\section{\FPUChapterName}
\label{sec:FPU}

\newcommand{\FNURLSTACK}{\footnote{\href{http://go.yurichev.com/17123}{wikipedia.org/wiki/Stack\_machine}}}
\newcommand{\FNURLFORTH}{\footnote{\href{http://go.yurichev.com/17124}{wikipedia.org/wiki/Forth\_(programming\_language)}}}
\newcommand{\FNURLIEEE}{\footnote{\href{http://go.yurichev.com/17125}{wikipedia.org/wiki/IEEE\_floating\_point}}}
\newcommand{\FNURLSP}{\footnote{\href{http://go.yurichev.com/17126}{wikipedia.org/wiki/Single-precision\_floating-point\_format}}}
\newcommand{\FNURLDP}{\footnote{\href{http://go.yurichev.com/17127}{wikipedia.org/wiki/Double-precision\_floating-point\_format}}}
\newcommand{\FNURLEP}{\footnote{\href{http://go.yurichev.com/17128}{wikipedia.org/wiki/Extended\_precision}}}

The \ac{FPU} is a device within the main \ac{CPU}, specially designed to deal with floating point numbers.

It was called \q{coprocessor} in the past and it stays somewhat aside of the main \ac{CPU}.

\subsection{IEEE 754}

A number in the IEEE 754 format consists of a \IT{sign}, a \IT{significand} (also called \IT{fraction}) and an \IT{exponent}.

\subsection{x86}

It is worth looking into stack machines\FNURLSTACK or learning the basics of the Forth language\FNURLFORTH,
before studying the \ac{FPU} in x86.

\myindex{Intel!80486}
\myindex{Intel!FPU}
It is interesting to know that in the past (before the 80486 CPU) the coprocessor was a separate chip 
and it was not always pre-installed on the motherboard. It was possible to buy it separately and install it
\footnote{For example, John Carmack used fixed-point arithmetic 
(\href{http://go.yurichev.com/17356}{wikipedia.org/wiki/Fixed-point\_arithmetic}) values in his Doom video game, stored in 
32-bit \ac{GPR} registers (16 bit for integral part and another 16 bit for fractional part), so Doom
could work on 32-bit computers without FPU, i.e., 80386 and 80486 SX.}.

Starting with the 80486 DX CPU, the \ac{FPU} is integrated in the \ac{CPU}.

\myindex{x86!\Instructions!FWAIT}
The \INS{FWAIT} instruction reminds us of that fact---it switches the \ac{CPU} to a waiting state, so it can wait until the \ac{FPU} has finished with its work.

Another rudiment is the fact that the \ac{FPU} instruction 
opcodes start with the so called \q{escape}-opcodes (\GTT{D8..DF}), i.e., 
opcodes passed to a separate coprocessor.

\myindex{IEEE 754}
\label{FPU_is_stack}

The FPU has a stack capable to holding 8 80-bit registers, and each register can hold a number 
in the IEEE 754\FNURLIEEE format.

They are \ST{0}..\ST{7}. For brevity, \IDA and \olly show \ST{0} as \GTT{ST}, 
which is represented in some textbooks and manuals as \q{Stack Top}.

\subsection{ARM, MIPS, x86/x64 SIMD}

In ARM and MIPS the FPU is not a stack, but a set of registers, which can be accessed randomly, like \ac{GPR}.

The same ideology is used in the SIMD extensions of x86/x64 CPUs.

\subsection{\CCpp}

\myindex{float}
\myindex{double}

The standard \CCpp languages offer at least two floating number types, \Tfloat (\IT{single-precision}\FNURLSP, 32 bits)
\footnote{the single precision floating point number format is also addressed in 
the \IT{\WorkingWithFloatAsWithStructSubSubSectionName}~(\myref{sec:floatasstruct}) section}
and \Tdouble (\IT{double-precision}\FNURLDP, 64 bits).

In \InSqBrackets{\TAOCPvolII 246} we can find the \IT{single-precision} means that the floating point value can be placed into a single
[32-bit] machine word, \IT{double-precision} means it can be stored in two words (64 bits).

\myindex{long double}

GCC also supports the \IT{long double} type (\IT{extended precision}\FNURLEP, 80 bit), which MSVC doesn't.

The \Tfloat type requires the same number of bits as the \Tint type in 32-bit environments, 
but the number representation is completely different.

\ifdefined\RUSSIAN
\subsection{Простой пример}

Рассмотрим простой пример:
\fi

\ifdefined\ENGLISH
\subsection{Simple example}

Let's consider this simple example:
\fi

\ifdefined\GERMAN
\subsection{\DEph{}}

\DEph{}

\fi

\ifdefined\FRENCH
\subsection{Exemple simple}

Considérons cet exemple simple:
\fi

\lstinputlisting[style=customc]{patterns/12_FPU/1_simple/simple.c}

\subsubsection{x86}

% subsubsections
\EN{\input{patterns/12_FPU/1_simple/MSVC_EN}}
\RU{\input{patterns/12_FPU/1_simple/MSVC_RU}}
\DE{\input{patterns/12_FPU/1_simple/MSVC_DE}}
\FR{\myparagraph{MSVC}

Compilons-le avec MSVC 2010:

\lstinputlisting[caption=MSVC 2010: \ttf{},style=customasmx86]{patterns/12_FPU/1_simple/MSVC_FR.asm}

\FLD prend 8 octets depuis la pile et charge le nombre dans le registre \ST{0}, en
le convertissant automatiquement dans le format interne sur 80-bit (\IT{précision
étendue}):

\myindex{x86!\Instructions!FDIV}

\FDIV divise la valeur dans \ST{0} par le nombre stocké à l'adresse \\
\GTT{\_\_real@40091eb851eb851f}~---la valeur 3.14 est encodée ici.
La syntaxe assembleur ne supporte pas les nombres à virgule flottante, donc ce que
l'on voit ici est la représentation hexadécimale de 3.14 au format 64-bit IEEE 754.

Après l'exécution de \FDIV, \ST{0} contient le \gls{quotient}.

\myindex{x86!\Instructions!FDIVP}

A propos, il y a aussi l'instruction \FDIVP, qui divise \ST{1} par \ST{0}, prenant
ces deux valeurs dans la pile et poussant le résultant.
Si vous connaissez le langage Forth\FNURLFORTH, vous pouvez comprendre rapidement
que ceci est une machine à pile\FNURLSTACK.

L'instruction \FLD subséquente pousse la valaeur de $b$ sur la pile.

Après cela, le quotient est placé dans \ST{1}, et \ST{0} a la valeur de $b$.

\myindex{x86!\Instructions!FMUL}

L'instruction suivante effectue la multiplication: $b$ de \ST{0} est multiplié par
la valeur en \GTT{\_\_real@4010666666666666} (le nombre 4.1 est là) et met le résultat
dans le registre \ST{0}.

\myindex{x86!\Instructions!FADDP}

La dernière instruction \FADDP ajoute les deux valeurs au sommet de la pile, stockant
le résultat dans \ST{1} et supprimant la valeur de \ST{0}, laissant ainsi le résultat
au sommet de la pile, dans \ST{0}.

La fonction doit renvoyer son résultat dans le registre \ST{0}, donc il n'y a aucune
autre instruction après \FADDP, excepté l'épilogue de la fonction.

\input{patterns/12_FPU/1_simple/olly_FR.tex}
}

\EN{\input{patterns/12_FPU/1_simple/GCC_EN}}
\RU{\input{patterns/12_FPU/1_simple/GCC_RU}}
\DE{\myparagraph{GCC 4.4.1}

\lstinputlisting[caption=GCC
4.4.1,style=customasmx86]{patterns/12_FPU/3_comparison/x86/GCC_DE.asm}

\myindex{x86!\Instructions!FUCOMPP}
\FUCOMPP{} ist fast wie like \FCOM, nimmt aber beide Werte vom Stand und
behandelt \q{undefinierte Zahlenwerte} anders.


\myindex{Non-a-numbers (NaNs)}
Ein wenig über \IT{undefinierte Zahlenwerte}.

\newcommand{\NANFN}{\footnote{\href{http://go.yurichev.com/17130}{wikipedia.org/wiki/NaN}}}
Die FPU ist in der Lage mit speziellen undefinieten Werten, den sogenannten
\IT{not-a-number}(kurz \gls{NaN})\NANFN umzugehen. Beispiele sind etwa der Wert
unendlich, das Ergebnis einer Division durch 0, etc. Undefinierte Werte können
entwder \q{quiet} oder \q{signaling} sein. Es ist möglich mit \q{quiet} NaNs zu
arbeiten, aber beim Versuch einen Befehl auf \q{signaling} NaNs auszuführen,
wird eine Exception geworfen. 

\myindex{x86!\Instructions!FCOM}
\myindex{x86!\Instructions!FUCOM}
\FCOM erzeugt eine Exception, falls irgendein Operand ein \gls{NaN} ist.
\FUCOM erzeugt eine Exception nur dann, wenn ein Operand eine \q{signaling}
\gls{NaN} (SNaN) ist.

\myindex{x86!\Instructions!SAHF}
\label{SAHF}
Der nächste Befehl ist \SAHF (\IT{Store AH into Flags})~---es handelt sich
hierbei um einen seltenen Befehl, der nicht mit der FPU zusammenhängt.
8 Bits aus AH werden in die niederen 8 Bit der CPU Flags in der folgenden
Reihenfolge verschoben:

\input{SAHF_LAHF}

\myindex{x86!\Instructions!FNSTSW}
Erinnern wir uns, dass \FNSTSW die für uns interessanten Bits (\CThreeBits) auf
den Stellen 6,2,0 im AH Register setzt:

\input{C3_in_AH}
Mit anderen Worten: der Befehl \INS{fnstsw ax / sahf} verschiebt \CThreeBits
nach \ZF, \PF und \CF. 

Überlegen wir uns auch die Werte der \CThreeBits in unterschiedlichen Szenarien:

\begin{itemize} 
  \item Falls in unserem Beispiel $a$ größer als $b$ ist, dann werden die
  \CThreeBits auf 0,0,0 gesetzt.
  \item Falls $a$ kleiner als $b$ ist, werden die Bits auf 0,0,1 gesetzt.
  \item Falls $a=b$, dann werden die Bits auf 1,0,0 gesetzt.
\end{itemize}
% TODO: table?
Mit anderen Worten, die folgenden Zustände der CPU Flags sind nach drei
\FUCOMPP/\FNSTSW/\SAHF Befehlen möglich:

\begin{itemize}
\item Falls $a>b$, werden die CPU Flags wie folgt gesetzt \GTT{ZF=0, PF=0,
CF=0}.
\item Falls $a<b$, werden die CPU Flags wie folgt gesetzt: \GTT{ZF=0, PF=0,
CF=1}.
\item Und falls $a=b$, dann gilt: \GTT{ZF=1, PF=0, CF=0}.
\end{itemize}
% TODO: table?

\myindex{x86!\Instructions!SETcc}
\myindex{x86!\Instructions!JNBE}
Abhängig von den CPU Flags und Bedingungen, speichert \SETNBE entweder 1 oder 0
in AL.
Es ist also quasi das Gegenstück von \JNBE mit dem Unterschied, dass \SETcc

Depending on the CPU flags and conditions, \SETNBE stores 1 or 0 to AL. 
It is almost the counterpart of \JNBE, with the exception that \SETcc
\footnote{\IT{cc} is \IT{condition code}} eine 1 oder 0 in \AL speichert, aber
\Jcc tatsächlich auch springt.
\SETNBE speicher 1 nur, falls \GTT{CF=0} und \GTT{ZF=0}.
Wenn dies nicht der Fall ist, dann wird 0 in \AL gespeichert.

Nur in einem Fall sind \CF und \ZF beide 0: falls $a>b$.

In diesem Fall wird 1 in \AL gespeichert, der nachfolgende \JZ Sprung wird nicht
ausgeführt und die Funktion liefert {\_a} zurück. In allen anderen Fällen wird
{\_b} zurückgegeben.
}
\FR{\myparagraph{GCC}

GCC 4.4.1 (avec l'option \Othree) génère le même code, juste un peu différent:

\lstinputlisting[caption=\Optimizing GCC 4.4.1,style=customasmx86]{patterns/12_FPU/1_simple/GCC_FR.asm}

La différence est que, tout d'abord, 3.14 est poussé sur la pile (dans \ST{0}), et
ensuite la valeur dans \GTT{arg\_0} est divisée par la valeur dans \ST{0}.

\myindex{x86!\Instructions!FDIVR}

\FDIVR signifie \IT{Reverse Divide}~---pour diviser avec le diviseur et le dividende
échangés l'un avec l'autre.
Il n'y a pas dînstruction de ce genre pour la multiplication puisque c'est une opération
commutative, donc nous avons seulement \FMUL sans son homologue \GTT{-R}.

\myindex{x86!\Instructions!FADDP}

\FADDP ajoute les deux valeurs mais prend (pop) aussi une valeur depuis la pile.
Après cette opération, \ST{0} contient la somme.

}


\EN{\section{Returning Values}
\label{ret_val_func}

Another simple function is the one that simply returns a constant value:

\lstinputlisting[caption=\EN{\CCpp Code},style=customc]{patterns/011_ret/1.c}

Let's compile it.

\subsection{x86}

Here's what both the GCC and MSVC compilers produce (with optimization) on the x86 platform:

\lstinputlisting[caption=\Optimizing GCC/MSVC (\assemblyOutput),style=customasmx86]{patterns/011_ret/1.s}

\myindex{x86!\Instructions!RET}
There are just two instructions: the first places the value 123 into the \EAX register,
which is used by convention for storing the return
value, and the second one is \RET, which returns execution to the \gls{caller}.

The caller will take the result from the \EAX register.

\subsection{ARM}

There are a few differences on the ARM platform:

\lstinputlisting[caption=\OptimizingKeilVI (\ARMMode) ASM Output,style=customasmARM]{patterns/011_ret/1_Keil_ARM_O3.s}

ARM uses the register \Reg{0} for returning the results of functions, so 123 is copied into \Reg{0}.

\myindex{ARM!\Instructions!MOV}
\myindex{x86!\Instructions!MOV}
It is worth noting that \MOV is a misleading name for the instruction in both the x86 and ARM \ac{ISA}s.

The data is not in fact \IT{moved}, but \IT{copied}.

\subsection{MIPS}

\label{MIPS_leaf_function_ex1}

The GCC assembly output below lists registers by number:

\lstinputlisting[caption=\Optimizing GCC 4.4.5 (\assemblyOutput),style=customasmMIPS]{patterns/011_ret/MIPS.s}

\dots while \IDA does it by their pseudo names:

\lstinputlisting[caption=\Optimizing GCC 4.4.5 (IDA),style=customasmMIPS]{patterns/011_ret/MIPS_IDA.lst}

The \$2 (or \$V0) register is used to store the function's return value.
\myindex{MIPS!\Pseudoinstructions!LI}
\INS{LI} stands for ``Load Immediate'' and is the MIPS equivalent to \MOV.

\myindex{MIPS!\Instructions!J}
The other instruction is the jump instruction (J or JR) which returns the execution flow to the \gls{caller}.

\myindex{MIPS!Branch delay slot}
You might be wondering why the positions of the load instruction (LI) and the jump instruction (J or JR) are swapped. This is due to a \ac{RISC} feature called ``branch delay slot''.

The reason this happens is a quirk in the architecture of some RISC \ac{ISA}s and isn't important for our
purposes---we must simply keep in mind that in MIPS, the instruction following a jump or branch instruction
is executed \IT{before} the jump/branch instruction itself.

As a consequence, branch instructions always swap places with the instruction executed immediately beforehand.


In practice, functions which merely return 1 (\IT{true}) or 0 (\IT{false}) are very frequent.

The smallest ever of the standard UNIX utilities, \IT{/bin/true} and \IT{/bin/false} return 0 and 1 respectively, as an exit code.
(Zero as an exit code usually means success, non-zero means error.)
}
\RU{\subsubsection{std::string}
\myindex{\Cpp!STL!std::string}
\label{std_string}

\myparagraph{Как устроена структура}

Многие строковые библиотеки \InSqBrackets{\CNotes 2.2} обеспечивают структуру содержащую ссылку 
на буфер собственно со строкой, переменная всегда содержащую длину строки 
(что очень удобно для массы функций \InSqBrackets{\CNotes 2.2.1}) и переменную содержащую текущий размер буфера.

Строка в буфере обыкновенно оканчивается нулем: это для того чтобы указатель на буфер можно было
передавать в функции требующие на вход обычную сишную \ac{ASCIIZ}-строку.

Стандарт \Cpp не описывает, как именно нужно реализовывать std::string,
но, как правило, они реализованы как описано выше, с небольшими дополнениями.

Строки в \Cpp это не класс (как, например, QString в Qt), а темплейт (basic\_string), 
это сделано для того чтобы поддерживать 
строки содержащие разного типа символы: как минимум \Tchar и \IT{wchar\_t}.

Так что, std::string это класс с базовым типом \Tchar.

А std::wstring это класс с базовым типом \IT{wchar\_t}.

\mysubparagraph{MSVC}

В реализации MSVC, вместо ссылки на буфер может содержаться сам буфер (если строка короче 16-и символов).

Это означает, что каждая короткая строка будет занимать в памяти по крайней мере $16 + 4 + 4 = 24$ 
байт для 32-битной среды либо $16 + 8 + 8 = 32$ 
байта в 64-битной, а если строка длиннее 16-и символов, то прибавьте еще длину самой строки.

\lstinputlisting[caption=пример для MSVC,style=customc]{\CURPATH/STL/string/MSVC_RU.cpp}

Собственно, из этого исходника почти всё ясно.

Несколько замечаний:

Если строка короче 16-и символов, 
то отдельный буфер для строки в \glslink{heap}{куче} выделяться не будет.

Это удобно потому что на практике, основная часть строк действительно короткие.
Вероятно, разработчики в Microsoft выбрали размер в 16 символов как разумный баланс.

Теперь очень важный момент в конце функции main(): мы не пользуемся методом c\_str(), тем не менее,
если это скомпилировать и запустить, то обе строки появятся в консоли!

Работает это вот почему.

В первом случае строка короче 16-и символов и в начале объекта std::string (его можно рассматривать
просто как структуру) расположен буфер с этой строкой.
\printf трактует указатель как указатель на массив символов оканчивающийся нулем и поэтому всё работает.

Вывод второй строки (длиннее 16-и символов) даже еще опаснее: это вообще типичная программистская ошибка 
(или опечатка), забыть дописать c\_str().
Это работает потому что в это время в начале структуры расположен указатель на буфер.
Это может надолго остаться незамеченным: до тех пока там не появится строка 
короче 16-и символов, тогда процесс упадет.

\mysubparagraph{GCC}

В реализации GCC в структуре есть еще одна переменная --- reference count.

Интересно, что указатель на экземпляр класса std::string в GCC указывает не на начало самой структуры, 
а на указатель на буфера.
В libstdc++-v3\textbackslash{}include\textbackslash{}bits\textbackslash{}basic\_string.h 
мы можем прочитать что это сделано для удобства отладки:

\begin{lstlisting}
   *  The reason you want _M_data pointing to the character %array and
   *  not the _Rep is so that the debugger can see the string
   *  contents. (Probably we should add a non-inline member to get
   *  the _Rep for the debugger to use, so users can check the actual
   *  string length.)
\end{lstlisting}

\href{http://go.yurichev.com/17085}{исходный код basic\_string.h}

В нашем примере мы учитываем это:

\lstinputlisting[caption=пример для GCC,style=customc]{\CURPATH/STL/string/GCC_RU.cpp}

Нужны еще небольшие хаки чтобы сымитировать типичную ошибку, которую мы уже видели выше, из-за
более ужесточенной проверки типов в GCC, тем не менее, printf() работает и здесь без c\_str().

\myparagraph{Чуть более сложный пример}

\lstinputlisting[style=customc]{\CURPATH/STL/string/3.cpp}

\lstinputlisting[caption=MSVC 2012,style=customasmx86]{\CURPATH/STL/string/3_MSVC_RU.asm}

Собственно, компилятор не конструирует строки статически: да в общем-то и как
это возможно, если буфер с ней нужно хранить в \glslink{heap}{куче}?

Вместо этого в сегменте данных хранятся обычные \ac{ASCIIZ}-строки, а позже, во время выполнения, 
при помощи метода \q{assign}, конструируются строки s1 и s2
.
При помощи \TT{operator+}, создается строка s3.

Обратите внимание на то что вызов метода c\_str() отсутствует,
потому что его код достаточно короткий и компилятор вставил его прямо здесь:
если строка короче 16-и байт, то в регистре EAX остается указатель на буфер,
а если длиннее, то из этого же места достается адрес на буфер расположенный в \glslink{heap}{куче}.

Далее следуют вызовы трех деструкторов, причем, они вызываются только если строка длиннее 16-и байт:
тогда нужно освободить буфера в \glslink{heap}{куче}.
В противном случае, так как все три объекта std::string хранятся в стеке,
они освобождаются автоматически после выхода из функции.

Следовательно, работа с короткими строками более быстрая из-за м\'{е}ньшего обращения к \glslink{heap}{куче}.

Код на GCC даже проще (из-за того, что в GCC, как мы уже видели, не реализована возможность хранить короткую
строку прямо в структуре):

% TODO1 comment each function meaning
\lstinputlisting[caption=GCC 4.8.1,style=customasmx86]{\CURPATH/STL/string/3_GCC_RU.s}

Можно заметить, что в деструкторы передается не указатель на объект,
а указатель на место за 12 байт (или 3 слова) перед ним, то есть, на настоящее начало структуры.

\myparagraph{std::string как глобальная переменная}
\label{sec:std_string_as_global_variable}

Опытные программисты на \Cpp знают, что глобальные переменные \ac{STL}-типов вполне можно объявлять.

Да, действительно:

\lstinputlisting[style=customc]{\CURPATH/STL/string/5.cpp}

Но как и где будет вызываться конструктор \TT{std::string}?

На самом деле, эта переменная будет инициализирована даже перед началом \main.

\lstinputlisting[caption=MSVC 2012: здесь конструируется глобальная переменная{,} а также регистрируется её деструктор,style=customasmx86]{\CURPATH/STL/string/5_MSVC_p2.asm}

\lstinputlisting[caption=MSVC 2012: здесь глобальная переменная используется в \main,style=customasmx86]{\CURPATH/STL/string/5_MSVC_p1.asm}

\lstinputlisting[caption=MSVC 2012: эта функция-деструктор вызывается перед выходом,style=customasmx86]{\CURPATH/STL/string/5_MSVC_p3.asm}

\myindex{\CStandardLibrary!atexit()}
В реальности, из \ac{CRT}, еще до вызова main(), вызывается специальная функция,
в которой перечислены все конструкторы подобных переменных.
Более того: при помощи atexit() регистрируется функция, которая будет вызвана в конце работы программы:
в этой функции компилятор собирает вызовы деструкторов всех подобных глобальных переменных.

GCC работает похожим образом:

\lstinputlisting[caption=GCC 4.8.1,style=customasmx86]{\CURPATH/STL/string/5_GCC.s}

Но он не выделяет отдельной функции в которой будут собраны деструкторы: 
каждый деструктор передается в atexit() по одному.

% TODO а если глобальная STL-переменная в другом модуле? надо проверить.

}
\DE{\subsection{Einfachste XOR-Verschlüsselung überhaupt}

Ich habe einmal eine Software gesehen, bei der alle Debugging-Ausgaben mit XOR mit dem Wert 3
verschlüsselt wurden. Mit anderen Worten, die beiden niedrigsten Bits aller Buchstaben wurden invertiert.

``Hello, world'' wurde zu ``Kfool/\#tlqog'':

\begin{lstlisting}
#!/usr/bin/python

msg="Hello, world!"

print "".join(map(lambda x: chr(ord(x)^3), msg))
\end{lstlisting}

Das ist eine ziemlich interessante Verschlüsselung (oder besser eine Verschleierung),
weil sie zwei wichtige Eigenschaften hat:
1) es ist eine einzige Funktion zum Verschlüsseln und entschlüsseln, sie muss nur wiederholt angewendet werden
2) die entstehenden Buchstaben befinden sich im druckbaren Bereich, also die ganze Zeichenkette kann ohne
Escape-Symbole im Code verwendet werden.

Die zweite Eigenschaft nutzt die Tatsache, dass alle druckbaren Zeichen in Reihen organisiert sind: 0x2x-0x7x,
und wenn die beiden niederwertigsten Bits invertiert werden, wird der Buchstabe um eine oder drei Stellen nach
links oder rechts \IT{verschoben}, aber niemals in eine andere Reihe:

\begin{figure}[H]
\centering
\includegraphics[width=0.7\textwidth]{ascii_clean.png}
\caption{7-Bit \ac{ASCII} Tabelle in Emacs}
\end{figure}

\dots mit dem Zeichen 0x7F als einziger Ausnahme.

Im Folgenden werden also beispielsweise die Zeichen A-Z \IT{verschlüsselt}:

\begin{lstlisting}
#!/usr/bin/python

msg="@ABCDEFGHIJKLMNO"

print "".join(map(lambda x: chr(ord(x)^3), msg))
\end{lstlisting}

Ergebnis:
% FIXME \verb  --  relevant comment for German?
\begin{lstlisting}
CBA@GFEDKJIHONML
\end{lstlisting}

Es sieht so aus als würden die Zeichen ``@'' und ``C'' sowie ``B'' und ``A'' vertauscht werden.

Hier ist noch ein interessantes Beispiel, in dem gezeigt wird, wie die Eigenschaften von XOR
ausgenutzt werden können: Exakt den gleichen Effekt, dass druckbare Zeichen auch druckbar bleiben,
kann man dadurch erzielen, dass irgendeine Kombination der niedrigsten vier Bits invertiert wird.
}
\FR{\subsubsection{ARM: \OptimizingXcodeIV (\ARMMode)}

Jusqu'à la standardisation du support de la virgule flottante, certains fabricants
de processeur ont ajouté leur propre instructions étendues.
Ensuite, VFP (\IT{Vector Floating Point}) a été standardisé.

Une différence importante par rapport au x86 est qu'en ARM, il n'y a pas de pile,
vous travaillez seulement avec des registres.

\lstinputlisting[label=ARM_leaf_example10,caption=\OptimizingXcodeIV (\ARMMode),style=customasmARM]{patterns/12_FPU/1_simple/ARM/Xcode_ARM_O3_FR.asm}

\myindex{ARM!D-\registers{}}
\myindex{ARM!S-\registers{}}

Donc, nous voyons ici que des nouveaux registres sont utilisés, avec le préfixe D.

Ce sont des registres 64-bits, il y en a 32, et ils peuvent être utilisés tant pour
des nombres à virgules flottantes (double) que pour des opérations SIMD (c'est appelé
NEON ici en ARM).

Il y a aussi 32 S-registres 32 bits, destinés à être utilisés pour les nombres à
virgules flottantes simple précision (float).

C'est facile à retenir: les registres D sont pour les nombres en double précision,
tandis que les registres S----pour les nombres en simple précision
Pour aller plus loin: \myref{ARM_VFP_registers}.

Les deux constantes (3.14 et 4.1) sont stockées en mémoire au format IEEE 754.

\myindex{ARM!\Instructions!VLDR}
\myindex{ARM!\Instructions!VMOV}
\INS{VLDR} et \INS{VMOV}, comme il peut en être facilement déduit, sont analogues
aux instructions \INS{LDR} et \MOV, mais travaillent avec des registres D.

Il est à noter que ces instructions, tout comme les registres D, sont destinées non
seulement pour les nombres à virgules flottantes, mais peuvent aussi être utilisées
pour des opérations SIMD (NEON) et cela va être montré bientôt.

Les arguments sont passés à la fonction de manière classique, via les registres-R,
toutefois, chaque nombres en double précision a une taille de 64 bits, donc deux
registres-R sont nécessaire pour passer chacun d'entre eux.

\INS{VMOV D17, R0, R1} au début, combine les deux valeurs 32-bit de \Reg{0} et \Reg{1}
en une valeur 64-bit et la sauve dans \GTT{D17}.

\INS{VMOV R0, R1, D16} est l'opération inverse: ce qui est dans \GTT{D16} est séparé
dans deux registres, \Reg{0} et \Reg{1}, car un nombre en double précision qui
nécessite 64 bit pour le stockage, est renvoyé dans \Reg{0} et \Reg{1}.

\myindex{ARM!\Instructions!VDIV}
\myindex{ARM!\Instructions!VMUL}
\myindex{ARM!\Instructions!VADD}
\INS{VDIV}, \INS{VMUL} and \INS{VADD}, 
sont des instructions pour traiter des nombres à virgule flottante, qui calculent
respectivement le \gls{quotient}, \glslink{product}{produit} et la somme.

Le code pour Thumb-2 est similaire.

\subsubsection{ARM: \OptimizingKeilVI (\ThumbMode)}

\lstinputlisting[style=customasmARM]{patterns/12_FPU/1_simple/ARM/Keil_O3_thumb_FR.asm}

Code généré par Keil pour un processeur sans FPU ou support pour NEON.

Les nombres en virgule flottante double précision sont passés par des registres-R
génériques et au lieu d'instructions FPU, des fonctions d'une bibliothèque de service
sont appelées (comme \GTT{\_\_aeabi\_dmul}, \GTT{\_\_aeabi\_ddiv}, \GTT{\_\_aeabi\_dadd})
qui émulent la multiplication, la division et l'addition pour les nombres à virgule
flottante.

Bien sûr, c'est plus lent qu'un coprocesseur FPU, mais toujours mieux que rien.

A propos, de telles bibliothèques d'émulation de FPU étaient très populaire dans
le monde x86 lorsque les coprocesseurs étaient rares et chers, et étaient installés
seulement dans des ordinateurs coûteux.

\myindex{ARM!soft float}
\myindex{ARM!armel}
\myindex{ARM!armhf}
\myindex{ARM!hard float}

L'émulation d'un coprocesseur FPU est appelée \IT{soft float} ou \IT{armel} (\IT{emulation})
dans le monde ARM, alors que l'utilisation des instructions d'un coprocesseur FPU
est appelée \IT{hard float} ou \IT{armhf}.

\iffalse
% TODO разобраться...
\myindex{Raspberry Pi}

Par exemple, le noyau Linux pour Raspberry Pi est compilé en deux variantes.

Dans le case \IT{soft float}, les arguments sont passés par les registres-R, et dans
le cas \IT{hard float}---par les registes-D.

Et c'est ce qui empêche d'utiliser des bibliothèques armfh pour de code armel ou
vice-versa, et c'est pourquoi le code dans les distributions Linux doit être compilé
suivant une seule convention.
\fi

\subsubsection{ARM64: GCC \Optimizing (Linaro) 4.9}

Code très compact:

\lstinputlisting[caption=GCC \Optimizing (Linaro) 4.9,style=customasmARM]{patterns/12_FPU/1_simple/ARM/ARM64_GCC_O3_FR.s}

\subsubsection{ARM64: GCC \NonOptimizing (Linaro) 4.9}

\lstinputlisting[caption=GCC \NonOptimizing (Linaro) 4.9,style=customasmARM]{patterns/12_FPU/1_simple/ARM/ARM64_GCC_O0_FR.s}

GCC \NonOptimizing est plus verbeux.

Il y a des nombreuses modifications de valeurs inutile, incluant du code clairement
redondant (les deux dernières instructions \INS{FMOV}). Sans doute que GCC 4.9 n'est
pas encore très bon pour la génération de code ARM64.

Il est utile de noter qu'ARM64 possède des registres 64-bit, et que les registres-D
sont aussi 64-bit.

Donc le compilateur est libre de sauver des valeurs de type \Tdouble dans \ac{GPR}s
au lieu de la pile locale.
Ce n'est pas possible sur des CPUs 32-bit.

Et encore, à titre d'exercice, vous pouvez essayer d'optimiser manuellement cette
fonction, sans introduire de nouvelles instructions comme \INS{FMADD}.
}



\EN{\subsubsection{MIPS}

MIPS can support several coprocessors (up to 4), 
the zeroth of which\footnote{Starting at 0.} is a special control coprocessor,
and first coprocessor is the FPU.

As in ARM, the MIPS coprocessor is not a stack machine, it has 32 32-bit registers (\$F0-\$F31):
\myref{MIPS_FPU_registers}.

When one needs to work with 64-bit \Tdouble values, a pair of 32-bit F-registers is used.

\lstinputlisting[caption=\Optimizing GCC 4.4.5 (IDA),style=customasmMIPS]{patterns/12_FPU/1_simple/MIPS_O3_IDA_EN.lst}

The new instructions here are:

\myindex{MIPS!\Instructions!LWC1}
\myindex{MIPS!\Instructions!DIV.D}
\myindex{MIPS!\Instructions!MUL.D}
\myindex{MIPS!\Instructions!ADD.D}
\begin{itemize}

\item \INS{LWC1} loads a 32-bit word into a register of the first coprocessor (hence \q{1} in instruction name).
\myindex{MIPS!\Pseudoinstructions!L.D}

A pair of \INS{LWC1} instructions may be combined into a \INS{L.D} pseudo instruction.

\item \INS{DIV.D}, \INS{MUL.D}, \INS{ADD.D} do division, multiplication, and addition respectively 
(\q{.D} in the suffix stands for double precision, \q{.S} stands for single precision)

\end{itemize}

\myindex{MIPS!\Instructions!LUI}
\myindex{\CompilerAnomaly}
\label{MIPS_FPU_LUI}

There is also a weird compiler anomaly: the \INS{LUI} instructions that we've marked with a question mark.
It's hard for me to understand why load a part of a 64-bit constant of \Tdouble type into the \$V0 register.
These instructions has no effect.
% TODO did you try checking out compiler source code?
If someone knows more about it, please drop an email to author\footnote{\EMAIL}.

}
\RU{\subsubsection{MIPS}

MIPS может поддерживать несколько сопроцессоров (вплоть до 4), нулевой из которых\footnote{Если считать с нуля.} это специальный
управляющий сопроцессор, а первый~--- это FPU.

Как и в ARM, сопроцессор в MIPS это не стековая машина. Он имеет 32 32-битных регистра (\$F0-\$F31):

\myref{MIPS_FPU_registers}.
Когда нужно работать с 64-битными значениями типа \Tdouble, используется пара 32-битных F-регистров.

\lstinputlisting[caption=\Optimizing GCC 4.4.5 (IDA),style=customasmMIPS]{patterns/12_FPU/1_simple/MIPS_O3_IDA_RU.lst}

Новые инструкции:

\myindex{MIPS!\Instructions!LWC1}
\myindex{MIPS!\Instructions!DIV.D}
\myindex{MIPS!\Instructions!MUL.D}
\myindex{MIPS!\Instructions!ADD.D}
\begin{itemize}

\item \INS{LWC1} загружает 32-битное слово в регистр первого сопроцессора (отсюда \q{1} в названии инструкции).

\myindex{MIPS!\Pseudoinstructions!L.D}
Пара инструкций \INS{LWC1} может быть объединена в одну псевдоинструкцию \INS{L.D}.

\item \INS{DIV.D}, \INS{MUL.D}, \INS{ADD.D} производят деление, умножение и сложение соответственно 
(\q{.D} в суффиксе означает двойную точность, \q{.S}~--- одинарную точность)

\end{itemize}

\myindex{MIPS!\Instructions!LUI}
\myindex{\CompilerAnomaly}
\label{MIPS_FPU_LUI}
Здесь также имеется странная аномалия компилятора: инструкция \INS{LUI} помеченная нами вопросительным знаком.%

Мне трудно понять, зачем загружать часть 64-битной константы типа \Tdouble в регистр \$V0.

От этих инструкций нет толка.
% TODO did you try checking out compiler source code?
Если кто-то об этом что-то знает, пожалуйста, напишите автору емейл \footnote{\EMAIL}.

}
\DE{\subsubsection{MIPS}

MIPS unterstütz mehrere Koprozessoren (bis zu 4); der nullte\footnote{TBT} in ein spezeiller
Kontroll-Koprozessor und der erste Koprozessor ist die FPU.

Genau wie in ARM ist der MIPS Koprozessor keine Stackmaschine, sondern hat 32
32-bit-Register (\$F0-\$F31):
\myref{MIPS_FPU_registers}.

Muss man mit 64-bit \Tdouble Werten arbeiten, wird ein Paar 32-bit F-Register
hierfür verwendet.

\lstinputlisting[caption=\Optimizing GCC 4.4.5
(IDA)]{patterns/12_FPU/1_simple/MIPS_O3_IDA_DE.lst}

Die neuen Befehl sind im Einzelnen:

\myindex{MIPS!\Instructions!LWC1}
\myindex{MIPS!\Instructions!DIV.D}
\myindex{MIPS!\Instructions!MUL.D}
\myindex{MIPS!\Instructions!ADD.D}
\begin{itemize}

\item \INS{LWC1} lädt ein 32-bit-Wort in ein Register des ersten Koprozessors
(daher \q{1} im Namen des Befehls).
\myindex{MIPS!\Pseudoinstructions!L.D}

Ein Parr \INS{LWC1} Befehle kann zu einem \INS{L.D} Pseudobefehl zusammengefasst
werden.

\item \INS{DIV.D}, \INS{MUL.D}, \INS{ADD.D} führen Division, Multiplikation bzw.
Addition aus (das \q{D.} im Suffix steht für doppelte Genauigkeit, \q{S.}
bedeutet entsprechend einfache Genauigkeit).

\end{itemize}

\myindex{MIPS!\Instructions!LUI}
\myindex{\CompilerAnomaly}
\label{MIPS_FPU_LUI}
Es gibt ein verrückte Anomalie im Compiler: die \INS{LUI} Befehle, die wir mit
einem Fragezeichen versehen haben. 
Es ist sehr schwer zu verstehen, warum ein Teil einer
64-bit-Konstante vom Typ \Tdouble in das \$V0 Register geladen wird. 
Dieser Befehl hat keine Auswirkungen. 

% TODO did you try checking out compiler source code?
Sollte jemand mehr zu dieser Anomalie wissen, bittet der Autor um eine
Mail\footnote{\EMAIL}.

}
\FR{\subsubsection{MIPS}

MIPS peut supporter plusieurs coprocesseur (jusqu'à 4), le zérotième\footnote{Barbarisme pour rappeler que les indices commencent à zéro.} est un coprocesseur
contrôleur spécial, et celui d'indice 1 est le FPU.

Comme en ARM, le coprocesseur MIPS n'est pas une machine à pile, il comprend 32 registres
32-bit (\$F0-\$F31):
\myref{MIPS_FPU_registers}.

Lorsque l'on doit travailler avec des valeurs \Tdouble 64-bit, une paire de registre-F
32-bit est utilisée.

\lstinputlisting[caption=GCC 4.4.5 \Optimizing (IDA),style=customasmMIPS]{patterns/12_FPU/1_simple/MIPS_O3_IDA_FR.lst}

Les nouvelles instructions ici sont:

\myindex{MIPS!\Instructions!LWC1}
\myindex{MIPS!\Instructions!DIV.D}
\myindex{MIPS!\Instructions!MUL.D}
\myindex{MIPS!\Instructions!ADD.D}
\begin{itemize}

\item \INS{LWC1} charge un mot de 32-bit dans un registre du premier coprocesseur
(d'où le \q{1} dans le nom de l'instruction).
\myindex{MIPS!\Pseudoinstructions!L.D}

Une paire d'instructions \INS{LWC1} peut être combinée en une pseudo instruction \INS{L.D}.

\item \INS{DIV.D}, \INS{MUL.D}, \INS{ADD.D} effectuent respectivement la division,
la multiplication, et l'addition (\q{.D} est le suffixe standard pour la double précision,
\q{.S} pour la simple précision)

\end{itemize}

\myindex{MIPS!\Instructions!LUI}
\myindex{\CompilerAnomaly}
\label{MIPS_FPU_LUI}

Il y a une anomalie bizarre du compilateur: l'instruction \INS{LUI} que nous avons
marqué avec un point d'interrogation.
Il m'est difficile de comprendre pourquoi charger une partie de la constante de type
64-bit \Tdouble dans le registre \$V0. Cette instruction n'a pas d'effet.
% TODO did you try checking out compiler source code?
Si quelqu'un en sait plus sur ceci, s'il vous plait, envoyez un email à l'auteur\footnote{\EMAIL}.

}


\subsection{\RU{Передача чисел с плавающей запятой в аргументах}\EN{Passing floating point numbers via arguments}\DEph{}\FR{Passage de nombres en virgule flottante par les arguments}}
\myindex{\CStandardLibrary!pow()}

\lstinputlisting[style=customc]{patterns/12_FPU/2_passing_floats/pow.c}

\EN{\input{patterns/12_FPU/2_passing_floats/x86_EN}}
\RU{\input{patterns/12_FPU/2_passing_floats/x86_RU}}
\DE{\input{patterns/12_FPU/2_passing_floats/x86_DE}}
\FR{\subsubsection{x86}

Regardons ce que nous obtenons avec MSVC 2010:

\lstinputlisting[caption=MSVC 2010,style=customasmx86]{patterns/12_FPU/2_passing_floats/MSVC_FR.asm}

\myindex{x86!\Instructions!FLD}
\myindex{x86!\Instructions!FSTP}

\FLD et \FSTP déplacent des variables entre le segment de données et la pile du FPU.
\GTT{pow()}\footnote{une fonction C standard, qui élève un nombre à la puissance
donnée (puissance)} prend deux valeurs depuis la pile du FPU et renvoie son résultat
dans le registre \ST{0}.
\printf prend 8 octets de la pile locale et les interprète comme des variables de
type \Tdouble.

Á prosos, une paire d'instructions \MOV pourrait être utilisée ici pour déplacer
les valeurs depuis la mémoire vers la pile, car les valeurs en mémoire sont stockées
au format IEEE 754, et pow() les prend aussi dans ce format, donc aucune conversion
n'est nécessaire.
C'est comme ceci que c'est fait dans l'exemple suivant, pour ARM: \myref{FPU_passing_floats_ARM}.

}

\EN{\input{patterns/12_FPU/2_passing_floats/ARM_EN}}
\RU{\input{patterns/12_FPU/2_passing_floats/ARM_RU}}
\DE{\input{patterns/12_FPU/2_passing_floats/ARM_DE}}
\FR{\subsubsection{ARM + \NonOptimizingXcodeIV (\ThumbTwoMode)}
\label{FPU_passing_floats_ARM}

\lstinputlisting[style=customasmARM]{patterns/12_FPU/2_passing_floats/Xcode_thumb_O0.asm}

Comme nous l'avons déjà mentionné, les pointeurs sur des nombres flottants 64-bit
sont passés dans une paire de R-registres.

Ce code est un peu redondant (probablement car l'optimisation est désactivée),
puisqu'il est possible de charger les valeurs directement dans les R-registres sans
toucher les D-registres.

Donc, comme nous le voyons, la fonction \GTT{\_pow} reçoit son premier argument dans
\Reg{0} et \Reg{1}, et le second dans \Reg{2} et \Reg{3}.
La fonction laisse son résultat dans \Reg{0} et \Reg{1}.
Le résultat de \GTT{\_pow} est déplacé dans \GTT{D16}, puis dans la paire \Reg{1}
et \Reg{2}, d'où \printf prend le nombre résultant.

\subsubsection{ARM + \NonOptimizingKeilVI (\ARMMode)}

\lstinputlisting[style=customasmARM]{patterns/12_FPU/2_passing_floats/Keil_ARM_O0.asm}

%%D-registers are not used here, just R-register pairs.
Les D-registres ne sont pas utilisés ici, juste des paires de R-registres.

\subsubsection{ARM64 + GCC (Linaro) 4.9 \Optimizing}

\lstinputlisting[caption=GCC (Linaro) 4.9 \Optimizing,style=customasmARM]{patterns/12_FPU/2_passing_floats/ARM64_FR.s}

Les constantes sont chargées dans \RegD{0} et \RegD{1}: \TT{pow()} les prend d'ici.
Le résultat sera dans \RegD{0} après l'exécution de \TT{pow()}.
Il est passé à  \printf sans aucune modification ni déplacement, car \printf
prend ces arguments de \glslink{integral type}{type intégral} et pointeurs depuis
des X-registres, et les arguments en virgule flottante depuis des D-registres.

}

\EN{\subsubsection{MIPS}

MIPS can support several coprocessors (up to 4), 
the zeroth of which\footnote{Starting at 0.} is a special control coprocessor,
and first coprocessor is the FPU.

As in ARM, the MIPS coprocessor is not a stack machine, it has 32 32-bit registers (\$F0-\$F31):
\myref{MIPS_FPU_registers}.

When one needs to work with 64-bit \Tdouble values, a pair of 32-bit F-registers is used.

\lstinputlisting[caption=\Optimizing GCC 4.4.5 (IDA),style=customasmMIPS]{patterns/12_FPU/1_simple/MIPS_O3_IDA_EN.lst}

The new instructions here are:

\myindex{MIPS!\Instructions!LWC1}
\myindex{MIPS!\Instructions!DIV.D}
\myindex{MIPS!\Instructions!MUL.D}
\myindex{MIPS!\Instructions!ADD.D}
\begin{itemize}

\item \INS{LWC1} loads a 32-bit word into a register of the first coprocessor (hence \q{1} in instruction name).
\myindex{MIPS!\Pseudoinstructions!L.D}

A pair of \INS{LWC1} instructions may be combined into a \INS{L.D} pseudo instruction.

\item \INS{DIV.D}, \INS{MUL.D}, \INS{ADD.D} do division, multiplication, and addition respectively 
(\q{.D} in the suffix stands for double precision, \q{.S} stands for single precision)

\end{itemize}

\myindex{MIPS!\Instructions!LUI}
\myindex{\CompilerAnomaly}
\label{MIPS_FPU_LUI}

There is also a weird compiler anomaly: the \INS{LUI} instructions that we've marked with a question mark.
It's hard for me to understand why load a part of a 64-bit constant of \Tdouble type into the \$V0 register.
These instructions has no effect.
% TODO did you try checking out compiler source code?
If someone knows more about it, please drop an email to author\footnote{\EMAIL}.

}
\RU{\subsubsection{MIPS}

MIPS может поддерживать несколько сопроцессоров (вплоть до 4), нулевой из которых\footnote{Если считать с нуля.} это специальный
управляющий сопроцессор, а первый~--- это FPU.

Как и в ARM, сопроцессор в MIPS это не стековая машина. Он имеет 32 32-битных регистра (\$F0-\$F31):

\myref{MIPS_FPU_registers}.
Когда нужно работать с 64-битными значениями типа \Tdouble, используется пара 32-битных F-регистров.

\lstinputlisting[caption=\Optimizing GCC 4.4.5 (IDA),style=customasmMIPS]{patterns/12_FPU/1_simple/MIPS_O3_IDA_RU.lst}

Новые инструкции:

\myindex{MIPS!\Instructions!LWC1}
\myindex{MIPS!\Instructions!DIV.D}
\myindex{MIPS!\Instructions!MUL.D}
\myindex{MIPS!\Instructions!ADD.D}
\begin{itemize}

\item \INS{LWC1} загружает 32-битное слово в регистр первого сопроцессора (отсюда \q{1} в названии инструкции).

\myindex{MIPS!\Pseudoinstructions!L.D}
Пара инструкций \INS{LWC1} может быть объединена в одну псевдоинструкцию \INS{L.D}.

\item \INS{DIV.D}, \INS{MUL.D}, \INS{ADD.D} производят деление, умножение и сложение соответственно 
(\q{.D} в суффиксе означает двойную точность, \q{.S}~--- одинарную точность)

\end{itemize}

\myindex{MIPS!\Instructions!LUI}
\myindex{\CompilerAnomaly}
\label{MIPS_FPU_LUI}
Здесь также имеется странная аномалия компилятора: инструкция \INS{LUI} помеченная нами вопросительным знаком.%

Мне трудно понять, зачем загружать часть 64-битной константы типа \Tdouble в регистр \$V0.

От этих инструкций нет толка.
% TODO did you try checking out compiler source code?
Если кто-то об этом что-то знает, пожалуйста, напишите автору емейл \footnote{\EMAIL}.

}
\DE{\subsubsection{MIPS}

MIPS unterstütz mehrere Koprozessoren (bis zu 4); der nullte\footnote{TBT} in ein spezeiller
Kontroll-Koprozessor und der erste Koprozessor ist die FPU.

Genau wie in ARM ist der MIPS Koprozessor keine Stackmaschine, sondern hat 32
32-bit-Register (\$F0-\$F31):
\myref{MIPS_FPU_registers}.

Muss man mit 64-bit \Tdouble Werten arbeiten, wird ein Paar 32-bit F-Register
hierfür verwendet.

\lstinputlisting[caption=\Optimizing GCC 4.4.5
(IDA)]{patterns/12_FPU/1_simple/MIPS_O3_IDA_DE.lst}

Die neuen Befehl sind im Einzelnen:

\myindex{MIPS!\Instructions!LWC1}
\myindex{MIPS!\Instructions!DIV.D}
\myindex{MIPS!\Instructions!MUL.D}
\myindex{MIPS!\Instructions!ADD.D}
\begin{itemize}

\item \INS{LWC1} lädt ein 32-bit-Wort in ein Register des ersten Koprozessors
(daher \q{1} im Namen des Befehls).
\myindex{MIPS!\Pseudoinstructions!L.D}

Ein Parr \INS{LWC1} Befehle kann zu einem \INS{L.D} Pseudobefehl zusammengefasst
werden.

\item \INS{DIV.D}, \INS{MUL.D}, \INS{ADD.D} führen Division, Multiplikation bzw.
Addition aus (das \q{D.} im Suffix steht für doppelte Genauigkeit, \q{S.}
bedeutet entsprechend einfache Genauigkeit).

\end{itemize}

\myindex{MIPS!\Instructions!LUI}
\myindex{\CompilerAnomaly}
\label{MIPS_FPU_LUI}
Es gibt ein verrückte Anomalie im Compiler: die \INS{LUI} Befehle, die wir mit
einem Fragezeichen versehen haben. 
Es ist sehr schwer zu verstehen, warum ein Teil einer
64-bit-Konstante vom Typ \Tdouble in das \$V0 Register geladen wird. 
Dieser Befehl hat keine Auswirkungen. 

% TODO did you try checking out compiler source code?
Sollte jemand mehr zu dieser Anomalie wissen, bittet der Autor um eine
Mail\footnote{\EMAIL}.

}
\FR{\subsubsection{MIPS}

\lstinputlisting[caption=\Optimizing GCC 4.4.5 (IDA),style=customasmMIPS]{patterns/12_FPU/2_passing_floats/MIPS_O3_IDA_FR.lst}

À nouveau, nous voyons ici \INS{LUI} qui charge une partie 32-bit d'un nombre \Tdouble
dans \$V0.
À nouveau, c'est difficile de comprendre pourquoi.

\myindex{MIPS!\Instructions!MFC1}

La nouvelle instruction pour nous ici est \INS{MFC1} (\q{Move From Coprocessor 1}
charger depuis le coprocesseur 1).
Le FPU est le coprocesseur numéro 1, d'où le \q{1} dans le nom de l'instruction.
Cette instruction transfère des valeurs depuis des registres du coprocesseur dans
les registres du CPU (\ac{GPR}).
Donc à la fin, le résultat de \TT{pow()} est transfèré dans les registres \$A3 et
\$A2, et \printf prend une valeur double 64-bit depuis cette paire de registre.

}


\ifdefined\SPANISH
\chapter{Patrones de código}
\fi % SPANISH

\ifdefined\GERMAN
\chapter{Code-Muster}
\fi % GERMAN

\ifdefined\ENGLISH
\chapter{Code Patterns}
\fi % ENGLISH

\ifdefined\ITALIAN
\chapter{Forme di codice}
\fi % ITALIAN

\ifdefined\RUSSIAN
\chapter{Образцы кода}
\fi % RUSSIAN

\ifdefined\BRAZILIAN
\chapter{Padrões de códigos}
\fi % BRAZILIAN

\ifdefined\THAI
\chapter{รูปแบบของโค้ด}
\fi % THAI

\ifdefined\FRENCH
\chapter{Modèle de code}
\fi % FRENCH

\ifdefined\POLISH
\chapter{\PLph{}}
\fi % POLISH

% sections
\EN{\section{The method}

When the author of this book first started learning C and, later, \Cpp, he used to write small pieces of code, compile them,
and then look at the assembly language output. This made it very easy for him to understand what was going on in the code that he had written.
\footnote{In fact, he still does this when he can't understand what a particular bit of code does.}.
He did this so many times that the relationship between the \CCpp code and what the compiler produced was imprinted deeply in his mind.
It's now easy for him to imagine instantly a rough outline of a C code's appearance and function.
Perhaps this technique could be helpful for others.

%There are a lot of examples for both x86/x64 and ARM.
%Those who already familiar with one of architectures, may freely skim over pages.

By the way, there is a great website where you can do the same, with various compilers, instead of installing them on your box.
You can use it as well: \url{https://gcc.godbolt.org/}.

\section*{\Exercises}

When the author of this book studied assembly language, he also often compiled small C functions and then rewrote
them gradually to assembly, trying to make their code as short as possible.
This probably is not worth doing in real-world scenarios today,
because it's hard to compete with the latest compilers in terms of efficiency. It is, however, a very good way to gain a better understanding of assembly.
Feel free, therefore, to take any assembly code from this book and try to make it shorter.
However, don't forget to test what you have written.

% rewrote to show that debug\release and optimisations levels are orthogonal concepts.
\section*{Optimization levels and debug information}

Source code can be compiled by different compilers with various optimization levels.
A typical compiler has about three such levels, where level zero means that optimization is completely disabled.
Optimization can also be targeted towards code size or code speed.
A non-optimizing compiler is faster and produces more understandable (albeit verbose) code,
whereas an optimizing compiler is slower and tries to produce code that runs faster (but is not necessarily more compact).
In addition to optimization levels, a compiler can include some debug information in the resulting file,
producing code that is easy to debug.
One of the important features of the ´debug' code is that it might contain links
between each line of the source code and its respective machine code address.
Optimizing compilers, on the other hand, tend to produce output where entire lines of source code
can be optimized away and thus not even be present in the resulting machine code.
Reverse engineers can encounter either version, simply because some developers turn on the compiler's optimization flags and others do not.
Because of this, we'll try to work on examples of both debug and release versions of the code featured in this book, wherever possible.

Sometimes some pretty ancient compilers are used in this book, in order to get the shortest (or simplest) possible code snippet.
}
\ES{\input{patterns/patterns_opt_dbg_ES}}
\ITA{\input{patterns/patterns_opt_dbg_ITA}}
\PTBR{\input{patterns/patterns_opt_dbg_PTBR}}
\RU{\input{patterns/patterns_opt_dbg_RU}}
\THA{\input{patterns/patterns_opt_dbg_THA}}
\DE{\input{patterns/patterns_opt_dbg_DE}}
\FR{\input{patterns/patterns_opt_dbg_FR}}
\PL{\input{patterns/patterns_opt_dbg_PL}}

\RU{\section{Некоторые базовые понятия}}
\EN{\section{Some basics}}
\DE{\section{Einige Grundlagen}}
\FR{\section{Quelques bases}}
\ES{\section{\ESph{}}}
\ITA{\section{Alcune basi teoriche}}
\PTBR{\section{\PTBRph{}}}
\THA{\section{\THAph{}}}
\PL{\section{\PLph{}}}

% sections:
\EN{\input{patterns/intro_CPU_ISA_EN}}
\ES{\input{patterns/intro_CPU_ISA_ES}}
\ITA{\input{patterns/intro_CPU_ISA_ITA}}
\PTBR{\input{patterns/intro_CPU_ISA_PTBR}}
\RU{\input{patterns/intro_CPU_ISA_RU}}
\DE{\input{patterns/intro_CPU_ISA_DE}}
\FR{\input{patterns/intro_CPU_ISA_FR}}
\PL{\input{patterns/intro_CPU_ISA_PL}}

\EN{\subsection{Numeral Systems}

Humans have become accustomed to a decimal numeral system, probably because almost everyone has 10 fingers.
Nevertheless, the number \q{10} has no significant meaning in science and mathematics.
The natural numeral system in digital electronics is binary: 0 is for an absence of current in the wire, and 1 for presence.
10 in binary is 2 in decimal, 100 in binary is 4 in decimal, and so on.

% This sentence is a bit unweildy - maybe try 'Our ten-digit system would be described as having a radix...' - Renaissance
If the numeral system has 10 digits, it has a \IT{radix} (or \IT{base}) of 10.
The binary numeral system has a \IT{radix} of 2.

Important things to recall:

1) A \IT{number} is a number, while a \IT{digit} is a term from writing systems, and is usually one character

% The original is 'number' is not changed; I think the intent is value, and changed it - Renaissance
2) The value of a number does not change when converted to another radix; only the writing notation for that value has changed (and therefore the way of representing it in \ac{RAM}).

\subsection{Converting From One Radix To Another}

Positional notation is used almost every numerical system. This means that a digit has weight relative to where it is placed inside of the larger number.
If 2 is placed at the rightmost place, it's 2, but if it's placed one digit before rightmost, it's 20.

What does $1234$ stand for?

$10^3 \cdot 1 + 10^2 \cdot 2 + 10^1 \cdot 3 + 1 \cdot 4 = 1234$ or
$1000 \cdot 1 + 100 \cdot 2 + 10 \cdot 3 + 4 = 1234$

It's the same story for binary numbers, but the base is 2 instead of 10.
What does 0b101011 stand for?

$2^5 \cdot 1 + 2^4 \cdot 0 + 2^3 \cdot 1 + 2^2 \cdot 0 + 2^1 \cdot 1 + 2^0 \cdot 1 = 43$ or
$32 \cdot 1 + 16 \cdot 0 + 8 \cdot 1 + 4 \cdot 0 + 2 \cdot 1 + 1 = 43$

There is such a thing as non-positional notation, such as the Roman numeral system.
\footnote{About numeric system evolution, see \InSqBrackets{\TAOCPvolII{}, 195--213.}}.
% Maybe add a sentence to fill in that X is always 10, and is therefore non-positional, even though putting an I before subtracts and after adds, and is in that sense positional
Perhaps, humankind switched to positional notation because it's easier to do basic operations (addition, multiplication, etc.) on paper by hand.

Binary numbers can be added, subtracted and so on in the very same as taught in schools, but only 2 digits are available.

Binary numbers are bulky when represented in source code and dumps, so that is where the hexadecimal numeral system can be useful.
A hexadecimal radix uses the digits 0..9, and also 6 Latin characters: A..F.
Each hexadecimal digit takes 4 bits or 4 binary digits, so it's very easy to convert from binary number to hexadecimal and back, even manually, in one's mind.

\begin{center}
\begin{longtable}{ | l | l | l | }
\hline
\HeaderColor hexadecimal & \HeaderColor binary & \HeaderColor decimal \\
\hline
0	&0000	&0 \\
1	&0001	&1 \\
2	&0010	&2 \\
3	&0011	&3 \\
4	&0100	&4 \\
5	&0101	&5 \\
6	&0110	&6 \\
7	&0111	&7 \\
8	&1000	&8 \\
9	&1001	&9 \\
A	&1010	&10 \\
B	&1011	&11 \\
C	&1100	&12 \\
D	&1101	&13 \\
E	&1110	&14 \\
F	&1111	&15 \\
\hline
\end{longtable}
\end{center}

How can one tell which radix is being used in a specific instance?

Decimal numbers are usually written as is, i.e., 1234. Some assemblers allow an identifier on decimal radix numbers, in which the number would be written with a "d" suffix: 1234d.

Binary numbers are sometimes prepended with the "0b" prefix: 0b100110111 (\ac{GCC} has a non-standard language extension for this\footnote{\url{https://gcc.gnu.org/onlinedocs/gcc/Binary-constants.html}}).
There is also another way: using a "b" suffix, for example: 100110111b.
This book tries to use the "0b" prefix consistently throughout the book for binary numbers.

Hexadecimal numbers are prepended with "0x" prefix in \CCpp and other \ac{PL}s: 0x1234ABCD.
Alternatively, they are given a "h" suffix: 1234ABCDh. This is common way of representing them in assemblers and debuggers.
In this convention, if the number is started with a Latin (A..F) digit, a 0 is added at the beginning: 0ABCDEFh.
There was also convention that was popular in 8-bit home computers era, using \$ prefix, like \$ABCD.
The book will try to stick to "0x" prefix throughout the book for hexadecimal numbers.

Should one learn to convert numbers mentally? A table of 1-digit hexadecimal numbers can easily be memorized.
As for larger numbers, it's probably not worth tormenting yourself.

Perhaps the most visible hexadecimal numbers are in \ac{URL}s.
This is the way that non-Latin characters are encoded.
For example:
\url{https://en.wiktionary.org/wiki/na\%C3\%AFvet\%C3\%A9} is the \ac{URL} of Wiktionary article about \q{naïveté} word.

\subsubsection{Octal Radix}

Another numeral system heavily used in the past of computer programming is octal. In octal there are 8 digits (0..7), and each is mapped to 3 bits, so it's easy to convert numbers back and forth.
It has been superseded by the hexadecimal system almost everywhere, but, surprisingly, there is a *NIX utility, used often by many people, which takes octal numbers as argument: \TT{chmod}.

\myindex{UNIX!chmod}
As many *NIX users know, \TT{chmod} argument can be a number of 3 digits. The first digit represents the rights of the owner of the file (read, write and/or execute), the second is the rights for the group to which the file belongs, and the third is for everyone else.
Each digit that \TT{chmod} takes can be represented in binary form:

\begin{center}
\begin{longtable}{ | l | l | l | }
\hline
\HeaderColor decimal & \HeaderColor binary & \HeaderColor meaning \\
\hline
7	&111	&\textbf{rwx} \\
6	&110	&\textbf{rw-} \\
5	&101	&\textbf{r-x} \\
4	&100	&\textbf{r-{}-} \\
3	&011	&\textbf{-wx} \\
2	&010	&\textbf{-w-} \\
1	&001	&\textbf{-{}-x} \\
0	&000	&\textbf{-{}-{}-} \\
\hline
\end{longtable}
\end{center}

So each bit is mapped to a flag: read/write/execute.

The importance of \TT{chmod} here is that the whole number in argument can be represented as octal number.
Let's take, for example, 644.
When you run \TT{chmod 644 file}, you set read/write permissions for owner, read permissions for group and again, read permissions for everyone else.
If we convert the octal number 644 to binary, it would be \TT{110100100}, or, in groups of 3 bits, \TT{110 100 100}.

Now we see that each triplet describe permissions for owner/group/others: first is \TT{rw-}, second is \TT{r--} and third is \TT{r--}.

The octal numeral system was also popular on old computers like PDP-8, because word there could be 12, 24 or 36 bits, and these numbers are all divisible by 3, so the octal system was natural in that environment.
Nowadays, all popular computers employ word/address sizes of 16, 32 or 64 bits, and these numbers are all divisible by 4, so the hexadecimal system is more natural there.

The octal numeral system is supported by all standard \CCpp compilers.
This is a source of confusion sometimes, because octal numbers are encoded with a zero prepended, for example, 0377 is 255.
Sometimes, you might make a typo and write "09" instead of 9, and the compiler would report an error.
GCC might report something like this:\\
\TT{error: invalid digit "9" in octal constant}.

Also, the octal system is somewhat popular in Java. When the IDA shows Java strings with non-printable characters,
they are encoded in the octal system instead of hexadecimal.
\myindex{JAD}
The JAD Java decompiler behaves the same way.

\subsubsection{Divisibility}

When you see a decimal number like 120, you can quickly deduce that it's divisible by 10, because the last digit is zero.
In the same way, 123400 is divisible by 100, because the two last digits are zeros.

Likewise, the hexadecimal number 0x1230 is divisible by 0x10 (or 16), 0x123000 is divisible by 0x1000 (or 4096), etc.

The binary number 0b1000101000 is divisible by 0b1000 (8), etc.

This property can often be used to quickly realize if the size of some block in memory is padded to some boundary.
For example, sections in \ac{PE} files are almost always started at addresses ending with 3 hexadecimal zeros: 0x41000, 0x10001000, etc.
The reason behind this is the fact that almost all \ac{PE} sections are padded to a boundary of 0x1000 (4096) bytes.

\subsubsection{Multi-Precision Arithmetic and Radix}

\index{RSA}
Multi-precision arithmetic can use huge numbers, and each one may be stored in several bytes.
For example, RSA keys, both public and private, span up to 4096 bits, and maybe even more.

% I'm not sure how to change this, but the normal format for quoting would be just to mention the author or book, and footnote to the full reference
In \InSqBrackets{\TAOCPvolII, 265} we find the following idea: when you store a multi-precision number in several bytes,
the whole number can be represented as having a radix of $2^8=256$, and each digit goes to the corresponding byte.
Likewise, if you store a multi-precision number in several 32-bit integer values, each digit goes to each 32-bit slot,
and you may think about this number as stored in radix of $2^{32}$.

\subsubsection{How to Pronounce Non-Decimal Numbers}

Numbers in a non-decimal base are usually pronounced by digit by digit: ``one-zero-zero-one-one-...''.
Words like ``ten'' and ``thousand'' are usually not pronounced, to prevent confusion with the decimal base system.

\subsubsection{Floating point numbers}

To distinguish floating point numbers from integers, they are usually written with ``.0'' at the end,
like $0.0$, $123.0$, etc.
}
\RU{\subsection{Представление чисел}

Люди привыкли к десятичной системе счисления вероятно потому что почти у каждого есть по 10 пальцев.
Тем не менее, число 10 не имеет особого значения в науке и математике.
Двоичная система естествена для цифровой электроники: 0 означает отсутствие тока в проводе и 1 --- его присутствие.
10 в двоичной системе это 2 в десятичной; 100 в двоичной это 4 в десятичной, итд.

Если в системе счисления есть 10 цифр, её \IT{основание} или \IT{radix} это 10.
Двоичная система имеет \IT{основание} 2.

Важные вещи, которые полезно вспомнить:
1) \IT{число} это число, в то время как \IT{цифра} это термин из системы письменности, и это обычно один символ;
2) само число не меняется, когда конвертируется из одного основания в другое: меняется способ его записи (или представления
в памяти).

Как сконвертировать число из одного основания в другое?

Позиционная нотация используется почти везде, это означает, что всякая цифра имеет свой вес, в зависимости от её расположения
внутри числа.
Если 2 расположена в самом последнем месте справа, это 2.
Если она расположена в месте перед последним, это 20.

Что означает $1234$?

$10^3 \cdot 1 + 10^2 \cdot 2 + 10^1 \cdot 3 + 1 \cdot 4$ = 1234 или
$1000 \cdot 1 + 100 \cdot 2 + 10 \cdot 3 + 4 = 1234$

Та же история и для двоичных чисел, только основание там 2 вместо 10.
Что означает 0b101011?

$2^5 \cdot 1 + 2^4 \cdot 0 + 2^3 \cdot 1 + 2^2 \cdot 0 + 2^1 \cdot 1 + 2^0 \cdot 1 = 43$ или
$32 \cdot 1 + 16 \cdot 0 + 8 \cdot 1 + 4 \cdot 0 + 2 \cdot 1 + 1 = 43$

Позиционную нотацию можно противопоставить непозиционной нотации, такой как римская система записи чисел
\footnote{Об эволюции способов записи чисел, см.также: \InSqBrackets{\TAOCPvolII{}, 195--213.}}.
Вероятно, человечество перешло на позиционную нотацию, потому что так проще работать с числами (сложение, умножение, итд)
на бумаге, в ручную.

Действительно, двоичные числа можно складывать, вычитать, итд, точно также, как этому обычно обучают в школах,
только доступны лишь 2 цифры.

Двоичные числа громоздки, когда их используют в исходных кодах и дампах, так что в этих случаях применяется шестнадцатеричная
система.
Используются цифры 0..9 и еще 6 латинских букв: A..F.
Каждая шестнадцатеричная цифра занимает 4 бита или 4 двоичных цифры, так что конвертировать из двоичной системы в
шестнадцатеричную и назад, можно легко вручную, или даже в уме.

\begin{center}
\begin{longtable}{ | l | l | l | }
\hline
\HeaderColor шестнадцатеричная & \HeaderColor двоичная & \HeaderColor десятичная \\
\hline
0	&0000	&0 \\
1	&0001	&1 \\
2	&0010	&2 \\
3	&0011	&3 \\
4	&0100	&4 \\
5	&0101	&5 \\
6	&0110	&6 \\
7	&0111	&7 \\
8	&1000	&8 \\
9	&1001	&9 \\
A	&1010	&10 \\
B	&1011	&11 \\
C	&1100	&12 \\
D	&1101	&13 \\
E	&1110	&14 \\
F	&1111	&15 \\
\hline
\end{longtable}
\end{center}

Как понять, какое основание используется в конкретном месте?

Десятичные числа обычно записываются как есть, т.е., 1234. Но некоторые ассемблеры позволяют подчеркивать
этот факт для ясности, и это число может быть дополнено суффиксом "d": 1234d.

К двоичным числам иногда спереди добавляют префикс "0b": 0b100110111
(В \ac{GCC} для этого есть нестандартное расширение языка
\footnote{\url{https://gcc.gnu.org/onlinedocs/gcc/Binary-constants.html}}).
Есть также еще один способ: суффикс "b", например: 100110111b.
В этой книге я буду пытаться придерживаться префикса "0b" для двоичных чисел.

Шестнадцатеричные числа имеют префикс "0x" в \CCpp и некоторых других \ac{PL}: 0x1234ABCD.
Либо они имеют суффикс "h": 1234ABCDh --- обычно так они представляются в ассемблерах и отладчиках.
Если число начинается с цифры A..F, перед ним добавляется 0: 0ABCDEFh.
Во времена 8-битных домашних компьютеров, был также способ записи чисел используя префикс \$, например, \$ABCD.
В книге я попытаюсь придерживаться префикса "0x" для шестнадцатеричных чисел.

Нужно ли учиться конвертировать числа в уме? Таблицу шестнадцатеричных чисел из одной цифры легко запомнить.
А запоминать б\'{о}льшие числа, наверное, не стоит.

Наверное, чаще всего шестнадцатеричные числа можно увидеть в \ac{URL}-ах.
Так кодируются буквы не из числа латинских.
Например:
\url{https://en.wiktionary.org/wiki/na\%C3\%AFvet\%C3\%A9} это \ac{URL} страницы в Wiktionary о слове \q{naïveté}.

\subsubsection{Восьмеричная система}

Еще одна система, которая в прошлом много использовалась в программировании это восьмеричная: есть 8 цифр (0..7) и каждая
описывает 3 бита, так что легко конвертировать числа туда и назад.
Она почти везде была заменена шестнадцатеричной, но удивительно, в *NIX имеется утилита использующаяся многими людьми,
которая принимает на вход восьмеричное число: \TT{chmod}.

\myindex{UNIX!chmod}
Как знают многие пользователи *NIX, аргумент \TT{chmod} это число из трех цифр. Первая цифра это права владельца файла,
вторая это права группы (которой файл принадлежит), третья для всех остальных.
И каждая цифра может быть представлена в двоичном виде:

\begin{center}
\begin{longtable}{ | l | l | l | }
\hline
\HeaderColor десятичная & \HeaderColor двоичная & \HeaderColor значение \\
\hline
7	&111	&\textbf{rwx} \\
6	&110	&\textbf{rw-} \\
5	&101	&\textbf{r-x} \\
4	&100	&\textbf{r-{}-} \\
3	&011	&\textbf{-wx} \\
2	&010	&\textbf{-w-} \\
1	&001	&\textbf{-{}-x} \\
0	&000	&\textbf{-{}-{}-} \\
\hline
\end{longtable}
\end{center}

Так что каждый бит привязан к флагу: read/write/execute (чтение/запись/исполнение).

И вот почему я вспомнил здесь о \TT{chmod}, это потому что всё число может быть представлено как число в восьмеричной системе.
Для примера возьмем 644.
Когда вы запускаете \TT{chmod 644 file}, вы выставляете права read/write для владельца, права read для группы, и снова,
read для всех остальных.
Сконвертируем число 644 из восьмеричной системы в двоичную, это будет \TT{110100100}, или (в группах по 3 бита) \TT{110 100 100}.

Теперь мы видим, что каждая тройка описывает права для владельца/группы/остальных:
первая это \TT{rw-}, вторая это \TT{r--} и третья это \TT{r--}.

Восьмеричная система была также популярная на старых компьютерах вроде PDP-8, потому что слово там могло содержать 12, 24 или
36 бит, и эти числа делятся на 3, так что выбор восьмеричной системы в той среде был логичен.
Сейчас, все популярные компьютеры имеют размер слова/адреса 16, 32 или 64 бита, и эти числа делятся на 4,
так что шестнадцатеричная система здесь удобнее.

Восьмеричная система поддерживается всеми стандартными компиляторами \CCpp{}.
Это иногда источник недоумения, потому что восьмеричные числа кодируются с нулем вперед, например, 0377 это 255.
И иногда, вы можете сделать опечатку, и написать "09" вместо 9, и компилятор выдаст ошибку.
GCC может выдать что-то вроде:\\
\TT{error: invalid digit "9" in octal constant}.

Также, восьмеричная система популярна в Java: когда IDA показывает строку с непечатаемыми символами,
они кодируются в восьмеричной системе вместо шестнадцатеричной.
\myindex{JAD}
Точно также себя ведет декомпилятор с Java JAD.

\subsubsection{Делимость}

Когда вы видите десятичное число вроде 120, вы можете быстро понять что оно делится на 10, потому что последняя цифра это 0.
Точно также, 123400 делится на 100, потому что две последних цифры это нули.

Точно также, шестнадцатеричное число 0x1230 делится на 0x10 (или 16), 0x123000 делится на 0x1000 (или 4096), итд.

Двоичное число 0b1000101000 делится на 0b1000 (8), итд.

Это свойство можно часто использовать, чтобы быстро понять,
что длина какого-либо блока в памяти выровнена по некоторой границе.
Например, секции в \ac{PE}-файлах почти всегда начинаются с адресов заканчивающихся 3 шестнадцатеричными нулями:
0x41000, 0x10001000, итд.
Причина в том, что почти все секции в \ac{PE} выровнены по границе 0x1000 (4096) байт.

\subsubsection{Арифметика произвольной точности и основание}

\index{RSA}
Арифметика произвольной точности (multi-precision arithmetic) может использовать огромные числа,
которые могут храниться в нескольких байтах.
Например, ключи RSA, и открытые и закрытые, могут занимать до 4096 бит и даже больше.

В \InSqBrackets{\TAOCPvolII, 265} можно найти такую идею: когда вы сохраняете число произвольной точности в нескольких байтах,
всё число может быть представлено как имеющую систему счисления по основанию $2^8=256$, и каждая цифра находится
в соответствующем байте.
Точно также, если вы сохраняете число произвольной точности в нескольких 32-битных целочисленных значениях,
каждая цифра отправляется в каждый 32-битный слот, и вы можете считать что это число записано в системе с основанием $2^{32}$.

\subsubsection{Произношение}

Числа в недесятичных системах счислениях обычно произносятся по одной цифре: ``один-ноль-ноль-один-один-...''.
Слова вроде ``десять'', ``тысяча'', итд, обычно не произносятся, потому что тогда можно спутать с десятичной системой.

\subsubsection{Числа с плавающей запятой}

Чтобы отличать числа с плавающей запятой от целочисленных, часто, в конце добавляют ``.0'',
например $0.0$, $123.0$, итд.

}
\ITA{\input{patterns/numeral_ITA}}
\DE{\input{patterns/numeral_DE}}
\FR{\input{patterns/numeral_FR}}
\PL{\input{patterns/numeral_PL}}

% chapters
\ifdefined\SPANISH
\chapter{Patrones de código}
\fi % SPANISH

\ifdefined\GERMAN
\chapter{Code-Muster}
\fi % GERMAN

\ifdefined\ENGLISH
\chapter{Code Patterns}
\fi % ENGLISH

\ifdefined\ITALIAN
\chapter{Forme di codice}
\fi % ITALIAN

\ifdefined\RUSSIAN
\chapter{Образцы кода}
\fi % RUSSIAN

\ifdefined\BRAZILIAN
\chapter{Padrões de códigos}
\fi % BRAZILIAN

\ifdefined\THAI
\chapter{รูปแบบของโค้ด}
\fi % THAI

\ifdefined\FRENCH
\chapter{Modèle de code}
\fi % FRENCH

\ifdefined\POLISH
\chapter{\PLph{}}
\fi % POLISH

% sections
\EN{\section{The method}

When the author of this book first started learning C and, later, \Cpp, he used to write small pieces of code, compile them,
and then look at the assembly language output. This made it very easy for him to understand what was going on in the code that he had written.
\footnote{In fact, he still does this when he can't understand what a particular bit of code does.}.
He did this so many times that the relationship between the \CCpp code and what the compiler produced was imprinted deeply in his mind.
It's now easy for him to imagine instantly a rough outline of a C code's appearance and function.
Perhaps this technique could be helpful for others.

%There are a lot of examples for both x86/x64 and ARM.
%Those who already familiar with one of architectures, may freely skim over pages.

By the way, there is a great website where you can do the same, with various compilers, instead of installing them on your box.
You can use it as well: \url{https://gcc.godbolt.org/}.

\section*{\Exercises}

When the author of this book studied assembly language, he also often compiled small C functions and then rewrote
them gradually to assembly, trying to make their code as short as possible.
This probably is not worth doing in real-world scenarios today,
because it's hard to compete with the latest compilers in terms of efficiency. It is, however, a very good way to gain a better understanding of assembly.
Feel free, therefore, to take any assembly code from this book and try to make it shorter.
However, don't forget to test what you have written.

% rewrote to show that debug\release and optimisations levels are orthogonal concepts.
\section*{Optimization levels and debug information}

Source code can be compiled by different compilers with various optimization levels.
A typical compiler has about three such levels, where level zero means that optimization is completely disabled.
Optimization can also be targeted towards code size or code speed.
A non-optimizing compiler is faster and produces more understandable (albeit verbose) code,
whereas an optimizing compiler is slower and tries to produce code that runs faster (but is not necessarily more compact).
In addition to optimization levels, a compiler can include some debug information in the resulting file,
producing code that is easy to debug.
One of the important features of the ´debug' code is that it might contain links
between each line of the source code and its respective machine code address.
Optimizing compilers, on the other hand, tend to produce output where entire lines of source code
can be optimized away and thus not even be present in the resulting machine code.
Reverse engineers can encounter either version, simply because some developers turn on the compiler's optimization flags and others do not.
Because of this, we'll try to work on examples of both debug and release versions of the code featured in this book, wherever possible.

Sometimes some pretty ancient compilers are used in this book, in order to get the shortest (or simplest) possible code snippet.
}
\ES{\input{patterns/patterns_opt_dbg_ES}}
\ITA{\input{patterns/patterns_opt_dbg_ITA}}
\PTBR{\input{patterns/patterns_opt_dbg_PTBR}}
\RU{\input{patterns/patterns_opt_dbg_RU}}
\THA{\input{patterns/patterns_opt_dbg_THA}}
\DE{\input{patterns/patterns_opt_dbg_DE}}
\FR{\input{patterns/patterns_opt_dbg_FR}}
\PL{\input{patterns/patterns_opt_dbg_PL}}

\RU{\section{Некоторые базовые понятия}}
\EN{\section{Some basics}}
\DE{\section{Einige Grundlagen}}
\FR{\section{Quelques bases}}
\ES{\section{\ESph{}}}
\ITA{\section{Alcune basi teoriche}}
\PTBR{\section{\PTBRph{}}}
\THA{\section{\THAph{}}}
\PL{\section{\PLph{}}}

% sections:
\EN{\input{patterns/intro_CPU_ISA_EN}}
\ES{\input{patterns/intro_CPU_ISA_ES}}
\ITA{\input{patterns/intro_CPU_ISA_ITA}}
\PTBR{\input{patterns/intro_CPU_ISA_PTBR}}
\RU{\input{patterns/intro_CPU_ISA_RU}}
\DE{\input{patterns/intro_CPU_ISA_DE}}
\FR{\input{patterns/intro_CPU_ISA_FR}}
\PL{\input{patterns/intro_CPU_ISA_PL}}

\EN{\subsection{Numeral Systems}

Humans have become accustomed to a decimal numeral system, probably because almost everyone has 10 fingers.
Nevertheless, the number \q{10} has no significant meaning in science and mathematics.
The natural numeral system in digital electronics is binary: 0 is for an absence of current in the wire, and 1 for presence.
10 in binary is 2 in decimal, 100 in binary is 4 in decimal, and so on.

% This sentence is a bit unweildy - maybe try 'Our ten-digit system would be described as having a radix...' - Renaissance
If the numeral system has 10 digits, it has a \IT{radix} (or \IT{base}) of 10.
The binary numeral system has a \IT{radix} of 2.

Important things to recall:

1) A \IT{number} is a number, while a \IT{digit} is a term from writing systems, and is usually one character

% The original is 'number' is not changed; I think the intent is value, and changed it - Renaissance
2) The value of a number does not change when converted to another radix; only the writing notation for that value has changed (and therefore the way of representing it in \ac{RAM}).

\subsection{Converting From One Radix To Another}

Positional notation is used almost every numerical system. This means that a digit has weight relative to where it is placed inside of the larger number.
If 2 is placed at the rightmost place, it's 2, but if it's placed one digit before rightmost, it's 20.

What does $1234$ stand for?

$10^3 \cdot 1 + 10^2 \cdot 2 + 10^1 \cdot 3 + 1 \cdot 4 = 1234$ or
$1000 \cdot 1 + 100 \cdot 2 + 10 \cdot 3 + 4 = 1234$

It's the same story for binary numbers, but the base is 2 instead of 10.
What does 0b101011 stand for?

$2^5 \cdot 1 + 2^4 \cdot 0 + 2^3 \cdot 1 + 2^2 \cdot 0 + 2^1 \cdot 1 + 2^0 \cdot 1 = 43$ or
$32 \cdot 1 + 16 \cdot 0 + 8 \cdot 1 + 4 \cdot 0 + 2 \cdot 1 + 1 = 43$

There is such a thing as non-positional notation, such as the Roman numeral system.
\footnote{About numeric system evolution, see \InSqBrackets{\TAOCPvolII{}, 195--213.}}.
% Maybe add a sentence to fill in that X is always 10, and is therefore non-positional, even though putting an I before subtracts and after adds, and is in that sense positional
Perhaps, humankind switched to positional notation because it's easier to do basic operations (addition, multiplication, etc.) on paper by hand.

Binary numbers can be added, subtracted and so on in the very same as taught in schools, but only 2 digits are available.

Binary numbers are bulky when represented in source code and dumps, so that is where the hexadecimal numeral system can be useful.
A hexadecimal radix uses the digits 0..9, and also 6 Latin characters: A..F.
Each hexadecimal digit takes 4 bits or 4 binary digits, so it's very easy to convert from binary number to hexadecimal and back, even manually, in one's mind.

\begin{center}
\begin{longtable}{ | l | l | l | }
\hline
\HeaderColor hexadecimal & \HeaderColor binary & \HeaderColor decimal \\
\hline
0	&0000	&0 \\
1	&0001	&1 \\
2	&0010	&2 \\
3	&0011	&3 \\
4	&0100	&4 \\
5	&0101	&5 \\
6	&0110	&6 \\
7	&0111	&7 \\
8	&1000	&8 \\
9	&1001	&9 \\
A	&1010	&10 \\
B	&1011	&11 \\
C	&1100	&12 \\
D	&1101	&13 \\
E	&1110	&14 \\
F	&1111	&15 \\
\hline
\end{longtable}
\end{center}

How can one tell which radix is being used in a specific instance?

Decimal numbers are usually written as is, i.e., 1234. Some assemblers allow an identifier on decimal radix numbers, in which the number would be written with a "d" suffix: 1234d.

Binary numbers are sometimes prepended with the "0b" prefix: 0b100110111 (\ac{GCC} has a non-standard language extension for this\footnote{\url{https://gcc.gnu.org/onlinedocs/gcc/Binary-constants.html}}).
There is also another way: using a "b" suffix, for example: 100110111b.
This book tries to use the "0b" prefix consistently throughout the book for binary numbers.

Hexadecimal numbers are prepended with "0x" prefix in \CCpp and other \ac{PL}s: 0x1234ABCD.
Alternatively, they are given a "h" suffix: 1234ABCDh. This is common way of representing them in assemblers and debuggers.
In this convention, if the number is started with a Latin (A..F) digit, a 0 is added at the beginning: 0ABCDEFh.
There was also convention that was popular in 8-bit home computers era, using \$ prefix, like \$ABCD.
The book will try to stick to "0x" prefix throughout the book for hexadecimal numbers.

Should one learn to convert numbers mentally? A table of 1-digit hexadecimal numbers can easily be memorized.
As for larger numbers, it's probably not worth tormenting yourself.

Perhaps the most visible hexadecimal numbers are in \ac{URL}s.
This is the way that non-Latin characters are encoded.
For example:
\url{https://en.wiktionary.org/wiki/na\%C3\%AFvet\%C3\%A9} is the \ac{URL} of Wiktionary article about \q{naïveté} word.

\subsubsection{Octal Radix}

Another numeral system heavily used in the past of computer programming is octal. In octal there are 8 digits (0..7), and each is mapped to 3 bits, so it's easy to convert numbers back and forth.
It has been superseded by the hexadecimal system almost everywhere, but, surprisingly, there is a *NIX utility, used often by many people, which takes octal numbers as argument: \TT{chmod}.

\myindex{UNIX!chmod}
As many *NIX users know, \TT{chmod} argument can be a number of 3 digits. The first digit represents the rights of the owner of the file (read, write and/or execute), the second is the rights for the group to which the file belongs, and the third is for everyone else.
Each digit that \TT{chmod} takes can be represented in binary form:

\begin{center}
\begin{longtable}{ | l | l | l | }
\hline
\HeaderColor decimal & \HeaderColor binary & \HeaderColor meaning \\
\hline
7	&111	&\textbf{rwx} \\
6	&110	&\textbf{rw-} \\
5	&101	&\textbf{r-x} \\
4	&100	&\textbf{r-{}-} \\
3	&011	&\textbf{-wx} \\
2	&010	&\textbf{-w-} \\
1	&001	&\textbf{-{}-x} \\
0	&000	&\textbf{-{}-{}-} \\
\hline
\end{longtable}
\end{center}

So each bit is mapped to a flag: read/write/execute.

The importance of \TT{chmod} here is that the whole number in argument can be represented as octal number.
Let's take, for example, 644.
When you run \TT{chmod 644 file}, you set read/write permissions for owner, read permissions for group and again, read permissions for everyone else.
If we convert the octal number 644 to binary, it would be \TT{110100100}, or, in groups of 3 bits, \TT{110 100 100}.

Now we see that each triplet describe permissions for owner/group/others: first is \TT{rw-}, second is \TT{r--} and third is \TT{r--}.

The octal numeral system was also popular on old computers like PDP-8, because word there could be 12, 24 or 36 bits, and these numbers are all divisible by 3, so the octal system was natural in that environment.
Nowadays, all popular computers employ word/address sizes of 16, 32 or 64 bits, and these numbers are all divisible by 4, so the hexadecimal system is more natural there.

The octal numeral system is supported by all standard \CCpp compilers.
This is a source of confusion sometimes, because octal numbers are encoded with a zero prepended, for example, 0377 is 255.
Sometimes, you might make a typo and write "09" instead of 9, and the compiler would report an error.
GCC might report something like this:\\
\TT{error: invalid digit "9" in octal constant}.

Also, the octal system is somewhat popular in Java. When the IDA shows Java strings with non-printable characters,
they are encoded in the octal system instead of hexadecimal.
\myindex{JAD}
The JAD Java decompiler behaves the same way.

\subsubsection{Divisibility}

When you see a decimal number like 120, you can quickly deduce that it's divisible by 10, because the last digit is zero.
In the same way, 123400 is divisible by 100, because the two last digits are zeros.

Likewise, the hexadecimal number 0x1230 is divisible by 0x10 (or 16), 0x123000 is divisible by 0x1000 (or 4096), etc.

The binary number 0b1000101000 is divisible by 0b1000 (8), etc.

This property can often be used to quickly realize if the size of some block in memory is padded to some boundary.
For example, sections in \ac{PE} files are almost always started at addresses ending with 3 hexadecimal zeros: 0x41000, 0x10001000, etc.
The reason behind this is the fact that almost all \ac{PE} sections are padded to a boundary of 0x1000 (4096) bytes.

\subsubsection{Multi-Precision Arithmetic and Radix}

\index{RSA}
Multi-precision arithmetic can use huge numbers, and each one may be stored in several bytes.
For example, RSA keys, both public and private, span up to 4096 bits, and maybe even more.

% I'm not sure how to change this, but the normal format for quoting would be just to mention the author or book, and footnote to the full reference
In \InSqBrackets{\TAOCPvolII, 265} we find the following idea: when you store a multi-precision number in several bytes,
the whole number can be represented as having a radix of $2^8=256$, and each digit goes to the corresponding byte.
Likewise, if you store a multi-precision number in several 32-bit integer values, each digit goes to each 32-bit slot,
and you may think about this number as stored in radix of $2^{32}$.

\subsubsection{How to Pronounce Non-Decimal Numbers}

Numbers in a non-decimal base are usually pronounced by digit by digit: ``one-zero-zero-one-one-...''.
Words like ``ten'' and ``thousand'' are usually not pronounced, to prevent confusion with the decimal base system.

\subsubsection{Floating point numbers}

To distinguish floating point numbers from integers, they are usually written with ``.0'' at the end,
like $0.0$, $123.0$, etc.
}
\RU{\subsection{Представление чисел}

Люди привыкли к десятичной системе счисления вероятно потому что почти у каждого есть по 10 пальцев.
Тем не менее, число 10 не имеет особого значения в науке и математике.
Двоичная система естествена для цифровой электроники: 0 означает отсутствие тока в проводе и 1 --- его присутствие.
10 в двоичной системе это 2 в десятичной; 100 в двоичной это 4 в десятичной, итд.

Если в системе счисления есть 10 цифр, её \IT{основание} или \IT{radix} это 10.
Двоичная система имеет \IT{основание} 2.

Важные вещи, которые полезно вспомнить:
1) \IT{число} это число, в то время как \IT{цифра} это термин из системы письменности, и это обычно один символ;
2) само число не меняется, когда конвертируется из одного основания в другое: меняется способ его записи (или представления
в памяти).

Как сконвертировать число из одного основания в другое?

Позиционная нотация используется почти везде, это означает, что всякая цифра имеет свой вес, в зависимости от её расположения
внутри числа.
Если 2 расположена в самом последнем месте справа, это 2.
Если она расположена в месте перед последним, это 20.

Что означает $1234$?

$10^3 \cdot 1 + 10^2 \cdot 2 + 10^1 \cdot 3 + 1 \cdot 4$ = 1234 или
$1000 \cdot 1 + 100 \cdot 2 + 10 \cdot 3 + 4 = 1234$

Та же история и для двоичных чисел, только основание там 2 вместо 10.
Что означает 0b101011?

$2^5 \cdot 1 + 2^4 \cdot 0 + 2^3 \cdot 1 + 2^2 \cdot 0 + 2^1 \cdot 1 + 2^0 \cdot 1 = 43$ или
$32 \cdot 1 + 16 \cdot 0 + 8 \cdot 1 + 4 \cdot 0 + 2 \cdot 1 + 1 = 43$

Позиционную нотацию можно противопоставить непозиционной нотации, такой как римская система записи чисел
\footnote{Об эволюции способов записи чисел, см.также: \InSqBrackets{\TAOCPvolII{}, 195--213.}}.
Вероятно, человечество перешло на позиционную нотацию, потому что так проще работать с числами (сложение, умножение, итд)
на бумаге, в ручную.

Действительно, двоичные числа можно складывать, вычитать, итд, точно также, как этому обычно обучают в школах,
только доступны лишь 2 цифры.

Двоичные числа громоздки, когда их используют в исходных кодах и дампах, так что в этих случаях применяется шестнадцатеричная
система.
Используются цифры 0..9 и еще 6 латинских букв: A..F.
Каждая шестнадцатеричная цифра занимает 4 бита или 4 двоичных цифры, так что конвертировать из двоичной системы в
шестнадцатеричную и назад, можно легко вручную, или даже в уме.

\begin{center}
\begin{longtable}{ | l | l | l | }
\hline
\HeaderColor шестнадцатеричная & \HeaderColor двоичная & \HeaderColor десятичная \\
\hline
0	&0000	&0 \\
1	&0001	&1 \\
2	&0010	&2 \\
3	&0011	&3 \\
4	&0100	&4 \\
5	&0101	&5 \\
6	&0110	&6 \\
7	&0111	&7 \\
8	&1000	&8 \\
9	&1001	&9 \\
A	&1010	&10 \\
B	&1011	&11 \\
C	&1100	&12 \\
D	&1101	&13 \\
E	&1110	&14 \\
F	&1111	&15 \\
\hline
\end{longtable}
\end{center}

Как понять, какое основание используется в конкретном месте?

Десятичные числа обычно записываются как есть, т.е., 1234. Но некоторые ассемблеры позволяют подчеркивать
этот факт для ясности, и это число может быть дополнено суффиксом "d": 1234d.

К двоичным числам иногда спереди добавляют префикс "0b": 0b100110111
(В \ac{GCC} для этого есть нестандартное расширение языка
\footnote{\url{https://gcc.gnu.org/onlinedocs/gcc/Binary-constants.html}}).
Есть также еще один способ: суффикс "b", например: 100110111b.
В этой книге я буду пытаться придерживаться префикса "0b" для двоичных чисел.

Шестнадцатеричные числа имеют префикс "0x" в \CCpp и некоторых других \ac{PL}: 0x1234ABCD.
Либо они имеют суффикс "h": 1234ABCDh --- обычно так они представляются в ассемблерах и отладчиках.
Если число начинается с цифры A..F, перед ним добавляется 0: 0ABCDEFh.
Во времена 8-битных домашних компьютеров, был также способ записи чисел используя префикс \$, например, \$ABCD.
В книге я попытаюсь придерживаться префикса "0x" для шестнадцатеричных чисел.

Нужно ли учиться конвертировать числа в уме? Таблицу шестнадцатеричных чисел из одной цифры легко запомнить.
А запоминать б\'{о}льшие числа, наверное, не стоит.

Наверное, чаще всего шестнадцатеричные числа можно увидеть в \ac{URL}-ах.
Так кодируются буквы не из числа латинских.
Например:
\url{https://en.wiktionary.org/wiki/na\%C3\%AFvet\%C3\%A9} это \ac{URL} страницы в Wiktionary о слове \q{naïveté}.

\subsubsection{Восьмеричная система}

Еще одна система, которая в прошлом много использовалась в программировании это восьмеричная: есть 8 цифр (0..7) и каждая
описывает 3 бита, так что легко конвертировать числа туда и назад.
Она почти везде была заменена шестнадцатеричной, но удивительно, в *NIX имеется утилита использующаяся многими людьми,
которая принимает на вход восьмеричное число: \TT{chmod}.

\myindex{UNIX!chmod}
Как знают многие пользователи *NIX, аргумент \TT{chmod} это число из трех цифр. Первая цифра это права владельца файла,
вторая это права группы (которой файл принадлежит), третья для всех остальных.
И каждая цифра может быть представлена в двоичном виде:

\begin{center}
\begin{longtable}{ | l | l | l | }
\hline
\HeaderColor десятичная & \HeaderColor двоичная & \HeaderColor значение \\
\hline
7	&111	&\textbf{rwx} \\
6	&110	&\textbf{rw-} \\
5	&101	&\textbf{r-x} \\
4	&100	&\textbf{r-{}-} \\
3	&011	&\textbf{-wx} \\
2	&010	&\textbf{-w-} \\
1	&001	&\textbf{-{}-x} \\
0	&000	&\textbf{-{}-{}-} \\
\hline
\end{longtable}
\end{center}

Так что каждый бит привязан к флагу: read/write/execute (чтение/запись/исполнение).

И вот почему я вспомнил здесь о \TT{chmod}, это потому что всё число может быть представлено как число в восьмеричной системе.
Для примера возьмем 644.
Когда вы запускаете \TT{chmod 644 file}, вы выставляете права read/write для владельца, права read для группы, и снова,
read для всех остальных.
Сконвертируем число 644 из восьмеричной системы в двоичную, это будет \TT{110100100}, или (в группах по 3 бита) \TT{110 100 100}.

Теперь мы видим, что каждая тройка описывает права для владельца/группы/остальных:
первая это \TT{rw-}, вторая это \TT{r--} и третья это \TT{r--}.

Восьмеричная система была также популярная на старых компьютерах вроде PDP-8, потому что слово там могло содержать 12, 24 или
36 бит, и эти числа делятся на 3, так что выбор восьмеричной системы в той среде был логичен.
Сейчас, все популярные компьютеры имеют размер слова/адреса 16, 32 или 64 бита, и эти числа делятся на 4,
так что шестнадцатеричная система здесь удобнее.

Восьмеричная система поддерживается всеми стандартными компиляторами \CCpp{}.
Это иногда источник недоумения, потому что восьмеричные числа кодируются с нулем вперед, например, 0377 это 255.
И иногда, вы можете сделать опечатку, и написать "09" вместо 9, и компилятор выдаст ошибку.
GCC может выдать что-то вроде:\\
\TT{error: invalid digit "9" in octal constant}.

Также, восьмеричная система популярна в Java: когда IDA показывает строку с непечатаемыми символами,
они кодируются в восьмеричной системе вместо шестнадцатеричной.
\myindex{JAD}
Точно также себя ведет декомпилятор с Java JAD.

\subsubsection{Делимость}

Когда вы видите десятичное число вроде 120, вы можете быстро понять что оно делится на 10, потому что последняя цифра это 0.
Точно также, 123400 делится на 100, потому что две последних цифры это нули.

Точно также, шестнадцатеричное число 0x1230 делится на 0x10 (или 16), 0x123000 делится на 0x1000 (или 4096), итд.

Двоичное число 0b1000101000 делится на 0b1000 (8), итд.

Это свойство можно часто использовать, чтобы быстро понять,
что длина какого-либо блока в памяти выровнена по некоторой границе.
Например, секции в \ac{PE}-файлах почти всегда начинаются с адресов заканчивающихся 3 шестнадцатеричными нулями:
0x41000, 0x10001000, итд.
Причина в том, что почти все секции в \ac{PE} выровнены по границе 0x1000 (4096) байт.

\subsubsection{Арифметика произвольной точности и основание}

\index{RSA}
Арифметика произвольной точности (multi-precision arithmetic) может использовать огромные числа,
которые могут храниться в нескольких байтах.
Например, ключи RSA, и открытые и закрытые, могут занимать до 4096 бит и даже больше.

В \InSqBrackets{\TAOCPvolII, 265} можно найти такую идею: когда вы сохраняете число произвольной точности в нескольких байтах,
всё число может быть представлено как имеющую систему счисления по основанию $2^8=256$, и каждая цифра находится
в соответствующем байте.
Точно также, если вы сохраняете число произвольной точности в нескольких 32-битных целочисленных значениях,
каждая цифра отправляется в каждый 32-битный слот, и вы можете считать что это число записано в системе с основанием $2^{32}$.

\subsubsection{Произношение}

Числа в недесятичных системах счислениях обычно произносятся по одной цифре: ``один-ноль-ноль-один-один-...''.
Слова вроде ``десять'', ``тысяча'', итд, обычно не произносятся, потому что тогда можно спутать с десятичной системой.

\subsubsection{Числа с плавающей запятой}

Чтобы отличать числа с плавающей запятой от целочисленных, часто, в конце добавляют ``.0'',
например $0.0$, $123.0$, итд.

}
\ITA{\input{patterns/numeral_ITA}}
\DE{\input{patterns/numeral_DE}}
\FR{\input{patterns/numeral_FR}}
\PL{\input{patterns/numeral_PL}}

% chapters
\ifdefined\SPANISH
\chapter{Patrones de código}
\fi % SPANISH

\ifdefined\GERMAN
\chapter{Code-Muster}
\fi % GERMAN

\ifdefined\ENGLISH
\chapter{Code Patterns}
\fi % ENGLISH

\ifdefined\ITALIAN
\chapter{Forme di codice}
\fi % ITALIAN

\ifdefined\RUSSIAN
\chapter{Образцы кода}
\fi % RUSSIAN

\ifdefined\BRAZILIAN
\chapter{Padrões de códigos}
\fi % BRAZILIAN

\ifdefined\THAI
\chapter{รูปแบบของโค้ด}
\fi % THAI

\ifdefined\FRENCH
\chapter{Modèle de code}
\fi % FRENCH

\ifdefined\POLISH
\chapter{\PLph{}}
\fi % POLISH

% sections
\EN{\input{patterns/patterns_opt_dbg_EN}}
\ES{\input{patterns/patterns_opt_dbg_ES}}
\ITA{\input{patterns/patterns_opt_dbg_ITA}}
\PTBR{\input{patterns/patterns_opt_dbg_PTBR}}
\RU{\input{patterns/patterns_opt_dbg_RU}}
\THA{\input{patterns/patterns_opt_dbg_THA}}
\DE{\input{patterns/patterns_opt_dbg_DE}}
\FR{\input{patterns/patterns_opt_dbg_FR}}
\PL{\input{patterns/patterns_opt_dbg_PL}}

\RU{\section{Некоторые базовые понятия}}
\EN{\section{Some basics}}
\DE{\section{Einige Grundlagen}}
\FR{\section{Quelques bases}}
\ES{\section{\ESph{}}}
\ITA{\section{Alcune basi teoriche}}
\PTBR{\section{\PTBRph{}}}
\THA{\section{\THAph{}}}
\PL{\section{\PLph{}}}

% sections:
\EN{\input{patterns/intro_CPU_ISA_EN}}
\ES{\input{patterns/intro_CPU_ISA_ES}}
\ITA{\input{patterns/intro_CPU_ISA_ITA}}
\PTBR{\input{patterns/intro_CPU_ISA_PTBR}}
\RU{\input{patterns/intro_CPU_ISA_RU}}
\DE{\input{patterns/intro_CPU_ISA_DE}}
\FR{\input{patterns/intro_CPU_ISA_FR}}
\PL{\input{patterns/intro_CPU_ISA_PL}}

\EN{\input{patterns/numeral_EN}}
\RU{\input{patterns/numeral_RU}}
\ITA{\input{patterns/numeral_ITA}}
\DE{\input{patterns/numeral_DE}}
\FR{\input{patterns/numeral_FR}}
\PL{\input{patterns/numeral_PL}}

% chapters
\input{patterns/00_empty/main}
\input{patterns/011_ret/main}
\input{patterns/01_helloworld/main}
\input{patterns/015_prolog_epilogue/main}
\input{patterns/02_stack/main}
\input{patterns/03_printf/main}
\input{patterns/04_scanf/main}
\input{patterns/05_passing_arguments/main}
\input{patterns/06_return_results/main}
\input{patterns/061_pointers/main}
\input{patterns/065_GOTO/main}
\input{patterns/07_jcc/main}
\input{patterns/08_switch/main}
\input{patterns/09_loops/main}
\input{patterns/10_strings/main}
\input{patterns/11_arith_optimizations/main}
\input{patterns/12_FPU/main}
\input{patterns/13_arrays/main}
\input{patterns/14_bitfields/main}
\EN{\input{patterns/145_LCG/main_EN}}
\RU{\input{patterns/145_LCG/main_RU}}
\input{patterns/15_structs/main}
\input{patterns/17_unions/main}
\input{patterns/18_pointers_to_functions/main}
\input{patterns/185_64bit_in_32_env/main}

\EN{\input{patterns/19_SIMD/main_EN}}
\RU{\input{patterns/19_SIMD/main_RU}}
\DE{\input{patterns/19_SIMD/main_DE}}

\EN{\input{patterns/20_x64/main_EN}}
\RU{\input{patterns/20_x64/main_RU}}

\EN{\input{patterns/205_floating_SIMD/main_EN}}
\RU{\input{patterns/205_floating_SIMD/main_RU}}
\DE{\input{patterns/205_floating_SIMD/main_DE}}

\EN{\input{patterns/ARM/main_EN}}
\RU{\input{patterns/ARM/main_RU}}
\DE{\input{patterns/ARM/main_DE}}

\input{patterns/MIPS/main}

\ifdefined\SPANISH
\chapter{Patrones de código}
\fi % SPANISH

\ifdefined\GERMAN
\chapter{Code-Muster}
\fi % GERMAN

\ifdefined\ENGLISH
\chapter{Code Patterns}
\fi % ENGLISH

\ifdefined\ITALIAN
\chapter{Forme di codice}
\fi % ITALIAN

\ifdefined\RUSSIAN
\chapter{Образцы кода}
\fi % RUSSIAN

\ifdefined\BRAZILIAN
\chapter{Padrões de códigos}
\fi % BRAZILIAN

\ifdefined\THAI
\chapter{รูปแบบของโค้ด}
\fi % THAI

\ifdefined\FRENCH
\chapter{Modèle de code}
\fi % FRENCH

\ifdefined\POLISH
\chapter{\PLph{}}
\fi % POLISH

% sections
\EN{\input{patterns/patterns_opt_dbg_EN}}
\ES{\input{patterns/patterns_opt_dbg_ES}}
\ITA{\input{patterns/patterns_opt_dbg_ITA}}
\PTBR{\input{patterns/patterns_opt_dbg_PTBR}}
\RU{\input{patterns/patterns_opt_dbg_RU}}
\THA{\input{patterns/patterns_opt_dbg_THA}}
\DE{\input{patterns/patterns_opt_dbg_DE}}
\FR{\input{patterns/patterns_opt_dbg_FR}}
\PL{\input{patterns/patterns_opt_dbg_PL}}

\RU{\section{Некоторые базовые понятия}}
\EN{\section{Some basics}}
\DE{\section{Einige Grundlagen}}
\FR{\section{Quelques bases}}
\ES{\section{\ESph{}}}
\ITA{\section{Alcune basi teoriche}}
\PTBR{\section{\PTBRph{}}}
\THA{\section{\THAph{}}}
\PL{\section{\PLph{}}}

% sections:
\EN{\input{patterns/intro_CPU_ISA_EN}}
\ES{\input{patterns/intro_CPU_ISA_ES}}
\ITA{\input{patterns/intro_CPU_ISA_ITA}}
\PTBR{\input{patterns/intro_CPU_ISA_PTBR}}
\RU{\input{patterns/intro_CPU_ISA_RU}}
\DE{\input{patterns/intro_CPU_ISA_DE}}
\FR{\input{patterns/intro_CPU_ISA_FR}}
\PL{\input{patterns/intro_CPU_ISA_PL}}

\EN{\input{patterns/numeral_EN}}
\RU{\input{patterns/numeral_RU}}
\ITA{\input{patterns/numeral_ITA}}
\DE{\input{patterns/numeral_DE}}
\FR{\input{patterns/numeral_FR}}
\PL{\input{patterns/numeral_PL}}

% chapters
\input{patterns/00_empty/main}
\input{patterns/011_ret/main}
\input{patterns/01_helloworld/main}
\input{patterns/015_prolog_epilogue/main}
\input{patterns/02_stack/main}
\input{patterns/03_printf/main}
\input{patterns/04_scanf/main}
\input{patterns/05_passing_arguments/main}
\input{patterns/06_return_results/main}
\input{patterns/061_pointers/main}
\input{patterns/065_GOTO/main}
\input{patterns/07_jcc/main}
\input{patterns/08_switch/main}
\input{patterns/09_loops/main}
\input{patterns/10_strings/main}
\input{patterns/11_arith_optimizations/main}
\input{patterns/12_FPU/main}
\input{patterns/13_arrays/main}
\input{patterns/14_bitfields/main}
\EN{\input{patterns/145_LCG/main_EN}}
\RU{\input{patterns/145_LCG/main_RU}}
\input{patterns/15_structs/main}
\input{patterns/17_unions/main}
\input{patterns/18_pointers_to_functions/main}
\input{patterns/185_64bit_in_32_env/main}

\EN{\input{patterns/19_SIMD/main_EN}}
\RU{\input{patterns/19_SIMD/main_RU}}
\DE{\input{patterns/19_SIMD/main_DE}}

\EN{\input{patterns/20_x64/main_EN}}
\RU{\input{patterns/20_x64/main_RU}}

\EN{\input{patterns/205_floating_SIMD/main_EN}}
\RU{\input{patterns/205_floating_SIMD/main_RU}}
\DE{\input{patterns/205_floating_SIMD/main_DE}}

\EN{\input{patterns/ARM/main_EN}}
\RU{\input{patterns/ARM/main_RU}}
\DE{\input{patterns/ARM/main_DE}}

\input{patterns/MIPS/main}

\ifdefined\SPANISH
\chapter{Patrones de código}
\fi % SPANISH

\ifdefined\GERMAN
\chapter{Code-Muster}
\fi % GERMAN

\ifdefined\ENGLISH
\chapter{Code Patterns}
\fi % ENGLISH

\ifdefined\ITALIAN
\chapter{Forme di codice}
\fi % ITALIAN

\ifdefined\RUSSIAN
\chapter{Образцы кода}
\fi % RUSSIAN

\ifdefined\BRAZILIAN
\chapter{Padrões de códigos}
\fi % BRAZILIAN

\ifdefined\THAI
\chapter{รูปแบบของโค้ด}
\fi % THAI

\ifdefined\FRENCH
\chapter{Modèle de code}
\fi % FRENCH

\ifdefined\POLISH
\chapter{\PLph{}}
\fi % POLISH

% sections
\EN{\input{patterns/patterns_opt_dbg_EN}}
\ES{\input{patterns/patterns_opt_dbg_ES}}
\ITA{\input{patterns/patterns_opt_dbg_ITA}}
\PTBR{\input{patterns/patterns_opt_dbg_PTBR}}
\RU{\input{patterns/patterns_opt_dbg_RU}}
\THA{\input{patterns/patterns_opt_dbg_THA}}
\DE{\input{patterns/patterns_opt_dbg_DE}}
\FR{\input{patterns/patterns_opt_dbg_FR}}
\PL{\input{patterns/patterns_opt_dbg_PL}}

\RU{\section{Некоторые базовые понятия}}
\EN{\section{Some basics}}
\DE{\section{Einige Grundlagen}}
\FR{\section{Quelques bases}}
\ES{\section{\ESph{}}}
\ITA{\section{Alcune basi teoriche}}
\PTBR{\section{\PTBRph{}}}
\THA{\section{\THAph{}}}
\PL{\section{\PLph{}}}

% sections:
\EN{\input{patterns/intro_CPU_ISA_EN}}
\ES{\input{patterns/intro_CPU_ISA_ES}}
\ITA{\input{patterns/intro_CPU_ISA_ITA}}
\PTBR{\input{patterns/intro_CPU_ISA_PTBR}}
\RU{\input{patterns/intro_CPU_ISA_RU}}
\DE{\input{patterns/intro_CPU_ISA_DE}}
\FR{\input{patterns/intro_CPU_ISA_FR}}
\PL{\input{patterns/intro_CPU_ISA_PL}}

\EN{\input{patterns/numeral_EN}}
\RU{\input{patterns/numeral_RU}}
\ITA{\input{patterns/numeral_ITA}}
\DE{\input{patterns/numeral_DE}}
\FR{\input{patterns/numeral_FR}}
\PL{\input{patterns/numeral_PL}}

% chapters
\input{patterns/00_empty/main}
\input{patterns/011_ret/main}
\input{patterns/01_helloworld/main}
\input{patterns/015_prolog_epilogue/main}
\input{patterns/02_stack/main}
\input{patterns/03_printf/main}
\input{patterns/04_scanf/main}
\input{patterns/05_passing_arguments/main}
\input{patterns/06_return_results/main}
\input{patterns/061_pointers/main}
\input{patterns/065_GOTO/main}
\input{patterns/07_jcc/main}
\input{patterns/08_switch/main}
\input{patterns/09_loops/main}
\input{patterns/10_strings/main}
\input{patterns/11_arith_optimizations/main}
\input{patterns/12_FPU/main}
\input{patterns/13_arrays/main}
\input{patterns/14_bitfields/main}
\EN{\input{patterns/145_LCG/main_EN}}
\RU{\input{patterns/145_LCG/main_RU}}
\input{patterns/15_structs/main}
\input{patterns/17_unions/main}
\input{patterns/18_pointers_to_functions/main}
\input{patterns/185_64bit_in_32_env/main}

\EN{\input{patterns/19_SIMD/main_EN}}
\RU{\input{patterns/19_SIMD/main_RU}}
\DE{\input{patterns/19_SIMD/main_DE}}

\EN{\input{patterns/20_x64/main_EN}}
\RU{\input{patterns/20_x64/main_RU}}

\EN{\input{patterns/205_floating_SIMD/main_EN}}
\RU{\input{patterns/205_floating_SIMD/main_RU}}
\DE{\input{patterns/205_floating_SIMD/main_DE}}

\EN{\input{patterns/ARM/main_EN}}
\RU{\input{patterns/ARM/main_RU}}
\DE{\input{patterns/ARM/main_DE}}

\input{patterns/MIPS/main}

\ifdefined\SPANISH
\chapter{Patrones de código}
\fi % SPANISH

\ifdefined\GERMAN
\chapter{Code-Muster}
\fi % GERMAN

\ifdefined\ENGLISH
\chapter{Code Patterns}
\fi % ENGLISH

\ifdefined\ITALIAN
\chapter{Forme di codice}
\fi % ITALIAN

\ifdefined\RUSSIAN
\chapter{Образцы кода}
\fi % RUSSIAN

\ifdefined\BRAZILIAN
\chapter{Padrões de códigos}
\fi % BRAZILIAN

\ifdefined\THAI
\chapter{รูปแบบของโค้ด}
\fi % THAI

\ifdefined\FRENCH
\chapter{Modèle de code}
\fi % FRENCH

\ifdefined\POLISH
\chapter{\PLph{}}
\fi % POLISH

% sections
\EN{\input{patterns/patterns_opt_dbg_EN}}
\ES{\input{patterns/patterns_opt_dbg_ES}}
\ITA{\input{patterns/patterns_opt_dbg_ITA}}
\PTBR{\input{patterns/patterns_opt_dbg_PTBR}}
\RU{\input{patterns/patterns_opt_dbg_RU}}
\THA{\input{patterns/patterns_opt_dbg_THA}}
\DE{\input{patterns/patterns_opt_dbg_DE}}
\FR{\input{patterns/patterns_opt_dbg_FR}}
\PL{\input{patterns/patterns_opt_dbg_PL}}

\RU{\section{Некоторые базовые понятия}}
\EN{\section{Some basics}}
\DE{\section{Einige Grundlagen}}
\FR{\section{Quelques bases}}
\ES{\section{\ESph{}}}
\ITA{\section{Alcune basi teoriche}}
\PTBR{\section{\PTBRph{}}}
\THA{\section{\THAph{}}}
\PL{\section{\PLph{}}}

% sections:
\EN{\input{patterns/intro_CPU_ISA_EN}}
\ES{\input{patterns/intro_CPU_ISA_ES}}
\ITA{\input{patterns/intro_CPU_ISA_ITA}}
\PTBR{\input{patterns/intro_CPU_ISA_PTBR}}
\RU{\input{patterns/intro_CPU_ISA_RU}}
\DE{\input{patterns/intro_CPU_ISA_DE}}
\FR{\input{patterns/intro_CPU_ISA_FR}}
\PL{\input{patterns/intro_CPU_ISA_PL}}

\EN{\input{patterns/numeral_EN}}
\RU{\input{patterns/numeral_RU}}
\ITA{\input{patterns/numeral_ITA}}
\DE{\input{patterns/numeral_DE}}
\FR{\input{patterns/numeral_FR}}
\PL{\input{patterns/numeral_PL}}

% chapters
\input{patterns/00_empty/main}
\input{patterns/011_ret/main}
\input{patterns/01_helloworld/main}
\input{patterns/015_prolog_epilogue/main}
\input{patterns/02_stack/main}
\input{patterns/03_printf/main}
\input{patterns/04_scanf/main}
\input{patterns/05_passing_arguments/main}
\input{patterns/06_return_results/main}
\input{patterns/061_pointers/main}
\input{patterns/065_GOTO/main}
\input{patterns/07_jcc/main}
\input{patterns/08_switch/main}
\input{patterns/09_loops/main}
\input{patterns/10_strings/main}
\input{patterns/11_arith_optimizations/main}
\input{patterns/12_FPU/main}
\input{patterns/13_arrays/main}
\input{patterns/14_bitfields/main}
\EN{\input{patterns/145_LCG/main_EN}}
\RU{\input{patterns/145_LCG/main_RU}}
\input{patterns/15_structs/main}
\input{patterns/17_unions/main}
\input{patterns/18_pointers_to_functions/main}
\input{patterns/185_64bit_in_32_env/main}

\EN{\input{patterns/19_SIMD/main_EN}}
\RU{\input{patterns/19_SIMD/main_RU}}
\DE{\input{patterns/19_SIMD/main_DE}}

\EN{\input{patterns/20_x64/main_EN}}
\RU{\input{patterns/20_x64/main_RU}}

\EN{\input{patterns/205_floating_SIMD/main_EN}}
\RU{\input{patterns/205_floating_SIMD/main_RU}}
\DE{\input{patterns/205_floating_SIMD/main_DE}}

\EN{\input{patterns/ARM/main_EN}}
\RU{\input{patterns/ARM/main_RU}}
\DE{\input{patterns/ARM/main_DE}}

\input{patterns/MIPS/main}

\ifdefined\SPANISH
\chapter{Patrones de código}
\fi % SPANISH

\ifdefined\GERMAN
\chapter{Code-Muster}
\fi % GERMAN

\ifdefined\ENGLISH
\chapter{Code Patterns}
\fi % ENGLISH

\ifdefined\ITALIAN
\chapter{Forme di codice}
\fi % ITALIAN

\ifdefined\RUSSIAN
\chapter{Образцы кода}
\fi % RUSSIAN

\ifdefined\BRAZILIAN
\chapter{Padrões de códigos}
\fi % BRAZILIAN

\ifdefined\THAI
\chapter{รูปแบบของโค้ด}
\fi % THAI

\ifdefined\FRENCH
\chapter{Modèle de code}
\fi % FRENCH

\ifdefined\POLISH
\chapter{\PLph{}}
\fi % POLISH

% sections
\EN{\input{patterns/patterns_opt_dbg_EN}}
\ES{\input{patterns/patterns_opt_dbg_ES}}
\ITA{\input{patterns/patterns_opt_dbg_ITA}}
\PTBR{\input{patterns/patterns_opt_dbg_PTBR}}
\RU{\input{patterns/patterns_opt_dbg_RU}}
\THA{\input{patterns/patterns_opt_dbg_THA}}
\DE{\input{patterns/patterns_opt_dbg_DE}}
\FR{\input{patterns/patterns_opt_dbg_FR}}
\PL{\input{patterns/patterns_opt_dbg_PL}}

\RU{\section{Некоторые базовые понятия}}
\EN{\section{Some basics}}
\DE{\section{Einige Grundlagen}}
\FR{\section{Quelques bases}}
\ES{\section{\ESph{}}}
\ITA{\section{Alcune basi teoriche}}
\PTBR{\section{\PTBRph{}}}
\THA{\section{\THAph{}}}
\PL{\section{\PLph{}}}

% sections:
\EN{\input{patterns/intro_CPU_ISA_EN}}
\ES{\input{patterns/intro_CPU_ISA_ES}}
\ITA{\input{patterns/intro_CPU_ISA_ITA}}
\PTBR{\input{patterns/intro_CPU_ISA_PTBR}}
\RU{\input{patterns/intro_CPU_ISA_RU}}
\DE{\input{patterns/intro_CPU_ISA_DE}}
\FR{\input{patterns/intro_CPU_ISA_FR}}
\PL{\input{patterns/intro_CPU_ISA_PL}}

\EN{\input{patterns/numeral_EN}}
\RU{\input{patterns/numeral_RU}}
\ITA{\input{patterns/numeral_ITA}}
\DE{\input{patterns/numeral_DE}}
\FR{\input{patterns/numeral_FR}}
\PL{\input{patterns/numeral_PL}}

% chapters
\input{patterns/00_empty/main}
\input{patterns/011_ret/main}
\input{patterns/01_helloworld/main}
\input{patterns/015_prolog_epilogue/main}
\input{patterns/02_stack/main}
\input{patterns/03_printf/main}
\input{patterns/04_scanf/main}
\input{patterns/05_passing_arguments/main}
\input{patterns/06_return_results/main}
\input{patterns/061_pointers/main}
\input{patterns/065_GOTO/main}
\input{patterns/07_jcc/main}
\input{patterns/08_switch/main}
\input{patterns/09_loops/main}
\input{patterns/10_strings/main}
\input{patterns/11_arith_optimizations/main}
\input{patterns/12_FPU/main}
\input{patterns/13_arrays/main}
\input{patterns/14_bitfields/main}
\EN{\input{patterns/145_LCG/main_EN}}
\RU{\input{patterns/145_LCG/main_RU}}
\input{patterns/15_structs/main}
\input{patterns/17_unions/main}
\input{patterns/18_pointers_to_functions/main}
\input{patterns/185_64bit_in_32_env/main}

\EN{\input{patterns/19_SIMD/main_EN}}
\RU{\input{patterns/19_SIMD/main_RU}}
\DE{\input{patterns/19_SIMD/main_DE}}

\EN{\input{patterns/20_x64/main_EN}}
\RU{\input{patterns/20_x64/main_RU}}

\EN{\input{patterns/205_floating_SIMD/main_EN}}
\RU{\input{patterns/205_floating_SIMD/main_RU}}
\DE{\input{patterns/205_floating_SIMD/main_DE}}

\EN{\input{patterns/ARM/main_EN}}
\RU{\input{patterns/ARM/main_RU}}
\DE{\input{patterns/ARM/main_DE}}

\input{patterns/MIPS/main}

\ifdefined\SPANISH
\chapter{Patrones de código}
\fi % SPANISH

\ifdefined\GERMAN
\chapter{Code-Muster}
\fi % GERMAN

\ifdefined\ENGLISH
\chapter{Code Patterns}
\fi % ENGLISH

\ifdefined\ITALIAN
\chapter{Forme di codice}
\fi % ITALIAN

\ifdefined\RUSSIAN
\chapter{Образцы кода}
\fi % RUSSIAN

\ifdefined\BRAZILIAN
\chapter{Padrões de códigos}
\fi % BRAZILIAN

\ifdefined\THAI
\chapter{รูปแบบของโค้ด}
\fi % THAI

\ifdefined\FRENCH
\chapter{Modèle de code}
\fi % FRENCH

\ifdefined\POLISH
\chapter{\PLph{}}
\fi % POLISH

% sections
\EN{\input{patterns/patterns_opt_dbg_EN}}
\ES{\input{patterns/patterns_opt_dbg_ES}}
\ITA{\input{patterns/patterns_opt_dbg_ITA}}
\PTBR{\input{patterns/patterns_opt_dbg_PTBR}}
\RU{\input{patterns/patterns_opt_dbg_RU}}
\THA{\input{patterns/patterns_opt_dbg_THA}}
\DE{\input{patterns/patterns_opt_dbg_DE}}
\FR{\input{patterns/patterns_opt_dbg_FR}}
\PL{\input{patterns/patterns_opt_dbg_PL}}

\RU{\section{Некоторые базовые понятия}}
\EN{\section{Some basics}}
\DE{\section{Einige Grundlagen}}
\FR{\section{Quelques bases}}
\ES{\section{\ESph{}}}
\ITA{\section{Alcune basi teoriche}}
\PTBR{\section{\PTBRph{}}}
\THA{\section{\THAph{}}}
\PL{\section{\PLph{}}}

% sections:
\EN{\input{patterns/intro_CPU_ISA_EN}}
\ES{\input{patterns/intro_CPU_ISA_ES}}
\ITA{\input{patterns/intro_CPU_ISA_ITA}}
\PTBR{\input{patterns/intro_CPU_ISA_PTBR}}
\RU{\input{patterns/intro_CPU_ISA_RU}}
\DE{\input{patterns/intro_CPU_ISA_DE}}
\FR{\input{patterns/intro_CPU_ISA_FR}}
\PL{\input{patterns/intro_CPU_ISA_PL}}

\EN{\input{patterns/numeral_EN}}
\RU{\input{patterns/numeral_RU}}
\ITA{\input{patterns/numeral_ITA}}
\DE{\input{patterns/numeral_DE}}
\FR{\input{patterns/numeral_FR}}
\PL{\input{patterns/numeral_PL}}

% chapters
\input{patterns/00_empty/main}
\input{patterns/011_ret/main}
\input{patterns/01_helloworld/main}
\input{patterns/015_prolog_epilogue/main}
\input{patterns/02_stack/main}
\input{patterns/03_printf/main}
\input{patterns/04_scanf/main}
\input{patterns/05_passing_arguments/main}
\input{patterns/06_return_results/main}
\input{patterns/061_pointers/main}
\input{patterns/065_GOTO/main}
\input{patterns/07_jcc/main}
\input{patterns/08_switch/main}
\input{patterns/09_loops/main}
\input{patterns/10_strings/main}
\input{patterns/11_arith_optimizations/main}
\input{patterns/12_FPU/main}
\input{patterns/13_arrays/main}
\input{patterns/14_bitfields/main}
\EN{\input{patterns/145_LCG/main_EN}}
\RU{\input{patterns/145_LCG/main_RU}}
\input{patterns/15_structs/main}
\input{patterns/17_unions/main}
\input{patterns/18_pointers_to_functions/main}
\input{patterns/185_64bit_in_32_env/main}

\EN{\input{patterns/19_SIMD/main_EN}}
\RU{\input{patterns/19_SIMD/main_RU}}
\DE{\input{patterns/19_SIMD/main_DE}}

\EN{\input{patterns/20_x64/main_EN}}
\RU{\input{patterns/20_x64/main_RU}}

\EN{\input{patterns/205_floating_SIMD/main_EN}}
\RU{\input{patterns/205_floating_SIMD/main_RU}}
\DE{\input{patterns/205_floating_SIMD/main_DE}}

\EN{\input{patterns/ARM/main_EN}}
\RU{\input{patterns/ARM/main_RU}}
\DE{\input{patterns/ARM/main_DE}}

\input{patterns/MIPS/main}

\ifdefined\SPANISH
\chapter{Patrones de código}
\fi % SPANISH

\ifdefined\GERMAN
\chapter{Code-Muster}
\fi % GERMAN

\ifdefined\ENGLISH
\chapter{Code Patterns}
\fi % ENGLISH

\ifdefined\ITALIAN
\chapter{Forme di codice}
\fi % ITALIAN

\ifdefined\RUSSIAN
\chapter{Образцы кода}
\fi % RUSSIAN

\ifdefined\BRAZILIAN
\chapter{Padrões de códigos}
\fi % BRAZILIAN

\ifdefined\THAI
\chapter{รูปแบบของโค้ด}
\fi % THAI

\ifdefined\FRENCH
\chapter{Modèle de code}
\fi % FRENCH

\ifdefined\POLISH
\chapter{\PLph{}}
\fi % POLISH

% sections
\EN{\input{patterns/patterns_opt_dbg_EN}}
\ES{\input{patterns/patterns_opt_dbg_ES}}
\ITA{\input{patterns/patterns_opt_dbg_ITA}}
\PTBR{\input{patterns/patterns_opt_dbg_PTBR}}
\RU{\input{patterns/patterns_opt_dbg_RU}}
\THA{\input{patterns/patterns_opt_dbg_THA}}
\DE{\input{patterns/patterns_opt_dbg_DE}}
\FR{\input{patterns/patterns_opt_dbg_FR}}
\PL{\input{patterns/patterns_opt_dbg_PL}}

\RU{\section{Некоторые базовые понятия}}
\EN{\section{Some basics}}
\DE{\section{Einige Grundlagen}}
\FR{\section{Quelques bases}}
\ES{\section{\ESph{}}}
\ITA{\section{Alcune basi teoriche}}
\PTBR{\section{\PTBRph{}}}
\THA{\section{\THAph{}}}
\PL{\section{\PLph{}}}

% sections:
\EN{\input{patterns/intro_CPU_ISA_EN}}
\ES{\input{patterns/intro_CPU_ISA_ES}}
\ITA{\input{patterns/intro_CPU_ISA_ITA}}
\PTBR{\input{patterns/intro_CPU_ISA_PTBR}}
\RU{\input{patterns/intro_CPU_ISA_RU}}
\DE{\input{patterns/intro_CPU_ISA_DE}}
\FR{\input{patterns/intro_CPU_ISA_FR}}
\PL{\input{patterns/intro_CPU_ISA_PL}}

\EN{\input{patterns/numeral_EN}}
\RU{\input{patterns/numeral_RU}}
\ITA{\input{patterns/numeral_ITA}}
\DE{\input{patterns/numeral_DE}}
\FR{\input{patterns/numeral_FR}}
\PL{\input{patterns/numeral_PL}}

% chapters
\input{patterns/00_empty/main}
\input{patterns/011_ret/main}
\input{patterns/01_helloworld/main}
\input{patterns/015_prolog_epilogue/main}
\input{patterns/02_stack/main}
\input{patterns/03_printf/main}
\input{patterns/04_scanf/main}
\input{patterns/05_passing_arguments/main}
\input{patterns/06_return_results/main}
\input{patterns/061_pointers/main}
\input{patterns/065_GOTO/main}
\input{patterns/07_jcc/main}
\input{patterns/08_switch/main}
\input{patterns/09_loops/main}
\input{patterns/10_strings/main}
\input{patterns/11_arith_optimizations/main}
\input{patterns/12_FPU/main}
\input{patterns/13_arrays/main}
\input{patterns/14_bitfields/main}
\EN{\input{patterns/145_LCG/main_EN}}
\RU{\input{patterns/145_LCG/main_RU}}
\input{patterns/15_structs/main}
\input{patterns/17_unions/main}
\input{patterns/18_pointers_to_functions/main}
\input{patterns/185_64bit_in_32_env/main}

\EN{\input{patterns/19_SIMD/main_EN}}
\RU{\input{patterns/19_SIMD/main_RU}}
\DE{\input{patterns/19_SIMD/main_DE}}

\EN{\input{patterns/20_x64/main_EN}}
\RU{\input{patterns/20_x64/main_RU}}

\EN{\input{patterns/205_floating_SIMD/main_EN}}
\RU{\input{patterns/205_floating_SIMD/main_RU}}
\DE{\input{patterns/205_floating_SIMD/main_DE}}

\EN{\input{patterns/ARM/main_EN}}
\RU{\input{patterns/ARM/main_RU}}
\DE{\input{patterns/ARM/main_DE}}

\input{patterns/MIPS/main}

\ifdefined\SPANISH
\chapter{Patrones de código}
\fi % SPANISH

\ifdefined\GERMAN
\chapter{Code-Muster}
\fi % GERMAN

\ifdefined\ENGLISH
\chapter{Code Patterns}
\fi % ENGLISH

\ifdefined\ITALIAN
\chapter{Forme di codice}
\fi % ITALIAN

\ifdefined\RUSSIAN
\chapter{Образцы кода}
\fi % RUSSIAN

\ifdefined\BRAZILIAN
\chapter{Padrões de códigos}
\fi % BRAZILIAN

\ifdefined\THAI
\chapter{รูปแบบของโค้ด}
\fi % THAI

\ifdefined\FRENCH
\chapter{Modèle de code}
\fi % FRENCH

\ifdefined\POLISH
\chapter{\PLph{}}
\fi % POLISH

% sections
\EN{\input{patterns/patterns_opt_dbg_EN}}
\ES{\input{patterns/patterns_opt_dbg_ES}}
\ITA{\input{patterns/patterns_opt_dbg_ITA}}
\PTBR{\input{patterns/patterns_opt_dbg_PTBR}}
\RU{\input{patterns/patterns_opt_dbg_RU}}
\THA{\input{patterns/patterns_opt_dbg_THA}}
\DE{\input{patterns/patterns_opt_dbg_DE}}
\FR{\input{patterns/patterns_opt_dbg_FR}}
\PL{\input{patterns/patterns_opt_dbg_PL}}

\RU{\section{Некоторые базовые понятия}}
\EN{\section{Some basics}}
\DE{\section{Einige Grundlagen}}
\FR{\section{Quelques bases}}
\ES{\section{\ESph{}}}
\ITA{\section{Alcune basi teoriche}}
\PTBR{\section{\PTBRph{}}}
\THA{\section{\THAph{}}}
\PL{\section{\PLph{}}}

% sections:
\EN{\input{patterns/intro_CPU_ISA_EN}}
\ES{\input{patterns/intro_CPU_ISA_ES}}
\ITA{\input{patterns/intro_CPU_ISA_ITA}}
\PTBR{\input{patterns/intro_CPU_ISA_PTBR}}
\RU{\input{patterns/intro_CPU_ISA_RU}}
\DE{\input{patterns/intro_CPU_ISA_DE}}
\FR{\input{patterns/intro_CPU_ISA_FR}}
\PL{\input{patterns/intro_CPU_ISA_PL}}

\EN{\input{patterns/numeral_EN}}
\RU{\input{patterns/numeral_RU}}
\ITA{\input{patterns/numeral_ITA}}
\DE{\input{patterns/numeral_DE}}
\FR{\input{patterns/numeral_FR}}
\PL{\input{patterns/numeral_PL}}

% chapters
\input{patterns/00_empty/main}
\input{patterns/011_ret/main}
\input{patterns/01_helloworld/main}
\input{patterns/015_prolog_epilogue/main}
\input{patterns/02_stack/main}
\input{patterns/03_printf/main}
\input{patterns/04_scanf/main}
\input{patterns/05_passing_arguments/main}
\input{patterns/06_return_results/main}
\input{patterns/061_pointers/main}
\input{patterns/065_GOTO/main}
\input{patterns/07_jcc/main}
\input{patterns/08_switch/main}
\input{patterns/09_loops/main}
\input{patterns/10_strings/main}
\input{patterns/11_arith_optimizations/main}
\input{patterns/12_FPU/main}
\input{patterns/13_arrays/main}
\input{patterns/14_bitfields/main}
\EN{\input{patterns/145_LCG/main_EN}}
\RU{\input{patterns/145_LCG/main_RU}}
\input{patterns/15_structs/main}
\input{patterns/17_unions/main}
\input{patterns/18_pointers_to_functions/main}
\input{patterns/185_64bit_in_32_env/main}

\EN{\input{patterns/19_SIMD/main_EN}}
\RU{\input{patterns/19_SIMD/main_RU}}
\DE{\input{patterns/19_SIMD/main_DE}}

\EN{\input{patterns/20_x64/main_EN}}
\RU{\input{patterns/20_x64/main_RU}}

\EN{\input{patterns/205_floating_SIMD/main_EN}}
\RU{\input{patterns/205_floating_SIMD/main_RU}}
\DE{\input{patterns/205_floating_SIMD/main_DE}}

\EN{\input{patterns/ARM/main_EN}}
\RU{\input{patterns/ARM/main_RU}}
\DE{\input{patterns/ARM/main_DE}}

\input{patterns/MIPS/main}

\ifdefined\SPANISH
\chapter{Patrones de código}
\fi % SPANISH

\ifdefined\GERMAN
\chapter{Code-Muster}
\fi % GERMAN

\ifdefined\ENGLISH
\chapter{Code Patterns}
\fi % ENGLISH

\ifdefined\ITALIAN
\chapter{Forme di codice}
\fi % ITALIAN

\ifdefined\RUSSIAN
\chapter{Образцы кода}
\fi % RUSSIAN

\ifdefined\BRAZILIAN
\chapter{Padrões de códigos}
\fi % BRAZILIAN

\ifdefined\THAI
\chapter{รูปแบบของโค้ด}
\fi % THAI

\ifdefined\FRENCH
\chapter{Modèle de code}
\fi % FRENCH

\ifdefined\POLISH
\chapter{\PLph{}}
\fi % POLISH

% sections
\EN{\input{patterns/patterns_opt_dbg_EN}}
\ES{\input{patterns/patterns_opt_dbg_ES}}
\ITA{\input{patterns/patterns_opt_dbg_ITA}}
\PTBR{\input{patterns/patterns_opt_dbg_PTBR}}
\RU{\input{patterns/patterns_opt_dbg_RU}}
\THA{\input{patterns/patterns_opt_dbg_THA}}
\DE{\input{patterns/patterns_opt_dbg_DE}}
\FR{\input{patterns/patterns_opt_dbg_FR}}
\PL{\input{patterns/patterns_opt_dbg_PL}}

\RU{\section{Некоторые базовые понятия}}
\EN{\section{Some basics}}
\DE{\section{Einige Grundlagen}}
\FR{\section{Quelques bases}}
\ES{\section{\ESph{}}}
\ITA{\section{Alcune basi teoriche}}
\PTBR{\section{\PTBRph{}}}
\THA{\section{\THAph{}}}
\PL{\section{\PLph{}}}

% sections:
\EN{\input{patterns/intro_CPU_ISA_EN}}
\ES{\input{patterns/intro_CPU_ISA_ES}}
\ITA{\input{patterns/intro_CPU_ISA_ITA}}
\PTBR{\input{patterns/intro_CPU_ISA_PTBR}}
\RU{\input{patterns/intro_CPU_ISA_RU}}
\DE{\input{patterns/intro_CPU_ISA_DE}}
\FR{\input{patterns/intro_CPU_ISA_FR}}
\PL{\input{patterns/intro_CPU_ISA_PL}}

\EN{\input{patterns/numeral_EN}}
\RU{\input{patterns/numeral_RU}}
\ITA{\input{patterns/numeral_ITA}}
\DE{\input{patterns/numeral_DE}}
\FR{\input{patterns/numeral_FR}}
\PL{\input{patterns/numeral_PL}}

% chapters
\input{patterns/00_empty/main}
\input{patterns/011_ret/main}
\input{patterns/01_helloworld/main}
\input{patterns/015_prolog_epilogue/main}
\input{patterns/02_stack/main}
\input{patterns/03_printf/main}
\input{patterns/04_scanf/main}
\input{patterns/05_passing_arguments/main}
\input{patterns/06_return_results/main}
\input{patterns/061_pointers/main}
\input{patterns/065_GOTO/main}
\input{patterns/07_jcc/main}
\input{patterns/08_switch/main}
\input{patterns/09_loops/main}
\input{patterns/10_strings/main}
\input{patterns/11_arith_optimizations/main}
\input{patterns/12_FPU/main}
\input{patterns/13_arrays/main}
\input{patterns/14_bitfields/main}
\EN{\input{patterns/145_LCG/main_EN}}
\RU{\input{patterns/145_LCG/main_RU}}
\input{patterns/15_structs/main}
\input{patterns/17_unions/main}
\input{patterns/18_pointers_to_functions/main}
\input{patterns/185_64bit_in_32_env/main}

\EN{\input{patterns/19_SIMD/main_EN}}
\RU{\input{patterns/19_SIMD/main_RU}}
\DE{\input{patterns/19_SIMD/main_DE}}

\EN{\input{patterns/20_x64/main_EN}}
\RU{\input{patterns/20_x64/main_RU}}

\EN{\input{patterns/205_floating_SIMD/main_EN}}
\RU{\input{patterns/205_floating_SIMD/main_RU}}
\DE{\input{patterns/205_floating_SIMD/main_DE}}

\EN{\input{patterns/ARM/main_EN}}
\RU{\input{patterns/ARM/main_RU}}
\DE{\input{patterns/ARM/main_DE}}

\input{patterns/MIPS/main}

\ifdefined\SPANISH
\chapter{Patrones de código}
\fi % SPANISH

\ifdefined\GERMAN
\chapter{Code-Muster}
\fi % GERMAN

\ifdefined\ENGLISH
\chapter{Code Patterns}
\fi % ENGLISH

\ifdefined\ITALIAN
\chapter{Forme di codice}
\fi % ITALIAN

\ifdefined\RUSSIAN
\chapter{Образцы кода}
\fi % RUSSIAN

\ifdefined\BRAZILIAN
\chapter{Padrões de códigos}
\fi % BRAZILIAN

\ifdefined\THAI
\chapter{รูปแบบของโค้ด}
\fi % THAI

\ifdefined\FRENCH
\chapter{Modèle de code}
\fi % FRENCH

\ifdefined\POLISH
\chapter{\PLph{}}
\fi % POLISH

% sections
\EN{\input{patterns/patterns_opt_dbg_EN}}
\ES{\input{patterns/patterns_opt_dbg_ES}}
\ITA{\input{patterns/patterns_opt_dbg_ITA}}
\PTBR{\input{patterns/patterns_opt_dbg_PTBR}}
\RU{\input{patterns/patterns_opt_dbg_RU}}
\THA{\input{patterns/patterns_opt_dbg_THA}}
\DE{\input{patterns/patterns_opt_dbg_DE}}
\FR{\input{patterns/patterns_opt_dbg_FR}}
\PL{\input{patterns/patterns_opt_dbg_PL}}

\RU{\section{Некоторые базовые понятия}}
\EN{\section{Some basics}}
\DE{\section{Einige Grundlagen}}
\FR{\section{Quelques bases}}
\ES{\section{\ESph{}}}
\ITA{\section{Alcune basi teoriche}}
\PTBR{\section{\PTBRph{}}}
\THA{\section{\THAph{}}}
\PL{\section{\PLph{}}}

% sections:
\EN{\input{patterns/intro_CPU_ISA_EN}}
\ES{\input{patterns/intro_CPU_ISA_ES}}
\ITA{\input{patterns/intro_CPU_ISA_ITA}}
\PTBR{\input{patterns/intro_CPU_ISA_PTBR}}
\RU{\input{patterns/intro_CPU_ISA_RU}}
\DE{\input{patterns/intro_CPU_ISA_DE}}
\FR{\input{patterns/intro_CPU_ISA_FR}}
\PL{\input{patterns/intro_CPU_ISA_PL}}

\EN{\input{patterns/numeral_EN}}
\RU{\input{patterns/numeral_RU}}
\ITA{\input{patterns/numeral_ITA}}
\DE{\input{patterns/numeral_DE}}
\FR{\input{patterns/numeral_FR}}
\PL{\input{patterns/numeral_PL}}

% chapters
\input{patterns/00_empty/main}
\input{patterns/011_ret/main}
\input{patterns/01_helloworld/main}
\input{patterns/015_prolog_epilogue/main}
\input{patterns/02_stack/main}
\input{patterns/03_printf/main}
\input{patterns/04_scanf/main}
\input{patterns/05_passing_arguments/main}
\input{patterns/06_return_results/main}
\input{patterns/061_pointers/main}
\input{patterns/065_GOTO/main}
\input{patterns/07_jcc/main}
\input{patterns/08_switch/main}
\input{patterns/09_loops/main}
\input{patterns/10_strings/main}
\input{patterns/11_arith_optimizations/main}
\input{patterns/12_FPU/main}
\input{patterns/13_arrays/main}
\input{patterns/14_bitfields/main}
\EN{\input{patterns/145_LCG/main_EN}}
\RU{\input{patterns/145_LCG/main_RU}}
\input{patterns/15_structs/main}
\input{patterns/17_unions/main}
\input{patterns/18_pointers_to_functions/main}
\input{patterns/185_64bit_in_32_env/main}

\EN{\input{patterns/19_SIMD/main_EN}}
\RU{\input{patterns/19_SIMD/main_RU}}
\DE{\input{patterns/19_SIMD/main_DE}}

\EN{\input{patterns/20_x64/main_EN}}
\RU{\input{patterns/20_x64/main_RU}}

\EN{\input{patterns/205_floating_SIMD/main_EN}}
\RU{\input{patterns/205_floating_SIMD/main_RU}}
\DE{\input{patterns/205_floating_SIMD/main_DE}}

\EN{\input{patterns/ARM/main_EN}}
\RU{\input{patterns/ARM/main_RU}}
\DE{\input{patterns/ARM/main_DE}}

\input{patterns/MIPS/main}

\ifdefined\SPANISH
\chapter{Patrones de código}
\fi % SPANISH

\ifdefined\GERMAN
\chapter{Code-Muster}
\fi % GERMAN

\ifdefined\ENGLISH
\chapter{Code Patterns}
\fi % ENGLISH

\ifdefined\ITALIAN
\chapter{Forme di codice}
\fi % ITALIAN

\ifdefined\RUSSIAN
\chapter{Образцы кода}
\fi % RUSSIAN

\ifdefined\BRAZILIAN
\chapter{Padrões de códigos}
\fi % BRAZILIAN

\ifdefined\THAI
\chapter{รูปแบบของโค้ด}
\fi % THAI

\ifdefined\FRENCH
\chapter{Modèle de code}
\fi % FRENCH

\ifdefined\POLISH
\chapter{\PLph{}}
\fi % POLISH

% sections
\EN{\input{patterns/patterns_opt_dbg_EN}}
\ES{\input{patterns/patterns_opt_dbg_ES}}
\ITA{\input{patterns/patterns_opt_dbg_ITA}}
\PTBR{\input{patterns/patterns_opt_dbg_PTBR}}
\RU{\input{patterns/patterns_opt_dbg_RU}}
\THA{\input{patterns/patterns_opt_dbg_THA}}
\DE{\input{patterns/patterns_opt_dbg_DE}}
\FR{\input{patterns/patterns_opt_dbg_FR}}
\PL{\input{patterns/patterns_opt_dbg_PL}}

\RU{\section{Некоторые базовые понятия}}
\EN{\section{Some basics}}
\DE{\section{Einige Grundlagen}}
\FR{\section{Quelques bases}}
\ES{\section{\ESph{}}}
\ITA{\section{Alcune basi teoriche}}
\PTBR{\section{\PTBRph{}}}
\THA{\section{\THAph{}}}
\PL{\section{\PLph{}}}

% sections:
\EN{\input{patterns/intro_CPU_ISA_EN}}
\ES{\input{patterns/intro_CPU_ISA_ES}}
\ITA{\input{patterns/intro_CPU_ISA_ITA}}
\PTBR{\input{patterns/intro_CPU_ISA_PTBR}}
\RU{\input{patterns/intro_CPU_ISA_RU}}
\DE{\input{patterns/intro_CPU_ISA_DE}}
\FR{\input{patterns/intro_CPU_ISA_FR}}
\PL{\input{patterns/intro_CPU_ISA_PL}}

\EN{\input{patterns/numeral_EN}}
\RU{\input{patterns/numeral_RU}}
\ITA{\input{patterns/numeral_ITA}}
\DE{\input{patterns/numeral_DE}}
\FR{\input{patterns/numeral_FR}}
\PL{\input{patterns/numeral_PL}}

% chapters
\input{patterns/00_empty/main}
\input{patterns/011_ret/main}
\input{patterns/01_helloworld/main}
\input{patterns/015_prolog_epilogue/main}
\input{patterns/02_stack/main}
\input{patterns/03_printf/main}
\input{patterns/04_scanf/main}
\input{patterns/05_passing_arguments/main}
\input{patterns/06_return_results/main}
\input{patterns/061_pointers/main}
\input{patterns/065_GOTO/main}
\input{patterns/07_jcc/main}
\input{patterns/08_switch/main}
\input{patterns/09_loops/main}
\input{patterns/10_strings/main}
\input{patterns/11_arith_optimizations/main}
\input{patterns/12_FPU/main}
\input{patterns/13_arrays/main}
\input{patterns/14_bitfields/main}
\EN{\input{patterns/145_LCG/main_EN}}
\RU{\input{patterns/145_LCG/main_RU}}
\input{patterns/15_structs/main}
\input{patterns/17_unions/main}
\input{patterns/18_pointers_to_functions/main}
\input{patterns/185_64bit_in_32_env/main}

\EN{\input{patterns/19_SIMD/main_EN}}
\RU{\input{patterns/19_SIMD/main_RU}}
\DE{\input{patterns/19_SIMD/main_DE}}

\EN{\input{patterns/20_x64/main_EN}}
\RU{\input{patterns/20_x64/main_RU}}

\EN{\input{patterns/205_floating_SIMD/main_EN}}
\RU{\input{patterns/205_floating_SIMD/main_RU}}
\DE{\input{patterns/205_floating_SIMD/main_DE}}

\EN{\input{patterns/ARM/main_EN}}
\RU{\input{patterns/ARM/main_RU}}
\DE{\input{patterns/ARM/main_DE}}

\input{patterns/MIPS/main}

\ifdefined\SPANISH
\chapter{Patrones de código}
\fi % SPANISH

\ifdefined\GERMAN
\chapter{Code-Muster}
\fi % GERMAN

\ifdefined\ENGLISH
\chapter{Code Patterns}
\fi % ENGLISH

\ifdefined\ITALIAN
\chapter{Forme di codice}
\fi % ITALIAN

\ifdefined\RUSSIAN
\chapter{Образцы кода}
\fi % RUSSIAN

\ifdefined\BRAZILIAN
\chapter{Padrões de códigos}
\fi % BRAZILIAN

\ifdefined\THAI
\chapter{รูปแบบของโค้ด}
\fi % THAI

\ifdefined\FRENCH
\chapter{Modèle de code}
\fi % FRENCH

\ifdefined\POLISH
\chapter{\PLph{}}
\fi % POLISH

% sections
\EN{\input{patterns/patterns_opt_dbg_EN}}
\ES{\input{patterns/patterns_opt_dbg_ES}}
\ITA{\input{patterns/patterns_opt_dbg_ITA}}
\PTBR{\input{patterns/patterns_opt_dbg_PTBR}}
\RU{\input{patterns/patterns_opt_dbg_RU}}
\THA{\input{patterns/patterns_opt_dbg_THA}}
\DE{\input{patterns/patterns_opt_dbg_DE}}
\FR{\input{patterns/patterns_opt_dbg_FR}}
\PL{\input{patterns/patterns_opt_dbg_PL}}

\RU{\section{Некоторые базовые понятия}}
\EN{\section{Some basics}}
\DE{\section{Einige Grundlagen}}
\FR{\section{Quelques bases}}
\ES{\section{\ESph{}}}
\ITA{\section{Alcune basi teoriche}}
\PTBR{\section{\PTBRph{}}}
\THA{\section{\THAph{}}}
\PL{\section{\PLph{}}}

% sections:
\EN{\input{patterns/intro_CPU_ISA_EN}}
\ES{\input{patterns/intro_CPU_ISA_ES}}
\ITA{\input{patterns/intro_CPU_ISA_ITA}}
\PTBR{\input{patterns/intro_CPU_ISA_PTBR}}
\RU{\input{patterns/intro_CPU_ISA_RU}}
\DE{\input{patterns/intro_CPU_ISA_DE}}
\FR{\input{patterns/intro_CPU_ISA_FR}}
\PL{\input{patterns/intro_CPU_ISA_PL}}

\EN{\input{patterns/numeral_EN}}
\RU{\input{patterns/numeral_RU}}
\ITA{\input{patterns/numeral_ITA}}
\DE{\input{patterns/numeral_DE}}
\FR{\input{patterns/numeral_FR}}
\PL{\input{patterns/numeral_PL}}

% chapters
\input{patterns/00_empty/main}
\input{patterns/011_ret/main}
\input{patterns/01_helloworld/main}
\input{patterns/015_prolog_epilogue/main}
\input{patterns/02_stack/main}
\input{patterns/03_printf/main}
\input{patterns/04_scanf/main}
\input{patterns/05_passing_arguments/main}
\input{patterns/06_return_results/main}
\input{patterns/061_pointers/main}
\input{patterns/065_GOTO/main}
\input{patterns/07_jcc/main}
\input{patterns/08_switch/main}
\input{patterns/09_loops/main}
\input{patterns/10_strings/main}
\input{patterns/11_arith_optimizations/main}
\input{patterns/12_FPU/main}
\input{patterns/13_arrays/main}
\input{patterns/14_bitfields/main}
\EN{\input{patterns/145_LCG/main_EN}}
\RU{\input{patterns/145_LCG/main_RU}}
\input{patterns/15_structs/main}
\input{patterns/17_unions/main}
\input{patterns/18_pointers_to_functions/main}
\input{patterns/185_64bit_in_32_env/main}

\EN{\input{patterns/19_SIMD/main_EN}}
\RU{\input{patterns/19_SIMD/main_RU}}
\DE{\input{patterns/19_SIMD/main_DE}}

\EN{\input{patterns/20_x64/main_EN}}
\RU{\input{patterns/20_x64/main_RU}}

\EN{\input{patterns/205_floating_SIMD/main_EN}}
\RU{\input{patterns/205_floating_SIMD/main_RU}}
\DE{\input{patterns/205_floating_SIMD/main_DE}}

\EN{\input{patterns/ARM/main_EN}}
\RU{\input{patterns/ARM/main_RU}}
\DE{\input{patterns/ARM/main_DE}}

\input{patterns/MIPS/main}

\ifdefined\SPANISH
\chapter{Patrones de código}
\fi % SPANISH

\ifdefined\GERMAN
\chapter{Code-Muster}
\fi % GERMAN

\ifdefined\ENGLISH
\chapter{Code Patterns}
\fi % ENGLISH

\ifdefined\ITALIAN
\chapter{Forme di codice}
\fi % ITALIAN

\ifdefined\RUSSIAN
\chapter{Образцы кода}
\fi % RUSSIAN

\ifdefined\BRAZILIAN
\chapter{Padrões de códigos}
\fi % BRAZILIAN

\ifdefined\THAI
\chapter{รูปแบบของโค้ด}
\fi % THAI

\ifdefined\FRENCH
\chapter{Modèle de code}
\fi % FRENCH

\ifdefined\POLISH
\chapter{\PLph{}}
\fi % POLISH

% sections
\EN{\input{patterns/patterns_opt_dbg_EN}}
\ES{\input{patterns/patterns_opt_dbg_ES}}
\ITA{\input{patterns/patterns_opt_dbg_ITA}}
\PTBR{\input{patterns/patterns_opt_dbg_PTBR}}
\RU{\input{patterns/patterns_opt_dbg_RU}}
\THA{\input{patterns/patterns_opt_dbg_THA}}
\DE{\input{patterns/patterns_opt_dbg_DE}}
\FR{\input{patterns/patterns_opt_dbg_FR}}
\PL{\input{patterns/patterns_opt_dbg_PL}}

\RU{\section{Некоторые базовые понятия}}
\EN{\section{Some basics}}
\DE{\section{Einige Grundlagen}}
\FR{\section{Quelques bases}}
\ES{\section{\ESph{}}}
\ITA{\section{Alcune basi teoriche}}
\PTBR{\section{\PTBRph{}}}
\THA{\section{\THAph{}}}
\PL{\section{\PLph{}}}

% sections:
\EN{\input{patterns/intro_CPU_ISA_EN}}
\ES{\input{patterns/intro_CPU_ISA_ES}}
\ITA{\input{patterns/intro_CPU_ISA_ITA}}
\PTBR{\input{patterns/intro_CPU_ISA_PTBR}}
\RU{\input{patterns/intro_CPU_ISA_RU}}
\DE{\input{patterns/intro_CPU_ISA_DE}}
\FR{\input{patterns/intro_CPU_ISA_FR}}
\PL{\input{patterns/intro_CPU_ISA_PL}}

\EN{\input{patterns/numeral_EN}}
\RU{\input{patterns/numeral_RU}}
\ITA{\input{patterns/numeral_ITA}}
\DE{\input{patterns/numeral_DE}}
\FR{\input{patterns/numeral_FR}}
\PL{\input{patterns/numeral_PL}}

% chapters
\input{patterns/00_empty/main}
\input{patterns/011_ret/main}
\input{patterns/01_helloworld/main}
\input{patterns/015_prolog_epilogue/main}
\input{patterns/02_stack/main}
\input{patterns/03_printf/main}
\input{patterns/04_scanf/main}
\input{patterns/05_passing_arguments/main}
\input{patterns/06_return_results/main}
\input{patterns/061_pointers/main}
\input{patterns/065_GOTO/main}
\input{patterns/07_jcc/main}
\input{patterns/08_switch/main}
\input{patterns/09_loops/main}
\input{patterns/10_strings/main}
\input{patterns/11_arith_optimizations/main}
\input{patterns/12_FPU/main}
\input{patterns/13_arrays/main}
\input{patterns/14_bitfields/main}
\EN{\input{patterns/145_LCG/main_EN}}
\RU{\input{patterns/145_LCG/main_RU}}
\input{patterns/15_structs/main}
\input{patterns/17_unions/main}
\input{patterns/18_pointers_to_functions/main}
\input{patterns/185_64bit_in_32_env/main}

\EN{\input{patterns/19_SIMD/main_EN}}
\RU{\input{patterns/19_SIMD/main_RU}}
\DE{\input{patterns/19_SIMD/main_DE}}

\EN{\input{patterns/20_x64/main_EN}}
\RU{\input{patterns/20_x64/main_RU}}

\EN{\input{patterns/205_floating_SIMD/main_EN}}
\RU{\input{patterns/205_floating_SIMD/main_RU}}
\DE{\input{patterns/205_floating_SIMD/main_DE}}

\EN{\input{patterns/ARM/main_EN}}
\RU{\input{patterns/ARM/main_RU}}
\DE{\input{patterns/ARM/main_DE}}

\input{patterns/MIPS/main}

\ifdefined\SPANISH
\chapter{Patrones de código}
\fi % SPANISH

\ifdefined\GERMAN
\chapter{Code-Muster}
\fi % GERMAN

\ifdefined\ENGLISH
\chapter{Code Patterns}
\fi % ENGLISH

\ifdefined\ITALIAN
\chapter{Forme di codice}
\fi % ITALIAN

\ifdefined\RUSSIAN
\chapter{Образцы кода}
\fi % RUSSIAN

\ifdefined\BRAZILIAN
\chapter{Padrões de códigos}
\fi % BRAZILIAN

\ifdefined\THAI
\chapter{รูปแบบของโค้ด}
\fi % THAI

\ifdefined\FRENCH
\chapter{Modèle de code}
\fi % FRENCH

\ifdefined\POLISH
\chapter{\PLph{}}
\fi % POLISH

% sections
\EN{\input{patterns/patterns_opt_dbg_EN}}
\ES{\input{patterns/patterns_opt_dbg_ES}}
\ITA{\input{patterns/patterns_opt_dbg_ITA}}
\PTBR{\input{patterns/patterns_opt_dbg_PTBR}}
\RU{\input{patterns/patterns_opt_dbg_RU}}
\THA{\input{patterns/patterns_opt_dbg_THA}}
\DE{\input{patterns/patterns_opt_dbg_DE}}
\FR{\input{patterns/patterns_opt_dbg_FR}}
\PL{\input{patterns/patterns_opt_dbg_PL}}

\RU{\section{Некоторые базовые понятия}}
\EN{\section{Some basics}}
\DE{\section{Einige Grundlagen}}
\FR{\section{Quelques bases}}
\ES{\section{\ESph{}}}
\ITA{\section{Alcune basi teoriche}}
\PTBR{\section{\PTBRph{}}}
\THA{\section{\THAph{}}}
\PL{\section{\PLph{}}}

% sections:
\EN{\input{patterns/intro_CPU_ISA_EN}}
\ES{\input{patterns/intro_CPU_ISA_ES}}
\ITA{\input{patterns/intro_CPU_ISA_ITA}}
\PTBR{\input{patterns/intro_CPU_ISA_PTBR}}
\RU{\input{patterns/intro_CPU_ISA_RU}}
\DE{\input{patterns/intro_CPU_ISA_DE}}
\FR{\input{patterns/intro_CPU_ISA_FR}}
\PL{\input{patterns/intro_CPU_ISA_PL}}

\EN{\input{patterns/numeral_EN}}
\RU{\input{patterns/numeral_RU}}
\ITA{\input{patterns/numeral_ITA}}
\DE{\input{patterns/numeral_DE}}
\FR{\input{patterns/numeral_FR}}
\PL{\input{patterns/numeral_PL}}

% chapters
\input{patterns/00_empty/main}
\input{patterns/011_ret/main}
\input{patterns/01_helloworld/main}
\input{patterns/015_prolog_epilogue/main}
\input{patterns/02_stack/main}
\input{patterns/03_printf/main}
\input{patterns/04_scanf/main}
\input{patterns/05_passing_arguments/main}
\input{patterns/06_return_results/main}
\input{patterns/061_pointers/main}
\input{patterns/065_GOTO/main}
\input{patterns/07_jcc/main}
\input{patterns/08_switch/main}
\input{patterns/09_loops/main}
\input{patterns/10_strings/main}
\input{patterns/11_arith_optimizations/main}
\input{patterns/12_FPU/main}
\input{patterns/13_arrays/main}
\input{patterns/14_bitfields/main}
\EN{\input{patterns/145_LCG/main_EN}}
\RU{\input{patterns/145_LCG/main_RU}}
\input{patterns/15_structs/main}
\input{patterns/17_unions/main}
\input{patterns/18_pointers_to_functions/main}
\input{patterns/185_64bit_in_32_env/main}

\EN{\input{patterns/19_SIMD/main_EN}}
\RU{\input{patterns/19_SIMD/main_RU}}
\DE{\input{patterns/19_SIMD/main_DE}}

\EN{\input{patterns/20_x64/main_EN}}
\RU{\input{patterns/20_x64/main_RU}}

\EN{\input{patterns/205_floating_SIMD/main_EN}}
\RU{\input{patterns/205_floating_SIMD/main_RU}}
\DE{\input{patterns/205_floating_SIMD/main_DE}}

\EN{\input{patterns/ARM/main_EN}}
\RU{\input{patterns/ARM/main_RU}}
\DE{\input{patterns/ARM/main_DE}}

\input{patterns/MIPS/main}

\ifdefined\SPANISH
\chapter{Patrones de código}
\fi % SPANISH

\ifdefined\GERMAN
\chapter{Code-Muster}
\fi % GERMAN

\ifdefined\ENGLISH
\chapter{Code Patterns}
\fi % ENGLISH

\ifdefined\ITALIAN
\chapter{Forme di codice}
\fi % ITALIAN

\ifdefined\RUSSIAN
\chapter{Образцы кода}
\fi % RUSSIAN

\ifdefined\BRAZILIAN
\chapter{Padrões de códigos}
\fi % BRAZILIAN

\ifdefined\THAI
\chapter{รูปแบบของโค้ด}
\fi % THAI

\ifdefined\FRENCH
\chapter{Modèle de code}
\fi % FRENCH

\ifdefined\POLISH
\chapter{\PLph{}}
\fi % POLISH

% sections
\EN{\input{patterns/patterns_opt_dbg_EN}}
\ES{\input{patterns/patterns_opt_dbg_ES}}
\ITA{\input{patterns/patterns_opt_dbg_ITA}}
\PTBR{\input{patterns/patterns_opt_dbg_PTBR}}
\RU{\input{patterns/patterns_opt_dbg_RU}}
\THA{\input{patterns/patterns_opt_dbg_THA}}
\DE{\input{patterns/patterns_opt_dbg_DE}}
\FR{\input{patterns/patterns_opt_dbg_FR}}
\PL{\input{patterns/patterns_opt_dbg_PL}}

\RU{\section{Некоторые базовые понятия}}
\EN{\section{Some basics}}
\DE{\section{Einige Grundlagen}}
\FR{\section{Quelques bases}}
\ES{\section{\ESph{}}}
\ITA{\section{Alcune basi teoriche}}
\PTBR{\section{\PTBRph{}}}
\THA{\section{\THAph{}}}
\PL{\section{\PLph{}}}

% sections:
\EN{\input{patterns/intro_CPU_ISA_EN}}
\ES{\input{patterns/intro_CPU_ISA_ES}}
\ITA{\input{patterns/intro_CPU_ISA_ITA}}
\PTBR{\input{patterns/intro_CPU_ISA_PTBR}}
\RU{\input{patterns/intro_CPU_ISA_RU}}
\DE{\input{patterns/intro_CPU_ISA_DE}}
\FR{\input{patterns/intro_CPU_ISA_FR}}
\PL{\input{patterns/intro_CPU_ISA_PL}}

\EN{\input{patterns/numeral_EN}}
\RU{\input{patterns/numeral_RU}}
\ITA{\input{patterns/numeral_ITA}}
\DE{\input{patterns/numeral_DE}}
\FR{\input{patterns/numeral_FR}}
\PL{\input{patterns/numeral_PL}}

% chapters
\input{patterns/00_empty/main}
\input{patterns/011_ret/main}
\input{patterns/01_helloworld/main}
\input{patterns/015_prolog_epilogue/main}
\input{patterns/02_stack/main}
\input{patterns/03_printf/main}
\input{patterns/04_scanf/main}
\input{patterns/05_passing_arguments/main}
\input{patterns/06_return_results/main}
\input{patterns/061_pointers/main}
\input{patterns/065_GOTO/main}
\input{patterns/07_jcc/main}
\input{patterns/08_switch/main}
\input{patterns/09_loops/main}
\input{patterns/10_strings/main}
\input{patterns/11_arith_optimizations/main}
\input{patterns/12_FPU/main}
\input{patterns/13_arrays/main}
\input{patterns/14_bitfields/main}
\EN{\input{patterns/145_LCG/main_EN}}
\RU{\input{patterns/145_LCG/main_RU}}
\input{patterns/15_structs/main}
\input{patterns/17_unions/main}
\input{patterns/18_pointers_to_functions/main}
\input{patterns/185_64bit_in_32_env/main}

\EN{\input{patterns/19_SIMD/main_EN}}
\RU{\input{patterns/19_SIMD/main_RU}}
\DE{\input{patterns/19_SIMD/main_DE}}

\EN{\input{patterns/20_x64/main_EN}}
\RU{\input{patterns/20_x64/main_RU}}

\EN{\input{patterns/205_floating_SIMD/main_EN}}
\RU{\input{patterns/205_floating_SIMD/main_RU}}
\DE{\input{patterns/205_floating_SIMD/main_DE}}

\EN{\input{patterns/ARM/main_EN}}
\RU{\input{patterns/ARM/main_RU}}
\DE{\input{patterns/ARM/main_DE}}

\input{patterns/MIPS/main}

\ifdefined\SPANISH
\chapter{Patrones de código}
\fi % SPANISH

\ifdefined\GERMAN
\chapter{Code-Muster}
\fi % GERMAN

\ifdefined\ENGLISH
\chapter{Code Patterns}
\fi % ENGLISH

\ifdefined\ITALIAN
\chapter{Forme di codice}
\fi % ITALIAN

\ifdefined\RUSSIAN
\chapter{Образцы кода}
\fi % RUSSIAN

\ifdefined\BRAZILIAN
\chapter{Padrões de códigos}
\fi % BRAZILIAN

\ifdefined\THAI
\chapter{รูปแบบของโค้ด}
\fi % THAI

\ifdefined\FRENCH
\chapter{Modèle de code}
\fi % FRENCH

\ifdefined\POLISH
\chapter{\PLph{}}
\fi % POLISH

% sections
\EN{\input{patterns/patterns_opt_dbg_EN}}
\ES{\input{patterns/patterns_opt_dbg_ES}}
\ITA{\input{patterns/patterns_opt_dbg_ITA}}
\PTBR{\input{patterns/patterns_opt_dbg_PTBR}}
\RU{\input{patterns/patterns_opt_dbg_RU}}
\THA{\input{patterns/patterns_opt_dbg_THA}}
\DE{\input{patterns/patterns_opt_dbg_DE}}
\FR{\input{patterns/patterns_opt_dbg_FR}}
\PL{\input{patterns/patterns_opt_dbg_PL}}

\RU{\section{Некоторые базовые понятия}}
\EN{\section{Some basics}}
\DE{\section{Einige Grundlagen}}
\FR{\section{Quelques bases}}
\ES{\section{\ESph{}}}
\ITA{\section{Alcune basi teoriche}}
\PTBR{\section{\PTBRph{}}}
\THA{\section{\THAph{}}}
\PL{\section{\PLph{}}}

% sections:
\EN{\input{patterns/intro_CPU_ISA_EN}}
\ES{\input{patterns/intro_CPU_ISA_ES}}
\ITA{\input{patterns/intro_CPU_ISA_ITA}}
\PTBR{\input{patterns/intro_CPU_ISA_PTBR}}
\RU{\input{patterns/intro_CPU_ISA_RU}}
\DE{\input{patterns/intro_CPU_ISA_DE}}
\FR{\input{patterns/intro_CPU_ISA_FR}}
\PL{\input{patterns/intro_CPU_ISA_PL}}

\EN{\input{patterns/numeral_EN}}
\RU{\input{patterns/numeral_RU}}
\ITA{\input{patterns/numeral_ITA}}
\DE{\input{patterns/numeral_DE}}
\FR{\input{patterns/numeral_FR}}
\PL{\input{patterns/numeral_PL}}

% chapters
\input{patterns/00_empty/main}
\input{patterns/011_ret/main}
\input{patterns/01_helloworld/main}
\input{patterns/015_prolog_epilogue/main}
\input{patterns/02_stack/main}
\input{patterns/03_printf/main}
\input{patterns/04_scanf/main}
\input{patterns/05_passing_arguments/main}
\input{patterns/06_return_results/main}
\input{patterns/061_pointers/main}
\input{patterns/065_GOTO/main}
\input{patterns/07_jcc/main}
\input{patterns/08_switch/main}
\input{patterns/09_loops/main}
\input{patterns/10_strings/main}
\input{patterns/11_arith_optimizations/main}
\input{patterns/12_FPU/main}
\input{patterns/13_arrays/main}
\input{patterns/14_bitfields/main}
\EN{\input{patterns/145_LCG/main_EN}}
\RU{\input{patterns/145_LCG/main_RU}}
\input{patterns/15_structs/main}
\input{patterns/17_unions/main}
\input{patterns/18_pointers_to_functions/main}
\input{patterns/185_64bit_in_32_env/main}

\EN{\input{patterns/19_SIMD/main_EN}}
\RU{\input{patterns/19_SIMD/main_RU}}
\DE{\input{patterns/19_SIMD/main_DE}}

\EN{\input{patterns/20_x64/main_EN}}
\RU{\input{patterns/20_x64/main_RU}}

\EN{\input{patterns/205_floating_SIMD/main_EN}}
\RU{\input{patterns/205_floating_SIMD/main_RU}}
\DE{\input{patterns/205_floating_SIMD/main_DE}}

\EN{\input{patterns/ARM/main_EN}}
\RU{\input{patterns/ARM/main_RU}}
\DE{\input{patterns/ARM/main_DE}}

\input{patterns/MIPS/main}

\EN{\input{patterns/12_FPU/main_EN}}
\RU{\input{patterns/12_FPU/main_RU}}
\DE{\input{patterns/12_FPU/main_DE}}
\FR{\input{patterns/12_FPU/main_FR}}


\ifdefined\SPANISH
\chapter{Patrones de código}
\fi % SPANISH

\ifdefined\GERMAN
\chapter{Code-Muster}
\fi % GERMAN

\ifdefined\ENGLISH
\chapter{Code Patterns}
\fi % ENGLISH

\ifdefined\ITALIAN
\chapter{Forme di codice}
\fi % ITALIAN

\ifdefined\RUSSIAN
\chapter{Образцы кода}
\fi % RUSSIAN

\ifdefined\BRAZILIAN
\chapter{Padrões de códigos}
\fi % BRAZILIAN

\ifdefined\THAI
\chapter{รูปแบบของโค้ด}
\fi % THAI

\ifdefined\FRENCH
\chapter{Modèle de code}
\fi % FRENCH

\ifdefined\POLISH
\chapter{\PLph{}}
\fi % POLISH

% sections
\EN{\input{patterns/patterns_opt_dbg_EN}}
\ES{\input{patterns/patterns_opt_dbg_ES}}
\ITA{\input{patterns/patterns_opt_dbg_ITA}}
\PTBR{\input{patterns/patterns_opt_dbg_PTBR}}
\RU{\input{patterns/patterns_opt_dbg_RU}}
\THA{\input{patterns/patterns_opt_dbg_THA}}
\DE{\input{patterns/patterns_opt_dbg_DE}}
\FR{\input{patterns/patterns_opt_dbg_FR}}
\PL{\input{patterns/patterns_opt_dbg_PL}}

\RU{\section{Некоторые базовые понятия}}
\EN{\section{Some basics}}
\DE{\section{Einige Grundlagen}}
\FR{\section{Quelques bases}}
\ES{\section{\ESph{}}}
\ITA{\section{Alcune basi teoriche}}
\PTBR{\section{\PTBRph{}}}
\THA{\section{\THAph{}}}
\PL{\section{\PLph{}}}

% sections:
\EN{\input{patterns/intro_CPU_ISA_EN}}
\ES{\input{patterns/intro_CPU_ISA_ES}}
\ITA{\input{patterns/intro_CPU_ISA_ITA}}
\PTBR{\input{patterns/intro_CPU_ISA_PTBR}}
\RU{\input{patterns/intro_CPU_ISA_RU}}
\DE{\input{patterns/intro_CPU_ISA_DE}}
\FR{\input{patterns/intro_CPU_ISA_FR}}
\PL{\input{patterns/intro_CPU_ISA_PL}}

\EN{\input{patterns/numeral_EN}}
\RU{\input{patterns/numeral_RU}}
\ITA{\input{patterns/numeral_ITA}}
\DE{\input{patterns/numeral_DE}}
\FR{\input{patterns/numeral_FR}}
\PL{\input{patterns/numeral_PL}}

% chapters
\input{patterns/00_empty/main}
\input{patterns/011_ret/main}
\input{patterns/01_helloworld/main}
\input{patterns/015_prolog_epilogue/main}
\input{patterns/02_stack/main}
\input{patterns/03_printf/main}
\input{patterns/04_scanf/main}
\input{patterns/05_passing_arguments/main}
\input{patterns/06_return_results/main}
\input{patterns/061_pointers/main}
\input{patterns/065_GOTO/main}
\input{patterns/07_jcc/main}
\input{patterns/08_switch/main}
\input{patterns/09_loops/main}
\input{patterns/10_strings/main}
\input{patterns/11_arith_optimizations/main}
\input{patterns/12_FPU/main}
\input{patterns/13_arrays/main}
\input{patterns/14_bitfields/main}
\EN{\input{patterns/145_LCG/main_EN}}
\RU{\input{patterns/145_LCG/main_RU}}
\input{patterns/15_structs/main}
\input{patterns/17_unions/main}
\input{patterns/18_pointers_to_functions/main}
\input{patterns/185_64bit_in_32_env/main}

\EN{\input{patterns/19_SIMD/main_EN}}
\RU{\input{patterns/19_SIMD/main_RU}}
\DE{\input{patterns/19_SIMD/main_DE}}

\EN{\input{patterns/20_x64/main_EN}}
\RU{\input{patterns/20_x64/main_RU}}

\EN{\input{patterns/205_floating_SIMD/main_EN}}
\RU{\input{patterns/205_floating_SIMD/main_RU}}
\DE{\input{patterns/205_floating_SIMD/main_DE}}

\EN{\input{patterns/ARM/main_EN}}
\RU{\input{patterns/ARM/main_RU}}
\DE{\input{patterns/ARM/main_DE}}

\input{patterns/MIPS/main}

\ifdefined\SPANISH
\chapter{Patrones de código}
\fi % SPANISH

\ifdefined\GERMAN
\chapter{Code-Muster}
\fi % GERMAN

\ifdefined\ENGLISH
\chapter{Code Patterns}
\fi % ENGLISH

\ifdefined\ITALIAN
\chapter{Forme di codice}
\fi % ITALIAN

\ifdefined\RUSSIAN
\chapter{Образцы кода}
\fi % RUSSIAN

\ifdefined\BRAZILIAN
\chapter{Padrões de códigos}
\fi % BRAZILIAN

\ifdefined\THAI
\chapter{รูปแบบของโค้ด}
\fi % THAI

\ifdefined\FRENCH
\chapter{Modèle de code}
\fi % FRENCH

\ifdefined\POLISH
\chapter{\PLph{}}
\fi % POLISH

% sections
\EN{\input{patterns/patterns_opt_dbg_EN}}
\ES{\input{patterns/patterns_opt_dbg_ES}}
\ITA{\input{patterns/patterns_opt_dbg_ITA}}
\PTBR{\input{patterns/patterns_opt_dbg_PTBR}}
\RU{\input{patterns/patterns_opt_dbg_RU}}
\THA{\input{patterns/patterns_opt_dbg_THA}}
\DE{\input{patterns/patterns_opt_dbg_DE}}
\FR{\input{patterns/patterns_opt_dbg_FR}}
\PL{\input{patterns/patterns_opt_dbg_PL}}

\RU{\section{Некоторые базовые понятия}}
\EN{\section{Some basics}}
\DE{\section{Einige Grundlagen}}
\FR{\section{Quelques bases}}
\ES{\section{\ESph{}}}
\ITA{\section{Alcune basi teoriche}}
\PTBR{\section{\PTBRph{}}}
\THA{\section{\THAph{}}}
\PL{\section{\PLph{}}}

% sections:
\EN{\input{patterns/intro_CPU_ISA_EN}}
\ES{\input{patterns/intro_CPU_ISA_ES}}
\ITA{\input{patterns/intro_CPU_ISA_ITA}}
\PTBR{\input{patterns/intro_CPU_ISA_PTBR}}
\RU{\input{patterns/intro_CPU_ISA_RU}}
\DE{\input{patterns/intro_CPU_ISA_DE}}
\FR{\input{patterns/intro_CPU_ISA_FR}}
\PL{\input{patterns/intro_CPU_ISA_PL}}

\EN{\input{patterns/numeral_EN}}
\RU{\input{patterns/numeral_RU}}
\ITA{\input{patterns/numeral_ITA}}
\DE{\input{patterns/numeral_DE}}
\FR{\input{patterns/numeral_FR}}
\PL{\input{patterns/numeral_PL}}

% chapters
\input{patterns/00_empty/main}
\input{patterns/011_ret/main}
\input{patterns/01_helloworld/main}
\input{patterns/015_prolog_epilogue/main}
\input{patterns/02_stack/main}
\input{patterns/03_printf/main}
\input{patterns/04_scanf/main}
\input{patterns/05_passing_arguments/main}
\input{patterns/06_return_results/main}
\input{patterns/061_pointers/main}
\input{patterns/065_GOTO/main}
\input{patterns/07_jcc/main}
\input{patterns/08_switch/main}
\input{patterns/09_loops/main}
\input{patterns/10_strings/main}
\input{patterns/11_arith_optimizations/main}
\input{patterns/12_FPU/main}
\input{patterns/13_arrays/main}
\input{patterns/14_bitfields/main}
\EN{\input{patterns/145_LCG/main_EN}}
\RU{\input{patterns/145_LCG/main_RU}}
\input{patterns/15_structs/main}
\input{patterns/17_unions/main}
\input{patterns/18_pointers_to_functions/main}
\input{patterns/185_64bit_in_32_env/main}

\EN{\input{patterns/19_SIMD/main_EN}}
\RU{\input{patterns/19_SIMD/main_RU}}
\DE{\input{patterns/19_SIMD/main_DE}}

\EN{\input{patterns/20_x64/main_EN}}
\RU{\input{patterns/20_x64/main_RU}}

\EN{\input{patterns/205_floating_SIMD/main_EN}}
\RU{\input{patterns/205_floating_SIMD/main_RU}}
\DE{\input{patterns/205_floating_SIMD/main_DE}}

\EN{\input{patterns/ARM/main_EN}}
\RU{\input{patterns/ARM/main_RU}}
\DE{\input{patterns/ARM/main_DE}}

\input{patterns/MIPS/main}

\EN{\section{Returning Values}
\label{ret_val_func}

Another simple function is the one that simply returns a constant value:

\lstinputlisting[caption=\EN{\CCpp Code},style=customc]{patterns/011_ret/1.c}

Let's compile it.

\subsection{x86}

Here's what both the GCC and MSVC compilers produce (with optimization) on the x86 platform:

\lstinputlisting[caption=\Optimizing GCC/MSVC (\assemblyOutput),style=customasmx86]{patterns/011_ret/1.s}

\myindex{x86!\Instructions!RET}
There are just two instructions: the first places the value 123 into the \EAX register,
which is used by convention for storing the return
value, and the second one is \RET, which returns execution to the \gls{caller}.

The caller will take the result from the \EAX register.

\subsection{ARM}

There are a few differences on the ARM platform:

\lstinputlisting[caption=\OptimizingKeilVI (\ARMMode) ASM Output,style=customasmARM]{patterns/011_ret/1_Keil_ARM_O3.s}

ARM uses the register \Reg{0} for returning the results of functions, so 123 is copied into \Reg{0}.

\myindex{ARM!\Instructions!MOV}
\myindex{x86!\Instructions!MOV}
It is worth noting that \MOV is a misleading name for the instruction in both the x86 and ARM \ac{ISA}s.

The data is not in fact \IT{moved}, but \IT{copied}.

\subsection{MIPS}

\label{MIPS_leaf_function_ex1}

The GCC assembly output below lists registers by number:

\lstinputlisting[caption=\Optimizing GCC 4.4.5 (\assemblyOutput),style=customasmMIPS]{patterns/011_ret/MIPS.s}

\dots while \IDA does it by their pseudo names:

\lstinputlisting[caption=\Optimizing GCC 4.4.5 (IDA),style=customasmMIPS]{patterns/011_ret/MIPS_IDA.lst}

The \$2 (or \$V0) register is used to store the function's return value.
\myindex{MIPS!\Pseudoinstructions!LI}
\INS{LI} stands for ``Load Immediate'' and is the MIPS equivalent to \MOV.

\myindex{MIPS!\Instructions!J}
The other instruction is the jump instruction (J or JR) which returns the execution flow to the \gls{caller}.

\myindex{MIPS!Branch delay slot}
You might be wondering why the positions of the load instruction (LI) and the jump instruction (J or JR) are swapped. This is due to a \ac{RISC} feature called ``branch delay slot''.

The reason this happens is a quirk in the architecture of some RISC \ac{ISA}s and isn't important for our
purposes---we must simply keep in mind that in MIPS, the instruction following a jump or branch instruction
is executed \IT{before} the jump/branch instruction itself.

As a consequence, branch instructions always swap places with the instruction executed immediately beforehand.


In practice, functions which merely return 1 (\IT{true}) or 0 (\IT{false}) are very frequent.

The smallest ever of the standard UNIX utilities, \IT{/bin/true} and \IT{/bin/false} return 0 and 1 respectively, as an exit code.
(Zero as an exit code usually means success, non-zero means error.)
}
\RU{\subsubsection{std::string}
\myindex{\Cpp!STL!std::string}
\label{std_string}

\myparagraph{Как устроена структура}

Многие строковые библиотеки \InSqBrackets{\CNotes 2.2} обеспечивают структуру содержащую ссылку 
на буфер собственно со строкой, переменная всегда содержащую длину строки 
(что очень удобно для массы функций \InSqBrackets{\CNotes 2.2.1}) и переменную содержащую текущий размер буфера.

Строка в буфере обыкновенно оканчивается нулем: это для того чтобы указатель на буфер можно было
передавать в функции требующие на вход обычную сишную \ac{ASCIIZ}-строку.

Стандарт \Cpp не описывает, как именно нужно реализовывать std::string,
но, как правило, они реализованы как описано выше, с небольшими дополнениями.

Строки в \Cpp это не класс (как, например, QString в Qt), а темплейт (basic\_string), 
это сделано для того чтобы поддерживать 
строки содержащие разного типа символы: как минимум \Tchar и \IT{wchar\_t}.

Так что, std::string это класс с базовым типом \Tchar.

А std::wstring это класс с базовым типом \IT{wchar\_t}.

\mysubparagraph{MSVC}

В реализации MSVC, вместо ссылки на буфер может содержаться сам буфер (если строка короче 16-и символов).

Это означает, что каждая короткая строка будет занимать в памяти по крайней мере $16 + 4 + 4 = 24$ 
байт для 32-битной среды либо $16 + 8 + 8 = 32$ 
байта в 64-битной, а если строка длиннее 16-и символов, то прибавьте еще длину самой строки.

\lstinputlisting[caption=пример для MSVC,style=customc]{\CURPATH/STL/string/MSVC_RU.cpp}

Собственно, из этого исходника почти всё ясно.

Несколько замечаний:

Если строка короче 16-и символов, 
то отдельный буфер для строки в \glslink{heap}{куче} выделяться не будет.

Это удобно потому что на практике, основная часть строк действительно короткие.
Вероятно, разработчики в Microsoft выбрали размер в 16 символов как разумный баланс.

Теперь очень важный момент в конце функции main(): мы не пользуемся методом c\_str(), тем не менее,
если это скомпилировать и запустить, то обе строки появятся в консоли!

Работает это вот почему.

В первом случае строка короче 16-и символов и в начале объекта std::string (его можно рассматривать
просто как структуру) расположен буфер с этой строкой.
\printf трактует указатель как указатель на массив символов оканчивающийся нулем и поэтому всё работает.

Вывод второй строки (длиннее 16-и символов) даже еще опаснее: это вообще типичная программистская ошибка 
(или опечатка), забыть дописать c\_str().
Это работает потому что в это время в начале структуры расположен указатель на буфер.
Это может надолго остаться незамеченным: до тех пока там не появится строка 
короче 16-и символов, тогда процесс упадет.

\mysubparagraph{GCC}

В реализации GCC в структуре есть еще одна переменная --- reference count.

Интересно, что указатель на экземпляр класса std::string в GCC указывает не на начало самой структуры, 
а на указатель на буфера.
В libstdc++-v3\textbackslash{}include\textbackslash{}bits\textbackslash{}basic\_string.h 
мы можем прочитать что это сделано для удобства отладки:

\begin{lstlisting}
   *  The reason you want _M_data pointing to the character %array and
   *  not the _Rep is so that the debugger can see the string
   *  contents. (Probably we should add a non-inline member to get
   *  the _Rep for the debugger to use, so users can check the actual
   *  string length.)
\end{lstlisting}

\href{http://go.yurichev.com/17085}{исходный код basic\_string.h}

В нашем примере мы учитываем это:

\lstinputlisting[caption=пример для GCC,style=customc]{\CURPATH/STL/string/GCC_RU.cpp}

Нужны еще небольшие хаки чтобы сымитировать типичную ошибку, которую мы уже видели выше, из-за
более ужесточенной проверки типов в GCC, тем не менее, printf() работает и здесь без c\_str().

\myparagraph{Чуть более сложный пример}

\lstinputlisting[style=customc]{\CURPATH/STL/string/3.cpp}

\lstinputlisting[caption=MSVC 2012,style=customasmx86]{\CURPATH/STL/string/3_MSVC_RU.asm}

Собственно, компилятор не конструирует строки статически: да в общем-то и как
это возможно, если буфер с ней нужно хранить в \glslink{heap}{куче}?

Вместо этого в сегменте данных хранятся обычные \ac{ASCIIZ}-строки, а позже, во время выполнения, 
при помощи метода \q{assign}, конструируются строки s1 и s2
.
При помощи \TT{operator+}, создается строка s3.

Обратите внимание на то что вызов метода c\_str() отсутствует,
потому что его код достаточно короткий и компилятор вставил его прямо здесь:
если строка короче 16-и байт, то в регистре EAX остается указатель на буфер,
а если длиннее, то из этого же места достается адрес на буфер расположенный в \glslink{heap}{куче}.

Далее следуют вызовы трех деструкторов, причем, они вызываются только если строка длиннее 16-и байт:
тогда нужно освободить буфера в \glslink{heap}{куче}.
В противном случае, так как все три объекта std::string хранятся в стеке,
они освобождаются автоматически после выхода из функции.

Следовательно, работа с короткими строками более быстрая из-за м\'{е}ньшего обращения к \glslink{heap}{куче}.

Код на GCC даже проще (из-за того, что в GCC, как мы уже видели, не реализована возможность хранить короткую
строку прямо в структуре):

% TODO1 comment each function meaning
\lstinputlisting[caption=GCC 4.8.1,style=customasmx86]{\CURPATH/STL/string/3_GCC_RU.s}

Можно заметить, что в деструкторы передается не указатель на объект,
а указатель на место за 12 байт (или 3 слова) перед ним, то есть, на настоящее начало структуры.

\myparagraph{std::string как глобальная переменная}
\label{sec:std_string_as_global_variable}

Опытные программисты на \Cpp знают, что глобальные переменные \ac{STL}-типов вполне можно объявлять.

Да, действительно:

\lstinputlisting[style=customc]{\CURPATH/STL/string/5.cpp}

Но как и где будет вызываться конструктор \TT{std::string}?

На самом деле, эта переменная будет инициализирована даже перед началом \main.

\lstinputlisting[caption=MSVC 2012: здесь конструируется глобальная переменная{,} а также регистрируется её деструктор,style=customasmx86]{\CURPATH/STL/string/5_MSVC_p2.asm}

\lstinputlisting[caption=MSVC 2012: здесь глобальная переменная используется в \main,style=customasmx86]{\CURPATH/STL/string/5_MSVC_p1.asm}

\lstinputlisting[caption=MSVC 2012: эта функция-деструктор вызывается перед выходом,style=customasmx86]{\CURPATH/STL/string/5_MSVC_p3.asm}

\myindex{\CStandardLibrary!atexit()}
В реальности, из \ac{CRT}, еще до вызова main(), вызывается специальная функция,
в которой перечислены все конструкторы подобных переменных.
Более того: при помощи atexit() регистрируется функция, которая будет вызвана в конце работы программы:
в этой функции компилятор собирает вызовы деструкторов всех подобных глобальных переменных.

GCC работает похожим образом:

\lstinputlisting[caption=GCC 4.8.1,style=customasmx86]{\CURPATH/STL/string/5_GCC.s}

Но он не выделяет отдельной функции в которой будут собраны деструкторы: 
каждый деструктор передается в atexit() по одному.

% TODO а если глобальная STL-переменная в другом модуле? надо проверить.

}
\ifdefined\SPANISH
\chapter{Patrones de código}
\fi % SPANISH

\ifdefined\GERMAN
\chapter{Code-Muster}
\fi % GERMAN

\ifdefined\ENGLISH
\chapter{Code Patterns}
\fi % ENGLISH

\ifdefined\ITALIAN
\chapter{Forme di codice}
\fi % ITALIAN

\ifdefined\RUSSIAN
\chapter{Образцы кода}
\fi % RUSSIAN

\ifdefined\BRAZILIAN
\chapter{Padrões de códigos}
\fi % BRAZILIAN

\ifdefined\THAI
\chapter{รูปแบบของโค้ด}
\fi % THAI

\ifdefined\FRENCH
\chapter{Modèle de code}
\fi % FRENCH

\ifdefined\POLISH
\chapter{\PLph{}}
\fi % POLISH

% sections
\EN{\input{patterns/patterns_opt_dbg_EN}}
\ES{\input{patterns/patterns_opt_dbg_ES}}
\ITA{\input{patterns/patterns_opt_dbg_ITA}}
\PTBR{\input{patterns/patterns_opt_dbg_PTBR}}
\RU{\input{patterns/patterns_opt_dbg_RU}}
\THA{\input{patterns/patterns_opt_dbg_THA}}
\DE{\input{patterns/patterns_opt_dbg_DE}}
\FR{\input{patterns/patterns_opt_dbg_FR}}
\PL{\input{patterns/patterns_opt_dbg_PL}}

\RU{\section{Некоторые базовые понятия}}
\EN{\section{Some basics}}
\DE{\section{Einige Grundlagen}}
\FR{\section{Quelques bases}}
\ES{\section{\ESph{}}}
\ITA{\section{Alcune basi teoriche}}
\PTBR{\section{\PTBRph{}}}
\THA{\section{\THAph{}}}
\PL{\section{\PLph{}}}

% sections:
\EN{\input{patterns/intro_CPU_ISA_EN}}
\ES{\input{patterns/intro_CPU_ISA_ES}}
\ITA{\input{patterns/intro_CPU_ISA_ITA}}
\PTBR{\input{patterns/intro_CPU_ISA_PTBR}}
\RU{\input{patterns/intro_CPU_ISA_RU}}
\DE{\input{patterns/intro_CPU_ISA_DE}}
\FR{\input{patterns/intro_CPU_ISA_FR}}
\PL{\input{patterns/intro_CPU_ISA_PL}}

\EN{\input{patterns/numeral_EN}}
\RU{\input{patterns/numeral_RU}}
\ITA{\input{patterns/numeral_ITA}}
\DE{\input{patterns/numeral_DE}}
\FR{\input{patterns/numeral_FR}}
\PL{\input{patterns/numeral_PL}}

% chapters
\input{patterns/00_empty/main}
\input{patterns/011_ret/main}
\input{patterns/01_helloworld/main}
\input{patterns/015_prolog_epilogue/main}
\input{patterns/02_stack/main}
\input{patterns/03_printf/main}
\input{patterns/04_scanf/main}
\input{patterns/05_passing_arguments/main}
\input{patterns/06_return_results/main}
\input{patterns/061_pointers/main}
\input{patterns/065_GOTO/main}
\input{patterns/07_jcc/main}
\input{patterns/08_switch/main}
\input{patterns/09_loops/main}
\input{patterns/10_strings/main}
\input{patterns/11_arith_optimizations/main}
\input{patterns/12_FPU/main}
\input{patterns/13_arrays/main}
\input{patterns/14_bitfields/main}
\EN{\input{patterns/145_LCG/main_EN}}
\RU{\input{patterns/145_LCG/main_RU}}
\input{patterns/15_structs/main}
\input{patterns/17_unions/main}
\input{patterns/18_pointers_to_functions/main}
\input{patterns/185_64bit_in_32_env/main}

\EN{\input{patterns/19_SIMD/main_EN}}
\RU{\input{patterns/19_SIMD/main_RU}}
\DE{\input{patterns/19_SIMD/main_DE}}

\EN{\input{patterns/20_x64/main_EN}}
\RU{\input{patterns/20_x64/main_RU}}

\EN{\input{patterns/205_floating_SIMD/main_EN}}
\RU{\input{patterns/205_floating_SIMD/main_RU}}
\DE{\input{patterns/205_floating_SIMD/main_DE}}

\EN{\input{patterns/ARM/main_EN}}
\RU{\input{patterns/ARM/main_RU}}
\DE{\input{patterns/ARM/main_DE}}

\input{patterns/MIPS/main}

\ifdefined\SPANISH
\chapter{Patrones de código}
\fi % SPANISH

\ifdefined\GERMAN
\chapter{Code-Muster}
\fi % GERMAN

\ifdefined\ENGLISH
\chapter{Code Patterns}
\fi % ENGLISH

\ifdefined\ITALIAN
\chapter{Forme di codice}
\fi % ITALIAN

\ifdefined\RUSSIAN
\chapter{Образцы кода}
\fi % RUSSIAN

\ifdefined\BRAZILIAN
\chapter{Padrões de códigos}
\fi % BRAZILIAN

\ifdefined\THAI
\chapter{รูปแบบของโค้ด}
\fi % THAI

\ifdefined\FRENCH
\chapter{Modèle de code}
\fi % FRENCH

\ifdefined\POLISH
\chapter{\PLph{}}
\fi % POLISH

% sections
\EN{\input{patterns/patterns_opt_dbg_EN}}
\ES{\input{patterns/patterns_opt_dbg_ES}}
\ITA{\input{patterns/patterns_opt_dbg_ITA}}
\PTBR{\input{patterns/patterns_opt_dbg_PTBR}}
\RU{\input{patterns/patterns_opt_dbg_RU}}
\THA{\input{patterns/patterns_opt_dbg_THA}}
\DE{\input{patterns/patterns_opt_dbg_DE}}
\FR{\input{patterns/patterns_opt_dbg_FR}}
\PL{\input{patterns/patterns_opt_dbg_PL}}

\RU{\section{Некоторые базовые понятия}}
\EN{\section{Some basics}}
\DE{\section{Einige Grundlagen}}
\FR{\section{Quelques bases}}
\ES{\section{\ESph{}}}
\ITA{\section{Alcune basi teoriche}}
\PTBR{\section{\PTBRph{}}}
\THA{\section{\THAph{}}}
\PL{\section{\PLph{}}}

% sections:
\EN{\input{patterns/intro_CPU_ISA_EN}}
\ES{\input{patterns/intro_CPU_ISA_ES}}
\ITA{\input{patterns/intro_CPU_ISA_ITA}}
\PTBR{\input{patterns/intro_CPU_ISA_PTBR}}
\RU{\input{patterns/intro_CPU_ISA_RU}}
\DE{\input{patterns/intro_CPU_ISA_DE}}
\FR{\input{patterns/intro_CPU_ISA_FR}}
\PL{\input{patterns/intro_CPU_ISA_PL}}

\EN{\input{patterns/numeral_EN}}
\RU{\input{patterns/numeral_RU}}
\ITA{\input{patterns/numeral_ITA}}
\DE{\input{patterns/numeral_DE}}
\FR{\input{patterns/numeral_FR}}
\PL{\input{patterns/numeral_PL}}

% chapters
\input{patterns/00_empty/main}
\input{patterns/011_ret/main}
\input{patterns/01_helloworld/main}
\input{patterns/015_prolog_epilogue/main}
\input{patterns/02_stack/main}
\input{patterns/03_printf/main}
\input{patterns/04_scanf/main}
\input{patterns/05_passing_arguments/main}
\input{patterns/06_return_results/main}
\input{patterns/061_pointers/main}
\input{patterns/065_GOTO/main}
\input{patterns/07_jcc/main}
\input{patterns/08_switch/main}
\input{patterns/09_loops/main}
\input{patterns/10_strings/main}
\input{patterns/11_arith_optimizations/main}
\input{patterns/12_FPU/main}
\input{patterns/13_arrays/main}
\input{patterns/14_bitfields/main}
\EN{\input{patterns/145_LCG/main_EN}}
\RU{\input{patterns/145_LCG/main_RU}}
\input{patterns/15_structs/main}
\input{patterns/17_unions/main}
\input{patterns/18_pointers_to_functions/main}
\input{patterns/185_64bit_in_32_env/main}

\EN{\input{patterns/19_SIMD/main_EN}}
\RU{\input{patterns/19_SIMD/main_RU}}
\DE{\input{patterns/19_SIMD/main_DE}}

\EN{\input{patterns/20_x64/main_EN}}
\RU{\input{patterns/20_x64/main_RU}}

\EN{\input{patterns/205_floating_SIMD/main_EN}}
\RU{\input{patterns/205_floating_SIMD/main_RU}}
\DE{\input{patterns/205_floating_SIMD/main_DE}}

\EN{\input{patterns/ARM/main_EN}}
\RU{\input{patterns/ARM/main_RU}}
\DE{\input{patterns/ARM/main_DE}}

\input{patterns/MIPS/main}

\ifdefined\SPANISH
\chapter{Patrones de código}
\fi % SPANISH

\ifdefined\GERMAN
\chapter{Code-Muster}
\fi % GERMAN

\ifdefined\ENGLISH
\chapter{Code Patterns}
\fi % ENGLISH

\ifdefined\ITALIAN
\chapter{Forme di codice}
\fi % ITALIAN

\ifdefined\RUSSIAN
\chapter{Образцы кода}
\fi % RUSSIAN

\ifdefined\BRAZILIAN
\chapter{Padrões de códigos}
\fi % BRAZILIAN

\ifdefined\THAI
\chapter{รูปแบบของโค้ด}
\fi % THAI

\ifdefined\FRENCH
\chapter{Modèle de code}
\fi % FRENCH

\ifdefined\POLISH
\chapter{\PLph{}}
\fi % POLISH

% sections
\EN{\input{patterns/patterns_opt_dbg_EN}}
\ES{\input{patterns/patterns_opt_dbg_ES}}
\ITA{\input{patterns/patterns_opt_dbg_ITA}}
\PTBR{\input{patterns/patterns_opt_dbg_PTBR}}
\RU{\input{patterns/patterns_opt_dbg_RU}}
\THA{\input{patterns/patterns_opt_dbg_THA}}
\DE{\input{patterns/patterns_opt_dbg_DE}}
\FR{\input{patterns/patterns_opt_dbg_FR}}
\PL{\input{patterns/patterns_opt_dbg_PL}}

\RU{\section{Некоторые базовые понятия}}
\EN{\section{Some basics}}
\DE{\section{Einige Grundlagen}}
\FR{\section{Quelques bases}}
\ES{\section{\ESph{}}}
\ITA{\section{Alcune basi teoriche}}
\PTBR{\section{\PTBRph{}}}
\THA{\section{\THAph{}}}
\PL{\section{\PLph{}}}

% sections:
\EN{\input{patterns/intro_CPU_ISA_EN}}
\ES{\input{patterns/intro_CPU_ISA_ES}}
\ITA{\input{patterns/intro_CPU_ISA_ITA}}
\PTBR{\input{patterns/intro_CPU_ISA_PTBR}}
\RU{\input{patterns/intro_CPU_ISA_RU}}
\DE{\input{patterns/intro_CPU_ISA_DE}}
\FR{\input{patterns/intro_CPU_ISA_FR}}
\PL{\input{patterns/intro_CPU_ISA_PL}}

\EN{\input{patterns/numeral_EN}}
\RU{\input{patterns/numeral_RU}}
\ITA{\input{patterns/numeral_ITA}}
\DE{\input{patterns/numeral_DE}}
\FR{\input{patterns/numeral_FR}}
\PL{\input{patterns/numeral_PL}}

% chapters
\input{patterns/00_empty/main}
\input{patterns/011_ret/main}
\input{patterns/01_helloworld/main}
\input{patterns/015_prolog_epilogue/main}
\input{patterns/02_stack/main}
\input{patterns/03_printf/main}
\input{patterns/04_scanf/main}
\input{patterns/05_passing_arguments/main}
\input{patterns/06_return_results/main}
\input{patterns/061_pointers/main}
\input{patterns/065_GOTO/main}
\input{patterns/07_jcc/main}
\input{patterns/08_switch/main}
\input{patterns/09_loops/main}
\input{patterns/10_strings/main}
\input{patterns/11_arith_optimizations/main}
\input{patterns/12_FPU/main}
\input{patterns/13_arrays/main}
\input{patterns/14_bitfields/main}
\EN{\input{patterns/145_LCG/main_EN}}
\RU{\input{patterns/145_LCG/main_RU}}
\input{patterns/15_structs/main}
\input{patterns/17_unions/main}
\input{patterns/18_pointers_to_functions/main}
\input{patterns/185_64bit_in_32_env/main}

\EN{\input{patterns/19_SIMD/main_EN}}
\RU{\input{patterns/19_SIMD/main_RU}}
\DE{\input{patterns/19_SIMD/main_DE}}

\EN{\input{patterns/20_x64/main_EN}}
\RU{\input{patterns/20_x64/main_RU}}

\EN{\input{patterns/205_floating_SIMD/main_EN}}
\RU{\input{patterns/205_floating_SIMD/main_RU}}
\DE{\input{patterns/205_floating_SIMD/main_DE}}

\EN{\input{patterns/ARM/main_EN}}
\RU{\input{patterns/ARM/main_RU}}
\DE{\input{patterns/ARM/main_DE}}

\input{patterns/MIPS/main}

\ifdefined\SPANISH
\chapter{Patrones de código}
\fi % SPANISH

\ifdefined\GERMAN
\chapter{Code-Muster}
\fi % GERMAN

\ifdefined\ENGLISH
\chapter{Code Patterns}
\fi % ENGLISH

\ifdefined\ITALIAN
\chapter{Forme di codice}
\fi % ITALIAN

\ifdefined\RUSSIAN
\chapter{Образцы кода}
\fi % RUSSIAN

\ifdefined\BRAZILIAN
\chapter{Padrões de códigos}
\fi % BRAZILIAN

\ifdefined\THAI
\chapter{รูปแบบของโค้ด}
\fi % THAI

\ifdefined\FRENCH
\chapter{Modèle de code}
\fi % FRENCH

\ifdefined\POLISH
\chapter{\PLph{}}
\fi % POLISH

% sections
\EN{\input{patterns/patterns_opt_dbg_EN}}
\ES{\input{patterns/patterns_opt_dbg_ES}}
\ITA{\input{patterns/patterns_opt_dbg_ITA}}
\PTBR{\input{patterns/patterns_opt_dbg_PTBR}}
\RU{\input{patterns/patterns_opt_dbg_RU}}
\THA{\input{patterns/patterns_opt_dbg_THA}}
\DE{\input{patterns/patterns_opt_dbg_DE}}
\FR{\input{patterns/patterns_opt_dbg_FR}}
\PL{\input{patterns/patterns_opt_dbg_PL}}

\RU{\section{Некоторые базовые понятия}}
\EN{\section{Some basics}}
\DE{\section{Einige Grundlagen}}
\FR{\section{Quelques bases}}
\ES{\section{\ESph{}}}
\ITA{\section{Alcune basi teoriche}}
\PTBR{\section{\PTBRph{}}}
\THA{\section{\THAph{}}}
\PL{\section{\PLph{}}}

% sections:
\EN{\input{patterns/intro_CPU_ISA_EN}}
\ES{\input{patterns/intro_CPU_ISA_ES}}
\ITA{\input{patterns/intro_CPU_ISA_ITA}}
\PTBR{\input{patterns/intro_CPU_ISA_PTBR}}
\RU{\input{patterns/intro_CPU_ISA_RU}}
\DE{\input{patterns/intro_CPU_ISA_DE}}
\FR{\input{patterns/intro_CPU_ISA_FR}}
\PL{\input{patterns/intro_CPU_ISA_PL}}

\EN{\input{patterns/numeral_EN}}
\RU{\input{patterns/numeral_RU}}
\ITA{\input{patterns/numeral_ITA}}
\DE{\input{patterns/numeral_DE}}
\FR{\input{patterns/numeral_FR}}
\PL{\input{patterns/numeral_PL}}

% chapters
\input{patterns/00_empty/main}
\input{patterns/011_ret/main}
\input{patterns/01_helloworld/main}
\input{patterns/015_prolog_epilogue/main}
\input{patterns/02_stack/main}
\input{patterns/03_printf/main}
\input{patterns/04_scanf/main}
\input{patterns/05_passing_arguments/main}
\input{patterns/06_return_results/main}
\input{patterns/061_pointers/main}
\input{patterns/065_GOTO/main}
\input{patterns/07_jcc/main}
\input{patterns/08_switch/main}
\input{patterns/09_loops/main}
\input{patterns/10_strings/main}
\input{patterns/11_arith_optimizations/main}
\input{patterns/12_FPU/main}
\input{patterns/13_arrays/main}
\input{patterns/14_bitfields/main}
\EN{\input{patterns/145_LCG/main_EN}}
\RU{\input{patterns/145_LCG/main_RU}}
\input{patterns/15_structs/main}
\input{patterns/17_unions/main}
\input{patterns/18_pointers_to_functions/main}
\input{patterns/185_64bit_in_32_env/main}

\EN{\input{patterns/19_SIMD/main_EN}}
\RU{\input{patterns/19_SIMD/main_RU}}
\DE{\input{patterns/19_SIMD/main_DE}}

\EN{\input{patterns/20_x64/main_EN}}
\RU{\input{patterns/20_x64/main_RU}}

\EN{\input{patterns/205_floating_SIMD/main_EN}}
\RU{\input{patterns/205_floating_SIMD/main_RU}}
\DE{\input{patterns/205_floating_SIMD/main_DE}}

\EN{\input{patterns/ARM/main_EN}}
\RU{\input{patterns/ARM/main_RU}}
\DE{\input{patterns/ARM/main_DE}}

\input{patterns/MIPS/main}


\EN{\section{Returning Values}
\label{ret_val_func}

Another simple function is the one that simply returns a constant value:

\lstinputlisting[caption=\EN{\CCpp Code},style=customc]{patterns/011_ret/1.c}

Let's compile it.

\subsection{x86}

Here's what both the GCC and MSVC compilers produce (with optimization) on the x86 platform:

\lstinputlisting[caption=\Optimizing GCC/MSVC (\assemblyOutput),style=customasmx86]{patterns/011_ret/1.s}

\myindex{x86!\Instructions!RET}
There are just two instructions: the first places the value 123 into the \EAX register,
which is used by convention for storing the return
value, and the second one is \RET, which returns execution to the \gls{caller}.

The caller will take the result from the \EAX register.

\subsection{ARM}

There are a few differences on the ARM platform:

\lstinputlisting[caption=\OptimizingKeilVI (\ARMMode) ASM Output,style=customasmARM]{patterns/011_ret/1_Keil_ARM_O3.s}

ARM uses the register \Reg{0} for returning the results of functions, so 123 is copied into \Reg{0}.

\myindex{ARM!\Instructions!MOV}
\myindex{x86!\Instructions!MOV}
It is worth noting that \MOV is a misleading name for the instruction in both the x86 and ARM \ac{ISA}s.

The data is not in fact \IT{moved}, but \IT{copied}.

\subsection{MIPS}

\label{MIPS_leaf_function_ex1}

The GCC assembly output below lists registers by number:

\lstinputlisting[caption=\Optimizing GCC 4.4.5 (\assemblyOutput),style=customasmMIPS]{patterns/011_ret/MIPS.s}

\dots while \IDA does it by their pseudo names:

\lstinputlisting[caption=\Optimizing GCC 4.4.5 (IDA),style=customasmMIPS]{patterns/011_ret/MIPS_IDA.lst}

The \$2 (or \$V0) register is used to store the function's return value.
\myindex{MIPS!\Pseudoinstructions!LI}
\INS{LI} stands for ``Load Immediate'' and is the MIPS equivalent to \MOV.

\myindex{MIPS!\Instructions!J}
The other instruction is the jump instruction (J or JR) which returns the execution flow to the \gls{caller}.

\myindex{MIPS!Branch delay slot}
You might be wondering why the positions of the load instruction (LI) and the jump instruction (J or JR) are swapped. This is due to a \ac{RISC} feature called ``branch delay slot''.

The reason this happens is a quirk in the architecture of some RISC \ac{ISA}s and isn't important for our
purposes---we must simply keep in mind that in MIPS, the instruction following a jump or branch instruction
is executed \IT{before} the jump/branch instruction itself.

As a consequence, branch instructions always swap places with the instruction executed immediately beforehand.


In practice, functions which merely return 1 (\IT{true}) or 0 (\IT{false}) are very frequent.

The smallest ever of the standard UNIX utilities, \IT{/bin/true} and \IT{/bin/false} return 0 and 1 respectively, as an exit code.
(Zero as an exit code usually means success, non-zero means error.)
}
\RU{\subsubsection{std::string}
\myindex{\Cpp!STL!std::string}
\label{std_string}

\myparagraph{Как устроена структура}

Многие строковые библиотеки \InSqBrackets{\CNotes 2.2} обеспечивают структуру содержащую ссылку 
на буфер собственно со строкой, переменная всегда содержащую длину строки 
(что очень удобно для массы функций \InSqBrackets{\CNotes 2.2.1}) и переменную содержащую текущий размер буфера.

Строка в буфере обыкновенно оканчивается нулем: это для того чтобы указатель на буфер можно было
передавать в функции требующие на вход обычную сишную \ac{ASCIIZ}-строку.

Стандарт \Cpp не описывает, как именно нужно реализовывать std::string,
но, как правило, они реализованы как описано выше, с небольшими дополнениями.

Строки в \Cpp это не класс (как, например, QString в Qt), а темплейт (basic\_string), 
это сделано для того чтобы поддерживать 
строки содержащие разного типа символы: как минимум \Tchar и \IT{wchar\_t}.

Так что, std::string это класс с базовым типом \Tchar.

А std::wstring это класс с базовым типом \IT{wchar\_t}.

\mysubparagraph{MSVC}

В реализации MSVC, вместо ссылки на буфер может содержаться сам буфер (если строка короче 16-и символов).

Это означает, что каждая короткая строка будет занимать в памяти по крайней мере $16 + 4 + 4 = 24$ 
байт для 32-битной среды либо $16 + 8 + 8 = 32$ 
байта в 64-битной, а если строка длиннее 16-и символов, то прибавьте еще длину самой строки.

\lstinputlisting[caption=пример для MSVC,style=customc]{\CURPATH/STL/string/MSVC_RU.cpp}

Собственно, из этого исходника почти всё ясно.

Несколько замечаний:

Если строка короче 16-и символов, 
то отдельный буфер для строки в \glslink{heap}{куче} выделяться не будет.

Это удобно потому что на практике, основная часть строк действительно короткие.
Вероятно, разработчики в Microsoft выбрали размер в 16 символов как разумный баланс.

Теперь очень важный момент в конце функции main(): мы не пользуемся методом c\_str(), тем не менее,
если это скомпилировать и запустить, то обе строки появятся в консоли!

Работает это вот почему.

В первом случае строка короче 16-и символов и в начале объекта std::string (его можно рассматривать
просто как структуру) расположен буфер с этой строкой.
\printf трактует указатель как указатель на массив символов оканчивающийся нулем и поэтому всё работает.

Вывод второй строки (длиннее 16-и символов) даже еще опаснее: это вообще типичная программистская ошибка 
(или опечатка), забыть дописать c\_str().
Это работает потому что в это время в начале структуры расположен указатель на буфер.
Это может надолго остаться незамеченным: до тех пока там не появится строка 
короче 16-и символов, тогда процесс упадет.

\mysubparagraph{GCC}

В реализации GCC в структуре есть еще одна переменная --- reference count.

Интересно, что указатель на экземпляр класса std::string в GCC указывает не на начало самой структуры, 
а на указатель на буфера.
В libstdc++-v3\textbackslash{}include\textbackslash{}bits\textbackslash{}basic\_string.h 
мы можем прочитать что это сделано для удобства отладки:

\begin{lstlisting}
   *  The reason you want _M_data pointing to the character %array and
   *  not the _Rep is so that the debugger can see the string
   *  contents. (Probably we should add a non-inline member to get
   *  the _Rep for the debugger to use, so users can check the actual
   *  string length.)
\end{lstlisting}

\href{http://go.yurichev.com/17085}{исходный код basic\_string.h}

В нашем примере мы учитываем это:

\lstinputlisting[caption=пример для GCC,style=customc]{\CURPATH/STL/string/GCC_RU.cpp}

Нужны еще небольшие хаки чтобы сымитировать типичную ошибку, которую мы уже видели выше, из-за
более ужесточенной проверки типов в GCC, тем не менее, printf() работает и здесь без c\_str().

\myparagraph{Чуть более сложный пример}

\lstinputlisting[style=customc]{\CURPATH/STL/string/3.cpp}

\lstinputlisting[caption=MSVC 2012,style=customasmx86]{\CURPATH/STL/string/3_MSVC_RU.asm}

Собственно, компилятор не конструирует строки статически: да в общем-то и как
это возможно, если буфер с ней нужно хранить в \glslink{heap}{куче}?

Вместо этого в сегменте данных хранятся обычные \ac{ASCIIZ}-строки, а позже, во время выполнения, 
при помощи метода \q{assign}, конструируются строки s1 и s2
.
При помощи \TT{operator+}, создается строка s3.

Обратите внимание на то что вызов метода c\_str() отсутствует,
потому что его код достаточно короткий и компилятор вставил его прямо здесь:
если строка короче 16-и байт, то в регистре EAX остается указатель на буфер,
а если длиннее, то из этого же места достается адрес на буфер расположенный в \glslink{heap}{куче}.

Далее следуют вызовы трех деструкторов, причем, они вызываются только если строка длиннее 16-и байт:
тогда нужно освободить буфера в \glslink{heap}{куче}.
В противном случае, так как все три объекта std::string хранятся в стеке,
они освобождаются автоматически после выхода из функции.

Следовательно, работа с короткими строками более быстрая из-за м\'{е}ньшего обращения к \glslink{heap}{куче}.

Код на GCC даже проще (из-за того, что в GCC, как мы уже видели, не реализована возможность хранить короткую
строку прямо в структуре):

% TODO1 comment each function meaning
\lstinputlisting[caption=GCC 4.8.1,style=customasmx86]{\CURPATH/STL/string/3_GCC_RU.s}

Можно заметить, что в деструкторы передается не указатель на объект,
а указатель на место за 12 байт (или 3 слова) перед ним, то есть, на настоящее начало структуры.

\myparagraph{std::string как глобальная переменная}
\label{sec:std_string_as_global_variable}

Опытные программисты на \Cpp знают, что глобальные переменные \ac{STL}-типов вполне можно объявлять.

Да, действительно:

\lstinputlisting[style=customc]{\CURPATH/STL/string/5.cpp}

Но как и где будет вызываться конструктор \TT{std::string}?

На самом деле, эта переменная будет инициализирована даже перед началом \main.

\lstinputlisting[caption=MSVC 2012: здесь конструируется глобальная переменная{,} а также регистрируется её деструктор,style=customasmx86]{\CURPATH/STL/string/5_MSVC_p2.asm}

\lstinputlisting[caption=MSVC 2012: здесь глобальная переменная используется в \main,style=customasmx86]{\CURPATH/STL/string/5_MSVC_p1.asm}

\lstinputlisting[caption=MSVC 2012: эта функция-деструктор вызывается перед выходом,style=customasmx86]{\CURPATH/STL/string/5_MSVC_p3.asm}

\myindex{\CStandardLibrary!atexit()}
В реальности, из \ac{CRT}, еще до вызова main(), вызывается специальная функция,
в которой перечислены все конструкторы подобных переменных.
Более того: при помощи atexit() регистрируется функция, которая будет вызвана в конце работы программы:
в этой функции компилятор собирает вызовы деструкторов всех подобных глобальных переменных.

GCC работает похожим образом:

\lstinputlisting[caption=GCC 4.8.1,style=customasmx86]{\CURPATH/STL/string/5_GCC.s}

Но он не выделяет отдельной функции в которой будут собраны деструкторы: 
каждый деструктор передается в atexit() по одному.

% TODO а если глобальная STL-переменная в другом модуле? надо проверить.

}
\DE{\subsection{Einfachste XOR-Verschlüsselung überhaupt}

Ich habe einmal eine Software gesehen, bei der alle Debugging-Ausgaben mit XOR mit dem Wert 3
verschlüsselt wurden. Mit anderen Worten, die beiden niedrigsten Bits aller Buchstaben wurden invertiert.

``Hello, world'' wurde zu ``Kfool/\#tlqog'':

\begin{lstlisting}
#!/usr/bin/python

msg="Hello, world!"

print "".join(map(lambda x: chr(ord(x)^3), msg))
\end{lstlisting}

Das ist eine ziemlich interessante Verschlüsselung (oder besser eine Verschleierung),
weil sie zwei wichtige Eigenschaften hat:
1) es ist eine einzige Funktion zum Verschlüsseln und entschlüsseln, sie muss nur wiederholt angewendet werden
2) die entstehenden Buchstaben befinden sich im druckbaren Bereich, also die ganze Zeichenkette kann ohne
Escape-Symbole im Code verwendet werden.

Die zweite Eigenschaft nutzt die Tatsache, dass alle druckbaren Zeichen in Reihen organisiert sind: 0x2x-0x7x,
und wenn die beiden niederwertigsten Bits invertiert werden, wird der Buchstabe um eine oder drei Stellen nach
links oder rechts \IT{verschoben}, aber niemals in eine andere Reihe:

\begin{figure}[H]
\centering
\includegraphics[width=0.7\textwidth]{ascii_clean.png}
\caption{7-Bit \ac{ASCII} Tabelle in Emacs}
\end{figure}

\dots mit dem Zeichen 0x7F als einziger Ausnahme.

Im Folgenden werden also beispielsweise die Zeichen A-Z \IT{verschlüsselt}:

\begin{lstlisting}
#!/usr/bin/python

msg="@ABCDEFGHIJKLMNO"

print "".join(map(lambda x: chr(ord(x)^3), msg))
\end{lstlisting}

Ergebnis:
% FIXME \verb  --  relevant comment for German?
\begin{lstlisting}
CBA@GFEDKJIHONML
\end{lstlisting}

Es sieht so aus als würden die Zeichen ``@'' und ``C'' sowie ``B'' und ``A'' vertauscht werden.

Hier ist noch ein interessantes Beispiel, in dem gezeigt wird, wie die Eigenschaften von XOR
ausgenutzt werden können: Exakt den gleichen Effekt, dass druckbare Zeichen auch druckbar bleiben,
kann man dadurch erzielen, dass irgendeine Kombination der niedrigsten vier Bits invertiert wird.
}

\EN{\section{Returning Values}
\label{ret_val_func}

Another simple function is the one that simply returns a constant value:

\lstinputlisting[caption=\EN{\CCpp Code},style=customc]{patterns/011_ret/1.c}

Let's compile it.

\subsection{x86}

Here's what both the GCC and MSVC compilers produce (with optimization) on the x86 platform:

\lstinputlisting[caption=\Optimizing GCC/MSVC (\assemblyOutput),style=customasmx86]{patterns/011_ret/1.s}

\myindex{x86!\Instructions!RET}
There are just two instructions: the first places the value 123 into the \EAX register,
which is used by convention for storing the return
value, and the second one is \RET, which returns execution to the \gls{caller}.

The caller will take the result from the \EAX register.

\subsection{ARM}

There are a few differences on the ARM platform:

\lstinputlisting[caption=\OptimizingKeilVI (\ARMMode) ASM Output,style=customasmARM]{patterns/011_ret/1_Keil_ARM_O3.s}

ARM uses the register \Reg{0} for returning the results of functions, so 123 is copied into \Reg{0}.

\myindex{ARM!\Instructions!MOV}
\myindex{x86!\Instructions!MOV}
It is worth noting that \MOV is a misleading name for the instruction in both the x86 and ARM \ac{ISA}s.

The data is not in fact \IT{moved}, but \IT{copied}.

\subsection{MIPS}

\label{MIPS_leaf_function_ex1}

The GCC assembly output below lists registers by number:

\lstinputlisting[caption=\Optimizing GCC 4.4.5 (\assemblyOutput),style=customasmMIPS]{patterns/011_ret/MIPS.s}

\dots while \IDA does it by their pseudo names:

\lstinputlisting[caption=\Optimizing GCC 4.4.5 (IDA),style=customasmMIPS]{patterns/011_ret/MIPS_IDA.lst}

The \$2 (or \$V0) register is used to store the function's return value.
\myindex{MIPS!\Pseudoinstructions!LI}
\INS{LI} stands for ``Load Immediate'' and is the MIPS equivalent to \MOV.

\myindex{MIPS!\Instructions!J}
The other instruction is the jump instruction (J or JR) which returns the execution flow to the \gls{caller}.

\myindex{MIPS!Branch delay slot}
You might be wondering why the positions of the load instruction (LI) and the jump instruction (J or JR) are swapped. This is due to a \ac{RISC} feature called ``branch delay slot''.

The reason this happens is a quirk in the architecture of some RISC \ac{ISA}s and isn't important for our
purposes---we must simply keep in mind that in MIPS, the instruction following a jump or branch instruction
is executed \IT{before} the jump/branch instruction itself.

As a consequence, branch instructions always swap places with the instruction executed immediately beforehand.


In practice, functions which merely return 1 (\IT{true}) or 0 (\IT{false}) are very frequent.

The smallest ever of the standard UNIX utilities, \IT{/bin/true} and \IT{/bin/false} return 0 and 1 respectively, as an exit code.
(Zero as an exit code usually means success, non-zero means error.)
}
\RU{\subsubsection{std::string}
\myindex{\Cpp!STL!std::string}
\label{std_string}

\myparagraph{Как устроена структура}

Многие строковые библиотеки \InSqBrackets{\CNotes 2.2} обеспечивают структуру содержащую ссылку 
на буфер собственно со строкой, переменная всегда содержащую длину строки 
(что очень удобно для массы функций \InSqBrackets{\CNotes 2.2.1}) и переменную содержащую текущий размер буфера.

Строка в буфере обыкновенно оканчивается нулем: это для того чтобы указатель на буфер можно было
передавать в функции требующие на вход обычную сишную \ac{ASCIIZ}-строку.

Стандарт \Cpp не описывает, как именно нужно реализовывать std::string,
но, как правило, они реализованы как описано выше, с небольшими дополнениями.

Строки в \Cpp это не класс (как, например, QString в Qt), а темплейт (basic\_string), 
это сделано для того чтобы поддерживать 
строки содержащие разного типа символы: как минимум \Tchar и \IT{wchar\_t}.

Так что, std::string это класс с базовым типом \Tchar.

А std::wstring это класс с базовым типом \IT{wchar\_t}.

\mysubparagraph{MSVC}

В реализации MSVC, вместо ссылки на буфер может содержаться сам буфер (если строка короче 16-и символов).

Это означает, что каждая короткая строка будет занимать в памяти по крайней мере $16 + 4 + 4 = 24$ 
байт для 32-битной среды либо $16 + 8 + 8 = 32$ 
байта в 64-битной, а если строка длиннее 16-и символов, то прибавьте еще длину самой строки.

\lstinputlisting[caption=пример для MSVC,style=customc]{\CURPATH/STL/string/MSVC_RU.cpp}

Собственно, из этого исходника почти всё ясно.

Несколько замечаний:

Если строка короче 16-и символов, 
то отдельный буфер для строки в \glslink{heap}{куче} выделяться не будет.

Это удобно потому что на практике, основная часть строк действительно короткие.
Вероятно, разработчики в Microsoft выбрали размер в 16 символов как разумный баланс.

Теперь очень важный момент в конце функции main(): мы не пользуемся методом c\_str(), тем не менее,
если это скомпилировать и запустить, то обе строки появятся в консоли!

Работает это вот почему.

В первом случае строка короче 16-и символов и в начале объекта std::string (его можно рассматривать
просто как структуру) расположен буфер с этой строкой.
\printf трактует указатель как указатель на массив символов оканчивающийся нулем и поэтому всё работает.

Вывод второй строки (длиннее 16-и символов) даже еще опаснее: это вообще типичная программистская ошибка 
(или опечатка), забыть дописать c\_str().
Это работает потому что в это время в начале структуры расположен указатель на буфер.
Это может надолго остаться незамеченным: до тех пока там не появится строка 
короче 16-и символов, тогда процесс упадет.

\mysubparagraph{GCC}

В реализации GCC в структуре есть еще одна переменная --- reference count.

Интересно, что указатель на экземпляр класса std::string в GCC указывает не на начало самой структуры, 
а на указатель на буфера.
В libstdc++-v3\textbackslash{}include\textbackslash{}bits\textbackslash{}basic\_string.h 
мы можем прочитать что это сделано для удобства отладки:

\begin{lstlisting}
   *  The reason you want _M_data pointing to the character %array and
   *  not the _Rep is so that the debugger can see the string
   *  contents. (Probably we should add a non-inline member to get
   *  the _Rep for the debugger to use, so users can check the actual
   *  string length.)
\end{lstlisting}

\href{http://go.yurichev.com/17085}{исходный код basic\_string.h}

В нашем примере мы учитываем это:

\lstinputlisting[caption=пример для GCC,style=customc]{\CURPATH/STL/string/GCC_RU.cpp}

Нужны еще небольшие хаки чтобы сымитировать типичную ошибку, которую мы уже видели выше, из-за
более ужесточенной проверки типов в GCC, тем не менее, printf() работает и здесь без c\_str().

\myparagraph{Чуть более сложный пример}

\lstinputlisting[style=customc]{\CURPATH/STL/string/3.cpp}

\lstinputlisting[caption=MSVC 2012,style=customasmx86]{\CURPATH/STL/string/3_MSVC_RU.asm}

Собственно, компилятор не конструирует строки статически: да в общем-то и как
это возможно, если буфер с ней нужно хранить в \glslink{heap}{куче}?

Вместо этого в сегменте данных хранятся обычные \ac{ASCIIZ}-строки, а позже, во время выполнения, 
при помощи метода \q{assign}, конструируются строки s1 и s2
.
При помощи \TT{operator+}, создается строка s3.

Обратите внимание на то что вызов метода c\_str() отсутствует,
потому что его код достаточно короткий и компилятор вставил его прямо здесь:
если строка короче 16-и байт, то в регистре EAX остается указатель на буфер,
а если длиннее, то из этого же места достается адрес на буфер расположенный в \glslink{heap}{куче}.

Далее следуют вызовы трех деструкторов, причем, они вызываются только если строка длиннее 16-и байт:
тогда нужно освободить буфера в \glslink{heap}{куче}.
В противном случае, так как все три объекта std::string хранятся в стеке,
они освобождаются автоматически после выхода из функции.

Следовательно, работа с короткими строками более быстрая из-за м\'{е}ньшего обращения к \glslink{heap}{куче}.

Код на GCC даже проще (из-за того, что в GCC, как мы уже видели, не реализована возможность хранить короткую
строку прямо в структуре):

% TODO1 comment each function meaning
\lstinputlisting[caption=GCC 4.8.1,style=customasmx86]{\CURPATH/STL/string/3_GCC_RU.s}

Можно заметить, что в деструкторы передается не указатель на объект,
а указатель на место за 12 байт (или 3 слова) перед ним, то есть, на настоящее начало структуры.

\myparagraph{std::string как глобальная переменная}
\label{sec:std_string_as_global_variable}

Опытные программисты на \Cpp знают, что глобальные переменные \ac{STL}-типов вполне можно объявлять.

Да, действительно:

\lstinputlisting[style=customc]{\CURPATH/STL/string/5.cpp}

Но как и где будет вызываться конструктор \TT{std::string}?

На самом деле, эта переменная будет инициализирована даже перед началом \main.

\lstinputlisting[caption=MSVC 2012: здесь конструируется глобальная переменная{,} а также регистрируется её деструктор,style=customasmx86]{\CURPATH/STL/string/5_MSVC_p2.asm}

\lstinputlisting[caption=MSVC 2012: здесь глобальная переменная используется в \main,style=customasmx86]{\CURPATH/STL/string/5_MSVC_p1.asm}

\lstinputlisting[caption=MSVC 2012: эта функция-деструктор вызывается перед выходом,style=customasmx86]{\CURPATH/STL/string/5_MSVC_p3.asm}

\myindex{\CStandardLibrary!atexit()}
В реальности, из \ac{CRT}, еще до вызова main(), вызывается специальная функция,
в которой перечислены все конструкторы подобных переменных.
Более того: при помощи atexit() регистрируется функция, которая будет вызвана в конце работы программы:
в этой функции компилятор собирает вызовы деструкторов всех подобных глобальных переменных.

GCC работает похожим образом:

\lstinputlisting[caption=GCC 4.8.1,style=customasmx86]{\CURPATH/STL/string/5_GCC.s}

Но он не выделяет отдельной функции в которой будут собраны деструкторы: 
каждый деструктор передается в atexit() по одному.

% TODO а если глобальная STL-переменная в другом модуле? надо проверить.

}

\EN{\section{Returning Values}
\label{ret_val_func}

Another simple function is the one that simply returns a constant value:

\lstinputlisting[caption=\EN{\CCpp Code},style=customc]{patterns/011_ret/1.c}

Let's compile it.

\subsection{x86}

Here's what both the GCC and MSVC compilers produce (with optimization) on the x86 platform:

\lstinputlisting[caption=\Optimizing GCC/MSVC (\assemblyOutput),style=customasmx86]{patterns/011_ret/1.s}

\myindex{x86!\Instructions!RET}
There are just two instructions: the first places the value 123 into the \EAX register,
which is used by convention for storing the return
value, and the second one is \RET, which returns execution to the \gls{caller}.

The caller will take the result from the \EAX register.

\subsection{ARM}

There are a few differences on the ARM platform:

\lstinputlisting[caption=\OptimizingKeilVI (\ARMMode) ASM Output,style=customasmARM]{patterns/011_ret/1_Keil_ARM_O3.s}

ARM uses the register \Reg{0} for returning the results of functions, so 123 is copied into \Reg{0}.

\myindex{ARM!\Instructions!MOV}
\myindex{x86!\Instructions!MOV}
It is worth noting that \MOV is a misleading name for the instruction in both the x86 and ARM \ac{ISA}s.

The data is not in fact \IT{moved}, but \IT{copied}.

\subsection{MIPS}

\label{MIPS_leaf_function_ex1}

The GCC assembly output below lists registers by number:

\lstinputlisting[caption=\Optimizing GCC 4.4.5 (\assemblyOutput),style=customasmMIPS]{patterns/011_ret/MIPS.s}

\dots while \IDA does it by their pseudo names:

\lstinputlisting[caption=\Optimizing GCC 4.4.5 (IDA),style=customasmMIPS]{patterns/011_ret/MIPS_IDA.lst}

The \$2 (or \$V0) register is used to store the function's return value.
\myindex{MIPS!\Pseudoinstructions!LI}
\INS{LI} stands for ``Load Immediate'' and is the MIPS equivalent to \MOV.

\myindex{MIPS!\Instructions!J}
The other instruction is the jump instruction (J or JR) which returns the execution flow to the \gls{caller}.

\myindex{MIPS!Branch delay slot}
You might be wondering why the positions of the load instruction (LI) and the jump instruction (J or JR) are swapped. This is due to a \ac{RISC} feature called ``branch delay slot''.

The reason this happens is a quirk in the architecture of some RISC \ac{ISA}s and isn't important for our
purposes---we must simply keep in mind that in MIPS, the instruction following a jump or branch instruction
is executed \IT{before} the jump/branch instruction itself.

As a consequence, branch instructions always swap places with the instruction executed immediately beforehand.


In practice, functions which merely return 1 (\IT{true}) or 0 (\IT{false}) are very frequent.

The smallest ever of the standard UNIX utilities, \IT{/bin/true} and \IT{/bin/false} return 0 and 1 respectively, as an exit code.
(Zero as an exit code usually means success, non-zero means error.)
}
\RU{\subsubsection{std::string}
\myindex{\Cpp!STL!std::string}
\label{std_string}

\myparagraph{Как устроена структура}

Многие строковые библиотеки \InSqBrackets{\CNotes 2.2} обеспечивают структуру содержащую ссылку 
на буфер собственно со строкой, переменная всегда содержащую длину строки 
(что очень удобно для массы функций \InSqBrackets{\CNotes 2.2.1}) и переменную содержащую текущий размер буфера.

Строка в буфере обыкновенно оканчивается нулем: это для того чтобы указатель на буфер можно было
передавать в функции требующие на вход обычную сишную \ac{ASCIIZ}-строку.

Стандарт \Cpp не описывает, как именно нужно реализовывать std::string,
но, как правило, они реализованы как описано выше, с небольшими дополнениями.

Строки в \Cpp это не класс (как, например, QString в Qt), а темплейт (basic\_string), 
это сделано для того чтобы поддерживать 
строки содержащие разного типа символы: как минимум \Tchar и \IT{wchar\_t}.

Так что, std::string это класс с базовым типом \Tchar.

А std::wstring это класс с базовым типом \IT{wchar\_t}.

\mysubparagraph{MSVC}

В реализации MSVC, вместо ссылки на буфер может содержаться сам буфер (если строка короче 16-и символов).

Это означает, что каждая короткая строка будет занимать в памяти по крайней мере $16 + 4 + 4 = 24$ 
байт для 32-битной среды либо $16 + 8 + 8 = 32$ 
байта в 64-битной, а если строка длиннее 16-и символов, то прибавьте еще длину самой строки.

\lstinputlisting[caption=пример для MSVC,style=customc]{\CURPATH/STL/string/MSVC_RU.cpp}

Собственно, из этого исходника почти всё ясно.

Несколько замечаний:

Если строка короче 16-и символов, 
то отдельный буфер для строки в \glslink{heap}{куче} выделяться не будет.

Это удобно потому что на практике, основная часть строк действительно короткие.
Вероятно, разработчики в Microsoft выбрали размер в 16 символов как разумный баланс.

Теперь очень важный момент в конце функции main(): мы не пользуемся методом c\_str(), тем не менее,
если это скомпилировать и запустить, то обе строки появятся в консоли!

Работает это вот почему.

В первом случае строка короче 16-и символов и в начале объекта std::string (его можно рассматривать
просто как структуру) расположен буфер с этой строкой.
\printf трактует указатель как указатель на массив символов оканчивающийся нулем и поэтому всё работает.

Вывод второй строки (длиннее 16-и символов) даже еще опаснее: это вообще типичная программистская ошибка 
(или опечатка), забыть дописать c\_str().
Это работает потому что в это время в начале структуры расположен указатель на буфер.
Это может надолго остаться незамеченным: до тех пока там не появится строка 
короче 16-и символов, тогда процесс упадет.

\mysubparagraph{GCC}

В реализации GCC в структуре есть еще одна переменная --- reference count.

Интересно, что указатель на экземпляр класса std::string в GCC указывает не на начало самой структуры, 
а на указатель на буфера.
В libstdc++-v3\textbackslash{}include\textbackslash{}bits\textbackslash{}basic\_string.h 
мы можем прочитать что это сделано для удобства отладки:

\begin{lstlisting}
   *  The reason you want _M_data pointing to the character %array and
   *  not the _Rep is so that the debugger can see the string
   *  contents. (Probably we should add a non-inline member to get
   *  the _Rep for the debugger to use, so users can check the actual
   *  string length.)
\end{lstlisting}

\href{http://go.yurichev.com/17085}{исходный код basic\_string.h}

В нашем примере мы учитываем это:

\lstinputlisting[caption=пример для GCC,style=customc]{\CURPATH/STL/string/GCC_RU.cpp}

Нужны еще небольшие хаки чтобы сымитировать типичную ошибку, которую мы уже видели выше, из-за
более ужесточенной проверки типов в GCC, тем не менее, printf() работает и здесь без c\_str().

\myparagraph{Чуть более сложный пример}

\lstinputlisting[style=customc]{\CURPATH/STL/string/3.cpp}

\lstinputlisting[caption=MSVC 2012,style=customasmx86]{\CURPATH/STL/string/3_MSVC_RU.asm}

Собственно, компилятор не конструирует строки статически: да в общем-то и как
это возможно, если буфер с ней нужно хранить в \glslink{heap}{куче}?

Вместо этого в сегменте данных хранятся обычные \ac{ASCIIZ}-строки, а позже, во время выполнения, 
при помощи метода \q{assign}, конструируются строки s1 и s2
.
При помощи \TT{operator+}, создается строка s3.

Обратите внимание на то что вызов метода c\_str() отсутствует,
потому что его код достаточно короткий и компилятор вставил его прямо здесь:
если строка короче 16-и байт, то в регистре EAX остается указатель на буфер,
а если длиннее, то из этого же места достается адрес на буфер расположенный в \glslink{heap}{куче}.

Далее следуют вызовы трех деструкторов, причем, они вызываются только если строка длиннее 16-и байт:
тогда нужно освободить буфера в \glslink{heap}{куче}.
В противном случае, так как все три объекта std::string хранятся в стеке,
они освобождаются автоматически после выхода из функции.

Следовательно, работа с короткими строками более быстрая из-за м\'{е}ньшего обращения к \glslink{heap}{куче}.

Код на GCC даже проще (из-за того, что в GCC, как мы уже видели, не реализована возможность хранить короткую
строку прямо в структуре):

% TODO1 comment each function meaning
\lstinputlisting[caption=GCC 4.8.1,style=customasmx86]{\CURPATH/STL/string/3_GCC_RU.s}

Можно заметить, что в деструкторы передается не указатель на объект,
а указатель на место за 12 байт (или 3 слова) перед ним, то есть, на настоящее начало структуры.

\myparagraph{std::string как глобальная переменная}
\label{sec:std_string_as_global_variable}

Опытные программисты на \Cpp знают, что глобальные переменные \ac{STL}-типов вполне можно объявлять.

Да, действительно:

\lstinputlisting[style=customc]{\CURPATH/STL/string/5.cpp}

Но как и где будет вызываться конструктор \TT{std::string}?

На самом деле, эта переменная будет инициализирована даже перед началом \main.

\lstinputlisting[caption=MSVC 2012: здесь конструируется глобальная переменная{,} а также регистрируется её деструктор,style=customasmx86]{\CURPATH/STL/string/5_MSVC_p2.asm}

\lstinputlisting[caption=MSVC 2012: здесь глобальная переменная используется в \main,style=customasmx86]{\CURPATH/STL/string/5_MSVC_p1.asm}

\lstinputlisting[caption=MSVC 2012: эта функция-деструктор вызывается перед выходом,style=customasmx86]{\CURPATH/STL/string/5_MSVC_p3.asm}

\myindex{\CStandardLibrary!atexit()}
В реальности, из \ac{CRT}, еще до вызова main(), вызывается специальная функция,
в которой перечислены все конструкторы подобных переменных.
Более того: при помощи atexit() регистрируется функция, которая будет вызвана в конце работы программы:
в этой функции компилятор собирает вызовы деструкторов всех подобных глобальных переменных.

GCC работает похожим образом:

\lstinputlisting[caption=GCC 4.8.1,style=customasmx86]{\CURPATH/STL/string/5_GCC.s}

Но он не выделяет отдельной функции в которой будут собраны деструкторы: 
каждый деструктор передается в atexit() по одному.

% TODO а если глобальная STL-переменная в другом модуле? надо проверить.

}
\DE{\subsection{Einfachste XOR-Verschlüsselung überhaupt}

Ich habe einmal eine Software gesehen, bei der alle Debugging-Ausgaben mit XOR mit dem Wert 3
verschlüsselt wurden. Mit anderen Worten, die beiden niedrigsten Bits aller Buchstaben wurden invertiert.

``Hello, world'' wurde zu ``Kfool/\#tlqog'':

\begin{lstlisting}
#!/usr/bin/python

msg="Hello, world!"

print "".join(map(lambda x: chr(ord(x)^3), msg))
\end{lstlisting}

Das ist eine ziemlich interessante Verschlüsselung (oder besser eine Verschleierung),
weil sie zwei wichtige Eigenschaften hat:
1) es ist eine einzige Funktion zum Verschlüsseln und entschlüsseln, sie muss nur wiederholt angewendet werden
2) die entstehenden Buchstaben befinden sich im druckbaren Bereich, also die ganze Zeichenkette kann ohne
Escape-Symbole im Code verwendet werden.

Die zweite Eigenschaft nutzt die Tatsache, dass alle druckbaren Zeichen in Reihen organisiert sind: 0x2x-0x7x,
und wenn die beiden niederwertigsten Bits invertiert werden, wird der Buchstabe um eine oder drei Stellen nach
links oder rechts \IT{verschoben}, aber niemals in eine andere Reihe:

\begin{figure}[H]
\centering
\includegraphics[width=0.7\textwidth]{ascii_clean.png}
\caption{7-Bit \ac{ASCII} Tabelle in Emacs}
\end{figure}

\dots mit dem Zeichen 0x7F als einziger Ausnahme.

Im Folgenden werden also beispielsweise die Zeichen A-Z \IT{verschlüsselt}:

\begin{lstlisting}
#!/usr/bin/python

msg="@ABCDEFGHIJKLMNO"

print "".join(map(lambda x: chr(ord(x)^3), msg))
\end{lstlisting}

Ergebnis:
% FIXME \verb  --  relevant comment for German?
\begin{lstlisting}
CBA@GFEDKJIHONML
\end{lstlisting}

Es sieht so aus als würden die Zeichen ``@'' und ``C'' sowie ``B'' und ``A'' vertauscht werden.

Hier ist noch ein interessantes Beispiel, in dem gezeigt wird, wie die Eigenschaften von XOR
ausgenutzt werden können: Exakt den gleichen Effekt, dass druckbare Zeichen auch druckbar bleiben,
kann man dadurch erzielen, dass irgendeine Kombination der niedrigsten vier Bits invertiert wird.
}

\EN{\section{Returning Values}
\label{ret_val_func}

Another simple function is the one that simply returns a constant value:

\lstinputlisting[caption=\EN{\CCpp Code},style=customc]{patterns/011_ret/1.c}

Let's compile it.

\subsection{x86}

Here's what both the GCC and MSVC compilers produce (with optimization) on the x86 platform:

\lstinputlisting[caption=\Optimizing GCC/MSVC (\assemblyOutput),style=customasmx86]{patterns/011_ret/1.s}

\myindex{x86!\Instructions!RET}
There are just two instructions: the first places the value 123 into the \EAX register,
which is used by convention for storing the return
value, and the second one is \RET, which returns execution to the \gls{caller}.

The caller will take the result from the \EAX register.

\subsection{ARM}

There are a few differences on the ARM platform:

\lstinputlisting[caption=\OptimizingKeilVI (\ARMMode) ASM Output,style=customasmARM]{patterns/011_ret/1_Keil_ARM_O3.s}

ARM uses the register \Reg{0} for returning the results of functions, so 123 is copied into \Reg{0}.

\myindex{ARM!\Instructions!MOV}
\myindex{x86!\Instructions!MOV}
It is worth noting that \MOV is a misleading name for the instruction in both the x86 and ARM \ac{ISA}s.

The data is not in fact \IT{moved}, but \IT{copied}.

\subsection{MIPS}

\label{MIPS_leaf_function_ex1}

The GCC assembly output below lists registers by number:

\lstinputlisting[caption=\Optimizing GCC 4.4.5 (\assemblyOutput),style=customasmMIPS]{patterns/011_ret/MIPS.s}

\dots while \IDA does it by their pseudo names:

\lstinputlisting[caption=\Optimizing GCC 4.4.5 (IDA),style=customasmMIPS]{patterns/011_ret/MIPS_IDA.lst}

The \$2 (or \$V0) register is used to store the function's return value.
\myindex{MIPS!\Pseudoinstructions!LI}
\INS{LI} stands for ``Load Immediate'' and is the MIPS equivalent to \MOV.

\myindex{MIPS!\Instructions!J}
The other instruction is the jump instruction (J or JR) which returns the execution flow to the \gls{caller}.

\myindex{MIPS!Branch delay slot}
You might be wondering why the positions of the load instruction (LI) and the jump instruction (J or JR) are swapped. This is due to a \ac{RISC} feature called ``branch delay slot''.

The reason this happens is a quirk in the architecture of some RISC \ac{ISA}s and isn't important for our
purposes---we must simply keep in mind that in MIPS, the instruction following a jump or branch instruction
is executed \IT{before} the jump/branch instruction itself.

As a consequence, branch instructions always swap places with the instruction executed immediately beforehand.


In practice, functions which merely return 1 (\IT{true}) or 0 (\IT{false}) are very frequent.

The smallest ever of the standard UNIX utilities, \IT{/bin/true} and \IT{/bin/false} return 0 and 1 respectively, as an exit code.
(Zero as an exit code usually means success, non-zero means error.)
}
\RU{\subsubsection{std::string}
\myindex{\Cpp!STL!std::string}
\label{std_string}

\myparagraph{Как устроена структура}

Многие строковые библиотеки \InSqBrackets{\CNotes 2.2} обеспечивают структуру содержащую ссылку 
на буфер собственно со строкой, переменная всегда содержащую длину строки 
(что очень удобно для массы функций \InSqBrackets{\CNotes 2.2.1}) и переменную содержащую текущий размер буфера.

Строка в буфере обыкновенно оканчивается нулем: это для того чтобы указатель на буфер можно было
передавать в функции требующие на вход обычную сишную \ac{ASCIIZ}-строку.

Стандарт \Cpp не описывает, как именно нужно реализовывать std::string,
но, как правило, они реализованы как описано выше, с небольшими дополнениями.

Строки в \Cpp это не класс (как, например, QString в Qt), а темплейт (basic\_string), 
это сделано для того чтобы поддерживать 
строки содержащие разного типа символы: как минимум \Tchar и \IT{wchar\_t}.

Так что, std::string это класс с базовым типом \Tchar.

А std::wstring это класс с базовым типом \IT{wchar\_t}.

\mysubparagraph{MSVC}

В реализации MSVC, вместо ссылки на буфер может содержаться сам буфер (если строка короче 16-и символов).

Это означает, что каждая короткая строка будет занимать в памяти по крайней мере $16 + 4 + 4 = 24$ 
байт для 32-битной среды либо $16 + 8 + 8 = 32$ 
байта в 64-битной, а если строка длиннее 16-и символов, то прибавьте еще длину самой строки.

\lstinputlisting[caption=пример для MSVC,style=customc]{\CURPATH/STL/string/MSVC_RU.cpp}

Собственно, из этого исходника почти всё ясно.

Несколько замечаний:

Если строка короче 16-и символов, 
то отдельный буфер для строки в \glslink{heap}{куче} выделяться не будет.

Это удобно потому что на практике, основная часть строк действительно короткие.
Вероятно, разработчики в Microsoft выбрали размер в 16 символов как разумный баланс.

Теперь очень важный момент в конце функции main(): мы не пользуемся методом c\_str(), тем не менее,
если это скомпилировать и запустить, то обе строки появятся в консоли!

Работает это вот почему.

В первом случае строка короче 16-и символов и в начале объекта std::string (его можно рассматривать
просто как структуру) расположен буфер с этой строкой.
\printf трактует указатель как указатель на массив символов оканчивающийся нулем и поэтому всё работает.

Вывод второй строки (длиннее 16-и символов) даже еще опаснее: это вообще типичная программистская ошибка 
(или опечатка), забыть дописать c\_str().
Это работает потому что в это время в начале структуры расположен указатель на буфер.
Это может надолго остаться незамеченным: до тех пока там не появится строка 
короче 16-и символов, тогда процесс упадет.

\mysubparagraph{GCC}

В реализации GCC в структуре есть еще одна переменная --- reference count.

Интересно, что указатель на экземпляр класса std::string в GCC указывает не на начало самой структуры, 
а на указатель на буфера.
В libstdc++-v3\textbackslash{}include\textbackslash{}bits\textbackslash{}basic\_string.h 
мы можем прочитать что это сделано для удобства отладки:

\begin{lstlisting}
   *  The reason you want _M_data pointing to the character %array and
   *  not the _Rep is so that the debugger can see the string
   *  contents. (Probably we should add a non-inline member to get
   *  the _Rep for the debugger to use, so users can check the actual
   *  string length.)
\end{lstlisting}

\href{http://go.yurichev.com/17085}{исходный код basic\_string.h}

В нашем примере мы учитываем это:

\lstinputlisting[caption=пример для GCC,style=customc]{\CURPATH/STL/string/GCC_RU.cpp}

Нужны еще небольшие хаки чтобы сымитировать типичную ошибку, которую мы уже видели выше, из-за
более ужесточенной проверки типов в GCC, тем не менее, printf() работает и здесь без c\_str().

\myparagraph{Чуть более сложный пример}

\lstinputlisting[style=customc]{\CURPATH/STL/string/3.cpp}

\lstinputlisting[caption=MSVC 2012,style=customasmx86]{\CURPATH/STL/string/3_MSVC_RU.asm}

Собственно, компилятор не конструирует строки статически: да в общем-то и как
это возможно, если буфер с ней нужно хранить в \glslink{heap}{куче}?

Вместо этого в сегменте данных хранятся обычные \ac{ASCIIZ}-строки, а позже, во время выполнения, 
при помощи метода \q{assign}, конструируются строки s1 и s2
.
При помощи \TT{operator+}, создается строка s3.

Обратите внимание на то что вызов метода c\_str() отсутствует,
потому что его код достаточно короткий и компилятор вставил его прямо здесь:
если строка короче 16-и байт, то в регистре EAX остается указатель на буфер,
а если длиннее, то из этого же места достается адрес на буфер расположенный в \glslink{heap}{куче}.

Далее следуют вызовы трех деструкторов, причем, они вызываются только если строка длиннее 16-и байт:
тогда нужно освободить буфера в \glslink{heap}{куче}.
В противном случае, так как все три объекта std::string хранятся в стеке,
они освобождаются автоматически после выхода из функции.

Следовательно, работа с короткими строками более быстрая из-за м\'{е}ньшего обращения к \glslink{heap}{куче}.

Код на GCC даже проще (из-за того, что в GCC, как мы уже видели, не реализована возможность хранить короткую
строку прямо в структуре):

% TODO1 comment each function meaning
\lstinputlisting[caption=GCC 4.8.1,style=customasmx86]{\CURPATH/STL/string/3_GCC_RU.s}

Можно заметить, что в деструкторы передается не указатель на объект,
а указатель на место за 12 байт (или 3 слова) перед ним, то есть, на настоящее начало структуры.

\myparagraph{std::string как глобальная переменная}
\label{sec:std_string_as_global_variable}

Опытные программисты на \Cpp знают, что глобальные переменные \ac{STL}-типов вполне можно объявлять.

Да, действительно:

\lstinputlisting[style=customc]{\CURPATH/STL/string/5.cpp}

Но как и где будет вызываться конструктор \TT{std::string}?

На самом деле, эта переменная будет инициализирована даже перед началом \main.

\lstinputlisting[caption=MSVC 2012: здесь конструируется глобальная переменная{,} а также регистрируется её деструктор,style=customasmx86]{\CURPATH/STL/string/5_MSVC_p2.asm}

\lstinputlisting[caption=MSVC 2012: здесь глобальная переменная используется в \main,style=customasmx86]{\CURPATH/STL/string/5_MSVC_p1.asm}

\lstinputlisting[caption=MSVC 2012: эта функция-деструктор вызывается перед выходом,style=customasmx86]{\CURPATH/STL/string/5_MSVC_p3.asm}

\myindex{\CStandardLibrary!atexit()}
В реальности, из \ac{CRT}, еще до вызова main(), вызывается специальная функция,
в которой перечислены все конструкторы подобных переменных.
Более того: при помощи atexit() регистрируется функция, которая будет вызвана в конце работы программы:
в этой функции компилятор собирает вызовы деструкторов всех подобных глобальных переменных.

GCC работает похожим образом:

\lstinputlisting[caption=GCC 4.8.1,style=customasmx86]{\CURPATH/STL/string/5_GCC.s}

Но он не выделяет отдельной функции в которой будут собраны деструкторы: 
каждый деструктор передается в atexit() по одному.

% TODO а если глобальная STL-переменная в другом модуле? надо проверить.

}
\DE{\subsection{Einfachste XOR-Verschlüsselung überhaupt}

Ich habe einmal eine Software gesehen, bei der alle Debugging-Ausgaben mit XOR mit dem Wert 3
verschlüsselt wurden. Mit anderen Worten, die beiden niedrigsten Bits aller Buchstaben wurden invertiert.

``Hello, world'' wurde zu ``Kfool/\#tlqog'':

\begin{lstlisting}
#!/usr/bin/python

msg="Hello, world!"

print "".join(map(lambda x: chr(ord(x)^3), msg))
\end{lstlisting}

Das ist eine ziemlich interessante Verschlüsselung (oder besser eine Verschleierung),
weil sie zwei wichtige Eigenschaften hat:
1) es ist eine einzige Funktion zum Verschlüsseln und entschlüsseln, sie muss nur wiederholt angewendet werden
2) die entstehenden Buchstaben befinden sich im druckbaren Bereich, also die ganze Zeichenkette kann ohne
Escape-Symbole im Code verwendet werden.

Die zweite Eigenschaft nutzt die Tatsache, dass alle druckbaren Zeichen in Reihen organisiert sind: 0x2x-0x7x,
und wenn die beiden niederwertigsten Bits invertiert werden, wird der Buchstabe um eine oder drei Stellen nach
links oder rechts \IT{verschoben}, aber niemals in eine andere Reihe:

\begin{figure}[H]
\centering
\includegraphics[width=0.7\textwidth]{ascii_clean.png}
\caption{7-Bit \ac{ASCII} Tabelle in Emacs}
\end{figure}

\dots mit dem Zeichen 0x7F als einziger Ausnahme.

Im Folgenden werden also beispielsweise die Zeichen A-Z \IT{verschlüsselt}:

\begin{lstlisting}
#!/usr/bin/python

msg="@ABCDEFGHIJKLMNO"

print "".join(map(lambda x: chr(ord(x)^3), msg))
\end{lstlisting}

Ergebnis:
% FIXME \verb  --  relevant comment for German?
\begin{lstlisting}
CBA@GFEDKJIHONML
\end{lstlisting}

Es sieht so aus als würden die Zeichen ``@'' und ``C'' sowie ``B'' und ``A'' vertauscht werden.

Hier ist noch ein interessantes Beispiel, in dem gezeigt wird, wie die Eigenschaften von XOR
ausgenutzt werden können: Exakt den gleichen Effekt, dass druckbare Zeichen auch druckbar bleiben,
kann man dadurch erzielen, dass irgendeine Kombination der niedrigsten vier Bits invertiert wird.
}

\ifdefined\SPANISH
\chapter{Patrones de código}
\fi % SPANISH

\ifdefined\GERMAN
\chapter{Code-Muster}
\fi % GERMAN

\ifdefined\ENGLISH
\chapter{Code Patterns}
\fi % ENGLISH

\ifdefined\ITALIAN
\chapter{Forme di codice}
\fi % ITALIAN

\ifdefined\RUSSIAN
\chapter{Образцы кода}
\fi % RUSSIAN

\ifdefined\BRAZILIAN
\chapter{Padrões de códigos}
\fi % BRAZILIAN

\ifdefined\THAI
\chapter{รูปแบบของโค้ด}
\fi % THAI

\ifdefined\FRENCH
\chapter{Modèle de code}
\fi % FRENCH

\ifdefined\POLISH
\chapter{\PLph{}}
\fi % POLISH

% sections
\EN{\input{patterns/patterns_opt_dbg_EN}}
\ES{\input{patterns/patterns_opt_dbg_ES}}
\ITA{\input{patterns/patterns_opt_dbg_ITA}}
\PTBR{\input{patterns/patterns_opt_dbg_PTBR}}
\RU{\input{patterns/patterns_opt_dbg_RU}}
\THA{\input{patterns/patterns_opt_dbg_THA}}
\DE{\input{patterns/patterns_opt_dbg_DE}}
\FR{\input{patterns/patterns_opt_dbg_FR}}
\PL{\input{patterns/patterns_opt_dbg_PL}}

\RU{\section{Некоторые базовые понятия}}
\EN{\section{Some basics}}
\DE{\section{Einige Grundlagen}}
\FR{\section{Quelques bases}}
\ES{\section{\ESph{}}}
\ITA{\section{Alcune basi teoriche}}
\PTBR{\section{\PTBRph{}}}
\THA{\section{\THAph{}}}
\PL{\section{\PLph{}}}

% sections:
\EN{\input{patterns/intro_CPU_ISA_EN}}
\ES{\input{patterns/intro_CPU_ISA_ES}}
\ITA{\input{patterns/intro_CPU_ISA_ITA}}
\PTBR{\input{patterns/intro_CPU_ISA_PTBR}}
\RU{\input{patterns/intro_CPU_ISA_RU}}
\DE{\input{patterns/intro_CPU_ISA_DE}}
\FR{\input{patterns/intro_CPU_ISA_FR}}
\PL{\input{patterns/intro_CPU_ISA_PL}}

\EN{\input{patterns/numeral_EN}}
\RU{\input{patterns/numeral_RU}}
\ITA{\input{patterns/numeral_ITA}}
\DE{\input{patterns/numeral_DE}}
\FR{\input{patterns/numeral_FR}}
\PL{\input{patterns/numeral_PL}}

% chapters
\input{patterns/00_empty/main}
\input{patterns/011_ret/main}
\input{patterns/01_helloworld/main}
\input{patterns/015_prolog_epilogue/main}
\input{patterns/02_stack/main}
\input{patterns/03_printf/main}
\input{patterns/04_scanf/main}
\input{patterns/05_passing_arguments/main}
\input{patterns/06_return_results/main}
\input{patterns/061_pointers/main}
\input{patterns/065_GOTO/main}
\input{patterns/07_jcc/main}
\input{patterns/08_switch/main}
\input{patterns/09_loops/main}
\input{patterns/10_strings/main}
\input{patterns/11_arith_optimizations/main}
\input{patterns/12_FPU/main}
\input{patterns/13_arrays/main}
\input{patterns/14_bitfields/main}
\EN{\input{patterns/145_LCG/main_EN}}
\RU{\input{patterns/145_LCG/main_RU}}
\input{patterns/15_structs/main}
\input{patterns/17_unions/main}
\input{patterns/18_pointers_to_functions/main}
\input{patterns/185_64bit_in_32_env/main}

\EN{\input{patterns/19_SIMD/main_EN}}
\RU{\input{patterns/19_SIMD/main_RU}}
\DE{\input{patterns/19_SIMD/main_DE}}

\EN{\input{patterns/20_x64/main_EN}}
\RU{\input{patterns/20_x64/main_RU}}

\EN{\input{patterns/205_floating_SIMD/main_EN}}
\RU{\input{patterns/205_floating_SIMD/main_RU}}
\DE{\input{patterns/205_floating_SIMD/main_DE}}

\EN{\input{patterns/ARM/main_EN}}
\RU{\input{patterns/ARM/main_RU}}
\DE{\input{patterns/ARM/main_DE}}

\input{patterns/MIPS/main}


\ifdefined\SPANISH
\chapter{Patrones de código}
\fi % SPANISH

\ifdefined\GERMAN
\chapter{Code-Muster}
\fi % GERMAN

\ifdefined\ENGLISH
\chapter{Code Patterns}
\fi % ENGLISH

\ifdefined\ITALIAN
\chapter{Forme di codice}
\fi % ITALIAN

\ifdefined\RUSSIAN
\chapter{Образцы кода}
\fi % RUSSIAN

\ifdefined\BRAZILIAN
\chapter{Padrões de códigos}
\fi % BRAZILIAN

\ifdefined\THAI
\chapter{รูปแบบของโค้ด}
\fi % THAI

\ifdefined\FRENCH
\chapter{Modèle de code}
\fi % FRENCH

\ifdefined\POLISH
\chapter{\PLph{}}
\fi % POLISH

% sections
\EN{\section{The method}

When the author of this book first started learning C and, later, \Cpp, he used to write small pieces of code, compile them,
and then look at the assembly language output. This made it very easy for him to understand what was going on in the code that he had written.
\footnote{In fact, he still does this when he can't understand what a particular bit of code does.}.
He did this so many times that the relationship between the \CCpp code and what the compiler produced was imprinted deeply in his mind.
It's now easy for him to imagine instantly a rough outline of a C code's appearance and function.
Perhaps this technique could be helpful for others.

%There are a lot of examples for both x86/x64 and ARM.
%Those who already familiar with one of architectures, may freely skim over pages.

By the way, there is a great website where you can do the same, with various compilers, instead of installing them on your box.
You can use it as well: \url{https://gcc.godbolt.org/}.

\section*{\Exercises}

When the author of this book studied assembly language, he also often compiled small C functions and then rewrote
them gradually to assembly, trying to make their code as short as possible.
This probably is not worth doing in real-world scenarios today,
because it's hard to compete with the latest compilers in terms of efficiency. It is, however, a very good way to gain a better understanding of assembly.
Feel free, therefore, to take any assembly code from this book and try to make it shorter.
However, don't forget to test what you have written.

% rewrote to show that debug\release and optimisations levels are orthogonal concepts.
\section*{Optimization levels and debug information}

Source code can be compiled by different compilers with various optimization levels.
A typical compiler has about three such levels, where level zero means that optimization is completely disabled.
Optimization can also be targeted towards code size or code speed.
A non-optimizing compiler is faster and produces more understandable (albeit verbose) code,
whereas an optimizing compiler is slower and tries to produce code that runs faster (but is not necessarily more compact).
In addition to optimization levels, a compiler can include some debug information in the resulting file,
producing code that is easy to debug.
One of the important features of the ´debug' code is that it might contain links
between each line of the source code and its respective machine code address.
Optimizing compilers, on the other hand, tend to produce output where entire lines of source code
can be optimized away and thus not even be present in the resulting machine code.
Reverse engineers can encounter either version, simply because some developers turn on the compiler's optimization flags and others do not.
Because of this, we'll try to work on examples of both debug and release versions of the code featured in this book, wherever possible.

Sometimes some pretty ancient compilers are used in this book, in order to get the shortest (or simplest) possible code snippet.
}
\ES{\input{patterns/patterns_opt_dbg_ES}}
\ITA{\input{patterns/patterns_opt_dbg_ITA}}
\PTBR{\input{patterns/patterns_opt_dbg_PTBR}}
\RU{\input{patterns/patterns_opt_dbg_RU}}
\THA{\input{patterns/patterns_opt_dbg_THA}}
\DE{\input{patterns/patterns_opt_dbg_DE}}
\FR{\input{patterns/patterns_opt_dbg_FR}}
\PL{\input{patterns/patterns_opt_dbg_PL}}

\RU{\section{Некоторые базовые понятия}}
\EN{\section{Some basics}}
\DE{\section{Einige Grundlagen}}
\FR{\section{Quelques bases}}
\ES{\section{\ESph{}}}
\ITA{\section{Alcune basi teoriche}}
\PTBR{\section{\PTBRph{}}}
\THA{\section{\THAph{}}}
\PL{\section{\PLph{}}}

% sections:
\EN{\input{patterns/intro_CPU_ISA_EN}}
\ES{\input{patterns/intro_CPU_ISA_ES}}
\ITA{\input{patterns/intro_CPU_ISA_ITA}}
\PTBR{\input{patterns/intro_CPU_ISA_PTBR}}
\RU{\input{patterns/intro_CPU_ISA_RU}}
\DE{\input{patterns/intro_CPU_ISA_DE}}
\FR{\input{patterns/intro_CPU_ISA_FR}}
\PL{\input{patterns/intro_CPU_ISA_PL}}

\EN{\subsection{Numeral Systems}

Humans have become accustomed to a decimal numeral system, probably because almost everyone has 10 fingers.
Nevertheless, the number \q{10} has no significant meaning in science and mathematics.
The natural numeral system in digital electronics is binary: 0 is for an absence of current in the wire, and 1 for presence.
10 in binary is 2 in decimal, 100 in binary is 4 in decimal, and so on.

% This sentence is a bit unweildy - maybe try 'Our ten-digit system would be described as having a radix...' - Renaissance
If the numeral system has 10 digits, it has a \IT{radix} (or \IT{base}) of 10.
The binary numeral system has a \IT{radix} of 2.

Important things to recall:

1) A \IT{number} is a number, while a \IT{digit} is a term from writing systems, and is usually one character

% The original is 'number' is not changed; I think the intent is value, and changed it - Renaissance
2) The value of a number does not change when converted to another radix; only the writing notation for that value has changed (and therefore the way of representing it in \ac{RAM}).

\subsection{Converting From One Radix To Another}

Positional notation is used almost every numerical system. This means that a digit has weight relative to where it is placed inside of the larger number.
If 2 is placed at the rightmost place, it's 2, but if it's placed one digit before rightmost, it's 20.

What does $1234$ stand for?

$10^3 \cdot 1 + 10^2 \cdot 2 + 10^1 \cdot 3 + 1 \cdot 4 = 1234$ or
$1000 \cdot 1 + 100 \cdot 2 + 10 \cdot 3 + 4 = 1234$

It's the same story for binary numbers, but the base is 2 instead of 10.
What does 0b101011 stand for?

$2^5 \cdot 1 + 2^4 \cdot 0 + 2^3 \cdot 1 + 2^2 \cdot 0 + 2^1 \cdot 1 + 2^0 \cdot 1 = 43$ or
$32 \cdot 1 + 16 \cdot 0 + 8 \cdot 1 + 4 \cdot 0 + 2 \cdot 1 + 1 = 43$

There is such a thing as non-positional notation, such as the Roman numeral system.
\footnote{About numeric system evolution, see \InSqBrackets{\TAOCPvolII{}, 195--213.}}.
% Maybe add a sentence to fill in that X is always 10, and is therefore non-positional, even though putting an I before subtracts and after adds, and is in that sense positional
Perhaps, humankind switched to positional notation because it's easier to do basic operations (addition, multiplication, etc.) on paper by hand.

Binary numbers can be added, subtracted and so on in the very same as taught in schools, but only 2 digits are available.

Binary numbers are bulky when represented in source code and dumps, so that is where the hexadecimal numeral system can be useful.
A hexadecimal radix uses the digits 0..9, and also 6 Latin characters: A..F.
Each hexadecimal digit takes 4 bits or 4 binary digits, so it's very easy to convert from binary number to hexadecimal and back, even manually, in one's mind.

\begin{center}
\begin{longtable}{ | l | l | l | }
\hline
\HeaderColor hexadecimal & \HeaderColor binary & \HeaderColor decimal \\
\hline
0	&0000	&0 \\
1	&0001	&1 \\
2	&0010	&2 \\
3	&0011	&3 \\
4	&0100	&4 \\
5	&0101	&5 \\
6	&0110	&6 \\
7	&0111	&7 \\
8	&1000	&8 \\
9	&1001	&9 \\
A	&1010	&10 \\
B	&1011	&11 \\
C	&1100	&12 \\
D	&1101	&13 \\
E	&1110	&14 \\
F	&1111	&15 \\
\hline
\end{longtable}
\end{center}

How can one tell which radix is being used in a specific instance?

Decimal numbers are usually written as is, i.e., 1234. Some assemblers allow an identifier on decimal radix numbers, in which the number would be written with a "d" suffix: 1234d.

Binary numbers are sometimes prepended with the "0b" prefix: 0b100110111 (\ac{GCC} has a non-standard language extension for this\footnote{\url{https://gcc.gnu.org/onlinedocs/gcc/Binary-constants.html}}).
There is also another way: using a "b" suffix, for example: 100110111b.
This book tries to use the "0b" prefix consistently throughout the book for binary numbers.

Hexadecimal numbers are prepended with "0x" prefix in \CCpp and other \ac{PL}s: 0x1234ABCD.
Alternatively, they are given a "h" suffix: 1234ABCDh. This is common way of representing them in assemblers and debuggers.
In this convention, if the number is started with a Latin (A..F) digit, a 0 is added at the beginning: 0ABCDEFh.
There was also convention that was popular in 8-bit home computers era, using \$ prefix, like \$ABCD.
The book will try to stick to "0x" prefix throughout the book for hexadecimal numbers.

Should one learn to convert numbers mentally? A table of 1-digit hexadecimal numbers can easily be memorized.
As for larger numbers, it's probably not worth tormenting yourself.

Perhaps the most visible hexadecimal numbers are in \ac{URL}s.
This is the way that non-Latin characters are encoded.
For example:
\url{https://en.wiktionary.org/wiki/na\%C3\%AFvet\%C3\%A9} is the \ac{URL} of Wiktionary article about \q{naïveté} word.

\subsubsection{Octal Radix}

Another numeral system heavily used in the past of computer programming is octal. In octal there are 8 digits (0..7), and each is mapped to 3 bits, so it's easy to convert numbers back and forth.
It has been superseded by the hexadecimal system almost everywhere, but, surprisingly, there is a *NIX utility, used often by many people, which takes octal numbers as argument: \TT{chmod}.

\myindex{UNIX!chmod}
As many *NIX users know, \TT{chmod} argument can be a number of 3 digits. The first digit represents the rights of the owner of the file (read, write and/or execute), the second is the rights for the group to which the file belongs, and the third is for everyone else.
Each digit that \TT{chmod} takes can be represented in binary form:

\begin{center}
\begin{longtable}{ | l | l | l | }
\hline
\HeaderColor decimal & \HeaderColor binary & \HeaderColor meaning \\
\hline
7	&111	&\textbf{rwx} \\
6	&110	&\textbf{rw-} \\
5	&101	&\textbf{r-x} \\
4	&100	&\textbf{r-{}-} \\
3	&011	&\textbf{-wx} \\
2	&010	&\textbf{-w-} \\
1	&001	&\textbf{-{}-x} \\
0	&000	&\textbf{-{}-{}-} \\
\hline
\end{longtable}
\end{center}

So each bit is mapped to a flag: read/write/execute.

The importance of \TT{chmod} here is that the whole number in argument can be represented as octal number.
Let's take, for example, 644.
When you run \TT{chmod 644 file}, you set read/write permissions for owner, read permissions for group and again, read permissions for everyone else.
If we convert the octal number 644 to binary, it would be \TT{110100100}, or, in groups of 3 bits, \TT{110 100 100}.

Now we see that each triplet describe permissions for owner/group/others: first is \TT{rw-}, second is \TT{r--} and third is \TT{r--}.

The octal numeral system was also popular on old computers like PDP-8, because word there could be 12, 24 or 36 bits, and these numbers are all divisible by 3, so the octal system was natural in that environment.
Nowadays, all popular computers employ word/address sizes of 16, 32 or 64 bits, and these numbers are all divisible by 4, so the hexadecimal system is more natural there.

The octal numeral system is supported by all standard \CCpp compilers.
This is a source of confusion sometimes, because octal numbers are encoded with a zero prepended, for example, 0377 is 255.
Sometimes, you might make a typo and write "09" instead of 9, and the compiler would report an error.
GCC might report something like this:\\
\TT{error: invalid digit "9" in octal constant}.

Also, the octal system is somewhat popular in Java. When the IDA shows Java strings with non-printable characters,
they are encoded in the octal system instead of hexadecimal.
\myindex{JAD}
The JAD Java decompiler behaves the same way.

\subsubsection{Divisibility}

When you see a decimal number like 120, you can quickly deduce that it's divisible by 10, because the last digit is zero.
In the same way, 123400 is divisible by 100, because the two last digits are zeros.

Likewise, the hexadecimal number 0x1230 is divisible by 0x10 (or 16), 0x123000 is divisible by 0x1000 (or 4096), etc.

The binary number 0b1000101000 is divisible by 0b1000 (8), etc.

This property can often be used to quickly realize if the size of some block in memory is padded to some boundary.
For example, sections in \ac{PE} files are almost always started at addresses ending with 3 hexadecimal zeros: 0x41000, 0x10001000, etc.
The reason behind this is the fact that almost all \ac{PE} sections are padded to a boundary of 0x1000 (4096) bytes.

\subsubsection{Multi-Precision Arithmetic and Radix}

\index{RSA}
Multi-precision arithmetic can use huge numbers, and each one may be stored in several bytes.
For example, RSA keys, both public and private, span up to 4096 bits, and maybe even more.

% I'm not sure how to change this, but the normal format for quoting would be just to mention the author or book, and footnote to the full reference
In \InSqBrackets{\TAOCPvolII, 265} we find the following idea: when you store a multi-precision number in several bytes,
the whole number can be represented as having a radix of $2^8=256$, and each digit goes to the corresponding byte.
Likewise, if you store a multi-precision number in several 32-bit integer values, each digit goes to each 32-bit slot,
and you may think about this number as stored in radix of $2^{32}$.

\subsubsection{How to Pronounce Non-Decimal Numbers}

Numbers in a non-decimal base are usually pronounced by digit by digit: ``one-zero-zero-one-one-...''.
Words like ``ten'' and ``thousand'' are usually not pronounced, to prevent confusion with the decimal base system.

\subsubsection{Floating point numbers}

To distinguish floating point numbers from integers, they are usually written with ``.0'' at the end,
like $0.0$, $123.0$, etc.
}
\RU{\subsection{Представление чисел}

Люди привыкли к десятичной системе счисления вероятно потому что почти у каждого есть по 10 пальцев.
Тем не менее, число 10 не имеет особого значения в науке и математике.
Двоичная система естествена для цифровой электроники: 0 означает отсутствие тока в проводе и 1 --- его присутствие.
10 в двоичной системе это 2 в десятичной; 100 в двоичной это 4 в десятичной, итд.

Если в системе счисления есть 10 цифр, её \IT{основание} или \IT{radix} это 10.
Двоичная система имеет \IT{основание} 2.

Важные вещи, которые полезно вспомнить:
1) \IT{число} это число, в то время как \IT{цифра} это термин из системы письменности, и это обычно один символ;
2) само число не меняется, когда конвертируется из одного основания в другое: меняется способ его записи (или представления
в памяти).

Как сконвертировать число из одного основания в другое?

Позиционная нотация используется почти везде, это означает, что всякая цифра имеет свой вес, в зависимости от её расположения
внутри числа.
Если 2 расположена в самом последнем месте справа, это 2.
Если она расположена в месте перед последним, это 20.

Что означает $1234$?

$10^3 \cdot 1 + 10^2 \cdot 2 + 10^1 \cdot 3 + 1 \cdot 4$ = 1234 или
$1000 \cdot 1 + 100 \cdot 2 + 10 \cdot 3 + 4 = 1234$

Та же история и для двоичных чисел, только основание там 2 вместо 10.
Что означает 0b101011?

$2^5 \cdot 1 + 2^4 \cdot 0 + 2^3 \cdot 1 + 2^2 \cdot 0 + 2^1 \cdot 1 + 2^0 \cdot 1 = 43$ или
$32 \cdot 1 + 16 \cdot 0 + 8 \cdot 1 + 4 \cdot 0 + 2 \cdot 1 + 1 = 43$

Позиционную нотацию можно противопоставить непозиционной нотации, такой как римская система записи чисел
\footnote{Об эволюции способов записи чисел, см.также: \InSqBrackets{\TAOCPvolII{}, 195--213.}}.
Вероятно, человечество перешло на позиционную нотацию, потому что так проще работать с числами (сложение, умножение, итд)
на бумаге, в ручную.

Действительно, двоичные числа можно складывать, вычитать, итд, точно также, как этому обычно обучают в школах,
только доступны лишь 2 цифры.

Двоичные числа громоздки, когда их используют в исходных кодах и дампах, так что в этих случаях применяется шестнадцатеричная
система.
Используются цифры 0..9 и еще 6 латинских букв: A..F.
Каждая шестнадцатеричная цифра занимает 4 бита или 4 двоичных цифры, так что конвертировать из двоичной системы в
шестнадцатеричную и назад, можно легко вручную, или даже в уме.

\begin{center}
\begin{longtable}{ | l | l | l | }
\hline
\HeaderColor шестнадцатеричная & \HeaderColor двоичная & \HeaderColor десятичная \\
\hline
0	&0000	&0 \\
1	&0001	&1 \\
2	&0010	&2 \\
3	&0011	&3 \\
4	&0100	&4 \\
5	&0101	&5 \\
6	&0110	&6 \\
7	&0111	&7 \\
8	&1000	&8 \\
9	&1001	&9 \\
A	&1010	&10 \\
B	&1011	&11 \\
C	&1100	&12 \\
D	&1101	&13 \\
E	&1110	&14 \\
F	&1111	&15 \\
\hline
\end{longtable}
\end{center}

Как понять, какое основание используется в конкретном месте?

Десятичные числа обычно записываются как есть, т.е., 1234. Но некоторые ассемблеры позволяют подчеркивать
этот факт для ясности, и это число может быть дополнено суффиксом "d": 1234d.

К двоичным числам иногда спереди добавляют префикс "0b": 0b100110111
(В \ac{GCC} для этого есть нестандартное расширение языка
\footnote{\url{https://gcc.gnu.org/onlinedocs/gcc/Binary-constants.html}}).
Есть также еще один способ: суффикс "b", например: 100110111b.
В этой книге я буду пытаться придерживаться префикса "0b" для двоичных чисел.

Шестнадцатеричные числа имеют префикс "0x" в \CCpp и некоторых других \ac{PL}: 0x1234ABCD.
Либо они имеют суффикс "h": 1234ABCDh --- обычно так они представляются в ассемблерах и отладчиках.
Если число начинается с цифры A..F, перед ним добавляется 0: 0ABCDEFh.
Во времена 8-битных домашних компьютеров, был также способ записи чисел используя префикс \$, например, \$ABCD.
В книге я попытаюсь придерживаться префикса "0x" для шестнадцатеричных чисел.

Нужно ли учиться конвертировать числа в уме? Таблицу шестнадцатеричных чисел из одной цифры легко запомнить.
А запоминать б\'{о}льшие числа, наверное, не стоит.

Наверное, чаще всего шестнадцатеричные числа можно увидеть в \ac{URL}-ах.
Так кодируются буквы не из числа латинских.
Например:
\url{https://en.wiktionary.org/wiki/na\%C3\%AFvet\%C3\%A9} это \ac{URL} страницы в Wiktionary о слове \q{naïveté}.

\subsubsection{Восьмеричная система}

Еще одна система, которая в прошлом много использовалась в программировании это восьмеричная: есть 8 цифр (0..7) и каждая
описывает 3 бита, так что легко конвертировать числа туда и назад.
Она почти везде была заменена шестнадцатеричной, но удивительно, в *NIX имеется утилита использующаяся многими людьми,
которая принимает на вход восьмеричное число: \TT{chmod}.

\myindex{UNIX!chmod}
Как знают многие пользователи *NIX, аргумент \TT{chmod} это число из трех цифр. Первая цифра это права владельца файла,
вторая это права группы (которой файл принадлежит), третья для всех остальных.
И каждая цифра может быть представлена в двоичном виде:

\begin{center}
\begin{longtable}{ | l | l | l | }
\hline
\HeaderColor десятичная & \HeaderColor двоичная & \HeaderColor значение \\
\hline
7	&111	&\textbf{rwx} \\
6	&110	&\textbf{rw-} \\
5	&101	&\textbf{r-x} \\
4	&100	&\textbf{r-{}-} \\
3	&011	&\textbf{-wx} \\
2	&010	&\textbf{-w-} \\
1	&001	&\textbf{-{}-x} \\
0	&000	&\textbf{-{}-{}-} \\
\hline
\end{longtable}
\end{center}

Так что каждый бит привязан к флагу: read/write/execute (чтение/запись/исполнение).

И вот почему я вспомнил здесь о \TT{chmod}, это потому что всё число может быть представлено как число в восьмеричной системе.
Для примера возьмем 644.
Когда вы запускаете \TT{chmod 644 file}, вы выставляете права read/write для владельца, права read для группы, и снова,
read для всех остальных.
Сконвертируем число 644 из восьмеричной системы в двоичную, это будет \TT{110100100}, или (в группах по 3 бита) \TT{110 100 100}.

Теперь мы видим, что каждая тройка описывает права для владельца/группы/остальных:
первая это \TT{rw-}, вторая это \TT{r--} и третья это \TT{r--}.

Восьмеричная система была также популярная на старых компьютерах вроде PDP-8, потому что слово там могло содержать 12, 24 или
36 бит, и эти числа делятся на 3, так что выбор восьмеричной системы в той среде был логичен.
Сейчас, все популярные компьютеры имеют размер слова/адреса 16, 32 или 64 бита, и эти числа делятся на 4,
так что шестнадцатеричная система здесь удобнее.

Восьмеричная система поддерживается всеми стандартными компиляторами \CCpp{}.
Это иногда источник недоумения, потому что восьмеричные числа кодируются с нулем вперед, например, 0377 это 255.
И иногда, вы можете сделать опечатку, и написать "09" вместо 9, и компилятор выдаст ошибку.
GCC может выдать что-то вроде:\\
\TT{error: invalid digit "9" in octal constant}.

Также, восьмеричная система популярна в Java: когда IDA показывает строку с непечатаемыми символами,
они кодируются в восьмеричной системе вместо шестнадцатеричной.
\myindex{JAD}
Точно также себя ведет декомпилятор с Java JAD.

\subsubsection{Делимость}

Когда вы видите десятичное число вроде 120, вы можете быстро понять что оно делится на 10, потому что последняя цифра это 0.
Точно также, 123400 делится на 100, потому что две последних цифры это нули.

Точно также, шестнадцатеричное число 0x1230 делится на 0x10 (или 16), 0x123000 делится на 0x1000 (или 4096), итд.

Двоичное число 0b1000101000 делится на 0b1000 (8), итд.

Это свойство можно часто использовать, чтобы быстро понять,
что длина какого-либо блока в памяти выровнена по некоторой границе.
Например, секции в \ac{PE}-файлах почти всегда начинаются с адресов заканчивающихся 3 шестнадцатеричными нулями:
0x41000, 0x10001000, итд.
Причина в том, что почти все секции в \ac{PE} выровнены по границе 0x1000 (4096) байт.

\subsubsection{Арифметика произвольной точности и основание}

\index{RSA}
Арифметика произвольной точности (multi-precision arithmetic) может использовать огромные числа,
которые могут храниться в нескольких байтах.
Например, ключи RSA, и открытые и закрытые, могут занимать до 4096 бит и даже больше.

В \InSqBrackets{\TAOCPvolII, 265} можно найти такую идею: когда вы сохраняете число произвольной точности в нескольких байтах,
всё число может быть представлено как имеющую систему счисления по основанию $2^8=256$, и каждая цифра находится
в соответствующем байте.
Точно также, если вы сохраняете число произвольной точности в нескольких 32-битных целочисленных значениях,
каждая цифра отправляется в каждый 32-битный слот, и вы можете считать что это число записано в системе с основанием $2^{32}$.

\subsubsection{Произношение}

Числа в недесятичных системах счислениях обычно произносятся по одной цифре: ``один-ноль-ноль-один-один-...''.
Слова вроде ``десять'', ``тысяча'', итд, обычно не произносятся, потому что тогда можно спутать с десятичной системой.

\subsubsection{Числа с плавающей запятой}

Чтобы отличать числа с плавающей запятой от целочисленных, часто, в конце добавляют ``.0'',
например $0.0$, $123.0$, итд.

}
\ITA{\input{patterns/numeral_ITA}}
\DE{\input{patterns/numeral_DE}}
\FR{\input{patterns/numeral_FR}}
\PL{\input{patterns/numeral_PL}}

% chapters
\ifdefined\SPANISH
\chapter{Patrones de código}
\fi % SPANISH

\ifdefined\GERMAN
\chapter{Code-Muster}
\fi % GERMAN

\ifdefined\ENGLISH
\chapter{Code Patterns}
\fi % ENGLISH

\ifdefined\ITALIAN
\chapter{Forme di codice}
\fi % ITALIAN

\ifdefined\RUSSIAN
\chapter{Образцы кода}
\fi % RUSSIAN

\ifdefined\BRAZILIAN
\chapter{Padrões de códigos}
\fi % BRAZILIAN

\ifdefined\THAI
\chapter{รูปแบบของโค้ด}
\fi % THAI

\ifdefined\FRENCH
\chapter{Modèle de code}
\fi % FRENCH

\ifdefined\POLISH
\chapter{\PLph{}}
\fi % POLISH

% sections
\EN{\input{patterns/patterns_opt_dbg_EN}}
\ES{\input{patterns/patterns_opt_dbg_ES}}
\ITA{\input{patterns/patterns_opt_dbg_ITA}}
\PTBR{\input{patterns/patterns_opt_dbg_PTBR}}
\RU{\input{patterns/patterns_opt_dbg_RU}}
\THA{\input{patterns/patterns_opt_dbg_THA}}
\DE{\input{patterns/patterns_opt_dbg_DE}}
\FR{\input{patterns/patterns_opt_dbg_FR}}
\PL{\input{patterns/patterns_opt_dbg_PL}}

\RU{\section{Некоторые базовые понятия}}
\EN{\section{Some basics}}
\DE{\section{Einige Grundlagen}}
\FR{\section{Quelques bases}}
\ES{\section{\ESph{}}}
\ITA{\section{Alcune basi teoriche}}
\PTBR{\section{\PTBRph{}}}
\THA{\section{\THAph{}}}
\PL{\section{\PLph{}}}

% sections:
\EN{\input{patterns/intro_CPU_ISA_EN}}
\ES{\input{patterns/intro_CPU_ISA_ES}}
\ITA{\input{patterns/intro_CPU_ISA_ITA}}
\PTBR{\input{patterns/intro_CPU_ISA_PTBR}}
\RU{\input{patterns/intro_CPU_ISA_RU}}
\DE{\input{patterns/intro_CPU_ISA_DE}}
\FR{\input{patterns/intro_CPU_ISA_FR}}
\PL{\input{patterns/intro_CPU_ISA_PL}}

\EN{\input{patterns/numeral_EN}}
\RU{\input{patterns/numeral_RU}}
\ITA{\input{patterns/numeral_ITA}}
\DE{\input{patterns/numeral_DE}}
\FR{\input{patterns/numeral_FR}}
\PL{\input{patterns/numeral_PL}}

% chapters
\input{patterns/00_empty/main}
\input{patterns/011_ret/main}
\input{patterns/01_helloworld/main}
\input{patterns/015_prolog_epilogue/main}
\input{patterns/02_stack/main}
\input{patterns/03_printf/main}
\input{patterns/04_scanf/main}
\input{patterns/05_passing_arguments/main}
\input{patterns/06_return_results/main}
\input{patterns/061_pointers/main}
\input{patterns/065_GOTO/main}
\input{patterns/07_jcc/main}
\input{patterns/08_switch/main}
\input{patterns/09_loops/main}
\input{patterns/10_strings/main}
\input{patterns/11_arith_optimizations/main}
\input{patterns/12_FPU/main}
\input{patterns/13_arrays/main}
\input{patterns/14_bitfields/main}
\EN{\input{patterns/145_LCG/main_EN}}
\RU{\input{patterns/145_LCG/main_RU}}
\input{patterns/15_structs/main}
\input{patterns/17_unions/main}
\input{patterns/18_pointers_to_functions/main}
\input{patterns/185_64bit_in_32_env/main}

\EN{\input{patterns/19_SIMD/main_EN}}
\RU{\input{patterns/19_SIMD/main_RU}}
\DE{\input{patterns/19_SIMD/main_DE}}

\EN{\input{patterns/20_x64/main_EN}}
\RU{\input{patterns/20_x64/main_RU}}

\EN{\input{patterns/205_floating_SIMD/main_EN}}
\RU{\input{patterns/205_floating_SIMD/main_RU}}
\DE{\input{patterns/205_floating_SIMD/main_DE}}

\EN{\input{patterns/ARM/main_EN}}
\RU{\input{patterns/ARM/main_RU}}
\DE{\input{patterns/ARM/main_DE}}

\input{patterns/MIPS/main}

\ifdefined\SPANISH
\chapter{Patrones de código}
\fi % SPANISH

\ifdefined\GERMAN
\chapter{Code-Muster}
\fi % GERMAN

\ifdefined\ENGLISH
\chapter{Code Patterns}
\fi % ENGLISH

\ifdefined\ITALIAN
\chapter{Forme di codice}
\fi % ITALIAN

\ifdefined\RUSSIAN
\chapter{Образцы кода}
\fi % RUSSIAN

\ifdefined\BRAZILIAN
\chapter{Padrões de códigos}
\fi % BRAZILIAN

\ifdefined\THAI
\chapter{รูปแบบของโค้ด}
\fi % THAI

\ifdefined\FRENCH
\chapter{Modèle de code}
\fi % FRENCH

\ifdefined\POLISH
\chapter{\PLph{}}
\fi % POLISH

% sections
\EN{\input{patterns/patterns_opt_dbg_EN}}
\ES{\input{patterns/patterns_opt_dbg_ES}}
\ITA{\input{patterns/patterns_opt_dbg_ITA}}
\PTBR{\input{patterns/patterns_opt_dbg_PTBR}}
\RU{\input{patterns/patterns_opt_dbg_RU}}
\THA{\input{patterns/patterns_opt_dbg_THA}}
\DE{\input{patterns/patterns_opt_dbg_DE}}
\FR{\input{patterns/patterns_opt_dbg_FR}}
\PL{\input{patterns/patterns_opt_dbg_PL}}

\RU{\section{Некоторые базовые понятия}}
\EN{\section{Some basics}}
\DE{\section{Einige Grundlagen}}
\FR{\section{Quelques bases}}
\ES{\section{\ESph{}}}
\ITA{\section{Alcune basi teoriche}}
\PTBR{\section{\PTBRph{}}}
\THA{\section{\THAph{}}}
\PL{\section{\PLph{}}}

% sections:
\EN{\input{patterns/intro_CPU_ISA_EN}}
\ES{\input{patterns/intro_CPU_ISA_ES}}
\ITA{\input{patterns/intro_CPU_ISA_ITA}}
\PTBR{\input{patterns/intro_CPU_ISA_PTBR}}
\RU{\input{patterns/intro_CPU_ISA_RU}}
\DE{\input{patterns/intro_CPU_ISA_DE}}
\FR{\input{patterns/intro_CPU_ISA_FR}}
\PL{\input{patterns/intro_CPU_ISA_PL}}

\EN{\input{patterns/numeral_EN}}
\RU{\input{patterns/numeral_RU}}
\ITA{\input{patterns/numeral_ITA}}
\DE{\input{patterns/numeral_DE}}
\FR{\input{patterns/numeral_FR}}
\PL{\input{patterns/numeral_PL}}

% chapters
\input{patterns/00_empty/main}
\input{patterns/011_ret/main}
\input{patterns/01_helloworld/main}
\input{patterns/015_prolog_epilogue/main}
\input{patterns/02_stack/main}
\input{patterns/03_printf/main}
\input{patterns/04_scanf/main}
\input{patterns/05_passing_arguments/main}
\input{patterns/06_return_results/main}
\input{patterns/061_pointers/main}
\input{patterns/065_GOTO/main}
\input{patterns/07_jcc/main}
\input{patterns/08_switch/main}
\input{patterns/09_loops/main}
\input{patterns/10_strings/main}
\input{patterns/11_arith_optimizations/main}
\input{patterns/12_FPU/main}
\input{patterns/13_arrays/main}
\input{patterns/14_bitfields/main}
\EN{\input{patterns/145_LCG/main_EN}}
\RU{\input{patterns/145_LCG/main_RU}}
\input{patterns/15_structs/main}
\input{patterns/17_unions/main}
\input{patterns/18_pointers_to_functions/main}
\input{patterns/185_64bit_in_32_env/main}

\EN{\input{patterns/19_SIMD/main_EN}}
\RU{\input{patterns/19_SIMD/main_RU}}
\DE{\input{patterns/19_SIMD/main_DE}}

\EN{\input{patterns/20_x64/main_EN}}
\RU{\input{patterns/20_x64/main_RU}}

\EN{\input{patterns/205_floating_SIMD/main_EN}}
\RU{\input{patterns/205_floating_SIMD/main_RU}}
\DE{\input{patterns/205_floating_SIMD/main_DE}}

\EN{\input{patterns/ARM/main_EN}}
\RU{\input{patterns/ARM/main_RU}}
\DE{\input{patterns/ARM/main_DE}}

\input{patterns/MIPS/main}

\ifdefined\SPANISH
\chapter{Patrones de código}
\fi % SPANISH

\ifdefined\GERMAN
\chapter{Code-Muster}
\fi % GERMAN

\ifdefined\ENGLISH
\chapter{Code Patterns}
\fi % ENGLISH

\ifdefined\ITALIAN
\chapter{Forme di codice}
\fi % ITALIAN

\ifdefined\RUSSIAN
\chapter{Образцы кода}
\fi % RUSSIAN

\ifdefined\BRAZILIAN
\chapter{Padrões de códigos}
\fi % BRAZILIAN

\ifdefined\THAI
\chapter{รูปแบบของโค้ด}
\fi % THAI

\ifdefined\FRENCH
\chapter{Modèle de code}
\fi % FRENCH

\ifdefined\POLISH
\chapter{\PLph{}}
\fi % POLISH

% sections
\EN{\input{patterns/patterns_opt_dbg_EN}}
\ES{\input{patterns/patterns_opt_dbg_ES}}
\ITA{\input{patterns/patterns_opt_dbg_ITA}}
\PTBR{\input{patterns/patterns_opt_dbg_PTBR}}
\RU{\input{patterns/patterns_opt_dbg_RU}}
\THA{\input{patterns/patterns_opt_dbg_THA}}
\DE{\input{patterns/patterns_opt_dbg_DE}}
\FR{\input{patterns/patterns_opt_dbg_FR}}
\PL{\input{patterns/patterns_opt_dbg_PL}}

\RU{\section{Некоторые базовые понятия}}
\EN{\section{Some basics}}
\DE{\section{Einige Grundlagen}}
\FR{\section{Quelques bases}}
\ES{\section{\ESph{}}}
\ITA{\section{Alcune basi teoriche}}
\PTBR{\section{\PTBRph{}}}
\THA{\section{\THAph{}}}
\PL{\section{\PLph{}}}

% sections:
\EN{\input{patterns/intro_CPU_ISA_EN}}
\ES{\input{patterns/intro_CPU_ISA_ES}}
\ITA{\input{patterns/intro_CPU_ISA_ITA}}
\PTBR{\input{patterns/intro_CPU_ISA_PTBR}}
\RU{\input{patterns/intro_CPU_ISA_RU}}
\DE{\input{patterns/intro_CPU_ISA_DE}}
\FR{\input{patterns/intro_CPU_ISA_FR}}
\PL{\input{patterns/intro_CPU_ISA_PL}}

\EN{\input{patterns/numeral_EN}}
\RU{\input{patterns/numeral_RU}}
\ITA{\input{patterns/numeral_ITA}}
\DE{\input{patterns/numeral_DE}}
\FR{\input{patterns/numeral_FR}}
\PL{\input{patterns/numeral_PL}}

% chapters
\input{patterns/00_empty/main}
\input{patterns/011_ret/main}
\input{patterns/01_helloworld/main}
\input{patterns/015_prolog_epilogue/main}
\input{patterns/02_stack/main}
\input{patterns/03_printf/main}
\input{patterns/04_scanf/main}
\input{patterns/05_passing_arguments/main}
\input{patterns/06_return_results/main}
\input{patterns/061_pointers/main}
\input{patterns/065_GOTO/main}
\input{patterns/07_jcc/main}
\input{patterns/08_switch/main}
\input{patterns/09_loops/main}
\input{patterns/10_strings/main}
\input{patterns/11_arith_optimizations/main}
\input{patterns/12_FPU/main}
\input{patterns/13_arrays/main}
\input{patterns/14_bitfields/main}
\EN{\input{patterns/145_LCG/main_EN}}
\RU{\input{patterns/145_LCG/main_RU}}
\input{patterns/15_structs/main}
\input{patterns/17_unions/main}
\input{patterns/18_pointers_to_functions/main}
\input{patterns/185_64bit_in_32_env/main}

\EN{\input{patterns/19_SIMD/main_EN}}
\RU{\input{patterns/19_SIMD/main_RU}}
\DE{\input{patterns/19_SIMD/main_DE}}

\EN{\input{patterns/20_x64/main_EN}}
\RU{\input{patterns/20_x64/main_RU}}

\EN{\input{patterns/205_floating_SIMD/main_EN}}
\RU{\input{patterns/205_floating_SIMD/main_RU}}
\DE{\input{patterns/205_floating_SIMD/main_DE}}

\EN{\input{patterns/ARM/main_EN}}
\RU{\input{patterns/ARM/main_RU}}
\DE{\input{patterns/ARM/main_DE}}

\input{patterns/MIPS/main}

\ifdefined\SPANISH
\chapter{Patrones de código}
\fi % SPANISH

\ifdefined\GERMAN
\chapter{Code-Muster}
\fi % GERMAN

\ifdefined\ENGLISH
\chapter{Code Patterns}
\fi % ENGLISH

\ifdefined\ITALIAN
\chapter{Forme di codice}
\fi % ITALIAN

\ifdefined\RUSSIAN
\chapter{Образцы кода}
\fi % RUSSIAN

\ifdefined\BRAZILIAN
\chapter{Padrões de códigos}
\fi % BRAZILIAN

\ifdefined\THAI
\chapter{รูปแบบของโค้ด}
\fi % THAI

\ifdefined\FRENCH
\chapter{Modèle de code}
\fi % FRENCH

\ifdefined\POLISH
\chapter{\PLph{}}
\fi % POLISH

% sections
\EN{\input{patterns/patterns_opt_dbg_EN}}
\ES{\input{patterns/patterns_opt_dbg_ES}}
\ITA{\input{patterns/patterns_opt_dbg_ITA}}
\PTBR{\input{patterns/patterns_opt_dbg_PTBR}}
\RU{\input{patterns/patterns_opt_dbg_RU}}
\THA{\input{patterns/patterns_opt_dbg_THA}}
\DE{\input{patterns/patterns_opt_dbg_DE}}
\FR{\input{patterns/patterns_opt_dbg_FR}}
\PL{\input{patterns/patterns_opt_dbg_PL}}

\RU{\section{Некоторые базовые понятия}}
\EN{\section{Some basics}}
\DE{\section{Einige Grundlagen}}
\FR{\section{Quelques bases}}
\ES{\section{\ESph{}}}
\ITA{\section{Alcune basi teoriche}}
\PTBR{\section{\PTBRph{}}}
\THA{\section{\THAph{}}}
\PL{\section{\PLph{}}}

% sections:
\EN{\input{patterns/intro_CPU_ISA_EN}}
\ES{\input{patterns/intro_CPU_ISA_ES}}
\ITA{\input{patterns/intro_CPU_ISA_ITA}}
\PTBR{\input{patterns/intro_CPU_ISA_PTBR}}
\RU{\input{patterns/intro_CPU_ISA_RU}}
\DE{\input{patterns/intro_CPU_ISA_DE}}
\FR{\input{patterns/intro_CPU_ISA_FR}}
\PL{\input{patterns/intro_CPU_ISA_PL}}

\EN{\input{patterns/numeral_EN}}
\RU{\input{patterns/numeral_RU}}
\ITA{\input{patterns/numeral_ITA}}
\DE{\input{patterns/numeral_DE}}
\FR{\input{patterns/numeral_FR}}
\PL{\input{patterns/numeral_PL}}

% chapters
\input{patterns/00_empty/main}
\input{patterns/011_ret/main}
\input{patterns/01_helloworld/main}
\input{patterns/015_prolog_epilogue/main}
\input{patterns/02_stack/main}
\input{patterns/03_printf/main}
\input{patterns/04_scanf/main}
\input{patterns/05_passing_arguments/main}
\input{patterns/06_return_results/main}
\input{patterns/061_pointers/main}
\input{patterns/065_GOTO/main}
\input{patterns/07_jcc/main}
\input{patterns/08_switch/main}
\input{patterns/09_loops/main}
\input{patterns/10_strings/main}
\input{patterns/11_arith_optimizations/main}
\input{patterns/12_FPU/main}
\input{patterns/13_arrays/main}
\input{patterns/14_bitfields/main}
\EN{\input{patterns/145_LCG/main_EN}}
\RU{\input{patterns/145_LCG/main_RU}}
\input{patterns/15_structs/main}
\input{patterns/17_unions/main}
\input{patterns/18_pointers_to_functions/main}
\input{patterns/185_64bit_in_32_env/main}

\EN{\input{patterns/19_SIMD/main_EN}}
\RU{\input{patterns/19_SIMD/main_RU}}
\DE{\input{patterns/19_SIMD/main_DE}}

\EN{\input{patterns/20_x64/main_EN}}
\RU{\input{patterns/20_x64/main_RU}}

\EN{\input{patterns/205_floating_SIMD/main_EN}}
\RU{\input{patterns/205_floating_SIMD/main_RU}}
\DE{\input{patterns/205_floating_SIMD/main_DE}}

\EN{\input{patterns/ARM/main_EN}}
\RU{\input{patterns/ARM/main_RU}}
\DE{\input{patterns/ARM/main_DE}}

\input{patterns/MIPS/main}

\ifdefined\SPANISH
\chapter{Patrones de código}
\fi % SPANISH

\ifdefined\GERMAN
\chapter{Code-Muster}
\fi % GERMAN

\ifdefined\ENGLISH
\chapter{Code Patterns}
\fi % ENGLISH

\ifdefined\ITALIAN
\chapter{Forme di codice}
\fi % ITALIAN

\ifdefined\RUSSIAN
\chapter{Образцы кода}
\fi % RUSSIAN

\ifdefined\BRAZILIAN
\chapter{Padrões de códigos}
\fi % BRAZILIAN

\ifdefined\THAI
\chapter{รูปแบบของโค้ด}
\fi % THAI

\ifdefined\FRENCH
\chapter{Modèle de code}
\fi % FRENCH

\ifdefined\POLISH
\chapter{\PLph{}}
\fi % POLISH

% sections
\EN{\input{patterns/patterns_opt_dbg_EN}}
\ES{\input{patterns/patterns_opt_dbg_ES}}
\ITA{\input{patterns/patterns_opt_dbg_ITA}}
\PTBR{\input{patterns/patterns_opt_dbg_PTBR}}
\RU{\input{patterns/patterns_opt_dbg_RU}}
\THA{\input{patterns/patterns_opt_dbg_THA}}
\DE{\input{patterns/patterns_opt_dbg_DE}}
\FR{\input{patterns/patterns_opt_dbg_FR}}
\PL{\input{patterns/patterns_opt_dbg_PL}}

\RU{\section{Некоторые базовые понятия}}
\EN{\section{Some basics}}
\DE{\section{Einige Grundlagen}}
\FR{\section{Quelques bases}}
\ES{\section{\ESph{}}}
\ITA{\section{Alcune basi teoriche}}
\PTBR{\section{\PTBRph{}}}
\THA{\section{\THAph{}}}
\PL{\section{\PLph{}}}

% sections:
\EN{\input{patterns/intro_CPU_ISA_EN}}
\ES{\input{patterns/intro_CPU_ISA_ES}}
\ITA{\input{patterns/intro_CPU_ISA_ITA}}
\PTBR{\input{patterns/intro_CPU_ISA_PTBR}}
\RU{\input{patterns/intro_CPU_ISA_RU}}
\DE{\input{patterns/intro_CPU_ISA_DE}}
\FR{\input{patterns/intro_CPU_ISA_FR}}
\PL{\input{patterns/intro_CPU_ISA_PL}}

\EN{\input{patterns/numeral_EN}}
\RU{\input{patterns/numeral_RU}}
\ITA{\input{patterns/numeral_ITA}}
\DE{\input{patterns/numeral_DE}}
\FR{\input{patterns/numeral_FR}}
\PL{\input{patterns/numeral_PL}}

% chapters
\input{patterns/00_empty/main}
\input{patterns/011_ret/main}
\input{patterns/01_helloworld/main}
\input{patterns/015_prolog_epilogue/main}
\input{patterns/02_stack/main}
\input{patterns/03_printf/main}
\input{patterns/04_scanf/main}
\input{patterns/05_passing_arguments/main}
\input{patterns/06_return_results/main}
\input{patterns/061_pointers/main}
\input{patterns/065_GOTO/main}
\input{patterns/07_jcc/main}
\input{patterns/08_switch/main}
\input{patterns/09_loops/main}
\input{patterns/10_strings/main}
\input{patterns/11_arith_optimizations/main}
\input{patterns/12_FPU/main}
\input{patterns/13_arrays/main}
\input{patterns/14_bitfields/main}
\EN{\input{patterns/145_LCG/main_EN}}
\RU{\input{patterns/145_LCG/main_RU}}
\input{patterns/15_structs/main}
\input{patterns/17_unions/main}
\input{patterns/18_pointers_to_functions/main}
\input{patterns/185_64bit_in_32_env/main}

\EN{\input{patterns/19_SIMD/main_EN}}
\RU{\input{patterns/19_SIMD/main_RU}}
\DE{\input{patterns/19_SIMD/main_DE}}

\EN{\input{patterns/20_x64/main_EN}}
\RU{\input{patterns/20_x64/main_RU}}

\EN{\input{patterns/205_floating_SIMD/main_EN}}
\RU{\input{patterns/205_floating_SIMD/main_RU}}
\DE{\input{patterns/205_floating_SIMD/main_DE}}

\EN{\input{patterns/ARM/main_EN}}
\RU{\input{patterns/ARM/main_RU}}
\DE{\input{patterns/ARM/main_DE}}

\input{patterns/MIPS/main}

\ifdefined\SPANISH
\chapter{Patrones de código}
\fi % SPANISH

\ifdefined\GERMAN
\chapter{Code-Muster}
\fi % GERMAN

\ifdefined\ENGLISH
\chapter{Code Patterns}
\fi % ENGLISH

\ifdefined\ITALIAN
\chapter{Forme di codice}
\fi % ITALIAN

\ifdefined\RUSSIAN
\chapter{Образцы кода}
\fi % RUSSIAN

\ifdefined\BRAZILIAN
\chapter{Padrões de códigos}
\fi % BRAZILIAN

\ifdefined\THAI
\chapter{รูปแบบของโค้ด}
\fi % THAI

\ifdefined\FRENCH
\chapter{Modèle de code}
\fi % FRENCH

\ifdefined\POLISH
\chapter{\PLph{}}
\fi % POLISH

% sections
\EN{\input{patterns/patterns_opt_dbg_EN}}
\ES{\input{patterns/patterns_opt_dbg_ES}}
\ITA{\input{patterns/patterns_opt_dbg_ITA}}
\PTBR{\input{patterns/patterns_opt_dbg_PTBR}}
\RU{\input{patterns/patterns_opt_dbg_RU}}
\THA{\input{patterns/patterns_opt_dbg_THA}}
\DE{\input{patterns/patterns_opt_dbg_DE}}
\FR{\input{patterns/patterns_opt_dbg_FR}}
\PL{\input{patterns/patterns_opt_dbg_PL}}

\RU{\section{Некоторые базовые понятия}}
\EN{\section{Some basics}}
\DE{\section{Einige Grundlagen}}
\FR{\section{Quelques bases}}
\ES{\section{\ESph{}}}
\ITA{\section{Alcune basi teoriche}}
\PTBR{\section{\PTBRph{}}}
\THA{\section{\THAph{}}}
\PL{\section{\PLph{}}}

% sections:
\EN{\input{patterns/intro_CPU_ISA_EN}}
\ES{\input{patterns/intro_CPU_ISA_ES}}
\ITA{\input{patterns/intro_CPU_ISA_ITA}}
\PTBR{\input{patterns/intro_CPU_ISA_PTBR}}
\RU{\input{patterns/intro_CPU_ISA_RU}}
\DE{\input{patterns/intro_CPU_ISA_DE}}
\FR{\input{patterns/intro_CPU_ISA_FR}}
\PL{\input{patterns/intro_CPU_ISA_PL}}

\EN{\input{patterns/numeral_EN}}
\RU{\input{patterns/numeral_RU}}
\ITA{\input{patterns/numeral_ITA}}
\DE{\input{patterns/numeral_DE}}
\FR{\input{patterns/numeral_FR}}
\PL{\input{patterns/numeral_PL}}

% chapters
\input{patterns/00_empty/main}
\input{patterns/011_ret/main}
\input{patterns/01_helloworld/main}
\input{patterns/015_prolog_epilogue/main}
\input{patterns/02_stack/main}
\input{patterns/03_printf/main}
\input{patterns/04_scanf/main}
\input{patterns/05_passing_arguments/main}
\input{patterns/06_return_results/main}
\input{patterns/061_pointers/main}
\input{patterns/065_GOTO/main}
\input{patterns/07_jcc/main}
\input{patterns/08_switch/main}
\input{patterns/09_loops/main}
\input{patterns/10_strings/main}
\input{patterns/11_arith_optimizations/main}
\input{patterns/12_FPU/main}
\input{patterns/13_arrays/main}
\input{patterns/14_bitfields/main}
\EN{\input{patterns/145_LCG/main_EN}}
\RU{\input{patterns/145_LCG/main_RU}}
\input{patterns/15_structs/main}
\input{patterns/17_unions/main}
\input{patterns/18_pointers_to_functions/main}
\input{patterns/185_64bit_in_32_env/main}

\EN{\input{patterns/19_SIMD/main_EN}}
\RU{\input{patterns/19_SIMD/main_RU}}
\DE{\input{patterns/19_SIMD/main_DE}}

\EN{\input{patterns/20_x64/main_EN}}
\RU{\input{patterns/20_x64/main_RU}}

\EN{\input{patterns/205_floating_SIMD/main_EN}}
\RU{\input{patterns/205_floating_SIMD/main_RU}}
\DE{\input{patterns/205_floating_SIMD/main_DE}}

\EN{\input{patterns/ARM/main_EN}}
\RU{\input{patterns/ARM/main_RU}}
\DE{\input{patterns/ARM/main_DE}}

\input{patterns/MIPS/main}

\ifdefined\SPANISH
\chapter{Patrones de código}
\fi % SPANISH

\ifdefined\GERMAN
\chapter{Code-Muster}
\fi % GERMAN

\ifdefined\ENGLISH
\chapter{Code Patterns}
\fi % ENGLISH

\ifdefined\ITALIAN
\chapter{Forme di codice}
\fi % ITALIAN

\ifdefined\RUSSIAN
\chapter{Образцы кода}
\fi % RUSSIAN

\ifdefined\BRAZILIAN
\chapter{Padrões de códigos}
\fi % BRAZILIAN

\ifdefined\THAI
\chapter{รูปแบบของโค้ด}
\fi % THAI

\ifdefined\FRENCH
\chapter{Modèle de code}
\fi % FRENCH

\ifdefined\POLISH
\chapter{\PLph{}}
\fi % POLISH

% sections
\EN{\input{patterns/patterns_opt_dbg_EN}}
\ES{\input{patterns/patterns_opt_dbg_ES}}
\ITA{\input{patterns/patterns_opt_dbg_ITA}}
\PTBR{\input{patterns/patterns_opt_dbg_PTBR}}
\RU{\input{patterns/patterns_opt_dbg_RU}}
\THA{\input{patterns/patterns_opt_dbg_THA}}
\DE{\input{patterns/patterns_opt_dbg_DE}}
\FR{\input{patterns/patterns_opt_dbg_FR}}
\PL{\input{patterns/patterns_opt_dbg_PL}}

\RU{\section{Некоторые базовые понятия}}
\EN{\section{Some basics}}
\DE{\section{Einige Grundlagen}}
\FR{\section{Quelques bases}}
\ES{\section{\ESph{}}}
\ITA{\section{Alcune basi teoriche}}
\PTBR{\section{\PTBRph{}}}
\THA{\section{\THAph{}}}
\PL{\section{\PLph{}}}

% sections:
\EN{\input{patterns/intro_CPU_ISA_EN}}
\ES{\input{patterns/intro_CPU_ISA_ES}}
\ITA{\input{patterns/intro_CPU_ISA_ITA}}
\PTBR{\input{patterns/intro_CPU_ISA_PTBR}}
\RU{\input{patterns/intro_CPU_ISA_RU}}
\DE{\input{patterns/intro_CPU_ISA_DE}}
\FR{\input{patterns/intro_CPU_ISA_FR}}
\PL{\input{patterns/intro_CPU_ISA_PL}}

\EN{\input{patterns/numeral_EN}}
\RU{\input{patterns/numeral_RU}}
\ITA{\input{patterns/numeral_ITA}}
\DE{\input{patterns/numeral_DE}}
\FR{\input{patterns/numeral_FR}}
\PL{\input{patterns/numeral_PL}}

% chapters
\input{patterns/00_empty/main}
\input{patterns/011_ret/main}
\input{patterns/01_helloworld/main}
\input{patterns/015_prolog_epilogue/main}
\input{patterns/02_stack/main}
\input{patterns/03_printf/main}
\input{patterns/04_scanf/main}
\input{patterns/05_passing_arguments/main}
\input{patterns/06_return_results/main}
\input{patterns/061_pointers/main}
\input{patterns/065_GOTO/main}
\input{patterns/07_jcc/main}
\input{patterns/08_switch/main}
\input{patterns/09_loops/main}
\input{patterns/10_strings/main}
\input{patterns/11_arith_optimizations/main}
\input{patterns/12_FPU/main}
\input{patterns/13_arrays/main}
\input{patterns/14_bitfields/main}
\EN{\input{patterns/145_LCG/main_EN}}
\RU{\input{patterns/145_LCG/main_RU}}
\input{patterns/15_structs/main}
\input{patterns/17_unions/main}
\input{patterns/18_pointers_to_functions/main}
\input{patterns/185_64bit_in_32_env/main}

\EN{\input{patterns/19_SIMD/main_EN}}
\RU{\input{patterns/19_SIMD/main_RU}}
\DE{\input{patterns/19_SIMD/main_DE}}

\EN{\input{patterns/20_x64/main_EN}}
\RU{\input{patterns/20_x64/main_RU}}

\EN{\input{patterns/205_floating_SIMD/main_EN}}
\RU{\input{patterns/205_floating_SIMD/main_RU}}
\DE{\input{patterns/205_floating_SIMD/main_DE}}

\EN{\input{patterns/ARM/main_EN}}
\RU{\input{patterns/ARM/main_RU}}
\DE{\input{patterns/ARM/main_DE}}

\input{patterns/MIPS/main}

\ifdefined\SPANISH
\chapter{Patrones de código}
\fi % SPANISH

\ifdefined\GERMAN
\chapter{Code-Muster}
\fi % GERMAN

\ifdefined\ENGLISH
\chapter{Code Patterns}
\fi % ENGLISH

\ifdefined\ITALIAN
\chapter{Forme di codice}
\fi % ITALIAN

\ifdefined\RUSSIAN
\chapter{Образцы кода}
\fi % RUSSIAN

\ifdefined\BRAZILIAN
\chapter{Padrões de códigos}
\fi % BRAZILIAN

\ifdefined\THAI
\chapter{รูปแบบของโค้ด}
\fi % THAI

\ifdefined\FRENCH
\chapter{Modèle de code}
\fi % FRENCH

\ifdefined\POLISH
\chapter{\PLph{}}
\fi % POLISH

% sections
\EN{\input{patterns/patterns_opt_dbg_EN}}
\ES{\input{patterns/patterns_opt_dbg_ES}}
\ITA{\input{patterns/patterns_opt_dbg_ITA}}
\PTBR{\input{patterns/patterns_opt_dbg_PTBR}}
\RU{\input{patterns/patterns_opt_dbg_RU}}
\THA{\input{patterns/patterns_opt_dbg_THA}}
\DE{\input{patterns/patterns_opt_dbg_DE}}
\FR{\input{patterns/patterns_opt_dbg_FR}}
\PL{\input{patterns/patterns_opt_dbg_PL}}

\RU{\section{Некоторые базовые понятия}}
\EN{\section{Some basics}}
\DE{\section{Einige Grundlagen}}
\FR{\section{Quelques bases}}
\ES{\section{\ESph{}}}
\ITA{\section{Alcune basi teoriche}}
\PTBR{\section{\PTBRph{}}}
\THA{\section{\THAph{}}}
\PL{\section{\PLph{}}}

% sections:
\EN{\input{patterns/intro_CPU_ISA_EN}}
\ES{\input{patterns/intro_CPU_ISA_ES}}
\ITA{\input{patterns/intro_CPU_ISA_ITA}}
\PTBR{\input{patterns/intro_CPU_ISA_PTBR}}
\RU{\input{patterns/intro_CPU_ISA_RU}}
\DE{\input{patterns/intro_CPU_ISA_DE}}
\FR{\input{patterns/intro_CPU_ISA_FR}}
\PL{\input{patterns/intro_CPU_ISA_PL}}

\EN{\input{patterns/numeral_EN}}
\RU{\input{patterns/numeral_RU}}
\ITA{\input{patterns/numeral_ITA}}
\DE{\input{patterns/numeral_DE}}
\FR{\input{patterns/numeral_FR}}
\PL{\input{patterns/numeral_PL}}

% chapters
\input{patterns/00_empty/main}
\input{patterns/011_ret/main}
\input{patterns/01_helloworld/main}
\input{patterns/015_prolog_epilogue/main}
\input{patterns/02_stack/main}
\input{patterns/03_printf/main}
\input{patterns/04_scanf/main}
\input{patterns/05_passing_arguments/main}
\input{patterns/06_return_results/main}
\input{patterns/061_pointers/main}
\input{patterns/065_GOTO/main}
\input{patterns/07_jcc/main}
\input{patterns/08_switch/main}
\input{patterns/09_loops/main}
\input{patterns/10_strings/main}
\input{patterns/11_arith_optimizations/main}
\input{patterns/12_FPU/main}
\input{patterns/13_arrays/main}
\input{patterns/14_bitfields/main}
\EN{\input{patterns/145_LCG/main_EN}}
\RU{\input{patterns/145_LCG/main_RU}}
\input{patterns/15_structs/main}
\input{patterns/17_unions/main}
\input{patterns/18_pointers_to_functions/main}
\input{patterns/185_64bit_in_32_env/main}

\EN{\input{patterns/19_SIMD/main_EN}}
\RU{\input{patterns/19_SIMD/main_RU}}
\DE{\input{patterns/19_SIMD/main_DE}}

\EN{\input{patterns/20_x64/main_EN}}
\RU{\input{patterns/20_x64/main_RU}}

\EN{\input{patterns/205_floating_SIMD/main_EN}}
\RU{\input{patterns/205_floating_SIMD/main_RU}}
\DE{\input{patterns/205_floating_SIMD/main_DE}}

\EN{\input{patterns/ARM/main_EN}}
\RU{\input{patterns/ARM/main_RU}}
\DE{\input{patterns/ARM/main_DE}}

\input{patterns/MIPS/main}

\ifdefined\SPANISH
\chapter{Patrones de código}
\fi % SPANISH

\ifdefined\GERMAN
\chapter{Code-Muster}
\fi % GERMAN

\ifdefined\ENGLISH
\chapter{Code Patterns}
\fi % ENGLISH

\ifdefined\ITALIAN
\chapter{Forme di codice}
\fi % ITALIAN

\ifdefined\RUSSIAN
\chapter{Образцы кода}
\fi % RUSSIAN

\ifdefined\BRAZILIAN
\chapter{Padrões de códigos}
\fi % BRAZILIAN

\ifdefined\THAI
\chapter{รูปแบบของโค้ด}
\fi % THAI

\ifdefined\FRENCH
\chapter{Modèle de code}
\fi % FRENCH

\ifdefined\POLISH
\chapter{\PLph{}}
\fi % POLISH

% sections
\EN{\input{patterns/patterns_opt_dbg_EN}}
\ES{\input{patterns/patterns_opt_dbg_ES}}
\ITA{\input{patterns/patterns_opt_dbg_ITA}}
\PTBR{\input{patterns/patterns_opt_dbg_PTBR}}
\RU{\input{patterns/patterns_opt_dbg_RU}}
\THA{\input{patterns/patterns_opt_dbg_THA}}
\DE{\input{patterns/patterns_opt_dbg_DE}}
\FR{\input{patterns/patterns_opt_dbg_FR}}
\PL{\input{patterns/patterns_opt_dbg_PL}}

\RU{\section{Некоторые базовые понятия}}
\EN{\section{Some basics}}
\DE{\section{Einige Grundlagen}}
\FR{\section{Quelques bases}}
\ES{\section{\ESph{}}}
\ITA{\section{Alcune basi teoriche}}
\PTBR{\section{\PTBRph{}}}
\THA{\section{\THAph{}}}
\PL{\section{\PLph{}}}

% sections:
\EN{\input{patterns/intro_CPU_ISA_EN}}
\ES{\input{patterns/intro_CPU_ISA_ES}}
\ITA{\input{patterns/intro_CPU_ISA_ITA}}
\PTBR{\input{patterns/intro_CPU_ISA_PTBR}}
\RU{\input{patterns/intro_CPU_ISA_RU}}
\DE{\input{patterns/intro_CPU_ISA_DE}}
\FR{\input{patterns/intro_CPU_ISA_FR}}
\PL{\input{patterns/intro_CPU_ISA_PL}}

\EN{\input{patterns/numeral_EN}}
\RU{\input{patterns/numeral_RU}}
\ITA{\input{patterns/numeral_ITA}}
\DE{\input{patterns/numeral_DE}}
\FR{\input{patterns/numeral_FR}}
\PL{\input{patterns/numeral_PL}}

% chapters
\input{patterns/00_empty/main}
\input{patterns/011_ret/main}
\input{patterns/01_helloworld/main}
\input{patterns/015_prolog_epilogue/main}
\input{patterns/02_stack/main}
\input{patterns/03_printf/main}
\input{patterns/04_scanf/main}
\input{patterns/05_passing_arguments/main}
\input{patterns/06_return_results/main}
\input{patterns/061_pointers/main}
\input{patterns/065_GOTO/main}
\input{patterns/07_jcc/main}
\input{patterns/08_switch/main}
\input{patterns/09_loops/main}
\input{patterns/10_strings/main}
\input{patterns/11_arith_optimizations/main}
\input{patterns/12_FPU/main}
\input{patterns/13_arrays/main}
\input{patterns/14_bitfields/main}
\EN{\input{patterns/145_LCG/main_EN}}
\RU{\input{patterns/145_LCG/main_RU}}
\input{patterns/15_structs/main}
\input{patterns/17_unions/main}
\input{patterns/18_pointers_to_functions/main}
\input{patterns/185_64bit_in_32_env/main}

\EN{\input{patterns/19_SIMD/main_EN}}
\RU{\input{patterns/19_SIMD/main_RU}}
\DE{\input{patterns/19_SIMD/main_DE}}

\EN{\input{patterns/20_x64/main_EN}}
\RU{\input{patterns/20_x64/main_RU}}

\EN{\input{patterns/205_floating_SIMD/main_EN}}
\RU{\input{patterns/205_floating_SIMD/main_RU}}
\DE{\input{patterns/205_floating_SIMD/main_DE}}

\EN{\input{patterns/ARM/main_EN}}
\RU{\input{patterns/ARM/main_RU}}
\DE{\input{patterns/ARM/main_DE}}

\input{patterns/MIPS/main}

\ifdefined\SPANISH
\chapter{Patrones de código}
\fi % SPANISH

\ifdefined\GERMAN
\chapter{Code-Muster}
\fi % GERMAN

\ifdefined\ENGLISH
\chapter{Code Patterns}
\fi % ENGLISH

\ifdefined\ITALIAN
\chapter{Forme di codice}
\fi % ITALIAN

\ifdefined\RUSSIAN
\chapter{Образцы кода}
\fi % RUSSIAN

\ifdefined\BRAZILIAN
\chapter{Padrões de códigos}
\fi % BRAZILIAN

\ifdefined\THAI
\chapter{รูปแบบของโค้ด}
\fi % THAI

\ifdefined\FRENCH
\chapter{Modèle de code}
\fi % FRENCH

\ifdefined\POLISH
\chapter{\PLph{}}
\fi % POLISH

% sections
\EN{\input{patterns/patterns_opt_dbg_EN}}
\ES{\input{patterns/patterns_opt_dbg_ES}}
\ITA{\input{patterns/patterns_opt_dbg_ITA}}
\PTBR{\input{patterns/patterns_opt_dbg_PTBR}}
\RU{\input{patterns/patterns_opt_dbg_RU}}
\THA{\input{patterns/patterns_opt_dbg_THA}}
\DE{\input{patterns/patterns_opt_dbg_DE}}
\FR{\input{patterns/patterns_opt_dbg_FR}}
\PL{\input{patterns/patterns_opt_dbg_PL}}

\RU{\section{Некоторые базовые понятия}}
\EN{\section{Some basics}}
\DE{\section{Einige Grundlagen}}
\FR{\section{Quelques bases}}
\ES{\section{\ESph{}}}
\ITA{\section{Alcune basi teoriche}}
\PTBR{\section{\PTBRph{}}}
\THA{\section{\THAph{}}}
\PL{\section{\PLph{}}}

% sections:
\EN{\input{patterns/intro_CPU_ISA_EN}}
\ES{\input{patterns/intro_CPU_ISA_ES}}
\ITA{\input{patterns/intro_CPU_ISA_ITA}}
\PTBR{\input{patterns/intro_CPU_ISA_PTBR}}
\RU{\input{patterns/intro_CPU_ISA_RU}}
\DE{\input{patterns/intro_CPU_ISA_DE}}
\FR{\input{patterns/intro_CPU_ISA_FR}}
\PL{\input{patterns/intro_CPU_ISA_PL}}

\EN{\input{patterns/numeral_EN}}
\RU{\input{patterns/numeral_RU}}
\ITA{\input{patterns/numeral_ITA}}
\DE{\input{patterns/numeral_DE}}
\FR{\input{patterns/numeral_FR}}
\PL{\input{patterns/numeral_PL}}

% chapters
\input{patterns/00_empty/main}
\input{patterns/011_ret/main}
\input{patterns/01_helloworld/main}
\input{patterns/015_prolog_epilogue/main}
\input{patterns/02_stack/main}
\input{patterns/03_printf/main}
\input{patterns/04_scanf/main}
\input{patterns/05_passing_arguments/main}
\input{patterns/06_return_results/main}
\input{patterns/061_pointers/main}
\input{patterns/065_GOTO/main}
\input{patterns/07_jcc/main}
\input{patterns/08_switch/main}
\input{patterns/09_loops/main}
\input{patterns/10_strings/main}
\input{patterns/11_arith_optimizations/main}
\input{patterns/12_FPU/main}
\input{patterns/13_arrays/main}
\input{patterns/14_bitfields/main}
\EN{\input{patterns/145_LCG/main_EN}}
\RU{\input{patterns/145_LCG/main_RU}}
\input{patterns/15_structs/main}
\input{patterns/17_unions/main}
\input{patterns/18_pointers_to_functions/main}
\input{patterns/185_64bit_in_32_env/main}

\EN{\input{patterns/19_SIMD/main_EN}}
\RU{\input{patterns/19_SIMD/main_RU}}
\DE{\input{patterns/19_SIMD/main_DE}}

\EN{\input{patterns/20_x64/main_EN}}
\RU{\input{patterns/20_x64/main_RU}}

\EN{\input{patterns/205_floating_SIMD/main_EN}}
\RU{\input{patterns/205_floating_SIMD/main_RU}}
\DE{\input{patterns/205_floating_SIMD/main_DE}}

\EN{\input{patterns/ARM/main_EN}}
\RU{\input{patterns/ARM/main_RU}}
\DE{\input{patterns/ARM/main_DE}}

\input{patterns/MIPS/main}

\ifdefined\SPANISH
\chapter{Patrones de código}
\fi % SPANISH

\ifdefined\GERMAN
\chapter{Code-Muster}
\fi % GERMAN

\ifdefined\ENGLISH
\chapter{Code Patterns}
\fi % ENGLISH

\ifdefined\ITALIAN
\chapter{Forme di codice}
\fi % ITALIAN

\ifdefined\RUSSIAN
\chapter{Образцы кода}
\fi % RUSSIAN

\ifdefined\BRAZILIAN
\chapter{Padrões de códigos}
\fi % BRAZILIAN

\ifdefined\THAI
\chapter{รูปแบบของโค้ด}
\fi % THAI

\ifdefined\FRENCH
\chapter{Modèle de code}
\fi % FRENCH

\ifdefined\POLISH
\chapter{\PLph{}}
\fi % POLISH

% sections
\EN{\input{patterns/patterns_opt_dbg_EN}}
\ES{\input{patterns/patterns_opt_dbg_ES}}
\ITA{\input{patterns/patterns_opt_dbg_ITA}}
\PTBR{\input{patterns/patterns_opt_dbg_PTBR}}
\RU{\input{patterns/patterns_opt_dbg_RU}}
\THA{\input{patterns/patterns_opt_dbg_THA}}
\DE{\input{patterns/patterns_opt_dbg_DE}}
\FR{\input{patterns/patterns_opt_dbg_FR}}
\PL{\input{patterns/patterns_opt_dbg_PL}}

\RU{\section{Некоторые базовые понятия}}
\EN{\section{Some basics}}
\DE{\section{Einige Grundlagen}}
\FR{\section{Quelques bases}}
\ES{\section{\ESph{}}}
\ITA{\section{Alcune basi teoriche}}
\PTBR{\section{\PTBRph{}}}
\THA{\section{\THAph{}}}
\PL{\section{\PLph{}}}

% sections:
\EN{\input{patterns/intro_CPU_ISA_EN}}
\ES{\input{patterns/intro_CPU_ISA_ES}}
\ITA{\input{patterns/intro_CPU_ISA_ITA}}
\PTBR{\input{patterns/intro_CPU_ISA_PTBR}}
\RU{\input{patterns/intro_CPU_ISA_RU}}
\DE{\input{patterns/intro_CPU_ISA_DE}}
\FR{\input{patterns/intro_CPU_ISA_FR}}
\PL{\input{patterns/intro_CPU_ISA_PL}}

\EN{\input{patterns/numeral_EN}}
\RU{\input{patterns/numeral_RU}}
\ITA{\input{patterns/numeral_ITA}}
\DE{\input{patterns/numeral_DE}}
\FR{\input{patterns/numeral_FR}}
\PL{\input{patterns/numeral_PL}}

% chapters
\input{patterns/00_empty/main}
\input{patterns/011_ret/main}
\input{patterns/01_helloworld/main}
\input{patterns/015_prolog_epilogue/main}
\input{patterns/02_stack/main}
\input{patterns/03_printf/main}
\input{patterns/04_scanf/main}
\input{patterns/05_passing_arguments/main}
\input{patterns/06_return_results/main}
\input{patterns/061_pointers/main}
\input{patterns/065_GOTO/main}
\input{patterns/07_jcc/main}
\input{patterns/08_switch/main}
\input{patterns/09_loops/main}
\input{patterns/10_strings/main}
\input{patterns/11_arith_optimizations/main}
\input{patterns/12_FPU/main}
\input{patterns/13_arrays/main}
\input{patterns/14_bitfields/main}
\EN{\input{patterns/145_LCG/main_EN}}
\RU{\input{patterns/145_LCG/main_RU}}
\input{patterns/15_structs/main}
\input{patterns/17_unions/main}
\input{patterns/18_pointers_to_functions/main}
\input{patterns/185_64bit_in_32_env/main}

\EN{\input{patterns/19_SIMD/main_EN}}
\RU{\input{patterns/19_SIMD/main_RU}}
\DE{\input{patterns/19_SIMD/main_DE}}

\EN{\input{patterns/20_x64/main_EN}}
\RU{\input{patterns/20_x64/main_RU}}

\EN{\input{patterns/205_floating_SIMD/main_EN}}
\RU{\input{patterns/205_floating_SIMD/main_RU}}
\DE{\input{patterns/205_floating_SIMD/main_DE}}

\EN{\input{patterns/ARM/main_EN}}
\RU{\input{patterns/ARM/main_RU}}
\DE{\input{patterns/ARM/main_DE}}

\input{patterns/MIPS/main}

\ifdefined\SPANISH
\chapter{Patrones de código}
\fi % SPANISH

\ifdefined\GERMAN
\chapter{Code-Muster}
\fi % GERMAN

\ifdefined\ENGLISH
\chapter{Code Patterns}
\fi % ENGLISH

\ifdefined\ITALIAN
\chapter{Forme di codice}
\fi % ITALIAN

\ifdefined\RUSSIAN
\chapter{Образцы кода}
\fi % RUSSIAN

\ifdefined\BRAZILIAN
\chapter{Padrões de códigos}
\fi % BRAZILIAN

\ifdefined\THAI
\chapter{รูปแบบของโค้ด}
\fi % THAI

\ifdefined\FRENCH
\chapter{Modèle de code}
\fi % FRENCH

\ifdefined\POLISH
\chapter{\PLph{}}
\fi % POLISH

% sections
\EN{\input{patterns/patterns_opt_dbg_EN}}
\ES{\input{patterns/patterns_opt_dbg_ES}}
\ITA{\input{patterns/patterns_opt_dbg_ITA}}
\PTBR{\input{patterns/patterns_opt_dbg_PTBR}}
\RU{\input{patterns/patterns_opt_dbg_RU}}
\THA{\input{patterns/patterns_opt_dbg_THA}}
\DE{\input{patterns/patterns_opt_dbg_DE}}
\FR{\input{patterns/patterns_opt_dbg_FR}}
\PL{\input{patterns/patterns_opt_dbg_PL}}

\RU{\section{Некоторые базовые понятия}}
\EN{\section{Some basics}}
\DE{\section{Einige Grundlagen}}
\FR{\section{Quelques bases}}
\ES{\section{\ESph{}}}
\ITA{\section{Alcune basi teoriche}}
\PTBR{\section{\PTBRph{}}}
\THA{\section{\THAph{}}}
\PL{\section{\PLph{}}}

% sections:
\EN{\input{patterns/intro_CPU_ISA_EN}}
\ES{\input{patterns/intro_CPU_ISA_ES}}
\ITA{\input{patterns/intro_CPU_ISA_ITA}}
\PTBR{\input{patterns/intro_CPU_ISA_PTBR}}
\RU{\input{patterns/intro_CPU_ISA_RU}}
\DE{\input{patterns/intro_CPU_ISA_DE}}
\FR{\input{patterns/intro_CPU_ISA_FR}}
\PL{\input{patterns/intro_CPU_ISA_PL}}

\EN{\input{patterns/numeral_EN}}
\RU{\input{patterns/numeral_RU}}
\ITA{\input{patterns/numeral_ITA}}
\DE{\input{patterns/numeral_DE}}
\FR{\input{patterns/numeral_FR}}
\PL{\input{patterns/numeral_PL}}

% chapters
\input{patterns/00_empty/main}
\input{patterns/011_ret/main}
\input{patterns/01_helloworld/main}
\input{patterns/015_prolog_epilogue/main}
\input{patterns/02_stack/main}
\input{patterns/03_printf/main}
\input{patterns/04_scanf/main}
\input{patterns/05_passing_arguments/main}
\input{patterns/06_return_results/main}
\input{patterns/061_pointers/main}
\input{patterns/065_GOTO/main}
\input{patterns/07_jcc/main}
\input{patterns/08_switch/main}
\input{patterns/09_loops/main}
\input{patterns/10_strings/main}
\input{patterns/11_arith_optimizations/main}
\input{patterns/12_FPU/main}
\input{patterns/13_arrays/main}
\input{patterns/14_bitfields/main}
\EN{\input{patterns/145_LCG/main_EN}}
\RU{\input{patterns/145_LCG/main_RU}}
\input{patterns/15_structs/main}
\input{patterns/17_unions/main}
\input{patterns/18_pointers_to_functions/main}
\input{patterns/185_64bit_in_32_env/main}

\EN{\input{patterns/19_SIMD/main_EN}}
\RU{\input{patterns/19_SIMD/main_RU}}
\DE{\input{patterns/19_SIMD/main_DE}}

\EN{\input{patterns/20_x64/main_EN}}
\RU{\input{patterns/20_x64/main_RU}}

\EN{\input{patterns/205_floating_SIMD/main_EN}}
\RU{\input{patterns/205_floating_SIMD/main_RU}}
\DE{\input{patterns/205_floating_SIMD/main_DE}}

\EN{\input{patterns/ARM/main_EN}}
\RU{\input{patterns/ARM/main_RU}}
\DE{\input{patterns/ARM/main_DE}}

\input{patterns/MIPS/main}

\ifdefined\SPANISH
\chapter{Patrones de código}
\fi % SPANISH

\ifdefined\GERMAN
\chapter{Code-Muster}
\fi % GERMAN

\ifdefined\ENGLISH
\chapter{Code Patterns}
\fi % ENGLISH

\ifdefined\ITALIAN
\chapter{Forme di codice}
\fi % ITALIAN

\ifdefined\RUSSIAN
\chapter{Образцы кода}
\fi % RUSSIAN

\ifdefined\BRAZILIAN
\chapter{Padrões de códigos}
\fi % BRAZILIAN

\ifdefined\THAI
\chapter{รูปแบบของโค้ด}
\fi % THAI

\ifdefined\FRENCH
\chapter{Modèle de code}
\fi % FRENCH

\ifdefined\POLISH
\chapter{\PLph{}}
\fi % POLISH

% sections
\EN{\input{patterns/patterns_opt_dbg_EN}}
\ES{\input{patterns/patterns_opt_dbg_ES}}
\ITA{\input{patterns/patterns_opt_dbg_ITA}}
\PTBR{\input{patterns/patterns_opt_dbg_PTBR}}
\RU{\input{patterns/patterns_opt_dbg_RU}}
\THA{\input{patterns/patterns_opt_dbg_THA}}
\DE{\input{patterns/patterns_opt_dbg_DE}}
\FR{\input{patterns/patterns_opt_dbg_FR}}
\PL{\input{patterns/patterns_opt_dbg_PL}}

\RU{\section{Некоторые базовые понятия}}
\EN{\section{Some basics}}
\DE{\section{Einige Grundlagen}}
\FR{\section{Quelques bases}}
\ES{\section{\ESph{}}}
\ITA{\section{Alcune basi teoriche}}
\PTBR{\section{\PTBRph{}}}
\THA{\section{\THAph{}}}
\PL{\section{\PLph{}}}

% sections:
\EN{\input{patterns/intro_CPU_ISA_EN}}
\ES{\input{patterns/intro_CPU_ISA_ES}}
\ITA{\input{patterns/intro_CPU_ISA_ITA}}
\PTBR{\input{patterns/intro_CPU_ISA_PTBR}}
\RU{\input{patterns/intro_CPU_ISA_RU}}
\DE{\input{patterns/intro_CPU_ISA_DE}}
\FR{\input{patterns/intro_CPU_ISA_FR}}
\PL{\input{patterns/intro_CPU_ISA_PL}}

\EN{\input{patterns/numeral_EN}}
\RU{\input{patterns/numeral_RU}}
\ITA{\input{patterns/numeral_ITA}}
\DE{\input{patterns/numeral_DE}}
\FR{\input{patterns/numeral_FR}}
\PL{\input{patterns/numeral_PL}}

% chapters
\input{patterns/00_empty/main}
\input{patterns/011_ret/main}
\input{patterns/01_helloworld/main}
\input{patterns/015_prolog_epilogue/main}
\input{patterns/02_stack/main}
\input{patterns/03_printf/main}
\input{patterns/04_scanf/main}
\input{patterns/05_passing_arguments/main}
\input{patterns/06_return_results/main}
\input{patterns/061_pointers/main}
\input{patterns/065_GOTO/main}
\input{patterns/07_jcc/main}
\input{patterns/08_switch/main}
\input{patterns/09_loops/main}
\input{patterns/10_strings/main}
\input{patterns/11_arith_optimizations/main}
\input{patterns/12_FPU/main}
\input{patterns/13_arrays/main}
\input{patterns/14_bitfields/main}
\EN{\input{patterns/145_LCG/main_EN}}
\RU{\input{patterns/145_LCG/main_RU}}
\input{patterns/15_structs/main}
\input{patterns/17_unions/main}
\input{patterns/18_pointers_to_functions/main}
\input{patterns/185_64bit_in_32_env/main}

\EN{\input{patterns/19_SIMD/main_EN}}
\RU{\input{patterns/19_SIMD/main_RU}}
\DE{\input{patterns/19_SIMD/main_DE}}

\EN{\input{patterns/20_x64/main_EN}}
\RU{\input{patterns/20_x64/main_RU}}

\EN{\input{patterns/205_floating_SIMD/main_EN}}
\RU{\input{patterns/205_floating_SIMD/main_RU}}
\DE{\input{patterns/205_floating_SIMD/main_DE}}

\EN{\input{patterns/ARM/main_EN}}
\RU{\input{patterns/ARM/main_RU}}
\DE{\input{patterns/ARM/main_DE}}

\input{patterns/MIPS/main}

\ifdefined\SPANISH
\chapter{Patrones de código}
\fi % SPANISH

\ifdefined\GERMAN
\chapter{Code-Muster}
\fi % GERMAN

\ifdefined\ENGLISH
\chapter{Code Patterns}
\fi % ENGLISH

\ifdefined\ITALIAN
\chapter{Forme di codice}
\fi % ITALIAN

\ifdefined\RUSSIAN
\chapter{Образцы кода}
\fi % RUSSIAN

\ifdefined\BRAZILIAN
\chapter{Padrões de códigos}
\fi % BRAZILIAN

\ifdefined\THAI
\chapter{รูปแบบของโค้ด}
\fi % THAI

\ifdefined\FRENCH
\chapter{Modèle de code}
\fi % FRENCH

\ifdefined\POLISH
\chapter{\PLph{}}
\fi % POLISH

% sections
\EN{\input{patterns/patterns_opt_dbg_EN}}
\ES{\input{patterns/patterns_opt_dbg_ES}}
\ITA{\input{patterns/patterns_opt_dbg_ITA}}
\PTBR{\input{patterns/patterns_opt_dbg_PTBR}}
\RU{\input{patterns/patterns_opt_dbg_RU}}
\THA{\input{patterns/patterns_opt_dbg_THA}}
\DE{\input{patterns/patterns_opt_dbg_DE}}
\FR{\input{patterns/patterns_opt_dbg_FR}}
\PL{\input{patterns/patterns_opt_dbg_PL}}

\RU{\section{Некоторые базовые понятия}}
\EN{\section{Some basics}}
\DE{\section{Einige Grundlagen}}
\FR{\section{Quelques bases}}
\ES{\section{\ESph{}}}
\ITA{\section{Alcune basi teoriche}}
\PTBR{\section{\PTBRph{}}}
\THA{\section{\THAph{}}}
\PL{\section{\PLph{}}}

% sections:
\EN{\input{patterns/intro_CPU_ISA_EN}}
\ES{\input{patterns/intro_CPU_ISA_ES}}
\ITA{\input{patterns/intro_CPU_ISA_ITA}}
\PTBR{\input{patterns/intro_CPU_ISA_PTBR}}
\RU{\input{patterns/intro_CPU_ISA_RU}}
\DE{\input{patterns/intro_CPU_ISA_DE}}
\FR{\input{patterns/intro_CPU_ISA_FR}}
\PL{\input{patterns/intro_CPU_ISA_PL}}

\EN{\input{patterns/numeral_EN}}
\RU{\input{patterns/numeral_RU}}
\ITA{\input{patterns/numeral_ITA}}
\DE{\input{patterns/numeral_DE}}
\FR{\input{patterns/numeral_FR}}
\PL{\input{patterns/numeral_PL}}

% chapters
\input{patterns/00_empty/main}
\input{patterns/011_ret/main}
\input{patterns/01_helloworld/main}
\input{patterns/015_prolog_epilogue/main}
\input{patterns/02_stack/main}
\input{patterns/03_printf/main}
\input{patterns/04_scanf/main}
\input{patterns/05_passing_arguments/main}
\input{patterns/06_return_results/main}
\input{patterns/061_pointers/main}
\input{patterns/065_GOTO/main}
\input{patterns/07_jcc/main}
\input{patterns/08_switch/main}
\input{patterns/09_loops/main}
\input{patterns/10_strings/main}
\input{patterns/11_arith_optimizations/main}
\input{patterns/12_FPU/main}
\input{patterns/13_arrays/main}
\input{patterns/14_bitfields/main}
\EN{\input{patterns/145_LCG/main_EN}}
\RU{\input{patterns/145_LCG/main_RU}}
\input{patterns/15_structs/main}
\input{patterns/17_unions/main}
\input{patterns/18_pointers_to_functions/main}
\input{patterns/185_64bit_in_32_env/main}

\EN{\input{patterns/19_SIMD/main_EN}}
\RU{\input{patterns/19_SIMD/main_RU}}
\DE{\input{patterns/19_SIMD/main_DE}}

\EN{\input{patterns/20_x64/main_EN}}
\RU{\input{patterns/20_x64/main_RU}}

\EN{\input{patterns/205_floating_SIMD/main_EN}}
\RU{\input{patterns/205_floating_SIMD/main_RU}}
\DE{\input{patterns/205_floating_SIMD/main_DE}}

\EN{\input{patterns/ARM/main_EN}}
\RU{\input{patterns/ARM/main_RU}}
\DE{\input{patterns/ARM/main_DE}}

\input{patterns/MIPS/main}

\ifdefined\SPANISH
\chapter{Patrones de código}
\fi % SPANISH

\ifdefined\GERMAN
\chapter{Code-Muster}
\fi % GERMAN

\ifdefined\ENGLISH
\chapter{Code Patterns}
\fi % ENGLISH

\ifdefined\ITALIAN
\chapter{Forme di codice}
\fi % ITALIAN

\ifdefined\RUSSIAN
\chapter{Образцы кода}
\fi % RUSSIAN

\ifdefined\BRAZILIAN
\chapter{Padrões de códigos}
\fi % BRAZILIAN

\ifdefined\THAI
\chapter{รูปแบบของโค้ด}
\fi % THAI

\ifdefined\FRENCH
\chapter{Modèle de code}
\fi % FRENCH

\ifdefined\POLISH
\chapter{\PLph{}}
\fi % POLISH

% sections
\EN{\input{patterns/patterns_opt_dbg_EN}}
\ES{\input{patterns/patterns_opt_dbg_ES}}
\ITA{\input{patterns/patterns_opt_dbg_ITA}}
\PTBR{\input{patterns/patterns_opt_dbg_PTBR}}
\RU{\input{patterns/patterns_opt_dbg_RU}}
\THA{\input{patterns/patterns_opt_dbg_THA}}
\DE{\input{patterns/patterns_opt_dbg_DE}}
\FR{\input{patterns/patterns_opt_dbg_FR}}
\PL{\input{patterns/patterns_opt_dbg_PL}}

\RU{\section{Некоторые базовые понятия}}
\EN{\section{Some basics}}
\DE{\section{Einige Grundlagen}}
\FR{\section{Quelques bases}}
\ES{\section{\ESph{}}}
\ITA{\section{Alcune basi teoriche}}
\PTBR{\section{\PTBRph{}}}
\THA{\section{\THAph{}}}
\PL{\section{\PLph{}}}

% sections:
\EN{\input{patterns/intro_CPU_ISA_EN}}
\ES{\input{patterns/intro_CPU_ISA_ES}}
\ITA{\input{patterns/intro_CPU_ISA_ITA}}
\PTBR{\input{patterns/intro_CPU_ISA_PTBR}}
\RU{\input{patterns/intro_CPU_ISA_RU}}
\DE{\input{patterns/intro_CPU_ISA_DE}}
\FR{\input{patterns/intro_CPU_ISA_FR}}
\PL{\input{patterns/intro_CPU_ISA_PL}}

\EN{\input{patterns/numeral_EN}}
\RU{\input{patterns/numeral_RU}}
\ITA{\input{patterns/numeral_ITA}}
\DE{\input{patterns/numeral_DE}}
\FR{\input{patterns/numeral_FR}}
\PL{\input{patterns/numeral_PL}}

% chapters
\input{patterns/00_empty/main}
\input{patterns/011_ret/main}
\input{patterns/01_helloworld/main}
\input{patterns/015_prolog_epilogue/main}
\input{patterns/02_stack/main}
\input{patterns/03_printf/main}
\input{patterns/04_scanf/main}
\input{patterns/05_passing_arguments/main}
\input{patterns/06_return_results/main}
\input{patterns/061_pointers/main}
\input{patterns/065_GOTO/main}
\input{patterns/07_jcc/main}
\input{patterns/08_switch/main}
\input{patterns/09_loops/main}
\input{patterns/10_strings/main}
\input{patterns/11_arith_optimizations/main}
\input{patterns/12_FPU/main}
\input{patterns/13_arrays/main}
\input{patterns/14_bitfields/main}
\EN{\input{patterns/145_LCG/main_EN}}
\RU{\input{patterns/145_LCG/main_RU}}
\input{patterns/15_structs/main}
\input{patterns/17_unions/main}
\input{patterns/18_pointers_to_functions/main}
\input{patterns/185_64bit_in_32_env/main}

\EN{\input{patterns/19_SIMD/main_EN}}
\RU{\input{patterns/19_SIMD/main_RU}}
\DE{\input{patterns/19_SIMD/main_DE}}

\EN{\input{patterns/20_x64/main_EN}}
\RU{\input{patterns/20_x64/main_RU}}

\EN{\input{patterns/205_floating_SIMD/main_EN}}
\RU{\input{patterns/205_floating_SIMD/main_RU}}
\DE{\input{patterns/205_floating_SIMD/main_DE}}

\EN{\input{patterns/ARM/main_EN}}
\RU{\input{patterns/ARM/main_RU}}
\DE{\input{patterns/ARM/main_DE}}

\input{patterns/MIPS/main}

\ifdefined\SPANISH
\chapter{Patrones de código}
\fi % SPANISH

\ifdefined\GERMAN
\chapter{Code-Muster}
\fi % GERMAN

\ifdefined\ENGLISH
\chapter{Code Patterns}
\fi % ENGLISH

\ifdefined\ITALIAN
\chapter{Forme di codice}
\fi % ITALIAN

\ifdefined\RUSSIAN
\chapter{Образцы кода}
\fi % RUSSIAN

\ifdefined\BRAZILIAN
\chapter{Padrões de códigos}
\fi % BRAZILIAN

\ifdefined\THAI
\chapter{รูปแบบของโค้ด}
\fi % THAI

\ifdefined\FRENCH
\chapter{Modèle de code}
\fi % FRENCH

\ifdefined\POLISH
\chapter{\PLph{}}
\fi % POLISH

% sections
\EN{\input{patterns/patterns_opt_dbg_EN}}
\ES{\input{patterns/patterns_opt_dbg_ES}}
\ITA{\input{patterns/patterns_opt_dbg_ITA}}
\PTBR{\input{patterns/patterns_opt_dbg_PTBR}}
\RU{\input{patterns/patterns_opt_dbg_RU}}
\THA{\input{patterns/patterns_opt_dbg_THA}}
\DE{\input{patterns/patterns_opt_dbg_DE}}
\FR{\input{patterns/patterns_opt_dbg_FR}}
\PL{\input{patterns/patterns_opt_dbg_PL}}

\RU{\section{Некоторые базовые понятия}}
\EN{\section{Some basics}}
\DE{\section{Einige Grundlagen}}
\FR{\section{Quelques bases}}
\ES{\section{\ESph{}}}
\ITA{\section{Alcune basi teoriche}}
\PTBR{\section{\PTBRph{}}}
\THA{\section{\THAph{}}}
\PL{\section{\PLph{}}}

% sections:
\EN{\input{patterns/intro_CPU_ISA_EN}}
\ES{\input{patterns/intro_CPU_ISA_ES}}
\ITA{\input{patterns/intro_CPU_ISA_ITA}}
\PTBR{\input{patterns/intro_CPU_ISA_PTBR}}
\RU{\input{patterns/intro_CPU_ISA_RU}}
\DE{\input{patterns/intro_CPU_ISA_DE}}
\FR{\input{patterns/intro_CPU_ISA_FR}}
\PL{\input{patterns/intro_CPU_ISA_PL}}

\EN{\input{patterns/numeral_EN}}
\RU{\input{patterns/numeral_RU}}
\ITA{\input{patterns/numeral_ITA}}
\DE{\input{patterns/numeral_DE}}
\FR{\input{patterns/numeral_FR}}
\PL{\input{patterns/numeral_PL}}

% chapters
\input{patterns/00_empty/main}
\input{patterns/011_ret/main}
\input{patterns/01_helloworld/main}
\input{patterns/015_prolog_epilogue/main}
\input{patterns/02_stack/main}
\input{patterns/03_printf/main}
\input{patterns/04_scanf/main}
\input{patterns/05_passing_arguments/main}
\input{patterns/06_return_results/main}
\input{patterns/061_pointers/main}
\input{patterns/065_GOTO/main}
\input{patterns/07_jcc/main}
\input{patterns/08_switch/main}
\input{patterns/09_loops/main}
\input{patterns/10_strings/main}
\input{patterns/11_arith_optimizations/main}
\input{patterns/12_FPU/main}
\input{patterns/13_arrays/main}
\input{patterns/14_bitfields/main}
\EN{\input{patterns/145_LCG/main_EN}}
\RU{\input{patterns/145_LCG/main_RU}}
\input{patterns/15_structs/main}
\input{patterns/17_unions/main}
\input{patterns/18_pointers_to_functions/main}
\input{patterns/185_64bit_in_32_env/main}

\EN{\input{patterns/19_SIMD/main_EN}}
\RU{\input{patterns/19_SIMD/main_RU}}
\DE{\input{patterns/19_SIMD/main_DE}}

\EN{\input{patterns/20_x64/main_EN}}
\RU{\input{patterns/20_x64/main_RU}}

\EN{\input{patterns/205_floating_SIMD/main_EN}}
\RU{\input{patterns/205_floating_SIMD/main_RU}}
\DE{\input{patterns/205_floating_SIMD/main_DE}}

\EN{\input{patterns/ARM/main_EN}}
\RU{\input{patterns/ARM/main_RU}}
\DE{\input{patterns/ARM/main_DE}}

\input{patterns/MIPS/main}

\EN{\input{patterns/12_FPU/main_EN}}
\RU{\input{patterns/12_FPU/main_RU}}
\DE{\input{patterns/12_FPU/main_DE}}
\FR{\input{patterns/12_FPU/main_FR}}


\ifdefined\SPANISH
\chapter{Patrones de código}
\fi % SPANISH

\ifdefined\GERMAN
\chapter{Code-Muster}
\fi % GERMAN

\ifdefined\ENGLISH
\chapter{Code Patterns}
\fi % ENGLISH

\ifdefined\ITALIAN
\chapter{Forme di codice}
\fi % ITALIAN

\ifdefined\RUSSIAN
\chapter{Образцы кода}
\fi % RUSSIAN

\ifdefined\BRAZILIAN
\chapter{Padrões de códigos}
\fi % BRAZILIAN

\ifdefined\THAI
\chapter{รูปแบบของโค้ด}
\fi % THAI

\ifdefined\FRENCH
\chapter{Modèle de code}
\fi % FRENCH

\ifdefined\POLISH
\chapter{\PLph{}}
\fi % POLISH

% sections
\EN{\input{patterns/patterns_opt_dbg_EN}}
\ES{\input{patterns/patterns_opt_dbg_ES}}
\ITA{\input{patterns/patterns_opt_dbg_ITA}}
\PTBR{\input{patterns/patterns_opt_dbg_PTBR}}
\RU{\input{patterns/patterns_opt_dbg_RU}}
\THA{\input{patterns/patterns_opt_dbg_THA}}
\DE{\input{patterns/patterns_opt_dbg_DE}}
\FR{\input{patterns/patterns_opt_dbg_FR}}
\PL{\input{patterns/patterns_opt_dbg_PL}}

\RU{\section{Некоторые базовые понятия}}
\EN{\section{Some basics}}
\DE{\section{Einige Grundlagen}}
\FR{\section{Quelques bases}}
\ES{\section{\ESph{}}}
\ITA{\section{Alcune basi teoriche}}
\PTBR{\section{\PTBRph{}}}
\THA{\section{\THAph{}}}
\PL{\section{\PLph{}}}

% sections:
\EN{\input{patterns/intro_CPU_ISA_EN}}
\ES{\input{patterns/intro_CPU_ISA_ES}}
\ITA{\input{patterns/intro_CPU_ISA_ITA}}
\PTBR{\input{patterns/intro_CPU_ISA_PTBR}}
\RU{\input{patterns/intro_CPU_ISA_RU}}
\DE{\input{patterns/intro_CPU_ISA_DE}}
\FR{\input{patterns/intro_CPU_ISA_FR}}
\PL{\input{patterns/intro_CPU_ISA_PL}}

\EN{\input{patterns/numeral_EN}}
\RU{\input{patterns/numeral_RU}}
\ITA{\input{patterns/numeral_ITA}}
\DE{\input{patterns/numeral_DE}}
\FR{\input{patterns/numeral_FR}}
\PL{\input{patterns/numeral_PL}}

% chapters
\input{patterns/00_empty/main}
\input{patterns/011_ret/main}
\input{patterns/01_helloworld/main}
\input{patterns/015_prolog_epilogue/main}
\input{patterns/02_stack/main}
\input{patterns/03_printf/main}
\input{patterns/04_scanf/main}
\input{patterns/05_passing_arguments/main}
\input{patterns/06_return_results/main}
\input{patterns/061_pointers/main}
\input{patterns/065_GOTO/main}
\input{patterns/07_jcc/main}
\input{patterns/08_switch/main}
\input{patterns/09_loops/main}
\input{patterns/10_strings/main}
\input{patterns/11_arith_optimizations/main}
\input{patterns/12_FPU/main}
\input{patterns/13_arrays/main}
\input{patterns/14_bitfields/main}
\EN{\input{patterns/145_LCG/main_EN}}
\RU{\input{patterns/145_LCG/main_RU}}
\input{patterns/15_structs/main}
\input{patterns/17_unions/main}
\input{patterns/18_pointers_to_functions/main}
\input{patterns/185_64bit_in_32_env/main}

\EN{\input{patterns/19_SIMD/main_EN}}
\RU{\input{patterns/19_SIMD/main_RU}}
\DE{\input{patterns/19_SIMD/main_DE}}

\EN{\input{patterns/20_x64/main_EN}}
\RU{\input{patterns/20_x64/main_RU}}

\EN{\input{patterns/205_floating_SIMD/main_EN}}
\RU{\input{patterns/205_floating_SIMD/main_RU}}
\DE{\input{patterns/205_floating_SIMD/main_DE}}

\EN{\input{patterns/ARM/main_EN}}
\RU{\input{patterns/ARM/main_RU}}
\DE{\input{patterns/ARM/main_DE}}

\input{patterns/MIPS/main}

\ifdefined\SPANISH
\chapter{Patrones de código}
\fi % SPANISH

\ifdefined\GERMAN
\chapter{Code-Muster}
\fi % GERMAN

\ifdefined\ENGLISH
\chapter{Code Patterns}
\fi % ENGLISH

\ifdefined\ITALIAN
\chapter{Forme di codice}
\fi % ITALIAN

\ifdefined\RUSSIAN
\chapter{Образцы кода}
\fi % RUSSIAN

\ifdefined\BRAZILIAN
\chapter{Padrões de códigos}
\fi % BRAZILIAN

\ifdefined\THAI
\chapter{รูปแบบของโค้ด}
\fi % THAI

\ifdefined\FRENCH
\chapter{Modèle de code}
\fi % FRENCH

\ifdefined\POLISH
\chapter{\PLph{}}
\fi % POLISH

% sections
\EN{\input{patterns/patterns_opt_dbg_EN}}
\ES{\input{patterns/patterns_opt_dbg_ES}}
\ITA{\input{patterns/patterns_opt_dbg_ITA}}
\PTBR{\input{patterns/patterns_opt_dbg_PTBR}}
\RU{\input{patterns/patterns_opt_dbg_RU}}
\THA{\input{patterns/patterns_opt_dbg_THA}}
\DE{\input{patterns/patterns_opt_dbg_DE}}
\FR{\input{patterns/patterns_opt_dbg_FR}}
\PL{\input{patterns/patterns_opt_dbg_PL}}

\RU{\section{Некоторые базовые понятия}}
\EN{\section{Some basics}}
\DE{\section{Einige Grundlagen}}
\FR{\section{Quelques bases}}
\ES{\section{\ESph{}}}
\ITA{\section{Alcune basi teoriche}}
\PTBR{\section{\PTBRph{}}}
\THA{\section{\THAph{}}}
\PL{\section{\PLph{}}}

% sections:
\EN{\input{patterns/intro_CPU_ISA_EN}}
\ES{\input{patterns/intro_CPU_ISA_ES}}
\ITA{\input{patterns/intro_CPU_ISA_ITA}}
\PTBR{\input{patterns/intro_CPU_ISA_PTBR}}
\RU{\input{patterns/intro_CPU_ISA_RU}}
\DE{\input{patterns/intro_CPU_ISA_DE}}
\FR{\input{patterns/intro_CPU_ISA_FR}}
\PL{\input{patterns/intro_CPU_ISA_PL}}

\EN{\input{patterns/numeral_EN}}
\RU{\input{patterns/numeral_RU}}
\ITA{\input{patterns/numeral_ITA}}
\DE{\input{patterns/numeral_DE}}
\FR{\input{patterns/numeral_FR}}
\PL{\input{patterns/numeral_PL}}

% chapters
\input{patterns/00_empty/main}
\input{patterns/011_ret/main}
\input{patterns/01_helloworld/main}
\input{patterns/015_prolog_epilogue/main}
\input{patterns/02_stack/main}
\input{patterns/03_printf/main}
\input{patterns/04_scanf/main}
\input{patterns/05_passing_arguments/main}
\input{patterns/06_return_results/main}
\input{patterns/061_pointers/main}
\input{patterns/065_GOTO/main}
\input{patterns/07_jcc/main}
\input{patterns/08_switch/main}
\input{patterns/09_loops/main}
\input{patterns/10_strings/main}
\input{patterns/11_arith_optimizations/main}
\input{patterns/12_FPU/main}
\input{patterns/13_arrays/main}
\input{patterns/14_bitfields/main}
\EN{\input{patterns/145_LCG/main_EN}}
\RU{\input{patterns/145_LCG/main_RU}}
\input{patterns/15_structs/main}
\input{patterns/17_unions/main}
\input{patterns/18_pointers_to_functions/main}
\input{patterns/185_64bit_in_32_env/main}

\EN{\input{patterns/19_SIMD/main_EN}}
\RU{\input{patterns/19_SIMD/main_RU}}
\DE{\input{patterns/19_SIMD/main_DE}}

\EN{\input{patterns/20_x64/main_EN}}
\RU{\input{patterns/20_x64/main_RU}}

\EN{\input{patterns/205_floating_SIMD/main_EN}}
\RU{\input{patterns/205_floating_SIMD/main_RU}}
\DE{\input{patterns/205_floating_SIMD/main_DE}}

\EN{\input{patterns/ARM/main_EN}}
\RU{\input{patterns/ARM/main_RU}}
\DE{\input{patterns/ARM/main_DE}}

\input{patterns/MIPS/main}

\EN{\section{Returning Values}
\label{ret_val_func}

Another simple function is the one that simply returns a constant value:

\lstinputlisting[caption=\EN{\CCpp Code},style=customc]{patterns/011_ret/1.c}

Let's compile it.

\subsection{x86}

Here's what both the GCC and MSVC compilers produce (with optimization) on the x86 platform:

\lstinputlisting[caption=\Optimizing GCC/MSVC (\assemblyOutput),style=customasmx86]{patterns/011_ret/1.s}

\myindex{x86!\Instructions!RET}
There are just two instructions: the first places the value 123 into the \EAX register,
which is used by convention for storing the return
value, and the second one is \RET, which returns execution to the \gls{caller}.

The caller will take the result from the \EAX register.

\subsection{ARM}

There are a few differences on the ARM platform:

\lstinputlisting[caption=\OptimizingKeilVI (\ARMMode) ASM Output,style=customasmARM]{patterns/011_ret/1_Keil_ARM_O3.s}

ARM uses the register \Reg{0} for returning the results of functions, so 123 is copied into \Reg{0}.

\myindex{ARM!\Instructions!MOV}
\myindex{x86!\Instructions!MOV}
It is worth noting that \MOV is a misleading name for the instruction in both the x86 and ARM \ac{ISA}s.

The data is not in fact \IT{moved}, but \IT{copied}.

\subsection{MIPS}

\label{MIPS_leaf_function_ex1}

The GCC assembly output below lists registers by number:

\lstinputlisting[caption=\Optimizing GCC 4.4.5 (\assemblyOutput),style=customasmMIPS]{patterns/011_ret/MIPS.s}

\dots while \IDA does it by their pseudo names:

\lstinputlisting[caption=\Optimizing GCC 4.4.5 (IDA),style=customasmMIPS]{patterns/011_ret/MIPS_IDA.lst}

The \$2 (or \$V0) register is used to store the function's return value.
\myindex{MIPS!\Pseudoinstructions!LI}
\INS{LI} stands for ``Load Immediate'' and is the MIPS equivalent to \MOV.

\myindex{MIPS!\Instructions!J}
The other instruction is the jump instruction (J or JR) which returns the execution flow to the \gls{caller}.

\myindex{MIPS!Branch delay slot}
You might be wondering why the positions of the load instruction (LI) and the jump instruction (J or JR) are swapped. This is due to a \ac{RISC} feature called ``branch delay slot''.

The reason this happens is a quirk in the architecture of some RISC \ac{ISA}s and isn't important for our
purposes---we must simply keep in mind that in MIPS, the instruction following a jump or branch instruction
is executed \IT{before} the jump/branch instruction itself.

As a consequence, branch instructions always swap places with the instruction executed immediately beforehand.


In practice, functions which merely return 1 (\IT{true}) or 0 (\IT{false}) are very frequent.

The smallest ever of the standard UNIX utilities, \IT{/bin/true} and \IT{/bin/false} return 0 and 1 respectively, as an exit code.
(Zero as an exit code usually means success, non-zero means error.)
}
\RU{\subsubsection{std::string}
\myindex{\Cpp!STL!std::string}
\label{std_string}

\myparagraph{Как устроена структура}

Многие строковые библиотеки \InSqBrackets{\CNotes 2.2} обеспечивают структуру содержащую ссылку 
на буфер собственно со строкой, переменная всегда содержащую длину строки 
(что очень удобно для массы функций \InSqBrackets{\CNotes 2.2.1}) и переменную содержащую текущий размер буфера.

Строка в буфере обыкновенно оканчивается нулем: это для того чтобы указатель на буфер можно было
передавать в функции требующие на вход обычную сишную \ac{ASCIIZ}-строку.

Стандарт \Cpp не описывает, как именно нужно реализовывать std::string,
но, как правило, они реализованы как описано выше, с небольшими дополнениями.

Строки в \Cpp это не класс (как, например, QString в Qt), а темплейт (basic\_string), 
это сделано для того чтобы поддерживать 
строки содержащие разного типа символы: как минимум \Tchar и \IT{wchar\_t}.

Так что, std::string это класс с базовым типом \Tchar.

А std::wstring это класс с базовым типом \IT{wchar\_t}.

\mysubparagraph{MSVC}

В реализации MSVC, вместо ссылки на буфер может содержаться сам буфер (если строка короче 16-и символов).

Это означает, что каждая короткая строка будет занимать в памяти по крайней мере $16 + 4 + 4 = 24$ 
байт для 32-битной среды либо $16 + 8 + 8 = 32$ 
байта в 64-битной, а если строка длиннее 16-и символов, то прибавьте еще длину самой строки.

\lstinputlisting[caption=пример для MSVC,style=customc]{\CURPATH/STL/string/MSVC_RU.cpp}

Собственно, из этого исходника почти всё ясно.

Несколько замечаний:

Если строка короче 16-и символов, 
то отдельный буфер для строки в \glslink{heap}{куче} выделяться не будет.

Это удобно потому что на практике, основная часть строк действительно короткие.
Вероятно, разработчики в Microsoft выбрали размер в 16 символов как разумный баланс.

Теперь очень важный момент в конце функции main(): мы не пользуемся методом c\_str(), тем не менее,
если это скомпилировать и запустить, то обе строки появятся в консоли!

Работает это вот почему.

В первом случае строка короче 16-и символов и в начале объекта std::string (его можно рассматривать
просто как структуру) расположен буфер с этой строкой.
\printf трактует указатель как указатель на массив символов оканчивающийся нулем и поэтому всё работает.

Вывод второй строки (длиннее 16-и символов) даже еще опаснее: это вообще типичная программистская ошибка 
(или опечатка), забыть дописать c\_str().
Это работает потому что в это время в начале структуры расположен указатель на буфер.
Это может надолго остаться незамеченным: до тех пока там не появится строка 
короче 16-и символов, тогда процесс упадет.

\mysubparagraph{GCC}

В реализации GCC в структуре есть еще одна переменная --- reference count.

Интересно, что указатель на экземпляр класса std::string в GCC указывает не на начало самой структуры, 
а на указатель на буфера.
В libstdc++-v3\textbackslash{}include\textbackslash{}bits\textbackslash{}basic\_string.h 
мы можем прочитать что это сделано для удобства отладки:

\begin{lstlisting}
   *  The reason you want _M_data pointing to the character %array and
   *  not the _Rep is so that the debugger can see the string
   *  contents. (Probably we should add a non-inline member to get
   *  the _Rep for the debugger to use, so users can check the actual
   *  string length.)
\end{lstlisting}

\href{http://go.yurichev.com/17085}{исходный код basic\_string.h}

В нашем примере мы учитываем это:

\lstinputlisting[caption=пример для GCC,style=customc]{\CURPATH/STL/string/GCC_RU.cpp}

Нужны еще небольшие хаки чтобы сымитировать типичную ошибку, которую мы уже видели выше, из-за
более ужесточенной проверки типов в GCC, тем не менее, printf() работает и здесь без c\_str().

\myparagraph{Чуть более сложный пример}

\lstinputlisting[style=customc]{\CURPATH/STL/string/3.cpp}

\lstinputlisting[caption=MSVC 2012,style=customasmx86]{\CURPATH/STL/string/3_MSVC_RU.asm}

Собственно, компилятор не конструирует строки статически: да в общем-то и как
это возможно, если буфер с ней нужно хранить в \glslink{heap}{куче}?

Вместо этого в сегменте данных хранятся обычные \ac{ASCIIZ}-строки, а позже, во время выполнения, 
при помощи метода \q{assign}, конструируются строки s1 и s2
.
При помощи \TT{operator+}, создается строка s3.

Обратите внимание на то что вызов метода c\_str() отсутствует,
потому что его код достаточно короткий и компилятор вставил его прямо здесь:
если строка короче 16-и байт, то в регистре EAX остается указатель на буфер,
а если длиннее, то из этого же места достается адрес на буфер расположенный в \glslink{heap}{куче}.

Далее следуют вызовы трех деструкторов, причем, они вызываются только если строка длиннее 16-и байт:
тогда нужно освободить буфера в \glslink{heap}{куче}.
В противном случае, так как все три объекта std::string хранятся в стеке,
они освобождаются автоматически после выхода из функции.

Следовательно, работа с короткими строками более быстрая из-за м\'{е}ньшего обращения к \glslink{heap}{куче}.

Код на GCC даже проще (из-за того, что в GCC, как мы уже видели, не реализована возможность хранить короткую
строку прямо в структуре):

% TODO1 comment each function meaning
\lstinputlisting[caption=GCC 4.8.1,style=customasmx86]{\CURPATH/STL/string/3_GCC_RU.s}

Можно заметить, что в деструкторы передается не указатель на объект,
а указатель на место за 12 байт (или 3 слова) перед ним, то есть, на настоящее начало структуры.

\myparagraph{std::string как глобальная переменная}
\label{sec:std_string_as_global_variable}

Опытные программисты на \Cpp знают, что глобальные переменные \ac{STL}-типов вполне можно объявлять.

Да, действительно:

\lstinputlisting[style=customc]{\CURPATH/STL/string/5.cpp}

Но как и где будет вызываться конструктор \TT{std::string}?

На самом деле, эта переменная будет инициализирована даже перед началом \main.

\lstinputlisting[caption=MSVC 2012: здесь конструируется глобальная переменная{,} а также регистрируется её деструктор,style=customasmx86]{\CURPATH/STL/string/5_MSVC_p2.asm}

\lstinputlisting[caption=MSVC 2012: здесь глобальная переменная используется в \main,style=customasmx86]{\CURPATH/STL/string/5_MSVC_p1.asm}

\lstinputlisting[caption=MSVC 2012: эта функция-деструктор вызывается перед выходом,style=customasmx86]{\CURPATH/STL/string/5_MSVC_p3.asm}

\myindex{\CStandardLibrary!atexit()}
В реальности, из \ac{CRT}, еще до вызова main(), вызывается специальная функция,
в которой перечислены все конструкторы подобных переменных.
Более того: при помощи atexit() регистрируется функция, которая будет вызвана в конце работы программы:
в этой функции компилятор собирает вызовы деструкторов всех подобных глобальных переменных.

GCC работает похожим образом:

\lstinputlisting[caption=GCC 4.8.1,style=customasmx86]{\CURPATH/STL/string/5_GCC.s}

Но он не выделяет отдельной функции в которой будут собраны деструкторы: 
каждый деструктор передается в atexit() по одному.

% TODO а если глобальная STL-переменная в другом модуле? надо проверить.

}
\ifdefined\SPANISH
\chapter{Patrones de código}
\fi % SPANISH

\ifdefined\GERMAN
\chapter{Code-Muster}
\fi % GERMAN

\ifdefined\ENGLISH
\chapter{Code Patterns}
\fi % ENGLISH

\ifdefined\ITALIAN
\chapter{Forme di codice}
\fi % ITALIAN

\ifdefined\RUSSIAN
\chapter{Образцы кода}
\fi % RUSSIAN

\ifdefined\BRAZILIAN
\chapter{Padrões de códigos}
\fi % BRAZILIAN

\ifdefined\THAI
\chapter{รูปแบบของโค้ด}
\fi % THAI

\ifdefined\FRENCH
\chapter{Modèle de code}
\fi % FRENCH

\ifdefined\POLISH
\chapter{\PLph{}}
\fi % POLISH

% sections
\EN{\input{patterns/patterns_opt_dbg_EN}}
\ES{\input{patterns/patterns_opt_dbg_ES}}
\ITA{\input{patterns/patterns_opt_dbg_ITA}}
\PTBR{\input{patterns/patterns_opt_dbg_PTBR}}
\RU{\input{patterns/patterns_opt_dbg_RU}}
\THA{\input{patterns/patterns_opt_dbg_THA}}
\DE{\input{patterns/patterns_opt_dbg_DE}}
\FR{\input{patterns/patterns_opt_dbg_FR}}
\PL{\input{patterns/patterns_opt_dbg_PL}}

\RU{\section{Некоторые базовые понятия}}
\EN{\section{Some basics}}
\DE{\section{Einige Grundlagen}}
\FR{\section{Quelques bases}}
\ES{\section{\ESph{}}}
\ITA{\section{Alcune basi teoriche}}
\PTBR{\section{\PTBRph{}}}
\THA{\section{\THAph{}}}
\PL{\section{\PLph{}}}

% sections:
\EN{\input{patterns/intro_CPU_ISA_EN}}
\ES{\input{patterns/intro_CPU_ISA_ES}}
\ITA{\input{patterns/intro_CPU_ISA_ITA}}
\PTBR{\input{patterns/intro_CPU_ISA_PTBR}}
\RU{\input{patterns/intro_CPU_ISA_RU}}
\DE{\input{patterns/intro_CPU_ISA_DE}}
\FR{\input{patterns/intro_CPU_ISA_FR}}
\PL{\input{patterns/intro_CPU_ISA_PL}}

\EN{\input{patterns/numeral_EN}}
\RU{\input{patterns/numeral_RU}}
\ITA{\input{patterns/numeral_ITA}}
\DE{\input{patterns/numeral_DE}}
\FR{\input{patterns/numeral_FR}}
\PL{\input{patterns/numeral_PL}}

% chapters
\input{patterns/00_empty/main}
\input{patterns/011_ret/main}
\input{patterns/01_helloworld/main}
\input{patterns/015_prolog_epilogue/main}
\input{patterns/02_stack/main}
\input{patterns/03_printf/main}
\input{patterns/04_scanf/main}
\input{patterns/05_passing_arguments/main}
\input{patterns/06_return_results/main}
\input{patterns/061_pointers/main}
\input{patterns/065_GOTO/main}
\input{patterns/07_jcc/main}
\input{patterns/08_switch/main}
\input{patterns/09_loops/main}
\input{patterns/10_strings/main}
\input{patterns/11_arith_optimizations/main}
\input{patterns/12_FPU/main}
\input{patterns/13_arrays/main}
\input{patterns/14_bitfields/main}
\EN{\input{patterns/145_LCG/main_EN}}
\RU{\input{patterns/145_LCG/main_RU}}
\input{patterns/15_structs/main}
\input{patterns/17_unions/main}
\input{patterns/18_pointers_to_functions/main}
\input{patterns/185_64bit_in_32_env/main}

\EN{\input{patterns/19_SIMD/main_EN}}
\RU{\input{patterns/19_SIMD/main_RU}}
\DE{\input{patterns/19_SIMD/main_DE}}

\EN{\input{patterns/20_x64/main_EN}}
\RU{\input{patterns/20_x64/main_RU}}

\EN{\input{patterns/205_floating_SIMD/main_EN}}
\RU{\input{patterns/205_floating_SIMD/main_RU}}
\DE{\input{patterns/205_floating_SIMD/main_DE}}

\EN{\input{patterns/ARM/main_EN}}
\RU{\input{patterns/ARM/main_RU}}
\DE{\input{patterns/ARM/main_DE}}

\input{patterns/MIPS/main}

\ifdefined\SPANISH
\chapter{Patrones de código}
\fi % SPANISH

\ifdefined\GERMAN
\chapter{Code-Muster}
\fi % GERMAN

\ifdefined\ENGLISH
\chapter{Code Patterns}
\fi % ENGLISH

\ifdefined\ITALIAN
\chapter{Forme di codice}
\fi % ITALIAN

\ifdefined\RUSSIAN
\chapter{Образцы кода}
\fi % RUSSIAN

\ifdefined\BRAZILIAN
\chapter{Padrões de códigos}
\fi % BRAZILIAN

\ifdefined\THAI
\chapter{รูปแบบของโค้ด}
\fi % THAI

\ifdefined\FRENCH
\chapter{Modèle de code}
\fi % FRENCH

\ifdefined\POLISH
\chapter{\PLph{}}
\fi % POLISH

% sections
\EN{\input{patterns/patterns_opt_dbg_EN}}
\ES{\input{patterns/patterns_opt_dbg_ES}}
\ITA{\input{patterns/patterns_opt_dbg_ITA}}
\PTBR{\input{patterns/patterns_opt_dbg_PTBR}}
\RU{\input{patterns/patterns_opt_dbg_RU}}
\THA{\input{patterns/patterns_opt_dbg_THA}}
\DE{\input{patterns/patterns_opt_dbg_DE}}
\FR{\input{patterns/patterns_opt_dbg_FR}}
\PL{\input{patterns/patterns_opt_dbg_PL}}

\RU{\section{Некоторые базовые понятия}}
\EN{\section{Some basics}}
\DE{\section{Einige Grundlagen}}
\FR{\section{Quelques bases}}
\ES{\section{\ESph{}}}
\ITA{\section{Alcune basi teoriche}}
\PTBR{\section{\PTBRph{}}}
\THA{\section{\THAph{}}}
\PL{\section{\PLph{}}}

% sections:
\EN{\input{patterns/intro_CPU_ISA_EN}}
\ES{\input{patterns/intro_CPU_ISA_ES}}
\ITA{\input{patterns/intro_CPU_ISA_ITA}}
\PTBR{\input{patterns/intro_CPU_ISA_PTBR}}
\RU{\input{patterns/intro_CPU_ISA_RU}}
\DE{\input{patterns/intro_CPU_ISA_DE}}
\FR{\input{patterns/intro_CPU_ISA_FR}}
\PL{\input{patterns/intro_CPU_ISA_PL}}

\EN{\input{patterns/numeral_EN}}
\RU{\input{patterns/numeral_RU}}
\ITA{\input{patterns/numeral_ITA}}
\DE{\input{patterns/numeral_DE}}
\FR{\input{patterns/numeral_FR}}
\PL{\input{patterns/numeral_PL}}

% chapters
\input{patterns/00_empty/main}
\input{patterns/011_ret/main}
\input{patterns/01_helloworld/main}
\input{patterns/015_prolog_epilogue/main}
\input{patterns/02_stack/main}
\input{patterns/03_printf/main}
\input{patterns/04_scanf/main}
\input{patterns/05_passing_arguments/main}
\input{patterns/06_return_results/main}
\input{patterns/061_pointers/main}
\input{patterns/065_GOTO/main}
\input{patterns/07_jcc/main}
\input{patterns/08_switch/main}
\input{patterns/09_loops/main}
\input{patterns/10_strings/main}
\input{patterns/11_arith_optimizations/main}
\input{patterns/12_FPU/main}
\input{patterns/13_arrays/main}
\input{patterns/14_bitfields/main}
\EN{\input{patterns/145_LCG/main_EN}}
\RU{\input{patterns/145_LCG/main_RU}}
\input{patterns/15_structs/main}
\input{patterns/17_unions/main}
\input{patterns/18_pointers_to_functions/main}
\input{patterns/185_64bit_in_32_env/main}

\EN{\input{patterns/19_SIMD/main_EN}}
\RU{\input{patterns/19_SIMD/main_RU}}
\DE{\input{patterns/19_SIMD/main_DE}}

\EN{\input{patterns/20_x64/main_EN}}
\RU{\input{patterns/20_x64/main_RU}}

\EN{\input{patterns/205_floating_SIMD/main_EN}}
\RU{\input{patterns/205_floating_SIMD/main_RU}}
\DE{\input{patterns/205_floating_SIMD/main_DE}}

\EN{\input{patterns/ARM/main_EN}}
\RU{\input{patterns/ARM/main_RU}}
\DE{\input{patterns/ARM/main_DE}}

\input{patterns/MIPS/main}

\ifdefined\SPANISH
\chapter{Patrones de código}
\fi % SPANISH

\ifdefined\GERMAN
\chapter{Code-Muster}
\fi % GERMAN

\ifdefined\ENGLISH
\chapter{Code Patterns}
\fi % ENGLISH

\ifdefined\ITALIAN
\chapter{Forme di codice}
\fi % ITALIAN

\ifdefined\RUSSIAN
\chapter{Образцы кода}
\fi % RUSSIAN

\ifdefined\BRAZILIAN
\chapter{Padrões de códigos}
\fi % BRAZILIAN

\ifdefined\THAI
\chapter{รูปแบบของโค้ด}
\fi % THAI

\ifdefined\FRENCH
\chapter{Modèle de code}
\fi % FRENCH

\ifdefined\POLISH
\chapter{\PLph{}}
\fi % POLISH

% sections
\EN{\input{patterns/patterns_opt_dbg_EN}}
\ES{\input{patterns/patterns_opt_dbg_ES}}
\ITA{\input{patterns/patterns_opt_dbg_ITA}}
\PTBR{\input{patterns/patterns_opt_dbg_PTBR}}
\RU{\input{patterns/patterns_opt_dbg_RU}}
\THA{\input{patterns/patterns_opt_dbg_THA}}
\DE{\input{patterns/patterns_opt_dbg_DE}}
\FR{\input{patterns/patterns_opt_dbg_FR}}
\PL{\input{patterns/patterns_opt_dbg_PL}}

\RU{\section{Некоторые базовые понятия}}
\EN{\section{Some basics}}
\DE{\section{Einige Grundlagen}}
\FR{\section{Quelques bases}}
\ES{\section{\ESph{}}}
\ITA{\section{Alcune basi teoriche}}
\PTBR{\section{\PTBRph{}}}
\THA{\section{\THAph{}}}
\PL{\section{\PLph{}}}

% sections:
\EN{\input{patterns/intro_CPU_ISA_EN}}
\ES{\input{patterns/intro_CPU_ISA_ES}}
\ITA{\input{patterns/intro_CPU_ISA_ITA}}
\PTBR{\input{patterns/intro_CPU_ISA_PTBR}}
\RU{\input{patterns/intro_CPU_ISA_RU}}
\DE{\input{patterns/intro_CPU_ISA_DE}}
\FR{\input{patterns/intro_CPU_ISA_FR}}
\PL{\input{patterns/intro_CPU_ISA_PL}}

\EN{\input{patterns/numeral_EN}}
\RU{\input{patterns/numeral_RU}}
\ITA{\input{patterns/numeral_ITA}}
\DE{\input{patterns/numeral_DE}}
\FR{\input{patterns/numeral_FR}}
\PL{\input{patterns/numeral_PL}}

% chapters
\input{patterns/00_empty/main}
\input{patterns/011_ret/main}
\input{patterns/01_helloworld/main}
\input{patterns/015_prolog_epilogue/main}
\input{patterns/02_stack/main}
\input{patterns/03_printf/main}
\input{patterns/04_scanf/main}
\input{patterns/05_passing_arguments/main}
\input{patterns/06_return_results/main}
\input{patterns/061_pointers/main}
\input{patterns/065_GOTO/main}
\input{patterns/07_jcc/main}
\input{patterns/08_switch/main}
\input{patterns/09_loops/main}
\input{patterns/10_strings/main}
\input{patterns/11_arith_optimizations/main}
\input{patterns/12_FPU/main}
\input{patterns/13_arrays/main}
\input{patterns/14_bitfields/main}
\EN{\input{patterns/145_LCG/main_EN}}
\RU{\input{patterns/145_LCG/main_RU}}
\input{patterns/15_structs/main}
\input{patterns/17_unions/main}
\input{patterns/18_pointers_to_functions/main}
\input{patterns/185_64bit_in_32_env/main}

\EN{\input{patterns/19_SIMD/main_EN}}
\RU{\input{patterns/19_SIMD/main_RU}}
\DE{\input{patterns/19_SIMD/main_DE}}

\EN{\input{patterns/20_x64/main_EN}}
\RU{\input{patterns/20_x64/main_RU}}

\EN{\input{patterns/205_floating_SIMD/main_EN}}
\RU{\input{patterns/205_floating_SIMD/main_RU}}
\DE{\input{patterns/205_floating_SIMD/main_DE}}

\EN{\input{patterns/ARM/main_EN}}
\RU{\input{patterns/ARM/main_RU}}
\DE{\input{patterns/ARM/main_DE}}

\input{patterns/MIPS/main}

\ifdefined\SPANISH
\chapter{Patrones de código}
\fi % SPANISH

\ifdefined\GERMAN
\chapter{Code-Muster}
\fi % GERMAN

\ifdefined\ENGLISH
\chapter{Code Patterns}
\fi % ENGLISH

\ifdefined\ITALIAN
\chapter{Forme di codice}
\fi % ITALIAN

\ifdefined\RUSSIAN
\chapter{Образцы кода}
\fi % RUSSIAN

\ifdefined\BRAZILIAN
\chapter{Padrões de códigos}
\fi % BRAZILIAN

\ifdefined\THAI
\chapter{รูปแบบของโค้ด}
\fi % THAI

\ifdefined\FRENCH
\chapter{Modèle de code}
\fi % FRENCH

\ifdefined\POLISH
\chapter{\PLph{}}
\fi % POLISH

% sections
\EN{\input{patterns/patterns_opt_dbg_EN}}
\ES{\input{patterns/patterns_opt_dbg_ES}}
\ITA{\input{patterns/patterns_opt_dbg_ITA}}
\PTBR{\input{patterns/patterns_opt_dbg_PTBR}}
\RU{\input{patterns/patterns_opt_dbg_RU}}
\THA{\input{patterns/patterns_opt_dbg_THA}}
\DE{\input{patterns/patterns_opt_dbg_DE}}
\FR{\input{patterns/patterns_opt_dbg_FR}}
\PL{\input{patterns/patterns_opt_dbg_PL}}

\RU{\section{Некоторые базовые понятия}}
\EN{\section{Some basics}}
\DE{\section{Einige Grundlagen}}
\FR{\section{Quelques bases}}
\ES{\section{\ESph{}}}
\ITA{\section{Alcune basi teoriche}}
\PTBR{\section{\PTBRph{}}}
\THA{\section{\THAph{}}}
\PL{\section{\PLph{}}}

% sections:
\EN{\input{patterns/intro_CPU_ISA_EN}}
\ES{\input{patterns/intro_CPU_ISA_ES}}
\ITA{\input{patterns/intro_CPU_ISA_ITA}}
\PTBR{\input{patterns/intro_CPU_ISA_PTBR}}
\RU{\input{patterns/intro_CPU_ISA_RU}}
\DE{\input{patterns/intro_CPU_ISA_DE}}
\FR{\input{patterns/intro_CPU_ISA_FR}}
\PL{\input{patterns/intro_CPU_ISA_PL}}

\EN{\input{patterns/numeral_EN}}
\RU{\input{patterns/numeral_RU}}
\ITA{\input{patterns/numeral_ITA}}
\DE{\input{patterns/numeral_DE}}
\FR{\input{patterns/numeral_FR}}
\PL{\input{patterns/numeral_PL}}

% chapters
\input{patterns/00_empty/main}
\input{patterns/011_ret/main}
\input{patterns/01_helloworld/main}
\input{patterns/015_prolog_epilogue/main}
\input{patterns/02_stack/main}
\input{patterns/03_printf/main}
\input{patterns/04_scanf/main}
\input{patterns/05_passing_arguments/main}
\input{patterns/06_return_results/main}
\input{patterns/061_pointers/main}
\input{patterns/065_GOTO/main}
\input{patterns/07_jcc/main}
\input{patterns/08_switch/main}
\input{patterns/09_loops/main}
\input{patterns/10_strings/main}
\input{patterns/11_arith_optimizations/main}
\input{patterns/12_FPU/main}
\input{patterns/13_arrays/main}
\input{patterns/14_bitfields/main}
\EN{\input{patterns/145_LCG/main_EN}}
\RU{\input{patterns/145_LCG/main_RU}}
\input{patterns/15_structs/main}
\input{patterns/17_unions/main}
\input{patterns/18_pointers_to_functions/main}
\input{patterns/185_64bit_in_32_env/main}

\EN{\input{patterns/19_SIMD/main_EN}}
\RU{\input{patterns/19_SIMD/main_RU}}
\DE{\input{patterns/19_SIMD/main_DE}}

\EN{\input{patterns/20_x64/main_EN}}
\RU{\input{patterns/20_x64/main_RU}}

\EN{\input{patterns/205_floating_SIMD/main_EN}}
\RU{\input{patterns/205_floating_SIMD/main_RU}}
\DE{\input{patterns/205_floating_SIMD/main_DE}}

\EN{\input{patterns/ARM/main_EN}}
\RU{\input{patterns/ARM/main_RU}}
\DE{\input{patterns/ARM/main_DE}}

\input{patterns/MIPS/main}


\EN{\section{Returning Values}
\label{ret_val_func}

Another simple function is the one that simply returns a constant value:

\lstinputlisting[caption=\EN{\CCpp Code},style=customc]{patterns/011_ret/1.c}

Let's compile it.

\subsection{x86}

Here's what both the GCC and MSVC compilers produce (with optimization) on the x86 platform:

\lstinputlisting[caption=\Optimizing GCC/MSVC (\assemblyOutput),style=customasmx86]{patterns/011_ret/1.s}

\myindex{x86!\Instructions!RET}
There are just two instructions: the first places the value 123 into the \EAX register,
which is used by convention for storing the return
value, and the second one is \RET, which returns execution to the \gls{caller}.

The caller will take the result from the \EAX register.

\subsection{ARM}

There are a few differences on the ARM platform:

\lstinputlisting[caption=\OptimizingKeilVI (\ARMMode) ASM Output,style=customasmARM]{patterns/011_ret/1_Keil_ARM_O3.s}

ARM uses the register \Reg{0} for returning the results of functions, so 123 is copied into \Reg{0}.

\myindex{ARM!\Instructions!MOV}
\myindex{x86!\Instructions!MOV}
It is worth noting that \MOV is a misleading name for the instruction in both the x86 and ARM \ac{ISA}s.

The data is not in fact \IT{moved}, but \IT{copied}.

\subsection{MIPS}

\label{MIPS_leaf_function_ex1}

The GCC assembly output below lists registers by number:

\lstinputlisting[caption=\Optimizing GCC 4.4.5 (\assemblyOutput),style=customasmMIPS]{patterns/011_ret/MIPS.s}

\dots while \IDA does it by their pseudo names:

\lstinputlisting[caption=\Optimizing GCC 4.4.5 (IDA),style=customasmMIPS]{patterns/011_ret/MIPS_IDA.lst}

The \$2 (or \$V0) register is used to store the function's return value.
\myindex{MIPS!\Pseudoinstructions!LI}
\INS{LI} stands for ``Load Immediate'' and is the MIPS equivalent to \MOV.

\myindex{MIPS!\Instructions!J}
The other instruction is the jump instruction (J or JR) which returns the execution flow to the \gls{caller}.

\myindex{MIPS!Branch delay slot}
You might be wondering why the positions of the load instruction (LI) and the jump instruction (J or JR) are swapped. This is due to a \ac{RISC} feature called ``branch delay slot''.

The reason this happens is a quirk in the architecture of some RISC \ac{ISA}s and isn't important for our
purposes---we must simply keep in mind that in MIPS, the instruction following a jump or branch instruction
is executed \IT{before} the jump/branch instruction itself.

As a consequence, branch instructions always swap places with the instruction executed immediately beforehand.


In practice, functions which merely return 1 (\IT{true}) or 0 (\IT{false}) are very frequent.

The smallest ever of the standard UNIX utilities, \IT{/bin/true} and \IT{/bin/false} return 0 and 1 respectively, as an exit code.
(Zero as an exit code usually means success, non-zero means error.)
}
\RU{\subsubsection{std::string}
\myindex{\Cpp!STL!std::string}
\label{std_string}

\myparagraph{Как устроена структура}

Многие строковые библиотеки \InSqBrackets{\CNotes 2.2} обеспечивают структуру содержащую ссылку 
на буфер собственно со строкой, переменная всегда содержащую длину строки 
(что очень удобно для массы функций \InSqBrackets{\CNotes 2.2.1}) и переменную содержащую текущий размер буфера.

Строка в буфере обыкновенно оканчивается нулем: это для того чтобы указатель на буфер можно было
передавать в функции требующие на вход обычную сишную \ac{ASCIIZ}-строку.

Стандарт \Cpp не описывает, как именно нужно реализовывать std::string,
но, как правило, они реализованы как описано выше, с небольшими дополнениями.

Строки в \Cpp это не класс (как, например, QString в Qt), а темплейт (basic\_string), 
это сделано для того чтобы поддерживать 
строки содержащие разного типа символы: как минимум \Tchar и \IT{wchar\_t}.

Так что, std::string это класс с базовым типом \Tchar.

А std::wstring это класс с базовым типом \IT{wchar\_t}.

\mysubparagraph{MSVC}

В реализации MSVC, вместо ссылки на буфер может содержаться сам буфер (если строка короче 16-и символов).

Это означает, что каждая короткая строка будет занимать в памяти по крайней мере $16 + 4 + 4 = 24$ 
байт для 32-битной среды либо $16 + 8 + 8 = 32$ 
байта в 64-битной, а если строка длиннее 16-и символов, то прибавьте еще длину самой строки.

\lstinputlisting[caption=пример для MSVC,style=customc]{\CURPATH/STL/string/MSVC_RU.cpp}

Собственно, из этого исходника почти всё ясно.

Несколько замечаний:

Если строка короче 16-и символов, 
то отдельный буфер для строки в \glslink{heap}{куче} выделяться не будет.

Это удобно потому что на практике, основная часть строк действительно короткие.
Вероятно, разработчики в Microsoft выбрали размер в 16 символов как разумный баланс.

Теперь очень важный момент в конце функции main(): мы не пользуемся методом c\_str(), тем не менее,
если это скомпилировать и запустить, то обе строки появятся в консоли!

Работает это вот почему.

В первом случае строка короче 16-и символов и в начале объекта std::string (его можно рассматривать
просто как структуру) расположен буфер с этой строкой.
\printf трактует указатель как указатель на массив символов оканчивающийся нулем и поэтому всё работает.

Вывод второй строки (длиннее 16-и символов) даже еще опаснее: это вообще типичная программистская ошибка 
(или опечатка), забыть дописать c\_str().
Это работает потому что в это время в начале структуры расположен указатель на буфер.
Это может надолго остаться незамеченным: до тех пока там не появится строка 
короче 16-и символов, тогда процесс упадет.

\mysubparagraph{GCC}

В реализации GCC в структуре есть еще одна переменная --- reference count.

Интересно, что указатель на экземпляр класса std::string в GCC указывает не на начало самой структуры, 
а на указатель на буфера.
В libstdc++-v3\textbackslash{}include\textbackslash{}bits\textbackslash{}basic\_string.h 
мы можем прочитать что это сделано для удобства отладки:

\begin{lstlisting}
   *  The reason you want _M_data pointing to the character %array and
   *  not the _Rep is so that the debugger can see the string
   *  contents. (Probably we should add a non-inline member to get
   *  the _Rep for the debugger to use, so users can check the actual
   *  string length.)
\end{lstlisting}

\href{http://go.yurichev.com/17085}{исходный код basic\_string.h}

В нашем примере мы учитываем это:

\lstinputlisting[caption=пример для GCC,style=customc]{\CURPATH/STL/string/GCC_RU.cpp}

Нужны еще небольшие хаки чтобы сымитировать типичную ошибку, которую мы уже видели выше, из-за
более ужесточенной проверки типов в GCC, тем не менее, printf() работает и здесь без c\_str().

\myparagraph{Чуть более сложный пример}

\lstinputlisting[style=customc]{\CURPATH/STL/string/3.cpp}

\lstinputlisting[caption=MSVC 2012,style=customasmx86]{\CURPATH/STL/string/3_MSVC_RU.asm}

Собственно, компилятор не конструирует строки статически: да в общем-то и как
это возможно, если буфер с ней нужно хранить в \glslink{heap}{куче}?

Вместо этого в сегменте данных хранятся обычные \ac{ASCIIZ}-строки, а позже, во время выполнения, 
при помощи метода \q{assign}, конструируются строки s1 и s2
.
При помощи \TT{operator+}, создается строка s3.

Обратите внимание на то что вызов метода c\_str() отсутствует,
потому что его код достаточно короткий и компилятор вставил его прямо здесь:
если строка короче 16-и байт, то в регистре EAX остается указатель на буфер,
а если длиннее, то из этого же места достается адрес на буфер расположенный в \glslink{heap}{куче}.

Далее следуют вызовы трех деструкторов, причем, они вызываются только если строка длиннее 16-и байт:
тогда нужно освободить буфера в \glslink{heap}{куче}.
В противном случае, так как все три объекта std::string хранятся в стеке,
они освобождаются автоматически после выхода из функции.

Следовательно, работа с короткими строками более быстрая из-за м\'{е}ньшего обращения к \glslink{heap}{куче}.

Код на GCC даже проще (из-за того, что в GCC, как мы уже видели, не реализована возможность хранить короткую
строку прямо в структуре):

% TODO1 comment each function meaning
\lstinputlisting[caption=GCC 4.8.1,style=customasmx86]{\CURPATH/STL/string/3_GCC_RU.s}

Можно заметить, что в деструкторы передается не указатель на объект,
а указатель на место за 12 байт (или 3 слова) перед ним, то есть, на настоящее начало структуры.

\myparagraph{std::string как глобальная переменная}
\label{sec:std_string_as_global_variable}

Опытные программисты на \Cpp знают, что глобальные переменные \ac{STL}-типов вполне можно объявлять.

Да, действительно:

\lstinputlisting[style=customc]{\CURPATH/STL/string/5.cpp}

Но как и где будет вызываться конструктор \TT{std::string}?

На самом деле, эта переменная будет инициализирована даже перед началом \main.

\lstinputlisting[caption=MSVC 2012: здесь конструируется глобальная переменная{,} а также регистрируется её деструктор,style=customasmx86]{\CURPATH/STL/string/5_MSVC_p2.asm}

\lstinputlisting[caption=MSVC 2012: здесь глобальная переменная используется в \main,style=customasmx86]{\CURPATH/STL/string/5_MSVC_p1.asm}

\lstinputlisting[caption=MSVC 2012: эта функция-деструктор вызывается перед выходом,style=customasmx86]{\CURPATH/STL/string/5_MSVC_p3.asm}

\myindex{\CStandardLibrary!atexit()}
В реальности, из \ac{CRT}, еще до вызова main(), вызывается специальная функция,
в которой перечислены все конструкторы подобных переменных.
Более того: при помощи atexit() регистрируется функция, которая будет вызвана в конце работы программы:
в этой функции компилятор собирает вызовы деструкторов всех подобных глобальных переменных.

GCC работает похожим образом:

\lstinputlisting[caption=GCC 4.8.1,style=customasmx86]{\CURPATH/STL/string/5_GCC.s}

Но он не выделяет отдельной функции в которой будут собраны деструкторы: 
каждый деструктор передается в atexit() по одному.

% TODO а если глобальная STL-переменная в другом модуле? надо проверить.

}
\DE{\subsection{Einfachste XOR-Verschlüsselung überhaupt}

Ich habe einmal eine Software gesehen, bei der alle Debugging-Ausgaben mit XOR mit dem Wert 3
verschlüsselt wurden. Mit anderen Worten, die beiden niedrigsten Bits aller Buchstaben wurden invertiert.

``Hello, world'' wurde zu ``Kfool/\#tlqog'':

\begin{lstlisting}
#!/usr/bin/python

msg="Hello, world!"

print "".join(map(lambda x: chr(ord(x)^3), msg))
\end{lstlisting}

Das ist eine ziemlich interessante Verschlüsselung (oder besser eine Verschleierung),
weil sie zwei wichtige Eigenschaften hat:
1) es ist eine einzige Funktion zum Verschlüsseln und entschlüsseln, sie muss nur wiederholt angewendet werden
2) die entstehenden Buchstaben befinden sich im druckbaren Bereich, also die ganze Zeichenkette kann ohne
Escape-Symbole im Code verwendet werden.

Die zweite Eigenschaft nutzt die Tatsache, dass alle druckbaren Zeichen in Reihen organisiert sind: 0x2x-0x7x,
und wenn die beiden niederwertigsten Bits invertiert werden, wird der Buchstabe um eine oder drei Stellen nach
links oder rechts \IT{verschoben}, aber niemals in eine andere Reihe:

\begin{figure}[H]
\centering
\includegraphics[width=0.7\textwidth]{ascii_clean.png}
\caption{7-Bit \ac{ASCII} Tabelle in Emacs}
\end{figure}

\dots mit dem Zeichen 0x7F als einziger Ausnahme.

Im Folgenden werden also beispielsweise die Zeichen A-Z \IT{verschlüsselt}:

\begin{lstlisting}
#!/usr/bin/python

msg="@ABCDEFGHIJKLMNO"

print "".join(map(lambda x: chr(ord(x)^3), msg))
\end{lstlisting}

Ergebnis:
% FIXME \verb  --  relevant comment for German?
\begin{lstlisting}
CBA@GFEDKJIHONML
\end{lstlisting}

Es sieht so aus als würden die Zeichen ``@'' und ``C'' sowie ``B'' und ``A'' vertauscht werden.

Hier ist noch ein interessantes Beispiel, in dem gezeigt wird, wie die Eigenschaften von XOR
ausgenutzt werden können: Exakt den gleichen Effekt, dass druckbare Zeichen auch druckbar bleiben,
kann man dadurch erzielen, dass irgendeine Kombination der niedrigsten vier Bits invertiert wird.
}

\EN{\section{Returning Values}
\label{ret_val_func}

Another simple function is the one that simply returns a constant value:

\lstinputlisting[caption=\EN{\CCpp Code},style=customc]{patterns/011_ret/1.c}

Let's compile it.

\subsection{x86}

Here's what both the GCC and MSVC compilers produce (with optimization) on the x86 platform:

\lstinputlisting[caption=\Optimizing GCC/MSVC (\assemblyOutput),style=customasmx86]{patterns/011_ret/1.s}

\myindex{x86!\Instructions!RET}
There are just two instructions: the first places the value 123 into the \EAX register,
which is used by convention for storing the return
value, and the second one is \RET, which returns execution to the \gls{caller}.

The caller will take the result from the \EAX register.

\subsection{ARM}

There are a few differences on the ARM platform:

\lstinputlisting[caption=\OptimizingKeilVI (\ARMMode) ASM Output,style=customasmARM]{patterns/011_ret/1_Keil_ARM_O3.s}

ARM uses the register \Reg{0} for returning the results of functions, so 123 is copied into \Reg{0}.

\myindex{ARM!\Instructions!MOV}
\myindex{x86!\Instructions!MOV}
It is worth noting that \MOV is a misleading name for the instruction in both the x86 and ARM \ac{ISA}s.

The data is not in fact \IT{moved}, but \IT{copied}.

\subsection{MIPS}

\label{MIPS_leaf_function_ex1}

The GCC assembly output below lists registers by number:

\lstinputlisting[caption=\Optimizing GCC 4.4.5 (\assemblyOutput),style=customasmMIPS]{patterns/011_ret/MIPS.s}

\dots while \IDA does it by their pseudo names:

\lstinputlisting[caption=\Optimizing GCC 4.4.5 (IDA),style=customasmMIPS]{patterns/011_ret/MIPS_IDA.lst}

The \$2 (or \$V0) register is used to store the function's return value.
\myindex{MIPS!\Pseudoinstructions!LI}
\INS{LI} stands for ``Load Immediate'' and is the MIPS equivalent to \MOV.

\myindex{MIPS!\Instructions!J}
The other instruction is the jump instruction (J or JR) which returns the execution flow to the \gls{caller}.

\myindex{MIPS!Branch delay slot}
You might be wondering why the positions of the load instruction (LI) and the jump instruction (J or JR) are swapped. This is due to a \ac{RISC} feature called ``branch delay slot''.

The reason this happens is a quirk in the architecture of some RISC \ac{ISA}s and isn't important for our
purposes---we must simply keep in mind that in MIPS, the instruction following a jump or branch instruction
is executed \IT{before} the jump/branch instruction itself.

As a consequence, branch instructions always swap places with the instruction executed immediately beforehand.


In practice, functions which merely return 1 (\IT{true}) or 0 (\IT{false}) are very frequent.

The smallest ever of the standard UNIX utilities, \IT{/bin/true} and \IT{/bin/false} return 0 and 1 respectively, as an exit code.
(Zero as an exit code usually means success, non-zero means error.)
}
\RU{\subsubsection{std::string}
\myindex{\Cpp!STL!std::string}
\label{std_string}

\myparagraph{Как устроена структура}

Многие строковые библиотеки \InSqBrackets{\CNotes 2.2} обеспечивают структуру содержащую ссылку 
на буфер собственно со строкой, переменная всегда содержащую длину строки 
(что очень удобно для массы функций \InSqBrackets{\CNotes 2.2.1}) и переменную содержащую текущий размер буфера.

Строка в буфере обыкновенно оканчивается нулем: это для того чтобы указатель на буфер можно было
передавать в функции требующие на вход обычную сишную \ac{ASCIIZ}-строку.

Стандарт \Cpp не описывает, как именно нужно реализовывать std::string,
но, как правило, они реализованы как описано выше, с небольшими дополнениями.

Строки в \Cpp это не класс (как, например, QString в Qt), а темплейт (basic\_string), 
это сделано для того чтобы поддерживать 
строки содержащие разного типа символы: как минимум \Tchar и \IT{wchar\_t}.

Так что, std::string это класс с базовым типом \Tchar.

А std::wstring это класс с базовым типом \IT{wchar\_t}.

\mysubparagraph{MSVC}

В реализации MSVC, вместо ссылки на буфер может содержаться сам буфер (если строка короче 16-и символов).

Это означает, что каждая короткая строка будет занимать в памяти по крайней мере $16 + 4 + 4 = 24$ 
байт для 32-битной среды либо $16 + 8 + 8 = 32$ 
байта в 64-битной, а если строка длиннее 16-и символов, то прибавьте еще длину самой строки.

\lstinputlisting[caption=пример для MSVC,style=customc]{\CURPATH/STL/string/MSVC_RU.cpp}

Собственно, из этого исходника почти всё ясно.

Несколько замечаний:

Если строка короче 16-и символов, 
то отдельный буфер для строки в \glslink{heap}{куче} выделяться не будет.

Это удобно потому что на практике, основная часть строк действительно короткие.
Вероятно, разработчики в Microsoft выбрали размер в 16 символов как разумный баланс.

Теперь очень важный момент в конце функции main(): мы не пользуемся методом c\_str(), тем не менее,
если это скомпилировать и запустить, то обе строки появятся в консоли!

Работает это вот почему.

В первом случае строка короче 16-и символов и в начале объекта std::string (его можно рассматривать
просто как структуру) расположен буфер с этой строкой.
\printf трактует указатель как указатель на массив символов оканчивающийся нулем и поэтому всё работает.

Вывод второй строки (длиннее 16-и символов) даже еще опаснее: это вообще типичная программистская ошибка 
(или опечатка), забыть дописать c\_str().
Это работает потому что в это время в начале структуры расположен указатель на буфер.
Это может надолго остаться незамеченным: до тех пока там не появится строка 
короче 16-и символов, тогда процесс упадет.

\mysubparagraph{GCC}

В реализации GCC в структуре есть еще одна переменная --- reference count.

Интересно, что указатель на экземпляр класса std::string в GCC указывает не на начало самой структуры, 
а на указатель на буфера.
В libstdc++-v3\textbackslash{}include\textbackslash{}bits\textbackslash{}basic\_string.h 
мы можем прочитать что это сделано для удобства отладки:

\begin{lstlisting}
   *  The reason you want _M_data pointing to the character %array and
   *  not the _Rep is so that the debugger can see the string
   *  contents. (Probably we should add a non-inline member to get
   *  the _Rep for the debugger to use, so users can check the actual
   *  string length.)
\end{lstlisting}

\href{http://go.yurichev.com/17085}{исходный код basic\_string.h}

В нашем примере мы учитываем это:

\lstinputlisting[caption=пример для GCC,style=customc]{\CURPATH/STL/string/GCC_RU.cpp}

Нужны еще небольшие хаки чтобы сымитировать типичную ошибку, которую мы уже видели выше, из-за
более ужесточенной проверки типов в GCC, тем не менее, printf() работает и здесь без c\_str().

\myparagraph{Чуть более сложный пример}

\lstinputlisting[style=customc]{\CURPATH/STL/string/3.cpp}

\lstinputlisting[caption=MSVC 2012,style=customasmx86]{\CURPATH/STL/string/3_MSVC_RU.asm}

Собственно, компилятор не конструирует строки статически: да в общем-то и как
это возможно, если буфер с ней нужно хранить в \glslink{heap}{куче}?

Вместо этого в сегменте данных хранятся обычные \ac{ASCIIZ}-строки, а позже, во время выполнения, 
при помощи метода \q{assign}, конструируются строки s1 и s2
.
При помощи \TT{operator+}, создается строка s3.

Обратите внимание на то что вызов метода c\_str() отсутствует,
потому что его код достаточно короткий и компилятор вставил его прямо здесь:
если строка короче 16-и байт, то в регистре EAX остается указатель на буфер,
а если длиннее, то из этого же места достается адрес на буфер расположенный в \glslink{heap}{куче}.

Далее следуют вызовы трех деструкторов, причем, они вызываются только если строка длиннее 16-и байт:
тогда нужно освободить буфера в \glslink{heap}{куче}.
В противном случае, так как все три объекта std::string хранятся в стеке,
они освобождаются автоматически после выхода из функции.

Следовательно, работа с короткими строками более быстрая из-за м\'{е}ньшего обращения к \glslink{heap}{куче}.

Код на GCC даже проще (из-за того, что в GCC, как мы уже видели, не реализована возможность хранить короткую
строку прямо в структуре):

% TODO1 comment each function meaning
\lstinputlisting[caption=GCC 4.8.1,style=customasmx86]{\CURPATH/STL/string/3_GCC_RU.s}

Можно заметить, что в деструкторы передается не указатель на объект,
а указатель на место за 12 байт (или 3 слова) перед ним, то есть, на настоящее начало структуры.

\myparagraph{std::string как глобальная переменная}
\label{sec:std_string_as_global_variable}

Опытные программисты на \Cpp знают, что глобальные переменные \ac{STL}-типов вполне можно объявлять.

Да, действительно:

\lstinputlisting[style=customc]{\CURPATH/STL/string/5.cpp}

Но как и где будет вызываться конструктор \TT{std::string}?

На самом деле, эта переменная будет инициализирована даже перед началом \main.

\lstinputlisting[caption=MSVC 2012: здесь конструируется глобальная переменная{,} а также регистрируется её деструктор,style=customasmx86]{\CURPATH/STL/string/5_MSVC_p2.asm}

\lstinputlisting[caption=MSVC 2012: здесь глобальная переменная используется в \main,style=customasmx86]{\CURPATH/STL/string/5_MSVC_p1.asm}

\lstinputlisting[caption=MSVC 2012: эта функция-деструктор вызывается перед выходом,style=customasmx86]{\CURPATH/STL/string/5_MSVC_p3.asm}

\myindex{\CStandardLibrary!atexit()}
В реальности, из \ac{CRT}, еще до вызова main(), вызывается специальная функция,
в которой перечислены все конструкторы подобных переменных.
Более того: при помощи atexit() регистрируется функция, которая будет вызвана в конце работы программы:
в этой функции компилятор собирает вызовы деструкторов всех подобных глобальных переменных.

GCC работает похожим образом:

\lstinputlisting[caption=GCC 4.8.1,style=customasmx86]{\CURPATH/STL/string/5_GCC.s}

Но он не выделяет отдельной функции в которой будут собраны деструкторы: 
каждый деструктор передается в atexit() по одному.

% TODO а если глобальная STL-переменная в другом модуле? надо проверить.

}

\EN{\section{Returning Values}
\label{ret_val_func}

Another simple function is the one that simply returns a constant value:

\lstinputlisting[caption=\EN{\CCpp Code},style=customc]{patterns/011_ret/1.c}

Let's compile it.

\subsection{x86}

Here's what both the GCC and MSVC compilers produce (with optimization) on the x86 platform:

\lstinputlisting[caption=\Optimizing GCC/MSVC (\assemblyOutput),style=customasmx86]{patterns/011_ret/1.s}

\myindex{x86!\Instructions!RET}
There are just two instructions: the first places the value 123 into the \EAX register,
which is used by convention for storing the return
value, and the second one is \RET, which returns execution to the \gls{caller}.

The caller will take the result from the \EAX register.

\subsection{ARM}

There are a few differences on the ARM platform:

\lstinputlisting[caption=\OptimizingKeilVI (\ARMMode) ASM Output,style=customasmARM]{patterns/011_ret/1_Keil_ARM_O3.s}

ARM uses the register \Reg{0} for returning the results of functions, so 123 is copied into \Reg{0}.

\myindex{ARM!\Instructions!MOV}
\myindex{x86!\Instructions!MOV}
It is worth noting that \MOV is a misleading name for the instruction in both the x86 and ARM \ac{ISA}s.

The data is not in fact \IT{moved}, but \IT{copied}.

\subsection{MIPS}

\label{MIPS_leaf_function_ex1}

The GCC assembly output below lists registers by number:

\lstinputlisting[caption=\Optimizing GCC 4.4.5 (\assemblyOutput),style=customasmMIPS]{patterns/011_ret/MIPS.s}

\dots while \IDA does it by their pseudo names:

\lstinputlisting[caption=\Optimizing GCC 4.4.5 (IDA),style=customasmMIPS]{patterns/011_ret/MIPS_IDA.lst}

The \$2 (or \$V0) register is used to store the function's return value.
\myindex{MIPS!\Pseudoinstructions!LI}
\INS{LI} stands for ``Load Immediate'' and is the MIPS equivalent to \MOV.

\myindex{MIPS!\Instructions!J}
The other instruction is the jump instruction (J or JR) which returns the execution flow to the \gls{caller}.

\myindex{MIPS!Branch delay slot}
You might be wondering why the positions of the load instruction (LI) and the jump instruction (J or JR) are swapped. This is due to a \ac{RISC} feature called ``branch delay slot''.

The reason this happens is a quirk in the architecture of some RISC \ac{ISA}s and isn't important for our
purposes---we must simply keep in mind that in MIPS, the instruction following a jump or branch instruction
is executed \IT{before} the jump/branch instruction itself.

As a consequence, branch instructions always swap places with the instruction executed immediately beforehand.


In practice, functions which merely return 1 (\IT{true}) or 0 (\IT{false}) are very frequent.

The smallest ever of the standard UNIX utilities, \IT{/bin/true} and \IT{/bin/false} return 0 and 1 respectively, as an exit code.
(Zero as an exit code usually means success, non-zero means error.)
}
\RU{\subsubsection{std::string}
\myindex{\Cpp!STL!std::string}
\label{std_string}

\myparagraph{Как устроена структура}

Многие строковые библиотеки \InSqBrackets{\CNotes 2.2} обеспечивают структуру содержащую ссылку 
на буфер собственно со строкой, переменная всегда содержащую длину строки 
(что очень удобно для массы функций \InSqBrackets{\CNotes 2.2.1}) и переменную содержащую текущий размер буфера.

Строка в буфере обыкновенно оканчивается нулем: это для того чтобы указатель на буфер можно было
передавать в функции требующие на вход обычную сишную \ac{ASCIIZ}-строку.

Стандарт \Cpp не описывает, как именно нужно реализовывать std::string,
но, как правило, они реализованы как описано выше, с небольшими дополнениями.

Строки в \Cpp это не класс (как, например, QString в Qt), а темплейт (basic\_string), 
это сделано для того чтобы поддерживать 
строки содержащие разного типа символы: как минимум \Tchar и \IT{wchar\_t}.

Так что, std::string это класс с базовым типом \Tchar.

А std::wstring это класс с базовым типом \IT{wchar\_t}.

\mysubparagraph{MSVC}

В реализации MSVC, вместо ссылки на буфер может содержаться сам буфер (если строка короче 16-и символов).

Это означает, что каждая короткая строка будет занимать в памяти по крайней мере $16 + 4 + 4 = 24$ 
байт для 32-битной среды либо $16 + 8 + 8 = 32$ 
байта в 64-битной, а если строка длиннее 16-и символов, то прибавьте еще длину самой строки.

\lstinputlisting[caption=пример для MSVC,style=customc]{\CURPATH/STL/string/MSVC_RU.cpp}

Собственно, из этого исходника почти всё ясно.

Несколько замечаний:

Если строка короче 16-и символов, 
то отдельный буфер для строки в \glslink{heap}{куче} выделяться не будет.

Это удобно потому что на практике, основная часть строк действительно короткие.
Вероятно, разработчики в Microsoft выбрали размер в 16 символов как разумный баланс.

Теперь очень важный момент в конце функции main(): мы не пользуемся методом c\_str(), тем не менее,
если это скомпилировать и запустить, то обе строки появятся в консоли!

Работает это вот почему.

В первом случае строка короче 16-и символов и в начале объекта std::string (его можно рассматривать
просто как структуру) расположен буфер с этой строкой.
\printf трактует указатель как указатель на массив символов оканчивающийся нулем и поэтому всё работает.

Вывод второй строки (длиннее 16-и символов) даже еще опаснее: это вообще типичная программистская ошибка 
(или опечатка), забыть дописать c\_str().
Это работает потому что в это время в начале структуры расположен указатель на буфер.
Это может надолго остаться незамеченным: до тех пока там не появится строка 
короче 16-и символов, тогда процесс упадет.

\mysubparagraph{GCC}

В реализации GCC в структуре есть еще одна переменная --- reference count.

Интересно, что указатель на экземпляр класса std::string в GCC указывает не на начало самой структуры, 
а на указатель на буфера.
В libstdc++-v3\textbackslash{}include\textbackslash{}bits\textbackslash{}basic\_string.h 
мы можем прочитать что это сделано для удобства отладки:

\begin{lstlisting}
   *  The reason you want _M_data pointing to the character %array and
   *  not the _Rep is so that the debugger can see the string
   *  contents. (Probably we should add a non-inline member to get
   *  the _Rep for the debugger to use, so users can check the actual
   *  string length.)
\end{lstlisting}

\href{http://go.yurichev.com/17085}{исходный код basic\_string.h}

В нашем примере мы учитываем это:

\lstinputlisting[caption=пример для GCC,style=customc]{\CURPATH/STL/string/GCC_RU.cpp}

Нужны еще небольшие хаки чтобы сымитировать типичную ошибку, которую мы уже видели выше, из-за
более ужесточенной проверки типов в GCC, тем не менее, printf() работает и здесь без c\_str().

\myparagraph{Чуть более сложный пример}

\lstinputlisting[style=customc]{\CURPATH/STL/string/3.cpp}

\lstinputlisting[caption=MSVC 2012,style=customasmx86]{\CURPATH/STL/string/3_MSVC_RU.asm}

Собственно, компилятор не конструирует строки статически: да в общем-то и как
это возможно, если буфер с ней нужно хранить в \glslink{heap}{куче}?

Вместо этого в сегменте данных хранятся обычные \ac{ASCIIZ}-строки, а позже, во время выполнения, 
при помощи метода \q{assign}, конструируются строки s1 и s2
.
При помощи \TT{operator+}, создается строка s3.

Обратите внимание на то что вызов метода c\_str() отсутствует,
потому что его код достаточно короткий и компилятор вставил его прямо здесь:
если строка короче 16-и байт, то в регистре EAX остается указатель на буфер,
а если длиннее, то из этого же места достается адрес на буфер расположенный в \glslink{heap}{куче}.

Далее следуют вызовы трех деструкторов, причем, они вызываются только если строка длиннее 16-и байт:
тогда нужно освободить буфера в \glslink{heap}{куче}.
В противном случае, так как все три объекта std::string хранятся в стеке,
они освобождаются автоматически после выхода из функции.

Следовательно, работа с короткими строками более быстрая из-за м\'{е}ньшего обращения к \glslink{heap}{куче}.

Код на GCC даже проще (из-за того, что в GCC, как мы уже видели, не реализована возможность хранить короткую
строку прямо в структуре):

% TODO1 comment each function meaning
\lstinputlisting[caption=GCC 4.8.1,style=customasmx86]{\CURPATH/STL/string/3_GCC_RU.s}

Можно заметить, что в деструкторы передается не указатель на объект,
а указатель на место за 12 байт (или 3 слова) перед ним, то есть, на настоящее начало структуры.

\myparagraph{std::string как глобальная переменная}
\label{sec:std_string_as_global_variable}

Опытные программисты на \Cpp знают, что глобальные переменные \ac{STL}-типов вполне можно объявлять.

Да, действительно:

\lstinputlisting[style=customc]{\CURPATH/STL/string/5.cpp}

Но как и где будет вызываться конструктор \TT{std::string}?

На самом деле, эта переменная будет инициализирована даже перед началом \main.

\lstinputlisting[caption=MSVC 2012: здесь конструируется глобальная переменная{,} а также регистрируется её деструктор,style=customasmx86]{\CURPATH/STL/string/5_MSVC_p2.asm}

\lstinputlisting[caption=MSVC 2012: здесь глобальная переменная используется в \main,style=customasmx86]{\CURPATH/STL/string/5_MSVC_p1.asm}

\lstinputlisting[caption=MSVC 2012: эта функция-деструктор вызывается перед выходом,style=customasmx86]{\CURPATH/STL/string/5_MSVC_p3.asm}

\myindex{\CStandardLibrary!atexit()}
В реальности, из \ac{CRT}, еще до вызова main(), вызывается специальная функция,
в которой перечислены все конструкторы подобных переменных.
Более того: при помощи atexit() регистрируется функция, которая будет вызвана в конце работы программы:
в этой функции компилятор собирает вызовы деструкторов всех подобных глобальных переменных.

GCC работает похожим образом:

\lstinputlisting[caption=GCC 4.8.1,style=customasmx86]{\CURPATH/STL/string/5_GCC.s}

Но он не выделяет отдельной функции в которой будут собраны деструкторы: 
каждый деструктор передается в atexit() по одному.

% TODO а если глобальная STL-переменная в другом модуле? надо проверить.

}
\DE{\subsection{Einfachste XOR-Verschlüsselung überhaupt}

Ich habe einmal eine Software gesehen, bei der alle Debugging-Ausgaben mit XOR mit dem Wert 3
verschlüsselt wurden. Mit anderen Worten, die beiden niedrigsten Bits aller Buchstaben wurden invertiert.

``Hello, world'' wurde zu ``Kfool/\#tlqog'':

\begin{lstlisting}
#!/usr/bin/python

msg="Hello, world!"

print "".join(map(lambda x: chr(ord(x)^3), msg))
\end{lstlisting}

Das ist eine ziemlich interessante Verschlüsselung (oder besser eine Verschleierung),
weil sie zwei wichtige Eigenschaften hat:
1) es ist eine einzige Funktion zum Verschlüsseln und entschlüsseln, sie muss nur wiederholt angewendet werden
2) die entstehenden Buchstaben befinden sich im druckbaren Bereich, also die ganze Zeichenkette kann ohne
Escape-Symbole im Code verwendet werden.

Die zweite Eigenschaft nutzt die Tatsache, dass alle druckbaren Zeichen in Reihen organisiert sind: 0x2x-0x7x,
und wenn die beiden niederwertigsten Bits invertiert werden, wird der Buchstabe um eine oder drei Stellen nach
links oder rechts \IT{verschoben}, aber niemals in eine andere Reihe:

\begin{figure}[H]
\centering
\includegraphics[width=0.7\textwidth]{ascii_clean.png}
\caption{7-Bit \ac{ASCII} Tabelle in Emacs}
\end{figure}

\dots mit dem Zeichen 0x7F als einziger Ausnahme.

Im Folgenden werden also beispielsweise die Zeichen A-Z \IT{verschlüsselt}:

\begin{lstlisting}
#!/usr/bin/python

msg="@ABCDEFGHIJKLMNO"

print "".join(map(lambda x: chr(ord(x)^3), msg))
\end{lstlisting}

Ergebnis:
% FIXME \verb  --  relevant comment for German?
\begin{lstlisting}
CBA@GFEDKJIHONML
\end{lstlisting}

Es sieht so aus als würden die Zeichen ``@'' und ``C'' sowie ``B'' und ``A'' vertauscht werden.

Hier ist noch ein interessantes Beispiel, in dem gezeigt wird, wie die Eigenschaften von XOR
ausgenutzt werden können: Exakt den gleichen Effekt, dass druckbare Zeichen auch druckbar bleiben,
kann man dadurch erzielen, dass irgendeine Kombination der niedrigsten vier Bits invertiert wird.
}

\EN{\section{Returning Values}
\label{ret_val_func}

Another simple function is the one that simply returns a constant value:

\lstinputlisting[caption=\EN{\CCpp Code},style=customc]{patterns/011_ret/1.c}

Let's compile it.

\subsection{x86}

Here's what both the GCC and MSVC compilers produce (with optimization) on the x86 platform:

\lstinputlisting[caption=\Optimizing GCC/MSVC (\assemblyOutput),style=customasmx86]{patterns/011_ret/1.s}

\myindex{x86!\Instructions!RET}
There are just two instructions: the first places the value 123 into the \EAX register,
which is used by convention for storing the return
value, and the second one is \RET, which returns execution to the \gls{caller}.

The caller will take the result from the \EAX register.

\subsection{ARM}

There are a few differences on the ARM platform:

\lstinputlisting[caption=\OptimizingKeilVI (\ARMMode) ASM Output,style=customasmARM]{patterns/011_ret/1_Keil_ARM_O3.s}

ARM uses the register \Reg{0} for returning the results of functions, so 123 is copied into \Reg{0}.

\myindex{ARM!\Instructions!MOV}
\myindex{x86!\Instructions!MOV}
It is worth noting that \MOV is a misleading name for the instruction in both the x86 and ARM \ac{ISA}s.

The data is not in fact \IT{moved}, but \IT{copied}.

\subsection{MIPS}

\label{MIPS_leaf_function_ex1}

The GCC assembly output below lists registers by number:

\lstinputlisting[caption=\Optimizing GCC 4.4.5 (\assemblyOutput),style=customasmMIPS]{patterns/011_ret/MIPS.s}

\dots while \IDA does it by their pseudo names:

\lstinputlisting[caption=\Optimizing GCC 4.4.5 (IDA),style=customasmMIPS]{patterns/011_ret/MIPS_IDA.lst}

The \$2 (or \$V0) register is used to store the function's return value.
\myindex{MIPS!\Pseudoinstructions!LI}
\INS{LI} stands for ``Load Immediate'' and is the MIPS equivalent to \MOV.

\myindex{MIPS!\Instructions!J}
The other instruction is the jump instruction (J or JR) which returns the execution flow to the \gls{caller}.

\myindex{MIPS!Branch delay slot}
You might be wondering why the positions of the load instruction (LI) and the jump instruction (J or JR) are swapped. This is due to a \ac{RISC} feature called ``branch delay slot''.

The reason this happens is a quirk in the architecture of some RISC \ac{ISA}s and isn't important for our
purposes---we must simply keep in mind that in MIPS, the instruction following a jump or branch instruction
is executed \IT{before} the jump/branch instruction itself.

As a consequence, branch instructions always swap places with the instruction executed immediately beforehand.


In practice, functions which merely return 1 (\IT{true}) or 0 (\IT{false}) are very frequent.

The smallest ever of the standard UNIX utilities, \IT{/bin/true} and \IT{/bin/false} return 0 and 1 respectively, as an exit code.
(Zero as an exit code usually means success, non-zero means error.)
}
\RU{\subsubsection{std::string}
\myindex{\Cpp!STL!std::string}
\label{std_string}

\myparagraph{Как устроена структура}

Многие строковые библиотеки \InSqBrackets{\CNotes 2.2} обеспечивают структуру содержащую ссылку 
на буфер собственно со строкой, переменная всегда содержащую длину строки 
(что очень удобно для массы функций \InSqBrackets{\CNotes 2.2.1}) и переменную содержащую текущий размер буфера.

Строка в буфере обыкновенно оканчивается нулем: это для того чтобы указатель на буфер можно было
передавать в функции требующие на вход обычную сишную \ac{ASCIIZ}-строку.

Стандарт \Cpp не описывает, как именно нужно реализовывать std::string,
но, как правило, они реализованы как описано выше, с небольшими дополнениями.

Строки в \Cpp это не класс (как, например, QString в Qt), а темплейт (basic\_string), 
это сделано для того чтобы поддерживать 
строки содержащие разного типа символы: как минимум \Tchar и \IT{wchar\_t}.

Так что, std::string это класс с базовым типом \Tchar.

А std::wstring это класс с базовым типом \IT{wchar\_t}.

\mysubparagraph{MSVC}

В реализации MSVC, вместо ссылки на буфер может содержаться сам буфер (если строка короче 16-и символов).

Это означает, что каждая короткая строка будет занимать в памяти по крайней мере $16 + 4 + 4 = 24$ 
байт для 32-битной среды либо $16 + 8 + 8 = 32$ 
байта в 64-битной, а если строка длиннее 16-и символов, то прибавьте еще длину самой строки.

\lstinputlisting[caption=пример для MSVC,style=customc]{\CURPATH/STL/string/MSVC_RU.cpp}

Собственно, из этого исходника почти всё ясно.

Несколько замечаний:

Если строка короче 16-и символов, 
то отдельный буфер для строки в \glslink{heap}{куче} выделяться не будет.

Это удобно потому что на практике, основная часть строк действительно короткие.
Вероятно, разработчики в Microsoft выбрали размер в 16 символов как разумный баланс.

Теперь очень важный момент в конце функции main(): мы не пользуемся методом c\_str(), тем не менее,
если это скомпилировать и запустить, то обе строки появятся в консоли!

Работает это вот почему.

В первом случае строка короче 16-и символов и в начале объекта std::string (его можно рассматривать
просто как структуру) расположен буфер с этой строкой.
\printf трактует указатель как указатель на массив символов оканчивающийся нулем и поэтому всё работает.

Вывод второй строки (длиннее 16-и символов) даже еще опаснее: это вообще типичная программистская ошибка 
(или опечатка), забыть дописать c\_str().
Это работает потому что в это время в начале структуры расположен указатель на буфер.
Это может надолго остаться незамеченным: до тех пока там не появится строка 
короче 16-и символов, тогда процесс упадет.

\mysubparagraph{GCC}

В реализации GCC в структуре есть еще одна переменная --- reference count.

Интересно, что указатель на экземпляр класса std::string в GCC указывает не на начало самой структуры, 
а на указатель на буфера.
В libstdc++-v3\textbackslash{}include\textbackslash{}bits\textbackslash{}basic\_string.h 
мы можем прочитать что это сделано для удобства отладки:

\begin{lstlisting}
   *  The reason you want _M_data pointing to the character %array and
   *  not the _Rep is so that the debugger can see the string
   *  contents. (Probably we should add a non-inline member to get
   *  the _Rep for the debugger to use, so users can check the actual
   *  string length.)
\end{lstlisting}

\href{http://go.yurichev.com/17085}{исходный код basic\_string.h}

В нашем примере мы учитываем это:

\lstinputlisting[caption=пример для GCC,style=customc]{\CURPATH/STL/string/GCC_RU.cpp}

Нужны еще небольшие хаки чтобы сымитировать типичную ошибку, которую мы уже видели выше, из-за
более ужесточенной проверки типов в GCC, тем не менее, printf() работает и здесь без c\_str().

\myparagraph{Чуть более сложный пример}

\lstinputlisting[style=customc]{\CURPATH/STL/string/3.cpp}

\lstinputlisting[caption=MSVC 2012,style=customasmx86]{\CURPATH/STL/string/3_MSVC_RU.asm}

Собственно, компилятор не конструирует строки статически: да в общем-то и как
это возможно, если буфер с ней нужно хранить в \glslink{heap}{куче}?

Вместо этого в сегменте данных хранятся обычные \ac{ASCIIZ}-строки, а позже, во время выполнения, 
при помощи метода \q{assign}, конструируются строки s1 и s2
.
При помощи \TT{operator+}, создается строка s3.

Обратите внимание на то что вызов метода c\_str() отсутствует,
потому что его код достаточно короткий и компилятор вставил его прямо здесь:
если строка короче 16-и байт, то в регистре EAX остается указатель на буфер,
а если длиннее, то из этого же места достается адрес на буфер расположенный в \glslink{heap}{куче}.

Далее следуют вызовы трех деструкторов, причем, они вызываются только если строка длиннее 16-и байт:
тогда нужно освободить буфера в \glslink{heap}{куче}.
В противном случае, так как все три объекта std::string хранятся в стеке,
они освобождаются автоматически после выхода из функции.

Следовательно, работа с короткими строками более быстрая из-за м\'{е}ньшего обращения к \glslink{heap}{куче}.

Код на GCC даже проще (из-за того, что в GCC, как мы уже видели, не реализована возможность хранить короткую
строку прямо в структуре):

% TODO1 comment each function meaning
\lstinputlisting[caption=GCC 4.8.1,style=customasmx86]{\CURPATH/STL/string/3_GCC_RU.s}

Можно заметить, что в деструкторы передается не указатель на объект,
а указатель на место за 12 байт (или 3 слова) перед ним, то есть, на настоящее начало структуры.

\myparagraph{std::string как глобальная переменная}
\label{sec:std_string_as_global_variable}

Опытные программисты на \Cpp знают, что глобальные переменные \ac{STL}-типов вполне можно объявлять.

Да, действительно:

\lstinputlisting[style=customc]{\CURPATH/STL/string/5.cpp}

Но как и где будет вызываться конструктор \TT{std::string}?

На самом деле, эта переменная будет инициализирована даже перед началом \main.

\lstinputlisting[caption=MSVC 2012: здесь конструируется глобальная переменная{,} а также регистрируется её деструктор,style=customasmx86]{\CURPATH/STL/string/5_MSVC_p2.asm}

\lstinputlisting[caption=MSVC 2012: здесь глобальная переменная используется в \main,style=customasmx86]{\CURPATH/STL/string/5_MSVC_p1.asm}

\lstinputlisting[caption=MSVC 2012: эта функция-деструктор вызывается перед выходом,style=customasmx86]{\CURPATH/STL/string/5_MSVC_p3.asm}

\myindex{\CStandardLibrary!atexit()}
В реальности, из \ac{CRT}, еще до вызова main(), вызывается специальная функция,
в которой перечислены все конструкторы подобных переменных.
Более того: при помощи atexit() регистрируется функция, которая будет вызвана в конце работы программы:
в этой функции компилятор собирает вызовы деструкторов всех подобных глобальных переменных.

GCC работает похожим образом:

\lstinputlisting[caption=GCC 4.8.1,style=customasmx86]{\CURPATH/STL/string/5_GCC.s}

Но он не выделяет отдельной функции в которой будут собраны деструкторы: 
каждый деструктор передается в atexit() по одному.

% TODO а если глобальная STL-переменная в другом модуле? надо проверить.

}
\DE{\subsection{Einfachste XOR-Verschlüsselung überhaupt}

Ich habe einmal eine Software gesehen, bei der alle Debugging-Ausgaben mit XOR mit dem Wert 3
verschlüsselt wurden. Mit anderen Worten, die beiden niedrigsten Bits aller Buchstaben wurden invertiert.

``Hello, world'' wurde zu ``Kfool/\#tlqog'':

\begin{lstlisting}
#!/usr/bin/python

msg="Hello, world!"

print "".join(map(lambda x: chr(ord(x)^3), msg))
\end{lstlisting}

Das ist eine ziemlich interessante Verschlüsselung (oder besser eine Verschleierung),
weil sie zwei wichtige Eigenschaften hat:
1) es ist eine einzige Funktion zum Verschlüsseln und entschlüsseln, sie muss nur wiederholt angewendet werden
2) die entstehenden Buchstaben befinden sich im druckbaren Bereich, also die ganze Zeichenkette kann ohne
Escape-Symbole im Code verwendet werden.

Die zweite Eigenschaft nutzt die Tatsache, dass alle druckbaren Zeichen in Reihen organisiert sind: 0x2x-0x7x,
und wenn die beiden niederwertigsten Bits invertiert werden, wird der Buchstabe um eine oder drei Stellen nach
links oder rechts \IT{verschoben}, aber niemals in eine andere Reihe:

\begin{figure}[H]
\centering
\includegraphics[width=0.7\textwidth]{ascii_clean.png}
\caption{7-Bit \ac{ASCII} Tabelle in Emacs}
\end{figure}

\dots mit dem Zeichen 0x7F als einziger Ausnahme.

Im Folgenden werden also beispielsweise die Zeichen A-Z \IT{verschlüsselt}:

\begin{lstlisting}
#!/usr/bin/python

msg="@ABCDEFGHIJKLMNO"

print "".join(map(lambda x: chr(ord(x)^3), msg))
\end{lstlisting}

Ergebnis:
% FIXME \verb  --  relevant comment for German?
\begin{lstlisting}
CBA@GFEDKJIHONML
\end{lstlisting}

Es sieht so aus als würden die Zeichen ``@'' und ``C'' sowie ``B'' und ``A'' vertauscht werden.

Hier ist noch ein interessantes Beispiel, in dem gezeigt wird, wie die Eigenschaften von XOR
ausgenutzt werden können: Exakt den gleichen Effekt, dass druckbare Zeichen auch druckbar bleiben,
kann man dadurch erzielen, dass irgendeine Kombination der niedrigsten vier Bits invertiert wird.
}

\ifdefined\SPANISH
\chapter{Patrones de código}
\fi % SPANISH

\ifdefined\GERMAN
\chapter{Code-Muster}
\fi % GERMAN

\ifdefined\ENGLISH
\chapter{Code Patterns}
\fi % ENGLISH

\ifdefined\ITALIAN
\chapter{Forme di codice}
\fi % ITALIAN

\ifdefined\RUSSIAN
\chapter{Образцы кода}
\fi % RUSSIAN

\ifdefined\BRAZILIAN
\chapter{Padrões de códigos}
\fi % BRAZILIAN

\ifdefined\THAI
\chapter{รูปแบบของโค้ด}
\fi % THAI

\ifdefined\FRENCH
\chapter{Modèle de code}
\fi % FRENCH

\ifdefined\POLISH
\chapter{\PLph{}}
\fi % POLISH

% sections
\EN{\input{patterns/patterns_opt_dbg_EN}}
\ES{\input{patterns/patterns_opt_dbg_ES}}
\ITA{\input{patterns/patterns_opt_dbg_ITA}}
\PTBR{\input{patterns/patterns_opt_dbg_PTBR}}
\RU{\input{patterns/patterns_opt_dbg_RU}}
\THA{\input{patterns/patterns_opt_dbg_THA}}
\DE{\input{patterns/patterns_opt_dbg_DE}}
\FR{\input{patterns/patterns_opt_dbg_FR}}
\PL{\input{patterns/patterns_opt_dbg_PL}}

\RU{\section{Некоторые базовые понятия}}
\EN{\section{Some basics}}
\DE{\section{Einige Grundlagen}}
\FR{\section{Quelques bases}}
\ES{\section{\ESph{}}}
\ITA{\section{Alcune basi teoriche}}
\PTBR{\section{\PTBRph{}}}
\THA{\section{\THAph{}}}
\PL{\section{\PLph{}}}

% sections:
\EN{\input{patterns/intro_CPU_ISA_EN}}
\ES{\input{patterns/intro_CPU_ISA_ES}}
\ITA{\input{patterns/intro_CPU_ISA_ITA}}
\PTBR{\input{patterns/intro_CPU_ISA_PTBR}}
\RU{\input{patterns/intro_CPU_ISA_RU}}
\DE{\input{patterns/intro_CPU_ISA_DE}}
\FR{\input{patterns/intro_CPU_ISA_FR}}
\PL{\input{patterns/intro_CPU_ISA_PL}}

\EN{\input{patterns/numeral_EN}}
\RU{\input{patterns/numeral_RU}}
\ITA{\input{patterns/numeral_ITA}}
\DE{\input{patterns/numeral_DE}}
\FR{\input{patterns/numeral_FR}}
\PL{\input{patterns/numeral_PL}}

% chapters
\input{patterns/00_empty/main}
\input{patterns/011_ret/main}
\input{patterns/01_helloworld/main}
\input{patterns/015_prolog_epilogue/main}
\input{patterns/02_stack/main}
\input{patterns/03_printf/main}
\input{patterns/04_scanf/main}
\input{patterns/05_passing_arguments/main}
\input{patterns/06_return_results/main}
\input{patterns/061_pointers/main}
\input{patterns/065_GOTO/main}
\input{patterns/07_jcc/main}
\input{patterns/08_switch/main}
\input{patterns/09_loops/main}
\input{patterns/10_strings/main}
\input{patterns/11_arith_optimizations/main}
\input{patterns/12_FPU/main}
\input{patterns/13_arrays/main}
\input{patterns/14_bitfields/main}
\EN{\input{patterns/145_LCG/main_EN}}
\RU{\input{patterns/145_LCG/main_RU}}
\input{patterns/15_structs/main}
\input{patterns/17_unions/main}
\input{patterns/18_pointers_to_functions/main}
\input{patterns/185_64bit_in_32_env/main}

\EN{\input{patterns/19_SIMD/main_EN}}
\RU{\input{patterns/19_SIMD/main_RU}}
\DE{\input{patterns/19_SIMD/main_DE}}

\EN{\input{patterns/20_x64/main_EN}}
\RU{\input{patterns/20_x64/main_RU}}

\EN{\input{patterns/205_floating_SIMD/main_EN}}
\RU{\input{patterns/205_floating_SIMD/main_RU}}
\DE{\input{patterns/205_floating_SIMD/main_DE}}

\EN{\input{patterns/ARM/main_EN}}
\RU{\input{patterns/ARM/main_RU}}
\DE{\input{patterns/ARM/main_DE}}

\input{patterns/MIPS/main}


\ifdefined\SPANISH
\chapter{Patrones de código}
\fi % SPANISH

\ifdefined\GERMAN
\chapter{Code-Muster}
\fi % GERMAN

\ifdefined\ENGLISH
\chapter{Code Patterns}
\fi % ENGLISH

\ifdefined\ITALIAN
\chapter{Forme di codice}
\fi % ITALIAN

\ifdefined\RUSSIAN
\chapter{Образцы кода}
\fi % RUSSIAN

\ifdefined\BRAZILIAN
\chapter{Padrões de códigos}
\fi % BRAZILIAN

\ifdefined\THAI
\chapter{รูปแบบของโค้ด}
\fi % THAI

\ifdefined\FRENCH
\chapter{Modèle de code}
\fi % FRENCH

\ifdefined\POLISH
\chapter{\PLph{}}
\fi % POLISH

% sections
\EN{\section{The method}

When the author of this book first started learning C and, later, \Cpp, he used to write small pieces of code, compile them,
and then look at the assembly language output. This made it very easy for him to understand what was going on in the code that he had written.
\footnote{In fact, he still does this when he can't understand what a particular bit of code does.}.
He did this so many times that the relationship between the \CCpp code and what the compiler produced was imprinted deeply in his mind.
It's now easy for him to imagine instantly a rough outline of a C code's appearance and function.
Perhaps this technique could be helpful for others.

%There are a lot of examples for both x86/x64 and ARM.
%Those who already familiar with one of architectures, may freely skim over pages.

By the way, there is a great website where you can do the same, with various compilers, instead of installing them on your box.
You can use it as well: \url{https://gcc.godbolt.org/}.

\section*{\Exercises}

When the author of this book studied assembly language, he also often compiled small C functions and then rewrote
them gradually to assembly, trying to make their code as short as possible.
This probably is not worth doing in real-world scenarios today,
because it's hard to compete with the latest compilers in terms of efficiency. It is, however, a very good way to gain a better understanding of assembly.
Feel free, therefore, to take any assembly code from this book and try to make it shorter.
However, don't forget to test what you have written.

% rewrote to show that debug\release and optimisations levels are orthogonal concepts.
\section*{Optimization levels and debug information}

Source code can be compiled by different compilers with various optimization levels.
A typical compiler has about three such levels, where level zero means that optimization is completely disabled.
Optimization can also be targeted towards code size or code speed.
A non-optimizing compiler is faster and produces more understandable (albeit verbose) code,
whereas an optimizing compiler is slower and tries to produce code that runs faster (but is not necessarily more compact).
In addition to optimization levels, a compiler can include some debug information in the resulting file,
producing code that is easy to debug.
One of the important features of the ´debug' code is that it might contain links
between each line of the source code and its respective machine code address.
Optimizing compilers, on the other hand, tend to produce output where entire lines of source code
can be optimized away and thus not even be present in the resulting machine code.
Reverse engineers can encounter either version, simply because some developers turn on the compiler's optimization flags and others do not.
Because of this, we'll try to work on examples of both debug and release versions of the code featured in this book, wherever possible.

Sometimes some pretty ancient compilers are used in this book, in order to get the shortest (or simplest) possible code snippet.
}
\ES{\input{patterns/patterns_opt_dbg_ES}}
\ITA{\input{patterns/patterns_opt_dbg_ITA}}
\PTBR{\input{patterns/patterns_opt_dbg_PTBR}}
\RU{\input{patterns/patterns_opt_dbg_RU}}
\THA{\input{patterns/patterns_opt_dbg_THA}}
\DE{\input{patterns/patterns_opt_dbg_DE}}
\FR{\input{patterns/patterns_opt_dbg_FR}}
\PL{\input{patterns/patterns_opt_dbg_PL}}

\RU{\section{Некоторые базовые понятия}}
\EN{\section{Some basics}}
\DE{\section{Einige Grundlagen}}
\FR{\section{Quelques bases}}
\ES{\section{\ESph{}}}
\ITA{\section{Alcune basi teoriche}}
\PTBR{\section{\PTBRph{}}}
\THA{\section{\THAph{}}}
\PL{\section{\PLph{}}}

% sections:
\EN{\input{patterns/intro_CPU_ISA_EN}}
\ES{\input{patterns/intro_CPU_ISA_ES}}
\ITA{\input{patterns/intro_CPU_ISA_ITA}}
\PTBR{\input{patterns/intro_CPU_ISA_PTBR}}
\RU{\input{patterns/intro_CPU_ISA_RU}}
\DE{\input{patterns/intro_CPU_ISA_DE}}
\FR{\input{patterns/intro_CPU_ISA_FR}}
\PL{\input{patterns/intro_CPU_ISA_PL}}

\EN{\subsection{Numeral Systems}

Humans have become accustomed to a decimal numeral system, probably because almost everyone has 10 fingers.
Nevertheless, the number \q{10} has no significant meaning in science and mathematics.
The natural numeral system in digital electronics is binary: 0 is for an absence of current in the wire, and 1 for presence.
10 in binary is 2 in decimal, 100 in binary is 4 in decimal, and so on.

% This sentence is a bit unweildy - maybe try 'Our ten-digit system would be described as having a radix...' - Renaissance
If the numeral system has 10 digits, it has a \IT{radix} (or \IT{base}) of 10.
The binary numeral system has a \IT{radix} of 2.

Important things to recall:

1) A \IT{number} is a number, while a \IT{digit} is a term from writing systems, and is usually one character

% The original is 'number' is not changed; I think the intent is value, and changed it - Renaissance
2) The value of a number does not change when converted to another radix; only the writing notation for that value has changed (and therefore the way of representing it in \ac{RAM}).

\subsection{Converting From One Radix To Another}

Positional notation is used almost every numerical system. This means that a digit has weight relative to where it is placed inside of the larger number.
If 2 is placed at the rightmost place, it's 2, but if it's placed one digit before rightmost, it's 20.

What does $1234$ stand for?

$10^3 \cdot 1 + 10^2 \cdot 2 + 10^1 \cdot 3 + 1 \cdot 4 = 1234$ or
$1000 \cdot 1 + 100 \cdot 2 + 10 \cdot 3 + 4 = 1234$

It's the same story for binary numbers, but the base is 2 instead of 10.
What does 0b101011 stand for?

$2^5 \cdot 1 + 2^4 \cdot 0 + 2^3 \cdot 1 + 2^2 \cdot 0 + 2^1 \cdot 1 + 2^0 \cdot 1 = 43$ or
$32 \cdot 1 + 16 \cdot 0 + 8 \cdot 1 + 4 \cdot 0 + 2 \cdot 1 + 1 = 43$

There is such a thing as non-positional notation, such as the Roman numeral system.
\footnote{About numeric system evolution, see \InSqBrackets{\TAOCPvolII{}, 195--213.}}.
% Maybe add a sentence to fill in that X is always 10, and is therefore non-positional, even though putting an I before subtracts and after adds, and is in that sense positional
Perhaps, humankind switched to positional notation because it's easier to do basic operations (addition, multiplication, etc.) on paper by hand.

Binary numbers can be added, subtracted and so on in the very same as taught in schools, but only 2 digits are available.

Binary numbers are bulky when represented in source code and dumps, so that is where the hexadecimal numeral system can be useful.
A hexadecimal radix uses the digits 0..9, and also 6 Latin characters: A..F.
Each hexadecimal digit takes 4 bits or 4 binary digits, so it's very easy to convert from binary number to hexadecimal and back, even manually, in one's mind.

\begin{center}
\begin{longtable}{ | l | l | l | }
\hline
\HeaderColor hexadecimal & \HeaderColor binary & \HeaderColor decimal \\
\hline
0	&0000	&0 \\
1	&0001	&1 \\
2	&0010	&2 \\
3	&0011	&3 \\
4	&0100	&4 \\
5	&0101	&5 \\
6	&0110	&6 \\
7	&0111	&7 \\
8	&1000	&8 \\
9	&1001	&9 \\
A	&1010	&10 \\
B	&1011	&11 \\
C	&1100	&12 \\
D	&1101	&13 \\
E	&1110	&14 \\
F	&1111	&15 \\
\hline
\end{longtable}
\end{center}

How can one tell which radix is being used in a specific instance?

Decimal numbers are usually written as is, i.e., 1234. Some assemblers allow an identifier on decimal radix numbers, in which the number would be written with a "d" suffix: 1234d.

Binary numbers are sometimes prepended with the "0b" prefix: 0b100110111 (\ac{GCC} has a non-standard language extension for this\footnote{\url{https://gcc.gnu.org/onlinedocs/gcc/Binary-constants.html}}).
There is also another way: using a "b" suffix, for example: 100110111b.
This book tries to use the "0b" prefix consistently throughout the book for binary numbers.

Hexadecimal numbers are prepended with "0x" prefix in \CCpp and other \ac{PL}s: 0x1234ABCD.
Alternatively, they are given a "h" suffix: 1234ABCDh. This is common way of representing them in assemblers and debuggers.
In this convention, if the number is started with a Latin (A..F) digit, a 0 is added at the beginning: 0ABCDEFh.
There was also convention that was popular in 8-bit home computers era, using \$ prefix, like \$ABCD.
The book will try to stick to "0x" prefix throughout the book for hexadecimal numbers.

Should one learn to convert numbers mentally? A table of 1-digit hexadecimal numbers can easily be memorized.
As for larger numbers, it's probably not worth tormenting yourself.

Perhaps the most visible hexadecimal numbers are in \ac{URL}s.
This is the way that non-Latin characters are encoded.
For example:
\url{https://en.wiktionary.org/wiki/na\%C3\%AFvet\%C3\%A9} is the \ac{URL} of Wiktionary article about \q{naïveté} word.

\subsubsection{Octal Radix}

Another numeral system heavily used in the past of computer programming is octal. In octal there are 8 digits (0..7), and each is mapped to 3 bits, so it's easy to convert numbers back and forth.
It has been superseded by the hexadecimal system almost everywhere, but, surprisingly, there is a *NIX utility, used often by many people, which takes octal numbers as argument: \TT{chmod}.

\myindex{UNIX!chmod}
As many *NIX users know, \TT{chmod} argument can be a number of 3 digits. The first digit represents the rights of the owner of the file (read, write and/or execute), the second is the rights for the group to which the file belongs, and the third is for everyone else.
Each digit that \TT{chmod} takes can be represented in binary form:

\begin{center}
\begin{longtable}{ | l | l | l | }
\hline
\HeaderColor decimal & \HeaderColor binary & \HeaderColor meaning \\
\hline
7	&111	&\textbf{rwx} \\
6	&110	&\textbf{rw-} \\
5	&101	&\textbf{r-x} \\
4	&100	&\textbf{r-{}-} \\
3	&011	&\textbf{-wx} \\
2	&010	&\textbf{-w-} \\
1	&001	&\textbf{-{}-x} \\
0	&000	&\textbf{-{}-{}-} \\
\hline
\end{longtable}
\end{center}

So each bit is mapped to a flag: read/write/execute.

The importance of \TT{chmod} here is that the whole number in argument can be represented as octal number.
Let's take, for example, 644.
When you run \TT{chmod 644 file}, you set read/write permissions for owner, read permissions for group and again, read permissions for everyone else.
If we convert the octal number 644 to binary, it would be \TT{110100100}, or, in groups of 3 bits, \TT{110 100 100}.

Now we see that each triplet describe permissions for owner/group/others: first is \TT{rw-}, second is \TT{r--} and third is \TT{r--}.

The octal numeral system was also popular on old computers like PDP-8, because word there could be 12, 24 or 36 bits, and these numbers are all divisible by 3, so the octal system was natural in that environment.
Nowadays, all popular computers employ word/address sizes of 16, 32 or 64 bits, and these numbers are all divisible by 4, so the hexadecimal system is more natural there.

The octal numeral system is supported by all standard \CCpp compilers.
This is a source of confusion sometimes, because octal numbers are encoded with a zero prepended, for example, 0377 is 255.
Sometimes, you might make a typo and write "09" instead of 9, and the compiler would report an error.
GCC might report something like this:\\
\TT{error: invalid digit "9" in octal constant}.

Also, the octal system is somewhat popular in Java. When the IDA shows Java strings with non-printable characters,
they are encoded in the octal system instead of hexadecimal.
\myindex{JAD}
The JAD Java decompiler behaves the same way.

\subsubsection{Divisibility}

When you see a decimal number like 120, you can quickly deduce that it's divisible by 10, because the last digit is zero.
In the same way, 123400 is divisible by 100, because the two last digits are zeros.

Likewise, the hexadecimal number 0x1230 is divisible by 0x10 (or 16), 0x123000 is divisible by 0x1000 (or 4096), etc.

The binary number 0b1000101000 is divisible by 0b1000 (8), etc.

This property can often be used to quickly realize if the size of some block in memory is padded to some boundary.
For example, sections in \ac{PE} files are almost always started at addresses ending with 3 hexadecimal zeros: 0x41000, 0x10001000, etc.
The reason behind this is the fact that almost all \ac{PE} sections are padded to a boundary of 0x1000 (4096) bytes.

\subsubsection{Multi-Precision Arithmetic and Radix}

\index{RSA}
Multi-precision arithmetic can use huge numbers, and each one may be stored in several bytes.
For example, RSA keys, both public and private, span up to 4096 bits, and maybe even more.

% I'm not sure how to change this, but the normal format for quoting would be just to mention the author or book, and footnote to the full reference
In \InSqBrackets{\TAOCPvolII, 265} we find the following idea: when you store a multi-precision number in several bytes,
the whole number can be represented as having a radix of $2^8=256$, and each digit goes to the corresponding byte.
Likewise, if you store a multi-precision number in several 32-bit integer values, each digit goes to each 32-bit slot,
and you may think about this number as stored in radix of $2^{32}$.

\subsubsection{How to Pronounce Non-Decimal Numbers}

Numbers in a non-decimal base are usually pronounced by digit by digit: ``one-zero-zero-one-one-...''.
Words like ``ten'' and ``thousand'' are usually not pronounced, to prevent confusion with the decimal base system.

\subsubsection{Floating point numbers}

To distinguish floating point numbers from integers, they are usually written with ``.0'' at the end,
like $0.0$, $123.0$, etc.
}
\RU{\subsection{Представление чисел}

Люди привыкли к десятичной системе счисления вероятно потому что почти у каждого есть по 10 пальцев.
Тем не менее, число 10 не имеет особого значения в науке и математике.
Двоичная система естествена для цифровой электроники: 0 означает отсутствие тока в проводе и 1 --- его присутствие.
10 в двоичной системе это 2 в десятичной; 100 в двоичной это 4 в десятичной, итд.

Если в системе счисления есть 10 цифр, её \IT{основание} или \IT{radix} это 10.
Двоичная система имеет \IT{основание} 2.

Важные вещи, которые полезно вспомнить:
1) \IT{число} это число, в то время как \IT{цифра} это термин из системы письменности, и это обычно один символ;
2) само число не меняется, когда конвертируется из одного основания в другое: меняется способ его записи (или представления
в памяти).

Как сконвертировать число из одного основания в другое?

Позиционная нотация используется почти везде, это означает, что всякая цифра имеет свой вес, в зависимости от её расположения
внутри числа.
Если 2 расположена в самом последнем месте справа, это 2.
Если она расположена в месте перед последним, это 20.

Что означает $1234$?

$10^3 \cdot 1 + 10^2 \cdot 2 + 10^1 \cdot 3 + 1 \cdot 4$ = 1234 или
$1000 \cdot 1 + 100 \cdot 2 + 10 \cdot 3 + 4 = 1234$

Та же история и для двоичных чисел, только основание там 2 вместо 10.
Что означает 0b101011?

$2^5 \cdot 1 + 2^4 \cdot 0 + 2^3 \cdot 1 + 2^2 \cdot 0 + 2^1 \cdot 1 + 2^0 \cdot 1 = 43$ или
$32 \cdot 1 + 16 \cdot 0 + 8 \cdot 1 + 4 \cdot 0 + 2 \cdot 1 + 1 = 43$

Позиционную нотацию можно противопоставить непозиционной нотации, такой как римская система записи чисел
\footnote{Об эволюции способов записи чисел, см.также: \InSqBrackets{\TAOCPvolII{}, 195--213.}}.
Вероятно, человечество перешло на позиционную нотацию, потому что так проще работать с числами (сложение, умножение, итд)
на бумаге, в ручную.

Действительно, двоичные числа можно складывать, вычитать, итд, точно также, как этому обычно обучают в школах,
только доступны лишь 2 цифры.

Двоичные числа громоздки, когда их используют в исходных кодах и дампах, так что в этих случаях применяется шестнадцатеричная
система.
Используются цифры 0..9 и еще 6 латинских букв: A..F.
Каждая шестнадцатеричная цифра занимает 4 бита или 4 двоичных цифры, так что конвертировать из двоичной системы в
шестнадцатеричную и назад, можно легко вручную, или даже в уме.

\begin{center}
\begin{longtable}{ | l | l | l | }
\hline
\HeaderColor шестнадцатеричная & \HeaderColor двоичная & \HeaderColor десятичная \\
\hline
0	&0000	&0 \\
1	&0001	&1 \\
2	&0010	&2 \\
3	&0011	&3 \\
4	&0100	&4 \\
5	&0101	&5 \\
6	&0110	&6 \\
7	&0111	&7 \\
8	&1000	&8 \\
9	&1001	&9 \\
A	&1010	&10 \\
B	&1011	&11 \\
C	&1100	&12 \\
D	&1101	&13 \\
E	&1110	&14 \\
F	&1111	&15 \\
\hline
\end{longtable}
\end{center}

Как понять, какое основание используется в конкретном месте?

Десятичные числа обычно записываются как есть, т.е., 1234. Но некоторые ассемблеры позволяют подчеркивать
этот факт для ясности, и это число может быть дополнено суффиксом "d": 1234d.

К двоичным числам иногда спереди добавляют префикс "0b": 0b100110111
(В \ac{GCC} для этого есть нестандартное расширение языка
\footnote{\url{https://gcc.gnu.org/onlinedocs/gcc/Binary-constants.html}}).
Есть также еще один способ: суффикс "b", например: 100110111b.
В этой книге я буду пытаться придерживаться префикса "0b" для двоичных чисел.

Шестнадцатеричные числа имеют префикс "0x" в \CCpp и некоторых других \ac{PL}: 0x1234ABCD.
Либо они имеют суффикс "h": 1234ABCDh --- обычно так они представляются в ассемблерах и отладчиках.
Если число начинается с цифры A..F, перед ним добавляется 0: 0ABCDEFh.
Во времена 8-битных домашних компьютеров, был также способ записи чисел используя префикс \$, например, \$ABCD.
В книге я попытаюсь придерживаться префикса "0x" для шестнадцатеричных чисел.

Нужно ли учиться конвертировать числа в уме? Таблицу шестнадцатеричных чисел из одной цифры легко запомнить.
А запоминать б\'{о}льшие числа, наверное, не стоит.

Наверное, чаще всего шестнадцатеричные числа можно увидеть в \ac{URL}-ах.
Так кодируются буквы не из числа латинских.
Например:
\url{https://en.wiktionary.org/wiki/na\%C3\%AFvet\%C3\%A9} это \ac{URL} страницы в Wiktionary о слове \q{naïveté}.

\subsubsection{Восьмеричная система}

Еще одна система, которая в прошлом много использовалась в программировании это восьмеричная: есть 8 цифр (0..7) и каждая
описывает 3 бита, так что легко конвертировать числа туда и назад.
Она почти везде была заменена шестнадцатеричной, но удивительно, в *NIX имеется утилита использующаяся многими людьми,
которая принимает на вход восьмеричное число: \TT{chmod}.

\myindex{UNIX!chmod}
Как знают многие пользователи *NIX, аргумент \TT{chmod} это число из трех цифр. Первая цифра это права владельца файла,
вторая это права группы (которой файл принадлежит), третья для всех остальных.
И каждая цифра может быть представлена в двоичном виде:

\begin{center}
\begin{longtable}{ | l | l | l | }
\hline
\HeaderColor десятичная & \HeaderColor двоичная & \HeaderColor значение \\
\hline
7	&111	&\textbf{rwx} \\
6	&110	&\textbf{rw-} \\
5	&101	&\textbf{r-x} \\
4	&100	&\textbf{r-{}-} \\
3	&011	&\textbf{-wx} \\
2	&010	&\textbf{-w-} \\
1	&001	&\textbf{-{}-x} \\
0	&000	&\textbf{-{}-{}-} \\
\hline
\end{longtable}
\end{center}

Так что каждый бит привязан к флагу: read/write/execute (чтение/запись/исполнение).

И вот почему я вспомнил здесь о \TT{chmod}, это потому что всё число может быть представлено как число в восьмеричной системе.
Для примера возьмем 644.
Когда вы запускаете \TT{chmod 644 file}, вы выставляете права read/write для владельца, права read для группы, и снова,
read для всех остальных.
Сконвертируем число 644 из восьмеричной системы в двоичную, это будет \TT{110100100}, или (в группах по 3 бита) \TT{110 100 100}.

Теперь мы видим, что каждая тройка описывает права для владельца/группы/остальных:
первая это \TT{rw-}, вторая это \TT{r--} и третья это \TT{r--}.

Восьмеричная система была также популярная на старых компьютерах вроде PDP-8, потому что слово там могло содержать 12, 24 или
36 бит, и эти числа делятся на 3, так что выбор восьмеричной системы в той среде был логичен.
Сейчас, все популярные компьютеры имеют размер слова/адреса 16, 32 или 64 бита, и эти числа делятся на 4,
так что шестнадцатеричная система здесь удобнее.

Восьмеричная система поддерживается всеми стандартными компиляторами \CCpp{}.
Это иногда источник недоумения, потому что восьмеричные числа кодируются с нулем вперед, например, 0377 это 255.
И иногда, вы можете сделать опечатку, и написать "09" вместо 9, и компилятор выдаст ошибку.
GCC может выдать что-то вроде:\\
\TT{error: invalid digit "9" in octal constant}.

Также, восьмеричная система популярна в Java: когда IDA показывает строку с непечатаемыми символами,
они кодируются в восьмеричной системе вместо шестнадцатеричной.
\myindex{JAD}
Точно также себя ведет декомпилятор с Java JAD.

\subsubsection{Делимость}

Когда вы видите десятичное число вроде 120, вы можете быстро понять что оно делится на 10, потому что последняя цифра это 0.
Точно также, 123400 делится на 100, потому что две последних цифры это нули.

Точно также, шестнадцатеричное число 0x1230 делится на 0x10 (или 16), 0x123000 делится на 0x1000 (или 4096), итд.

Двоичное число 0b1000101000 делится на 0b1000 (8), итд.

Это свойство можно часто использовать, чтобы быстро понять,
что длина какого-либо блока в памяти выровнена по некоторой границе.
Например, секции в \ac{PE}-файлах почти всегда начинаются с адресов заканчивающихся 3 шестнадцатеричными нулями:
0x41000, 0x10001000, итд.
Причина в том, что почти все секции в \ac{PE} выровнены по границе 0x1000 (4096) байт.

\subsubsection{Арифметика произвольной точности и основание}

\index{RSA}
Арифметика произвольной точности (multi-precision arithmetic) может использовать огромные числа,
которые могут храниться в нескольких байтах.
Например, ключи RSA, и открытые и закрытые, могут занимать до 4096 бит и даже больше.

В \InSqBrackets{\TAOCPvolII, 265} можно найти такую идею: когда вы сохраняете число произвольной точности в нескольких байтах,
всё число может быть представлено как имеющую систему счисления по основанию $2^8=256$, и каждая цифра находится
в соответствующем байте.
Точно также, если вы сохраняете число произвольной точности в нескольких 32-битных целочисленных значениях,
каждая цифра отправляется в каждый 32-битный слот, и вы можете считать что это число записано в системе с основанием $2^{32}$.

\subsubsection{Произношение}

Числа в недесятичных системах счислениях обычно произносятся по одной цифре: ``один-ноль-ноль-один-один-...''.
Слова вроде ``десять'', ``тысяча'', итд, обычно не произносятся, потому что тогда можно спутать с десятичной системой.

\subsubsection{Числа с плавающей запятой}

Чтобы отличать числа с плавающей запятой от целочисленных, часто, в конце добавляют ``.0'',
например $0.0$, $123.0$, итд.

}
\ITA{\input{patterns/numeral_ITA}}
\DE{\input{patterns/numeral_DE}}
\FR{\input{patterns/numeral_FR}}
\PL{\input{patterns/numeral_PL}}

% chapters
\ifdefined\SPANISH
\chapter{Patrones de código}
\fi % SPANISH

\ifdefined\GERMAN
\chapter{Code-Muster}
\fi % GERMAN

\ifdefined\ENGLISH
\chapter{Code Patterns}
\fi % ENGLISH

\ifdefined\ITALIAN
\chapter{Forme di codice}
\fi % ITALIAN

\ifdefined\RUSSIAN
\chapter{Образцы кода}
\fi % RUSSIAN

\ifdefined\BRAZILIAN
\chapter{Padrões de códigos}
\fi % BRAZILIAN

\ifdefined\THAI
\chapter{รูปแบบของโค้ด}
\fi % THAI

\ifdefined\FRENCH
\chapter{Modèle de code}
\fi % FRENCH

\ifdefined\POLISH
\chapter{\PLph{}}
\fi % POLISH

% sections
\EN{\input{patterns/patterns_opt_dbg_EN}}
\ES{\input{patterns/patterns_opt_dbg_ES}}
\ITA{\input{patterns/patterns_opt_dbg_ITA}}
\PTBR{\input{patterns/patterns_opt_dbg_PTBR}}
\RU{\input{patterns/patterns_opt_dbg_RU}}
\THA{\input{patterns/patterns_opt_dbg_THA}}
\DE{\input{patterns/patterns_opt_dbg_DE}}
\FR{\input{patterns/patterns_opt_dbg_FR}}
\PL{\input{patterns/patterns_opt_dbg_PL}}

\RU{\section{Некоторые базовые понятия}}
\EN{\section{Some basics}}
\DE{\section{Einige Grundlagen}}
\FR{\section{Quelques bases}}
\ES{\section{\ESph{}}}
\ITA{\section{Alcune basi teoriche}}
\PTBR{\section{\PTBRph{}}}
\THA{\section{\THAph{}}}
\PL{\section{\PLph{}}}

% sections:
\EN{\input{patterns/intro_CPU_ISA_EN}}
\ES{\input{patterns/intro_CPU_ISA_ES}}
\ITA{\input{patterns/intro_CPU_ISA_ITA}}
\PTBR{\input{patterns/intro_CPU_ISA_PTBR}}
\RU{\input{patterns/intro_CPU_ISA_RU}}
\DE{\input{patterns/intro_CPU_ISA_DE}}
\FR{\input{patterns/intro_CPU_ISA_FR}}
\PL{\input{patterns/intro_CPU_ISA_PL}}

\EN{\input{patterns/numeral_EN}}
\RU{\input{patterns/numeral_RU}}
\ITA{\input{patterns/numeral_ITA}}
\DE{\input{patterns/numeral_DE}}
\FR{\input{patterns/numeral_FR}}
\PL{\input{patterns/numeral_PL}}

% chapters
\input{patterns/00_empty/main}
\input{patterns/011_ret/main}
\input{patterns/01_helloworld/main}
\input{patterns/015_prolog_epilogue/main}
\input{patterns/02_stack/main}
\input{patterns/03_printf/main}
\input{patterns/04_scanf/main}
\input{patterns/05_passing_arguments/main}
\input{patterns/06_return_results/main}
\input{patterns/061_pointers/main}
\input{patterns/065_GOTO/main}
\input{patterns/07_jcc/main}
\input{patterns/08_switch/main}
\input{patterns/09_loops/main}
\input{patterns/10_strings/main}
\input{patterns/11_arith_optimizations/main}
\input{patterns/12_FPU/main}
\input{patterns/13_arrays/main}
\input{patterns/14_bitfields/main}
\EN{\input{patterns/145_LCG/main_EN}}
\RU{\input{patterns/145_LCG/main_RU}}
\input{patterns/15_structs/main}
\input{patterns/17_unions/main}
\input{patterns/18_pointers_to_functions/main}
\input{patterns/185_64bit_in_32_env/main}

\EN{\input{patterns/19_SIMD/main_EN}}
\RU{\input{patterns/19_SIMD/main_RU}}
\DE{\input{patterns/19_SIMD/main_DE}}

\EN{\input{patterns/20_x64/main_EN}}
\RU{\input{patterns/20_x64/main_RU}}

\EN{\input{patterns/205_floating_SIMD/main_EN}}
\RU{\input{patterns/205_floating_SIMD/main_RU}}
\DE{\input{patterns/205_floating_SIMD/main_DE}}

\EN{\input{patterns/ARM/main_EN}}
\RU{\input{patterns/ARM/main_RU}}
\DE{\input{patterns/ARM/main_DE}}

\input{patterns/MIPS/main}

\ifdefined\SPANISH
\chapter{Patrones de código}
\fi % SPANISH

\ifdefined\GERMAN
\chapter{Code-Muster}
\fi % GERMAN

\ifdefined\ENGLISH
\chapter{Code Patterns}
\fi % ENGLISH

\ifdefined\ITALIAN
\chapter{Forme di codice}
\fi % ITALIAN

\ifdefined\RUSSIAN
\chapter{Образцы кода}
\fi % RUSSIAN

\ifdefined\BRAZILIAN
\chapter{Padrões de códigos}
\fi % BRAZILIAN

\ifdefined\THAI
\chapter{รูปแบบของโค้ด}
\fi % THAI

\ifdefined\FRENCH
\chapter{Modèle de code}
\fi % FRENCH

\ifdefined\POLISH
\chapter{\PLph{}}
\fi % POLISH

% sections
\EN{\input{patterns/patterns_opt_dbg_EN}}
\ES{\input{patterns/patterns_opt_dbg_ES}}
\ITA{\input{patterns/patterns_opt_dbg_ITA}}
\PTBR{\input{patterns/patterns_opt_dbg_PTBR}}
\RU{\input{patterns/patterns_opt_dbg_RU}}
\THA{\input{patterns/patterns_opt_dbg_THA}}
\DE{\input{patterns/patterns_opt_dbg_DE}}
\FR{\input{patterns/patterns_opt_dbg_FR}}
\PL{\input{patterns/patterns_opt_dbg_PL}}

\RU{\section{Некоторые базовые понятия}}
\EN{\section{Some basics}}
\DE{\section{Einige Grundlagen}}
\FR{\section{Quelques bases}}
\ES{\section{\ESph{}}}
\ITA{\section{Alcune basi teoriche}}
\PTBR{\section{\PTBRph{}}}
\THA{\section{\THAph{}}}
\PL{\section{\PLph{}}}

% sections:
\EN{\input{patterns/intro_CPU_ISA_EN}}
\ES{\input{patterns/intro_CPU_ISA_ES}}
\ITA{\input{patterns/intro_CPU_ISA_ITA}}
\PTBR{\input{patterns/intro_CPU_ISA_PTBR}}
\RU{\input{patterns/intro_CPU_ISA_RU}}
\DE{\input{patterns/intro_CPU_ISA_DE}}
\FR{\input{patterns/intro_CPU_ISA_FR}}
\PL{\input{patterns/intro_CPU_ISA_PL}}

\EN{\input{patterns/numeral_EN}}
\RU{\input{patterns/numeral_RU}}
\ITA{\input{patterns/numeral_ITA}}
\DE{\input{patterns/numeral_DE}}
\FR{\input{patterns/numeral_FR}}
\PL{\input{patterns/numeral_PL}}

% chapters
\input{patterns/00_empty/main}
\input{patterns/011_ret/main}
\input{patterns/01_helloworld/main}
\input{patterns/015_prolog_epilogue/main}
\input{patterns/02_stack/main}
\input{patterns/03_printf/main}
\input{patterns/04_scanf/main}
\input{patterns/05_passing_arguments/main}
\input{patterns/06_return_results/main}
\input{patterns/061_pointers/main}
\input{patterns/065_GOTO/main}
\input{patterns/07_jcc/main}
\input{patterns/08_switch/main}
\input{patterns/09_loops/main}
\input{patterns/10_strings/main}
\input{patterns/11_arith_optimizations/main}
\input{patterns/12_FPU/main}
\input{patterns/13_arrays/main}
\input{patterns/14_bitfields/main}
\EN{\input{patterns/145_LCG/main_EN}}
\RU{\input{patterns/145_LCG/main_RU}}
\input{patterns/15_structs/main}
\input{patterns/17_unions/main}
\input{patterns/18_pointers_to_functions/main}
\input{patterns/185_64bit_in_32_env/main}

\EN{\input{patterns/19_SIMD/main_EN}}
\RU{\input{patterns/19_SIMD/main_RU}}
\DE{\input{patterns/19_SIMD/main_DE}}

\EN{\input{patterns/20_x64/main_EN}}
\RU{\input{patterns/20_x64/main_RU}}

\EN{\input{patterns/205_floating_SIMD/main_EN}}
\RU{\input{patterns/205_floating_SIMD/main_RU}}
\DE{\input{patterns/205_floating_SIMD/main_DE}}

\EN{\input{patterns/ARM/main_EN}}
\RU{\input{patterns/ARM/main_RU}}
\DE{\input{patterns/ARM/main_DE}}

\input{patterns/MIPS/main}

\ifdefined\SPANISH
\chapter{Patrones de código}
\fi % SPANISH

\ifdefined\GERMAN
\chapter{Code-Muster}
\fi % GERMAN

\ifdefined\ENGLISH
\chapter{Code Patterns}
\fi % ENGLISH

\ifdefined\ITALIAN
\chapter{Forme di codice}
\fi % ITALIAN

\ifdefined\RUSSIAN
\chapter{Образцы кода}
\fi % RUSSIAN

\ifdefined\BRAZILIAN
\chapter{Padrões de códigos}
\fi % BRAZILIAN

\ifdefined\THAI
\chapter{รูปแบบของโค้ด}
\fi % THAI

\ifdefined\FRENCH
\chapter{Modèle de code}
\fi % FRENCH

\ifdefined\POLISH
\chapter{\PLph{}}
\fi % POLISH

% sections
\EN{\input{patterns/patterns_opt_dbg_EN}}
\ES{\input{patterns/patterns_opt_dbg_ES}}
\ITA{\input{patterns/patterns_opt_dbg_ITA}}
\PTBR{\input{patterns/patterns_opt_dbg_PTBR}}
\RU{\input{patterns/patterns_opt_dbg_RU}}
\THA{\input{patterns/patterns_opt_dbg_THA}}
\DE{\input{patterns/patterns_opt_dbg_DE}}
\FR{\input{patterns/patterns_opt_dbg_FR}}
\PL{\input{patterns/patterns_opt_dbg_PL}}

\RU{\section{Некоторые базовые понятия}}
\EN{\section{Some basics}}
\DE{\section{Einige Grundlagen}}
\FR{\section{Quelques bases}}
\ES{\section{\ESph{}}}
\ITA{\section{Alcune basi teoriche}}
\PTBR{\section{\PTBRph{}}}
\THA{\section{\THAph{}}}
\PL{\section{\PLph{}}}

% sections:
\EN{\input{patterns/intro_CPU_ISA_EN}}
\ES{\input{patterns/intro_CPU_ISA_ES}}
\ITA{\input{patterns/intro_CPU_ISA_ITA}}
\PTBR{\input{patterns/intro_CPU_ISA_PTBR}}
\RU{\input{patterns/intro_CPU_ISA_RU}}
\DE{\input{patterns/intro_CPU_ISA_DE}}
\FR{\input{patterns/intro_CPU_ISA_FR}}
\PL{\input{patterns/intro_CPU_ISA_PL}}

\EN{\input{patterns/numeral_EN}}
\RU{\input{patterns/numeral_RU}}
\ITA{\input{patterns/numeral_ITA}}
\DE{\input{patterns/numeral_DE}}
\FR{\input{patterns/numeral_FR}}
\PL{\input{patterns/numeral_PL}}

% chapters
\input{patterns/00_empty/main}
\input{patterns/011_ret/main}
\input{patterns/01_helloworld/main}
\input{patterns/015_prolog_epilogue/main}
\input{patterns/02_stack/main}
\input{patterns/03_printf/main}
\input{patterns/04_scanf/main}
\input{patterns/05_passing_arguments/main}
\input{patterns/06_return_results/main}
\input{patterns/061_pointers/main}
\input{patterns/065_GOTO/main}
\input{patterns/07_jcc/main}
\input{patterns/08_switch/main}
\input{patterns/09_loops/main}
\input{patterns/10_strings/main}
\input{patterns/11_arith_optimizations/main}
\input{patterns/12_FPU/main}
\input{patterns/13_arrays/main}
\input{patterns/14_bitfields/main}
\EN{\input{patterns/145_LCG/main_EN}}
\RU{\input{patterns/145_LCG/main_RU}}
\input{patterns/15_structs/main}
\input{patterns/17_unions/main}
\input{patterns/18_pointers_to_functions/main}
\input{patterns/185_64bit_in_32_env/main}

\EN{\input{patterns/19_SIMD/main_EN}}
\RU{\input{patterns/19_SIMD/main_RU}}
\DE{\input{patterns/19_SIMD/main_DE}}

\EN{\input{patterns/20_x64/main_EN}}
\RU{\input{patterns/20_x64/main_RU}}

\EN{\input{patterns/205_floating_SIMD/main_EN}}
\RU{\input{patterns/205_floating_SIMD/main_RU}}
\DE{\input{patterns/205_floating_SIMD/main_DE}}

\EN{\input{patterns/ARM/main_EN}}
\RU{\input{patterns/ARM/main_RU}}
\DE{\input{patterns/ARM/main_DE}}

\input{patterns/MIPS/main}

\ifdefined\SPANISH
\chapter{Patrones de código}
\fi % SPANISH

\ifdefined\GERMAN
\chapter{Code-Muster}
\fi % GERMAN

\ifdefined\ENGLISH
\chapter{Code Patterns}
\fi % ENGLISH

\ifdefined\ITALIAN
\chapter{Forme di codice}
\fi % ITALIAN

\ifdefined\RUSSIAN
\chapter{Образцы кода}
\fi % RUSSIAN

\ifdefined\BRAZILIAN
\chapter{Padrões de códigos}
\fi % BRAZILIAN

\ifdefined\THAI
\chapter{รูปแบบของโค้ด}
\fi % THAI

\ifdefined\FRENCH
\chapter{Modèle de code}
\fi % FRENCH

\ifdefined\POLISH
\chapter{\PLph{}}
\fi % POLISH

% sections
\EN{\input{patterns/patterns_opt_dbg_EN}}
\ES{\input{patterns/patterns_opt_dbg_ES}}
\ITA{\input{patterns/patterns_opt_dbg_ITA}}
\PTBR{\input{patterns/patterns_opt_dbg_PTBR}}
\RU{\input{patterns/patterns_opt_dbg_RU}}
\THA{\input{patterns/patterns_opt_dbg_THA}}
\DE{\input{patterns/patterns_opt_dbg_DE}}
\FR{\input{patterns/patterns_opt_dbg_FR}}
\PL{\input{patterns/patterns_opt_dbg_PL}}

\RU{\section{Некоторые базовые понятия}}
\EN{\section{Some basics}}
\DE{\section{Einige Grundlagen}}
\FR{\section{Quelques bases}}
\ES{\section{\ESph{}}}
\ITA{\section{Alcune basi teoriche}}
\PTBR{\section{\PTBRph{}}}
\THA{\section{\THAph{}}}
\PL{\section{\PLph{}}}

% sections:
\EN{\input{patterns/intro_CPU_ISA_EN}}
\ES{\input{patterns/intro_CPU_ISA_ES}}
\ITA{\input{patterns/intro_CPU_ISA_ITA}}
\PTBR{\input{patterns/intro_CPU_ISA_PTBR}}
\RU{\input{patterns/intro_CPU_ISA_RU}}
\DE{\input{patterns/intro_CPU_ISA_DE}}
\FR{\input{patterns/intro_CPU_ISA_FR}}
\PL{\input{patterns/intro_CPU_ISA_PL}}

\EN{\input{patterns/numeral_EN}}
\RU{\input{patterns/numeral_RU}}
\ITA{\input{patterns/numeral_ITA}}
\DE{\input{patterns/numeral_DE}}
\FR{\input{patterns/numeral_FR}}
\PL{\input{patterns/numeral_PL}}

% chapters
\input{patterns/00_empty/main}
\input{patterns/011_ret/main}
\input{patterns/01_helloworld/main}
\input{patterns/015_prolog_epilogue/main}
\input{patterns/02_stack/main}
\input{patterns/03_printf/main}
\input{patterns/04_scanf/main}
\input{patterns/05_passing_arguments/main}
\input{patterns/06_return_results/main}
\input{patterns/061_pointers/main}
\input{patterns/065_GOTO/main}
\input{patterns/07_jcc/main}
\input{patterns/08_switch/main}
\input{patterns/09_loops/main}
\input{patterns/10_strings/main}
\input{patterns/11_arith_optimizations/main}
\input{patterns/12_FPU/main}
\input{patterns/13_arrays/main}
\input{patterns/14_bitfields/main}
\EN{\input{patterns/145_LCG/main_EN}}
\RU{\input{patterns/145_LCG/main_RU}}
\input{patterns/15_structs/main}
\input{patterns/17_unions/main}
\input{patterns/18_pointers_to_functions/main}
\input{patterns/185_64bit_in_32_env/main}

\EN{\input{patterns/19_SIMD/main_EN}}
\RU{\input{patterns/19_SIMD/main_RU}}
\DE{\input{patterns/19_SIMD/main_DE}}

\EN{\input{patterns/20_x64/main_EN}}
\RU{\input{patterns/20_x64/main_RU}}

\EN{\input{patterns/205_floating_SIMD/main_EN}}
\RU{\input{patterns/205_floating_SIMD/main_RU}}
\DE{\input{patterns/205_floating_SIMD/main_DE}}

\EN{\input{patterns/ARM/main_EN}}
\RU{\input{patterns/ARM/main_RU}}
\DE{\input{patterns/ARM/main_DE}}

\input{patterns/MIPS/main}

\ifdefined\SPANISH
\chapter{Patrones de código}
\fi % SPANISH

\ifdefined\GERMAN
\chapter{Code-Muster}
\fi % GERMAN

\ifdefined\ENGLISH
\chapter{Code Patterns}
\fi % ENGLISH

\ifdefined\ITALIAN
\chapter{Forme di codice}
\fi % ITALIAN

\ifdefined\RUSSIAN
\chapter{Образцы кода}
\fi % RUSSIAN

\ifdefined\BRAZILIAN
\chapter{Padrões de códigos}
\fi % BRAZILIAN

\ifdefined\THAI
\chapter{รูปแบบของโค้ด}
\fi % THAI

\ifdefined\FRENCH
\chapter{Modèle de code}
\fi % FRENCH

\ifdefined\POLISH
\chapter{\PLph{}}
\fi % POLISH

% sections
\EN{\input{patterns/patterns_opt_dbg_EN}}
\ES{\input{patterns/patterns_opt_dbg_ES}}
\ITA{\input{patterns/patterns_opt_dbg_ITA}}
\PTBR{\input{patterns/patterns_opt_dbg_PTBR}}
\RU{\input{patterns/patterns_opt_dbg_RU}}
\THA{\input{patterns/patterns_opt_dbg_THA}}
\DE{\input{patterns/patterns_opt_dbg_DE}}
\FR{\input{patterns/patterns_opt_dbg_FR}}
\PL{\input{patterns/patterns_opt_dbg_PL}}

\RU{\section{Некоторые базовые понятия}}
\EN{\section{Some basics}}
\DE{\section{Einige Grundlagen}}
\FR{\section{Quelques bases}}
\ES{\section{\ESph{}}}
\ITA{\section{Alcune basi teoriche}}
\PTBR{\section{\PTBRph{}}}
\THA{\section{\THAph{}}}
\PL{\section{\PLph{}}}

% sections:
\EN{\input{patterns/intro_CPU_ISA_EN}}
\ES{\input{patterns/intro_CPU_ISA_ES}}
\ITA{\input{patterns/intro_CPU_ISA_ITA}}
\PTBR{\input{patterns/intro_CPU_ISA_PTBR}}
\RU{\input{patterns/intro_CPU_ISA_RU}}
\DE{\input{patterns/intro_CPU_ISA_DE}}
\FR{\input{patterns/intro_CPU_ISA_FR}}
\PL{\input{patterns/intro_CPU_ISA_PL}}

\EN{\input{patterns/numeral_EN}}
\RU{\input{patterns/numeral_RU}}
\ITA{\input{patterns/numeral_ITA}}
\DE{\input{patterns/numeral_DE}}
\FR{\input{patterns/numeral_FR}}
\PL{\input{patterns/numeral_PL}}

% chapters
\input{patterns/00_empty/main}
\input{patterns/011_ret/main}
\input{patterns/01_helloworld/main}
\input{patterns/015_prolog_epilogue/main}
\input{patterns/02_stack/main}
\input{patterns/03_printf/main}
\input{patterns/04_scanf/main}
\input{patterns/05_passing_arguments/main}
\input{patterns/06_return_results/main}
\input{patterns/061_pointers/main}
\input{patterns/065_GOTO/main}
\input{patterns/07_jcc/main}
\input{patterns/08_switch/main}
\input{patterns/09_loops/main}
\input{patterns/10_strings/main}
\input{patterns/11_arith_optimizations/main}
\input{patterns/12_FPU/main}
\input{patterns/13_arrays/main}
\input{patterns/14_bitfields/main}
\EN{\input{patterns/145_LCG/main_EN}}
\RU{\input{patterns/145_LCG/main_RU}}
\input{patterns/15_structs/main}
\input{patterns/17_unions/main}
\input{patterns/18_pointers_to_functions/main}
\input{patterns/185_64bit_in_32_env/main}

\EN{\input{patterns/19_SIMD/main_EN}}
\RU{\input{patterns/19_SIMD/main_RU}}
\DE{\input{patterns/19_SIMD/main_DE}}

\EN{\input{patterns/20_x64/main_EN}}
\RU{\input{patterns/20_x64/main_RU}}

\EN{\input{patterns/205_floating_SIMD/main_EN}}
\RU{\input{patterns/205_floating_SIMD/main_RU}}
\DE{\input{patterns/205_floating_SIMD/main_DE}}

\EN{\input{patterns/ARM/main_EN}}
\RU{\input{patterns/ARM/main_RU}}
\DE{\input{patterns/ARM/main_DE}}

\input{patterns/MIPS/main}

\ifdefined\SPANISH
\chapter{Patrones de código}
\fi % SPANISH

\ifdefined\GERMAN
\chapter{Code-Muster}
\fi % GERMAN

\ifdefined\ENGLISH
\chapter{Code Patterns}
\fi % ENGLISH

\ifdefined\ITALIAN
\chapter{Forme di codice}
\fi % ITALIAN

\ifdefined\RUSSIAN
\chapter{Образцы кода}
\fi % RUSSIAN

\ifdefined\BRAZILIAN
\chapter{Padrões de códigos}
\fi % BRAZILIAN

\ifdefined\THAI
\chapter{รูปแบบของโค้ด}
\fi % THAI

\ifdefined\FRENCH
\chapter{Modèle de code}
\fi % FRENCH

\ifdefined\POLISH
\chapter{\PLph{}}
\fi % POLISH

% sections
\EN{\input{patterns/patterns_opt_dbg_EN}}
\ES{\input{patterns/patterns_opt_dbg_ES}}
\ITA{\input{patterns/patterns_opt_dbg_ITA}}
\PTBR{\input{patterns/patterns_opt_dbg_PTBR}}
\RU{\input{patterns/patterns_opt_dbg_RU}}
\THA{\input{patterns/patterns_opt_dbg_THA}}
\DE{\input{patterns/patterns_opt_dbg_DE}}
\FR{\input{patterns/patterns_opt_dbg_FR}}
\PL{\input{patterns/patterns_opt_dbg_PL}}

\RU{\section{Некоторые базовые понятия}}
\EN{\section{Some basics}}
\DE{\section{Einige Grundlagen}}
\FR{\section{Quelques bases}}
\ES{\section{\ESph{}}}
\ITA{\section{Alcune basi teoriche}}
\PTBR{\section{\PTBRph{}}}
\THA{\section{\THAph{}}}
\PL{\section{\PLph{}}}

% sections:
\EN{\input{patterns/intro_CPU_ISA_EN}}
\ES{\input{patterns/intro_CPU_ISA_ES}}
\ITA{\input{patterns/intro_CPU_ISA_ITA}}
\PTBR{\input{patterns/intro_CPU_ISA_PTBR}}
\RU{\input{patterns/intro_CPU_ISA_RU}}
\DE{\input{patterns/intro_CPU_ISA_DE}}
\FR{\input{patterns/intro_CPU_ISA_FR}}
\PL{\input{patterns/intro_CPU_ISA_PL}}

\EN{\input{patterns/numeral_EN}}
\RU{\input{patterns/numeral_RU}}
\ITA{\input{patterns/numeral_ITA}}
\DE{\input{patterns/numeral_DE}}
\FR{\input{patterns/numeral_FR}}
\PL{\input{patterns/numeral_PL}}

% chapters
\input{patterns/00_empty/main}
\input{patterns/011_ret/main}
\input{patterns/01_helloworld/main}
\input{patterns/015_prolog_epilogue/main}
\input{patterns/02_stack/main}
\input{patterns/03_printf/main}
\input{patterns/04_scanf/main}
\input{patterns/05_passing_arguments/main}
\input{patterns/06_return_results/main}
\input{patterns/061_pointers/main}
\input{patterns/065_GOTO/main}
\input{patterns/07_jcc/main}
\input{patterns/08_switch/main}
\input{patterns/09_loops/main}
\input{patterns/10_strings/main}
\input{patterns/11_arith_optimizations/main}
\input{patterns/12_FPU/main}
\input{patterns/13_arrays/main}
\input{patterns/14_bitfields/main}
\EN{\input{patterns/145_LCG/main_EN}}
\RU{\input{patterns/145_LCG/main_RU}}
\input{patterns/15_structs/main}
\input{patterns/17_unions/main}
\input{patterns/18_pointers_to_functions/main}
\input{patterns/185_64bit_in_32_env/main}

\EN{\input{patterns/19_SIMD/main_EN}}
\RU{\input{patterns/19_SIMD/main_RU}}
\DE{\input{patterns/19_SIMD/main_DE}}

\EN{\input{patterns/20_x64/main_EN}}
\RU{\input{patterns/20_x64/main_RU}}

\EN{\input{patterns/205_floating_SIMD/main_EN}}
\RU{\input{patterns/205_floating_SIMD/main_RU}}
\DE{\input{patterns/205_floating_SIMD/main_DE}}

\EN{\input{patterns/ARM/main_EN}}
\RU{\input{patterns/ARM/main_RU}}
\DE{\input{patterns/ARM/main_DE}}

\input{patterns/MIPS/main}

\ifdefined\SPANISH
\chapter{Patrones de código}
\fi % SPANISH

\ifdefined\GERMAN
\chapter{Code-Muster}
\fi % GERMAN

\ifdefined\ENGLISH
\chapter{Code Patterns}
\fi % ENGLISH

\ifdefined\ITALIAN
\chapter{Forme di codice}
\fi % ITALIAN

\ifdefined\RUSSIAN
\chapter{Образцы кода}
\fi % RUSSIAN

\ifdefined\BRAZILIAN
\chapter{Padrões de códigos}
\fi % BRAZILIAN

\ifdefined\THAI
\chapter{รูปแบบของโค้ด}
\fi % THAI

\ifdefined\FRENCH
\chapter{Modèle de code}
\fi % FRENCH

\ifdefined\POLISH
\chapter{\PLph{}}
\fi % POLISH

% sections
\EN{\input{patterns/patterns_opt_dbg_EN}}
\ES{\input{patterns/patterns_opt_dbg_ES}}
\ITA{\input{patterns/patterns_opt_dbg_ITA}}
\PTBR{\input{patterns/patterns_opt_dbg_PTBR}}
\RU{\input{patterns/patterns_opt_dbg_RU}}
\THA{\input{patterns/patterns_opt_dbg_THA}}
\DE{\input{patterns/patterns_opt_dbg_DE}}
\FR{\input{patterns/patterns_opt_dbg_FR}}
\PL{\input{patterns/patterns_opt_dbg_PL}}

\RU{\section{Некоторые базовые понятия}}
\EN{\section{Some basics}}
\DE{\section{Einige Grundlagen}}
\FR{\section{Quelques bases}}
\ES{\section{\ESph{}}}
\ITA{\section{Alcune basi teoriche}}
\PTBR{\section{\PTBRph{}}}
\THA{\section{\THAph{}}}
\PL{\section{\PLph{}}}

% sections:
\EN{\input{patterns/intro_CPU_ISA_EN}}
\ES{\input{patterns/intro_CPU_ISA_ES}}
\ITA{\input{patterns/intro_CPU_ISA_ITA}}
\PTBR{\input{patterns/intro_CPU_ISA_PTBR}}
\RU{\input{patterns/intro_CPU_ISA_RU}}
\DE{\input{patterns/intro_CPU_ISA_DE}}
\FR{\input{patterns/intro_CPU_ISA_FR}}
\PL{\input{patterns/intro_CPU_ISA_PL}}

\EN{\input{patterns/numeral_EN}}
\RU{\input{patterns/numeral_RU}}
\ITA{\input{patterns/numeral_ITA}}
\DE{\input{patterns/numeral_DE}}
\FR{\input{patterns/numeral_FR}}
\PL{\input{patterns/numeral_PL}}

% chapters
\input{patterns/00_empty/main}
\input{patterns/011_ret/main}
\input{patterns/01_helloworld/main}
\input{patterns/015_prolog_epilogue/main}
\input{patterns/02_stack/main}
\input{patterns/03_printf/main}
\input{patterns/04_scanf/main}
\input{patterns/05_passing_arguments/main}
\input{patterns/06_return_results/main}
\input{patterns/061_pointers/main}
\input{patterns/065_GOTO/main}
\input{patterns/07_jcc/main}
\input{patterns/08_switch/main}
\input{patterns/09_loops/main}
\input{patterns/10_strings/main}
\input{patterns/11_arith_optimizations/main}
\input{patterns/12_FPU/main}
\input{patterns/13_arrays/main}
\input{patterns/14_bitfields/main}
\EN{\input{patterns/145_LCG/main_EN}}
\RU{\input{patterns/145_LCG/main_RU}}
\input{patterns/15_structs/main}
\input{patterns/17_unions/main}
\input{patterns/18_pointers_to_functions/main}
\input{patterns/185_64bit_in_32_env/main}

\EN{\input{patterns/19_SIMD/main_EN}}
\RU{\input{patterns/19_SIMD/main_RU}}
\DE{\input{patterns/19_SIMD/main_DE}}

\EN{\input{patterns/20_x64/main_EN}}
\RU{\input{patterns/20_x64/main_RU}}

\EN{\input{patterns/205_floating_SIMD/main_EN}}
\RU{\input{patterns/205_floating_SIMD/main_RU}}
\DE{\input{patterns/205_floating_SIMD/main_DE}}

\EN{\input{patterns/ARM/main_EN}}
\RU{\input{patterns/ARM/main_RU}}
\DE{\input{patterns/ARM/main_DE}}

\input{patterns/MIPS/main}

\ifdefined\SPANISH
\chapter{Patrones de código}
\fi % SPANISH

\ifdefined\GERMAN
\chapter{Code-Muster}
\fi % GERMAN

\ifdefined\ENGLISH
\chapter{Code Patterns}
\fi % ENGLISH

\ifdefined\ITALIAN
\chapter{Forme di codice}
\fi % ITALIAN

\ifdefined\RUSSIAN
\chapter{Образцы кода}
\fi % RUSSIAN

\ifdefined\BRAZILIAN
\chapter{Padrões de códigos}
\fi % BRAZILIAN

\ifdefined\THAI
\chapter{รูปแบบของโค้ด}
\fi % THAI

\ifdefined\FRENCH
\chapter{Modèle de code}
\fi % FRENCH

\ifdefined\POLISH
\chapter{\PLph{}}
\fi % POLISH

% sections
\EN{\input{patterns/patterns_opt_dbg_EN}}
\ES{\input{patterns/patterns_opt_dbg_ES}}
\ITA{\input{patterns/patterns_opt_dbg_ITA}}
\PTBR{\input{patterns/patterns_opt_dbg_PTBR}}
\RU{\input{patterns/patterns_opt_dbg_RU}}
\THA{\input{patterns/patterns_opt_dbg_THA}}
\DE{\input{patterns/patterns_opt_dbg_DE}}
\FR{\input{patterns/patterns_opt_dbg_FR}}
\PL{\input{patterns/patterns_opt_dbg_PL}}

\RU{\section{Некоторые базовые понятия}}
\EN{\section{Some basics}}
\DE{\section{Einige Grundlagen}}
\FR{\section{Quelques bases}}
\ES{\section{\ESph{}}}
\ITA{\section{Alcune basi teoriche}}
\PTBR{\section{\PTBRph{}}}
\THA{\section{\THAph{}}}
\PL{\section{\PLph{}}}

% sections:
\EN{\input{patterns/intro_CPU_ISA_EN}}
\ES{\input{patterns/intro_CPU_ISA_ES}}
\ITA{\input{patterns/intro_CPU_ISA_ITA}}
\PTBR{\input{patterns/intro_CPU_ISA_PTBR}}
\RU{\input{patterns/intro_CPU_ISA_RU}}
\DE{\input{patterns/intro_CPU_ISA_DE}}
\FR{\input{patterns/intro_CPU_ISA_FR}}
\PL{\input{patterns/intro_CPU_ISA_PL}}

\EN{\input{patterns/numeral_EN}}
\RU{\input{patterns/numeral_RU}}
\ITA{\input{patterns/numeral_ITA}}
\DE{\input{patterns/numeral_DE}}
\FR{\input{patterns/numeral_FR}}
\PL{\input{patterns/numeral_PL}}

% chapters
\input{patterns/00_empty/main}
\input{patterns/011_ret/main}
\input{patterns/01_helloworld/main}
\input{patterns/015_prolog_epilogue/main}
\input{patterns/02_stack/main}
\input{patterns/03_printf/main}
\input{patterns/04_scanf/main}
\input{patterns/05_passing_arguments/main}
\input{patterns/06_return_results/main}
\input{patterns/061_pointers/main}
\input{patterns/065_GOTO/main}
\input{patterns/07_jcc/main}
\input{patterns/08_switch/main}
\input{patterns/09_loops/main}
\input{patterns/10_strings/main}
\input{patterns/11_arith_optimizations/main}
\input{patterns/12_FPU/main}
\input{patterns/13_arrays/main}
\input{patterns/14_bitfields/main}
\EN{\input{patterns/145_LCG/main_EN}}
\RU{\input{patterns/145_LCG/main_RU}}
\input{patterns/15_structs/main}
\input{patterns/17_unions/main}
\input{patterns/18_pointers_to_functions/main}
\input{patterns/185_64bit_in_32_env/main}

\EN{\input{patterns/19_SIMD/main_EN}}
\RU{\input{patterns/19_SIMD/main_RU}}
\DE{\input{patterns/19_SIMD/main_DE}}

\EN{\input{patterns/20_x64/main_EN}}
\RU{\input{patterns/20_x64/main_RU}}

\EN{\input{patterns/205_floating_SIMD/main_EN}}
\RU{\input{patterns/205_floating_SIMD/main_RU}}
\DE{\input{patterns/205_floating_SIMD/main_DE}}

\EN{\input{patterns/ARM/main_EN}}
\RU{\input{patterns/ARM/main_RU}}
\DE{\input{patterns/ARM/main_DE}}

\input{patterns/MIPS/main}

\ifdefined\SPANISH
\chapter{Patrones de código}
\fi % SPANISH

\ifdefined\GERMAN
\chapter{Code-Muster}
\fi % GERMAN

\ifdefined\ENGLISH
\chapter{Code Patterns}
\fi % ENGLISH

\ifdefined\ITALIAN
\chapter{Forme di codice}
\fi % ITALIAN

\ifdefined\RUSSIAN
\chapter{Образцы кода}
\fi % RUSSIAN

\ifdefined\BRAZILIAN
\chapter{Padrões de códigos}
\fi % BRAZILIAN

\ifdefined\THAI
\chapter{รูปแบบของโค้ด}
\fi % THAI

\ifdefined\FRENCH
\chapter{Modèle de code}
\fi % FRENCH

\ifdefined\POLISH
\chapter{\PLph{}}
\fi % POLISH

% sections
\EN{\input{patterns/patterns_opt_dbg_EN}}
\ES{\input{patterns/patterns_opt_dbg_ES}}
\ITA{\input{patterns/patterns_opt_dbg_ITA}}
\PTBR{\input{patterns/patterns_opt_dbg_PTBR}}
\RU{\input{patterns/patterns_opt_dbg_RU}}
\THA{\input{patterns/patterns_opt_dbg_THA}}
\DE{\input{patterns/patterns_opt_dbg_DE}}
\FR{\input{patterns/patterns_opt_dbg_FR}}
\PL{\input{patterns/patterns_opt_dbg_PL}}

\RU{\section{Некоторые базовые понятия}}
\EN{\section{Some basics}}
\DE{\section{Einige Grundlagen}}
\FR{\section{Quelques bases}}
\ES{\section{\ESph{}}}
\ITA{\section{Alcune basi teoriche}}
\PTBR{\section{\PTBRph{}}}
\THA{\section{\THAph{}}}
\PL{\section{\PLph{}}}

% sections:
\EN{\input{patterns/intro_CPU_ISA_EN}}
\ES{\input{patterns/intro_CPU_ISA_ES}}
\ITA{\input{patterns/intro_CPU_ISA_ITA}}
\PTBR{\input{patterns/intro_CPU_ISA_PTBR}}
\RU{\input{patterns/intro_CPU_ISA_RU}}
\DE{\input{patterns/intro_CPU_ISA_DE}}
\FR{\input{patterns/intro_CPU_ISA_FR}}
\PL{\input{patterns/intro_CPU_ISA_PL}}

\EN{\input{patterns/numeral_EN}}
\RU{\input{patterns/numeral_RU}}
\ITA{\input{patterns/numeral_ITA}}
\DE{\input{patterns/numeral_DE}}
\FR{\input{patterns/numeral_FR}}
\PL{\input{patterns/numeral_PL}}

% chapters
\input{patterns/00_empty/main}
\input{patterns/011_ret/main}
\input{patterns/01_helloworld/main}
\input{patterns/015_prolog_epilogue/main}
\input{patterns/02_stack/main}
\input{patterns/03_printf/main}
\input{patterns/04_scanf/main}
\input{patterns/05_passing_arguments/main}
\input{patterns/06_return_results/main}
\input{patterns/061_pointers/main}
\input{patterns/065_GOTO/main}
\input{patterns/07_jcc/main}
\input{patterns/08_switch/main}
\input{patterns/09_loops/main}
\input{patterns/10_strings/main}
\input{patterns/11_arith_optimizations/main}
\input{patterns/12_FPU/main}
\input{patterns/13_arrays/main}
\input{patterns/14_bitfields/main}
\EN{\input{patterns/145_LCG/main_EN}}
\RU{\input{patterns/145_LCG/main_RU}}
\input{patterns/15_structs/main}
\input{patterns/17_unions/main}
\input{patterns/18_pointers_to_functions/main}
\input{patterns/185_64bit_in_32_env/main}

\EN{\input{patterns/19_SIMD/main_EN}}
\RU{\input{patterns/19_SIMD/main_RU}}
\DE{\input{patterns/19_SIMD/main_DE}}

\EN{\input{patterns/20_x64/main_EN}}
\RU{\input{patterns/20_x64/main_RU}}

\EN{\input{patterns/205_floating_SIMD/main_EN}}
\RU{\input{patterns/205_floating_SIMD/main_RU}}
\DE{\input{patterns/205_floating_SIMD/main_DE}}

\EN{\input{patterns/ARM/main_EN}}
\RU{\input{patterns/ARM/main_RU}}
\DE{\input{patterns/ARM/main_DE}}

\input{patterns/MIPS/main}

\ifdefined\SPANISH
\chapter{Patrones de código}
\fi % SPANISH

\ifdefined\GERMAN
\chapter{Code-Muster}
\fi % GERMAN

\ifdefined\ENGLISH
\chapter{Code Patterns}
\fi % ENGLISH

\ifdefined\ITALIAN
\chapter{Forme di codice}
\fi % ITALIAN

\ifdefined\RUSSIAN
\chapter{Образцы кода}
\fi % RUSSIAN

\ifdefined\BRAZILIAN
\chapter{Padrões de códigos}
\fi % BRAZILIAN

\ifdefined\THAI
\chapter{รูปแบบของโค้ด}
\fi % THAI

\ifdefined\FRENCH
\chapter{Modèle de code}
\fi % FRENCH

\ifdefined\POLISH
\chapter{\PLph{}}
\fi % POLISH

% sections
\EN{\input{patterns/patterns_opt_dbg_EN}}
\ES{\input{patterns/patterns_opt_dbg_ES}}
\ITA{\input{patterns/patterns_opt_dbg_ITA}}
\PTBR{\input{patterns/patterns_opt_dbg_PTBR}}
\RU{\input{patterns/patterns_opt_dbg_RU}}
\THA{\input{patterns/patterns_opt_dbg_THA}}
\DE{\input{patterns/patterns_opt_dbg_DE}}
\FR{\input{patterns/patterns_opt_dbg_FR}}
\PL{\input{patterns/patterns_opt_dbg_PL}}

\RU{\section{Некоторые базовые понятия}}
\EN{\section{Some basics}}
\DE{\section{Einige Grundlagen}}
\FR{\section{Quelques bases}}
\ES{\section{\ESph{}}}
\ITA{\section{Alcune basi teoriche}}
\PTBR{\section{\PTBRph{}}}
\THA{\section{\THAph{}}}
\PL{\section{\PLph{}}}

% sections:
\EN{\input{patterns/intro_CPU_ISA_EN}}
\ES{\input{patterns/intro_CPU_ISA_ES}}
\ITA{\input{patterns/intro_CPU_ISA_ITA}}
\PTBR{\input{patterns/intro_CPU_ISA_PTBR}}
\RU{\input{patterns/intro_CPU_ISA_RU}}
\DE{\input{patterns/intro_CPU_ISA_DE}}
\FR{\input{patterns/intro_CPU_ISA_FR}}
\PL{\input{patterns/intro_CPU_ISA_PL}}

\EN{\input{patterns/numeral_EN}}
\RU{\input{patterns/numeral_RU}}
\ITA{\input{patterns/numeral_ITA}}
\DE{\input{patterns/numeral_DE}}
\FR{\input{patterns/numeral_FR}}
\PL{\input{patterns/numeral_PL}}

% chapters
\input{patterns/00_empty/main}
\input{patterns/011_ret/main}
\input{patterns/01_helloworld/main}
\input{patterns/015_prolog_epilogue/main}
\input{patterns/02_stack/main}
\input{patterns/03_printf/main}
\input{patterns/04_scanf/main}
\input{patterns/05_passing_arguments/main}
\input{patterns/06_return_results/main}
\input{patterns/061_pointers/main}
\input{patterns/065_GOTO/main}
\input{patterns/07_jcc/main}
\input{patterns/08_switch/main}
\input{patterns/09_loops/main}
\input{patterns/10_strings/main}
\input{patterns/11_arith_optimizations/main}
\input{patterns/12_FPU/main}
\input{patterns/13_arrays/main}
\input{patterns/14_bitfields/main}
\EN{\input{patterns/145_LCG/main_EN}}
\RU{\input{patterns/145_LCG/main_RU}}
\input{patterns/15_structs/main}
\input{patterns/17_unions/main}
\input{patterns/18_pointers_to_functions/main}
\input{patterns/185_64bit_in_32_env/main}

\EN{\input{patterns/19_SIMD/main_EN}}
\RU{\input{patterns/19_SIMD/main_RU}}
\DE{\input{patterns/19_SIMD/main_DE}}

\EN{\input{patterns/20_x64/main_EN}}
\RU{\input{patterns/20_x64/main_RU}}

\EN{\input{patterns/205_floating_SIMD/main_EN}}
\RU{\input{patterns/205_floating_SIMD/main_RU}}
\DE{\input{patterns/205_floating_SIMD/main_DE}}

\EN{\input{patterns/ARM/main_EN}}
\RU{\input{patterns/ARM/main_RU}}
\DE{\input{patterns/ARM/main_DE}}

\input{patterns/MIPS/main}

\ifdefined\SPANISH
\chapter{Patrones de código}
\fi % SPANISH

\ifdefined\GERMAN
\chapter{Code-Muster}
\fi % GERMAN

\ifdefined\ENGLISH
\chapter{Code Patterns}
\fi % ENGLISH

\ifdefined\ITALIAN
\chapter{Forme di codice}
\fi % ITALIAN

\ifdefined\RUSSIAN
\chapter{Образцы кода}
\fi % RUSSIAN

\ifdefined\BRAZILIAN
\chapter{Padrões de códigos}
\fi % BRAZILIAN

\ifdefined\THAI
\chapter{รูปแบบของโค้ด}
\fi % THAI

\ifdefined\FRENCH
\chapter{Modèle de code}
\fi % FRENCH

\ifdefined\POLISH
\chapter{\PLph{}}
\fi % POLISH

% sections
\EN{\input{patterns/patterns_opt_dbg_EN}}
\ES{\input{patterns/patterns_opt_dbg_ES}}
\ITA{\input{patterns/patterns_opt_dbg_ITA}}
\PTBR{\input{patterns/patterns_opt_dbg_PTBR}}
\RU{\input{patterns/patterns_opt_dbg_RU}}
\THA{\input{patterns/patterns_opt_dbg_THA}}
\DE{\input{patterns/patterns_opt_dbg_DE}}
\FR{\input{patterns/patterns_opt_dbg_FR}}
\PL{\input{patterns/patterns_opt_dbg_PL}}

\RU{\section{Некоторые базовые понятия}}
\EN{\section{Some basics}}
\DE{\section{Einige Grundlagen}}
\FR{\section{Quelques bases}}
\ES{\section{\ESph{}}}
\ITA{\section{Alcune basi teoriche}}
\PTBR{\section{\PTBRph{}}}
\THA{\section{\THAph{}}}
\PL{\section{\PLph{}}}

% sections:
\EN{\input{patterns/intro_CPU_ISA_EN}}
\ES{\input{patterns/intro_CPU_ISA_ES}}
\ITA{\input{patterns/intro_CPU_ISA_ITA}}
\PTBR{\input{patterns/intro_CPU_ISA_PTBR}}
\RU{\input{patterns/intro_CPU_ISA_RU}}
\DE{\input{patterns/intro_CPU_ISA_DE}}
\FR{\input{patterns/intro_CPU_ISA_FR}}
\PL{\input{patterns/intro_CPU_ISA_PL}}

\EN{\input{patterns/numeral_EN}}
\RU{\input{patterns/numeral_RU}}
\ITA{\input{patterns/numeral_ITA}}
\DE{\input{patterns/numeral_DE}}
\FR{\input{patterns/numeral_FR}}
\PL{\input{patterns/numeral_PL}}

% chapters
\input{patterns/00_empty/main}
\input{patterns/011_ret/main}
\input{patterns/01_helloworld/main}
\input{patterns/015_prolog_epilogue/main}
\input{patterns/02_stack/main}
\input{patterns/03_printf/main}
\input{patterns/04_scanf/main}
\input{patterns/05_passing_arguments/main}
\input{patterns/06_return_results/main}
\input{patterns/061_pointers/main}
\input{patterns/065_GOTO/main}
\input{patterns/07_jcc/main}
\input{patterns/08_switch/main}
\input{patterns/09_loops/main}
\input{patterns/10_strings/main}
\input{patterns/11_arith_optimizations/main}
\input{patterns/12_FPU/main}
\input{patterns/13_arrays/main}
\input{patterns/14_bitfields/main}
\EN{\input{patterns/145_LCG/main_EN}}
\RU{\input{patterns/145_LCG/main_RU}}
\input{patterns/15_structs/main}
\input{patterns/17_unions/main}
\input{patterns/18_pointers_to_functions/main}
\input{patterns/185_64bit_in_32_env/main}

\EN{\input{patterns/19_SIMD/main_EN}}
\RU{\input{patterns/19_SIMD/main_RU}}
\DE{\input{patterns/19_SIMD/main_DE}}

\EN{\input{patterns/20_x64/main_EN}}
\RU{\input{patterns/20_x64/main_RU}}

\EN{\input{patterns/205_floating_SIMD/main_EN}}
\RU{\input{patterns/205_floating_SIMD/main_RU}}
\DE{\input{patterns/205_floating_SIMD/main_DE}}

\EN{\input{patterns/ARM/main_EN}}
\RU{\input{patterns/ARM/main_RU}}
\DE{\input{patterns/ARM/main_DE}}

\input{patterns/MIPS/main}

\ifdefined\SPANISH
\chapter{Patrones de código}
\fi % SPANISH

\ifdefined\GERMAN
\chapter{Code-Muster}
\fi % GERMAN

\ifdefined\ENGLISH
\chapter{Code Patterns}
\fi % ENGLISH

\ifdefined\ITALIAN
\chapter{Forme di codice}
\fi % ITALIAN

\ifdefined\RUSSIAN
\chapter{Образцы кода}
\fi % RUSSIAN

\ifdefined\BRAZILIAN
\chapter{Padrões de códigos}
\fi % BRAZILIAN

\ifdefined\THAI
\chapter{รูปแบบของโค้ด}
\fi % THAI

\ifdefined\FRENCH
\chapter{Modèle de code}
\fi % FRENCH

\ifdefined\POLISH
\chapter{\PLph{}}
\fi % POLISH

% sections
\EN{\input{patterns/patterns_opt_dbg_EN}}
\ES{\input{patterns/patterns_opt_dbg_ES}}
\ITA{\input{patterns/patterns_opt_dbg_ITA}}
\PTBR{\input{patterns/patterns_opt_dbg_PTBR}}
\RU{\input{patterns/patterns_opt_dbg_RU}}
\THA{\input{patterns/patterns_opt_dbg_THA}}
\DE{\input{patterns/patterns_opt_dbg_DE}}
\FR{\input{patterns/patterns_opt_dbg_FR}}
\PL{\input{patterns/patterns_opt_dbg_PL}}

\RU{\section{Некоторые базовые понятия}}
\EN{\section{Some basics}}
\DE{\section{Einige Grundlagen}}
\FR{\section{Quelques bases}}
\ES{\section{\ESph{}}}
\ITA{\section{Alcune basi teoriche}}
\PTBR{\section{\PTBRph{}}}
\THA{\section{\THAph{}}}
\PL{\section{\PLph{}}}

% sections:
\EN{\input{patterns/intro_CPU_ISA_EN}}
\ES{\input{patterns/intro_CPU_ISA_ES}}
\ITA{\input{patterns/intro_CPU_ISA_ITA}}
\PTBR{\input{patterns/intro_CPU_ISA_PTBR}}
\RU{\input{patterns/intro_CPU_ISA_RU}}
\DE{\input{patterns/intro_CPU_ISA_DE}}
\FR{\input{patterns/intro_CPU_ISA_FR}}
\PL{\input{patterns/intro_CPU_ISA_PL}}

\EN{\input{patterns/numeral_EN}}
\RU{\input{patterns/numeral_RU}}
\ITA{\input{patterns/numeral_ITA}}
\DE{\input{patterns/numeral_DE}}
\FR{\input{patterns/numeral_FR}}
\PL{\input{patterns/numeral_PL}}

% chapters
\input{patterns/00_empty/main}
\input{patterns/011_ret/main}
\input{patterns/01_helloworld/main}
\input{patterns/015_prolog_epilogue/main}
\input{patterns/02_stack/main}
\input{patterns/03_printf/main}
\input{patterns/04_scanf/main}
\input{patterns/05_passing_arguments/main}
\input{patterns/06_return_results/main}
\input{patterns/061_pointers/main}
\input{patterns/065_GOTO/main}
\input{patterns/07_jcc/main}
\input{patterns/08_switch/main}
\input{patterns/09_loops/main}
\input{patterns/10_strings/main}
\input{patterns/11_arith_optimizations/main}
\input{patterns/12_FPU/main}
\input{patterns/13_arrays/main}
\input{patterns/14_bitfields/main}
\EN{\input{patterns/145_LCG/main_EN}}
\RU{\input{patterns/145_LCG/main_RU}}
\input{patterns/15_structs/main}
\input{patterns/17_unions/main}
\input{patterns/18_pointers_to_functions/main}
\input{patterns/185_64bit_in_32_env/main}

\EN{\input{patterns/19_SIMD/main_EN}}
\RU{\input{patterns/19_SIMD/main_RU}}
\DE{\input{patterns/19_SIMD/main_DE}}

\EN{\input{patterns/20_x64/main_EN}}
\RU{\input{patterns/20_x64/main_RU}}

\EN{\input{patterns/205_floating_SIMD/main_EN}}
\RU{\input{patterns/205_floating_SIMD/main_RU}}
\DE{\input{patterns/205_floating_SIMD/main_DE}}

\EN{\input{patterns/ARM/main_EN}}
\RU{\input{patterns/ARM/main_RU}}
\DE{\input{patterns/ARM/main_DE}}

\input{patterns/MIPS/main}

\ifdefined\SPANISH
\chapter{Patrones de código}
\fi % SPANISH

\ifdefined\GERMAN
\chapter{Code-Muster}
\fi % GERMAN

\ifdefined\ENGLISH
\chapter{Code Patterns}
\fi % ENGLISH

\ifdefined\ITALIAN
\chapter{Forme di codice}
\fi % ITALIAN

\ifdefined\RUSSIAN
\chapter{Образцы кода}
\fi % RUSSIAN

\ifdefined\BRAZILIAN
\chapter{Padrões de códigos}
\fi % BRAZILIAN

\ifdefined\THAI
\chapter{รูปแบบของโค้ด}
\fi % THAI

\ifdefined\FRENCH
\chapter{Modèle de code}
\fi % FRENCH

\ifdefined\POLISH
\chapter{\PLph{}}
\fi % POLISH

% sections
\EN{\input{patterns/patterns_opt_dbg_EN}}
\ES{\input{patterns/patterns_opt_dbg_ES}}
\ITA{\input{patterns/patterns_opt_dbg_ITA}}
\PTBR{\input{patterns/patterns_opt_dbg_PTBR}}
\RU{\input{patterns/patterns_opt_dbg_RU}}
\THA{\input{patterns/patterns_opt_dbg_THA}}
\DE{\input{patterns/patterns_opt_dbg_DE}}
\FR{\input{patterns/patterns_opt_dbg_FR}}
\PL{\input{patterns/patterns_opt_dbg_PL}}

\RU{\section{Некоторые базовые понятия}}
\EN{\section{Some basics}}
\DE{\section{Einige Grundlagen}}
\FR{\section{Quelques bases}}
\ES{\section{\ESph{}}}
\ITA{\section{Alcune basi teoriche}}
\PTBR{\section{\PTBRph{}}}
\THA{\section{\THAph{}}}
\PL{\section{\PLph{}}}

% sections:
\EN{\input{patterns/intro_CPU_ISA_EN}}
\ES{\input{patterns/intro_CPU_ISA_ES}}
\ITA{\input{patterns/intro_CPU_ISA_ITA}}
\PTBR{\input{patterns/intro_CPU_ISA_PTBR}}
\RU{\input{patterns/intro_CPU_ISA_RU}}
\DE{\input{patterns/intro_CPU_ISA_DE}}
\FR{\input{patterns/intro_CPU_ISA_FR}}
\PL{\input{patterns/intro_CPU_ISA_PL}}

\EN{\input{patterns/numeral_EN}}
\RU{\input{patterns/numeral_RU}}
\ITA{\input{patterns/numeral_ITA}}
\DE{\input{patterns/numeral_DE}}
\FR{\input{patterns/numeral_FR}}
\PL{\input{patterns/numeral_PL}}

% chapters
\input{patterns/00_empty/main}
\input{patterns/011_ret/main}
\input{patterns/01_helloworld/main}
\input{patterns/015_prolog_epilogue/main}
\input{patterns/02_stack/main}
\input{patterns/03_printf/main}
\input{patterns/04_scanf/main}
\input{patterns/05_passing_arguments/main}
\input{patterns/06_return_results/main}
\input{patterns/061_pointers/main}
\input{patterns/065_GOTO/main}
\input{patterns/07_jcc/main}
\input{patterns/08_switch/main}
\input{patterns/09_loops/main}
\input{patterns/10_strings/main}
\input{patterns/11_arith_optimizations/main}
\input{patterns/12_FPU/main}
\input{patterns/13_arrays/main}
\input{patterns/14_bitfields/main}
\EN{\input{patterns/145_LCG/main_EN}}
\RU{\input{patterns/145_LCG/main_RU}}
\input{patterns/15_structs/main}
\input{patterns/17_unions/main}
\input{patterns/18_pointers_to_functions/main}
\input{patterns/185_64bit_in_32_env/main}

\EN{\input{patterns/19_SIMD/main_EN}}
\RU{\input{patterns/19_SIMD/main_RU}}
\DE{\input{patterns/19_SIMD/main_DE}}

\EN{\input{patterns/20_x64/main_EN}}
\RU{\input{patterns/20_x64/main_RU}}

\EN{\input{patterns/205_floating_SIMD/main_EN}}
\RU{\input{patterns/205_floating_SIMD/main_RU}}
\DE{\input{patterns/205_floating_SIMD/main_DE}}

\EN{\input{patterns/ARM/main_EN}}
\RU{\input{patterns/ARM/main_RU}}
\DE{\input{patterns/ARM/main_DE}}

\input{patterns/MIPS/main}

\ifdefined\SPANISH
\chapter{Patrones de código}
\fi % SPANISH

\ifdefined\GERMAN
\chapter{Code-Muster}
\fi % GERMAN

\ifdefined\ENGLISH
\chapter{Code Patterns}
\fi % ENGLISH

\ifdefined\ITALIAN
\chapter{Forme di codice}
\fi % ITALIAN

\ifdefined\RUSSIAN
\chapter{Образцы кода}
\fi % RUSSIAN

\ifdefined\BRAZILIAN
\chapter{Padrões de códigos}
\fi % BRAZILIAN

\ifdefined\THAI
\chapter{รูปแบบของโค้ด}
\fi % THAI

\ifdefined\FRENCH
\chapter{Modèle de code}
\fi % FRENCH

\ifdefined\POLISH
\chapter{\PLph{}}
\fi % POLISH

% sections
\EN{\input{patterns/patterns_opt_dbg_EN}}
\ES{\input{patterns/patterns_opt_dbg_ES}}
\ITA{\input{patterns/patterns_opt_dbg_ITA}}
\PTBR{\input{patterns/patterns_opt_dbg_PTBR}}
\RU{\input{patterns/patterns_opt_dbg_RU}}
\THA{\input{patterns/patterns_opt_dbg_THA}}
\DE{\input{patterns/patterns_opt_dbg_DE}}
\FR{\input{patterns/patterns_opt_dbg_FR}}
\PL{\input{patterns/patterns_opt_dbg_PL}}

\RU{\section{Некоторые базовые понятия}}
\EN{\section{Some basics}}
\DE{\section{Einige Grundlagen}}
\FR{\section{Quelques bases}}
\ES{\section{\ESph{}}}
\ITA{\section{Alcune basi teoriche}}
\PTBR{\section{\PTBRph{}}}
\THA{\section{\THAph{}}}
\PL{\section{\PLph{}}}

% sections:
\EN{\input{patterns/intro_CPU_ISA_EN}}
\ES{\input{patterns/intro_CPU_ISA_ES}}
\ITA{\input{patterns/intro_CPU_ISA_ITA}}
\PTBR{\input{patterns/intro_CPU_ISA_PTBR}}
\RU{\input{patterns/intro_CPU_ISA_RU}}
\DE{\input{patterns/intro_CPU_ISA_DE}}
\FR{\input{patterns/intro_CPU_ISA_FR}}
\PL{\input{patterns/intro_CPU_ISA_PL}}

\EN{\input{patterns/numeral_EN}}
\RU{\input{patterns/numeral_RU}}
\ITA{\input{patterns/numeral_ITA}}
\DE{\input{patterns/numeral_DE}}
\FR{\input{patterns/numeral_FR}}
\PL{\input{patterns/numeral_PL}}

% chapters
\input{patterns/00_empty/main}
\input{patterns/011_ret/main}
\input{patterns/01_helloworld/main}
\input{patterns/015_prolog_epilogue/main}
\input{patterns/02_stack/main}
\input{patterns/03_printf/main}
\input{patterns/04_scanf/main}
\input{patterns/05_passing_arguments/main}
\input{patterns/06_return_results/main}
\input{patterns/061_pointers/main}
\input{patterns/065_GOTO/main}
\input{patterns/07_jcc/main}
\input{patterns/08_switch/main}
\input{patterns/09_loops/main}
\input{patterns/10_strings/main}
\input{patterns/11_arith_optimizations/main}
\input{patterns/12_FPU/main}
\input{patterns/13_arrays/main}
\input{patterns/14_bitfields/main}
\EN{\input{patterns/145_LCG/main_EN}}
\RU{\input{patterns/145_LCG/main_RU}}
\input{patterns/15_structs/main}
\input{patterns/17_unions/main}
\input{patterns/18_pointers_to_functions/main}
\input{patterns/185_64bit_in_32_env/main}

\EN{\input{patterns/19_SIMD/main_EN}}
\RU{\input{patterns/19_SIMD/main_RU}}
\DE{\input{patterns/19_SIMD/main_DE}}

\EN{\input{patterns/20_x64/main_EN}}
\RU{\input{patterns/20_x64/main_RU}}

\EN{\input{patterns/205_floating_SIMD/main_EN}}
\RU{\input{patterns/205_floating_SIMD/main_RU}}
\DE{\input{patterns/205_floating_SIMD/main_DE}}

\EN{\input{patterns/ARM/main_EN}}
\RU{\input{patterns/ARM/main_RU}}
\DE{\input{patterns/ARM/main_DE}}

\input{patterns/MIPS/main}

\ifdefined\SPANISH
\chapter{Patrones de código}
\fi % SPANISH

\ifdefined\GERMAN
\chapter{Code-Muster}
\fi % GERMAN

\ifdefined\ENGLISH
\chapter{Code Patterns}
\fi % ENGLISH

\ifdefined\ITALIAN
\chapter{Forme di codice}
\fi % ITALIAN

\ifdefined\RUSSIAN
\chapter{Образцы кода}
\fi % RUSSIAN

\ifdefined\BRAZILIAN
\chapter{Padrões de códigos}
\fi % BRAZILIAN

\ifdefined\THAI
\chapter{รูปแบบของโค้ด}
\fi % THAI

\ifdefined\FRENCH
\chapter{Modèle de code}
\fi % FRENCH

\ifdefined\POLISH
\chapter{\PLph{}}
\fi % POLISH

% sections
\EN{\input{patterns/patterns_opt_dbg_EN}}
\ES{\input{patterns/patterns_opt_dbg_ES}}
\ITA{\input{patterns/patterns_opt_dbg_ITA}}
\PTBR{\input{patterns/patterns_opt_dbg_PTBR}}
\RU{\input{patterns/patterns_opt_dbg_RU}}
\THA{\input{patterns/patterns_opt_dbg_THA}}
\DE{\input{patterns/patterns_opt_dbg_DE}}
\FR{\input{patterns/patterns_opt_dbg_FR}}
\PL{\input{patterns/patterns_opt_dbg_PL}}

\RU{\section{Некоторые базовые понятия}}
\EN{\section{Some basics}}
\DE{\section{Einige Grundlagen}}
\FR{\section{Quelques bases}}
\ES{\section{\ESph{}}}
\ITA{\section{Alcune basi teoriche}}
\PTBR{\section{\PTBRph{}}}
\THA{\section{\THAph{}}}
\PL{\section{\PLph{}}}

% sections:
\EN{\input{patterns/intro_CPU_ISA_EN}}
\ES{\input{patterns/intro_CPU_ISA_ES}}
\ITA{\input{patterns/intro_CPU_ISA_ITA}}
\PTBR{\input{patterns/intro_CPU_ISA_PTBR}}
\RU{\input{patterns/intro_CPU_ISA_RU}}
\DE{\input{patterns/intro_CPU_ISA_DE}}
\FR{\input{patterns/intro_CPU_ISA_FR}}
\PL{\input{patterns/intro_CPU_ISA_PL}}

\EN{\input{patterns/numeral_EN}}
\RU{\input{patterns/numeral_RU}}
\ITA{\input{patterns/numeral_ITA}}
\DE{\input{patterns/numeral_DE}}
\FR{\input{patterns/numeral_FR}}
\PL{\input{patterns/numeral_PL}}

% chapters
\input{patterns/00_empty/main}
\input{patterns/011_ret/main}
\input{patterns/01_helloworld/main}
\input{patterns/015_prolog_epilogue/main}
\input{patterns/02_stack/main}
\input{patterns/03_printf/main}
\input{patterns/04_scanf/main}
\input{patterns/05_passing_arguments/main}
\input{patterns/06_return_results/main}
\input{patterns/061_pointers/main}
\input{patterns/065_GOTO/main}
\input{patterns/07_jcc/main}
\input{patterns/08_switch/main}
\input{patterns/09_loops/main}
\input{patterns/10_strings/main}
\input{patterns/11_arith_optimizations/main}
\input{patterns/12_FPU/main}
\input{patterns/13_arrays/main}
\input{patterns/14_bitfields/main}
\EN{\input{patterns/145_LCG/main_EN}}
\RU{\input{patterns/145_LCG/main_RU}}
\input{patterns/15_structs/main}
\input{patterns/17_unions/main}
\input{patterns/18_pointers_to_functions/main}
\input{patterns/185_64bit_in_32_env/main}

\EN{\input{patterns/19_SIMD/main_EN}}
\RU{\input{patterns/19_SIMD/main_RU}}
\DE{\input{patterns/19_SIMD/main_DE}}

\EN{\input{patterns/20_x64/main_EN}}
\RU{\input{patterns/20_x64/main_RU}}

\EN{\input{patterns/205_floating_SIMD/main_EN}}
\RU{\input{patterns/205_floating_SIMD/main_RU}}
\DE{\input{patterns/205_floating_SIMD/main_DE}}

\EN{\input{patterns/ARM/main_EN}}
\RU{\input{patterns/ARM/main_RU}}
\DE{\input{patterns/ARM/main_DE}}

\input{patterns/MIPS/main}

\ifdefined\SPANISH
\chapter{Patrones de código}
\fi % SPANISH

\ifdefined\GERMAN
\chapter{Code-Muster}
\fi % GERMAN

\ifdefined\ENGLISH
\chapter{Code Patterns}
\fi % ENGLISH

\ifdefined\ITALIAN
\chapter{Forme di codice}
\fi % ITALIAN

\ifdefined\RUSSIAN
\chapter{Образцы кода}
\fi % RUSSIAN

\ifdefined\BRAZILIAN
\chapter{Padrões de códigos}
\fi % BRAZILIAN

\ifdefined\THAI
\chapter{รูปแบบของโค้ด}
\fi % THAI

\ifdefined\FRENCH
\chapter{Modèle de code}
\fi % FRENCH

\ifdefined\POLISH
\chapter{\PLph{}}
\fi % POLISH

% sections
\EN{\input{patterns/patterns_opt_dbg_EN}}
\ES{\input{patterns/patterns_opt_dbg_ES}}
\ITA{\input{patterns/patterns_opt_dbg_ITA}}
\PTBR{\input{patterns/patterns_opt_dbg_PTBR}}
\RU{\input{patterns/patterns_opt_dbg_RU}}
\THA{\input{patterns/patterns_opt_dbg_THA}}
\DE{\input{patterns/patterns_opt_dbg_DE}}
\FR{\input{patterns/patterns_opt_dbg_FR}}
\PL{\input{patterns/patterns_opt_dbg_PL}}

\RU{\section{Некоторые базовые понятия}}
\EN{\section{Some basics}}
\DE{\section{Einige Grundlagen}}
\FR{\section{Quelques bases}}
\ES{\section{\ESph{}}}
\ITA{\section{Alcune basi teoriche}}
\PTBR{\section{\PTBRph{}}}
\THA{\section{\THAph{}}}
\PL{\section{\PLph{}}}

% sections:
\EN{\input{patterns/intro_CPU_ISA_EN}}
\ES{\input{patterns/intro_CPU_ISA_ES}}
\ITA{\input{patterns/intro_CPU_ISA_ITA}}
\PTBR{\input{patterns/intro_CPU_ISA_PTBR}}
\RU{\input{patterns/intro_CPU_ISA_RU}}
\DE{\input{patterns/intro_CPU_ISA_DE}}
\FR{\input{patterns/intro_CPU_ISA_FR}}
\PL{\input{patterns/intro_CPU_ISA_PL}}

\EN{\input{patterns/numeral_EN}}
\RU{\input{patterns/numeral_RU}}
\ITA{\input{patterns/numeral_ITA}}
\DE{\input{patterns/numeral_DE}}
\FR{\input{patterns/numeral_FR}}
\PL{\input{patterns/numeral_PL}}

% chapters
\input{patterns/00_empty/main}
\input{patterns/011_ret/main}
\input{patterns/01_helloworld/main}
\input{patterns/015_prolog_epilogue/main}
\input{patterns/02_stack/main}
\input{patterns/03_printf/main}
\input{patterns/04_scanf/main}
\input{patterns/05_passing_arguments/main}
\input{patterns/06_return_results/main}
\input{patterns/061_pointers/main}
\input{patterns/065_GOTO/main}
\input{patterns/07_jcc/main}
\input{patterns/08_switch/main}
\input{patterns/09_loops/main}
\input{patterns/10_strings/main}
\input{patterns/11_arith_optimizations/main}
\input{patterns/12_FPU/main}
\input{patterns/13_arrays/main}
\input{patterns/14_bitfields/main}
\EN{\input{patterns/145_LCG/main_EN}}
\RU{\input{patterns/145_LCG/main_RU}}
\input{patterns/15_structs/main}
\input{patterns/17_unions/main}
\input{patterns/18_pointers_to_functions/main}
\input{patterns/185_64bit_in_32_env/main}

\EN{\input{patterns/19_SIMD/main_EN}}
\RU{\input{patterns/19_SIMD/main_RU}}
\DE{\input{patterns/19_SIMD/main_DE}}

\EN{\input{patterns/20_x64/main_EN}}
\RU{\input{patterns/20_x64/main_RU}}

\EN{\input{patterns/205_floating_SIMD/main_EN}}
\RU{\input{patterns/205_floating_SIMD/main_RU}}
\DE{\input{patterns/205_floating_SIMD/main_DE}}

\EN{\input{patterns/ARM/main_EN}}
\RU{\input{patterns/ARM/main_RU}}
\DE{\input{patterns/ARM/main_DE}}

\input{patterns/MIPS/main}

\EN{\input{patterns/12_FPU/main_EN}}
\RU{\input{patterns/12_FPU/main_RU}}
\DE{\input{patterns/12_FPU/main_DE}}
\FR{\input{patterns/12_FPU/main_FR}}


\ifdefined\SPANISH
\chapter{Patrones de código}
\fi % SPANISH

\ifdefined\GERMAN
\chapter{Code-Muster}
\fi % GERMAN

\ifdefined\ENGLISH
\chapter{Code Patterns}
\fi % ENGLISH

\ifdefined\ITALIAN
\chapter{Forme di codice}
\fi % ITALIAN

\ifdefined\RUSSIAN
\chapter{Образцы кода}
\fi % RUSSIAN

\ifdefined\BRAZILIAN
\chapter{Padrões de códigos}
\fi % BRAZILIAN

\ifdefined\THAI
\chapter{รูปแบบของโค้ด}
\fi % THAI

\ifdefined\FRENCH
\chapter{Modèle de code}
\fi % FRENCH

\ifdefined\POLISH
\chapter{\PLph{}}
\fi % POLISH

% sections
\EN{\input{patterns/patterns_opt_dbg_EN}}
\ES{\input{patterns/patterns_opt_dbg_ES}}
\ITA{\input{patterns/patterns_opt_dbg_ITA}}
\PTBR{\input{patterns/patterns_opt_dbg_PTBR}}
\RU{\input{patterns/patterns_opt_dbg_RU}}
\THA{\input{patterns/patterns_opt_dbg_THA}}
\DE{\input{patterns/patterns_opt_dbg_DE}}
\FR{\input{patterns/patterns_opt_dbg_FR}}
\PL{\input{patterns/patterns_opt_dbg_PL}}

\RU{\section{Некоторые базовые понятия}}
\EN{\section{Some basics}}
\DE{\section{Einige Grundlagen}}
\FR{\section{Quelques bases}}
\ES{\section{\ESph{}}}
\ITA{\section{Alcune basi teoriche}}
\PTBR{\section{\PTBRph{}}}
\THA{\section{\THAph{}}}
\PL{\section{\PLph{}}}

% sections:
\EN{\input{patterns/intro_CPU_ISA_EN}}
\ES{\input{patterns/intro_CPU_ISA_ES}}
\ITA{\input{patterns/intro_CPU_ISA_ITA}}
\PTBR{\input{patterns/intro_CPU_ISA_PTBR}}
\RU{\input{patterns/intro_CPU_ISA_RU}}
\DE{\input{patterns/intro_CPU_ISA_DE}}
\FR{\input{patterns/intro_CPU_ISA_FR}}
\PL{\input{patterns/intro_CPU_ISA_PL}}

\EN{\input{patterns/numeral_EN}}
\RU{\input{patterns/numeral_RU}}
\ITA{\input{patterns/numeral_ITA}}
\DE{\input{patterns/numeral_DE}}
\FR{\input{patterns/numeral_FR}}
\PL{\input{patterns/numeral_PL}}

% chapters
\input{patterns/00_empty/main}
\input{patterns/011_ret/main}
\input{patterns/01_helloworld/main}
\input{patterns/015_prolog_epilogue/main}
\input{patterns/02_stack/main}
\input{patterns/03_printf/main}
\input{patterns/04_scanf/main}
\input{patterns/05_passing_arguments/main}
\input{patterns/06_return_results/main}
\input{patterns/061_pointers/main}
\input{patterns/065_GOTO/main}
\input{patterns/07_jcc/main}
\input{patterns/08_switch/main}
\input{patterns/09_loops/main}
\input{patterns/10_strings/main}
\input{patterns/11_arith_optimizations/main}
\input{patterns/12_FPU/main}
\input{patterns/13_arrays/main}
\input{patterns/14_bitfields/main}
\EN{\input{patterns/145_LCG/main_EN}}
\RU{\input{patterns/145_LCG/main_RU}}
\input{patterns/15_structs/main}
\input{patterns/17_unions/main}
\input{patterns/18_pointers_to_functions/main}
\input{patterns/185_64bit_in_32_env/main}

\EN{\input{patterns/19_SIMD/main_EN}}
\RU{\input{patterns/19_SIMD/main_RU}}
\DE{\input{patterns/19_SIMD/main_DE}}

\EN{\input{patterns/20_x64/main_EN}}
\RU{\input{patterns/20_x64/main_RU}}

\EN{\input{patterns/205_floating_SIMD/main_EN}}
\RU{\input{patterns/205_floating_SIMD/main_RU}}
\DE{\input{patterns/205_floating_SIMD/main_DE}}

\EN{\input{patterns/ARM/main_EN}}
\RU{\input{patterns/ARM/main_RU}}
\DE{\input{patterns/ARM/main_DE}}

\input{patterns/MIPS/main}

\ifdefined\SPANISH
\chapter{Patrones de código}
\fi % SPANISH

\ifdefined\GERMAN
\chapter{Code-Muster}
\fi % GERMAN

\ifdefined\ENGLISH
\chapter{Code Patterns}
\fi % ENGLISH

\ifdefined\ITALIAN
\chapter{Forme di codice}
\fi % ITALIAN

\ifdefined\RUSSIAN
\chapter{Образцы кода}
\fi % RUSSIAN

\ifdefined\BRAZILIAN
\chapter{Padrões de códigos}
\fi % BRAZILIAN

\ifdefined\THAI
\chapter{รูปแบบของโค้ด}
\fi % THAI

\ifdefined\FRENCH
\chapter{Modèle de code}
\fi % FRENCH

\ifdefined\POLISH
\chapter{\PLph{}}
\fi % POLISH

% sections
\EN{\input{patterns/patterns_opt_dbg_EN}}
\ES{\input{patterns/patterns_opt_dbg_ES}}
\ITA{\input{patterns/patterns_opt_dbg_ITA}}
\PTBR{\input{patterns/patterns_opt_dbg_PTBR}}
\RU{\input{patterns/patterns_opt_dbg_RU}}
\THA{\input{patterns/patterns_opt_dbg_THA}}
\DE{\input{patterns/patterns_opt_dbg_DE}}
\FR{\input{patterns/patterns_opt_dbg_FR}}
\PL{\input{patterns/patterns_opt_dbg_PL}}

\RU{\section{Некоторые базовые понятия}}
\EN{\section{Some basics}}
\DE{\section{Einige Grundlagen}}
\FR{\section{Quelques bases}}
\ES{\section{\ESph{}}}
\ITA{\section{Alcune basi teoriche}}
\PTBR{\section{\PTBRph{}}}
\THA{\section{\THAph{}}}
\PL{\section{\PLph{}}}

% sections:
\EN{\input{patterns/intro_CPU_ISA_EN}}
\ES{\input{patterns/intro_CPU_ISA_ES}}
\ITA{\input{patterns/intro_CPU_ISA_ITA}}
\PTBR{\input{patterns/intro_CPU_ISA_PTBR}}
\RU{\input{patterns/intro_CPU_ISA_RU}}
\DE{\input{patterns/intro_CPU_ISA_DE}}
\FR{\input{patterns/intro_CPU_ISA_FR}}
\PL{\input{patterns/intro_CPU_ISA_PL}}

\EN{\input{patterns/numeral_EN}}
\RU{\input{patterns/numeral_RU}}
\ITA{\input{patterns/numeral_ITA}}
\DE{\input{patterns/numeral_DE}}
\FR{\input{patterns/numeral_FR}}
\PL{\input{patterns/numeral_PL}}

% chapters
\input{patterns/00_empty/main}
\input{patterns/011_ret/main}
\input{patterns/01_helloworld/main}
\input{patterns/015_prolog_epilogue/main}
\input{patterns/02_stack/main}
\input{patterns/03_printf/main}
\input{patterns/04_scanf/main}
\input{patterns/05_passing_arguments/main}
\input{patterns/06_return_results/main}
\input{patterns/061_pointers/main}
\input{patterns/065_GOTO/main}
\input{patterns/07_jcc/main}
\input{patterns/08_switch/main}
\input{patterns/09_loops/main}
\input{patterns/10_strings/main}
\input{patterns/11_arith_optimizations/main}
\input{patterns/12_FPU/main}
\input{patterns/13_arrays/main}
\input{patterns/14_bitfields/main}
\EN{\input{patterns/145_LCG/main_EN}}
\RU{\input{patterns/145_LCG/main_RU}}
\input{patterns/15_structs/main}
\input{patterns/17_unions/main}
\input{patterns/18_pointers_to_functions/main}
\input{patterns/185_64bit_in_32_env/main}

\EN{\input{patterns/19_SIMD/main_EN}}
\RU{\input{patterns/19_SIMD/main_RU}}
\DE{\input{patterns/19_SIMD/main_DE}}

\EN{\input{patterns/20_x64/main_EN}}
\RU{\input{patterns/20_x64/main_RU}}

\EN{\input{patterns/205_floating_SIMD/main_EN}}
\RU{\input{patterns/205_floating_SIMD/main_RU}}
\DE{\input{patterns/205_floating_SIMD/main_DE}}

\EN{\input{patterns/ARM/main_EN}}
\RU{\input{patterns/ARM/main_RU}}
\DE{\input{patterns/ARM/main_DE}}

\input{patterns/MIPS/main}

\EN{\section{Returning Values}
\label{ret_val_func}

Another simple function is the one that simply returns a constant value:

\lstinputlisting[caption=\EN{\CCpp Code},style=customc]{patterns/011_ret/1.c}

Let's compile it.

\subsection{x86}

Here's what both the GCC and MSVC compilers produce (with optimization) on the x86 platform:

\lstinputlisting[caption=\Optimizing GCC/MSVC (\assemblyOutput),style=customasmx86]{patterns/011_ret/1.s}

\myindex{x86!\Instructions!RET}
There are just two instructions: the first places the value 123 into the \EAX register,
which is used by convention for storing the return
value, and the second one is \RET, which returns execution to the \gls{caller}.

The caller will take the result from the \EAX register.

\subsection{ARM}

There are a few differences on the ARM platform:

\lstinputlisting[caption=\OptimizingKeilVI (\ARMMode) ASM Output,style=customasmARM]{patterns/011_ret/1_Keil_ARM_O3.s}

ARM uses the register \Reg{0} for returning the results of functions, so 123 is copied into \Reg{0}.

\myindex{ARM!\Instructions!MOV}
\myindex{x86!\Instructions!MOV}
It is worth noting that \MOV is a misleading name for the instruction in both the x86 and ARM \ac{ISA}s.

The data is not in fact \IT{moved}, but \IT{copied}.

\subsection{MIPS}

\label{MIPS_leaf_function_ex1}

The GCC assembly output below lists registers by number:

\lstinputlisting[caption=\Optimizing GCC 4.4.5 (\assemblyOutput),style=customasmMIPS]{patterns/011_ret/MIPS.s}

\dots while \IDA does it by their pseudo names:

\lstinputlisting[caption=\Optimizing GCC 4.4.5 (IDA),style=customasmMIPS]{patterns/011_ret/MIPS_IDA.lst}

The \$2 (or \$V0) register is used to store the function's return value.
\myindex{MIPS!\Pseudoinstructions!LI}
\INS{LI} stands for ``Load Immediate'' and is the MIPS equivalent to \MOV.

\myindex{MIPS!\Instructions!J}
The other instruction is the jump instruction (J or JR) which returns the execution flow to the \gls{caller}.

\myindex{MIPS!Branch delay slot}
You might be wondering why the positions of the load instruction (LI) and the jump instruction (J or JR) are swapped. This is due to a \ac{RISC} feature called ``branch delay slot''.

The reason this happens is a quirk in the architecture of some RISC \ac{ISA}s and isn't important for our
purposes---we must simply keep in mind that in MIPS, the instruction following a jump or branch instruction
is executed \IT{before} the jump/branch instruction itself.

As a consequence, branch instructions always swap places with the instruction executed immediately beforehand.


In practice, functions which merely return 1 (\IT{true}) or 0 (\IT{false}) are very frequent.

The smallest ever of the standard UNIX utilities, \IT{/bin/true} and \IT{/bin/false} return 0 and 1 respectively, as an exit code.
(Zero as an exit code usually means success, non-zero means error.)
}
\RU{\subsubsection{std::string}
\myindex{\Cpp!STL!std::string}
\label{std_string}

\myparagraph{Как устроена структура}

Многие строковые библиотеки \InSqBrackets{\CNotes 2.2} обеспечивают структуру содержащую ссылку 
на буфер собственно со строкой, переменная всегда содержащую длину строки 
(что очень удобно для массы функций \InSqBrackets{\CNotes 2.2.1}) и переменную содержащую текущий размер буфера.

Строка в буфере обыкновенно оканчивается нулем: это для того чтобы указатель на буфер можно было
передавать в функции требующие на вход обычную сишную \ac{ASCIIZ}-строку.

Стандарт \Cpp не описывает, как именно нужно реализовывать std::string,
но, как правило, они реализованы как описано выше, с небольшими дополнениями.

Строки в \Cpp это не класс (как, например, QString в Qt), а темплейт (basic\_string), 
это сделано для того чтобы поддерживать 
строки содержащие разного типа символы: как минимум \Tchar и \IT{wchar\_t}.

Так что, std::string это класс с базовым типом \Tchar.

А std::wstring это класс с базовым типом \IT{wchar\_t}.

\mysubparagraph{MSVC}

В реализации MSVC, вместо ссылки на буфер может содержаться сам буфер (если строка короче 16-и символов).

Это означает, что каждая короткая строка будет занимать в памяти по крайней мере $16 + 4 + 4 = 24$ 
байт для 32-битной среды либо $16 + 8 + 8 = 32$ 
байта в 64-битной, а если строка длиннее 16-и символов, то прибавьте еще длину самой строки.

\lstinputlisting[caption=пример для MSVC,style=customc]{\CURPATH/STL/string/MSVC_RU.cpp}

Собственно, из этого исходника почти всё ясно.

Несколько замечаний:

Если строка короче 16-и символов, 
то отдельный буфер для строки в \glslink{heap}{куче} выделяться не будет.

Это удобно потому что на практике, основная часть строк действительно короткие.
Вероятно, разработчики в Microsoft выбрали размер в 16 символов как разумный баланс.

Теперь очень важный момент в конце функции main(): мы не пользуемся методом c\_str(), тем не менее,
если это скомпилировать и запустить, то обе строки появятся в консоли!

Работает это вот почему.

В первом случае строка короче 16-и символов и в начале объекта std::string (его можно рассматривать
просто как структуру) расположен буфер с этой строкой.
\printf трактует указатель как указатель на массив символов оканчивающийся нулем и поэтому всё работает.

Вывод второй строки (длиннее 16-и символов) даже еще опаснее: это вообще типичная программистская ошибка 
(или опечатка), забыть дописать c\_str().
Это работает потому что в это время в начале структуры расположен указатель на буфер.
Это может надолго остаться незамеченным: до тех пока там не появится строка 
короче 16-и символов, тогда процесс упадет.

\mysubparagraph{GCC}

В реализации GCC в структуре есть еще одна переменная --- reference count.

Интересно, что указатель на экземпляр класса std::string в GCC указывает не на начало самой структуры, 
а на указатель на буфера.
В libstdc++-v3\textbackslash{}include\textbackslash{}bits\textbackslash{}basic\_string.h 
мы можем прочитать что это сделано для удобства отладки:

\begin{lstlisting}
   *  The reason you want _M_data pointing to the character %array and
   *  not the _Rep is so that the debugger can see the string
   *  contents. (Probably we should add a non-inline member to get
   *  the _Rep for the debugger to use, so users can check the actual
   *  string length.)
\end{lstlisting}

\href{http://go.yurichev.com/17085}{исходный код basic\_string.h}

В нашем примере мы учитываем это:

\lstinputlisting[caption=пример для GCC,style=customc]{\CURPATH/STL/string/GCC_RU.cpp}

Нужны еще небольшие хаки чтобы сымитировать типичную ошибку, которую мы уже видели выше, из-за
более ужесточенной проверки типов в GCC, тем не менее, printf() работает и здесь без c\_str().

\myparagraph{Чуть более сложный пример}

\lstinputlisting[style=customc]{\CURPATH/STL/string/3.cpp}

\lstinputlisting[caption=MSVC 2012,style=customasmx86]{\CURPATH/STL/string/3_MSVC_RU.asm}

Собственно, компилятор не конструирует строки статически: да в общем-то и как
это возможно, если буфер с ней нужно хранить в \glslink{heap}{куче}?

Вместо этого в сегменте данных хранятся обычные \ac{ASCIIZ}-строки, а позже, во время выполнения, 
при помощи метода \q{assign}, конструируются строки s1 и s2
.
При помощи \TT{operator+}, создается строка s3.

Обратите внимание на то что вызов метода c\_str() отсутствует,
потому что его код достаточно короткий и компилятор вставил его прямо здесь:
если строка короче 16-и байт, то в регистре EAX остается указатель на буфер,
а если длиннее, то из этого же места достается адрес на буфер расположенный в \glslink{heap}{куче}.

Далее следуют вызовы трех деструкторов, причем, они вызываются только если строка длиннее 16-и байт:
тогда нужно освободить буфера в \glslink{heap}{куче}.
В противном случае, так как все три объекта std::string хранятся в стеке,
они освобождаются автоматически после выхода из функции.

Следовательно, работа с короткими строками более быстрая из-за м\'{е}ньшего обращения к \glslink{heap}{куче}.

Код на GCC даже проще (из-за того, что в GCC, как мы уже видели, не реализована возможность хранить короткую
строку прямо в структуре):

% TODO1 comment each function meaning
\lstinputlisting[caption=GCC 4.8.1,style=customasmx86]{\CURPATH/STL/string/3_GCC_RU.s}

Можно заметить, что в деструкторы передается не указатель на объект,
а указатель на место за 12 байт (или 3 слова) перед ним, то есть, на настоящее начало структуры.

\myparagraph{std::string как глобальная переменная}
\label{sec:std_string_as_global_variable}

Опытные программисты на \Cpp знают, что глобальные переменные \ac{STL}-типов вполне можно объявлять.

Да, действительно:

\lstinputlisting[style=customc]{\CURPATH/STL/string/5.cpp}

Но как и где будет вызываться конструктор \TT{std::string}?

На самом деле, эта переменная будет инициализирована даже перед началом \main.

\lstinputlisting[caption=MSVC 2012: здесь конструируется глобальная переменная{,} а также регистрируется её деструктор,style=customasmx86]{\CURPATH/STL/string/5_MSVC_p2.asm}

\lstinputlisting[caption=MSVC 2012: здесь глобальная переменная используется в \main,style=customasmx86]{\CURPATH/STL/string/5_MSVC_p1.asm}

\lstinputlisting[caption=MSVC 2012: эта функция-деструктор вызывается перед выходом,style=customasmx86]{\CURPATH/STL/string/5_MSVC_p3.asm}

\myindex{\CStandardLibrary!atexit()}
В реальности, из \ac{CRT}, еще до вызова main(), вызывается специальная функция,
в которой перечислены все конструкторы подобных переменных.
Более того: при помощи atexit() регистрируется функция, которая будет вызвана в конце работы программы:
в этой функции компилятор собирает вызовы деструкторов всех подобных глобальных переменных.

GCC работает похожим образом:

\lstinputlisting[caption=GCC 4.8.1,style=customasmx86]{\CURPATH/STL/string/5_GCC.s}

Но он не выделяет отдельной функции в которой будут собраны деструкторы: 
каждый деструктор передается в atexit() по одному.

% TODO а если глобальная STL-переменная в другом модуле? надо проверить.

}
\ifdefined\SPANISH
\chapter{Patrones de código}
\fi % SPANISH

\ifdefined\GERMAN
\chapter{Code-Muster}
\fi % GERMAN

\ifdefined\ENGLISH
\chapter{Code Patterns}
\fi % ENGLISH

\ifdefined\ITALIAN
\chapter{Forme di codice}
\fi % ITALIAN

\ifdefined\RUSSIAN
\chapter{Образцы кода}
\fi % RUSSIAN

\ifdefined\BRAZILIAN
\chapter{Padrões de códigos}
\fi % BRAZILIAN

\ifdefined\THAI
\chapter{รูปแบบของโค้ด}
\fi % THAI

\ifdefined\FRENCH
\chapter{Modèle de code}
\fi % FRENCH

\ifdefined\POLISH
\chapter{\PLph{}}
\fi % POLISH

% sections
\EN{\input{patterns/patterns_opt_dbg_EN}}
\ES{\input{patterns/patterns_opt_dbg_ES}}
\ITA{\input{patterns/patterns_opt_dbg_ITA}}
\PTBR{\input{patterns/patterns_opt_dbg_PTBR}}
\RU{\input{patterns/patterns_opt_dbg_RU}}
\THA{\input{patterns/patterns_opt_dbg_THA}}
\DE{\input{patterns/patterns_opt_dbg_DE}}
\FR{\input{patterns/patterns_opt_dbg_FR}}
\PL{\input{patterns/patterns_opt_dbg_PL}}

\RU{\section{Некоторые базовые понятия}}
\EN{\section{Some basics}}
\DE{\section{Einige Grundlagen}}
\FR{\section{Quelques bases}}
\ES{\section{\ESph{}}}
\ITA{\section{Alcune basi teoriche}}
\PTBR{\section{\PTBRph{}}}
\THA{\section{\THAph{}}}
\PL{\section{\PLph{}}}

% sections:
\EN{\input{patterns/intro_CPU_ISA_EN}}
\ES{\input{patterns/intro_CPU_ISA_ES}}
\ITA{\input{patterns/intro_CPU_ISA_ITA}}
\PTBR{\input{patterns/intro_CPU_ISA_PTBR}}
\RU{\input{patterns/intro_CPU_ISA_RU}}
\DE{\input{patterns/intro_CPU_ISA_DE}}
\FR{\input{patterns/intro_CPU_ISA_FR}}
\PL{\input{patterns/intro_CPU_ISA_PL}}

\EN{\input{patterns/numeral_EN}}
\RU{\input{patterns/numeral_RU}}
\ITA{\input{patterns/numeral_ITA}}
\DE{\input{patterns/numeral_DE}}
\FR{\input{patterns/numeral_FR}}
\PL{\input{patterns/numeral_PL}}

% chapters
\input{patterns/00_empty/main}
\input{patterns/011_ret/main}
\input{patterns/01_helloworld/main}
\input{patterns/015_prolog_epilogue/main}
\input{patterns/02_stack/main}
\input{patterns/03_printf/main}
\input{patterns/04_scanf/main}
\input{patterns/05_passing_arguments/main}
\input{patterns/06_return_results/main}
\input{patterns/061_pointers/main}
\input{patterns/065_GOTO/main}
\input{patterns/07_jcc/main}
\input{patterns/08_switch/main}
\input{patterns/09_loops/main}
\input{patterns/10_strings/main}
\input{patterns/11_arith_optimizations/main}
\input{patterns/12_FPU/main}
\input{patterns/13_arrays/main}
\input{patterns/14_bitfields/main}
\EN{\input{patterns/145_LCG/main_EN}}
\RU{\input{patterns/145_LCG/main_RU}}
\input{patterns/15_structs/main}
\input{patterns/17_unions/main}
\input{patterns/18_pointers_to_functions/main}
\input{patterns/185_64bit_in_32_env/main}

\EN{\input{patterns/19_SIMD/main_EN}}
\RU{\input{patterns/19_SIMD/main_RU}}
\DE{\input{patterns/19_SIMD/main_DE}}

\EN{\input{patterns/20_x64/main_EN}}
\RU{\input{patterns/20_x64/main_RU}}

\EN{\input{patterns/205_floating_SIMD/main_EN}}
\RU{\input{patterns/205_floating_SIMD/main_RU}}
\DE{\input{patterns/205_floating_SIMD/main_DE}}

\EN{\input{patterns/ARM/main_EN}}
\RU{\input{patterns/ARM/main_RU}}
\DE{\input{patterns/ARM/main_DE}}

\input{patterns/MIPS/main}

\ifdefined\SPANISH
\chapter{Patrones de código}
\fi % SPANISH

\ifdefined\GERMAN
\chapter{Code-Muster}
\fi % GERMAN

\ifdefined\ENGLISH
\chapter{Code Patterns}
\fi % ENGLISH

\ifdefined\ITALIAN
\chapter{Forme di codice}
\fi % ITALIAN

\ifdefined\RUSSIAN
\chapter{Образцы кода}
\fi % RUSSIAN

\ifdefined\BRAZILIAN
\chapter{Padrões de códigos}
\fi % BRAZILIAN

\ifdefined\THAI
\chapter{รูปแบบของโค้ด}
\fi % THAI

\ifdefined\FRENCH
\chapter{Modèle de code}
\fi % FRENCH

\ifdefined\POLISH
\chapter{\PLph{}}
\fi % POLISH

% sections
\EN{\input{patterns/patterns_opt_dbg_EN}}
\ES{\input{patterns/patterns_opt_dbg_ES}}
\ITA{\input{patterns/patterns_opt_dbg_ITA}}
\PTBR{\input{patterns/patterns_opt_dbg_PTBR}}
\RU{\input{patterns/patterns_opt_dbg_RU}}
\THA{\input{patterns/patterns_opt_dbg_THA}}
\DE{\input{patterns/patterns_opt_dbg_DE}}
\FR{\input{patterns/patterns_opt_dbg_FR}}
\PL{\input{patterns/patterns_opt_dbg_PL}}

\RU{\section{Некоторые базовые понятия}}
\EN{\section{Some basics}}
\DE{\section{Einige Grundlagen}}
\FR{\section{Quelques bases}}
\ES{\section{\ESph{}}}
\ITA{\section{Alcune basi teoriche}}
\PTBR{\section{\PTBRph{}}}
\THA{\section{\THAph{}}}
\PL{\section{\PLph{}}}

% sections:
\EN{\input{patterns/intro_CPU_ISA_EN}}
\ES{\input{patterns/intro_CPU_ISA_ES}}
\ITA{\input{patterns/intro_CPU_ISA_ITA}}
\PTBR{\input{patterns/intro_CPU_ISA_PTBR}}
\RU{\input{patterns/intro_CPU_ISA_RU}}
\DE{\input{patterns/intro_CPU_ISA_DE}}
\FR{\input{patterns/intro_CPU_ISA_FR}}
\PL{\input{patterns/intro_CPU_ISA_PL}}

\EN{\input{patterns/numeral_EN}}
\RU{\input{patterns/numeral_RU}}
\ITA{\input{patterns/numeral_ITA}}
\DE{\input{patterns/numeral_DE}}
\FR{\input{patterns/numeral_FR}}
\PL{\input{patterns/numeral_PL}}

% chapters
\input{patterns/00_empty/main}
\input{patterns/011_ret/main}
\input{patterns/01_helloworld/main}
\input{patterns/015_prolog_epilogue/main}
\input{patterns/02_stack/main}
\input{patterns/03_printf/main}
\input{patterns/04_scanf/main}
\input{patterns/05_passing_arguments/main}
\input{patterns/06_return_results/main}
\input{patterns/061_pointers/main}
\input{patterns/065_GOTO/main}
\input{patterns/07_jcc/main}
\input{patterns/08_switch/main}
\input{patterns/09_loops/main}
\input{patterns/10_strings/main}
\input{patterns/11_arith_optimizations/main}
\input{patterns/12_FPU/main}
\input{patterns/13_arrays/main}
\input{patterns/14_bitfields/main}
\EN{\input{patterns/145_LCG/main_EN}}
\RU{\input{patterns/145_LCG/main_RU}}
\input{patterns/15_structs/main}
\input{patterns/17_unions/main}
\input{patterns/18_pointers_to_functions/main}
\input{patterns/185_64bit_in_32_env/main}

\EN{\input{patterns/19_SIMD/main_EN}}
\RU{\input{patterns/19_SIMD/main_RU}}
\DE{\input{patterns/19_SIMD/main_DE}}

\EN{\input{patterns/20_x64/main_EN}}
\RU{\input{patterns/20_x64/main_RU}}

\EN{\input{patterns/205_floating_SIMD/main_EN}}
\RU{\input{patterns/205_floating_SIMD/main_RU}}
\DE{\input{patterns/205_floating_SIMD/main_DE}}

\EN{\input{patterns/ARM/main_EN}}
\RU{\input{patterns/ARM/main_RU}}
\DE{\input{patterns/ARM/main_DE}}

\input{patterns/MIPS/main}

\ifdefined\SPANISH
\chapter{Patrones de código}
\fi % SPANISH

\ifdefined\GERMAN
\chapter{Code-Muster}
\fi % GERMAN

\ifdefined\ENGLISH
\chapter{Code Patterns}
\fi % ENGLISH

\ifdefined\ITALIAN
\chapter{Forme di codice}
\fi % ITALIAN

\ifdefined\RUSSIAN
\chapter{Образцы кода}
\fi % RUSSIAN

\ifdefined\BRAZILIAN
\chapter{Padrões de códigos}
\fi % BRAZILIAN

\ifdefined\THAI
\chapter{รูปแบบของโค้ด}
\fi % THAI

\ifdefined\FRENCH
\chapter{Modèle de code}
\fi % FRENCH

\ifdefined\POLISH
\chapter{\PLph{}}
\fi % POLISH

% sections
\EN{\input{patterns/patterns_opt_dbg_EN}}
\ES{\input{patterns/patterns_opt_dbg_ES}}
\ITA{\input{patterns/patterns_opt_dbg_ITA}}
\PTBR{\input{patterns/patterns_opt_dbg_PTBR}}
\RU{\input{patterns/patterns_opt_dbg_RU}}
\THA{\input{patterns/patterns_opt_dbg_THA}}
\DE{\input{patterns/patterns_opt_dbg_DE}}
\FR{\input{patterns/patterns_opt_dbg_FR}}
\PL{\input{patterns/patterns_opt_dbg_PL}}

\RU{\section{Некоторые базовые понятия}}
\EN{\section{Some basics}}
\DE{\section{Einige Grundlagen}}
\FR{\section{Quelques bases}}
\ES{\section{\ESph{}}}
\ITA{\section{Alcune basi teoriche}}
\PTBR{\section{\PTBRph{}}}
\THA{\section{\THAph{}}}
\PL{\section{\PLph{}}}

% sections:
\EN{\input{patterns/intro_CPU_ISA_EN}}
\ES{\input{patterns/intro_CPU_ISA_ES}}
\ITA{\input{patterns/intro_CPU_ISA_ITA}}
\PTBR{\input{patterns/intro_CPU_ISA_PTBR}}
\RU{\input{patterns/intro_CPU_ISA_RU}}
\DE{\input{patterns/intro_CPU_ISA_DE}}
\FR{\input{patterns/intro_CPU_ISA_FR}}
\PL{\input{patterns/intro_CPU_ISA_PL}}

\EN{\input{patterns/numeral_EN}}
\RU{\input{patterns/numeral_RU}}
\ITA{\input{patterns/numeral_ITA}}
\DE{\input{patterns/numeral_DE}}
\FR{\input{patterns/numeral_FR}}
\PL{\input{patterns/numeral_PL}}

% chapters
\input{patterns/00_empty/main}
\input{patterns/011_ret/main}
\input{patterns/01_helloworld/main}
\input{patterns/015_prolog_epilogue/main}
\input{patterns/02_stack/main}
\input{patterns/03_printf/main}
\input{patterns/04_scanf/main}
\input{patterns/05_passing_arguments/main}
\input{patterns/06_return_results/main}
\input{patterns/061_pointers/main}
\input{patterns/065_GOTO/main}
\input{patterns/07_jcc/main}
\input{patterns/08_switch/main}
\input{patterns/09_loops/main}
\input{patterns/10_strings/main}
\input{patterns/11_arith_optimizations/main}
\input{patterns/12_FPU/main}
\input{patterns/13_arrays/main}
\input{patterns/14_bitfields/main}
\EN{\input{patterns/145_LCG/main_EN}}
\RU{\input{patterns/145_LCG/main_RU}}
\input{patterns/15_structs/main}
\input{patterns/17_unions/main}
\input{patterns/18_pointers_to_functions/main}
\input{patterns/185_64bit_in_32_env/main}

\EN{\input{patterns/19_SIMD/main_EN}}
\RU{\input{patterns/19_SIMD/main_RU}}
\DE{\input{patterns/19_SIMD/main_DE}}

\EN{\input{patterns/20_x64/main_EN}}
\RU{\input{patterns/20_x64/main_RU}}

\EN{\input{patterns/205_floating_SIMD/main_EN}}
\RU{\input{patterns/205_floating_SIMD/main_RU}}
\DE{\input{patterns/205_floating_SIMD/main_DE}}

\EN{\input{patterns/ARM/main_EN}}
\RU{\input{patterns/ARM/main_RU}}
\DE{\input{patterns/ARM/main_DE}}

\input{patterns/MIPS/main}

\ifdefined\SPANISH
\chapter{Patrones de código}
\fi % SPANISH

\ifdefined\GERMAN
\chapter{Code-Muster}
\fi % GERMAN

\ifdefined\ENGLISH
\chapter{Code Patterns}
\fi % ENGLISH

\ifdefined\ITALIAN
\chapter{Forme di codice}
\fi % ITALIAN

\ifdefined\RUSSIAN
\chapter{Образцы кода}
\fi % RUSSIAN

\ifdefined\BRAZILIAN
\chapter{Padrões de códigos}
\fi % BRAZILIAN

\ifdefined\THAI
\chapter{รูปแบบของโค้ด}
\fi % THAI

\ifdefined\FRENCH
\chapter{Modèle de code}
\fi % FRENCH

\ifdefined\POLISH
\chapter{\PLph{}}
\fi % POLISH

% sections
\EN{\input{patterns/patterns_opt_dbg_EN}}
\ES{\input{patterns/patterns_opt_dbg_ES}}
\ITA{\input{patterns/patterns_opt_dbg_ITA}}
\PTBR{\input{patterns/patterns_opt_dbg_PTBR}}
\RU{\input{patterns/patterns_opt_dbg_RU}}
\THA{\input{patterns/patterns_opt_dbg_THA}}
\DE{\input{patterns/patterns_opt_dbg_DE}}
\FR{\input{patterns/patterns_opt_dbg_FR}}
\PL{\input{patterns/patterns_opt_dbg_PL}}

\RU{\section{Некоторые базовые понятия}}
\EN{\section{Some basics}}
\DE{\section{Einige Grundlagen}}
\FR{\section{Quelques bases}}
\ES{\section{\ESph{}}}
\ITA{\section{Alcune basi teoriche}}
\PTBR{\section{\PTBRph{}}}
\THA{\section{\THAph{}}}
\PL{\section{\PLph{}}}

% sections:
\EN{\input{patterns/intro_CPU_ISA_EN}}
\ES{\input{patterns/intro_CPU_ISA_ES}}
\ITA{\input{patterns/intro_CPU_ISA_ITA}}
\PTBR{\input{patterns/intro_CPU_ISA_PTBR}}
\RU{\input{patterns/intro_CPU_ISA_RU}}
\DE{\input{patterns/intro_CPU_ISA_DE}}
\FR{\input{patterns/intro_CPU_ISA_FR}}
\PL{\input{patterns/intro_CPU_ISA_PL}}

\EN{\input{patterns/numeral_EN}}
\RU{\input{patterns/numeral_RU}}
\ITA{\input{patterns/numeral_ITA}}
\DE{\input{patterns/numeral_DE}}
\FR{\input{patterns/numeral_FR}}
\PL{\input{patterns/numeral_PL}}

% chapters
\input{patterns/00_empty/main}
\input{patterns/011_ret/main}
\input{patterns/01_helloworld/main}
\input{patterns/015_prolog_epilogue/main}
\input{patterns/02_stack/main}
\input{patterns/03_printf/main}
\input{patterns/04_scanf/main}
\input{patterns/05_passing_arguments/main}
\input{patterns/06_return_results/main}
\input{patterns/061_pointers/main}
\input{patterns/065_GOTO/main}
\input{patterns/07_jcc/main}
\input{patterns/08_switch/main}
\input{patterns/09_loops/main}
\input{patterns/10_strings/main}
\input{patterns/11_arith_optimizations/main}
\input{patterns/12_FPU/main}
\input{patterns/13_arrays/main}
\input{patterns/14_bitfields/main}
\EN{\input{patterns/145_LCG/main_EN}}
\RU{\input{patterns/145_LCG/main_RU}}
\input{patterns/15_structs/main}
\input{patterns/17_unions/main}
\input{patterns/18_pointers_to_functions/main}
\input{patterns/185_64bit_in_32_env/main}

\EN{\input{patterns/19_SIMD/main_EN}}
\RU{\input{patterns/19_SIMD/main_RU}}
\DE{\input{patterns/19_SIMD/main_DE}}

\EN{\input{patterns/20_x64/main_EN}}
\RU{\input{patterns/20_x64/main_RU}}

\EN{\input{patterns/205_floating_SIMD/main_EN}}
\RU{\input{patterns/205_floating_SIMD/main_RU}}
\DE{\input{patterns/205_floating_SIMD/main_DE}}

\EN{\input{patterns/ARM/main_EN}}
\RU{\input{patterns/ARM/main_RU}}
\DE{\input{patterns/ARM/main_DE}}

\input{patterns/MIPS/main}


\EN{\section{Returning Values}
\label{ret_val_func}

Another simple function is the one that simply returns a constant value:

\lstinputlisting[caption=\EN{\CCpp Code},style=customc]{patterns/011_ret/1.c}

Let's compile it.

\subsection{x86}

Here's what both the GCC and MSVC compilers produce (with optimization) on the x86 platform:

\lstinputlisting[caption=\Optimizing GCC/MSVC (\assemblyOutput),style=customasmx86]{patterns/011_ret/1.s}

\myindex{x86!\Instructions!RET}
There are just two instructions: the first places the value 123 into the \EAX register,
which is used by convention for storing the return
value, and the second one is \RET, which returns execution to the \gls{caller}.

The caller will take the result from the \EAX register.

\subsection{ARM}

There are a few differences on the ARM platform:

\lstinputlisting[caption=\OptimizingKeilVI (\ARMMode) ASM Output,style=customasmARM]{patterns/011_ret/1_Keil_ARM_O3.s}

ARM uses the register \Reg{0} for returning the results of functions, so 123 is copied into \Reg{0}.

\myindex{ARM!\Instructions!MOV}
\myindex{x86!\Instructions!MOV}
It is worth noting that \MOV is a misleading name for the instruction in both the x86 and ARM \ac{ISA}s.

The data is not in fact \IT{moved}, but \IT{copied}.

\subsection{MIPS}

\label{MIPS_leaf_function_ex1}

The GCC assembly output below lists registers by number:

\lstinputlisting[caption=\Optimizing GCC 4.4.5 (\assemblyOutput),style=customasmMIPS]{patterns/011_ret/MIPS.s}

\dots while \IDA does it by their pseudo names:

\lstinputlisting[caption=\Optimizing GCC 4.4.5 (IDA),style=customasmMIPS]{patterns/011_ret/MIPS_IDA.lst}

The \$2 (or \$V0) register is used to store the function's return value.
\myindex{MIPS!\Pseudoinstructions!LI}
\INS{LI} stands for ``Load Immediate'' and is the MIPS equivalent to \MOV.

\myindex{MIPS!\Instructions!J}
The other instruction is the jump instruction (J or JR) which returns the execution flow to the \gls{caller}.

\myindex{MIPS!Branch delay slot}
You might be wondering why the positions of the load instruction (LI) and the jump instruction (J or JR) are swapped. This is due to a \ac{RISC} feature called ``branch delay slot''.

The reason this happens is a quirk in the architecture of some RISC \ac{ISA}s and isn't important for our
purposes---we must simply keep in mind that in MIPS, the instruction following a jump or branch instruction
is executed \IT{before} the jump/branch instruction itself.

As a consequence, branch instructions always swap places with the instruction executed immediately beforehand.


In practice, functions which merely return 1 (\IT{true}) or 0 (\IT{false}) are very frequent.

The smallest ever of the standard UNIX utilities, \IT{/bin/true} and \IT{/bin/false} return 0 and 1 respectively, as an exit code.
(Zero as an exit code usually means success, non-zero means error.)
}
\RU{\subsubsection{std::string}
\myindex{\Cpp!STL!std::string}
\label{std_string}

\myparagraph{Как устроена структура}

Многие строковые библиотеки \InSqBrackets{\CNotes 2.2} обеспечивают структуру содержащую ссылку 
на буфер собственно со строкой, переменная всегда содержащую длину строки 
(что очень удобно для массы функций \InSqBrackets{\CNotes 2.2.1}) и переменную содержащую текущий размер буфера.

Строка в буфере обыкновенно оканчивается нулем: это для того чтобы указатель на буфер можно было
передавать в функции требующие на вход обычную сишную \ac{ASCIIZ}-строку.

Стандарт \Cpp не описывает, как именно нужно реализовывать std::string,
но, как правило, они реализованы как описано выше, с небольшими дополнениями.

Строки в \Cpp это не класс (как, например, QString в Qt), а темплейт (basic\_string), 
это сделано для того чтобы поддерживать 
строки содержащие разного типа символы: как минимум \Tchar и \IT{wchar\_t}.

Так что, std::string это класс с базовым типом \Tchar.

А std::wstring это класс с базовым типом \IT{wchar\_t}.

\mysubparagraph{MSVC}

В реализации MSVC, вместо ссылки на буфер может содержаться сам буфер (если строка короче 16-и символов).

Это означает, что каждая короткая строка будет занимать в памяти по крайней мере $16 + 4 + 4 = 24$ 
байт для 32-битной среды либо $16 + 8 + 8 = 32$ 
байта в 64-битной, а если строка длиннее 16-и символов, то прибавьте еще длину самой строки.

\lstinputlisting[caption=пример для MSVC,style=customc]{\CURPATH/STL/string/MSVC_RU.cpp}

Собственно, из этого исходника почти всё ясно.

Несколько замечаний:

Если строка короче 16-и символов, 
то отдельный буфер для строки в \glslink{heap}{куче} выделяться не будет.

Это удобно потому что на практике, основная часть строк действительно короткие.
Вероятно, разработчики в Microsoft выбрали размер в 16 символов как разумный баланс.

Теперь очень важный момент в конце функции main(): мы не пользуемся методом c\_str(), тем не менее,
если это скомпилировать и запустить, то обе строки появятся в консоли!

Работает это вот почему.

В первом случае строка короче 16-и символов и в начале объекта std::string (его можно рассматривать
просто как структуру) расположен буфер с этой строкой.
\printf трактует указатель как указатель на массив символов оканчивающийся нулем и поэтому всё работает.

Вывод второй строки (длиннее 16-и символов) даже еще опаснее: это вообще типичная программистская ошибка 
(или опечатка), забыть дописать c\_str().
Это работает потому что в это время в начале структуры расположен указатель на буфер.
Это может надолго остаться незамеченным: до тех пока там не появится строка 
короче 16-и символов, тогда процесс упадет.

\mysubparagraph{GCC}

В реализации GCC в структуре есть еще одна переменная --- reference count.

Интересно, что указатель на экземпляр класса std::string в GCC указывает не на начало самой структуры, 
а на указатель на буфера.
В libstdc++-v3\textbackslash{}include\textbackslash{}bits\textbackslash{}basic\_string.h 
мы можем прочитать что это сделано для удобства отладки:

\begin{lstlisting}
   *  The reason you want _M_data pointing to the character %array and
   *  not the _Rep is so that the debugger can see the string
   *  contents. (Probably we should add a non-inline member to get
   *  the _Rep for the debugger to use, so users can check the actual
   *  string length.)
\end{lstlisting}

\href{http://go.yurichev.com/17085}{исходный код basic\_string.h}

В нашем примере мы учитываем это:

\lstinputlisting[caption=пример для GCC,style=customc]{\CURPATH/STL/string/GCC_RU.cpp}

Нужны еще небольшие хаки чтобы сымитировать типичную ошибку, которую мы уже видели выше, из-за
более ужесточенной проверки типов в GCC, тем не менее, printf() работает и здесь без c\_str().

\myparagraph{Чуть более сложный пример}

\lstinputlisting[style=customc]{\CURPATH/STL/string/3.cpp}

\lstinputlisting[caption=MSVC 2012,style=customasmx86]{\CURPATH/STL/string/3_MSVC_RU.asm}

Собственно, компилятор не конструирует строки статически: да в общем-то и как
это возможно, если буфер с ней нужно хранить в \glslink{heap}{куче}?

Вместо этого в сегменте данных хранятся обычные \ac{ASCIIZ}-строки, а позже, во время выполнения, 
при помощи метода \q{assign}, конструируются строки s1 и s2
.
При помощи \TT{operator+}, создается строка s3.

Обратите внимание на то что вызов метода c\_str() отсутствует,
потому что его код достаточно короткий и компилятор вставил его прямо здесь:
если строка короче 16-и байт, то в регистре EAX остается указатель на буфер,
а если длиннее, то из этого же места достается адрес на буфер расположенный в \glslink{heap}{куче}.

Далее следуют вызовы трех деструкторов, причем, они вызываются только если строка длиннее 16-и байт:
тогда нужно освободить буфера в \glslink{heap}{куче}.
В противном случае, так как все три объекта std::string хранятся в стеке,
они освобождаются автоматически после выхода из функции.

Следовательно, работа с короткими строками более быстрая из-за м\'{е}ньшего обращения к \glslink{heap}{куче}.

Код на GCC даже проще (из-за того, что в GCC, как мы уже видели, не реализована возможность хранить короткую
строку прямо в структуре):

% TODO1 comment each function meaning
\lstinputlisting[caption=GCC 4.8.1,style=customasmx86]{\CURPATH/STL/string/3_GCC_RU.s}

Можно заметить, что в деструкторы передается не указатель на объект,
а указатель на место за 12 байт (или 3 слова) перед ним, то есть, на настоящее начало структуры.

\myparagraph{std::string как глобальная переменная}
\label{sec:std_string_as_global_variable}

Опытные программисты на \Cpp знают, что глобальные переменные \ac{STL}-типов вполне можно объявлять.

Да, действительно:

\lstinputlisting[style=customc]{\CURPATH/STL/string/5.cpp}

Но как и где будет вызываться конструктор \TT{std::string}?

На самом деле, эта переменная будет инициализирована даже перед началом \main.

\lstinputlisting[caption=MSVC 2012: здесь конструируется глобальная переменная{,} а также регистрируется её деструктор,style=customasmx86]{\CURPATH/STL/string/5_MSVC_p2.asm}

\lstinputlisting[caption=MSVC 2012: здесь глобальная переменная используется в \main,style=customasmx86]{\CURPATH/STL/string/5_MSVC_p1.asm}

\lstinputlisting[caption=MSVC 2012: эта функция-деструктор вызывается перед выходом,style=customasmx86]{\CURPATH/STL/string/5_MSVC_p3.asm}

\myindex{\CStandardLibrary!atexit()}
В реальности, из \ac{CRT}, еще до вызова main(), вызывается специальная функция,
в которой перечислены все конструкторы подобных переменных.
Более того: при помощи atexit() регистрируется функция, которая будет вызвана в конце работы программы:
в этой функции компилятор собирает вызовы деструкторов всех подобных глобальных переменных.

GCC работает похожим образом:

\lstinputlisting[caption=GCC 4.8.1,style=customasmx86]{\CURPATH/STL/string/5_GCC.s}

Но он не выделяет отдельной функции в которой будут собраны деструкторы: 
каждый деструктор передается в atexit() по одному.

% TODO а если глобальная STL-переменная в другом модуле? надо проверить.

}
\DE{\subsection{Einfachste XOR-Verschlüsselung überhaupt}

Ich habe einmal eine Software gesehen, bei der alle Debugging-Ausgaben mit XOR mit dem Wert 3
verschlüsselt wurden. Mit anderen Worten, die beiden niedrigsten Bits aller Buchstaben wurden invertiert.

``Hello, world'' wurde zu ``Kfool/\#tlqog'':

\begin{lstlisting}
#!/usr/bin/python

msg="Hello, world!"

print "".join(map(lambda x: chr(ord(x)^3), msg))
\end{lstlisting}

Das ist eine ziemlich interessante Verschlüsselung (oder besser eine Verschleierung),
weil sie zwei wichtige Eigenschaften hat:
1) es ist eine einzige Funktion zum Verschlüsseln und entschlüsseln, sie muss nur wiederholt angewendet werden
2) die entstehenden Buchstaben befinden sich im druckbaren Bereich, also die ganze Zeichenkette kann ohne
Escape-Symbole im Code verwendet werden.

Die zweite Eigenschaft nutzt die Tatsache, dass alle druckbaren Zeichen in Reihen organisiert sind: 0x2x-0x7x,
und wenn die beiden niederwertigsten Bits invertiert werden, wird der Buchstabe um eine oder drei Stellen nach
links oder rechts \IT{verschoben}, aber niemals in eine andere Reihe:

\begin{figure}[H]
\centering
\includegraphics[width=0.7\textwidth]{ascii_clean.png}
\caption{7-Bit \ac{ASCII} Tabelle in Emacs}
\end{figure}

\dots mit dem Zeichen 0x7F als einziger Ausnahme.

Im Folgenden werden also beispielsweise die Zeichen A-Z \IT{verschlüsselt}:

\begin{lstlisting}
#!/usr/bin/python

msg="@ABCDEFGHIJKLMNO"

print "".join(map(lambda x: chr(ord(x)^3), msg))
\end{lstlisting}

Ergebnis:
% FIXME \verb  --  relevant comment for German?
\begin{lstlisting}
CBA@GFEDKJIHONML
\end{lstlisting}

Es sieht so aus als würden die Zeichen ``@'' und ``C'' sowie ``B'' und ``A'' vertauscht werden.

Hier ist noch ein interessantes Beispiel, in dem gezeigt wird, wie die Eigenschaften von XOR
ausgenutzt werden können: Exakt den gleichen Effekt, dass druckbare Zeichen auch druckbar bleiben,
kann man dadurch erzielen, dass irgendeine Kombination der niedrigsten vier Bits invertiert wird.
}

\EN{\section{Returning Values}
\label{ret_val_func}

Another simple function is the one that simply returns a constant value:

\lstinputlisting[caption=\EN{\CCpp Code},style=customc]{patterns/011_ret/1.c}

Let's compile it.

\subsection{x86}

Here's what both the GCC and MSVC compilers produce (with optimization) on the x86 platform:

\lstinputlisting[caption=\Optimizing GCC/MSVC (\assemblyOutput),style=customasmx86]{patterns/011_ret/1.s}

\myindex{x86!\Instructions!RET}
There are just two instructions: the first places the value 123 into the \EAX register,
which is used by convention for storing the return
value, and the second one is \RET, which returns execution to the \gls{caller}.

The caller will take the result from the \EAX register.

\subsection{ARM}

There are a few differences on the ARM platform:

\lstinputlisting[caption=\OptimizingKeilVI (\ARMMode) ASM Output,style=customasmARM]{patterns/011_ret/1_Keil_ARM_O3.s}

ARM uses the register \Reg{0} for returning the results of functions, so 123 is copied into \Reg{0}.

\myindex{ARM!\Instructions!MOV}
\myindex{x86!\Instructions!MOV}
It is worth noting that \MOV is a misleading name for the instruction in both the x86 and ARM \ac{ISA}s.

The data is not in fact \IT{moved}, but \IT{copied}.

\subsection{MIPS}

\label{MIPS_leaf_function_ex1}

The GCC assembly output below lists registers by number:

\lstinputlisting[caption=\Optimizing GCC 4.4.5 (\assemblyOutput),style=customasmMIPS]{patterns/011_ret/MIPS.s}

\dots while \IDA does it by their pseudo names:

\lstinputlisting[caption=\Optimizing GCC 4.4.5 (IDA),style=customasmMIPS]{patterns/011_ret/MIPS_IDA.lst}

The \$2 (or \$V0) register is used to store the function's return value.
\myindex{MIPS!\Pseudoinstructions!LI}
\INS{LI} stands for ``Load Immediate'' and is the MIPS equivalent to \MOV.

\myindex{MIPS!\Instructions!J}
The other instruction is the jump instruction (J or JR) which returns the execution flow to the \gls{caller}.

\myindex{MIPS!Branch delay slot}
You might be wondering why the positions of the load instruction (LI) and the jump instruction (J or JR) are swapped. This is due to a \ac{RISC} feature called ``branch delay slot''.

The reason this happens is a quirk in the architecture of some RISC \ac{ISA}s and isn't important for our
purposes---we must simply keep in mind that in MIPS, the instruction following a jump or branch instruction
is executed \IT{before} the jump/branch instruction itself.

As a consequence, branch instructions always swap places with the instruction executed immediately beforehand.


In practice, functions which merely return 1 (\IT{true}) or 0 (\IT{false}) are very frequent.

The smallest ever of the standard UNIX utilities, \IT{/bin/true} and \IT{/bin/false} return 0 and 1 respectively, as an exit code.
(Zero as an exit code usually means success, non-zero means error.)
}
\RU{\subsubsection{std::string}
\myindex{\Cpp!STL!std::string}
\label{std_string}

\myparagraph{Как устроена структура}

Многие строковые библиотеки \InSqBrackets{\CNotes 2.2} обеспечивают структуру содержащую ссылку 
на буфер собственно со строкой, переменная всегда содержащую длину строки 
(что очень удобно для массы функций \InSqBrackets{\CNotes 2.2.1}) и переменную содержащую текущий размер буфера.

Строка в буфере обыкновенно оканчивается нулем: это для того чтобы указатель на буфер можно было
передавать в функции требующие на вход обычную сишную \ac{ASCIIZ}-строку.

Стандарт \Cpp не описывает, как именно нужно реализовывать std::string,
но, как правило, они реализованы как описано выше, с небольшими дополнениями.

Строки в \Cpp это не класс (как, например, QString в Qt), а темплейт (basic\_string), 
это сделано для того чтобы поддерживать 
строки содержащие разного типа символы: как минимум \Tchar и \IT{wchar\_t}.

Так что, std::string это класс с базовым типом \Tchar.

А std::wstring это класс с базовым типом \IT{wchar\_t}.

\mysubparagraph{MSVC}

В реализации MSVC, вместо ссылки на буфер может содержаться сам буфер (если строка короче 16-и символов).

Это означает, что каждая короткая строка будет занимать в памяти по крайней мере $16 + 4 + 4 = 24$ 
байт для 32-битной среды либо $16 + 8 + 8 = 32$ 
байта в 64-битной, а если строка длиннее 16-и символов, то прибавьте еще длину самой строки.

\lstinputlisting[caption=пример для MSVC,style=customc]{\CURPATH/STL/string/MSVC_RU.cpp}

Собственно, из этого исходника почти всё ясно.

Несколько замечаний:

Если строка короче 16-и символов, 
то отдельный буфер для строки в \glslink{heap}{куче} выделяться не будет.

Это удобно потому что на практике, основная часть строк действительно короткие.
Вероятно, разработчики в Microsoft выбрали размер в 16 символов как разумный баланс.

Теперь очень важный момент в конце функции main(): мы не пользуемся методом c\_str(), тем не менее,
если это скомпилировать и запустить, то обе строки появятся в консоли!

Работает это вот почему.

В первом случае строка короче 16-и символов и в начале объекта std::string (его можно рассматривать
просто как структуру) расположен буфер с этой строкой.
\printf трактует указатель как указатель на массив символов оканчивающийся нулем и поэтому всё работает.

Вывод второй строки (длиннее 16-и символов) даже еще опаснее: это вообще типичная программистская ошибка 
(или опечатка), забыть дописать c\_str().
Это работает потому что в это время в начале структуры расположен указатель на буфер.
Это может надолго остаться незамеченным: до тех пока там не появится строка 
короче 16-и символов, тогда процесс упадет.

\mysubparagraph{GCC}

В реализации GCC в структуре есть еще одна переменная --- reference count.

Интересно, что указатель на экземпляр класса std::string в GCC указывает не на начало самой структуры, 
а на указатель на буфера.
В libstdc++-v3\textbackslash{}include\textbackslash{}bits\textbackslash{}basic\_string.h 
мы можем прочитать что это сделано для удобства отладки:

\begin{lstlisting}
   *  The reason you want _M_data pointing to the character %array and
   *  not the _Rep is so that the debugger can see the string
   *  contents. (Probably we should add a non-inline member to get
   *  the _Rep for the debugger to use, so users can check the actual
   *  string length.)
\end{lstlisting}

\href{http://go.yurichev.com/17085}{исходный код basic\_string.h}

В нашем примере мы учитываем это:

\lstinputlisting[caption=пример для GCC,style=customc]{\CURPATH/STL/string/GCC_RU.cpp}

Нужны еще небольшие хаки чтобы сымитировать типичную ошибку, которую мы уже видели выше, из-за
более ужесточенной проверки типов в GCC, тем не менее, printf() работает и здесь без c\_str().

\myparagraph{Чуть более сложный пример}

\lstinputlisting[style=customc]{\CURPATH/STL/string/3.cpp}

\lstinputlisting[caption=MSVC 2012,style=customasmx86]{\CURPATH/STL/string/3_MSVC_RU.asm}

Собственно, компилятор не конструирует строки статически: да в общем-то и как
это возможно, если буфер с ней нужно хранить в \glslink{heap}{куче}?

Вместо этого в сегменте данных хранятся обычные \ac{ASCIIZ}-строки, а позже, во время выполнения, 
при помощи метода \q{assign}, конструируются строки s1 и s2
.
При помощи \TT{operator+}, создается строка s3.

Обратите внимание на то что вызов метода c\_str() отсутствует,
потому что его код достаточно короткий и компилятор вставил его прямо здесь:
если строка короче 16-и байт, то в регистре EAX остается указатель на буфер,
а если длиннее, то из этого же места достается адрес на буфер расположенный в \glslink{heap}{куче}.

Далее следуют вызовы трех деструкторов, причем, они вызываются только если строка длиннее 16-и байт:
тогда нужно освободить буфера в \glslink{heap}{куче}.
В противном случае, так как все три объекта std::string хранятся в стеке,
они освобождаются автоматически после выхода из функции.

Следовательно, работа с короткими строками более быстрая из-за м\'{е}ньшего обращения к \glslink{heap}{куче}.

Код на GCC даже проще (из-за того, что в GCC, как мы уже видели, не реализована возможность хранить короткую
строку прямо в структуре):

% TODO1 comment each function meaning
\lstinputlisting[caption=GCC 4.8.1,style=customasmx86]{\CURPATH/STL/string/3_GCC_RU.s}

Можно заметить, что в деструкторы передается не указатель на объект,
а указатель на место за 12 байт (или 3 слова) перед ним, то есть, на настоящее начало структуры.

\myparagraph{std::string как глобальная переменная}
\label{sec:std_string_as_global_variable}

Опытные программисты на \Cpp знают, что глобальные переменные \ac{STL}-типов вполне можно объявлять.

Да, действительно:

\lstinputlisting[style=customc]{\CURPATH/STL/string/5.cpp}

Но как и где будет вызываться конструктор \TT{std::string}?

На самом деле, эта переменная будет инициализирована даже перед началом \main.

\lstinputlisting[caption=MSVC 2012: здесь конструируется глобальная переменная{,} а также регистрируется её деструктор,style=customasmx86]{\CURPATH/STL/string/5_MSVC_p2.asm}

\lstinputlisting[caption=MSVC 2012: здесь глобальная переменная используется в \main,style=customasmx86]{\CURPATH/STL/string/5_MSVC_p1.asm}

\lstinputlisting[caption=MSVC 2012: эта функция-деструктор вызывается перед выходом,style=customasmx86]{\CURPATH/STL/string/5_MSVC_p3.asm}

\myindex{\CStandardLibrary!atexit()}
В реальности, из \ac{CRT}, еще до вызова main(), вызывается специальная функция,
в которой перечислены все конструкторы подобных переменных.
Более того: при помощи atexit() регистрируется функция, которая будет вызвана в конце работы программы:
в этой функции компилятор собирает вызовы деструкторов всех подобных глобальных переменных.

GCC работает похожим образом:

\lstinputlisting[caption=GCC 4.8.1,style=customasmx86]{\CURPATH/STL/string/5_GCC.s}

Но он не выделяет отдельной функции в которой будут собраны деструкторы: 
каждый деструктор передается в atexit() по одному.

% TODO а если глобальная STL-переменная в другом модуле? надо проверить.

}

\EN{\section{Returning Values}
\label{ret_val_func}

Another simple function is the one that simply returns a constant value:

\lstinputlisting[caption=\EN{\CCpp Code},style=customc]{patterns/011_ret/1.c}

Let's compile it.

\subsection{x86}

Here's what both the GCC and MSVC compilers produce (with optimization) on the x86 platform:

\lstinputlisting[caption=\Optimizing GCC/MSVC (\assemblyOutput),style=customasmx86]{patterns/011_ret/1.s}

\myindex{x86!\Instructions!RET}
There are just two instructions: the first places the value 123 into the \EAX register,
which is used by convention for storing the return
value, and the second one is \RET, which returns execution to the \gls{caller}.

The caller will take the result from the \EAX register.

\subsection{ARM}

There are a few differences on the ARM platform:

\lstinputlisting[caption=\OptimizingKeilVI (\ARMMode) ASM Output,style=customasmARM]{patterns/011_ret/1_Keil_ARM_O3.s}

ARM uses the register \Reg{0} for returning the results of functions, so 123 is copied into \Reg{0}.

\myindex{ARM!\Instructions!MOV}
\myindex{x86!\Instructions!MOV}
It is worth noting that \MOV is a misleading name for the instruction in both the x86 and ARM \ac{ISA}s.

The data is not in fact \IT{moved}, but \IT{copied}.

\subsection{MIPS}

\label{MIPS_leaf_function_ex1}

The GCC assembly output below lists registers by number:

\lstinputlisting[caption=\Optimizing GCC 4.4.5 (\assemblyOutput),style=customasmMIPS]{patterns/011_ret/MIPS.s}

\dots while \IDA does it by their pseudo names:

\lstinputlisting[caption=\Optimizing GCC 4.4.5 (IDA),style=customasmMIPS]{patterns/011_ret/MIPS_IDA.lst}

The \$2 (or \$V0) register is used to store the function's return value.
\myindex{MIPS!\Pseudoinstructions!LI}
\INS{LI} stands for ``Load Immediate'' and is the MIPS equivalent to \MOV.

\myindex{MIPS!\Instructions!J}
The other instruction is the jump instruction (J or JR) which returns the execution flow to the \gls{caller}.

\myindex{MIPS!Branch delay slot}
You might be wondering why the positions of the load instruction (LI) and the jump instruction (J or JR) are swapped. This is due to a \ac{RISC} feature called ``branch delay slot''.

The reason this happens is a quirk in the architecture of some RISC \ac{ISA}s and isn't important for our
purposes---we must simply keep in mind that in MIPS, the instruction following a jump or branch instruction
is executed \IT{before} the jump/branch instruction itself.

As a consequence, branch instructions always swap places with the instruction executed immediately beforehand.


In practice, functions which merely return 1 (\IT{true}) or 0 (\IT{false}) are very frequent.

The smallest ever of the standard UNIX utilities, \IT{/bin/true} and \IT{/bin/false} return 0 and 1 respectively, as an exit code.
(Zero as an exit code usually means success, non-zero means error.)
}
\RU{\subsubsection{std::string}
\myindex{\Cpp!STL!std::string}
\label{std_string}

\myparagraph{Как устроена структура}

Многие строковые библиотеки \InSqBrackets{\CNotes 2.2} обеспечивают структуру содержащую ссылку 
на буфер собственно со строкой, переменная всегда содержащую длину строки 
(что очень удобно для массы функций \InSqBrackets{\CNotes 2.2.1}) и переменную содержащую текущий размер буфера.

Строка в буфере обыкновенно оканчивается нулем: это для того чтобы указатель на буфер можно было
передавать в функции требующие на вход обычную сишную \ac{ASCIIZ}-строку.

Стандарт \Cpp не описывает, как именно нужно реализовывать std::string,
но, как правило, они реализованы как описано выше, с небольшими дополнениями.

Строки в \Cpp это не класс (как, например, QString в Qt), а темплейт (basic\_string), 
это сделано для того чтобы поддерживать 
строки содержащие разного типа символы: как минимум \Tchar и \IT{wchar\_t}.

Так что, std::string это класс с базовым типом \Tchar.

А std::wstring это класс с базовым типом \IT{wchar\_t}.

\mysubparagraph{MSVC}

В реализации MSVC, вместо ссылки на буфер может содержаться сам буфер (если строка короче 16-и символов).

Это означает, что каждая короткая строка будет занимать в памяти по крайней мере $16 + 4 + 4 = 24$ 
байт для 32-битной среды либо $16 + 8 + 8 = 32$ 
байта в 64-битной, а если строка длиннее 16-и символов, то прибавьте еще длину самой строки.

\lstinputlisting[caption=пример для MSVC,style=customc]{\CURPATH/STL/string/MSVC_RU.cpp}

Собственно, из этого исходника почти всё ясно.

Несколько замечаний:

Если строка короче 16-и символов, 
то отдельный буфер для строки в \glslink{heap}{куче} выделяться не будет.

Это удобно потому что на практике, основная часть строк действительно короткие.
Вероятно, разработчики в Microsoft выбрали размер в 16 символов как разумный баланс.

Теперь очень важный момент в конце функции main(): мы не пользуемся методом c\_str(), тем не менее,
если это скомпилировать и запустить, то обе строки появятся в консоли!

Работает это вот почему.

В первом случае строка короче 16-и символов и в начале объекта std::string (его можно рассматривать
просто как структуру) расположен буфер с этой строкой.
\printf трактует указатель как указатель на массив символов оканчивающийся нулем и поэтому всё работает.

Вывод второй строки (длиннее 16-и символов) даже еще опаснее: это вообще типичная программистская ошибка 
(или опечатка), забыть дописать c\_str().
Это работает потому что в это время в начале структуры расположен указатель на буфер.
Это может надолго остаться незамеченным: до тех пока там не появится строка 
короче 16-и символов, тогда процесс упадет.

\mysubparagraph{GCC}

В реализации GCC в структуре есть еще одна переменная --- reference count.

Интересно, что указатель на экземпляр класса std::string в GCC указывает не на начало самой структуры, 
а на указатель на буфера.
В libstdc++-v3\textbackslash{}include\textbackslash{}bits\textbackslash{}basic\_string.h 
мы можем прочитать что это сделано для удобства отладки:

\begin{lstlisting}
   *  The reason you want _M_data pointing to the character %array and
   *  not the _Rep is so that the debugger can see the string
   *  contents. (Probably we should add a non-inline member to get
   *  the _Rep for the debugger to use, so users can check the actual
   *  string length.)
\end{lstlisting}

\href{http://go.yurichev.com/17085}{исходный код basic\_string.h}

В нашем примере мы учитываем это:

\lstinputlisting[caption=пример для GCC,style=customc]{\CURPATH/STL/string/GCC_RU.cpp}

Нужны еще небольшие хаки чтобы сымитировать типичную ошибку, которую мы уже видели выше, из-за
более ужесточенной проверки типов в GCC, тем не менее, printf() работает и здесь без c\_str().

\myparagraph{Чуть более сложный пример}

\lstinputlisting[style=customc]{\CURPATH/STL/string/3.cpp}

\lstinputlisting[caption=MSVC 2012,style=customasmx86]{\CURPATH/STL/string/3_MSVC_RU.asm}

Собственно, компилятор не конструирует строки статически: да в общем-то и как
это возможно, если буфер с ней нужно хранить в \glslink{heap}{куче}?

Вместо этого в сегменте данных хранятся обычные \ac{ASCIIZ}-строки, а позже, во время выполнения, 
при помощи метода \q{assign}, конструируются строки s1 и s2
.
При помощи \TT{operator+}, создается строка s3.

Обратите внимание на то что вызов метода c\_str() отсутствует,
потому что его код достаточно короткий и компилятор вставил его прямо здесь:
если строка короче 16-и байт, то в регистре EAX остается указатель на буфер,
а если длиннее, то из этого же места достается адрес на буфер расположенный в \glslink{heap}{куче}.

Далее следуют вызовы трех деструкторов, причем, они вызываются только если строка длиннее 16-и байт:
тогда нужно освободить буфера в \glslink{heap}{куче}.
В противном случае, так как все три объекта std::string хранятся в стеке,
они освобождаются автоматически после выхода из функции.

Следовательно, работа с короткими строками более быстрая из-за м\'{е}ньшего обращения к \glslink{heap}{куче}.

Код на GCC даже проще (из-за того, что в GCC, как мы уже видели, не реализована возможность хранить короткую
строку прямо в структуре):

% TODO1 comment each function meaning
\lstinputlisting[caption=GCC 4.8.1,style=customasmx86]{\CURPATH/STL/string/3_GCC_RU.s}

Можно заметить, что в деструкторы передается не указатель на объект,
а указатель на место за 12 байт (или 3 слова) перед ним, то есть, на настоящее начало структуры.

\myparagraph{std::string как глобальная переменная}
\label{sec:std_string_as_global_variable}

Опытные программисты на \Cpp знают, что глобальные переменные \ac{STL}-типов вполне можно объявлять.

Да, действительно:

\lstinputlisting[style=customc]{\CURPATH/STL/string/5.cpp}

Но как и где будет вызываться конструктор \TT{std::string}?

На самом деле, эта переменная будет инициализирована даже перед началом \main.

\lstinputlisting[caption=MSVC 2012: здесь конструируется глобальная переменная{,} а также регистрируется её деструктор,style=customasmx86]{\CURPATH/STL/string/5_MSVC_p2.asm}

\lstinputlisting[caption=MSVC 2012: здесь глобальная переменная используется в \main,style=customasmx86]{\CURPATH/STL/string/5_MSVC_p1.asm}

\lstinputlisting[caption=MSVC 2012: эта функция-деструктор вызывается перед выходом,style=customasmx86]{\CURPATH/STL/string/5_MSVC_p3.asm}

\myindex{\CStandardLibrary!atexit()}
В реальности, из \ac{CRT}, еще до вызова main(), вызывается специальная функция,
в которой перечислены все конструкторы подобных переменных.
Более того: при помощи atexit() регистрируется функция, которая будет вызвана в конце работы программы:
в этой функции компилятор собирает вызовы деструкторов всех подобных глобальных переменных.

GCC работает похожим образом:

\lstinputlisting[caption=GCC 4.8.1,style=customasmx86]{\CURPATH/STL/string/5_GCC.s}

Но он не выделяет отдельной функции в которой будут собраны деструкторы: 
каждый деструктор передается в atexit() по одному.

% TODO а если глобальная STL-переменная в другом модуле? надо проверить.

}
\DE{\subsection{Einfachste XOR-Verschlüsselung überhaupt}

Ich habe einmal eine Software gesehen, bei der alle Debugging-Ausgaben mit XOR mit dem Wert 3
verschlüsselt wurden. Mit anderen Worten, die beiden niedrigsten Bits aller Buchstaben wurden invertiert.

``Hello, world'' wurde zu ``Kfool/\#tlqog'':

\begin{lstlisting}
#!/usr/bin/python

msg="Hello, world!"

print "".join(map(lambda x: chr(ord(x)^3), msg))
\end{lstlisting}

Das ist eine ziemlich interessante Verschlüsselung (oder besser eine Verschleierung),
weil sie zwei wichtige Eigenschaften hat:
1) es ist eine einzige Funktion zum Verschlüsseln und entschlüsseln, sie muss nur wiederholt angewendet werden
2) die entstehenden Buchstaben befinden sich im druckbaren Bereich, also die ganze Zeichenkette kann ohne
Escape-Symbole im Code verwendet werden.

Die zweite Eigenschaft nutzt die Tatsache, dass alle druckbaren Zeichen in Reihen organisiert sind: 0x2x-0x7x,
und wenn die beiden niederwertigsten Bits invertiert werden, wird der Buchstabe um eine oder drei Stellen nach
links oder rechts \IT{verschoben}, aber niemals in eine andere Reihe:

\begin{figure}[H]
\centering
\includegraphics[width=0.7\textwidth]{ascii_clean.png}
\caption{7-Bit \ac{ASCII} Tabelle in Emacs}
\end{figure}

\dots mit dem Zeichen 0x7F als einziger Ausnahme.

Im Folgenden werden also beispielsweise die Zeichen A-Z \IT{verschlüsselt}:

\begin{lstlisting}
#!/usr/bin/python

msg="@ABCDEFGHIJKLMNO"

print "".join(map(lambda x: chr(ord(x)^3), msg))
\end{lstlisting}

Ergebnis:
% FIXME \verb  --  relevant comment for German?
\begin{lstlisting}
CBA@GFEDKJIHONML
\end{lstlisting}

Es sieht so aus als würden die Zeichen ``@'' und ``C'' sowie ``B'' und ``A'' vertauscht werden.

Hier ist noch ein interessantes Beispiel, in dem gezeigt wird, wie die Eigenschaften von XOR
ausgenutzt werden können: Exakt den gleichen Effekt, dass druckbare Zeichen auch druckbar bleiben,
kann man dadurch erzielen, dass irgendeine Kombination der niedrigsten vier Bits invertiert wird.
}

\EN{\section{Returning Values}
\label{ret_val_func}

Another simple function is the one that simply returns a constant value:

\lstinputlisting[caption=\EN{\CCpp Code},style=customc]{patterns/011_ret/1.c}

Let's compile it.

\subsection{x86}

Here's what both the GCC and MSVC compilers produce (with optimization) on the x86 platform:

\lstinputlisting[caption=\Optimizing GCC/MSVC (\assemblyOutput),style=customasmx86]{patterns/011_ret/1.s}

\myindex{x86!\Instructions!RET}
There are just two instructions: the first places the value 123 into the \EAX register,
which is used by convention for storing the return
value, and the second one is \RET, which returns execution to the \gls{caller}.

The caller will take the result from the \EAX register.

\subsection{ARM}

There are a few differences on the ARM platform:

\lstinputlisting[caption=\OptimizingKeilVI (\ARMMode) ASM Output,style=customasmARM]{patterns/011_ret/1_Keil_ARM_O3.s}

ARM uses the register \Reg{0} for returning the results of functions, so 123 is copied into \Reg{0}.

\myindex{ARM!\Instructions!MOV}
\myindex{x86!\Instructions!MOV}
It is worth noting that \MOV is a misleading name for the instruction in both the x86 and ARM \ac{ISA}s.

The data is not in fact \IT{moved}, but \IT{copied}.

\subsection{MIPS}

\label{MIPS_leaf_function_ex1}

The GCC assembly output below lists registers by number:

\lstinputlisting[caption=\Optimizing GCC 4.4.5 (\assemblyOutput),style=customasmMIPS]{patterns/011_ret/MIPS.s}

\dots while \IDA does it by their pseudo names:

\lstinputlisting[caption=\Optimizing GCC 4.4.5 (IDA),style=customasmMIPS]{patterns/011_ret/MIPS_IDA.lst}

The \$2 (or \$V0) register is used to store the function's return value.
\myindex{MIPS!\Pseudoinstructions!LI}
\INS{LI} stands for ``Load Immediate'' and is the MIPS equivalent to \MOV.

\myindex{MIPS!\Instructions!J}
The other instruction is the jump instruction (J or JR) which returns the execution flow to the \gls{caller}.

\myindex{MIPS!Branch delay slot}
You might be wondering why the positions of the load instruction (LI) and the jump instruction (J or JR) are swapped. This is due to a \ac{RISC} feature called ``branch delay slot''.

The reason this happens is a quirk in the architecture of some RISC \ac{ISA}s and isn't important for our
purposes---we must simply keep in mind that in MIPS, the instruction following a jump or branch instruction
is executed \IT{before} the jump/branch instruction itself.

As a consequence, branch instructions always swap places with the instruction executed immediately beforehand.


In practice, functions which merely return 1 (\IT{true}) or 0 (\IT{false}) are very frequent.

The smallest ever of the standard UNIX utilities, \IT{/bin/true} and \IT{/bin/false} return 0 and 1 respectively, as an exit code.
(Zero as an exit code usually means success, non-zero means error.)
}
\RU{\subsubsection{std::string}
\myindex{\Cpp!STL!std::string}
\label{std_string}

\myparagraph{Как устроена структура}

Многие строковые библиотеки \InSqBrackets{\CNotes 2.2} обеспечивают структуру содержащую ссылку 
на буфер собственно со строкой, переменная всегда содержащую длину строки 
(что очень удобно для массы функций \InSqBrackets{\CNotes 2.2.1}) и переменную содержащую текущий размер буфера.

Строка в буфере обыкновенно оканчивается нулем: это для того чтобы указатель на буфер можно было
передавать в функции требующие на вход обычную сишную \ac{ASCIIZ}-строку.

Стандарт \Cpp не описывает, как именно нужно реализовывать std::string,
но, как правило, они реализованы как описано выше, с небольшими дополнениями.

Строки в \Cpp это не класс (как, например, QString в Qt), а темплейт (basic\_string), 
это сделано для того чтобы поддерживать 
строки содержащие разного типа символы: как минимум \Tchar и \IT{wchar\_t}.

Так что, std::string это класс с базовым типом \Tchar.

А std::wstring это класс с базовым типом \IT{wchar\_t}.

\mysubparagraph{MSVC}

В реализации MSVC, вместо ссылки на буфер может содержаться сам буфер (если строка короче 16-и символов).

Это означает, что каждая короткая строка будет занимать в памяти по крайней мере $16 + 4 + 4 = 24$ 
байт для 32-битной среды либо $16 + 8 + 8 = 32$ 
байта в 64-битной, а если строка длиннее 16-и символов, то прибавьте еще длину самой строки.

\lstinputlisting[caption=пример для MSVC,style=customc]{\CURPATH/STL/string/MSVC_RU.cpp}

Собственно, из этого исходника почти всё ясно.

Несколько замечаний:

Если строка короче 16-и символов, 
то отдельный буфер для строки в \glslink{heap}{куче} выделяться не будет.

Это удобно потому что на практике, основная часть строк действительно короткие.
Вероятно, разработчики в Microsoft выбрали размер в 16 символов как разумный баланс.

Теперь очень важный момент в конце функции main(): мы не пользуемся методом c\_str(), тем не менее,
если это скомпилировать и запустить, то обе строки появятся в консоли!

Работает это вот почему.

В первом случае строка короче 16-и символов и в начале объекта std::string (его можно рассматривать
просто как структуру) расположен буфер с этой строкой.
\printf трактует указатель как указатель на массив символов оканчивающийся нулем и поэтому всё работает.

Вывод второй строки (длиннее 16-и символов) даже еще опаснее: это вообще типичная программистская ошибка 
(или опечатка), забыть дописать c\_str().
Это работает потому что в это время в начале структуры расположен указатель на буфер.
Это может надолго остаться незамеченным: до тех пока там не появится строка 
короче 16-и символов, тогда процесс упадет.

\mysubparagraph{GCC}

В реализации GCC в структуре есть еще одна переменная --- reference count.

Интересно, что указатель на экземпляр класса std::string в GCC указывает не на начало самой структуры, 
а на указатель на буфера.
В libstdc++-v3\textbackslash{}include\textbackslash{}bits\textbackslash{}basic\_string.h 
мы можем прочитать что это сделано для удобства отладки:

\begin{lstlisting}
   *  The reason you want _M_data pointing to the character %array and
   *  not the _Rep is so that the debugger can see the string
   *  contents. (Probably we should add a non-inline member to get
   *  the _Rep for the debugger to use, so users can check the actual
   *  string length.)
\end{lstlisting}

\href{http://go.yurichev.com/17085}{исходный код basic\_string.h}

В нашем примере мы учитываем это:

\lstinputlisting[caption=пример для GCC,style=customc]{\CURPATH/STL/string/GCC_RU.cpp}

Нужны еще небольшие хаки чтобы сымитировать типичную ошибку, которую мы уже видели выше, из-за
более ужесточенной проверки типов в GCC, тем не менее, printf() работает и здесь без c\_str().

\myparagraph{Чуть более сложный пример}

\lstinputlisting[style=customc]{\CURPATH/STL/string/3.cpp}

\lstinputlisting[caption=MSVC 2012,style=customasmx86]{\CURPATH/STL/string/3_MSVC_RU.asm}

Собственно, компилятор не конструирует строки статически: да в общем-то и как
это возможно, если буфер с ней нужно хранить в \glslink{heap}{куче}?

Вместо этого в сегменте данных хранятся обычные \ac{ASCIIZ}-строки, а позже, во время выполнения, 
при помощи метода \q{assign}, конструируются строки s1 и s2
.
При помощи \TT{operator+}, создается строка s3.

Обратите внимание на то что вызов метода c\_str() отсутствует,
потому что его код достаточно короткий и компилятор вставил его прямо здесь:
если строка короче 16-и байт, то в регистре EAX остается указатель на буфер,
а если длиннее, то из этого же места достается адрес на буфер расположенный в \glslink{heap}{куче}.

Далее следуют вызовы трех деструкторов, причем, они вызываются только если строка длиннее 16-и байт:
тогда нужно освободить буфера в \glslink{heap}{куче}.
В противном случае, так как все три объекта std::string хранятся в стеке,
они освобождаются автоматически после выхода из функции.

Следовательно, работа с короткими строками более быстрая из-за м\'{е}ньшего обращения к \glslink{heap}{куче}.

Код на GCC даже проще (из-за того, что в GCC, как мы уже видели, не реализована возможность хранить короткую
строку прямо в структуре):

% TODO1 comment each function meaning
\lstinputlisting[caption=GCC 4.8.1,style=customasmx86]{\CURPATH/STL/string/3_GCC_RU.s}

Можно заметить, что в деструкторы передается не указатель на объект,
а указатель на место за 12 байт (или 3 слова) перед ним, то есть, на настоящее начало структуры.

\myparagraph{std::string как глобальная переменная}
\label{sec:std_string_as_global_variable}

Опытные программисты на \Cpp знают, что глобальные переменные \ac{STL}-типов вполне можно объявлять.

Да, действительно:

\lstinputlisting[style=customc]{\CURPATH/STL/string/5.cpp}

Но как и где будет вызываться конструктор \TT{std::string}?

На самом деле, эта переменная будет инициализирована даже перед началом \main.

\lstinputlisting[caption=MSVC 2012: здесь конструируется глобальная переменная{,} а также регистрируется её деструктор,style=customasmx86]{\CURPATH/STL/string/5_MSVC_p2.asm}

\lstinputlisting[caption=MSVC 2012: здесь глобальная переменная используется в \main,style=customasmx86]{\CURPATH/STL/string/5_MSVC_p1.asm}

\lstinputlisting[caption=MSVC 2012: эта функция-деструктор вызывается перед выходом,style=customasmx86]{\CURPATH/STL/string/5_MSVC_p3.asm}

\myindex{\CStandardLibrary!atexit()}
В реальности, из \ac{CRT}, еще до вызова main(), вызывается специальная функция,
в которой перечислены все конструкторы подобных переменных.
Более того: при помощи atexit() регистрируется функция, которая будет вызвана в конце работы программы:
в этой функции компилятор собирает вызовы деструкторов всех подобных глобальных переменных.

GCC работает похожим образом:

\lstinputlisting[caption=GCC 4.8.1,style=customasmx86]{\CURPATH/STL/string/5_GCC.s}

Но он не выделяет отдельной функции в которой будут собраны деструкторы: 
каждый деструктор передается в atexit() по одному.

% TODO а если глобальная STL-переменная в другом модуле? надо проверить.

}
\DE{\subsection{Einfachste XOR-Verschlüsselung überhaupt}

Ich habe einmal eine Software gesehen, bei der alle Debugging-Ausgaben mit XOR mit dem Wert 3
verschlüsselt wurden. Mit anderen Worten, die beiden niedrigsten Bits aller Buchstaben wurden invertiert.

``Hello, world'' wurde zu ``Kfool/\#tlqog'':

\begin{lstlisting}
#!/usr/bin/python

msg="Hello, world!"

print "".join(map(lambda x: chr(ord(x)^3), msg))
\end{lstlisting}

Das ist eine ziemlich interessante Verschlüsselung (oder besser eine Verschleierung),
weil sie zwei wichtige Eigenschaften hat:
1) es ist eine einzige Funktion zum Verschlüsseln und entschlüsseln, sie muss nur wiederholt angewendet werden
2) die entstehenden Buchstaben befinden sich im druckbaren Bereich, also die ganze Zeichenkette kann ohne
Escape-Symbole im Code verwendet werden.

Die zweite Eigenschaft nutzt die Tatsache, dass alle druckbaren Zeichen in Reihen organisiert sind: 0x2x-0x7x,
und wenn die beiden niederwertigsten Bits invertiert werden, wird der Buchstabe um eine oder drei Stellen nach
links oder rechts \IT{verschoben}, aber niemals in eine andere Reihe:

\begin{figure}[H]
\centering
\includegraphics[width=0.7\textwidth]{ascii_clean.png}
\caption{7-Bit \ac{ASCII} Tabelle in Emacs}
\end{figure}

\dots mit dem Zeichen 0x7F als einziger Ausnahme.

Im Folgenden werden also beispielsweise die Zeichen A-Z \IT{verschlüsselt}:

\begin{lstlisting}
#!/usr/bin/python

msg="@ABCDEFGHIJKLMNO"

print "".join(map(lambda x: chr(ord(x)^3), msg))
\end{lstlisting}

Ergebnis:
% FIXME \verb  --  relevant comment for German?
\begin{lstlisting}
CBA@GFEDKJIHONML
\end{lstlisting}

Es sieht so aus als würden die Zeichen ``@'' und ``C'' sowie ``B'' und ``A'' vertauscht werden.

Hier ist noch ein interessantes Beispiel, in dem gezeigt wird, wie die Eigenschaften von XOR
ausgenutzt werden können: Exakt den gleichen Effekt, dass druckbare Zeichen auch druckbar bleiben,
kann man dadurch erzielen, dass irgendeine Kombination der niedrigsten vier Bits invertiert wird.
}

\ifdefined\SPANISH
\chapter{Patrones de código}
\fi % SPANISH

\ifdefined\GERMAN
\chapter{Code-Muster}
\fi % GERMAN

\ifdefined\ENGLISH
\chapter{Code Patterns}
\fi % ENGLISH

\ifdefined\ITALIAN
\chapter{Forme di codice}
\fi % ITALIAN

\ifdefined\RUSSIAN
\chapter{Образцы кода}
\fi % RUSSIAN

\ifdefined\BRAZILIAN
\chapter{Padrões de códigos}
\fi % BRAZILIAN

\ifdefined\THAI
\chapter{รูปแบบของโค้ด}
\fi % THAI

\ifdefined\FRENCH
\chapter{Modèle de code}
\fi % FRENCH

\ifdefined\POLISH
\chapter{\PLph{}}
\fi % POLISH

% sections
\EN{\input{patterns/patterns_opt_dbg_EN}}
\ES{\input{patterns/patterns_opt_dbg_ES}}
\ITA{\input{patterns/patterns_opt_dbg_ITA}}
\PTBR{\input{patterns/patterns_opt_dbg_PTBR}}
\RU{\input{patterns/patterns_opt_dbg_RU}}
\THA{\input{patterns/patterns_opt_dbg_THA}}
\DE{\input{patterns/patterns_opt_dbg_DE}}
\FR{\input{patterns/patterns_opt_dbg_FR}}
\PL{\input{patterns/patterns_opt_dbg_PL}}

\RU{\section{Некоторые базовые понятия}}
\EN{\section{Some basics}}
\DE{\section{Einige Grundlagen}}
\FR{\section{Quelques bases}}
\ES{\section{\ESph{}}}
\ITA{\section{Alcune basi teoriche}}
\PTBR{\section{\PTBRph{}}}
\THA{\section{\THAph{}}}
\PL{\section{\PLph{}}}

% sections:
\EN{\input{patterns/intro_CPU_ISA_EN}}
\ES{\input{patterns/intro_CPU_ISA_ES}}
\ITA{\input{patterns/intro_CPU_ISA_ITA}}
\PTBR{\input{patterns/intro_CPU_ISA_PTBR}}
\RU{\input{patterns/intro_CPU_ISA_RU}}
\DE{\input{patterns/intro_CPU_ISA_DE}}
\FR{\input{patterns/intro_CPU_ISA_FR}}
\PL{\input{patterns/intro_CPU_ISA_PL}}

\EN{\input{patterns/numeral_EN}}
\RU{\input{patterns/numeral_RU}}
\ITA{\input{patterns/numeral_ITA}}
\DE{\input{patterns/numeral_DE}}
\FR{\input{patterns/numeral_FR}}
\PL{\input{patterns/numeral_PL}}

% chapters
\input{patterns/00_empty/main}
\input{patterns/011_ret/main}
\input{patterns/01_helloworld/main}
\input{patterns/015_prolog_epilogue/main}
\input{patterns/02_stack/main}
\input{patterns/03_printf/main}
\input{patterns/04_scanf/main}
\input{patterns/05_passing_arguments/main}
\input{patterns/06_return_results/main}
\input{patterns/061_pointers/main}
\input{patterns/065_GOTO/main}
\input{patterns/07_jcc/main}
\input{patterns/08_switch/main}
\input{patterns/09_loops/main}
\input{patterns/10_strings/main}
\input{patterns/11_arith_optimizations/main}
\input{patterns/12_FPU/main}
\input{patterns/13_arrays/main}
\input{patterns/14_bitfields/main}
\EN{\input{patterns/145_LCG/main_EN}}
\RU{\input{patterns/145_LCG/main_RU}}
\input{patterns/15_structs/main}
\input{patterns/17_unions/main}
\input{patterns/18_pointers_to_functions/main}
\input{patterns/185_64bit_in_32_env/main}

\EN{\input{patterns/19_SIMD/main_EN}}
\RU{\input{patterns/19_SIMD/main_RU}}
\DE{\input{patterns/19_SIMD/main_DE}}

\EN{\input{patterns/20_x64/main_EN}}
\RU{\input{patterns/20_x64/main_RU}}

\EN{\input{patterns/205_floating_SIMD/main_EN}}
\RU{\input{patterns/205_floating_SIMD/main_RU}}
\DE{\input{patterns/205_floating_SIMD/main_DE}}

\EN{\input{patterns/ARM/main_EN}}
\RU{\input{patterns/ARM/main_RU}}
\DE{\input{patterns/ARM/main_DE}}

\input{patterns/MIPS/main}


\ifdefined\SPANISH
\chapter{Patrones de código}
\fi % SPANISH

\ifdefined\GERMAN
\chapter{Code-Muster}
\fi % GERMAN

\ifdefined\ENGLISH
\chapter{Code Patterns}
\fi % ENGLISH

\ifdefined\ITALIAN
\chapter{Forme di codice}
\fi % ITALIAN

\ifdefined\RUSSIAN
\chapter{Образцы кода}
\fi % RUSSIAN

\ifdefined\BRAZILIAN
\chapter{Padrões de códigos}
\fi % BRAZILIAN

\ifdefined\THAI
\chapter{รูปแบบของโค้ด}
\fi % THAI

\ifdefined\FRENCH
\chapter{Modèle de code}
\fi % FRENCH

\ifdefined\POLISH
\chapter{\PLph{}}
\fi % POLISH

% sections
\EN{\section{The method}

When the author of this book first started learning C and, later, \Cpp, he used to write small pieces of code, compile them,
and then look at the assembly language output. This made it very easy for him to understand what was going on in the code that he had written.
\footnote{In fact, he still does this when he can't understand what a particular bit of code does.}.
He did this so many times that the relationship between the \CCpp code and what the compiler produced was imprinted deeply in his mind.
It's now easy for him to imagine instantly a rough outline of a C code's appearance and function.
Perhaps this technique could be helpful for others.

%There are a lot of examples for both x86/x64 and ARM.
%Those who already familiar with one of architectures, may freely skim over pages.

By the way, there is a great website where you can do the same, with various compilers, instead of installing them on your box.
You can use it as well: \url{https://gcc.godbolt.org/}.

\section*{\Exercises}

When the author of this book studied assembly language, he also often compiled small C functions and then rewrote
them gradually to assembly, trying to make their code as short as possible.
This probably is not worth doing in real-world scenarios today,
because it's hard to compete with the latest compilers in terms of efficiency. It is, however, a very good way to gain a better understanding of assembly.
Feel free, therefore, to take any assembly code from this book and try to make it shorter.
However, don't forget to test what you have written.

% rewrote to show that debug\release and optimisations levels are orthogonal concepts.
\section*{Optimization levels and debug information}

Source code can be compiled by different compilers with various optimization levels.
A typical compiler has about three such levels, where level zero means that optimization is completely disabled.
Optimization can also be targeted towards code size or code speed.
A non-optimizing compiler is faster and produces more understandable (albeit verbose) code,
whereas an optimizing compiler is slower and tries to produce code that runs faster (but is not necessarily more compact).
In addition to optimization levels, a compiler can include some debug information in the resulting file,
producing code that is easy to debug.
One of the important features of the ´debug' code is that it might contain links
between each line of the source code and its respective machine code address.
Optimizing compilers, on the other hand, tend to produce output where entire lines of source code
can be optimized away and thus not even be present in the resulting machine code.
Reverse engineers can encounter either version, simply because some developers turn on the compiler's optimization flags and others do not.
Because of this, we'll try to work on examples of both debug and release versions of the code featured in this book, wherever possible.

Sometimes some pretty ancient compilers are used in this book, in order to get the shortest (or simplest) possible code snippet.
}
\ES{\input{patterns/patterns_opt_dbg_ES}}
\ITA{\input{patterns/patterns_opt_dbg_ITA}}
\PTBR{\input{patterns/patterns_opt_dbg_PTBR}}
\RU{\input{patterns/patterns_opt_dbg_RU}}
\THA{\input{patterns/patterns_opt_dbg_THA}}
\DE{\input{patterns/patterns_opt_dbg_DE}}
\FR{\input{patterns/patterns_opt_dbg_FR}}
\PL{\input{patterns/patterns_opt_dbg_PL}}

\RU{\section{Некоторые базовые понятия}}
\EN{\section{Some basics}}
\DE{\section{Einige Grundlagen}}
\FR{\section{Quelques bases}}
\ES{\section{\ESph{}}}
\ITA{\section{Alcune basi teoriche}}
\PTBR{\section{\PTBRph{}}}
\THA{\section{\THAph{}}}
\PL{\section{\PLph{}}}

% sections:
\EN{\input{patterns/intro_CPU_ISA_EN}}
\ES{\input{patterns/intro_CPU_ISA_ES}}
\ITA{\input{patterns/intro_CPU_ISA_ITA}}
\PTBR{\input{patterns/intro_CPU_ISA_PTBR}}
\RU{\input{patterns/intro_CPU_ISA_RU}}
\DE{\input{patterns/intro_CPU_ISA_DE}}
\FR{\input{patterns/intro_CPU_ISA_FR}}
\PL{\input{patterns/intro_CPU_ISA_PL}}

\EN{\subsection{Numeral Systems}

Humans have become accustomed to a decimal numeral system, probably because almost everyone has 10 fingers.
Nevertheless, the number \q{10} has no significant meaning in science and mathematics.
The natural numeral system in digital electronics is binary: 0 is for an absence of current in the wire, and 1 for presence.
10 in binary is 2 in decimal, 100 in binary is 4 in decimal, and so on.

% This sentence is a bit unweildy - maybe try 'Our ten-digit system would be described as having a radix...' - Renaissance
If the numeral system has 10 digits, it has a \IT{radix} (or \IT{base}) of 10.
The binary numeral system has a \IT{radix} of 2.

Important things to recall:

1) A \IT{number} is a number, while a \IT{digit} is a term from writing systems, and is usually one character

% The original is 'number' is not changed; I think the intent is value, and changed it - Renaissance
2) The value of a number does not change when converted to another radix; only the writing notation for that value has changed (and therefore the way of representing it in \ac{RAM}).

\subsection{Converting From One Radix To Another}

Positional notation is used almost every numerical system. This means that a digit has weight relative to where it is placed inside of the larger number.
If 2 is placed at the rightmost place, it's 2, but if it's placed one digit before rightmost, it's 20.

What does $1234$ stand for?

$10^3 \cdot 1 + 10^2 \cdot 2 + 10^1 \cdot 3 + 1 \cdot 4 = 1234$ or
$1000 \cdot 1 + 100 \cdot 2 + 10 \cdot 3 + 4 = 1234$

It's the same story for binary numbers, but the base is 2 instead of 10.
What does 0b101011 stand for?

$2^5 \cdot 1 + 2^4 \cdot 0 + 2^3 \cdot 1 + 2^2 \cdot 0 + 2^1 \cdot 1 + 2^0 \cdot 1 = 43$ or
$32 \cdot 1 + 16 \cdot 0 + 8 \cdot 1 + 4 \cdot 0 + 2 \cdot 1 + 1 = 43$

There is such a thing as non-positional notation, such as the Roman numeral system.
\footnote{About numeric system evolution, see \InSqBrackets{\TAOCPvolII{}, 195--213.}}.
% Maybe add a sentence to fill in that X is always 10, and is therefore non-positional, even though putting an I before subtracts and after adds, and is in that sense positional
Perhaps, humankind switched to positional notation because it's easier to do basic operations (addition, multiplication, etc.) on paper by hand.

Binary numbers can be added, subtracted and so on in the very same as taught in schools, but only 2 digits are available.

Binary numbers are bulky when represented in source code and dumps, so that is where the hexadecimal numeral system can be useful.
A hexadecimal radix uses the digits 0..9, and also 6 Latin characters: A..F.
Each hexadecimal digit takes 4 bits or 4 binary digits, so it's very easy to convert from binary number to hexadecimal and back, even manually, in one's mind.

\begin{center}
\begin{longtable}{ | l | l | l | }
\hline
\HeaderColor hexadecimal & \HeaderColor binary & \HeaderColor decimal \\
\hline
0	&0000	&0 \\
1	&0001	&1 \\
2	&0010	&2 \\
3	&0011	&3 \\
4	&0100	&4 \\
5	&0101	&5 \\
6	&0110	&6 \\
7	&0111	&7 \\
8	&1000	&8 \\
9	&1001	&9 \\
A	&1010	&10 \\
B	&1011	&11 \\
C	&1100	&12 \\
D	&1101	&13 \\
E	&1110	&14 \\
F	&1111	&15 \\
\hline
\end{longtable}
\end{center}

How can one tell which radix is being used in a specific instance?

Decimal numbers are usually written as is, i.e., 1234. Some assemblers allow an identifier on decimal radix numbers, in which the number would be written with a "d" suffix: 1234d.

Binary numbers are sometimes prepended with the "0b" prefix: 0b100110111 (\ac{GCC} has a non-standard language extension for this\footnote{\url{https://gcc.gnu.org/onlinedocs/gcc/Binary-constants.html}}).
There is also another way: using a "b" suffix, for example: 100110111b.
This book tries to use the "0b" prefix consistently throughout the book for binary numbers.

Hexadecimal numbers are prepended with "0x" prefix in \CCpp and other \ac{PL}s: 0x1234ABCD.
Alternatively, they are given a "h" suffix: 1234ABCDh. This is common way of representing them in assemblers and debuggers.
In this convention, if the number is started with a Latin (A..F) digit, a 0 is added at the beginning: 0ABCDEFh.
There was also convention that was popular in 8-bit home computers era, using \$ prefix, like \$ABCD.
The book will try to stick to "0x" prefix throughout the book for hexadecimal numbers.

Should one learn to convert numbers mentally? A table of 1-digit hexadecimal numbers can easily be memorized.
As for larger numbers, it's probably not worth tormenting yourself.

Perhaps the most visible hexadecimal numbers are in \ac{URL}s.
This is the way that non-Latin characters are encoded.
For example:
\url{https://en.wiktionary.org/wiki/na\%C3\%AFvet\%C3\%A9} is the \ac{URL} of Wiktionary article about \q{naïveté} word.

\subsubsection{Octal Radix}

Another numeral system heavily used in the past of computer programming is octal. In octal there are 8 digits (0..7), and each is mapped to 3 bits, so it's easy to convert numbers back and forth.
It has been superseded by the hexadecimal system almost everywhere, but, surprisingly, there is a *NIX utility, used often by many people, which takes octal numbers as argument: \TT{chmod}.

\myindex{UNIX!chmod}
As many *NIX users know, \TT{chmod} argument can be a number of 3 digits. The first digit represents the rights of the owner of the file (read, write and/or execute), the second is the rights for the group to which the file belongs, and the third is for everyone else.
Each digit that \TT{chmod} takes can be represented in binary form:

\begin{center}
\begin{longtable}{ | l | l | l | }
\hline
\HeaderColor decimal & \HeaderColor binary & \HeaderColor meaning \\
\hline
7	&111	&\textbf{rwx} \\
6	&110	&\textbf{rw-} \\
5	&101	&\textbf{r-x} \\
4	&100	&\textbf{r-{}-} \\
3	&011	&\textbf{-wx} \\
2	&010	&\textbf{-w-} \\
1	&001	&\textbf{-{}-x} \\
0	&000	&\textbf{-{}-{}-} \\
\hline
\end{longtable}
\end{center}

So each bit is mapped to a flag: read/write/execute.

The importance of \TT{chmod} here is that the whole number in argument can be represented as octal number.
Let's take, for example, 644.
When you run \TT{chmod 644 file}, you set read/write permissions for owner, read permissions for group and again, read permissions for everyone else.
If we convert the octal number 644 to binary, it would be \TT{110100100}, or, in groups of 3 bits, \TT{110 100 100}.

Now we see that each triplet describe permissions for owner/group/others: first is \TT{rw-}, second is \TT{r--} and third is \TT{r--}.

The octal numeral system was also popular on old computers like PDP-8, because word there could be 12, 24 or 36 bits, and these numbers are all divisible by 3, so the octal system was natural in that environment.
Nowadays, all popular computers employ word/address sizes of 16, 32 or 64 bits, and these numbers are all divisible by 4, so the hexadecimal system is more natural there.

The octal numeral system is supported by all standard \CCpp compilers.
This is a source of confusion sometimes, because octal numbers are encoded with a zero prepended, for example, 0377 is 255.
Sometimes, you might make a typo and write "09" instead of 9, and the compiler would report an error.
GCC might report something like this:\\
\TT{error: invalid digit "9" in octal constant}.

Also, the octal system is somewhat popular in Java. When the IDA shows Java strings with non-printable characters,
they are encoded in the octal system instead of hexadecimal.
\myindex{JAD}
The JAD Java decompiler behaves the same way.

\subsubsection{Divisibility}

When you see a decimal number like 120, you can quickly deduce that it's divisible by 10, because the last digit is zero.
In the same way, 123400 is divisible by 100, because the two last digits are zeros.

Likewise, the hexadecimal number 0x1230 is divisible by 0x10 (or 16), 0x123000 is divisible by 0x1000 (or 4096), etc.

The binary number 0b1000101000 is divisible by 0b1000 (8), etc.

This property can often be used to quickly realize if the size of some block in memory is padded to some boundary.
For example, sections in \ac{PE} files are almost always started at addresses ending with 3 hexadecimal zeros: 0x41000, 0x10001000, etc.
The reason behind this is the fact that almost all \ac{PE} sections are padded to a boundary of 0x1000 (4096) bytes.

\subsubsection{Multi-Precision Arithmetic and Radix}

\index{RSA}
Multi-precision arithmetic can use huge numbers, and each one may be stored in several bytes.
For example, RSA keys, both public and private, span up to 4096 bits, and maybe even more.

% I'm not sure how to change this, but the normal format for quoting would be just to mention the author or book, and footnote to the full reference
In \InSqBrackets{\TAOCPvolII, 265} we find the following idea: when you store a multi-precision number in several bytes,
the whole number can be represented as having a radix of $2^8=256$, and each digit goes to the corresponding byte.
Likewise, if you store a multi-precision number in several 32-bit integer values, each digit goes to each 32-bit slot,
and you may think about this number as stored in radix of $2^{32}$.

\subsubsection{How to Pronounce Non-Decimal Numbers}

Numbers in a non-decimal base are usually pronounced by digit by digit: ``one-zero-zero-one-one-...''.
Words like ``ten'' and ``thousand'' are usually not pronounced, to prevent confusion with the decimal base system.

\subsubsection{Floating point numbers}

To distinguish floating point numbers from integers, they are usually written with ``.0'' at the end,
like $0.0$, $123.0$, etc.
}
\RU{\subsection{Представление чисел}

Люди привыкли к десятичной системе счисления вероятно потому что почти у каждого есть по 10 пальцев.
Тем не менее, число 10 не имеет особого значения в науке и математике.
Двоичная система естествена для цифровой электроники: 0 означает отсутствие тока в проводе и 1 --- его присутствие.
10 в двоичной системе это 2 в десятичной; 100 в двоичной это 4 в десятичной, итд.

Если в системе счисления есть 10 цифр, её \IT{основание} или \IT{radix} это 10.
Двоичная система имеет \IT{основание} 2.

Важные вещи, которые полезно вспомнить:
1) \IT{число} это число, в то время как \IT{цифра} это термин из системы письменности, и это обычно один символ;
2) само число не меняется, когда конвертируется из одного основания в другое: меняется способ его записи (или представления
в памяти).

Как сконвертировать число из одного основания в другое?

Позиционная нотация используется почти везде, это означает, что всякая цифра имеет свой вес, в зависимости от её расположения
внутри числа.
Если 2 расположена в самом последнем месте справа, это 2.
Если она расположена в месте перед последним, это 20.

Что означает $1234$?

$10^3 \cdot 1 + 10^2 \cdot 2 + 10^1 \cdot 3 + 1 \cdot 4$ = 1234 или
$1000 \cdot 1 + 100 \cdot 2 + 10 \cdot 3 + 4 = 1234$

Та же история и для двоичных чисел, только основание там 2 вместо 10.
Что означает 0b101011?

$2^5 \cdot 1 + 2^4 \cdot 0 + 2^3 \cdot 1 + 2^2 \cdot 0 + 2^1 \cdot 1 + 2^0 \cdot 1 = 43$ или
$32 \cdot 1 + 16 \cdot 0 + 8 \cdot 1 + 4 \cdot 0 + 2 \cdot 1 + 1 = 43$

Позиционную нотацию можно противопоставить непозиционной нотации, такой как римская система записи чисел
\footnote{Об эволюции способов записи чисел, см.также: \InSqBrackets{\TAOCPvolII{}, 195--213.}}.
Вероятно, человечество перешло на позиционную нотацию, потому что так проще работать с числами (сложение, умножение, итд)
на бумаге, в ручную.

Действительно, двоичные числа можно складывать, вычитать, итд, точно также, как этому обычно обучают в школах,
только доступны лишь 2 цифры.

Двоичные числа громоздки, когда их используют в исходных кодах и дампах, так что в этих случаях применяется шестнадцатеричная
система.
Используются цифры 0..9 и еще 6 латинских букв: A..F.
Каждая шестнадцатеричная цифра занимает 4 бита или 4 двоичных цифры, так что конвертировать из двоичной системы в
шестнадцатеричную и назад, можно легко вручную, или даже в уме.

\begin{center}
\begin{longtable}{ | l | l | l | }
\hline
\HeaderColor шестнадцатеричная & \HeaderColor двоичная & \HeaderColor десятичная \\
\hline
0	&0000	&0 \\
1	&0001	&1 \\
2	&0010	&2 \\
3	&0011	&3 \\
4	&0100	&4 \\
5	&0101	&5 \\
6	&0110	&6 \\
7	&0111	&7 \\
8	&1000	&8 \\
9	&1001	&9 \\
A	&1010	&10 \\
B	&1011	&11 \\
C	&1100	&12 \\
D	&1101	&13 \\
E	&1110	&14 \\
F	&1111	&15 \\
\hline
\end{longtable}
\end{center}

Как понять, какое основание используется в конкретном месте?

Десятичные числа обычно записываются как есть, т.е., 1234. Но некоторые ассемблеры позволяют подчеркивать
этот факт для ясности, и это число может быть дополнено суффиксом "d": 1234d.

К двоичным числам иногда спереди добавляют префикс "0b": 0b100110111
(В \ac{GCC} для этого есть нестандартное расширение языка
\footnote{\url{https://gcc.gnu.org/onlinedocs/gcc/Binary-constants.html}}).
Есть также еще один способ: суффикс "b", например: 100110111b.
В этой книге я буду пытаться придерживаться префикса "0b" для двоичных чисел.

Шестнадцатеричные числа имеют префикс "0x" в \CCpp и некоторых других \ac{PL}: 0x1234ABCD.
Либо они имеют суффикс "h": 1234ABCDh --- обычно так они представляются в ассемблерах и отладчиках.
Если число начинается с цифры A..F, перед ним добавляется 0: 0ABCDEFh.
Во времена 8-битных домашних компьютеров, был также способ записи чисел используя префикс \$, например, \$ABCD.
В книге я попытаюсь придерживаться префикса "0x" для шестнадцатеричных чисел.

Нужно ли учиться конвертировать числа в уме? Таблицу шестнадцатеричных чисел из одной цифры легко запомнить.
А запоминать б\'{о}льшие числа, наверное, не стоит.

Наверное, чаще всего шестнадцатеричные числа можно увидеть в \ac{URL}-ах.
Так кодируются буквы не из числа латинских.
Например:
\url{https://en.wiktionary.org/wiki/na\%C3\%AFvet\%C3\%A9} это \ac{URL} страницы в Wiktionary о слове \q{naïveté}.

\subsubsection{Восьмеричная система}

Еще одна система, которая в прошлом много использовалась в программировании это восьмеричная: есть 8 цифр (0..7) и каждая
описывает 3 бита, так что легко конвертировать числа туда и назад.
Она почти везде была заменена шестнадцатеричной, но удивительно, в *NIX имеется утилита использующаяся многими людьми,
которая принимает на вход восьмеричное число: \TT{chmod}.

\myindex{UNIX!chmod}
Как знают многие пользователи *NIX, аргумент \TT{chmod} это число из трех цифр. Первая цифра это права владельца файла,
вторая это права группы (которой файл принадлежит), третья для всех остальных.
И каждая цифра может быть представлена в двоичном виде:

\begin{center}
\begin{longtable}{ | l | l | l | }
\hline
\HeaderColor десятичная & \HeaderColor двоичная & \HeaderColor значение \\
\hline
7	&111	&\textbf{rwx} \\
6	&110	&\textbf{rw-} \\
5	&101	&\textbf{r-x} \\
4	&100	&\textbf{r-{}-} \\
3	&011	&\textbf{-wx} \\
2	&010	&\textbf{-w-} \\
1	&001	&\textbf{-{}-x} \\
0	&000	&\textbf{-{}-{}-} \\
\hline
\end{longtable}
\end{center}

Так что каждый бит привязан к флагу: read/write/execute (чтение/запись/исполнение).

И вот почему я вспомнил здесь о \TT{chmod}, это потому что всё число может быть представлено как число в восьмеричной системе.
Для примера возьмем 644.
Когда вы запускаете \TT{chmod 644 file}, вы выставляете права read/write для владельца, права read для группы, и снова,
read для всех остальных.
Сконвертируем число 644 из восьмеричной системы в двоичную, это будет \TT{110100100}, или (в группах по 3 бита) \TT{110 100 100}.

Теперь мы видим, что каждая тройка описывает права для владельца/группы/остальных:
первая это \TT{rw-}, вторая это \TT{r--} и третья это \TT{r--}.

Восьмеричная система была также популярная на старых компьютерах вроде PDP-8, потому что слово там могло содержать 12, 24 или
36 бит, и эти числа делятся на 3, так что выбор восьмеричной системы в той среде был логичен.
Сейчас, все популярные компьютеры имеют размер слова/адреса 16, 32 или 64 бита, и эти числа делятся на 4,
так что шестнадцатеричная система здесь удобнее.

Восьмеричная система поддерживается всеми стандартными компиляторами \CCpp{}.
Это иногда источник недоумения, потому что восьмеричные числа кодируются с нулем вперед, например, 0377 это 255.
И иногда, вы можете сделать опечатку, и написать "09" вместо 9, и компилятор выдаст ошибку.
GCC может выдать что-то вроде:\\
\TT{error: invalid digit "9" in octal constant}.

Также, восьмеричная система популярна в Java: когда IDA показывает строку с непечатаемыми символами,
они кодируются в восьмеричной системе вместо шестнадцатеричной.
\myindex{JAD}
Точно также себя ведет декомпилятор с Java JAD.

\subsubsection{Делимость}

Когда вы видите десятичное число вроде 120, вы можете быстро понять что оно делится на 10, потому что последняя цифра это 0.
Точно также, 123400 делится на 100, потому что две последних цифры это нули.

Точно также, шестнадцатеричное число 0x1230 делится на 0x10 (или 16), 0x123000 делится на 0x1000 (или 4096), итд.

Двоичное число 0b1000101000 делится на 0b1000 (8), итд.

Это свойство можно часто использовать, чтобы быстро понять,
что длина какого-либо блока в памяти выровнена по некоторой границе.
Например, секции в \ac{PE}-файлах почти всегда начинаются с адресов заканчивающихся 3 шестнадцатеричными нулями:
0x41000, 0x10001000, итд.
Причина в том, что почти все секции в \ac{PE} выровнены по границе 0x1000 (4096) байт.

\subsubsection{Арифметика произвольной точности и основание}

\index{RSA}
Арифметика произвольной точности (multi-precision arithmetic) может использовать огромные числа,
которые могут храниться в нескольких байтах.
Например, ключи RSA, и открытые и закрытые, могут занимать до 4096 бит и даже больше.

В \InSqBrackets{\TAOCPvolII, 265} можно найти такую идею: когда вы сохраняете число произвольной точности в нескольких байтах,
всё число может быть представлено как имеющую систему счисления по основанию $2^8=256$, и каждая цифра находится
в соответствующем байте.
Точно также, если вы сохраняете число произвольной точности в нескольких 32-битных целочисленных значениях,
каждая цифра отправляется в каждый 32-битный слот, и вы можете считать что это число записано в системе с основанием $2^{32}$.

\subsubsection{Произношение}

Числа в недесятичных системах счислениях обычно произносятся по одной цифре: ``один-ноль-ноль-один-один-...''.
Слова вроде ``десять'', ``тысяча'', итд, обычно не произносятся, потому что тогда можно спутать с десятичной системой.

\subsubsection{Числа с плавающей запятой}

Чтобы отличать числа с плавающей запятой от целочисленных, часто, в конце добавляют ``.0'',
например $0.0$, $123.0$, итд.

}
\ITA{\input{patterns/numeral_ITA}}
\DE{\input{patterns/numeral_DE}}
\FR{\input{patterns/numeral_FR}}
\PL{\input{patterns/numeral_PL}}

% chapters
\ifdefined\SPANISH
\chapter{Patrones de código}
\fi % SPANISH

\ifdefined\GERMAN
\chapter{Code-Muster}
\fi % GERMAN

\ifdefined\ENGLISH
\chapter{Code Patterns}
\fi % ENGLISH

\ifdefined\ITALIAN
\chapter{Forme di codice}
\fi % ITALIAN

\ifdefined\RUSSIAN
\chapter{Образцы кода}
\fi % RUSSIAN

\ifdefined\BRAZILIAN
\chapter{Padrões de códigos}
\fi % BRAZILIAN

\ifdefined\THAI
\chapter{รูปแบบของโค้ด}
\fi % THAI

\ifdefined\FRENCH
\chapter{Modèle de code}
\fi % FRENCH

\ifdefined\POLISH
\chapter{\PLph{}}
\fi % POLISH

% sections
\EN{\input{patterns/patterns_opt_dbg_EN}}
\ES{\input{patterns/patterns_opt_dbg_ES}}
\ITA{\input{patterns/patterns_opt_dbg_ITA}}
\PTBR{\input{patterns/patterns_opt_dbg_PTBR}}
\RU{\input{patterns/patterns_opt_dbg_RU}}
\THA{\input{patterns/patterns_opt_dbg_THA}}
\DE{\input{patterns/patterns_opt_dbg_DE}}
\FR{\input{patterns/patterns_opt_dbg_FR}}
\PL{\input{patterns/patterns_opt_dbg_PL}}

\RU{\section{Некоторые базовые понятия}}
\EN{\section{Some basics}}
\DE{\section{Einige Grundlagen}}
\FR{\section{Quelques bases}}
\ES{\section{\ESph{}}}
\ITA{\section{Alcune basi teoriche}}
\PTBR{\section{\PTBRph{}}}
\THA{\section{\THAph{}}}
\PL{\section{\PLph{}}}

% sections:
\EN{\input{patterns/intro_CPU_ISA_EN}}
\ES{\input{patterns/intro_CPU_ISA_ES}}
\ITA{\input{patterns/intro_CPU_ISA_ITA}}
\PTBR{\input{patterns/intro_CPU_ISA_PTBR}}
\RU{\input{patterns/intro_CPU_ISA_RU}}
\DE{\input{patterns/intro_CPU_ISA_DE}}
\FR{\input{patterns/intro_CPU_ISA_FR}}
\PL{\input{patterns/intro_CPU_ISA_PL}}

\EN{\input{patterns/numeral_EN}}
\RU{\input{patterns/numeral_RU}}
\ITA{\input{patterns/numeral_ITA}}
\DE{\input{patterns/numeral_DE}}
\FR{\input{patterns/numeral_FR}}
\PL{\input{patterns/numeral_PL}}

% chapters
\input{patterns/00_empty/main}
\input{patterns/011_ret/main}
\input{patterns/01_helloworld/main}
\input{patterns/015_prolog_epilogue/main}
\input{patterns/02_stack/main}
\input{patterns/03_printf/main}
\input{patterns/04_scanf/main}
\input{patterns/05_passing_arguments/main}
\input{patterns/06_return_results/main}
\input{patterns/061_pointers/main}
\input{patterns/065_GOTO/main}
\input{patterns/07_jcc/main}
\input{patterns/08_switch/main}
\input{patterns/09_loops/main}
\input{patterns/10_strings/main}
\input{patterns/11_arith_optimizations/main}
\input{patterns/12_FPU/main}
\input{patterns/13_arrays/main}
\input{patterns/14_bitfields/main}
\EN{\input{patterns/145_LCG/main_EN}}
\RU{\input{patterns/145_LCG/main_RU}}
\input{patterns/15_structs/main}
\input{patterns/17_unions/main}
\input{patterns/18_pointers_to_functions/main}
\input{patterns/185_64bit_in_32_env/main}

\EN{\input{patterns/19_SIMD/main_EN}}
\RU{\input{patterns/19_SIMD/main_RU}}
\DE{\input{patterns/19_SIMD/main_DE}}

\EN{\input{patterns/20_x64/main_EN}}
\RU{\input{patterns/20_x64/main_RU}}

\EN{\input{patterns/205_floating_SIMD/main_EN}}
\RU{\input{patterns/205_floating_SIMD/main_RU}}
\DE{\input{patterns/205_floating_SIMD/main_DE}}

\EN{\input{patterns/ARM/main_EN}}
\RU{\input{patterns/ARM/main_RU}}
\DE{\input{patterns/ARM/main_DE}}

\input{patterns/MIPS/main}

\ifdefined\SPANISH
\chapter{Patrones de código}
\fi % SPANISH

\ifdefined\GERMAN
\chapter{Code-Muster}
\fi % GERMAN

\ifdefined\ENGLISH
\chapter{Code Patterns}
\fi % ENGLISH

\ifdefined\ITALIAN
\chapter{Forme di codice}
\fi % ITALIAN

\ifdefined\RUSSIAN
\chapter{Образцы кода}
\fi % RUSSIAN

\ifdefined\BRAZILIAN
\chapter{Padrões de códigos}
\fi % BRAZILIAN

\ifdefined\THAI
\chapter{รูปแบบของโค้ด}
\fi % THAI

\ifdefined\FRENCH
\chapter{Modèle de code}
\fi % FRENCH

\ifdefined\POLISH
\chapter{\PLph{}}
\fi % POLISH

% sections
\EN{\input{patterns/patterns_opt_dbg_EN}}
\ES{\input{patterns/patterns_opt_dbg_ES}}
\ITA{\input{patterns/patterns_opt_dbg_ITA}}
\PTBR{\input{patterns/patterns_opt_dbg_PTBR}}
\RU{\input{patterns/patterns_opt_dbg_RU}}
\THA{\input{patterns/patterns_opt_dbg_THA}}
\DE{\input{patterns/patterns_opt_dbg_DE}}
\FR{\input{patterns/patterns_opt_dbg_FR}}
\PL{\input{patterns/patterns_opt_dbg_PL}}

\RU{\section{Некоторые базовые понятия}}
\EN{\section{Some basics}}
\DE{\section{Einige Grundlagen}}
\FR{\section{Quelques bases}}
\ES{\section{\ESph{}}}
\ITA{\section{Alcune basi teoriche}}
\PTBR{\section{\PTBRph{}}}
\THA{\section{\THAph{}}}
\PL{\section{\PLph{}}}

% sections:
\EN{\input{patterns/intro_CPU_ISA_EN}}
\ES{\input{patterns/intro_CPU_ISA_ES}}
\ITA{\input{patterns/intro_CPU_ISA_ITA}}
\PTBR{\input{patterns/intro_CPU_ISA_PTBR}}
\RU{\input{patterns/intro_CPU_ISA_RU}}
\DE{\input{patterns/intro_CPU_ISA_DE}}
\FR{\input{patterns/intro_CPU_ISA_FR}}
\PL{\input{patterns/intro_CPU_ISA_PL}}

\EN{\input{patterns/numeral_EN}}
\RU{\input{patterns/numeral_RU}}
\ITA{\input{patterns/numeral_ITA}}
\DE{\input{patterns/numeral_DE}}
\FR{\input{patterns/numeral_FR}}
\PL{\input{patterns/numeral_PL}}

% chapters
\input{patterns/00_empty/main}
\input{patterns/011_ret/main}
\input{patterns/01_helloworld/main}
\input{patterns/015_prolog_epilogue/main}
\input{patterns/02_stack/main}
\input{patterns/03_printf/main}
\input{patterns/04_scanf/main}
\input{patterns/05_passing_arguments/main}
\input{patterns/06_return_results/main}
\input{patterns/061_pointers/main}
\input{patterns/065_GOTO/main}
\input{patterns/07_jcc/main}
\input{patterns/08_switch/main}
\input{patterns/09_loops/main}
\input{patterns/10_strings/main}
\input{patterns/11_arith_optimizations/main}
\input{patterns/12_FPU/main}
\input{patterns/13_arrays/main}
\input{patterns/14_bitfields/main}
\EN{\input{patterns/145_LCG/main_EN}}
\RU{\input{patterns/145_LCG/main_RU}}
\input{patterns/15_structs/main}
\input{patterns/17_unions/main}
\input{patterns/18_pointers_to_functions/main}
\input{patterns/185_64bit_in_32_env/main}

\EN{\input{patterns/19_SIMD/main_EN}}
\RU{\input{patterns/19_SIMD/main_RU}}
\DE{\input{patterns/19_SIMD/main_DE}}

\EN{\input{patterns/20_x64/main_EN}}
\RU{\input{patterns/20_x64/main_RU}}

\EN{\input{patterns/205_floating_SIMD/main_EN}}
\RU{\input{patterns/205_floating_SIMD/main_RU}}
\DE{\input{patterns/205_floating_SIMD/main_DE}}

\EN{\input{patterns/ARM/main_EN}}
\RU{\input{patterns/ARM/main_RU}}
\DE{\input{patterns/ARM/main_DE}}

\input{patterns/MIPS/main}

\ifdefined\SPANISH
\chapter{Patrones de código}
\fi % SPANISH

\ifdefined\GERMAN
\chapter{Code-Muster}
\fi % GERMAN

\ifdefined\ENGLISH
\chapter{Code Patterns}
\fi % ENGLISH

\ifdefined\ITALIAN
\chapter{Forme di codice}
\fi % ITALIAN

\ifdefined\RUSSIAN
\chapter{Образцы кода}
\fi % RUSSIAN

\ifdefined\BRAZILIAN
\chapter{Padrões de códigos}
\fi % BRAZILIAN

\ifdefined\THAI
\chapter{รูปแบบของโค้ด}
\fi % THAI

\ifdefined\FRENCH
\chapter{Modèle de code}
\fi % FRENCH

\ifdefined\POLISH
\chapter{\PLph{}}
\fi % POLISH

% sections
\EN{\input{patterns/patterns_opt_dbg_EN}}
\ES{\input{patterns/patterns_opt_dbg_ES}}
\ITA{\input{patterns/patterns_opt_dbg_ITA}}
\PTBR{\input{patterns/patterns_opt_dbg_PTBR}}
\RU{\input{patterns/patterns_opt_dbg_RU}}
\THA{\input{patterns/patterns_opt_dbg_THA}}
\DE{\input{patterns/patterns_opt_dbg_DE}}
\FR{\input{patterns/patterns_opt_dbg_FR}}
\PL{\input{patterns/patterns_opt_dbg_PL}}

\RU{\section{Некоторые базовые понятия}}
\EN{\section{Some basics}}
\DE{\section{Einige Grundlagen}}
\FR{\section{Quelques bases}}
\ES{\section{\ESph{}}}
\ITA{\section{Alcune basi teoriche}}
\PTBR{\section{\PTBRph{}}}
\THA{\section{\THAph{}}}
\PL{\section{\PLph{}}}

% sections:
\EN{\input{patterns/intro_CPU_ISA_EN}}
\ES{\input{patterns/intro_CPU_ISA_ES}}
\ITA{\input{patterns/intro_CPU_ISA_ITA}}
\PTBR{\input{patterns/intro_CPU_ISA_PTBR}}
\RU{\input{patterns/intro_CPU_ISA_RU}}
\DE{\input{patterns/intro_CPU_ISA_DE}}
\FR{\input{patterns/intro_CPU_ISA_FR}}
\PL{\input{patterns/intro_CPU_ISA_PL}}

\EN{\input{patterns/numeral_EN}}
\RU{\input{patterns/numeral_RU}}
\ITA{\input{patterns/numeral_ITA}}
\DE{\input{patterns/numeral_DE}}
\FR{\input{patterns/numeral_FR}}
\PL{\input{patterns/numeral_PL}}

% chapters
\input{patterns/00_empty/main}
\input{patterns/011_ret/main}
\input{patterns/01_helloworld/main}
\input{patterns/015_prolog_epilogue/main}
\input{patterns/02_stack/main}
\input{patterns/03_printf/main}
\input{patterns/04_scanf/main}
\input{patterns/05_passing_arguments/main}
\input{patterns/06_return_results/main}
\input{patterns/061_pointers/main}
\input{patterns/065_GOTO/main}
\input{patterns/07_jcc/main}
\input{patterns/08_switch/main}
\input{patterns/09_loops/main}
\input{patterns/10_strings/main}
\input{patterns/11_arith_optimizations/main}
\input{patterns/12_FPU/main}
\input{patterns/13_arrays/main}
\input{patterns/14_bitfields/main}
\EN{\input{patterns/145_LCG/main_EN}}
\RU{\input{patterns/145_LCG/main_RU}}
\input{patterns/15_structs/main}
\input{patterns/17_unions/main}
\input{patterns/18_pointers_to_functions/main}
\input{patterns/185_64bit_in_32_env/main}

\EN{\input{patterns/19_SIMD/main_EN}}
\RU{\input{patterns/19_SIMD/main_RU}}
\DE{\input{patterns/19_SIMD/main_DE}}

\EN{\input{patterns/20_x64/main_EN}}
\RU{\input{patterns/20_x64/main_RU}}

\EN{\input{patterns/205_floating_SIMD/main_EN}}
\RU{\input{patterns/205_floating_SIMD/main_RU}}
\DE{\input{patterns/205_floating_SIMD/main_DE}}

\EN{\input{patterns/ARM/main_EN}}
\RU{\input{patterns/ARM/main_RU}}
\DE{\input{patterns/ARM/main_DE}}

\input{patterns/MIPS/main}

\ifdefined\SPANISH
\chapter{Patrones de código}
\fi % SPANISH

\ifdefined\GERMAN
\chapter{Code-Muster}
\fi % GERMAN

\ifdefined\ENGLISH
\chapter{Code Patterns}
\fi % ENGLISH

\ifdefined\ITALIAN
\chapter{Forme di codice}
\fi % ITALIAN

\ifdefined\RUSSIAN
\chapter{Образцы кода}
\fi % RUSSIAN

\ifdefined\BRAZILIAN
\chapter{Padrões de códigos}
\fi % BRAZILIAN

\ifdefined\THAI
\chapter{รูปแบบของโค้ด}
\fi % THAI

\ifdefined\FRENCH
\chapter{Modèle de code}
\fi % FRENCH

\ifdefined\POLISH
\chapter{\PLph{}}
\fi % POLISH

% sections
\EN{\input{patterns/patterns_opt_dbg_EN}}
\ES{\input{patterns/patterns_opt_dbg_ES}}
\ITA{\input{patterns/patterns_opt_dbg_ITA}}
\PTBR{\input{patterns/patterns_opt_dbg_PTBR}}
\RU{\input{patterns/patterns_opt_dbg_RU}}
\THA{\input{patterns/patterns_opt_dbg_THA}}
\DE{\input{patterns/patterns_opt_dbg_DE}}
\FR{\input{patterns/patterns_opt_dbg_FR}}
\PL{\input{patterns/patterns_opt_dbg_PL}}

\RU{\section{Некоторые базовые понятия}}
\EN{\section{Some basics}}
\DE{\section{Einige Grundlagen}}
\FR{\section{Quelques bases}}
\ES{\section{\ESph{}}}
\ITA{\section{Alcune basi teoriche}}
\PTBR{\section{\PTBRph{}}}
\THA{\section{\THAph{}}}
\PL{\section{\PLph{}}}

% sections:
\EN{\input{patterns/intro_CPU_ISA_EN}}
\ES{\input{patterns/intro_CPU_ISA_ES}}
\ITA{\input{patterns/intro_CPU_ISA_ITA}}
\PTBR{\input{patterns/intro_CPU_ISA_PTBR}}
\RU{\input{patterns/intro_CPU_ISA_RU}}
\DE{\input{patterns/intro_CPU_ISA_DE}}
\FR{\input{patterns/intro_CPU_ISA_FR}}
\PL{\input{patterns/intro_CPU_ISA_PL}}

\EN{\input{patterns/numeral_EN}}
\RU{\input{patterns/numeral_RU}}
\ITA{\input{patterns/numeral_ITA}}
\DE{\input{patterns/numeral_DE}}
\FR{\input{patterns/numeral_FR}}
\PL{\input{patterns/numeral_PL}}

% chapters
\input{patterns/00_empty/main}
\input{patterns/011_ret/main}
\input{patterns/01_helloworld/main}
\input{patterns/015_prolog_epilogue/main}
\input{patterns/02_stack/main}
\input{patterns/03_printf/main}
\input{patterns/04_scanf/main}
\input{patterns/05_passing_arguments/main}
\input{patterns/06_return_results/main}
\input{patterns/061_pointers/main}
\input{patterns/065_GOTO/main}
\input{patterns/07_jcc/main}
\input{patterns/08_switch/main}
\input{patterns/09_loops/main}
\input{patterns/10_strings/main}
\input{patterns/11_arith_optimizations/main}
\input{patterns/12_FPU/main}
\input{patterns/13_arrays/main}
\input{patterns/14_bitfields/main}
\EN{\input{patterns/145_LCG/main_EN}}
\RU{\input{patterns/145_LCG/main_RU}}
\input{patterns/15_structs/main}
\input{patterns/17_unions/main}
\input{patterns/18_pointers_to_functions/main}
\input{patterns/185_64bit_in_32_env/main}

\EN{\input{patterns/19_SIMD/main_EN}}
\RU{\input{patterns/19_SIMD/main_RU}}
\DE{\input{patterns/19_SIMD/main_DE}}

\EN{\input{patterns/20_x64/main_EN}}
\RU{\input{patterns/20_x64/main_RU}}

\EN{\input{patterns/205_floating_SIMD/main_EN}}
\RU{\input{patterns/205_floating_SIMD/main_RU}}
\DE{\input{patterns/205_floating_SIMD/main_DE}}

\EN{\input{patterns/ARM/main_EN}}
\RU{\input{patterns/ARM/main_RU}}
\DE{\input{patterns/ARM/main_DE}}

\input{patterns/MIPS/main}

\ifdefined\SPANISH
\chapter{Patrones de código}
\fi % SPANISH

\ifdefined\GERMAN
\chapter{Code-Muster}
\fi % GERMAN

\ifdefined\ENGLISH
\chapter{Code Patterns}
\fi % ENGLISH

\ifdefined\ITALIAN
\chapter{Forme di codice}
\fi % ITALIAN

\ifdefined\RUSSIAN
\chapter{Образцы кода}
\fi % RUSSIAN

\ifdefined\BRAZILIAN
\chapter{Padrões de códigos}
\fi % BRAZILIAN

\ifdefined\THAI
\chapter{รูปแบบของโค้ด}
\fi % THAI

\ifdefined\FRENCH
\chapter{Modèle de code}
\fi % FRENCH

\ifdefined\POLISH
\chapter{\PLph{}}
\fi % POLISH

% sections
\EN{\input{patterns/patterns_opt_dbg_EN}}
\ES{\input{patterns/patterns_opt_dbg_ES}}
\ITA{\input{patterns/patterns_opt_dbg_ITA}}
\PTBR{\input{patterns/patterns_opt_dbg_PTBR}}
\RU{\input{patterns/patterns_opt_dbg_RU}}
\THA{\input{patterns/patterns_opt_dbg_THA}}
\DE{\input{patterns/patterns_opt_dbg_DE}}
\FR{\input{patterns/patterns_opt_dbg_FR}}
\PL{\input{patterns/patterns_opt_dbg_PL}}

\RU{\section{Некоторые базовые понятия}}
\EN{\section{Some basics}}
\DE{\section{Einige Grundlagen}}
\FR{\section{Quelques bases}}
\ES{\section{\ESph{}}}
\ITA{\section{Alcune basi teoriche}}
\PTBR{\section{\PTBRph{}}}
\THA{\section{\THAph{}}}
\PL{\section{\PLph{}}}

% sections:
\EN{\input{patterns/intro_CPU_ISA_EN}}
\ES{\input{patterns/intro_CPU_ISA_ES}}
\ITA{\input{patterns/intro_CPU_ISA_ITA}}
\PTBR{\input{patterns/intro_CPU_ISA_PTBR}}
\RU{\input{patterns/intro_CPU_ISA_RU}}
\DE{\input{patterns/intro_CPU_ISA_DE}}
\FR{\input{patterns/intro_CPU_ISA_FR}}
\PL{\input{patterns/intro_CPU_ISA_PL}}

\EN{\input{patterns/numeral_EN}}
\RU{\input{patterns/numeral_RU}}
\ITA{\input{patterns/numeral_ITA}}
\DE{\input{patterns/numeral_DE}}
\FR{\input{patterns/numeral_FR}}
\PL{\input{patterns/numeral_PL}}

% chapters
\input{patterns/00_empty/main}
\input{patterns/011_ret/main}
\input{patterns/01_helloworld/main}
\input{patterns/015_prolog_epilogue/main}
\input{patterns/02_stack/main}
\input{patterns/03_printf/main}
\input{patterns/04_scanf/main}
\input{patterns/05_passing_arguments/main}
\input{patterns/06_return_results/main}
\input{patterns/061_pointers/main}
\input{patterns/065_GOTO/main}
\input{patterns/07_jcc/main}
\input{patterns/08_switch/main}
\input{patterns/09_loops/main}
\input{patterns/10_strings/main}
\input{patterns/11_arith_optimizations/main}
\input{patterns/12_FPU/main}
\input{patterns/13_arrays/main}
\input{patterns/14_bitfields/main}
\EN{\input{patterns/145_LCG/main_EN}}
\RU{\input{patterns/145_LCG/main_RU}}
\input{patterns/15_structs/main}
\input{patterns/17_unions/main}
\input{patterns/18_pointers_to_functions/main}
\input{patterns/185_64bit_in_32_env/main}

\EN{\input{patterns/19_SIMD/main_EN}}
\RU{\input{patterns/19_SIMD/main_RU}}
\DE{\input{patterns/19_SIMD/main_DE}}

\EN{\input{patterns/20_x64/main_EN}}
\RU{\input{patterns/20_x64/main_RU}}

\EN{\input{patterns/205_floating_SIMD/main_EN}}
\RU{\input{patterns/205_floating_SIMD/main_RU}}
\DE{\input{patterns/205_floating_SIMD/main_DE}}

\EN{\input{patterns/ARM/main_EN}}
\RU{\input{patterns/ARM/main_RU}}
\DE{\input{patterns/ARM/main_DE}}

\input{patterns/MIPS/main}

\ifdefined\SPANISH
\chapter{Patrones de código}
\fi % SPANISH

\ifdefined\GERMAN
\chapter{Code-Muster}
\fi % GERMAN

\ifdefined\ENGLISH
\chapter{Code Patterns}
\fi % ENGLISH

\ifdefined\ITALIAN
\chapter{Forme di codice}
\fi % ITALIAN

\ifdefined\RUSSIAN
\chapter{Образцы кода}
\fi % RUSSIAN

\ifdefined\BRAZILIAN
\chapter{Padrões de códigos}
\fi % BRAZILIAN

\ifdefined\THAI
\chapter{รูปแบบของโค้ด}
\fi % THAI

\ifdefined\FRENCH
\chapter{Modèle de code}
\fi % FRENCH

\ifdefined\POLISH
\chapter{\PLph{}}
\fi % POLISH

% sections
\EN{\input{patterns/patterns_opt_dbg_EN}}
\ES{\input{patterns/patterns_opt_dbg_ES}}
\ITA{\input{patterns/patterns_opt_dbg_ITA}}
\PTBR{\input{patterns/patterns_opt_dbg_PTBR}}
\RU{\input{patterns/patterns_opt_dbg_RU}}
\THA{\input{patterns/patterns_opt_dbg_THA}}
\DE{\input{patterns/patterns_opt_dbg_DE}}
\FR{\input{patterns/patterns_opt_dbg_FR}}
\PL{\input{patterns/patterns_opt_dbg_PL}}

\RU{\section{Некоторые базовые понятия}}
\EN{\section{Some basics}}
\DE{\section{Einige Grundlagen}}
\FR{\section{Quelques bases}}
\ES{\section{\ESph{}}}
\ITA{\section{Alcune basi teoriche}}
\PTBR{\section{\PTBRph{}}}
\THA{\section{\THAph{}}}
\PL{\section{\PLph{}}}

% sections:
\EN{\input{patterns/intro_CPU_ISA_EN}}
\ES{\input{patterns/intro_CPU_ISA_ES}}
\ITA{\input{patterns/intro_CPU_ISA_ITA}}
\PTBR{\input{patterns/intro_CPU_ISA_PTBR}}
\RU{\input{patterns/intro_CPU_ISA_RU}}
\DE{\input{patterns/intro_CPU_ISA_DE}}
\FR{\input{patterns/intro_CPU_ISA_FR}}
\PL{\input{patterns/intro_CPU_ISA_PL}}

\EN{\input{patterns/numeral_EN}}
\RU{\input{patterns/numeral_RU}}
\ITA{\input{patterns/numeral_ITA}}
\DE{\input{patterns/numeral_DE}}
\FR{\input{patterns/numeral_FR}}
\PL{\input{patterns/numeral_PL}}

% chapters
\input{patterns/00_empty/main}
\input{patterns/011_ret/main}
\input{patterns/01_helloworld/main}
\input{patterns/015_prolog_epilogue/main}
\input{patterns/02_stack/main}
\input{patterns/03_printf/main}
\input{patterns/04_scanf/main}
\input{patterns/05_passing_arguments/main}
\input{patterns/06_return_results/main}
\input{patterns/061_pointers/main}
\input{patterns/065_GOTO/main}
\input{patterns/07_jcc/main}
\input{patterns/08_switch/main}
\input{patterns/09_loops/main}
\input{patterns/10_strings/main}
\input{patterns/11_arith_optimizations/main}
\input{patterns/12_FPU/main}
\input{patterns/13_arrays/main}
\input{patterns/14_bitfields/main}
\EN{\input{patterns/145_LCG/main_EN}}
\RU{\input{patterns/145_LCG/main_RU}}
\input{patterns/15_structs/main}
\input{patterns/17_unions/main}
\input{patterns/18_pointers_to_functions/main}
\input{patterns/185_64bit_in_32_env/main}

\EN{\input{patterns/19_SIMD/main_EN}}
\RU{\input{patterns/19_SIMD/main_RU}}
\DE{\input{patterns/19_SIMD/main_DE}}

\EN{\input{patterns/20_x64/main_EN}}
\RU{\input{patterns/20_x64/main_RU}}

\EN{\input{patterns/205_floating_SIMD/main_EN}}
\RU{\input{patterns/205_floating_SIMD/main_RU}}
\DE{\input{patterns/205_floating_SIMD/main_DE}}

\EN{\input{patterns/ARM/main_EN}}
\RU{\input{patterns/ARM/main_RU}}
\DE{\input{patterns/ARM/main_DE}}

\input{patterns/MIPS/main}

\ifdefined\SPANISH
\chapter{Patrones de código}
\fi % SPANISH

\ifdefined\GERMAN
\chapter{Code-Muster}
\fi % GERMAN

\ifdefined\ENGLISH
\chapter{Code Patterns}
\fi % ENGLISH

\ifdefined\ITALIAN
\chapter{Forme di codice}
\fi % ITALIAN

\ifdefined\RUSSIAN
\chapter{Образцы кода}
\fi % RUSSIAN

\ifdefined\BRAZILIAN
\chapter{Padrões de códigos}
\fi % BRAZILIAN

\ifdefined\THAI
\chapter{รูปแบบของโค้ด}
\fi % THAI

\ifdefined\FRENCH
\chapter{Modèle de code}
\fi % FRENCH

\ifdefined\POLISH
\chapter{\PLph{}}
\fi % POLISH

% sections
\EN{\input{patterns/patterns_opt_dbg_EN}}
\ES{\input{patterns/patterns_opt_dbg_ES}}
\ITA{\input{patterns/patterns_opt_dbg_ITA}}
\PTBR{\input{patterns/patterns_opt_dbg_PTBR}}
\RU{\input{patterns/patterns_opt_dbg_RU}}
\THA{\input{patterns/patterns_opt_dbg_THA}}
\DE{\input{patterns/patterns_opt_dbg_DE}}
\FR{\input{patterns/patterns_opt_dbg_FR}}
\PL{\input{patterns/patterns_opt_dbg_PL}}

\RU{\section{Некоторые базовые понятия}}
\EN{\section{Some basics}}
\DE{\section{Einige Grundlagen}}
\FR{\section{Quelques bases}}
\ES{\section{\ESph{}}}
\ITA{\section{Alcune basi teoriche}}
\PTBR{\section{\PTBRph{}}}
\THA{\section{\THAph{}}}
\PL{\section{\PLph{}}}

% sections:
\EN{\input{patterns/intro_CPU_ISA_EN}}
\ES{\input{patterns/intro_CPU_ISA_ES}}
\ITA{\input{patterns/intro_CPU_ISA_ITA}}
\PTBR{\input{patterns/intro_CPU_ISA_PTBR}}
\RU{\input{patterns/intro_CPU_ISA_RU}}
\DE{\input{patterns/intro_CPU_ISA_DE}}
\FR{\input{patterns/intro_CPU_ISA_FR}}
\PL{\input{patterns/intro_CPU_ISA_PL}}

\EN{\input{patterns/numeral_EN}}
\RU{\input{patterns/numeral_RU}}
\ITA{\input{patterns/numeral_ITA}}
\DE{\input{patterns/numeral_DE}}
\FR{\input{patterns/numeral_FR}}
\PL{\input{patterns/numeral_PL}}

% chapters
\input{patterns/00_empty/main}
\input{patterns/011_ret/main}
\input{patterns/01_helloworld/main}
\input{patterns/015_prolog_epilogue/main}
\input{patterns/02_stack/main}
\input{patterns/03_printf/main}
\input{patterns/04_scanf/main}
\input{patterns/05_passing_arguments/main}
\input{patterns/06_return_results/main}
\input{patterns/061_pointers/main}
\input{patterns/065_GOTO/main}
\input{patterns/07_jcc/main}
\input{patterns/08_switch/main}
\input{patterns/09_loops/main}
\input{patterns/10_strings/main}
\input{patterns/11_arith_optimizations/main}
\input{patterns/12_FPU/main}
\input{patterns/13_arrays/main}
\input{patterns/14_bitfields/main}
\EN{\input{patterns/145_LCG/main_EN}}
\RU{\input{patterns/145_LCG/main_RU}}
\input{patterns/15_structs/main}
\input{patterns/17_unions/main}
\input{patterns/18_pointers_to_functions/main}
\input{patterns/185_64bit_in_32_env/main}

\EN{\input{patterns/19_SIMD/main_EN}}
\RU{\input{patterns/19_SIMD/main_RU}}
\DE{\input{patterns/19_SIMD/main_DE}}

\EN{\input{patterns/20_x64/main_EN}}
\RU{\input{patterns/20_x64/main_RU}}

\EN{\input{patterns/205_floating_SIMD/main_EN}}
\RU{\input{patterns/205_floating_SIMD/main_RU}}
\DE{\input{patterns/205_floating_SIMD/main_DE}}

\EN{\input{patterns/ARM/main_EN}}
\RU{\input{patterns/ARM/main_RU}}
\DE{\input{patterns/ARM/main_DE}}

\input{patterns/MIPS/main}

\ifdefined\SPANISH
\chapter{Patrones de código}
\fi % SPANISH

\ifdefined\GERMAN
\chapter{Code-Muster}
\fi % GERMAN

\ifdefined\ENGLISH
\chapter{Code Patterns}
\fi % ENGLISH

\ifdefined\ITALIAN
\chapter{Forme di codice}
\fi % ITALIAN

\ifdefined\RUSSIAN
\chapter{Образцы кода}
\fi % RUSSIAN

\ifdefined\BRAZILIAN
\chapter{Padrões de códigos}
\fi % BRAZILIAN

\ifdefined\THAI
\chapter{รูปแบบของโค้ด}
\fi % THAI

\ifdefined\FRENCH
\chapter{Modèle de code}
\fi % FRENCH

\ifdefined\POLISH
\chapter{\PLph{}}
\fi % POLISH

% sections
\EN{\input{patterns/patterns_opt_dbg_EN}}
\ES{\input{patterns/patterns_opt_dbg_ES}}
\ITA{\input{patterns/patterns_opt_dbg_ITA}}
\PTBR{\input{patterns/patterns_opt_dbg_PTBR}}
\RU{\input{patterns/patterns_opt_dbg_RU}}
\THA{\input{patterns/patterns_opt_dbg_THA}}
\DE{\input{patterns/patterns_opt_dbg_DE}}
\FR{\input{patterns/patterns_opt_dbg_FR}}
\PL{\input{patterns/patterns_opt_dbg_PL}}

\RU{\section{Некоторые базовые понятия}}
\EN{\section{Some basics}}
\DE{\section{Einige Grundlagen}}
\FR{\section{Quelques bases}}
\ES{\section{\ESph{}}}
\ITA{\section{Alcune basi teoriche}}
\PTBR{\section{\PTBRph{}}}
\THA{\section{\THAph{}}}
\PL{\section{\PLph{}}}

% sections:
\EN{\input{patterns/intro_CPU_ISA_EN}}
\ES{\input{patterns/intro_CPU_ISA_ES}}
\ITA{\input{patterns/intro_CPU_ISA_ITA}}
\PTBR{\input{patterns/intro_CPU_ISA_PTBR}}
\RU{\input{patterns/intro_CPU_ISA_RU}}
\DE{\input{patterns/intro_CPU_ISA_DE}}
\FR{\input{patterns/intro_CPU_ISA_FR}}
\PL{\input{patterns/intro_CPU_ISA_PL}}

\EN{\input{patterns/numeral_EN}}
\RU{\input{patterns/numeral_RU}}
\ITA{\input{patterns/numeral_ITA}}
\DE{\input{patterns/numeral_DE}}
\FR{\input{patterns/numeral_FR}}
\PL{\input{patterns/numeral_PL}}

% chapters
\input{patterns/00_empty/main}
\input{patterns/011_ret/main}
\input{patterns/01_helloworld/main}
\input{patterns/015_prolog_epilogue/main}
\input{patterns/02_stack/main}
\input{patterns/03_printf/main}
\input{patterns/04_scanf/main}
\input{patterns/05_passing_arguments/main}
\input{patterns/06_return_results/main}
\input{patterns/061_pointers/main}
\input{patterns/065_GOTO/main}
\input{patterns/07_jcc/main}
\input{patterns/08_switch/main}
\input{patterns/09_loops/main}
\input{patterns/10_strings/main}
\input{patterns/11_arith_optimizations/main}
\input{patterns/12_FPU/main}
\input{patterns/13_arrays/main}
\input{patterns/14_bitfields/main}
\EN{\input{patterns/145_LCG/main_EN}}
\RU{\input{patterns/145_LCG/main_RU}}
\input{patterns/15_structs/main}
\input{patterns/17_unions/main}
\input{patterns/18_pointers_to_functions/main}
\input{patterns/185_64bit_in_32_env/main}

\EN{\input{patterns/19_SIMD/main_EN}}
\RU{\input{patterns/19_SIMD/main_RU}}
\DE{\input{patterns/19_SIMD/main_DE}}

\EN{\input{patterns/20_x64/main_EN}}
\RU{\input{patterns/20_x64/main_RU}}

\EN{\input{patterns/205_floating_SIMD/main_EN}}
\RU{\input{patterns/205_floating_SIMD/main_RU}}
\DE{\input{patterns/205_floating_SIMD/main_DE}}

\EN{\input{patterns/ARM/main_EN}}
\RU{\input{patterns/ARM/main_RU}}
\DE{\input{patterns/ARM/main_DE}}

\input{patterns/MIPS/main}

\ifdefined\SPANISH
\chapter{Patrones de código}
\fi % SPANISH

\ifdefined\GERMAN
\chapter{Code-Muster}
\fi % GERMAN

\ifdefined\ENGLISH
\chapter{Code Patterns}
\fi % ENGLISH

\ifdefined\ITALIAN
\chapter{Forme di codice}
\fi % ITALIAN

\ifdefined\RUSSIAN
\chapter{Образцы кода}
\fi % RUSSIAN

\ifdefined\BRAZILIAN
\chapter{Padrões de códigos}
\fi % BRAZILIAN

\ifdefined\THAI
\chapter{รูปแบบของโค้ด}
\fi % THAI

\ifdefined\FRENCH
\chapter{Modèle de code}
\fi % FRENCH

\ifdefined\POLISH
\chapter{\PLph{}}
\fi % POLISH

% sections
\EN{\input{patterns/patterns_opt_dbg_EN}}
\ES{\input{patterns/patterns_opt_dbg_ES}}
\ITA{\input{patterns/patterns_opt_dbg_ITA}}
\PTBR{\input{patterns/patterns_opt_dbg_PTBR}}
\RU{\input{patterns/patterns_opt_dbg_RU}}
\THA{\input{patterns/patterns_opt_dbg_THA}}
\DE{\input{patterns/patterns_opt_dbg_DE}}
\FR{\input{patterns/patterns_opt_dbg_FR}}
\PL{\input{patterns/patterns_opt_dbg_PL}}

\RU{\section{Некоторые базовые понятия}}
\EN{\section{Some basics}}
\DE{\section{Einige Grundlagen}}
\FR{\section{Quelques bases}}
\ES{\section{\ESph{}}}
\ITA{\section{Alcune basi teoriche}}
\PTBR{\section{\PTBRph{}}}
\THA{\section{\THAph{}}}
\PL{\section{\PLph{}}}

% sections:
\EN{\input{patterns/intro_CPU_ISA_EN}}
\ES{\input{patterns/intro_CPU_ISA_ES}}
\ITA{\input{patterns/intro_CPU_ISA_ITA}}
\PTBR{\input{patterns/intro_CPU_ISA_PTBR}}
\RU{\input{patterns/intro_CPU_ISA_RU}}
\DE{\input{patterns/intro_CPU_ISA_DE}}
\FR{\input{patterns/intro_CPU_ISA_FR}}
\PL{\input{patterns/intro_CPU_ISA_PL}}

\EN{\input{patterns/numeral_EN}}
\RU{\input{patterns/numeral_RU}}
\ITA{\input{patterns/numeral_ITA}}
\DE{\input{patterns/numeral_DE}}
\FR{\input{patterns/numeral_FR}}
\PL{\input{patterns/numeral_PL}}

% chapters
\input{patterns/00_empty/main}
\input{patterns/011_ret/main}
\input{patterns/01_helloworld/main}
\input{patterns/015_prolog_epilogue/main}
\input{patterns/02_stack/main}
\input{patterns/03_printf/main}
\input{patterns/04_scanf/main}
\input{patterns/05_passing_arguments/main}
\input{patterns/06_return_results/main}
\input{patterns/061_pointers/main}
\input{patterns/065_GOTO/main}
\input{patterns/07_jcc/main}
\input{patterns/08_switch/main}
\input{patterns/09_loops/main}
\input{patterns/10_strings/main}
\input{patterns/11_arith_optimizations/main}
\input{patterns/12_FPU/main}
\input{patterns/13_arrays/main}
\input{patterns/14_bitfields/main}
\EN{\input{patterns/145_LCG/main_EN}}
\RU{\input{patterns/145_LCG/main_RU}}
\input{patterns/15_structs/main}
\input{patterns/17_unions/main}
\input{patterns/18_pointers_to_functions/main}
\input{patterns/185_64bit_in_32_env/main}

\EN{\input{patterns/19_SIMD/main_EN}}
\RU{\input{patterns/19_SIMD/main_RU}}
\DE{\input{patterns/19_SIMD/main_DE}}

\EN{\input{patterns/20_x64/main_EN}}
\RU{\input{patterns/20_x64/main_RU}}

\EN{\input{patterns/205_floating_SIMD/main_EN}}
\RU{\input{patterns/205_floating_SIMD/main_RU}}
\DE{\input{patterns/205_floating_SIMD/main_DE}}

\EN{\input{patterns/ARM/main_EN}}
\RU{\input{patterns/ARM/main_RU}}
\DE{\input{patterns/ARM/main_DE}}

\input{patterns/MIPS/main}

\ifdefined\SPANISH
\chapter{Patrones de código}
\fi % SPANISH

\ifdefined\GERMAN
\chapter{Code-Muster}
\fi % GERMAN

\ifdefined\ENGLISH
\chapter{Code Patterns}
\fi % ENGLISH

\ifdefined\ITALIAN
\chapter{Forme di codice}
\fi % ITALIAN

\ifdefined\RUSSIAN
\chapter{Образцы кода}
\fi % RUSSIAN

\ifdefined\BRAZILIAN
\chapter{Padrões de códigos}
\fi % BRAZILIAN

\ifdefined\THAI
\chapter{รูปแบบของโค้ด}
\fi % THAI

\ifdefined\FRENCH
\chapter{Modèle de code}
\fi % FRENCH

\ifdefined\POLISH
\chapter{\PLph{}}
\fi % POLISH

% sections
\EN{\input{patterns/patterns_opt_dbg_EN}}
\ES{\input{patterns/patterns_opt_dbg_ES}}
\ITA{\input{patterns/patterns_opt_dbg_ITA}}
\PTBR{\input{patterns/patterns_opt_dbg_PTBR}}
\RU{\input{patterns/patterns_opt_dbg_RU}}
\THA{\input{patterns/patterns_opt_dbg_THA}}
\DE{\input{patterns/patterns_opt_dbg_DE}}
\FR{\input{patterns/patterns_opt_dbg_FR}}
\PL{\input{patterns/patterns_opt_dbg_PL}}

\RU{\section{Некоторые базовые понятия}}
\EN{\section{Some basics}}
\DE{\section{Einige Grundlagen}}
\FR{\section{Quelques bases}}
\ES{\section{\ESph{}}}
\ITA{\section{Alcune basi teoriche}}
\PTBR{\section{\PTBRph{}}}
\THA{\section{\THAph{}}}
\PL{\section{\PLph{}}}

% sections:
\EN{\input{patterns/intro_CPU_ISA_EN}}
\ES{\input{patterns/intro_CPU_ISA_ES}}
\ITA{\input{patterns/intro_CPU_ISA_ITA}}
\PTBR{\input{patterns/intro_CPU_ISA_PTBR}}
\RU{\input{patterns/intro_CPU_ISA_RU}}
\DE{\input{patterns/intro_CPU_ISA_DE}}
\FR{\input{patterns/intro_CPU_ISA_FR}}
\PL{\input{patterns/intro_CPU_ISA_PL}}

\EN{\input{patterns/numeral_EN}}
\RU{\input{patterns/numeral_RU}}
\ITA{\input{patterns/numeral_ITA}}
\DE{\input{patterns/numeral_DE}}
\FR{\input{patterns/numeral_FR}}
\PL{\input{patterns/numeral_PL}}

% chapters
\input{patterns/00_empty/main}
\input{patterns/011_ret/main}
\input{patterns/01_helloworld/main}
\input{patterns/015_prolog_epilogue/main}
\input{patterns/02_stack/main}
\input{patterns/03_printf/main}
\input{patterns/04_scanf/main}
\input{patterns/05_passing_arguments/main}
\input{patterns/06_return_results/main}
\input{patterns/061_pointers/main}
\input{patterns/065_GOTO/main}
\input{patterns/07_jcc/main}
\input{patterns/08_switch/main}
\input{patterns/09_loops/main}
\input{patterns/10_strings/main}
\input{patterns/11_arith_optimizations/main}
\input{patterns/12_FPU/main}
\input{patterns/13_arrays/main}
\input{patterns/14_bitfields/main}
\EN{\input{patterns/145_LCG/main_EN}}
\RU{\input{patterns/145_LCG/main_RU}}
\input{patterns/15_structs/main}
\input{patterns/17_unions/main}
\input{patterns/18_pointers_to_functions/main}
\input{patterns/185_64bit_in_32_env/main}

\EN{\input{patterns/19_SIMD/main_EN}}
\RU{\input{patterns/19_SIMD/main_RU}}
\DE{\input{patterns/19_SIMD/main_DE}}

\EN{\input{patterns/20_x64/main_EN}}
\RU{\input{patterns/20_x64/main_RU}}

\EN{\input{patterns/205_floating_SIMD/main_EN}}
\RU{\input{patterns/205_floating_SIMD/main_RU}}
\DE{\input{patterns/205_floating_SIMD/main_DE}}

\EN{\input{patterns/ARM/main_EN}}
\RU{\input{patterns/ARM/main_RU}}
\DE{\input{patterns/ARM/main_DE}}

\input{patterns/MIPS/main}

\ifdefined\SPANISH
\chapter{Patrones de código}
\fi % SPANISH

\ifdefined\GERMAN
\chapter{Code-Muster}
\fi % GERMAN

\ifdefined\ENGLISH
\chapter{Code Patterns}
\fi % ENGLISH

\ifdefined\ITALIAN
\chapter{Forme di codice}
\fi % ITALIAN

\ifdefined\RUSSIAN
\chapter{Образцы кода}
\fi % RUSSIAN

\ifdefined\BRAZILIAN
\chapter{Padrões de códigos}
\fi % BRAZILIAN

\ifdefined\THAI
\chapter{รูปแบบของโค้ด}
\fi % THAI

\ifdefined\FRENCH
\chapter{Modèle de code}
\fi % FRENCH

\ifdefined\POLISH
\chapter{\PLph{}}
\fi % POLISH

% sections
\EN{\input{patterns/patterns_opt_dbg_EN}}
\ES{\input{patterns/patterns_opt_dbg_ES}}
\ITA{\input{patterns/patterns_opt_dbg_ITA}}
\PTBR{\input{patterns/patterns_opt_dbg_PTBR}}
\RU{\input{patterns/patterns_opt_dbg_RU}}
\THA{\input{patterns/patterns_opt_dbg_THA}}
\DE{\input{patterns/patterns_opt_dbg_DE}}
\FR{\input{patterns/patterns_opt_dbg_FR}}
\PL{\input{patterns/patterns_opt_dbg_PL}}

\RU{\section{Некоторые базовые понятия}}
\EN{\section{Some basics}}
\DE{\section{Einige Grundlagen}}
\FR{\section{Quelques bases}}
\ES{\section{\ESph{}}}
\ITA{\section{Alcune basi teoriche}}
\PTBR{\section{\PTBRph{}}}
\THA{\section{\THAph{}}}
\PL{\section{\PLph{}}}

% sections:
\EN{\input{patterns/intro_CPU_ISA_EN}}
\ES{\input{patterns/intro_CPU_ISA_ES}}
\ITA{\input{patterns/intro_CPU_ISA_ITA}}
\PTBR{\input{patterns/intro_CPU_ISA_PTBR}}
\RU{\input{patterns/intro_CPU_ISA_RU}}
\DE{\input{patterns/intro_CPU_ISA_DE}}
\FR{\input{patterns/intro_CPU_ISA_FR}}
\PL{\input{patterns/intro_CPU_ISA_PL}}

\EN{\input{patterns/numeral_EN}}
\RU{\input{patterns/numeral_RU}}
\ITA{\input{patterns/numeral_ITA}}
\DE{\input{patterns/numeral_DE}}
\FR{\input{patterns/numeral_FR}}
\PL{\input{patterns/numeral_PL}}

% chapters
\input{patterns/00_empty/main}
\input{patterns/011_ret/main}
\input{patterns/01_helloworld/main}
\input{patterns/015_prolog_epilogue/main}
\input{patterns/02_stack/main}
\input{patterns/03_printf/main}
\input{patterns/04_scanf/main}
\input{patterns/05_passing_arguments/main}
\input{patterns/06_return_results/main}
\input{patterns/061_pointers/main}
\input{patterns/065_GOTO/main}
\input{patterns/07_jcc/main}
\input{patterns/08_switch/main}
\input{patterns/09_loops/main}
\input{patterns/10_strings/main}
\input{patterns/11_arith_optimizations/main}
\input{patterns/12_FPU/main}
\input{patterns/13_arrays/main}
\input{patterns/14_bitfields/main}
\EN{\input{patterns/145_LCG/main_EN}}
\RU{\input{patterns/145_LCG/main_RU}}
\input{patterns/15_structs/main}
\input{patterns/17_unions/main}
\input{patterns/18_pointers_to_functions/main}
\input{patterns/185_64bit_in_32_env/main}

\EN{\input{patterns/19_SIMD/main_EN}}
\RU{\input{patterns/19_SIMD/main_RU}}
\DE{\input{patterns/19_SIMD/main_DE}}

\EN{\input{patterns/20_x64/main_EN}}
\RU{\input{patterns/20_x64/main_RU}}

\EN{\input{patterns/205_floating_SIMD/main_EN}}
\RU{\input{patterns/205_floating_SIMD/main_RU}}
\DE{\input{patterns/205_floating_SIMD/main_DE}}

\EN{\input{patterns/ARM/main_EN}}
\RU{\input{patterns/ARM/main_RU}}
\DE{\input{patterns/ARM/main_DE}}

\input{patterns/MIPS/main}

\ifdefined\SPANISH
\chapter{Patrones de código}
\fi % SPANISH

\ifdefined\GERMAN
\chapter{Code-Muster}
\fi % GERMAN

\ifdefined\ENGLISH
\chapter{Code Patterns}
\fi % ENGLISH

\ifdefined\ITALIAN
\chapter{Forme di codice}
\fi % ITALIAN

\ifdefined\RUSSIAN
\chapter{Образцы кода}
\fi % RUSSIAN

\ifdefined\BRAZILIAN
\chapter{Padrões de códigos}
\fi % BRAZILIAN

\ifdefined\THAI
\chapter{รูปแบบของโค้ด}
\fi % THAI

\ifdefined\FRENCH
\chapter{Modèle de code}
\fi % FRENCH

\ifdefined\POLISH
\chapter{\PLph{}}
\fi % POLISH

% sections
\EN{\input{patterns/patterns_opt_dbg_EN}}
\ES{\input{patterns/patterns_opt_dbg_ES}}
\ITA{\input{patterns/patterns_opt_dbg_ITA}}
\PTBR{\input{patterns/patterns_opt_dbg_PTBR}}
\RU{\input{patterns/patterns_opt_dbg_RU}}
\THA{\input{patterns/patterns_opt_dbg_THA}}
\DE{\input{patterns/patterns_opt_dbg_DE}}
\FR{\input{patterns/patterns_opt_dbg_FR}}
\PL{\input{patterns/patterns_opt_dbg_PL}}

\RU{\section{Некоторые базовые понятия}}
\EN{\section{Some basics}}
\DE{\section{Einige Grundlagen}}
\FR{\section{Quelques bases}}
\ES{\section{\ESph{}}}
\ITA{\section{Alcune basi teoriche}}
\PTBR{\section{\PTBRph{}}}
\THA{\section{\THAph{}}}
\PL{\section{\PLph{}}}

% sections:
\EN{\input{patterns/intro_CPU_ISA_EN}}
\ES{\input{patterns/intro_CPU_ISA_ES}}
\ITA{\input{patterns/intro_CPU_ISA_ITA}}
\PTBR{\input{patterns/intro_CPU_ISA_PTBR}}
\RU{\input{patterns/intro_CPU_ISA_RU}}
\DE{\input{patterns/intro_CPU_ISA_DE}}
\FR{\input{patterns/intro_CPU_ISA_FR}}
\PL{\input{patterns/intro_CPU_ISA_PL}}

\EN{\input{patterns/numeral_EN}}
\RU{\input{patterns/numeral_RU}}
\ITA{\input{patterns/numeral_ITA}}
\DE{\input{patterns/numeral_DE}}
\FR{\input{patterns/numeral_FR}}
\PL{\input{patterns/numeral_PL}}

% chapters
\input{patterns/00_empty/main}
\input{patterns/011_ret/main}
\input{patterns/01_helloworld/main}
\input{patterns/015_prolog_epilogue/main}
\input{patterns/02_stack/main}
\input{patterns/03_printf/main}
\input{patterns/04_scanf/main}
\input{patterns/05_passing_arguments/main}
\input{patterns/06_return_results/main}
\input{patterns/061_pointers/main}
\input{patterns/065_GOTO/main}
\input{patterns/07_jcc/main}
\input{patterns/08_switch/main}
\input{patterns/09_loops/main}
\input{patterns/10_strings/main}
\input{patterns/11_arith_optimizations/main}
\input{patterns/12_FPU/main}
\input{patterns/13_arrays/main}
\input{patterns/14_bitfields/main}
\EN{\input{patterns/145_LCG/main_EN}}
\RU{\input{patterns/145_LCG/main_RU}}
\input{patterns/15_structs/main}
\input{patterns/17_unions/main}
\input{patterns/18_pointers_to_functions/main}
\input{patterns/185_64bit_in_32_env/main}

\EN{\input{patterns/19_SIMD/main_EN}}
\RU{\input{patterns/19_SIMD/main_RU}}
\DE{\input{patterns/19_SIMD/main_DE}}

\EN{\input{patterns/20_x64/main_EN}}
\RU{\input{patterns/20_x64/main_RU}}

\EN{\input{patterns/205_floating_SIMD/main_EN}}
\RU{\input{patterns/205_floating_SIMD/main_RU}}
\DE{\input{patterns/205_floating_SIMD/main_DE}}

\EN{\input{patterns/ARM/main_EN}}
\RU{\input{patterns/ARM/main_RU}}
\DE{\input{patterns/ARM/main_DE}}

\input{patterns/MIPS/main}

\ifdefined\SPANISH
\chapter{Patrones de código}
\fi % SPANISH

\ifdefined\GERMAN
\chapter{Code-Muster}
\fi % GERMAN

\ifdefined\ENGLISH
\chapter{Code Patterns}
\fi % ENGLISH

\ifdefined\ITALIAN
\chapter{Forme di codice}
\fi % ITALIAN

\ifdefined\RUSSIAN
\chapter{Образцы кода}
\fi % RUSSIAN

\ifdefined\BRAZILIAN
\chapter{Padrões de códigos}
\fi % BRAZILIAN

\ifdefined\THAI
\chapter{รูปแบบของโค้ด}
\fi % THAI

\ifdefined\FRENCH
\chapter{Modèle de code}
\fi % FRENCH

\ifdefined\POLISH
\chapter{\PLph{}}
\fi % POLISH

% sections
\EN{\input{patterns/patterns_opt_dbg_EN}}
\ES{\input{patterns/patterns_opt_dbg_ES}}
\ITA{\input{patterns/patterns_opt_dbg_ITA}}
\PTBR{\input{patterns/patterns_opt_dbg_PTBR}}
\RU{\input{patterns/patterns_opt_dbg_RU}}
\THA{\input{patterns/patterns_opt_dbg_THA}}
\DE{\input{patterns/patterns_opt_dbg_DE}}
\FR{\input{patterns/patterns_opt_dbg_FR}}
\PL{\input{patterns/patterns_opt_dbg_PL}}

\RU{\section{Некоторые базовые понятия}}
\EN{\section{Some basics}}
\DE{\section{Einige Grundlagen}}
\FR{\section{Quelques bases}}
\ES{\section{\ESph{}}}
\ITA{\section{Alcune basi teoriche}}
\PTBR{\section{\PTBRph{}}}
\THA{\section{\THAph{}}}
\PL{\section{\PLph{}}}

% sections:
\EN{\input{patterns/intro_CPU_ISA_EN}}
\ES{\input{patterns/intro_CPU_ISA_ES}}
\ITA{\input{patterns/intro_CPU_ISA_ITA}}
\PTBR{\input{patterns/intro_CPU_ISA_PTBR}}
\RU{\input{patterns/intro_CPU_ISA_RU}}
\DE{\input{patterns/intro_CPU_ISA_DE}}
\FR{\input{patterns/intro_CPU_ISA_FR}}
\PL{\input{patterns/intro_CPU_ISA_PL}}

\EN{\input{patterns/numeral_EN}}
\RU{\input{patterns/numeral_RU}}
\ITA{\input{patterns/numeral_ITA}}
\DE{\input{patterns/numeral_DE}}
\FR{\input{patterns/numeral_FR}}
\PL{\input{patterns/numeral_PL}}

% chapters
\input{patterns/00_empty/main}
\input{patterns/011_ret/main}
\input{patterns/01_helloworld/main}
\input{patterns/015_prolog_epilogue/main}
\input{patterns/02_stack/main}
\input{patterns/03_printf/main}
\input{patterns/04_scanf/main}
\input{patterns/05_passing_arguments/main}
\input{patterns/06_return_results/main}
\input{patterns/061_pointers/main}
\input{patterns/065_GOTO/main}
\input{patterns/07_jcc/main}
\input{patterns/08_switch/main}
\input{patterns/09_loops/main}
\input{patterns/10_strings/main}
\input{patterns/11_arith_optimizations/main}
\input{patterns/12_FPU/main}
\input{patterns/13_arrays/main}
\input{patterns/14_bitfields/main}
\EN{\input{patterns/145_LCG/main_EN}}
\RU{\input{patterns/145_LCG/main_RU}}
\input{patterns/15_structs/main}
\input{patterns/17_unions/main}
\input{patterns/18_pointers_to_functions/main}
\input{patterns/185_64bit_in_32_env/main}

\EN{\input{patterns/19_SIMD/main_EN}}
\RU{\input{patterns/19_SIMD/main_RU}}
\DE{\input{patterns/19_SIMD/main_DE}}

\EN{\input{patterns/20_x64/main_EN}}
\RU{\input{patterns/20_x64/main_RU}}

\EN{\input{patterns/205_floating_SIMD/main_EN}}
\RU{\input{patterns/205_floating_SIMD/main_RU}}
\DE{\input{patterns/205_floating_SIMD/main_DE}}

\EN{\input{patterns/ARM/main_EN}}
\RU{\input{patterns/ARM/main_RU}}
\DE{\input{patterns/ARM/main_DE}}

\input{patterns/MIPS/main}

\ifdefined\SPANISH
\chapter{Patrones de código}
\fi % SPANISH

\ifdefined\GERMAN
\chapter{Code-Muster}
\fi % GERMAN

\ifdefined\ENGLISH
\chapter{Code Patterns}
\fi % ENGLISH

\ifdefined\ITALIAN
\chapter{Forme di codice}
\fi % ITALIAN

\ifdefined\RUSSIAN
\chapter{Образцы кода}
\fi % RUSSIAN

\ifdefined\BRAZILIAN
\chapter{Padrões de códigos}
\fi % BRAZILIAN

\ifdefined\THAI
\chapter{รูปแบบของโค้ด}
\fi % THAI

\ifdefined\FRENCH
\chapter{Modèle de code}
\fi % FRENCH

\ifdefined\POLISH
\chapter{\PLph{}}
\fi % POLISH

% sections
\EN{\input{patterns/patterns_opt_dbg_EN}}
\ES{\input{patterns/patterns_opt_dbg_ES}}
\ITA{\input{patterns/patterns_opt_dbg_ITA}}
\PTBR{\input{patterns/patterns_opt_dbg_PTBR}}
\RU{\input{patterns/patterns_opt_dbg_RU}}
\THA{\input{patterns/patterns_opt_dbg_THA}}
\DE{\input{patterns/patterns_opt_dbg_DE}}
\FR{\input{patterns/patterns_opt_dbg_FR}}
\PL{\input{patterns/patterns_opt_dbg_PL}}

\RU{\section{Некоторые базовые понятия}}
\EN{\section{Some basics}}
\DE{\section{Einige Grundlagen}}
\FR{\section{Quelques bases}}
\ES{\section{\ESph{}}}
\ITA{\section{Alcune basi teoriche}}
\PTBR{\section{\PTBRph{}}}
\THA{\section{\THAph{}}}
\PL{\section{\PLph{}}}

% sections:
\EN{\input{patterns/intro_CPU_ISA_EN}}
\ES{\input{patterns/intro_CPU_ISA_ES}}
\ITA{\input{patterns/intro_CPU_ISA_ITA}}
\PTBR{\input{patterns/intro_CPU_ISA_PTBR}}
\RU{\input{patterns/intro_CPU_ISA_RU}}
\DE{\input{patterns/intro_CPU_ISA_DE}}
\FR{\input{patterns/intro_CPU_ISA_FR}}
\PL{\input{patterns/intro_CPU_ISA_PL}}

\EN{\input{patterns/numeral_EN}}
\RU{\input{patterns/numeral_RU}}
\ITA{\input{patterns/numeral_ITA}}
\DE{\input{patterns/numeral_DE}}
\FR{\input{patterns/numeral_FR}}
\PL{\input{patterns/numeral_PL}}

% chapters
\input{patterns/00_empty/main}
\input{patterns/011_ret/main}
\input{patterns/01_helloworld/main}
\input{patterns/015_prolog_epilogue/main}
\input{patterns/02_stack/main}
\input{patterns/03_printf/main}
\input{patterns/04_scanf/main}
\input{patterns/05_passing_arguments/main}
\input{patterns/06_return_results/main}
\input{patterns/061_pointers/main}
\input{patterns/065_GOTO/main}
\input{patterns/07_jcc/main}
\input{patterns/08_switch/main}
\input{patterns/09_loops/main}
\input{patterns/10_strings/main}
\input{patterns/11_arith_optimizations/main}
\input{patterns/12_FPU/main}
\input{patterns/13_arrays/main}
\input{patterns/14_bitfields/main}
\EN{\input{patterns/145_LCG/main_EN}}
\RU{\input{patterns/145_LCG/main_RU}}
\input{patterns/15_structs/main}
\input{patterns/17_unions/main}
\input{patterns/18_pointers_to_functions/main}
\input{patterns/185_64bit_in_32_env/main}

\EN{\input{patterns/19_SIMD/main_EN}}
\RU{\input{patterns/19_SIMD/main_RU}}
\DE{\input{patterns/19_SIMD/main_DE}}

\EN{\input{patterns/20_x64/main_EN}}
\RU{\input{patterns/20_x64/main_RU}}

\EN{\input{patterns/205_floating_SIMD/main_EN}}
\RU{\input{patterns/205_floating_SIMD/main_RU}}
\DE{\input{patterns/205_floating_SIMD/main_DE}}

\EN{\input{patterns/ARM/main_EN}}
\RU{\input{patterns/ARM/main_RU}}
\DE{\input{patterns/ARM/main_DE}}

\input{patterns/MIPS/main}

\ifdefined\SPANISH
\chapter{Patrones de código}
\fi % SPANISH

\ifdefined\GERMAN
\chapter{Code-Muster}
\fi % GERMAN

\ifdefined\ENGLISH
\chapter{Code Patterns}
\fi % ENGLISH

\ifdefined\ITALIAN
\chapter{Forme di codice}
\fi % ITALIAN

\ifdefined\RUSSIAN
\chapter{Образцы кода}
\fi % RUSSIAN

\ifdefined\BRAZILIAN
\chapter{Padrões de códigos}
\fi % BRAZILIAN

\ifdefined\THAI
\chapter{รูปแบบของโค้ด}
\fi % THAI

\ifdefined\FRENCH
\chapter{Modèle de code}
\fi % FRENCH

\ifdefined\POLISH
\chapter{\PLph{}}
\fi % POLISH

% sections
\EN{\input{patterns/patterns_opt_dbg_EN}}
\ES{\input{patterns/patterns_opt_dbg_ES}}
\ITA{\input{patterns/patterns_opt_dbg_ITA}}
\PTBR{\input{patterns/patterns_opt_dbg_PTBR}}
\RU{\input{patterns/patterns_opt_dbg_RU}}
\THA{\input{patterns/patterns_opt_dbg_THA}}
\DE{\input{patterns/patterns_opt_dbg_DE}}
\FR{\input{patterns/patterns_opt_dbg_FR}}
\PL{\input{patterns/patterns_opt_dbg_PL}}

\RU{\section{Некоторые базовые понятия}}
\EN{\section{Some basics}}
\DE{\section{Einige Grundlagen}}
\FR{\section{Quelques bases}}
\ES{\section{\ESph{}}}
\ITA{\section{Alcune basi teoriche}}
\PTBR{\section{\PTBRph{}}}
\THA{\section{\THAph{}}}
\PL{\section{\PLph{}}}

% sections:
\EN{\input{patterns/intro_CPU_ISA_EN}}
\ES{\input{patterns/intro_CPU_ISA_ES}}
\ITA{\input{patterns/intro_CPU_ISA_ITA}}
\PTBR{\input{patterns/intro_CPU_ISA_PTBR}}
\RU{\input{patterns/intro_CPU_ISA_RU}}
\DE{\input{patterns/intro_CPU_ISA_DE}}
\FR{\input{patterns/intro_CPU_ISA_FR}}
\PL{\input{patterns/intro_CPU_ISA_PL}}

\EN{\input{patterns/numeral_EN}}
\RU{\input{patterns/numeral_RU}}
\ITA{\input{patterns/numeral_ITA}}
\DE{\input{patterns/numeral_DE}}
\FR{\input{patterns/numeral_FR}}
\PL{\input{patterns/numeral_PL}}

% chapters
\input{patterns/00_empty/main}
\input{patterns/011_ret/main}
\input{patterns/01_helloworld/main}
\input{patterns/015_prolog_epilogue/main}
\input{patterns/02_stack/main}
\input{patterns/03_printf/main}
\input{patterns/04_scanf/main}
\input{patterns/05_passing_arguments/main}
\input{patterns/06_return_results/main}
\input{patterns/061_pointers/main}
\input{patterns/065_GOTO/main}
\input{patterns/07_jcc/main}
\input{patterns/08_switch/main}
\input{patterns/09_loops/main}
\input{patterns/10_strings/main}
\input{patterns/11_arith_optimizations/main}
\input{patterns/12_FPU/main}
\input{patterns/13_arrays/main}
\input{patterns/14_bitfields/main}
\EN{\input{patterns/145_LCG/main_EN}}
\RU{\input{patterns/145_LCG/main_RU}}
\input{patterns/15_structs/main}
\input{patterns/17_unions/main}
\input{patterns/18_pointers_to_functions/main}
\input{patterns/185_64bit_in_32_env/main}

\EN{\input{patterns/19_SIMD/main_EN}}
\RU{\input{patterns/19_SIMD/main_RU}}
\DE{\input{patterns/19_SIMD/main_DE}}

\EN{\input{patterns/20_x64/main_EN}}
\RU{\input{patterns/20_x64/main_RU}}

\EN{\input{patterns/205_floating_SIMD/main_EN}}
\RU{\input{patterns/205_floating_SIMD/main_RU}}
\DE{\input{patterns/205_floating_SIMD/main_DE}}

\EN{\input{patterns/ARM/main_EN}}
\RU{\input{patterns/ARM/main_RU}}
\DE{\input{patterns/ARM/main_DE}}

\input{patterns/MIPS/main}

\ifdefined\SPANISH
\chapter{Patrones de código}
\fi % SPANISH

\ifdefined\GERMAN
\chapter{Code-Muster}
\fi % GERMAN

\ifdefined\ENGLISH
\chapter{Code Patterns}
\fi % ENGLISH

\ifdefined\ITALIAN
\chapter{Forme di codice}
\fi % ITALIAN

\ifdefined\RUSSIAN
\chapter{Образцы кода}
\fi % RUSSIAN

\ifdefined\BRAZILIAN
\chapter{Padrões de códigos}
\fi % BRAZILIAN

\ifdefined\THAI
\chapter{รูปแบบของโค้ด}
\fi % THAI

\ifdefined\FRENCH
\chapter{Modèle de code}
\fi % FRENCH

\ifdefined\POLISH
\chapter{\PLph{}}
\fi % POLISH

% sections
\EN{\input{patterns/patterns_opt_dbg_EN}}
\ES{\input{patterns/patterns_opt_dbg_ES}}
\ITA{\input{patterns/patterns_opt_dbg_ITA}}
\PTBR{\input{patterns/patterns_opt_dbg_PTBR}}
\RU{\input{patterns/patterns_opt_dbg_RU}}
\THA{\input{patterns/patterns_opt_dbg_THA}}
\DE{\input{patterns/patterns_opt_dbg_DE}}
\FR{\input{patterns/patterns_opt_dbg_FR}}
\PL{\input{patterns/patterns_opt_dbg_PL}}

\RU{\section{Некоторые базовые понятия}}
\EN{\section{Some basics}}
\DE{\section{Einige Grundlagen}}
\FR{\section{Quelques bases}}
\ES{\section{\ESph{}}}
\ITA{\section{Alcune basi teoriche}}
\PTBR{\section{\PTBRph{}}}
\THA{\section{\THAph{}}}
\PL{\section{\PLph{}}}

% sections:
\EN{\input{patterns/intro_CPU_ISA_EN}}
\ES{\input{patterns/intro_CPU_ISA_ES}}
\ITA{\input{patterns/intro_CPU_ISA_ITA}}
\PTBR{\input{patterns/intro_CPU_ISA_PTBR}}
\RU{\input{patterns/intro_CPU_ISA_RU}}
\DE{\input{patterns/intro_CPU_ISA_DE}}
\FR{\input{patterns/intro_CPU_ISA_FR}}
\PL{\input{patterns/intro_CPU_ISA_PL}}

\EN{\input{patterns/numeral_EN}}
\RU{\input{patterns/numeral_RU}}
\ITA{\input{patterns/numeral_ITA}}
\DE{\input{patterns/numeral_DE}}
\FR{\input{patterns/numeral_FR}}
\PL{\input{patterns/numeral_PL}}

% chapters
\input{patterns/00_empty/main}
\input{patterns/011_ret/main}
\input{patterns/01_helloworld/main}
\input{patterns/015_prolog_epilogue/main}
\input{patterns/02_stack/main}
\input{patterns/03_printf/main}
\input{patterns/04_scanf/main}
\input{patterns/05_passing_arguments/main}
\input{patterns/06_return_results/main}
\input{patterns/061_pointers/main}
\input{patterns/065_GOTO/main}
\input{patterns/07_jcc/main}
\input{patterns/08_switch/main}
\input{patterns/09_loops/main}
\input{patterns/10_strings/main}
\input{patterns/11_arith_optimizations/main}
\input{patterns/12_FPU/main}
\input{patterns/13_arrays/main}
\input{patterns/14_bitfields/main}
\EN{\input{patterns/145_LCG/main_EN}}
\RU{\input{patterns/145_LCG/main_RU}}
\input{patterns/15_structs/main}
\input{patterns/17_unions/main}
\input{patterns/18_pointers_to_functions/main}
\input{patterns/185_64bit_in_32_env/main}

\EN{\input{patterns/19_SIMD/main_EN}}
\RU{\input{patterns/19_SIMD/main_RU}}
\DE{\input{patterns/19_SIMD/main_DE}}

\EN{\input{patterns/20_x64/main_EN}}
\RU{\input{patterns/20_x64/main_RU}}

\EN{\input{patterns/205_floating_SIMD/main_EN}}
\RU{\input{patterns/205_floating_SIMD/main_RU}}
\DE{\input{patterns/205_floating_SIMD/main_DE}}

\EN{\input{patterns/ARM/main_EN}}
\RU{\input{patterns/ARM/main_RU}}
\DE{\input{patterns/ARM/main_DE}}

\input{patterns/MIPS/main}

\EN{\input{patterns/12_FPU/main_EN}}
\RU{\input{patterns/12_FPU/main_RU}}
\DE{\input{patterns/12_FPU/main_DE}}
\FR{\input{patterns/12_FPU/main_FR}}


\ifdefined\SPANISH
\chapter{Patrones de código}
\fi % SPANISH

\ifdefined\GERMAN
\chapter{Code-Muster}
\fi % GERMAN

\ifdefined\ENGLISH
\chapter{Code Patterns}
\fi % ENGLISH

\ifdefined\ITALIAN
\chapter{Forme di codice}
\fi % ITALIAN

\ifdefined\RUSSIAN
\chapter{Образцы кода}
\fi % RUSSIAN

\ifdefined\BRAZILIAN
\chapter{Padrões de códigos}
\fi % BRAZILIAN

\ifdefined\THAI
\chapter{รูปแบบของโค้ด}
\fi % THAI

\ifdefined\FRENCH
\chapter{Modèle de code}
\fi % FRENCH

\ifdefined\POLISH
\chapter{\PLph{}}
\fi % POLISH

% sections
\EN{\input{patterns/patterns_opt_dbg_EN}}
\ES{\input{patterns/patterns_opt_dbg_ES}}
\ITA{\input{patterns/patterns_opt_dbg_ITA}}
\PTBR{\input{patterns/patterns_opt_dbg_PTBR}}
\RU{\input{patterns/patterns_opt_dbg_RU}}
\THA{\input{patterns/patterns_opt_dbg_THA}}
\DE{\input{patterns/patterns_opt_dbg_DE}}
\FR{\input{patterns/patterns_opt_dbg_FR}}
\PL{\input{patterns/patterns_opt_dbg_PL}}

\RU{\section{Некоторые базовые понятия}}
\EN{\section{Some basics}}
\DE{\section{Einige Grundlagen}}
\FR{\section{Quelques bases}}
\ES{\section{\ESph{}}}
\ITA{\section{Alcune basi teoriche}}
\PTBR{\section{\PTBRph{}}}
\THA{\section{\THAph{}}}
\PL{\section{\PLph{}}}

% sections:
\EN{\input{patterns/intro_CPU_ISA_EN}}
\ES{\input{patterns/intro_CPU_ISA_ES}}
\ITA{\input{patterns/intro_CPU_ISA_ITA}}
\PTBR{\input{patterns/intro_CPU_ISA_PTBR}}
\RU{\input{patterns/intro_CPU_ISA_RU}}
\DE{\input{patterns/intro_CPU_ISA_DE}}
\FR{\input{patterns/intro_CPU_ISA_FR}}
\PL{\input{patterns/intro_CPU_ISA_PL}}

\EN{\input{patterns/numeral_EN}}
\RU{\input{patterns/numeral_RU}}
\ITA{\input{patterns/numeral_ITA}}
\DE{\input{patterns/numeral_DE}}
\FR{\input{patterns/numeral_FR}}
\PL{\input{patterns/numeral_PL}}

% chapters
\input{patterns/00_empty/main}
\input{patterns/011_ret/main}
\input{patterns/01_helloworld/main}
\input{patterns/015_prolog_epilogue/main}
\input{patterns/02_stack/main}
\input{patterns/03_printf/main}
\input{patterns/04_scanf/main}
\input{patterns/05_passing_arguments/main}
\input{patterns/06_return_results/main}
\input{patterns/061_pointers/main}
\input{patterns/065_GOTO/main}
\input{patterns/07_jcc/main}
\input{patterns/08_switch/main}
\input{patterns/09_loops/main}
\input{patterns/10_strings/main}
\input{patterns/11_arith_optimizations/main}
\input{patterns/12_FPU/main}
\input{patterns/13_arrays/main}
\input{patterns/14_bitfields/main}
\EN{\input{patterns/145_LCG/main_EN}}
\RU{\input{patterns/145_LCG/main_RU}}
\input{patterns/15_structs/main}
\input{patterns/17_unions/main}
\input{patterns/18_pointers_to_functions/main}
\input{patterns/185_64bit_in_32_env/main}

\EN{\input{patterns/19_SIMD/main_EN}}
\RU{\input{patterns/19_SIMD/main_RU}}
\DE{\input{patterns/19_SIMD/main_DE}}

\EN{\input{patterns/20_x64/main_EN}}
\RU{\input{patterns/20_x64/main_RU}}

\EN{\input{patterns/205_floating_SIMD/main_EN}}
\RU{\input{patterns/205_floating_SIMD/main_RU}}
\DE{\input{patterns/205_floating_SIMD/main_DE}}

\EN{\input{patterns/ARM/main_EN}}
\RU{\input{patterns/ARM/main_RU}}
\DE{\input{patterns/ARM/main_DE}}

\input{patterns/MIPS/main}

\ifdefined\SPANISH
\chapter{Patrones de código}
\fi % SPANISH

\ifdefined\GERMAN
\chapter{Code-Muster}
\fi % GERMAN

\ifdefined\ENGLISH
\chapter{Code Patterns}
\fi % ENGLISH

\ifdefined\ITALIAN
\chapter{Forme di codice}
\fi % ITALIAN

\ifdefined\RUSSIAN
\chapter{Образцы кода}
\fi % RUSSIAN

\ifdefined\BRAZILIAN
\chapter{Padrões de códigos}
\fi % BRAZILIAN

\ifdefined\THAI
\chapter{รูปแบบของโค้ด}
\fi % THAI

\ifdefined\FRENCH
\chapter{Modèle de code}
\fi % FRENCH

\ifdefined\POLISH
\chapter{\PLph{}}
\fi % POLISH

% sections
\EN{\input{patterns/patterns_opt_dbg_EN}}
\ES{\input{patterns/patterns_opt_dbg_ES}}
\ITA{\input{patterns/patterns_opt_dbg_ITA}}
\PTBR{\input{patterns/patterns_opt_dbg_PTBR}}
\RU{\input{patterns/patterns_opt_dbg_RU}}
\THA{\input{patterns/patterns_opt_dbg_THA}}
\DE{\input{patterns/patterns_opt_dbg_DE}}
\FR{\input{patterns/patterns_opt_dbg_FR}}
\PL{\input{patterns/patterns_opt_dbg_PL}}

\RU{\section{Некоторые базовые понятия}}
\EN{\section{Some basics}}
\DE{\section{Einige Grundlagen}}
\FR{\section{Quelques bases}}
\ES{\section{\ESph{}}}
\ITA{\section{Alcune basi teoriche}}
\PTBR{\section{\PTBRph{}}}
\THA{\section{\THAph{}}}
\PL{\section{\PLph{}}}

% sections:
\EN{\input{patterns/intro_CPU_ISA_EN}}
\ES{\input{patterns/intro_CPU_ISA_ES}}
\ITA{\input{patterns/intro_CPU_ISA_ITA}}
\PTBR{\input{patterns/intro_CPU_ISA_PTBR}}
\RU{\input{patterns/intro_CPU_ISA_RU}}
\DE{\input{patterns/intro_CPU_ISA_DE}}
\FR{\input{patterns/intro_CPU_ISA_FR}}
\PL{\input{patterns/intro_CPU_ISA_PL}}

\EN{\input{patterns/numeral_EN}}
\RU{\input{patterns/numeral_RU}}
\ITA{\input{patterns/numeral_ITA}}
\DE{\input{patterns/numeral_DE}}
\FR{\input{patterns/numeral_FR}}
\PL{\input{patterns/numeral_PL}}

% chapters
\input{patterns/00_empty/main}
\input{patterns/011_ret/main}
\input{patterns/01_helloworld/main}
\input{patterns/015_prolog_epilogue/main}
\input{patterns/02_stack/main}
\input{patterns/03_printf/main}
\input{patterns/04_scanf/main}
\input{patterns/05_passing_arguments/main}
\input{patterns/06_return_results/main}
\input{patterns/061_pointers/main}
\input{patterns/065_GOTO/main}
\input{patterns/07_jcc/main}
\input{patterns/08_switch/main}
\input{patterns/09_loops/main}
\input{patterns/10_strings/main}
\input{patterns/11_arith_optimizations/main}
\input{patterns/12_FPU/main}
\input{patterns/13_arrays/main}
\input{patterns/14_bitfields/main}
\EN{\input{patterns/145_LCG/main_EN}}
\RU{\input{patterns/145_LCG/main_RU}}
\input{patterns/15_structs/main}
\input{patterns/17_unions/main}
\input{patterns/18_pointers_to_functions/main}
\input{patterns/185_64bit_in_32_env/main}

\EN{\input{patterns/19_SIMD/main_EN}}
\RU{\input{patterns/19_SIMD/main_RU}}
\DE{\input{patterns/19_SIMD/main_DE}}

\EN{\input{patterns/20_x64/main_EN}}
\RU{\input{patterns/20_x64/main_RU}}

\EN{\input{patterns/205_floating_SIMD/main_EN}}
\RU{\input{patterns/205_floating_SIMD/main_RU}}
\DE{\input{patterns/205_floating_SIMD/main_DE}}

\EN{\input{patterns/ARM/main_EN}}
\RU{\input{patterns/ARM/main_RU}}
\DE{\input{patterns/ARM/main_DE}}

\input{patterns/MIPS/main}

\EN{\section{Returning Values}
\label{ret_val_func}

Another simple function is the one that simply returns a constant value:

\lstinputlisting[caption=\EN{\CCpp Code},style=customc]{patterns/011_ret/1.c}

Let's compile it.

\subsection{x86}

Here's what both the GCC and MSVC compilers produce (with optimization) on the x86 platform:

\lstinputlisting[caption=\Optimizing GCC/MSVC (\assemblyOutput),style=customasmx86]{patterns/011_ret/1.s}

\myindex{x86!\Instructions!RET}
There are just two instructions: the first places the value 123 into the \EAX register,
which is used by convention for storing the return
value, and the second one is \RET, which returns execution to the \gls{caller}.

The caller will take the result from the \EAX register.

\subsection{ARM}

There are a few differences on the ARM platform:

\lstinputlisting[caption=\OptimizingKeilVI (\ARMMode) ASM Output,style=customasmARM]{patterns/011_ret/1_Keil_ARM_O3.s}

ARM uses the register \Reg{0} for returning the results of functions, so 123 is copied into \Reg{0}.

\myindex{ARM!\Instructions!MOV}
\myindex{x86!\Instructions!MOV}
It is worth noting that \MOV is a misleading name for the instruction in both the x86 and ARM \ac{ISA}s.

The data is not in fact \IT{moved}, but \IT{copied}.

\subsection{MIPS}

\label{MIPS_leaf_function_ex1}

The GCC assembly output below lists registers by number:

\lstinputlisting[caption=\Optimizing GCC 4.4.5 (\assemblyOutput),style=customasmMIPS]{patterns/011_ret/MIPS.s}

\dots while \IDA does it by their pseudo names:

\lstinputlisting[caption=\Optimizing GCC 4.4.5 (IDA),style=customasmMIPS]{patterns/011_ret/MIPS_IDA.lst}

The \$2 (or \$V0) register is used to store the function's return value.
\myindex{MIPS!\Pseudoinstructions!LI}
\INS{LI} stands for ``Load Immediate'' and is the MIPS equivalent to \MOV.

\myindex{MIPS!\Instructions!J}
The other instruction is the jump instruction (J or JR) which returns the execution flow to the \gls{caller}.

\myindex{MIPS!Branch delay slot}
You might be wondering why the positions of the load instruction (LI) and the jump instruction (J or JR) are swapped. This is due to a \ac{RISC} feature called ``branch delay slot''.

The reason this happens is a quirk in the architecture of some RISC \ac{ISA}s and isn't important for our
purposes---we must simply keep in mind that in MIPS, the instruction following a jump or branch instruction
is executed \IT{before} the jump/branch instruction itself.

As a consequence, branch instructions always swap places with the instruction executed immediately beforehand.


In practice, functions which merely return 1 (\IT{true}) or 0 (\IT{false}) are very frequent.

The smallest ever of the standard UNIX utilities, \IT{/bin/true} and \IT{/bin/false} return 0 and 1 respectively, as an exit code.
(Zero as an exit code usually means success, non-zero means error.)
}
\RU{\subsubsection{std::string}
\myindex{\Cpp!STL!std::string}
\label{std_string}

\myparagraph{Как устроена структура}

Многие строковые библиотеки \InSqBrackets{\CNotes 2.2} обеспечивают структуру содержащую ссылку 
на буфер собственно со строкой, переменная всегда содержащую длину строки 
(что очень удобно для массы функций \InSqBrackets{\CNotes 2.2.1}) и переменную содержащую текущий размер буфера.

Строка в буфере обыкновенно оканчивается нулем: это для того чтобы указатель на буфер можно было
передавать в функции требующие на вход обычную сишную \ac{ASCIIZ}-строку.

Стандарт \Cpp не описывает, как именно нужно реализовывать std::string,
но, как правило, они реализованы как описано выше, с небольшими дополнениями.

Строки в \Cpp это не класс (как, например, QString в Qt), а темплейт (basic\_string), 
это сделано для того чтобы поддерживать 
строки содержащие разного типа символы: как минимум \Tchar и \IT{wchar\_t}.

Так что, std::string это класс с базовым типом \Tchar.

А std::wstring это класс с базовым типом \IT{wchar\_t}.

\mysubparagraph{MSVC}

В реализации MSVC, вместо ссылки на буфер может содержаться сам буфер (если строка короче 16-и символов).

Это означает, что каждая короткая строка будет занимать в памяти по крайней мере $16 + 4 + 4 = 24$ 
байт для 32-битной среды либо $16 + 8 + 8 = 32$ 
байта в 64-битной, а если строка длиннее 16-и символов, то прибавьте еще длину самой строки.

\lstinputlisting[caption=пример для MSVC,style=customc]{\CURPATH/STL/string/MSVC_RU.cpp}

Собственно, из этого исходника почти всё ясно.

Несколько замечаний:

Если строка короче 16-и символов, 
то отдельный буфер для строки в \glslink{heap}{куче} выделяться не будет.

Это удобно потому что на практике, основная часть строк действительно короткие.
Вероятно, разработчики в Microsoft выбрали размер в 16 символов как разумный баланс.

Теперь очень важный момент в конце функции main(): мы не пользуемся методом c\_str(), тем не менее,
если это скомпилировать и запустить, то обе строки появятся в консоли!

Работает это вот почему.

В первом случае строка короче 16-и символов и в начале объекта std::string (его можно рассматривать
просто как структуру) расположен буфер с этой строкой.
\printf трактует указатель как указатель на массив символов оканчивающийся нулем и поэтому всё работает.

Вывод второй строки (длиннее 16-и символов) даже еще опаснее: это вообще типичная программистская ошибка 
(или опечатка), забыть дописать c\_str().
Это работает потому что в это время в начале структуры расположен указатель на буфер.
Это может надолго остаться незамеченным: до тех пока там не появится строка 
короче 16-и символов, тогда процесс упадет.

\mysubparagraph{GCC}

В реализации GCC в структуре есть еще одна переменная --- reference count.

Интересно, что указатель на экземпляр класса std::string в GCC указывает не на начало самой структуры, 
а на указатель на буфера.
В libstdc++-v3\textbackslash{}include\textbackslash{}bits\textbackslash{}basic\_string.h 
мы можем прочитать что это сделано для удобства отладки:

\begin{lstlisting}
   *  The reason you want _M_data pointing to the character %array and
   *  not the _Rep is so that the debugger can see the string
   *  contents. (Probably we should add a non-inline member to get
   *  the _Rep for the debugger to use, so users can check the actual
   *  string length.)
\end{lstlisting}

\href{http://go.yurichev.com/17085}{исходный код basic\_string.h}

В нашем примере мы учитываем это:

\lstinputlisting[caption=пример для GCC,style=customc]{\CURPATH/STL/string/GCC_RU.cpp}

Нужны еще небольшие хаки чтобы сымитировать типичную ошибку, которую мы уже видели выше, из-за
более ужесточенной проверки типов в GCC, тем не менее, printf() работает и здесь без c\_str().

\myparagraph{Чуть более сложный пример}

\lstinputlisting[style=customc]{\CURPATH/STL/string/3.cpp}

\lstinputlisting[caption=MSVC 2012,style=customasmx86]{\CURPATH/STL/string/3_MSVC_RU.asm}

Собственно, компилятор не конструирует строки статически: да в общем-то и как
это возможно, если буфер с ней нужно хранить в \glslink{heap}{куче}?

Вместо этого в сегменте данных хранятся обычные \ac{ASCIIZ}-строки, а позже, во время выполнения, 
при помощи метода \q{assign}, конструируются строки s1 и s2
.
При помощи \TT{operator+}, создается строка s3.

Обратите внимание на то что вызов метода c\_str() отсутствует,
потому что его код достаточно короткий и компилятор вставил его прямо здесь:
если строка короче 16-и байт, то в регистре EAX остается указатель на буфер,
а если длиннее, то из этого же места достается адрес на буфер расположенный в \glslink{heap}{куче}.

Далее следуют вызовы трех деструкторов, причем, они вызываются только если строка длиннее 16-и байт:
тогда нужно освободить буфера в \glslink{heap}{куче}.
В противном случае, так как все три объекта std::string хранятся в стеке,
они освобождаются автоматически после выхода из функции.

Следовательно, работа с короткими строками более быстрая из-за м\'{е}ньшего обращения к \glslink{heap}{куче}.

Код на GCC даже проще (из-за того, что в GCC, как мы уже видели, не реализована возможность хранить короткую
строку прямо в структуре):

% TODO1 comment each function meaning
\lstinputlisting[caption=GCC 4.8.1,style=customasmx86]{\CURPATH/STL/string/3_GCC_RU.s}

Можно заметить, что в деструкторы передается не указатель на объект,
а указатель на место за 12 байт (или 3 слова) перед ним, то есть, на настоящее начало структуры.

\myparagraph{std::string как глобальная переменная}
\label{sec:std_string_as_global_variable}

Опытные программисты на \Cpp знают, что глобальные переменные \ac{STL}-типов вполне можно объявлять.

Да, действительно:

\lstinputlisting[style=customc]{\CURPATH/STL/string/5.cpp}

Но как и где будет вызываться конструктор \TT{std::string}?

На самом деле, эта переменная будет инициализирована даже перед началом \main.

\lstinputlisting[caption=MSVC 2012: здесь конструируется глобальная переменная{,} а также регистрируется её деструктор,style=customasmx86]{\CURPATH/STL/string/5_MSVC_p2.asm}

\lstinputlisting[caption=MSVC 2012: здесь глобальная переменная используется в \main,style=customasmx86]{\CURPATH/STL/string/5_MSVC_p1.asm}

\lstinputlisting[caption=MSVC 2012: эта функция-деструктор вызывается перед выходом,style=customasmx86]{\CURPATH/STL/string/5_MSVC_p3.asm}

\myindex{\CStandardLibrary!atexit()}
В реальности, из \ac{CRT}, еще до вызова main(), вызывается специальная функция,
в которой перечислены все конструкторы подобных переменных.
Более того: при помощи atexit() регистрируется функция, которая будет вызвана в конце работы программы:
в этой функции компилятор собирает вызовы деструкторов всех подобных глобальных переменных.

GCC работает похожим образом:

\lstinputlisting[caption=GCC 4.8.1,style=customasmx86]{\CURPATH/STL/string/5_GCC.s}

Но он не выделяет отдельной функции в которой будут собраны деструкторы: 
каждый деструктор передается в atexit() по одному.

% TODO а если глобальная STL-переменная в другом модуле? надо проверить.

}
\ifdefined\SPANISH
\chapter{Patrones de código}
\fi % SPANISH

\ifdefined\GERMAN
\chapter{Code-Muster}
\fi % GERMAN

\ifdefined\ENGLISH
\chapter{Code Patterns}
\fi % ENGLISH

\ifdefined\ITALIAN
\chapter{Forme di codice}
\fi % ITALIAN

\ifdefined\RUSSIAN
\chapter{Образцы кода}
\fi % RUSSIAN

\ifdefined\BRAZILIAN
\chapter{Padrões de códigos}
\fi % BRAZILIAN

\ifdefined\THAI
\chapter{รูปแบบของโค้ด}
\fi % THAI

\ifdefined\FRENCH
\chapter{Modèle de code}
\fi % FRENCH

\ifdefined\POLISH
\chapter{\PLph{}}
\fi % POLISH

% sections
\EN{\input{patterns/patterns_opt_dbg_EN}}
\ES{\input{patterns/patterns_opt_dbg_ES}}
\ITA{\input{patterns/patterns_opt_dbg_ITA}}
\PTBR{\input{patterns/patterns_opt_dbg_PTBR}}
\RU{\input{patterns/patterns_opt_dbg_RU}}
\THA{\input{patterns/patterns_opt_dbg_THA}}
\DE{\input{patterns/patterns_opt_dbg_DE}}
\FR{\input{patterns/patterns_opt_dbg_FR}}
\PL{\input{patterns/patterns_opt_dbg_PL}}

\RU{\section{Некоторые базовые понятия}}
\EN{\section{Some basics}}
\DE{\section{Einige Grundlagen}}
\FR{\section{Quelques bases}}
\ES{\section{\ESph{}}}
\ITA{\section{Alcune basi teoriche}}
\PTBR{\section{\PTBRph{}}}
\THA{\section{\THAph{}}}
\PL{\section{\PLph{}}}

% sections:
\EN{\input{patterns/intro_CPU_ISA_EN}}
\ES{\input{patterns/intro_CPU_ISA_ES}}
\ITA{\input{patterns/intro_CPU_ISA_ITA}}
\PTBR{\input{patterns/intro_CPU_ISA_PTBR}}
\RU{\input{patterns/intro_CPU_ISA_RU}}
\DE{\input{patterns/intro_CPU_ISA_DE}}
\FR{\input{patterns/intro_CPU_ISA_FR}}
\PL{\input{patterns/intro_CPU_ISA_PL}}

\EN{\input{patterns/numeral_EN}}
\RU{\input{patterns/numeral_RU}}
\ITA{\input{patterns/numeral_ITA}}
\DE{\input{patterns/numeral_DE}}
\FR{\input{patterns/numeral_FR}}
\PL{\input{patterns/numeral_PL}}

% chapters
\input{patterns/00_empty/main}
\input{patterns/011_ret/main}
\input{patterns/01_helloworld/main}
\input{patterns/015_prolog_epilogue/main}
\input{patterns/02_stack/main}
\input{patterns/03_printf/main}
\input{patterns/04_scanf/main}
\input{patterns/05_passing_arguments/main}
\input{patterns/06_return_results/main}
\input{patterns/061_pointers/main}
\input{patterns/065_GOTO/main}
\input{patterns/07_jcc/main}
\input{patterns/08_switch/main}
\input{patterns/09_loops/main}
\input{patterns/10_strings/main}
\input{patterns/11_arith_optimizations/main}
\input{patterns/12_FPU/main}
\input{patterns/13_arrays/main}
\input{patterns/14_bitfields/main}
\EN{\input{patterns/145_LCG/main_EN}}
\RU{\input{patterns/145_LCG/main_RU}}
\input{patterns/15_structs/main}
\input{patterns/17_unions/main}
\input{patterns/18_pointers_to_functions/main}
\input{patterns/185_64bit_in_32_env/main}

\EN{\input{patterns/19_SIMD/main_EN}}
\RU{\input{patterns/19_SIMD/main_RU}}
\DE{\input{patterns/19_SIMD/main_DE}}

\EN{\input{patterns/20_x64/main_EN}}
\RU{\input{patterns/20_x64/main_RU}}

\EN{\input{patterns/205_floating_SIMD/main_EN}}
\RU{\input{patterns/205_floating_SIMD/main_RU}}
\DE{\input{patterns/205_floating_SIMD/main_DE}}

\EN{\input{patterns/ARM/main_EN}}
\RU{\input{patterns/ARM/main_RU}}
\DE{\input{patterns/ARM/main_DE}}

\input{patterns/MIPS/main}

\ifdefined\SPANISH
\chapter{Patrones de código}
\fi % SPANISH

\ifdefined\GERMAN
\chapter{Code-Muster}
\fi % GERMAN

\ifdefined\ENGLISH
\chapter{Code Patterns}
\fi % ENGLISH

\ifdefined\ITALIAN
\chapter{Forme di codice}
\fi % ITALIAN

\ifdefined\RUSSIAN
\chapter{Образцы кода}
\fi % RUSSIAN

\ifdefined\BRAZILIAN
\chapter{Padrões de códigos}
\fi % BRAZILIAN

\ifdefined\THAI
\chapter{รูปแบบของโค้ด}
\fi % THAI

\ifdefined\FRENCH
\chapter{Modèle de code}
\fi % FRENCH

\ifdefined\POLISH
\chapter{\PLph{}}
\fi % POLISH

% sections
\EN{\input{patterns/patterns_opt_dbg_EN}}
\ES{\input{patterns/patterns_opt_dbg_ES}}
\ITA{\input{patterns/patterns_opt_dbg_ITA}}
\PTBR{\input{patterns/patterns_opt_dbg_PTBR}}
\RU{\input{patterns/patterns_opt_dbg_RU}}
\THA{\input{patterns/patterns_opt_dbg_THA}}
\DE{\input{patterns/patterns_opt_dbg_DE}}
\FR{\input{patterns/patterns_opt_dbg_FR}}
\PL{\input{patterns/patterns_opt_dbg_PL}}

\RU{\section{Некоторые базовые понятия}}
\EN{\section{Some basics}}
\DE{\section{Einige Grundlagen}}
\FR{\section{Quelques bases}}
\ES{\section{\ESph{}}}
\ITA{\section{Alcune basi teoriche}}
\PTBR{\section{\PTBRph{}}}
\THA{\section{\THAph{}}}
\PL{\section{\PLph{}}}

% sections:
\EN{\input{patterns/intro_CPU_ISA_EN}}
\ES{\input{patterns/intro_CPU_ISA_ES}}
\ITA{\input{patterns/intro_CPU_ISA_ITA}}
\PTBR{\input{patterns/intro_CPU_ISA_PTBR}}
\RU{\input{patterns/intro_CPU_ISA_RU}}
\DE{\input{patterns/intro_CPU_ISA_DE}}
\FR{\input{patterns/intro_CPU_ISA_FR}}
\PL{\input{patterns/intro_CPU_ISA_PL}}

\EN{\input{patterns/numeral_EN}}
\RU{\input{patterns/numeral_RU}}
\ITA{\input{patterns/numeral_ITA}}
\DE{\input{patterns/numeral_DE}}
\FR{\input{patterns/numeral_FR}}
\PL{\input{patterns/numeral_PL}}

% chapters
\input{patterns/00_empty/main}
\input{patterns/011_ret/main}
\input{patterns/01_helloworld/main}
\input{patterns/015_prolog_epilogue/main}
\input{patterns/02_stack/main}
\input{patterns/03_printf/main}
\input{patterns/04_scanf/main}
\input{patterns/05_passing_arguments/main}
\input{patterns/06_return_results/main}
\input{patterns/061_pointers/main}
\input{patterns/065_GOTO/main}
\input{patterns/07_jcc/main}
\input{patterns/08_switch/main}
\input{patterns/09_loops/main}
\input{patterns/10_strings/main}
\input{patterns/11_arith_optimizations/main}
\input{patterns/12_FPU/main}
\input{patterns/13_arrays/main}
\input{patterns/14_bitfields/main}
\EN{\input{patterns/145_LCG/main_EN}}
\RU{\input{patterns/145_LCG/main_RU}}
\input{patterns/15_structs/main}
\input{patterns/17_unions/main}
\input{patterns/18_pointers_to_functions/main}
\input{patterns/185_64bit_in_32_env/main}

\EN{\input{patterns/19_SIMD/main_EN}}
\RU{\input{patterns/19_SIMD/main_RU}}
\DE{\input{patterns/19_SIMD/main_DE}}

\EN{\input{patterns/20_x64/main_EN}}
\RU{\input{patterns/20_x64/main_RU}}

\EN{\input{patterns/205_floating_SIMD/main_EN}}
\RU{\input{patterns/205_floating_SIMD/main_RU}}
\DE{\input{patterns/205_floating_SIMD/main_DE}}

\EN{\input{patterns/ARM/main_EN}}
\RU{\input{patterns/ARM/main_RU}}
\DE{\input{patterns/ARM/main_DE}}

\input{patterns/MIPS/main}

\ifdefined\SPANISH
\chapter{Patrones de código}
\fi % SPANISH

\ifdefined\GERMAN
\chapter{Code-Muster}
\fi % GERMAN

\ifdefined\ENGLISH
\chapter{Code Patterns}
\fi % ENGLISH

\ifdefined\ITALIAN
\chapter{Forme di codice}
\fi % ITALIAN

\ifdefined\RUSSIAN
\chapter{Образцы кода}
\fi % RUSSIAN

\ifdefined\BRAZILIAN
\chapter{Padrões de códigos}
\fi % BRAZILIAN

\ifdefined\THAI
\chapter{รูปแบบของโค้ด}
\fi % THAI

\ifdefined\FRENCH
\chapter{Modèle de code}
\fi % FRENCH

\ifdefined\POLISH
\chapter{\PLph{}}
\fi % POLISH

% sections
\EN{\input{patterns/patterns_opt_dbg_EN}}
\ES{\input{patterns/patterns_opt_dbg_ES}}
\ITA{\input{patterns/patterns_opt_dbg_ITA}}
\PTBR{\input{patterns/patterns_opt_dbg_PTBR}}
\RU{\input{patterns/patterns_opt_dbg_RU}}
\THA{\input{patterns/patterns_opt_dbg_THA}}
\DE{\input{patterns/patterns_opt_dbg_DE}}
\FR{\input{patterns/patterns_opt_dbg_FR}}
\PL{\input{patterns/patterns_opt_dbg_PL}}

\RU{\section{Некоторые базовые понятия}}
\EN{\section{Some basics}}
\DE{\section{Einige Grundlagen}}
\FR{\section{Quelques bases}}
\ES{\section{\ESph{}}}
\ITA{\section{Alcune basi teoriche}}
\PTBR{\section{\PTBRph{}}}
\THA{\section{\THAph{}}}
\PL{\section{\PLph{}}}

% sections:
\EN{\input{patterns/intro_CPU_ISA_EN}}
\ES{\input{patterns/intro_CPU_ISA_ES}}
\ITA{\input{patterns/intro_CPU_ISA_ITA}}
\PTBR{\input{patterns/intro_CPU_ISA_PTBR}}
\RU{\input{patterns/intro_CPU_ISA_RU}}
\DE{\input{patterns/intro_CPU_ISA_DE}}
\FR{\input{patterns/intro_CPU_ISA_FR}}
\PL{\input{patterns/intro_CPU_ISA_PL}}

\EN{\input{patterns/numeral_EN}}
\RU{\input{patterns/numeral_RU}}
\ITA{\input{patterns/numeral_ITA}}
\DE{\input{patterns/numeral_DE}}
\FR{\input{patterns/numeral_FR}}
\PL{\input{patterns/numeral_PL}}

% chapters
\input{patterns/00_empty/main}
\input{patterns/011_ret/main}
\input{patterns/01_helloworld/main}
\input{patterns/015_prolog_epilogue/main}
\input{patterns/02_stack/main}
\input{patterns/03_printf/main}
\input{patterns/04_scanf/main}
\input{patterns/05_passing_arguments/main}
\input{patterns/06_return_results/main}
\input{patterns/061_pointers/main}
\input{patterns/065_GOTO/main}
\input{patterns/07_jcc/main}
\input{patterns/08_switch/main}
\input{patterns/09_loops/main}
\input{patterns/10_strings/main}
\input{patterns/11_arith_optimizations/main}
\input{patterns/12_FPU/main}
\input{patterns/13_arrays/main}
\input{patterns/14_bitfields/main}
\EN{\input{patterns/145_LCG/main_EN}}
\RU{\input{patterns/145_LCG/main_RU}}
\input{patterns/15_structs/main}
\input{patterns/17_unions/main}
\input{patterns/18_pointers_to_functions/main}
\input{patterns/185_64bit_in_32_env/main}

\EN{\input{patterns/19_SIMD/main_EN}}
\RU{\input{patterns/19_SIMD/main_RU}}
\DE{\input{patterns/19_SIMD/main_DE}}

\EN{\input{patterns/20_x64/main_EN}}
\RU{\input{patterns/20_x64/main_RU}}

\EN{\input{patterns/205_floating_SIMD/main_EN}}
\RU{\input{patterns/205_floating_SIMD/main_RU}}
\DE{\input{patterns/205_floating_SIMD/main_DE}}

\EN{\input{patterns/ARM/main_EN}}
\RU{\input{patterns/ARM/main_RU}}
\DE{\input{patterns/ARM/main_DE}}

\input{patterns/MIPS/main}

\ifdefined\SPANISH
\chapter{Patrones de código}
\fi % SPANISH

\ifdefined\GERMAN
\chapter{Code-Muster}
\fi % GERMAN

\ifdefined\ENGLISH
\chapter{Code Patterns}
\fi % ENGLISH

\ifdefined\ITALIAN
\chapter{Forme di codice}
\fi % ITALIAN

\ifdefined\RUSSIAN
\chapter{Образцы кода}
\fi % RUSSIAN

\ifdefined\BRAZILIAN
\chapter{Padrões de códigos}
\fi % BRAZILIAN

\ifdefined\THAI
\chapter{รูปแบบของโค้ด}
\fi % THAI

\ifdefined\FRENCH
\chapter{Modèle de code}
\fi % FRENCH

\ifdefined\POLISH
\chapter{\PLph{}}
\fi % POLISH

% sections
\EN{\input{patterns/patterns_opt_dbg_EN}}
\ES{\input{patterns/patterns_opt_dbg_ES}}
\ITA{\input{patterns/patterns_opt_dbg_ITA}}
\PTBR{\input{patterns/patterns_opt_dbg_PTBR}}
\RU{\input{patterns/patterns_opt_dbg_RU}}
\THA{\input{patterns/patterns_opt_dbg_THA}}
\DE{\input{patterns/patterns_opt_dbg_DE}}
\FR{\input{patterns/patterns_opt_dbg_FR}}
\PL{\input{patterns/patterns_opt_dbg_PL}}

\RU{\section{Некоторые базовые понятия}}
\EN{\section{Some basics}}
\DE{\section{Einige Grundlagen}}
\FR{\section{Quelques bases}}
\ES{\section{\ESph{}}}
\ITA{\section{Alcune basi teoriche}}
\PTBR{\section{\PTBRph{}}}
\THA{\section{\THAph{}}}
\PL{\section{\PLph{}}}

% sections:
\EN{\input{patterns/intro_CPU_ISA_EN}}
\ES{\input{patterns/intro_CPU_ISA_ES}}
\ITA{\input{patterns/intro_CPU_ISA_ITA}}
\PTBR{\input{patterns/intro_CPU_ISA_PTBR}}
\RU{\input{patterns/intro_CPU_ISA_RU}}
\DE{\input{patterns/intro_CPU_ISA_DE}}
\FR{\input{patterns/intro_CPU_ISA_FR}}
\PL{\input{patterns/intro_CPU_ISA_PL}}

\EN{\input{patterns/numeral_EN}}
\RU{\input{patterns/numeral_RU}}
\ITA{\input{patterns/numeral_ITA}}
\DE{\input{patterns/numeral_DE}}
\FR{\input{patterns/numeral_FR}}
\PL{\input{patterns/numeral_PL}}

% chapters
\input{patterns/00_empty/main}
\input{patterns/011_ret/main}
\input{patterns/01_helloworld/main}
\input{patterns/015_prolog_epilogue/main}
\input{patterns/02_stack/main}
\input{patterns/03_printf/main}
\input{patterns/04_scanf/main}
\input{patterns/05_passing_arguments/main}
\input{patterns/06_return_results/main}
\input{patterns/061_pointers/main}
\input{patterns/065_GOTO/main}
\input{patterns/07_jcc/main}
\input{patterns/08_switch/main}
\input{patterns/09_loops/main}
\input{patterns/10_strings/main}
\input{patterns/11_arith_optimizations/main}
\input{patterns/12_FPU/main}
\input{patterns/13_arrays/main}
\input{patterns/14_bitfields/main}
\EN{\input{patterns/145_LCG/main_EN}}
\RU{\input{patterns/145_LCG/main_RU}}
\input{patterns/15_structs/main}
\input{patterns/17_unions/main}
\input{patterns/18_pointers_to_functions/main}
\input{patterns/185_64bit_in_32_env/main}

\EN{\input{patterns/19_SIMD/main_EN}}
\RU{\input{patterns/19_SIMD/main_RU}}
\DE{\input{patterns/19_SIMD/main_DE}}

\EN{\input{patterns/20_x64/main_EN}}
\RU{\input{patterns/20_x64/main_RU}}

\EN{\input{patterns/205_floating_SIMD/main_EN}}
\RU{\input{patterns/205_floating_SIMD/main_RU}}
\DE{\input{patterns/205_floating_SIMD/main_DE}}

\EN{\input{patterns/ARM/main_EN}}
\RU{\input{patterns/ARM/main_RU}}
\DE{\input{patterns/ARM/main_DE}}

\input{patterns/MIPS/main}


\EN{\section{Returning Values}
\label{ret_val_func}

Another simple function is the one that simply returns a constant value:

\lstinputlisting[caption=\EN{\CCpp Code},style=customc]{patterns/011_ret/1.c}

Let's compile it.

\subsection{x86}

Here's what both the GCC and MSVC compilers produce (with optimization) on the x86 platform:

\lstinputlisting[caption=\Optimizing GCC/MSVC (\assemblyOutput),style=customasmx86]{patterns/011_ret/1.s}

\myindex{x86!\Instructions!RET}
There are just two instructions: the first places the value 123 into the \EAX register,
which is used by convention for storing the return
value, and the second one is \RET, which returns execution to the \gls{caller}.

The caller will take the result from the \EAX register.

\subsection{ARM}

There are a few differences on the ARM platform:

\lstinputlisting[caption=\OptimizingKeilVI (\ARMMode) ASM Output,style=customasmARM]{patterns/011_ret/1_Keil_ARM_O3.s}

ARM uses the register \Reg{0} for returning the results of functions, so 123 is copied into \Reg{0}.

\myindex{ARM!\Instructions!MOV}
\myindex{x86!\Instructions!MOV}
It is worth noting that \MOV is a misleading name for the instruction in both the x86 and ARM \ac{ISA}s.

The data is not in fact \IT{moved}, but \IT{copied}.

\subsection{MIPS}

\label{MIPS_leaf_function_ex1}

The GCC assembly output below lists registers by number:

\lstinputlisting[caption=\Optimizing GCC 4.4.5 (\assemblyOutput),style=customasmMIPS]{patterns/011_ret/MIPS.s}

\dots while \IDA does it by their pseudo names:

\lstinputlisting[caption=\Optimizing GCC 4.4.5 (IDA),style=customasmMIPS]{patterns/011_ret/MIPS_IDA.lst}

The \$2 (or \$V0) register is used to store the function's return value.
\myindex{MIPS!\Pseudoinstructions!LI}
\INS{LI} stands for ``Load Immediate'' and is the MIPS equivalent to \MOV.

\myindex{MIPS!\Instructions!J}
The other instruction is the jump instruction (J or JR) which returns the execution flow to the \gls{caller}.

\myindex{MIPS!Branch delay slot}
You might be wondering why the positions of the load instruction (LI) and the jump instruction (J or JR) are swapped. This is due to a \ac{RISC} feature called ``branch delay slot''.

The reason this happens is a quirk in the architecture of some RISC \ac{ISA}s and isn't important for our
purposes---we must simply keep in mind that in MIPS, the instruction following a jump or branch instruction
is executed \IT{before} the jump/branch instruction itself.

As a consequence, branch instructions always swap places with the instruction executed immediately beforehand.


In practice, functions which merely return 1 (\IT{true}) or 0 (\IT{false}) are very frequent.

The smallest ever of the standard UNIX utilities, \IT{/bin/true} and \IT{/bin/false} return 0 and 1 respectively, as an exit code.
(Zero as an exit code usually means success, non-zero means error.)
}
\RU{\subsubsection{std::string}
\myindex{\Cpp!STL!std::string}
\label{std_string}

\myparagraph{Как устроена структура}

Многие строковые библиотеки \InSqBrackets{\CNotes 2.2} обеспечивают структуру содержащую ссылку 
на буфер собственно со строкой, переменная всегда содержащую длину строки 
(что очень удобно для массы функций \InSqBrackets{\CNotes 2.2.1}) и переменную содержащую текущий размер буфера.

Строка в буфере обыкновенно оканчивается нулем: это для того чтобы указатель на буфер можно было
передавать в функции требующие на вход обычную сишную \ac{ASCIIZ}-строку.

Стандарт \Cpp не описывает, как именно нужно реализовывать std::string,
но, как правило, они реализованы как описано выше, с небольшими дополнениями.

Строки в \Cpp это не класс (как, например, QString в Qt), а темплейт (basic\_string), 
это сделано для того чтобы поддерживать 
строки содержащие разного типа символы: как минимум \Tchar и \IT{wchar\_t}.

Так что, std::string это класс с базовым типом \Tchar.

А std::wstring это класс с базовым типом \IT{wchar\_t}.

\mysubparagraph{MSVC}

В реализации MSVC, вместо ссылки на буфер может содержаться сам буфер (если строка короче 16-и символов).

Это означает, что каждая короткая строка будет занимать в памяти по крайней мере $16 + 4 + 4 = 24$ 
байт для 32-битной среды либо $16 + 8 + 8 = 32$ 
байта в 64-битной, а если строка длиннее 16-и символов, то прибавьте еще длину самой строки.

\lstinputlisting[caption=пример для MSVC,style=customc]{\CURPATH/STL/string/MSVC_RU.cpp}

Собственно, из этого исходника почти всё ясно.

Несколько замечаний:

Если строка короче 16-и символов, 
то отдельный буфер для строки в \glslink{heap}{куче} выделяться не будет.

Это удобно потому что на практике, основная часть строк действительно короткие.
Вероятно, разработчики в Microsoft выбрали размер в 16 символов как разумный баланс.

Теперь очень важный момент в конце функции main(): мы не пользуемся методом c\_str(), тем не менее,
если это скомпилировать и запустить, то обе строки появятся в консоли!

Работает это вот почему.

В первом случае строка короче 16-и символов и в начале объекта std::string (его можно рассматривать
просто как структуру) расположен буфер с этой строкой.
\printf трактует указатель как указатель на массив символов оканчивающийся нулем и поэтому всё работает.

Вывод второй строки (длиннее 16-и символов) даже еще опаснее: это вообще типичная программистская ошибка 
(или опечатка), забыть дописать c\_str().
Это работает потому что в это время в начале структуры расположен указатель на буфер.
Это может надолго остаться незамеченным: до тех пока там не появится строка 
короче 16-и символов, тогда процесс упадет.

\mysubparagraph{GCC}

В реализации GCC в структуре есть еще одна переменная --- reference count.

Интересно, что указатель на экземпляр класса std::string в GCC указывает не на начало самой структуры, 
а на указатель на буфера.
В libstdc++-v3\textbackslash{}include\textbackslash{}bits\textbackslash{}basic\_string.h 
мы можем прочитать что это сделано для удобства отладки:

\begin{lstlisting}
   *  The reason you want _M_data pointing to the character %array and
   *  not the _Rep is so that the debugger can see the string
   *  contents. (Probably we should add a non-inline member to get
   *  the _Rep for the debugger to use, so users can check the actual
   *  string length.)
\end{lstlisting}

\href{http://go.yurichev.com/17085}{исходный код basic\_string.h}

В нашем примере мы учитываем это:

\lstinputlisting[caption=пример для GCC,style=customc]{\CURPATH/STL/string/GCC_RU.cpp}

Нужны еще небольшие хаки чтобы сымитировать типичную ошибку, которую мы уже видели выше, из-за
более ужесточенной проверки типов в GCC, тем не менее, printf() работает и здесь без c\_str().

\myparagraph{Чуть более сложный пример}

\lstinputlisting[style=customc]{\CURPATH/STL/string/3.cpp}

\lstinputlisting[caption=MSVC 2012,style=customasmx86]{\CURPATH/STL/string/3_MSVC_RU.asm}

Собственно, компилятор не конструирует строки статически: да в общем-то и как
это возможно, если буфер с ней нужно хранить в \glslink{heap}{куче}?

Вместо этого в сегменте данных хранятся обычные \ac{ASCIIZ}-строки, а позже, во время выполнения, 
при помощи метода \q{assign}, конструируются строки s1 и s2
.
При помощи \TT{operator+}, создается строка s3.

Обратите внимание на то что вызов метода c\_str() отсутствует,
потому что его код достаточно короткий и компилятор вставил его прямо здесь:
если строка короче 16-и байт, то в регистре EAX остается указатель на буфер,
а если длиннее, то из этого же места достается адрес на буфер расположенный в \glslink{heap}{куче}.

Далее следуют вызовы трех деструкторов, причем, они вызываются только если строка длиннее 16-и байт:
тогда нужно освободить буфера в \glslink{heap}{куче}.
В противном случае, так как все три объекта std::string хранятся в стеке,
они освобождаются автоматически после выхода из функции.

Следовательно, работа с короткими строками более быстрая из-за м\'{е}ньшего обращения к \glslink{heap}{куче}.

Код на GCC даже проще (из-за того, что в GCC, как мы уже видели, не реализована возможность хранить короткую
строку прямо в структуре):

% TODO1 comment each function meaning
\lstinputlisting[caption=GCC 4.8.1,style=customasmx86]{\CURPATH/STL/string/3_GCC_RU.s}

Можно заметить, что в деструкторы передается не указатель на объект,
а указатель на место за 12 байт (или 3 слова) перед ним, то есть, на настоящее начало структуры.

\myparagraph{std::string как глобальная переменная}
\label{sec:std_string_as_global_variable}

Опытные программисты на \Cpp знают, что глобальные переменные \ac{STL}-типов вполне можно объявлять.

Да, действительно:

\lstinputlisting[style=customc]{\CURPATH/STL/string/5.cpp}

Но как и где будет вызываться конструктор \TT{std::string}?

На самом деле, эта переменная будет инициализирована даже перед началом \main.

\lstinputlisting[caption=MSVC 2012: здесь конструируется глобальная переменная{,} а также регистрируется её деструктор,style=customasmx86]{\CURPATH/STL/string/5_MSVC_p2.asm}

\lstinputlisting[caption=MSVC 2012: здесь глобальная переменная используется в \main,style=customasmx86]{\CURPATH/STL/string/5_MSVC_p1.asm}

\lstinputlisting[caption=MSVC 2012: эта функция-деструктор вызывается перед выходом,style=customasmx86]{\CURPATH/STL/string/5_MSVC_p3.asm}

\myindex{\CStandardLibrary!atexit()}
В реальности, из \ac{CRT}, еще до вызова main(), вызывается специальная функция,
в которой перечислены все конструкторы подобных переменных.
Более того: при помощи atexit() регистрируется функция, которая будет вызвана в конце работы программы:
в этой функции компилятор собирает вызовы деструкторов всех подобных глобальных переменных.

GCC работает похожим образом:

\lstinputlisting[caption=GCC 4.8.1,style=customasmx86]{\CURPATH/STL/string/5_GCC.s}

Но он не выделяет отдельной функции в которой будут собраны деструкторы: 
каждый деструктор передается в atexit() по одному.

% TODO а если глобальная STL-переменная в другом модуле? надо проверить.

}
\DE{\subsection{Einfachste XOR-Verschlüsselung überhaupt}

Ich habe einmal eine Software gesehen, bei der alle Debugging-Ausgaben mit XOR mit dem Wert 3
verschlüsselt wurden. Mit anderen Worten, die beiden niedrigsten Bits aller Buchstaben wurden invertiert.

``Hello, world'' wurde zu ``Kfool/\#tlqog'':

\begin{lstlisting}
#!/usr/bin/python

msg="Hello, world!"

print "".join(map(lambda x: chr(ord(x)^3), msg))
\end{lstlisting}

Das ist eine ziemlich interessante Verschlüsselung (oder besser eine Verschleierung),
weil sie zwei wichtige Eigenschaften hat:
1) es ist eine einzige Funktion zum Verschlüsseln und entschlüsseln, sie muss nur wiederholt angewendet werden
2) die entstehenden Buchstaben befinden sich im druckbaren Bereich, also die ganze Zeichenkette kann ohne
Escape-Symbole im Code verwendet werden.

Die zweite Eigenschaft nutzt die Tatsache, dass alle druckbaren Zeichen in Reihen organisiert sind: 0x2x-0x7x,
und wenn die beiden niederwertigsten Bits invertiert werden, wird der Buchstabe um eine oder drei Stellen nach
links oder rechts \IT{verschoben}, aber niemals in eine andere Reihe:

\begin{figure}[H]
\centering
\includegraphics[width=0.7\textwidth]{ascii_clean.png}
\caption{7-Bit \ac{ASCII} Tabelle in Emacs}
\end{figure}

\dots mit dem Zeichen 0x7F als einziger Ausnahme.

Im Folgenden werden also beispielsweise die Zeichen A-Z \IT{verschlüsselt}:

\begin{lstlisting}
#!/usr/bin/python

msg="@ABCDEFGHIJKLMNO"

print "".join(map(lambda x: chr(ord(x)^3), msg))
\end{lstlisting}

Ergebnis:
% FIXME \verb  --  relevant comment for German?
\begin{lstlisting}
CBA@GFEDKJIHONML
\end{lstlisting}

Es sieht so aus als würden die Zeichen ``@'' und ``C'' sowie ``B'' und ``A'' vertauscht werden.

Hier ist noch ein interessantes Beispiel, in dem gezeigt wird, wie die Eigenschaften von XOR
ausgenutzt werden können: Exakt den gleichen Effekt, dass druckbare Zeichen auch druckbar bleiben,
kann man dadurch erzielen, dass irgendeine Kombination der niedrigsten vier Bits invertiert wird.
}

\EN{\section{Returning Values}
\label{ret_val_func}

Another simple function is the one that simply returns a constant value:

\lstinputlisting[caption=\EN{\CCpp Code},style=customc]{patterns/011_ret/1.c}

Let's compile it.

\subsection{x86}

Here's what both the GCC and MSVC compilers produce (with optimization) on the x86 platform:

\lstinputlisting[caption=\Optimizing GCC/MSVC (\assemblyOutput),style=customasmx86]{patterns/011_ret/1.s}

\myindex{x86!\Instructions!RET}
There are just two instructions: the first places the value 123 into the \EAX register,
which is used by convention for storing the return
value, and the second one is \RET, which returns execution to the \gls{caller}.

The caller will take the result from the \EAX register.

\subsection{ARM}

There are a few differences on the ARM platform:

\lstinputlisting[caption=\OptimizingKeilVI (\ARMMode) ASM Output,style=customasmARM]{patterns/011_ret/1_Keil_ARM_O3.s}

ARM uses the register \Reg{0} for returning the results of functions, so 123 is copied into \Reg{0}.

\myindex{ARM!\Instructions!MOV}
\myindex{x86!\Instructions!MOV}
It is worth noting that \MOV is a misleading name for the instruction in both the x86 and ARM \ac{ISA}s.

The data is not in fact \IT{moved}, but \IT{copied}.

\subsection{MIPS}

\label{MIPS_leaf_function_ex1}

The GCC assembly output below lists registers by number:

\lstinputlisting[caption=\Optimizing GCC 4.4.5 (\assemblyOutput),style=customasmMIPS]{patterns/011_ret/MIPS.s}

\dots while \IDA does it by their pseudo names:

\lstinputlisting[caption=\Optimizing GCC 4.4.5 (IDA),style=customasmMIPS]{patterns/011_ret/MIPS_IDA.lst}

The \$2 (or \$V0) register is used to store the function's return value.
\myindex{MIPS!\Pseudoinstructions!LI}
\INS{LI} stands for ``Load Immediate'' and is the MIPS equivalent to \MOV.

\myindex{MIPS!\Instructions!J}
The other instruction is the jump instruction (J or JR) which returns the execution flow to the \gls{caller}.

\myindex{MIPS!Branch delay slot}
You might be wondering why the positions of the load instruction (LI) and the jump instruction (J or JR) are swapped. This is due to a \ac{RISC} feature called ``branch delay slot''.

The reason this happens is a quirk in the architecture of some RISC \ac{ISA}s and isn't important for our
purposes---we must simply keep in mind that in MIPS, the instruction following a jump or branch instruction
is executed \IT{before} the jump/branch instruction itself.

As a consequence, branch instructions always swap places with the instruction executed immediately beforehand.


In practice, functions which merely return 1 (\IT{true}) or 0 (\IT{false}) are very frequent.

The smallest ever of the standard UNIX utilities, \IT{/bin/true} and \IT{/bin/false} return 0 and 1 respectively, as an exit code.
(Zero as an exit code usually means success, non-zero means error.)
}
\RU{\subsubsection{std::string}
\myindex{\Cpp!STL!std::string}
\label{std_string}

\myparagraph{Как устроена структура}

Многие строковые библиотеки \InSqBrackets{\CNotes 2.2} обеспечивают структуру содержащую ссылку 
на буфер собственно со строкой, переменная всегда содержащую длину строки 
(что очень удобно для массы функций \InSqBrackets{\CNotes 2.2.1}) и переменную содержащую текущий размер буфера.

Строка в буфере обыкновенно оканчивается нулем: это для того чтобы указатель на буфер можно было
передавать в функции требующие на вход обычную сишную \ac{ASCIIZ}-строку.

Стандарт \Cpp не описывает, как именно нужно реализовывать std::string,
но, как правило, они реализованы как описано выше, с небольшими дополнениями.

Строки в \Cpp это не класс (как, например, QString в Qt), а темплейт (basic\_string), 
это сделано для того чтобы поддерживать 
строки содержащие разного типа символы: как минимум \Tchar и \IT{wchar\_t}.

Так что, std::string это класс с базовым типом \Tchar.

А std::wstring это класс с базовым типом \IT{wchar\_t}.

\mysubparagraph{MSVC}

В реализации MSVC, вместо ссылки на буфер может содержаться сам буфер (если строка короче 16-и символов).

Это означает, что каждая короткая строка будет занимать в памяти по крайней мере $16 + 4 + 4 = 24$ 
байт для 32-битной среды либо $16 + 8 + 8 = 32$ 
байта в 64-битной, а если строка длиннее 16-и символов, то прибавьте еще длину самой строки.

\lstinputlisting[caption=пример для MSVC,style=customc]{\CURPATH/STL/string/MSVC_RU.cpp}

Собственно, из этого исходника почти всё ясно.

Несколько замечаний:

Если строка короче 16-и символов, 
то отдельный буфер для строки в \glslink{heap}{куче} выделяться не будет.

Это удобно потому что на практике, основная часть строк действительно короткие.
Вероятно, разработчики в Microsoft выбрали размер в 16 символов как разумный баланс.

Теперь очень важный момент в конце функции main(): мы не пользуемся методом c\_str(), тем не менее,
если это скомпилировать и запустить, то обе строки появятся в консоли!

Работает это вот почему.

В первом случае строка короче 16-и символов и в начале объекта std::string (его можно рассматривать
просто как структуру) расположен буфер с этой строкой.
\printf трактует указатель как указатель на массив символов оканчивающийся нулем и поэтому всё работает.

Вывод второй строки (длиннее 16-и символов) даже еще опаснее: это вообще типичная программистская ошибка 
(или опечатка), забыть дописать c\_str().
Это работает потому что в это время в начале структуры расположен указатель на буфер.
Это может надолго остаться незамеченным: до тех пока там не появится строка 
короче 16-и символов, тогда процесс упадет.

\mysubparagraph{GCC}

В реализации GCC в структуре есть еще одна переменная --- reference count.

Интересно, что указатель на экземпляр класса std::string в GCC указывает не на начало самой структуры, 
а на указатель на буфера.
В libstdc++-v3\textbackslash{}include\textbackslash{}bits\textbackslash{}basic\_string.h 
мы можем прочитать что это сделано для удобства отладки:

\begin{lstlisting}
   *  The reason you want _M_data pointing to the character %array and
   *  not the _Rep is so that the debugger can see the string
   *  contents. (Probably we should add a non-inline member to get
   *  the _Rep for the debugger to use, so users can check the actual
   *  string length.)
\end{lstlisting}

\href{http://go.yurichev.com/17085}{исходный код basic\_string.h}

В нашем примере мы учитываем это:

\lstinputlisting[caption=пример для GCC,style=customc]{\CURPATH/STL/string/GCC_RU.cpp}

Нужны еще небольшие хаки чтобы сымитировать типичную ошибку, которую мы уже видели выше, из-за
более ужесточенной проверки типов в GCC, тем не менее, printf() работает и здесь без c\_str().

\myparagraph{Чуть более сложный пример}

\lstinputlisting[style=customc]{\CURPATH/STL/string/3.cpp}

\lstinputlisting[caption=MSVC 2012,style=customasmx86]{\CURPATH/STL/string/3_MSVC_RU.asm}

Собственно, компилятор не конструирует строки статически: да в общем-то и как
это возможно, если буфер с ней нужно хранить в \glslink{heap}{куче}?

Вместо этого в сегменте данных хранятся обычные \ac{ASCIIZ}-строки, а позже, во время выполнения, 
при помощи метода \q{assign}, конструируются строки s1 и s2
.
При помощи \TT{operator+}, создается строка s3.

Обратите внимание на то что вызов метода c\_str() отсутствует,
потому что его код достаточно короткий и компилятор вставил его прямо здесь:
если строка короче 16-и байт, то в регистре EAX остается указатель на буфер,
а если длиннее, то из этого же места достается адрес на буфер расположенный в \glslink{heap}{куче}.

Далее следуют вызовы трех деструкторов, причем, они вызываются только если строка длиннее 16-и байт:
тогда нужно освободить буфера в \glslink{heap}{куче}.
В противном случае, так как все три объекта std::string хранятся в стеке,
они освобождаются автоматически после выхода из функции.

Следовательно, работа с короткими строками более быстрая из-за м\'{е}ньшего обращения к \glslink{heap}{куче}.

Код на GCC даже проще (из-за того, что в GCC, как мы уже видели, не реализована возможность хранить короткую
строку прямо в структуре):

% TODO1 comment each function meaning
\lstinputlisting[caption=GCC 4.8.1,style=customasmx86]{\CURPATH/STL/string/3_GCC_RU.s}

Можно заметить, что в деструкторы передается не указатель на объект,
а указатель на место за 12 байт (или 3 слова) перед ним, то есть, на настоящее начало структуры.

\myparagraph{std::string как глобальная переменная}
\label{sec:std_string_as_global_variable}

Опытные программисты на \Cpp знают, что глобальные переменные \ac{STL}-типов вполне можно объявлять.

Да, действительно:

\lstinputlisting[style=customc]{\CURPATH/STL/string/5.cpp}

Но как и где будет вызываться конструктор \TT{std::string}?

На самом деле, эта переменная будет инициализирована даже перед началом \main.

\lstinputlisting[caption=MSVC 2012: здесь конструируется глобальная переменная{,} а также регистрируется её деструктор,style=customasmx86]{\CURPATH/STL/string/5_MSVC_p2.asm}

\lstinputlisting[caption=MSVC 2012: здесь глобальная переменная используется в \main,style=customasmx86]{\CURPATH/STL/string/5_MSVC_p1.asm}

\lstinputlisting[caption=MSVC 2012: эта функция-деструктор вызывается перед выходом,style=customasmx86]{\CURPATH/STL/string/5_MSVC_p3.asm}

\myindex{\CStandardLibrary!atexit()}
В реальности, из \ac{CRT}, еще до вызова main(), вызывается специальная функция,
в которой перечислены все конструкторы подобных переменных.
Более того: при помощи atexit() регистрируется функция, которая будет вызвана в конце работы программы:
в этой функции компилятор собирает вызовы деструкторов всех подобных глобальных переменных.

GCC работает похожим образом:

\lstinputlisting[caption=GCC 4.8.1,style=customasmx86]{\CURPATH/STL/string/5_GCC.s}

Но он не выделяет отдельной функции в которой будут собраны деструкторы: 
каждый деструктор передается в atexit() по одному.

% TODO а если глобальная STL-переменная в другом модуле? надо проверить.

}

\EN{\section{Returning Values}
\label{ret_val_func}

Another simple function is the one that simply returns a constant value:

\lstinputlisting[caption=\EN{\CCpp Code},style=customc]{patterns/011_ret/1.c}

Let's compile it.

\subsection{x86}

Here's what both the GCC and MSVC compilers produce (with optimization) on the x86 platform:

\lstinputlisting[caption=\Optimizing GCC/MSVC (\assemblyOutput),style=customasmx86]{patterns/011_ret/1.s}

\myindex{x86!\Instructions!RET}
There are just two instructions: the first places the value 123 into the \EAX register,
which is used by convention for storing the return
value, and the second one is \RET, which returns execution to the \gls{caller}.

The caller will take the result from the \EAX register.

\subsection{ARM}

There are a few differences on the ARM platform:

\lstinputlisting[caption=\OptimizingKeilVI (\ARMMode) ASM Output,style=customasmARM]{patterns/011_ret/1_Keil_ARM_O3.s}

ARM uses the register \Reg{0} for returning the results of functions, so 123 is copied into \Reg{0}.

\myindex{ARM!\Instructions!MOV}
\myindex{x86!\Instructions!MOV}
It is worth noting that \MOV is a misleading name for the instruction in both the x86 and ARM \ac{ISA}s.

The data is not in fact \IT{moved}, but \IT{copied}.

\subsection{MIPS}

\label{MIPS_leaf_function_ex1}

The GCC assembly output below lists registers by number:

\lstinputlisting[caption=\Optimizing GCC 4.4.5 (\assemblyOutput),style=customasmMIPS]{patterns/011_ret/MIPS.s}

\dots while \IDA does it by their pseudo names:

\lstinputlisting[caption=\Optimizing GCC 4.4.5 (IDA),style=customasmMIPS]{patterns/011_ret/MIPS_IDA.lst}

The \$2 (or \$V0) register is used to store the function's return value.
\myindex{MIPS!\Pseudoinstructions!LI}
\INS{LI} stands for ``Load Immediate'' and is the MIPS equivalent to \MOV.

\myindex{MIPS!\Instructions!J}
The other instruction is the jump instruction (J or JR) which returns the execution flow to the \gls{caller}.

\myindex{MIPS!Branch delay slot}
You might be wondering why the positions of the load instruction (LI) and the jump instruction (J or JR) are swapped. This is due to a \ac{RISC} feature called ``branch delay slot''.

The reason this happens is a quirk in the architecture of some RISC \ac{ISA}s and isn't important for our
purposes---we must simply keep in mind that in MIPS, the instruction following a jump or branch instruction
is executed \IT{before} the jump/branch instruction itself.

As a consequence, branch instructions always swap places with the instruction executed immediately beforehand.


In practice, functions which merely return 1 (\IT{true}) or 0 (\IT{false}) are very frequent.

The smallest ever of the standard UNIX utilities, \IT{/bin/true} and \IT{/bin/false} return 0 and 1 respectively, as an exit code.
(Zero as an exit code usually means success, non-zero means error.)
}
\RU{\subsubsection{std::string}
\myindex{\Cpp!STL!std::string}
\label{std_string}

\myparagraph{Как устроена структура}

Многие строковые библиотеки \InSqBrackets{\CNotes 2.2} обеспечивают структуру содержащую ссылку 
на буфер собственно со строкой, переменная всегда содержащую длину строки 
(что очень удобно для массы функций \InSqBrackets{\CNotes 2.2.1}) и переменную содержащую текущий размер буфера.

Строка в буфере обыкновенно оканчивается нулем: это для того чтобы указатель на буфер можно было
передавать в функции требующие на вход обычную сишную \ac{ASCIIZ}-строку.

Стандарт \Cpp не описывает, как именно нужно реализовывать std::string,
но, как правило, они реализованы как описано выше, с небольшими дополнениями.

Строки в \Cpp это не класс (как, например, QString в Qt), а темплейт (basic\_string), 
это сделано для того чтобы поддерживать 
строки содержащие разного типа символы: как минимум \Tchar и \IT{wchar\_t}.

Так что, std::string это класс с базовым типом \Tchar.

А std::wstring это класс с базовым типом \IT{wchar\_t}.

\mysubparagraph{MSVC}

В реализации MSVC, вместо ссылки на буфер может содержаться сам буфер (если строка короче 16-и символов).

Это означает, что каждая короткая строка будет занимать в памяти по крайней мере $16 + 4 + 4 = 24$ 
байт для 32-битной среды либо $16 + 8 + 8 = 32$ 
байта в 64-битной, а если строка длиннее 16-и символов, то прибавьте еще длину самой строки.

\lstinputlisting[caption=пример для MSVC,style=customc]{\CURPATH/STL/string/MSVC_RU.cpp}

Собственно, из этого исходника почти всё ясно.

Несколько замечаний:

Если строка короче 16-и символов, 
то отдельный буфер для строки в \glslink{heap}{куче} выделяться не будет.

Это удобно потому что на практике, основная часть строк действительно короткие.
Вероятно, разработчики в Microsoft выбрали размер в 16 символов как разумный баланс.

Теперь очень важный момент в конце функции main(): мы не пользуемся методом c\_str(), тем не менее,
если это скомпилировать и запустить, то обе строки появятся в консоли!

Работает это вот почему.

В первом случае строка короче 16-и символов и в начале объекта std::string (его можно рассматривать
просто как структуру) расположен буфер с этой строкой.
\printf трактует указатель как указатель на массив символов оканчивающийся нулем и поэтому всё работает.

Вывод второй строки (длиннее 16-и символов) даже еще опаснее: это вообще типичная программистская ошибка 
(или опечатка), забыть дописать c\_str().
Это работает потому что в это время в начале структуры расположен указатель на буфер.
Это может надолго остаться незамеченным: до тех пока там не появится строка 
короче 16-и символов, тогда процесс упадет.

\mysubparagraph{GCC}

В реализации GCC в структуре есть еще одна переменная --- reference count.

Интересно, что указатель на экземпляр класса std::string в GCC указывает не на начало самой структуры, 
а на указатель на буфера.
В libstdc++-v3\textbackslash{}include\textbackslash{}bits\textbackslash{}basic\_string.h 
мы можем прочитать что это сделано для удобства отладки:

\begin{lstlisting}
   *  The reason you want _M_data pointing to the character %array and
   *  not the _Rep is so that the debugger can see the string
   *  contents. (Probably we should add a non-inline member to get
   *  the _Rep for the debugger to use, so users can check the actual
   *  string length.)
\end{lstlisting}

\href{http://go.yurichev.com/17085}{исходный код basic\_string.h}

В нашем примере мы учитываем это:

\lstinputlisting[caption=пример для GCC,style=customc]{\CURPATH/STL/string/GCC_RU.cpp}

Нужны еще небольшие хаки чтобы сымитировать типичную ошибку, которую мы уже видели выше, из-за
более ужесточенной проверки типов в GCC, тем не менее, printf() работает и здесь без c\_str().

\myparagraph{Чуть более сложный пример}

\lstinputlisting[style=customc]{\CURPATH/STL/string/3.cpp}

\lstinputlisting[caption=MSVC 2012,style=customasmx86]{\CURPATH/STL/string/3_MSVC_RU.asm}

Собственно, компилятор не конструирует строки статически: да в общем-то и как
это возможно, если буфер с ней нужно хранить в \glslink{heap}{куче}?

Вместо этого в сегменте данных хранятся обычные \ac{ASCIIZ}-строки, а позже, во время выполнения, 
при помощи метода \q{assign}, конструируются строки s1 и s2
.
При помощи \TT{operator+}, создается строка s3.

Обратите внимание на то что вызов метода c\_str() отсутствует,
потому что его код достаточно короткий и компилятор вставил его прямо здесь:
если строка короче 16-и байт, то в регистре EAX остается указатель на буфер,
а если длиннее, то из этого же места достается адрес на буфер расположенный в \glslink{heap}{куче}.

Далее следуют вызовы трех деструкторов, причем, они вызываются только если строка длиннее 16-и байт:
тогда нужно освободить буфера в \glslink{heap}{куче}.
В противном случае, так как все три объекта std::string хранятся в стеке,
они освобождаются автоматически после выхода из функции.

Следовательно, работа с короткими строками более быстрая из-за м\'{е}ньшего обращения к \glslink{heap}{куче}.

Код на GCC даже проще (из-за того, что в GCC, как мы уже видели, не реализована возможность хранить короткую
строку прямо в структуре):

% TODO1 comment each function meaning
\lstinputlisting[caption=GCC 4.8.1,style=customasmx86]{\CURPATH/STL/string/3_GCC_RU.s}

Можно заметить, что в деструкторы передается не указатель на объект,
а указатель на место за 12 байт (или 3 слова) перед ним, то есть, на настоящее начало структуры.

\myparagraph{std::string как глобальная переменная}
\label{sec:std_string_as_global_variable}

Опытные программисты на \Cpp знают, что глобальные переменные \ac{STL}-типов вполне можно объявлять.

Да, действительно:

\lstinputlisting[style=customc]{\CURPATH/STL/string/5.cpp}

Но как и где будет вызываться конструктор \TT{std::string}?

На самом деле, эта переменная будет инициализирована даже перед началом \main.

\lstinputlisting[caption=MSVC 2012: здесь конструируется глобальная переменная{,} а также регистрируется её деструктор,style=customasmx86]{\CURPATH/STL/string/5_MSVC_p2.asm}

\lstinputlisting[caption=MSVC 2012: здесь глобальная переменная используется в \main,style=customasmx86]{\CURPATH/STL/string/5_MSVC_p1.asm}

\lstinputlisting[caption=MSVC 2012: эта функция-деструктор вызывается перед выходом,style=customasmx86]{\CURPATH/STL/string/5_MSVC_p3.asm}

\myindex{\CStandardLibrary!atexit()}
В реальности, из \ac{CRT}, еще до вызова main(), вызывается специальная функция,
в которой перечислены все конструкторы подобных переменных.
Более того: при помощи atexit() регистрируется функция, которая будет вызвана в конце работы программы:
в этой функции компилятор собирает вызовы деструкторов всех подобных глобальных переменных.

GCC работает похожим образом:

\lstinputlisting[caption=GCC 4.8.1,style=customasmx86]{\CURPATH/STL/string/5_GCC.s}

Но он не выделяет отдельной функции в которой будут собраны деструкторы: 
каждый деструктор передается в atexit() по одному.

% TODO а если глобальная STL-переменная в другом модуле? надо проверить.

}
\DE{\subsection{Einfachste XOR-Verschlüsselung überhaupt}

Ich habe einmal eine Software gesehen, bei der alle Debugging-Ausgaben mit XOR mit dem Wert 3
verschlüsselt wurden. Mit anderen Worten, die beiden niedrigsten Bits aller Buchstaben wurden invertiert.

``Hello, world'' wurde zu ``Kfool/\#tlqog'':

\begin{lstlisting}
#!/usr/bin/python

msg="Hello, world!"

print "".join(map(lambda x: chr(ord(x)^3), msg))
\end{lstlisting}

Das ist eine ziemlich interessante Verschlüsselung (oder besser eine Verschleierung),
weil sie zwei wichtige Eigenschaften hat:
1) es ist eine einzige Funktion zum Verschlüsseln und entschlüsseln, sie muss nur wiederholt angewendet werden
2) die entstehenden Buchstaben befinden sich im druckbaren Bereich, also die ganze Zeichenkette kann ohne
Escape-Symbole im Code verwendet werden.

Die zweite Eigenschaft nutzt die Tatsache, dass alle druckbaren Zeichen in Reihen organisiert sind: 0x2x-0x7x,
und wenn die beiden niederwertigsten Bits invertiert werden, wird der Buchstabe um eine oder drei Stellen nach
links oder rechts \IT{verschoben}, aber niemals in eine andere Reihe:

\begin{figure}[H]
\centering
\includegraphics[width=0.7\textwidth]{ascii_clean.png}
\caption{7-Bit \ac{ASCII} Tabelle in Emacs}
\end{figure}

\dots mit dem Zeichen 0x7F als einziger Ausnahme.

Im Folgenden werden also beispielsweise die Zeichen A-Z \IT{verschlüsselt}:

\begin{lstlisting}
#!/usr/bin/python

msg="@ABCDEFGHIJKLMNO"

print "".join(map(lambda x: chr(ord(x)^3), msg))
\end{lstlisting}

Ergebnis:
% FIXME \verb  --  relevant comment for German?
\begin{lstlisting}
CBA@GFEDKJIHONML
\end{lstlisting}

Es sieht so aus als würden die Zeichen ``@'' und ``C'' sowie ``B'' und ``A'' vertauscht werden.

Hier ist noch ein interessantes Beispiel, in dem gezeigt wird, wie die Eigenschaften von XOR
ausgenutzt werden können: Exakt den gleichen Effekt, dass druckbare Zeichen auch druckbar bleiben,
kann man dadurch erzielen, dass irgendeine Kombination der niedrigsten vier Bits invertiert wird.
}

\EN{\section{Returning Values}
\label{ret_val_func}

Another simple function is the one that simply returns a constant value:

\lstinputlisting[caption=\EN{\CCpp Code},style=customc]{patterns/011_ret/1.c}

Let's compile it.

\subsection{x86}

Here's what both the GCC and MSVC compilers produce (with optimization) on the x86 platform:

\lstinputlisting[caption=\Optimizing GCC/MSVC (\assemblyOutput),style=customasmx86]{patterns/011_ret/1.s}

\myindex{x86!\Instructions!RET}
There are just two instructions: the first places the value 123 into the \EAX register,
which is used by convention for storing the return
value, and the second one is \RET, which returns execution to the \gls{caller}.

The caller will take the result from the \EAX register.

\subsection{ARM}

There are a few differences on the ARM platform:

\lstinputlisting[caption=\OptimizingKeilVI (\ARMMode) ASM Output,style=customasmARM]{patterns/011_ret/1_Keil_ARM_O3.s}

ARM uses the register \Reg{0} for returning the results of functions, so 123 is copied into \Reg{0}.

\myindex{ARM!\Instructions!MOV}
\myindex{x86!\Instructions!MOV}
It is worth noting that \MOV is a misleading name for the instruction in both the x86 and ARM \ac{ISA}s.

The data is not in fact \IT{moved}, but \IT{copied}.

\subsection{MIPS}

\label{MIPS_leaf_function_ex1}

The GCC assembly output below lists registers by number:

\lstinputlisting[caption=\Optimizing GCC 4.4.5 (\assemblyOutput),style=customasmMIPS]{patterns/011_ret/MIPS.s}

\dots while \IDA does it by their pseudo names:

\lstinputlisting[caption=\Optimizing GCC 4.4.5 (IDA),style=customasmMIPS]{patterns/011_ret/MIPS_IDA.lst}

The \$2 (or \$V0) register is used to store the function's return value.
\myindex{MIPS!\Pseudoinstructions!LI}
\INS{LI} stands for ``Load Immediate'' and is the MIPS equivalent to \MOV.

\myindex{MIPS!\Instructions!J}
The other instruction is the jump instruction (J or JR) which returns the execution flow to the \gls{caller}.

\myindex{MIPS!Branch delay slot}
You might be wondering why the positions of the load instruction (LI) and the jump instruction (J or JR) are swapped. This is due to a \ac{RISC} feature called ``branch delay slot''.

The reason this happens is a quirk in the architecture of some RISC \ac{ISA}s and isn't important for our
purposes---we must simply keep in mind that in MIPS, the instruction following a jump or branch instruction
is executed \IT{before} the jump/branch instruction itself.

As a consequence, branch instructions always swap places with the instruction executed immediately beforehand.


In practice, functions which merely return 1 (\IT{true}) or 0 (\IT{false}) are very frequent.

The smallest ever of the standard UNIX utilities, \IT{/bin/true} and \IT{/bin/false} return 0 and 1 respectively, as an exit code.
(Zero as an exit code usually means success, non-zero means error.)
}
\RU{\subsubsection{std::string}
\myindex{\Cpp!STL!std::string}
\label{std_string}

\myparagraph{Как устроена структура}

Многие строковые библиотеки \InSqBrackets{\CNotes 2.2} обеспечивают структуру содержащую ссылку 
на буфер собственно со строкой, переменная всегда содержащую длину строки 
(что очень удобно для массы функций \InSqBrackets{\CNotes 2.2.1}) и переменную содержащую текущий размер буфера.

Строка в буфере обыкновенно оканчивается нулем: это для того чтобы указатель на буфер можно было
передавать в функции требующие на вход обычную сишную \ac{ASCIIZ}-строку.

Стандарт \Cpp не описывает, как именно нужно реализовывать std::string,
но, как правило, они реализованы как описано выше, с небольшими дополнениями.

Строки в \Cpp это не класс (как, например, QString в Qt), а темплейт (basic\_string), 
это сделано для того чтобы поддерживать 
строки содержащие разного типа символы: как минимум \Tchar и \IT{wchar\_t}.

Так что, std::string это класс с базовым типом \Tchar.

А std::wstring это класс с базовым типом \IT{wchar\_t}.

\mysubparagraph{MSVC}

В реализации MSVC, вместо ссылки на буфер может содержаться сам буфер (если строка короче 16-и символов).

Это означает, что каждая короткая строка будет занимать в памяти по крайней мере $16 + 4 + 4 = 24$ 
байт для 32-битной среды либо $16 + 8 + 8 = 32$ 
байта в 64-битной, а если строка длиннее 16-и символов, то прибавьте еще длину самой строки.

\lstinputlisting[caption=пример для MSVC,style=customc]{\CURPATH/STL/string/MSVC_RU.cpp}

Собственно, из этого исходника почти всё ясно.

Несколько замечаний:

Если строка короче 16-и символов, 
то отдельный буфер для строки в \glslink{heap}{куче} выделяться не будет.

Это удобно потому что на практике, основная часть строк действительно короткие.
Вероятно, разработчики в Microsoft выбрали размер в 16 символов как разумный баланс.

Теперь очень важный момент в конце функции main(): мы не пользуемся методом c\_str(), тем не менее,
если это скомпилировать и запустить, то обе строки появятся в консоли!

Работает это вот почему.

В первом случае строка короче 16-и символов и в начале объекта std::string (его можно рассматривать
просто как структуру) расположен буфер с этой строкой.
\printf трактует указатель как указатель на массив символов оканчивающийся нулем и поэтому всё работает.

Вывод второй строки (длиннее 16-и символов) даже еще опаснее: это вообще типичная программистская ошибка 
(или опечатка), забыть дописать c\_str().
Это работает потому что в это время в начале структуры расположен указатель на буфер.
Это может надолго остаться незамеченным: до тех пока там не появится строка 
короче 16-и символов, тогда процесс упадет.

\mysubparagraph{GCC}

В реализации GCC в структуре есть еще одна переменная --- reference count.

Интересно, что указатель на экземпляр класса std::string в GCC указывает не на начало самой структуры, 
а на указатель на буфера.
В libstdc++-v3\textbackslash{}include\textbackslash{}bits\textbackslash{}basic\_string.h 
мы можем прочитать что это сделано для удобства отладки:

\begin{lstlisting}
   *  The reason you want _M_data pointing to the character %array and
   *  not the _Rep is so that the debugger can see the string
   *  contents. (Probably we should add a non-inline member to get
   *  the _Rep for the debugger to use, so users can check the actual
   *  string length.)
\end{lstlisting}

\href{http://go.yurichev.com/17085}{исходный код basic\_string.h}

В нашем примере мы учитываем это:

\lstinputlisting[caption=пример для GCC,style=customc]{\CURPATH/STL/string/GCC_RU.cpp}

Нужны еще небольшие хаки чтобы сымитировать типичную ошибку, которую мы уже видели выше, из-за
более ужесточенной проверки типов в GCC, тем не менее, printf() работает и здесь без c\_str().

\myparagraph{Чуть более сложный пример}

\lstinputlisting[style=customc]{\CURPATH/STL/string/3.cpp}

\lstinputlisting[caption=MSVC 2012,style=customasmx86]{\CURPATH/STL/string/3_MSVC_RU.asm}

Собственно, компилятор не конструирует строки статически: да в общем-то и как
это возможно, если буфер с ней нужно хранить в \glslink{heap}{куче}?

Вместо этого в сегменте данных хранятся обычные \ac{ASCIIZ}-строки, а позже, во время выполнения, 
при помощи метода \q{assign}, конструируются строки s1 и s2
.
При помощи \TT{operator+}, создается строка s3.

Обратите внимание на то что вызов метода c\_str() отсутствует,
потому что его код достаточно короткий и компилятор вставил его прямо здесь:
если строка короче 16-и байт, то в регистре EAX остается указатель на буфер,
а если длиннее, то из этого же места достается адрес на буфер расположенный в \glslink{heap}{куче}.

Далее следуют вызовы трех деструкторов, причем, они вызываются только если строка длиннее 16-и байт:
тогда нужно освободить буфера в \glslink{heap}{куче}.
В противном случае, так как все три объекта std::string хранятся в стеке,
они освобождаются автоматически после выхода из функции.

Следовательно, работа с короткими строками более быстрая из-за м\'{е}ньшего обращения к \glslink{heap}{куче}.

Код на GCC даже проще (из-за того, что в GCC, как мы уже видели, не реализована возможность хранить короткую
строку прямо в структуре):

% TODO1 comment each function meaning
\lstinputlisting[caption=GCC 4.8.1,style=customasmx86]{\CURPATH/STL/string/3_GCC_RU.s}

Можно заметить, что в деструкторы передается не указатель на объект,
а указатель на место за 12 байт (или 3 слова) перед ним, то есть, на настоящее начало структуры.

\myparagraph{std::string как глобальная переменная}
\label{sec:std_string_as_global_variable}

Опытные программисты на \Cpp знают, что глобальные переменные \ac{STL}-типов вполне можно объявлять.

Да, действительно:

\lstinputlisting[style=customc]{\CURPATH/STL/string/5.cpp}

Но как и где будет вызываться конструктор \TT{std::string}?

На самом деле, эта переменная будет инициализирована даже перед началом \main.

\lstinputlisting[caption=MSVC 2012: здесь конструируется глобальная переменная{,} а также регистрируется её деструктор,style=customasmx86]{\CURPATH/STL/string/5_MSVC_p2.asm}

\lstinputlisting[caption=MSVC 2012: здесь глобальная переменная используется в \main,style=customasmx86]{\CURPATH/STL/string/5_MSVC_p1.asm}

\lstinputlisting[caption=MSVC 2012: эта функция-деструктор вызывается перед выходом,style=customasmx86]{\CURPATH/STL/string/5_MSVC_p3.asm}

\myindex{\CStandardLibrary!atexit()}
В реальности, из \ac{CRT}, еще до вызова main(), вызывается специальная функция,
в которой перечислены все конструкторы подобных переменных.
Более того: при помощи atexit() регистрируется функция, которая будет вызвана в конце работы программы:
в этой функции компилятор собирает вызовы деструкторов всех подобных глобальных переменных.

GCC работает похожим образом:

\lstinputlisting[caption=GCC 4.8.1,style=customasmx86]{\CURPATH/STL/string/5_GCC.s}

Но он не выделяет отдельной функции в которой будут собраны деструкторы: 
каждый деструктор передается в atexit() по одному.

% TODO а если глобальная STL-переменная в другом модуле? надо проверить.

}
\DE{\subsection{Einfachste XOR-Verschlüsselung überhaupt}

Ich habe einmal eine Software gesehen, bei der alle Debugging-Ausgaben mit XOR mit dem Wert 3
verschlüsselt wurden. Mit anderen Worten, die beiden niedrigsten Bits aller Buchstaben wurden invertiert.

``Hello, world'' wurde zu ``Kfool/\#tlqog'':

\begin{lstlisting}
#!/usr/bin/python

msg="Hello, world!"

print "".join(map(lambda x: chr(ord(x)^3), msg))
\end{lstlisting}

Das ist eine ziemlich interessante Verschlüsselung (oder besser eine Verschleierung),
weil sie zwei wichtige Eigenschaften hat:
1) es ist eine einzige Funktion zum Verschlüsseln und entschlüsseln, sie muss nur wiederholt angewendet werden
2) die entstehenden Buchstaben befinden sich im druckbaren Bereich, also die ganze Zeichenkette kann ohne
Escape-Symbole im Code verwendet werden.

Die zweite Eigenschaft nutzt die Tatsache, dass alle druckbaren Zeichen in Reihen organisiert sind: 0x2x-0x7x,
und wenn die beiden niederwertigsten Bits invertiert werden, wird der Buchstabe um eine oder drei Stellen nach
links oder rechts \IT{verschoben}, aber niemals in eine andere Reihe:

\begin{figure}[H]
\centering
\includegraphics[width=0.7\textwidth]{ascii_clean.png}
\caption{7-Bit \ac{ASCII} Tabelle in Emacs}
\end{figure}

\dots mit dem Zeichen 0x7F als einziger Ausnahme.

Im Folgenden werden also beispielsweise die Zeichen A-Z \IT{verschlüsselt}:

\begin{lstlisting}
#!/usr/bin/python

msg="@ABCDEFGHIJKLMNO"

print "".join(map(lambda x: chr(ord(x)^3), msg))
\end{lstlisting}

Ergebnis:
% FIXME \verb  --  relevant comment for German?
\begin{lstlisting}
CBA@GFEDKJIHONML
\end{lstlisting}

Es sieht so aus als würden die Zeichen ``@'' und ``C'' sowie ``B'' und ``A'' vertauscht werden.

Hier ist noch ein interessantes Beispiel, in dem gezeigt wird, wie die Eigenschaften von XOR
ausgenutzt werden können: Exakt den gleichen Effekt, dass druckbare Zeichen auch druckbar bleiben,
kann man dadurch erzielen, dass irgendeine Kombination der niedrigsten vier Bits invertiert wird.
}

\ifdefined\SPANISH
\chapter{Patrones de código}
\fi % SPANISH

\ifdefined\GERMAN
\chapter{Code-Muster}
\fi % GERMAN

\ifdefined\ENGLISH
\chapter{Code Patterns}
\fi % ENGLISH

\ifdefined\ITALIAN
\chapter{Forme di codice}
\fi % ITALIAN

\ifdefined\RUSSIAN
\chapter{Образцы кода}
\fi % RUSSIAN

\ifdefined\BRAZILIAN
\chapter{Padrões de códigos}
\fi % BRAZILIAN

\ifdefined\THAI
\chapter{รูปแบบของโค้ด}
\fi % THAI

\ifdefined\FRENCH
\chapter{Modèle de code}
\fi % FRENCH

\ifdefined\POLISH
\chapter{\PLph{}}
\fi % POLISH

% sections
\EN{\input{patterns/patterns_opt_dbg_EN}}
\ES{\input{patterns/patterns_opt_dbg_ES}}
\ITA{\input{patterns/patterns_opt_dbg_ITA}}
\PTBR{\input{patterns/patterns_opt_dbg_PTBR}}
\RU{\input{patterns/patterns_opt_dbg_RU}}
\THA{\input{patterns/patterns_opt_dbg_THA}}
\DE{\input{patterns/patterns_opt_dbg_DE}}
\FR{\input{patterns/patterns_opt_dbg_FR}}
\PL{\input{patterns/patterns_opt_dbg_PL}}

\RU{\section{Некоторые базовые понятия}}
\EN{\section{Some basics}}
\DE{\section{Einige Grundlagen}}
\FR{\section{Quelques bases}}
\ES{\section{\ESph{}}}
\ITA{\section{Alcune basi teoriche}}
\PTBR{\section{\PTBRph{}}}
\THA{\section{\THAph{}}}
\PL{\section{\PLph{}}}

% sections:
\EN{\input{patterns/intro_CPU_ISA_EN}}
\ES{\input{patterns/intro_CPU_ISA_ES}}
\ITA{\input{patterns/intro_CPU_ISA_ITA}}
\PTBR{\input{patterns/intro_CPU_ISA_PTBR}}
\RU{\input{patterns/intro_CPU_ISA_RU}}
\DE{\input{patterns/intro_CPU_ISA_DE}}
\FR{\input{patterns/intro_CPU_ISA_FR}}
\PL{\input{patterns/intro_CPU_ISA_PL}}

\EN{\input{patterns/numeral_EN}}
\RU{\input{patterns/numeral_RU}}
\ITA{\input{patterns/numeral_ITA}}
\DE{\input{patterns/numeral_DE}}
\FR{\input{patterns/numeral_FR}}
\PL{\input{patterns/numeral_PL}}

% chapters
\input{patterns/00_empty/main}
\input{patterns/011_ret/main}
\input{patterns/01_helloworld/main}
\input{patterns/015_prolog_epilogue/main}
\input{patterns/02_stack/main}
\input{patterns/03_printf/main}
\input{patterns/04_scanf/main}
\input{patterns/05_passing_arguments/main}
\input{patterns/06_return_results/main}
\input{patterns/061_pointers/main}
\input{patterns/065_GOTO/main}
\input{patterns/07_jcc/main}
\input{patterns/08_switch/main}
\input{patterns/09_loops/main}
\input{patterns/10_strings/main}
\input{patterns/11_arith_optimizations/main}
\input{patterns/12_FPU/main}
\input{patterns/13_arrays/main}
\input{patterns/14_bitfields/main}
\EN{\input{patterns/145_LCG/main_EN}}
\RU{\input{patterns/145_LCG/main_RU}}
\input{patterns/15_structs/main}
\input{patterns/17_unions/main}
\input{patterns/18_pointers_to_functions/main}
\input{patterns/185_64bit_in_32_env/main}

\EN{\input{patterns/19_SIMD/main_EN}}
\RU{\input{patterns/19_SIMD/main_RU}}
\DE{\input{patterns/19_SIMD/main_DE}}

\EN{\input{patterns/20_x64/main_EN}}
\RU{\input{patterns/20_x64/main_RU}}

\EN{\input{patterns/205_floating_SIMD/main_EN}}
\RU{\input{patterns/205_floating_SIMD/main_RU}}
\DE{\input{patterns/205_floating_SIMD/main_DE}}

\EN{\input{patterns/ARM/main_EN}}
\RU{\input{patterns/ARM/main_RU}}
\DE{\input{patterns/ARM/main_DE}}

\input{patterns/MIPS/main}


\ifdefined\SPANISH
\chapter{Patrones de código}
\fi % SPANISH

\ifdefined\GERMAN
\chapter{Code-Muster}
\fi % GERMAN

\ifdefined\ENGLISH
\chapter{Code Patterns}
\fi % ENGLISH

\ifdefined\ITALIAN
\chapter{Forme di codice}
\fi % ITALIAN

\ifdefined\RUSSIAN
\chapter{Образцы кода}
\fi % RUSSIAN

\ifdefined\BRAZILIAN
\chapter{Padrões de códigos}
\fi % BRAZILIAN

\ifdefined\THAI
\chapter{รูปแบบของโค้ด}
\fi % THAI

\ifdefined\FRENCH
\chapter{Modèle de code}
\fi % FRENCH

\ifdefined\POLISH
\chapter{\PLph{}}
\fi % POLISH

% sections
\EN{\section{The method}

When the author of this book first started learning C and, later, \Cpp, he used to write small pieces of code, compile them,
and then look at the assembly language output. This made it very easy for him to understand what was going on in the code that he had written.
\footnote{In fact, he still does this when he can't understand what a particular bit of code does.}.
He did this so many times that the relationship between the \CCpp code and what the compiler produced was imprinted deeply in his mind.
It's now easy for him to imagine instantly a rough outline of a C code's appearance and function.
Perhaps this technique could be helpful for others.

%There are a lot of examples for both x86/x64 and ARM.
%Those who already familiar with one of architectures, may freely skim over pages.

By the way, there is a great website where you can do the same, with various compilers, instead of installing them on your box.
You can use it as well: \url{https://gcc.godbolt.org/}.

\section*{\Exercises}

When the author of this book studied assembly language, he also often compiled small C functions and then rewrote
them gradually to assembly, trying to make their code as short as possible.
This probably is not worth doing in real-world scenarios today,
because it's hard to compete with the latest compilers in terms of efficiency. It is, however, a very good way to gain a better understanding of assembly.
Feel free, therefore, to take any assembly code from this book and try to make it shorter.
However, don't forget to test what you have written.

% rewrote to show that debug\release and optimisations levels are orthogonal concepts.
\section*{Optimization levels and debug information}

Source code can be compiled by different compilers with various optimization levels.
A typical compiler has about three such levels, where level zero means that optimization is completely disabled.
Optimization can also be targeted towards code size or code speed.
A non-optimizing compiler is faster and produces more understandable (albeit verbose) code,
whereas an optimizing compiler is slower and tries to produce code that runs faster (but is not necessarily more compact).
In addition to optimization levels, a compiler can include some debug information in the resulting file,
producing code that is easy to debug.
One of the important features of the ´debug' code is that it might contain links
between each line of the source code and its respective machine code address.
Optimizing compilers, on the other hand, tend to produce output where entire lines of source code
can be optimized away and thus not even be present in the resulting machine code.
Reverse engineers can encounter either version, simply because some developers turn on the compiler's optimization flags and others do not.
Because of this, we'll try to work on examples of both debug and release versions of the code featured in this book, wherever possible.

Sometimes some pretty ancient compilers are used in this book, in order to get the shortest (or simplest) possible code snippet.
}
\ES{\input{patterns/patterns_opt_dbg_ES}}
\ITA{\input{patterns/patterns_opt_dbg_ITA}}
\PTBR{\input{patterns/patterns_opt_dbg_PTBR}}
\RU{\input{patterns/patterns_opt_dbg_RU}}
\THA{\input{patterns/patterns_opt_dbg_THA}}
\DE{\input{patterns/patterns_opt_dbg_DE}}
\FR{\input{patterns/patterns_opt_dbg_FR}}
\PL{\input{patterns/patterns_opt_dbg_PL}}

\RU{\section{Некоторые базовые понятия}}
\EN{\section{Some basics}}
\DE{\section{Einige Grundlagen}}
\FR{\section{Quelques bases}}
\ES{\section{\ESph{}}}
\ITA{\section{Alcune basi teoriche}}
\PTBR{\section{\PTBRph{}}}
\THA{\section{\THAph{}}}
\PL{\section{\PLph{}}}

% sections:
\EN{\input{patterns/intro_CPU_ISA_EN}}
\ES{\input{patterns/intro_CPU_ISA_ES}}
\ITA{\input{patterns/intro_CPU_ISA_ITA}}
\PTBR{\input{patterns/intro_CPU_ISA_PTBR}}
\RU{\input{patterns/intro_CPU_ISA_RU}}
\DE{\input{patterns/intro_CPU_ISA_DE}}
\FR{\input{patterns/intro_CPU_ISA_FR}}
\PL{\input{patterns/intro_CPU_ISA_PL}}

\EN{\subsection{Numeral Systems}

Humans have become accustomed to a decimal numeral system, probably because almost everyone has 10 fingers.
Nevertheless, the number \q{10} has no significant meaning in science and mathematics.
The natural numeral system in digital electronics is binary: 0 is for an absence of current in the wire, and 1 for presence.
10 in binary is 2 in decimal, 100 in binary is 4 in decimal, and so on.

% This sentence is a bit unweildy - maybe try 'Our ten-digit system would be described as having a radix...' - Renaissance
If the numeral system has 10 digits, it has a \IT{radix} (or \IT{base}) of 10.
The binary numeral system has a \IT{radix} of 2.

Important things to recall:

1) A \IT{number} is a number, while a \IT{digit} is a term from writing systems, and is usually one character

% The original is 'number' is not changed; I think the intent is value, and changed it - Renaissance
2) The value of a number does not change when converted to another radix; only the writing notation for that value has changed (and therefore the way of representing it in \ac{RAM}).

\subsection{Converting From One Radix To Another}

Positional notation is used almost every numerical system. This means that a digit has weight relative to where it is placed inside of the larger number.
If 2 is placed at the rightmost place, it's 2, but if it's placed one digit before rightmost, it's 20.

What does $1234$ stand for?

$10^3 \cdot 1 + 10^2 \cdot 2 + 10^1 \cdot 3 + 1 \cdot 4 = 1234$ or
$1000 \cdot 1 + 100 \cdot 2 + 10 \cdot 3 + 4 = 1234$

It's the same story for binary numbers, but the base is 2 instead of 10.
What does 0b101011 stand for?

$2^5 \cdot 1 + 2^4 \cdot 0 + 2^3 \cdot 1 + 2^2 \cdot 0 + 2^1 \cdot 1 + 2^0 \cdot 1 = 43$ or
$32 \cdot 1 + 16 \cdot 0 + 8 \cdot 1 + 4 \cdot 0 + 2 \cdot 1 + 1 = 43$

There is such a thing as non-positional notation, such as the Roman numeral system.
\footnote{About numeric system evolution, see \InSqBrackets{\TAOCPvolII{}, 195--213.}}.
% Maybe add a sentence to fill in that X is always 10, and is therefore non-positional, even though putting an I before subtracts and after adds, and is in that sense positional
Perhaps, humankind switched to positional notation because it's easier to do basic operations (addition, multiplication, etc.) on paper by hand.

Binary numbers can be added, subtracted and so on in the very same as taught in schools, but only 2 digits are available.

Binary numbers are bulky when represented in source code and dumps, so that is where the hexadecimal numeral system can be useful.
A hexadecimal radix uses the digits 0..9, and also 6 Latin characters: A..F.
Each hexadecimal digit takes 4 bits or 4 binary digits, so it's very easy to convert from binary number to hexadecimal and back, even manually, in one's mind.

\begin{center}
\begin{longtable}{ | l | l | l | }
\hline
\HeaderColor hexadecimal & \HeaderColor binary & \HeaderColor decimal \\
\hline
0	&0000	&0 \\
1	&0001	&1 \\
2	&0010	&2 \\
3	&0011	&3 \\
4	&0100	&4 \\
5	&0101	&5 \\
6	&0110	&6 \\
7	&0111	&7 \\
8	&1000	&8 \\
9	&1001	&9 \\
A	&1010	&10 \\
B	&1011	&11 \\
C	&1100	&12 \\
D	&1101	&13 \\
E	&1110	&14 \\
F	&1111	&15 \\
\hline
\end{longtable}
\end{center}

How can one tell which radix is being used in a specific instance?

Decimal numbers are usually written as is, i.e., 1234. Some assemblers allow an identifier on decimal radix numbers, in which the number would be written with a "d" suffix: 1234d.

Binary numbers are sometimes prepended with the "0b" prefix: 0b100110111 (\ac{GCC} has a non-standard language extension for this\footnote{\url{https://gcc.gnu.org/onlinedocs/gcc/Binary-constants.html}}).
There is also another way: using a "b" suffix, for example: 100110111b.
This book tries to use the "0b" prefix consistently throughout the book for binary numbers.

Hexadecimal numbers are prepended with "0x" prefix in \CCpp and other \ac{PL}s: 0x1234ABCD.
Alternatively, they are given a "h" suffix: 1234ABCDh. This is common way of representing them in assemblers and debuggers.
In this convention, if the number is started with a Latin (A..F) digit, a 0 is added at the beginning: 0ABCDEFh.
There was also convention that was popular in 8-bit home computers era, using \$ prefix, like \$ABCD.
The book will try to stick to "0x" prefix throughout the book for hexadecimal numbers.

Should one learn to convert numbers mentally? A table of 1-digit hexadecimal numbers can easily be memorized.
As for larger numbers, it's probably not worth tormenting yourself.

Perhaps the most visible hexadecimal numbers are in \ac{URL}s.
This is the way that non-Latin characters are encoded.
For example:
\url{https://en.wiktionary.org/wiki/na\%C3\%AFvet\%C3\%A9} is the \ac{URL} of Wiktionary article about \q{naïveté} word.

\subsubsection{Octal Radix}

Another numeral system heavily used in the past of computer programming is octal. In octal there are 8 digits (0..7), and each is mapped to 3 bits, so it's easy to convert numbers back and forth.
It has been superseded by the hexadecimal system almost everywhere, but, surprisingly, there is a *NIX utility, used often by many people, which takes octal numbers as argument: \TT{chmod}.

\myindex{UNIX!chmod}
As many *NIX users know, \TT{chmod} argument can be a number of 3 digits. The first digit represents the rights of the owner of the file (read, write and/or execute), the second is the rights for the group to which the file belongs, and the third is for everyone else.
Each digit that \TT{chmod} takes can be represented in binary form:

\begin{center}
\begin{longtable}{ | l | l | l | }
\hline
\HeaderColor decimal & \HeaderColor binary & \HeaderColor meaning \\
\hline
7	&111	&\textbf{rwx} \\
6	&110	&\textbf{rw-} \\
5	&101	&\textbf{r-x} \\
4	&100	&\textbf{r-{}-} \\
3	&011	&\textbf{-wx} \\
2	&010	&\textbf{-w-} \\
1	&001	&\textbf{-{}-x} \\
0	&000	&\textbf{-{}-{}-} \\
\hline
\end{longtable}
\end{center}

So each bit is mapped to a flag: read/write/execute.

The importance of \TT{chmod} here is that the whole number in argument can be represented as octal number.
Let's take, for example, 644.
When you run \TT{chmod 644 file}, you set read/write permissions for owner, read permissions for group and again, read permissions for everyone else.
If we convert the octal number 644 to binary, it would be \TT{110100100}, or, in groups of 3 bits, \TT{110 100 100}.

Now we see that each triplet describe permissions for owner/group/others: first is \TT{rw-}, second is \TT{r--} and third is \TT{r--}.

The octal numeral system was also popular on old computers like PDP-8, because word there could be 12, 24 or 36 bits, and these numbers are all divisible by 3, so the octal system was natural in that environment.
Nowadays, all popular computers employ word/address sizes of 16, 32 or 64 bits, and these numbers are all divisible by 4, so the hexadecimal system is more natural there.

The octal numeral system is supported by all standard \CCpp compilers.
This is a source of confusion sometimes, because octal numbers are encoded with a zero prepended, for example, 0377 is 255.
Sometimes, you might make a typo and write "09" instead of 9, and the compiler would report an error.
GCC might report something like this:\\
\TT{error: invalid digit "9" in octal constant}.

Also, the octal system is somewhat popular in Java. When the IDA shows Java strings with non-printable characters,
they are encoded in the octal system instead of hexadecimal.
\myindex{JAD}
The JAD Java decompiler behaves the same way.

\subsubsection{Divisibility}

When you see a decimal number like 120, you can quickly deduce that it's divisible by 10, because the last digit is zero.
In the same way, 123400 is divisible by 100, because the two last digits are zeros.

Likewise, the hexadecimal number 0x1230 is divisible by 0x10 (or 16), 0x123000 is divisible by 0x1000 (or 4096), etc.

The binary number 0b1000101000 is divisible by 0b1000 (8), etc.

This property can often be used to quickly realize if the size of some block in memory is padded to some boundary.
For example, sections in \ac{PE} files are almost always started at addresses ending with 3 hexadecimal zeros: 0x41000, 0x10001000, etc.
The reason behind this is the fact that almost all \ac{PE} sections are padded to a boundary of 0x1000 (4096) bytes.

\subsubsection{Multi-Precision Arithmetic and Radix}

\index{RSA}
Multi-precision arithmetic can use huge numbers, and each one may be stored in several bytes.
For example, RSA keys, both public and private, span up to 4096 bits, and maybe even more.

% I'm not sure how to change this, but the normal format for quoting would be just to mention the author or book, and footnote to the full reference
In \InSqBrackets{\TAOCPvolII, 265} we find the following idea: when you store a multi-precision number in several bytes,
the whole number can be represented as having a radix of $2^8=256$, and each digit goes to the corresponding byte.
Likewise, if you store a multi-precision number in several 32-bit integer values, each digit goes to each 32-bit slot,
and you may think about this number as stored in radix of $2^{32}$.

\subsubsection{How to Pronounce Non-Decimal Numbers}

Numbers in a non-decimal base are usually pronounced by digit by digit: ``one-zero-zero-one-one-...''.
Words like ``ten'' and ``thousand'' are usually not pronounced, to prevent confusion with the decimal base system.

\subsubsection{Floating point numbers}

To distinguish floating point numbers from integers, they are usually written with ``.0'' at the end,
like $0.0$, $123.0$, etc.
}
\RU{\subsection{Представление чисел}

Люди привыкли к десятичной системе счисления вероятно потому что почти у каждого есть по 10 пальцев.
Тем не менее, число 10 не имеет особого значения в науке и математике.
Двоичная система естествена для цифровой электроники: 0 означает отсутствие тока в проводе и 1 --- его присутствие.
10 в двоичной системе это 2 в десятичной; 100 в двоичной это 4 в десятичной, итд.

Если в системе счисления есть 10 цифр, её \IT{основание} или \IT{radix} это 10.
Двоичная система имеет \IT{основание} 2.

Важные вещи, которые полезно вспомнить:
1) \IT{число} это число, в то время как \IT{цифра} это термин из системы письменности, и это обычно один символ;
2) само число не меняется, когда конвертируется из одного основания в другое: меняется способ его записи (или представления
в памяти).

Как сконвертировать число из одного основания в другое?

Позиционная нотация используется почти везде, это означает, что всякая цифра имеет свой вес, в зависимости от её расположения
внутри числа.
Если 2 расположена в самом последнем месте справа, это 2.
Если она расположена в месте перед последним, это 20.

Что означает $1234$?

$10^3 \cdot 1 + 10^2 \cdot 2 + 10^1 \cdot 3 + 1 \cdot 4$ = 1234 или
$1000 \cdot 1 + 100 \cdot 2 + 10 \cdot 3 + 4 = 1234$

Та же история и для двоичных чисел, только основание там 2 вместо 10.
Что означает 0b101011?

$2^5 \cdot 1 + 2^4 \cdot 0 + 2^3 \cdot 1 + 2^2 \cdot 0 + 2^1 \cdot 1 + 2^0 \cdot 1 = 43$ или
$32 \cdot 1 + 16 \cdot 0 + 8 \cdot 1 + 4 \cdot 0 + 2 \cdot 1 + 1 = 43$

Позиционную нотацию можно противопоставить непозиционной нотации, такой как римская система записи чисел
\footnote{Об эволюции способов записи чисел, см.также: \InSqBrackets{\TAOCPvolII{}, 195--213.}}.
Вероятно, человечество перешло на позиционную нотацию, потому что так проще работать с числами (сложение, умножение, итд)
на бумаге, в ручную.

Действительно, двоичные числа можно складывать, вычитать, итд, точно также, как этому обычно обучают в школах,
только доступны лишь 2 цифры.

Двоичные числа громоздки, когда их используют в исходных кодах и дампах, так что в этих случаях применяется шестнадцатеричная
система.
Используются цифры 0..9 и еще 6 латинских букв: A..F.
Каждая шестнадцатеричная цифра занимает 4 бита или 4 двоичных цифры, так что конвертировать из двоичной системы в
шестнадцатеричную и назад, можно легко вручную, или даже в уме.

\begin{center}
\begin{longtable}{ | l | l | l | }
\hline
\HeaderColor шестнадцатеричная & \HeaderColor двоичная & \HeaderColor десятичная \\
\hline
0	&0000	&0 \\
1	&0001	&1 \\
2	&0010	&2 \\
3	&0011	&3 \\
4	&0100	&4 \\
5	&0101	&5 \\
6	&0110	&6 \\
7	&0111	&7 \\
8	&1000	&8 \\
9	&1001	&9 \\
A	&1010	&10 \\
B	&1011	&11 \\
C	&1100	&12 \\
D	&1101	&13 \\
E	&1110	&14 \\
F	&1111	&15 \\
\hline
\end{longtable}
\end{center}

Как понять, какое основание используется в конкретном месте?

Десятичные числа обычно записываются как есть, т.е., 1234. Но некоторые ассемблеры позволяют подчеркивать
этот факт для ясности, и это число может быть дополнено суффиксом "d": 1234d.

К двоичным числам иногда спереди добавляют префикс "0b": 0b100110111
(В \ac{GCC} для этого есть нестандартное расширение языка
\footnote{\url{https://gcc.gnu.org/onlinedocs/gcc/Binary-constants.html}}).
Есть также еще один способ: суффикс "b", например: 100110111b.
В этой книге я буду пытаться придерживаться префикса "0b" для двоичных чисел.

Шестнадцатеричные числа имеют префикс "0x" в \CCpp и некоторых других \ac{PL}: 0x1234ABCD.
Либо они имеют суффикс "h": 1234ABCDh --- обычно так они представляются в ассемблерах и отладчиках.
Если число начинается с цифры A..F, перед ним добавляется 0: 0ABCDEFh.
Во времена 8-битных домашних компьютеров, был также способ записи чисел используя префикс \$, например, \$ABCD.
В книге я попытаюсь придерживаться префикса "0x" для шестнадцатеричных чисел.

Нужно ли учиться конвертировать числа в уме? Таблицу шестнадцатеричных чисел из одной цифры легко запомнить.
А запоминать б\'{о}льшие числа, наверное, не стоит.

Наверное, чаще всего шестнадцатеричные числа можно увидеть в \ac{URL}-ах.
Так кодируются буквы не из числа латинских.
Например:
\url{https://en.wiktionary.org/wiki/na\%C3\%AFvet\%C3\%A9} это \ac{URL} страницы в Wiktionary о слове \q{naïveté}.

\subsubsection{Восьмеричная система}

Еще одна система, которая в прошлом много использовалась в программировании это восьмеричная: есть 8 цифр (0..7) и каждая
описывает 3 бита, так что легко конвертировать числа туда и назад.
Она почти везде была заменена шестнадцатеричной, но удивительно, в *NIX имеется утилита использующаяся многими людьми,
которая принимает на вход восьмеричное число: \TT{chmod}.

\myindex{UNIX!chmod}
Как знают многие пользователи *NIX, аргумент \TT{chmod} это число из трех цифр. Первая цифра это права владельца файла,
вторая это права группы (которой файл принадлежит), третья для всех остальных.
И каждая цифра может быть представлена в двоичном виде:

\begin{center}
\begin{longtable}{ | l | l | l | }
\hline
\HeaderColor десятичная & \HeaderColor двоичная & \HeaderColor значение \\
\hline
7	&111	&\textbf{rwx} \\
6	&110	&\textbf{rw-} \\
5	&101	&\textbf{r-x} \\
4	&100	&\textbf{r-{}-} \\
3	&011	&\textbf{-wx} \\
2	&010	&\textbf{-w-} \\
1	&001	&\textbf{-{}-x} \\
0	&000	&\textbf{-{}-{}-} \\
\hline
\end{longtable}
\end{center}

Так что каждый бит привязан к флагу: read/write/execute (чтение/запись/исполнение).

И вот почему я вспомнил здесь о \TT{chmod}, это потому что всё число может быть представлено как число в восьмеричной системе.
Для примера возьмем 644.
Когда вы запускаете \TT{chmod 644 file}, вы выставляете права read/write для владельца, права read для группы, и снова,
read для всех остальных.
Сконвертируем число 644 из восьмеричной системы в двоичную, это будет \TT{110100100}, или (в группах по 3 бита) \TT{110 100 100}.

Теперь мы видим, что каждая тройка описывает права для владельца/группы/остальных:
первая это \TT{rw-}, вторая это \TT{r--} и третья это \TT{r--}.

Восьмеричная система была также популярная на старых компьютерах вроде PDP-8, потому что слово там могло содержать 12, 24 или
36 бит, и эти числа делятся на 3, так что выбор восьмеричной системы в той среде был логичен.
Сейчас, все популярные компьютеры имеют размер слова/адреса 16, 32 или 64 бита, и эти числа делятся на 4,
так что шестнадцатеричная система здесь удобнее.

Восьмеричная система поддерживается всеми стандартными компиляторами \CCpp{}.
Это иногда источник недоумения, потому что восьмеричные числа кодируются с нулем вперед, например, 0377 это 255.
И иногда, вы можете сделать опечатку, и написать "09" вместо 9, и компилятор выдаст ошибку.
GCC может выдать что-то вроде:\\
\TT{error: invalid digit "9" in octal constant}.

Также, восьмеричная система популярна в Java: когда IDA показывает строку с непечатаемыми символами,
они кодируются в восьмеричной системе вместо шестнадцатеричной.
\myindex{JAD}
Точно также себя ведет декомпилятор с Java JAD.

\subsubsection{Делимость}

Когда вы видите десятичное число вроде 120, вы можете быстро понять что оно делится на 10, потому что последняя цифра это 0.
Точно также, 123400 делится на 100, потому что две последних цифры это нули.

Точно также, шестнадцатеричное число 0x1230 делится на 0x10 (или 16), 0x123000 делится на 0x1000 (или 4096), итд.

Двоичное число 0b1000101000 делится на 0b1000 (8), итд.

Это свойство можно часто использовать, чтобы быстро понять,
что длина какого-либо блока в памяти выровнена по некоторой границе.
Например, секции в \ac{PE}-файлах почти всегда начинаются с адресов заканчивающихся 3 шестнадцатеричными нулями:
0x41000, 0x10001000, итд.
Причина в том, что почти все секции в \ac{PE} выровнены по границе 0x1000 (4096) байт.

\subsubsection{Арифметика произвольной точности и основание}

\index{RSA}
Арифметика произвольной точности (multi-precision arithmetic) может использовать огромные числа,
которые могут храниться в нескольких байтах.
Например, ключи RSA, и открытые и закрытые, могут занимать до 4096 бит и даже больше.

В \InSqBrackets{\TAOCPvolII, 265} можно найти такую идею: когда вы сохраняете число произвольной точности в нескольких байтах,
всё число может быть представлено как имеющую систему счисления по основанию $2^8=256$, и каждая цифра находится
в соответствующем байте.
Точно также, если вы сохраняете число произвольной точности в нескольких 32-битных целочисленных значениях,
каждая цифра отправляется в каждый 32-битный слот, и вы можете считать что это число записано в системе с основанием $2^{32}$.

\subsubsection{Произношение}

Числа в недесятичных системах счислениях обычно произносятся по одной цифре: ``один-ноль-ноль-один-один-...''.
Слова вроде ``десять'', ``тысяча'', итд, обычно не произносятся, потому что тогда можно спутать с десятичной системой.

\subsubsection{Числа с плавающей запятой}

Чтобы отличать числа с плавающей запятой от целочисленных, часто, в конце добавляют ``.0'',
например $0.0$, $123.0$, итд.

}
\ITA{\input{patterns/numeral_ITA}}
\DE{\input{patterns/numeral_DE}}
\FR{\input{patterns/numeral_FR}}
\PL{\input{patterns/numeral_PL}}

% chapters
\ifdefined\SPANISH
\chapter{Patrones de código}
\fi % SPANISH

\ifdefined\GERMAN
\chapter{Code-Muster}
\fi % GERMAN

\ifdefined\ENGLISH
\chapter{Code Patterns}
\fi % ENGLISH

\ifdefined\ITALIAN
\chapter{Forme di codice}
\fi % ITALIAN

\ifdefined\RUSSIAN
\chapter{Образцы кода}
\fi % RUSSIAN

\ifdefined\BRAZILIAN
\chapter{Padrões de códigos}
\fi % BRAZILIAN

\ifdefined\THAI
\chapter{รูปแบบของโค้ด}
\fi % THAI

\ifdefined\FRENCH
\chapter{Modèle de code}
\fi % FRENCH

\ifdefined\POLISH
\chapter{\PLph{}}
\fi % POLISH

% sections
\EN{\input{patterns/patterns_opt_dbg_EN}}
\ES{\input{patterns/patterns_opt_dbg_ES}}
\ITA{\input{patterns/patterns_opt_dbg_ITA}}
\PTBR{\input{patterns/patterns_opt_dbg_PTBR}}
\RU{\input{patterns/patterns_opt_dbg_RU}}
\THA{\input{patterns/patterns_opt_dbg_THA}}
\DE{\input{patterns/patterns_opt_dbg_DE}}
\FR{\input{patterns/patterns_opt_dbg_FR}}
\PL{\input{patterns/patterns_opt_dbg_PL}}

\RU{\section{Некоторые базовые понятия}}
\EN{\section{Some basics}}
\DE{\section{Einige Grundlagen}}
\FR{\section{Quelques bases}}
\ES{\section{\ESph{}}}
\ITA{\section{Alcune basi teoriche}}
\PTBR{\section{\PTBRph{}}}
\THA{\section{\THAph{}}}
\PL{\section{\PLph{}}}

% sections:
\EN{\input{patterns/intro_CPU_ISA_EN}}
\ES{\input{patterns/intro_CPU_ISA_ES}}
\ITA{\input{patterns/intro_CPU_ISA_ITA}}
\PTBR{\input{patterns/intro_CPU_ISA_PTBR}}
\RU{\input{patterns/intro_CPU_ISA_RU}}
\DE{\input{patterns/intro_CPU_ISA_DE}}
\FR{\input{patterns/intro_CPU_ISA_FR}}
\PL{\input{patterns/intro_CPU_ISA_PL}}

\EN{\input{patterns/numeral_EN}}
\RU{\input{patterns/numeral_RU}}
\ITA{\input{patterns/numeral_ITA}}
\DE{\input{patterns/numeral_DE}}
\FR{\input{patterns/numeral_FR}}
\PL{\input{patterns/numeral_PL}}

% chapters
\input{patterns/00_empty/main}
\input{patterns/011_ret/main}
\input{patterns/01_helloworld/main}
\input{patterns/015_prolog_epilogue/main}
\input{patterns/02_stack/main}
\input{patterns/03_printf/main}
\input{patterns/04_scanf/main}
\input{patterns/05_passing_arguments/main}
\input{patterns/06_return_results/main}
\input{patterns/061_pointers/main}
\input{patterns/065_GOTO/main}
\input{patterns/07_jcc/main}
\input{patterns/08_switch/main}
\input{patterns/09_loops/main}
\input{patterns/10_strings/main}
\input{patterns/11_arith_optimizations/main}
\input{patterns/12_FPU/main}
\input{patterns/13_arrays/main}
\input{patterns/14_bitfields/main}
\EN{\input{patterns/145_LCG/main_EN}}
\RU{\input{patterns/145_LCG/main_RU}}
\input{patterns/15_structs/main}
\input{patterns/17_unions/main}
\input{patterns/18_pointers_to_functions/main}
\input{patterns/185_64bit_in_32_env/main}

\EN{\input{patterns/19_SIMD/main_EN}}
\RU{\input{patterns/19_SIMD/main_RU}}
\DE{\input{patterns/19_SIMD/main_DE}}

\EN{\input{patterns/20_x64/main_EN}}
\RU{\input{patterns/20_x64/main_RU}}

\EN{\input{patterns/205_floating_SIMD/main_EN}}
\RU{\input{patterns/205_floating_SIMD/main_RU}}
\DE{\input{patterns/205_floating_SIMD/main_DE}}

\EN{\input{patterns/ARM/main_EN}}
\RU{\input{patterns/ARM/main_RU}}
\DE{\input{patterns/ARM/main_DE}}

\input{patterns/MIPS/main}

\ifdefined\SPANISH
\chapter{Patrones de código}
\fi % SPANISH

\ifdefined\GERMAN
\chapter{Code-Muster}
\fi % GERMAN

\ifdefined\ENGLISH
\chapter{Code Patterns}
\fi % ENGLISH

\ifdefined\ITALIAN
\chapter{Forme di codice}
\fi % ITALIAN

\ifdefined\RUSSIAN
\chapter{Образцы кода}
\fi % RUSSIAN

\ifdefined\BRAZILIAN
\chapter{Padrões de códigos}
\fi % BRAZILIAN

\ifdefined\THAI
\chapter{รูปแบบของโค้ด}
\fi % THAI

\ifdefined\FRENCH
\chapter{Modèle de code}
\fi % FRENCH

\ifdefined\POLISH
\chapter{\PLph{}}
\fi % POLISH

% sections
\EN{\input{patterns/patterns_opt_dbg_EN}}
\ES{\input{patterns/patterns_opt_dbg_ES}}
\ITA{\input{patterns/patterns_opt_dbg_ITA}}
\PTBR{\input{patterns/patterns_opt_dbg_PTBR}}
\RU{\input{patterns/patterns_opt_dbg_RU}}
\THA{\input{patterns/patterns_opt_dbg_THA}}
\DE{\input{patterns/patterns_opt_dbg_DE}}
\FR{\input{patterns/patterns_opt_dbg_FR}}
\PL{\input{patterns/patterns_opt_dbg_PL}}

\RU{\section{Некоторые базовые понятия}}
\EN{\section{Some basics}}
\DE{\section{Einige Grundlagen}}
\FR{\section{Quelques bases}}
\ES{\section{\ESph{}}}
\ITA{\section{Alcune basi teoriche}}
\PTBR{\section{\PTBRph{}}}
\THA{\section{\THAph{}}}
\PL{\section{\PLph{}}}

% sections:
\EN{\input{patterns/intro_CPU_ISA_EN}}
\ES{\input{patterns/intro_CPU_ISA_ES}}
\ITA{\input{patterns/intro_CPU_ISA_ITA}}
\PTBR{\input{patterns/intro_CPU_ISA_PTBR}}
\RU{\input{patterns/intro_CPU_ISA_RU}}
\DE{\input{patterns/intro_CPU_ISA_DE}}
\FR{\input{patterns/intro_CPU_ISA_FR}}
\PL{\input{patterns/intro_CPU_ISA_PL}}

\EN{\input{patterns/numeral_EN}}
\RU{\input{patterns/numeral_RU}}
\ITA{\input{patterns/numeral_ITA}}
\DE{\input{patterns/numeral_DE}}
\FR{\input{patterns/numeral_FR}}
\PL{\input{patterns/numeral_PL}}

% chapters
\input{patterns/00_empty/main}
\input{patterns/011_ret/main}
\input{patterns/01_helloworld/main}
\input{patterns/015_prolog_epilogue/main}
\input{patterns/02_stack/main}
\input{patterns/03_printf/main}
\input{patterns/04_scanf/main}
\input{patterns/05_passing_arguments/main}
\input{patterns/06_return_results/main}
\input{patterns/061_pointers/main}
\input{patterns/065_GOTO/main}
\input{patterns/07_jcc/main}
\input{patterns/08_switch/main}
\input{patterns/09_loops/main}
\input{patterns/10_strings/main}
\input{patterns/11_arith_optimizations/main}
\input{patterns/12_FPU/main}
\input{patterns/13_arrays/main}
\input{patterns/14_bitfields/main}
\EN{\input{patterns/145_LCG/main_EN}}
\RU{\input{patterns/145_LCG/main_RU}}
\input{patterns/15_structs/main}
\input{patterns/17_unions/main}
\input{patterns/18_pointers_to_functions/main}
\input{patterns/185_64bit_in_32_env/main}

\EN{\input{patterns/19_SIMD/main_EN}}
\RU{\input{patterns/19_SIMD/main_RU}}
\DE{\input{patterns/19_SIMD/main_DE}}

\EN{\input{patterns/20_x64/main_EN}}
\RU{\input{patterns/20_x64/main_RU}}

\EN{\input{patterns/205_floating_SIMD/main_EN}}
\RU{\input{patterns/205_floating_SIMD/main_RU}}
\DE{\input{patterns/205_floating_SIMD/main_DE}}

\EN{\input{patterns/ARM/main_EN}}
\RU{\input{patterns/ARM/main_RU}}
\DE{\input{patterns/ARM/main_DE}}

\input{patterns/MIPS/main}

\ifdefined\SPANISH
\chapter{Patrones de código}
\fi % SPANISH

\ifdefined\GERMAN
\chapter{Code-Muster}
\fi % GERMAN

\ifdefined\ENGLISH
\chapter{Code Patterns}
\fi % ENGLISH

\ifdefined\ITALIAN
\chapter{Forme di codice}
\fi % ITALIAN

\ifdefined\RUSSIAN
\chapter{Образцы кода}
\fi % RUSSIAN

\ifdefined\BRAZILIAN
\chapter{Padrões de códigos}
\fi % BRAZILIAN

\ifdefined\THAI
\chapter{รูปแบบของโค้ด}
\fi % THAI

\ifdefined\FRENCH
\chapter{Modèle de code}
\fi % FRENCH

\ifdefined\POLISH
\chapter{\PLph{}}
\fi % POLISH

% sections
\EN{\input{patterns/patterns_opt_dbg_EN}}
\ES{\input{patterns/patterns_opt_dbg_ES}}
\ITA{\input{patterns/patterns_opt_dbg_ITA}}
\PTBR{\input{patterns/patterns_opt_dbg_PTBR}}
\RU{\input{patterns/patterns_opt_dbg_RU}}
\THA{\input{patterns/patterns_opt_dbg_THA}}
\DE{\input{patterns/patterns_opt_dbg_DE}}
\FR{\input{patterns/patterns_opt_dbg_FR}}
\PL{\input{patterns/patterns_opt_dbg_PL}}

\RU{\section{Некоторые базовые понятия}}
\EN{\section{Some basics}}
\DE{\section{Einige Grundlagen}}
\FR{\section{Quelques bases}}
\ES{\section{\ESph{}}}
\ITA{\section{Alcune basi teoriche}}
\PTBR{\section{\PTBRph{}}}
\THA{\section{\THAph{}}}
\PL{\section{\PLph{}}}

% sections:
\EN{\input{patterns/intro_CPU_ISA_EN}}
\ES{\input{patterns/intro_CPU_ISA_ES}}
\ITA{\input{patterns/intro_CPU_ISA_ITA}}
\PTBR{\input{patterns/intro_CPU_ISA_PTBR}}
\RU{\input{patterns/intro_CPU_ISA_RU}}
\DE{\input{patterns/intro_CPU_ISA_DE}}
\FR{\input{patterns/intro_CPU_ISA_FR}}
\PL{\input{patterns/intro_CPU_ISA_PL}}

\EN{\input{patterns/numeral_EN}}
\RU{\input{patterns/numeral_RU}}
\ITA{\input{patterns/numeral_ITA}}
\DE{\input{patterns/numeral_DE}}
\FR{\input{patterns/numeral_FR}}
\PL{\input{patterns/numeral_PL}}

% chapters
\input{patterns/00_empty/main}
\input{patterns/011_ret/main}
\input{patterns/01_helloworld/main}
\input{patterns/015_prolog_epilogue/main}
\input{patterns/02_stack/main}
\input{patterns/03_printf/main}
\input{patterns/04_scanf/main}
\input{patterns/05_passing_arguments/main}
\input{patterns/06_return_results/main}
\input{patterns/061_pointers/main}
\input{patterns/065_GOTO/main}
\input{patterns/07_jcc/main}
\input{patterns/08_switch/main}
\input{patterns/09_loops/main}
\input{patterns/10_strings/main}
\input{patterns/11_arith_optimizations/main}
\input{patterns/12_FPU/main}
\input{patterns/13_arrays/main}
\input{patterns/14_bitfields/main}
\EN{\input{patterns/145_LCG/main_EN}}
\RU{\input{patterns/145_LCG/main_RU}}
\input{patterns/15_structs/main}
\input{patterns/17_unions/main}
\input{patterns/18_pointers_to_functions/main}
\input{patterns/185_64bit_in_32_env/main}

\EN{\input{patterns/19_SIMD/main_EN}}
\RU{\input{patterns/19_SIMD/main_RU}}
\DE{\input{patterns/19_SIMD/main_DE}}

\EN{\input{patterns/20_x64/main_EN}}
\RU{\input{patterns/20_x64/main_RU}}

\EN{\input{patterns/205_floating_SIMD/main_EN}}
\RU{\input{patterns/205_floating_SIMD/main_RU}}
\DE{\input{patterns/205_floating_SIMD/main_DE}}

\EN{\input{patterns/ARM/main_EN}}
\RU{\input{patterns/ARM/main_RU}}
\DE{\input{patterns/ARM/main_DE}}

\input{patterns/MIPS/main}

\ifdefined\SPANISH
\chapter{Patrones de código}
\fi % SPANISH

\ifdefined\GERMAN
\chapter{Code-Muster}
\fi % GERMAN

\ifdefined\ENGLISH
\chapter{Code Patterns}
\fi % ENGLISH

\ifdefined\ITALIAN
\chapter{Forme di codice}
\fi % ITALIAN

\ifdefined\RUSSIAN
\chapter{Образцы кода}
\fi % RUSSIAN

\ifdefined\BRAZILIAN
\chapter{Padrões de códigos}
\fi % BRAZILIAN

\ifdefined\THAI
\chapter{รูปแบบของโค้ด}
\fi % THAI

\ifdefined\FRENCH
\chapter{Modèle de code}
\fi % FRENCH

\ifdefined\POLISH
\chapter{\PLph{}}
\fi % POLISH

% sections
\EN{\input{patterns/patterns_opt_dbg_EN}}
\ES{\input{patterns/patterns_opt_dbg_ES}}
\ITA{\input{patterns/patterns_opt_dbg_ITA}}
\PTBR{\input{patterns/patterns_opt_dbg_PTBR}}
\RU{\input{patterns/patterns_opt_dbg_RU}}
\THA{\input{patterns/patterns_opt_dbg_THA}}
\DE{\input{patterns/patterns_opt_dbg_DE}}
\FR{\input{patterns/patterns_opt_dbg_FR}}
\PL{\input{patterns/patterns_opt_dbg_PL}}

\RU{\section{Некоторые базовые понятия}}
\EN{\section{Some basics}}
\DE{\section{Einige Grundlagen}}
\FR{\section{Quelques bases}}
\ES{\section{\ESph{}}}
\ITA{\section{Alcune basi teoriche}}
\PTBR{\section{\PTBRph{}}}
\THA{\section{\THAph{}}}
\PL{\section{\PLph{}}}

% sections:
\EN{\input{patterns/intro_CPU_ISA_EN}}
\ES{\input{patterns/intro_CPU_ISA_ES}}
\ITA{\input{patterns/intro_CPU_ISA_ITA}}
\PTBR{\input{patterns/intro_CPU_ISA_PTBR}}
\RU{\input{patterns/intro_CPU_ISA_RU}}
\DE{\input{patterns/intro_CPU_ISA_DE}}
\FR{\input{patterns/intro_CPU_ISA_FR}}
\PL{\input{patterns/intro_CPU_ISA_PL}}

\EN{\input{patterns/numeral_EN}}
\RU{\input{patterns/numeral_RU}}
\ITA{\input{patterns/numeral_ITA}}
\DE{\input{patterns/numeral_DE}}
\FR{\input{patterns/numeral_FR}}
\PL{\input{patterns/numeral_PL}}

% chapters
\input{patterns/00_empty/main}
\input{patterns/011_ret/main}
\input{patterns/01_helloworld/main}
\input{patterns/015_prolog_epilogue/main}
\input{patterns/02_stack/main}
\input{patterns/03_printf/main}
\input{patterns/04_scanf/main}
\input{patterns/05_passing_arguments/main}
\input{patterns/06_return_results/main}
\input{patterns/061_pointers/main}
\input{patterns/065_GOTO/main}
\input{patterns/07_jcc/main}
\input{patterns/08_switch/main}
\input{patterns/09_loops/main}
\input{patterns/10_strings/main}
\input{patterns/11_arith_optimizations/main}
\input{patterns/12_FPU/main}
\input{patterns/13_arrays/main}
\input{patterns/14_bitfields/main}
\EN{\input{patterns/145_LCG/main_EN}}
\RU{\input{patterns/145_LCG/main_RU}}
\input{patterns/15_structs/main}
\input{patterns/17_unions/main}
\input{patterns/18_pointers_to_functions/main}
\input{patterns/185_64bit_in_32_env/main}

\EN{\input{patterns/19_SIMD/main_EN}}
\RU{\input{patterns/19_SIMD/main_RU}}
\DE{\input{patterns/19_SIMD/main_DE}}

\EN{\input{patterns/20_x64/main_EN}}
\RU{\input{patterns/20_x64/main_RU}}

\EN{\input{patterns/205_floating_SIMD/main_EN}}
\RU{\input{patterns/205_floating_SIMD/main_RU}}
\DE{\input{patterns/205_floating_SIMD/main_DE}}

\EN{\input{patterns/ARM/main_EN}}
\RU{\input{patterns/ARM/main_RU}}
\DE{\input{patterns/ARM/main_DE}}

\input{patterns/MIPS/main}

\ifdefined\SPANISH
\chapter{Patrones de código}
\fi % SPANISH

\ifdefined\GERMAN
\chapter{Code-Muster}
\fi % GERMAN

\ifdefined\ENGLISH
\chapter{Code Patterns}
\fi % ENGLISH

\ifdefined\ITALIAN
\chapter{Forme di codice}
\fi % ITALIAN

\ifdefined\RUSSIAN
\chapter{Образцы кода}
\fi % RUSSIAN

\ifdefined\BRAZILIAN
\chapter{Padrões de códigos}
\fi % BRAZILIAN

\ifdefined\THAI
\chapter{รูปแบบของโค้ด}
\fi % THAI

\ifdefined\FRENCH
\chapter{Modèle de code}
\fi % FRENCH

\ifdefined\POLISH
\chapter{\PLph{}}
\fi % POLISH

% sections
\EN{\input{patterns/patterns_opt_dbg_EN}}
\ES{\input{patterns/patterns_opt_dbg_ES}}
\ITA{\input{patterns/patterns_opt_dbg_ITA}}
\PTBR{\input{patterns/patterns_opt_dbg_PTBR}}
\RU{\input{patterns/patterns_opt_dbg_RU}}
\THA{\input{patterns/patterns_opt_dbg_THA}}
\DE{\input{patterns/patterns_opt_dbg_DE}}
\FR{\input{patterns/patterns_opt_dbg_FR}}
\PL{\input{patterns/patterns_opt_dbg_PL}}

\RU{\section{Некоторые базовые понятия}}
\EN{\section{Some basics}}
\DE{\section{Einige Grundlagen}}
\FR{\section{Quelques bases}}
\ES{\section{\ESph{}}}
\ITA{\section{Alcune basi teoriche}}
\PTBR{\section{\PTBRph{}}}
\THA{\section{\THAph{}}}
\PL{\section{\PLph{}}}

% sections:
\EN{\input{patterns/intro_CPU_ISA_EN}}
\ES{\input{patterns/intro_CPU_ISA_ES}}
\ITA{\input{patterns/intro_CPU_ISA_ITA}}
\PTBR{\input{patterns/intro_CPU_ISA_PTBR}}
\RU{\input{patterns/intro_CPU_ISA_RU}}
\DE{\input{patterns/intro_CPU_ISA_DE}}
\FR{\input{patterns/intro_CPU_ISA_FR}}
\PL{\input{patterns/intro_CPU_ISA_PL}}

\EN{\input{patterns/numeral_EN}}
\RU{\input{patterns/numeral_RU}}
\ITA{\input{patterns/numeral_ITA}}
\DE{\input{patterns/numeral_DE}}
\FR{\input{patterns/numeral_FR}}
\PL{\input{patterns/numeral_PL}}

% chapters
\input{patterns/00_empty/main}
\input{patterns/011_ret/main}
\input{patterns/01_helloworld/main}
\input{patterns/015_prolog_epilogue/main}
\input{patterns/02_stack/main}
\input{patterns/03_printf/main}
\input{patterns/04_scanf/main}
\input{patterns/05_passing_arguments/main}
\input{patterns/06_return_results/main}
\input{patterns/061_pointers/main}
\input{patterns/065_GOTO/main}
\input{patterns/07_jcc/main}
\input{patterns/08_switch/main}
\input{patterns/09_loops/main}
\input{patterns/10_strings/main}
\input{patterns/11_arith_optimizations/main}
\input{patterns/12_FPU/main}
\input{patterns/13_arrays/main}
\input{patterns/14_bitfields/main}
\EN{\input{patterns/145_LCG/main_EN}}
\RU{\input{patterns/145_LCG/main_RU}}
\input{patterns/15_structs/main}
\input{patterns/17_unions/main}
\input{patterns/18_pointers_to_functions/main}
\input{patterns/185_64bit_in_32_env/main}

\EN{\input{patterns/19_SIMD/main_EN}}
\RU{\input{patterns/19_SIMD/main_RU}}
\DE{\input{patterns/19_SIMD/main_DE}}

\EN{\input{patterns/20_x64/main_EN}}
\RU{\input{patterns/20_x64/main_RU}}

\EN{\input{patterns/205_floating_SIMD/main_EN}}
\RU{\input{patterns/205_floating_SIMD/main_RU}}
\DE{\input{patterns/205_floating_SIMD/main_DE}}

\EN{\input{patterns/ARM/main_EN}}
\RU{\input{patterns/ARM/main_RU}}
\DE{\input{patterns/ARM/main_DE}}

\input{patterns/MIPS/main}

\ifdefined\SPANISH
\chapter{Patrones de código}
\fi % SPANISH

\ifdefined\GERMAN
\chapter{Code-Muster}
\fi % GERMAN

\ifdefined\ENGLISH
\chapter{Code Patterns}
\fi % ENGLISH

\ifdefined\ITALIAN
\chapter{Forme di codice}
\fi % ITALIAN

\ifdefined\RUSSIAN
\chapter{Образцы кода}
\fi % RUSSIAN

\ifdefined\BRAZILIAN
\chapter{Padrões de códigos}
\fi % BRAZILIAN

\ifdefined\THAI
\chapter{รูปแบบของโค้ด}
\fi % THAI

\ifdefined\FRENCH
\chapter{Modèle de code}
\fi % FRENCH

\ifdefined\POLISH
\chapter{\PLph{}}
\fi % POLISH

% sections
\EN{\input{patterns/patterns_opt_dbg_EN}}
\ES{\input{patterns/patterns_opt_dbg_ES}}
\ITA{\input{patterns/patterns_opt_dbg_ITA}}
\PTBR{\input{patterns/patterns_opt_dbg_PTBR}}
\RU{\input{patterns/patterns_opt_dbg_RU}}
\THA{\input{patterns/patterns_opt_dbg_THA}}
\DE{\input{patterns/patterns_opt_dbg_DE}}
\FR{\input{patterns/patterns_opt_dbg_FR}}
\PL{\input{patterns/patterns_opt_dbg_PL}}

\RU{\section{Некоторые базовые понятия}}
\EN{\section{Some basics}}
\DE{\section{Einige Grundlagen}}
\FR{\section{Quelques bases}}
\ES{\section{\ESph{}}}
\ITA{\section{Alcune basi teoriche}}
\PTBR{\section{\PTBRph{}}}
\THA{\section{\THAph{}}}
\PL{\section{\PLph{}}}

% sections:
\EN{\input{patterns/intro_CPU_ISA_EN}}
\ES{\input{patterns/intro_CPU_ISA_ES}}
\ITA{\input{patterns/intro_CPU_ISA_ITA}}
\PTBR{\input{patterns/intro_CPU_ISA_PTBR}}
\RU{\input{patterns/intro_CPU_ISA_RU}}
\DE{\input{patterns/intro_CPU_ISA_DE}}
\FR{\input{patterns/intro_CPU_ISA_FR}}
\PL{\input{patterns/intro_CPU_ISA_PL}}

\EN{\input{patterns/numeral_EN}}
\RU{\input{patterns/numeral_RU}}
\ITA{\input{patterns/numeral_ITA}}
\DE{\input{patterns/numeral_DE}}
\FR{\input{patterns/numeral_FR}}
\PL{\input{patterns/numeral_PL}}

% chapters
\input{patterns/00_empty/main}
\input{patterns/011_ret/main}
\input{patterns/01_helloworld/main}
\input{patterns/015_prolog_epilogue/main}
\input{patterns/02_stack/main}
\input{patterns/03_printf/main}
\input{patterns/04_scanf/main}
\input{patterns/05_passing_arguments/main}
\input{patterns/06_return_results/main}
\input{patterns/061_pointers/main}
\input{patterns/065_GOTO/main}
\input{patterns/07_jcc/main}
\input{patterns/08_switch/main}
\input{patterns/09_loops/main}
\input{patterns/10_strings/main}
\input{patterns/11_arith_optimizations/main}
\input{patterns/12_FPU/main}
\input{patterns/13_arrays/main}
\input{patterns/14_bitfields/main}
\EN{\input{patterns/145_LCG/main_EN}}
\RU{\input{patterns/145_LCG/main_RU}}
\input{patterns/15_structs/main}
\input{patterns/17_unions/main}
\input{patterns/18_pointers_to_functions/main}
\input{patterns/185_64bit_in_32_env/main}

\EN{\input{patterns/19_SIMD/main_EN}}
\RU{\input{patterns/19_SIMD/main_RU}}
\DE{\input{patterns/19_SIMD/main_DE}}

\EN{\input{patterns/20_x64/main_EN}}
\RU{\input{patterns/20_x64/main_RU}}

\EN{\input{patterns/205_floating_SIMD/main_EN}}
\RU{\input{patterns/205_floating_SIMD/main_RU}}
\DE{\input{patterns/205_floating_SIMD/main_DE}}

\EN{\input{patterns/ARM/main_EN}}
\RU{\input{patterns/ARM/main_RU}}
\DE{\input{patterns/ARM/main_DE}}

\input{patterns/MIPS/main}

\ifdefined\SPANISH
\chapter{Patrones de código}
\fi % SPANISH

\ifdefined\GERMAN
\chapter{Code-Muster}
\fi % GERMAN

\ifdefined\ENGLISH
\chapter{Code Patterns}
\fi % ENGLISH

\ifdefined\ITALIAN
\chapter{Forme di codice}
\fi % ITALIAN

\ifdefined\RUSSIAN
\chapter{Образцы кода}
\fi % RUSSIAN

\ifdefined\BRAZILIAN
\chapter{Padrões de códigos}
\fi % BRAZILIAN

\ifdefined\THAI
\chapter{รูปแบบของโค้ด}
\fi % THAI

\ifdefined\FRENCH
\chapter{Modèle de code}
\fi % FRENCH

\ifdefined\POLISH
\chapter{\PLph{}}
\fi % POLISH

% sections
\EN{\input{patterns/patterns_opt_dbg_EN}}
\ES{\input{patterns/patterns_opt_dbg_ES}}
\ITA{\input{patterns/patterns_opt_dbg_ITA}}
\PTBR{\input{patterns/patterns_opt_dbg_PTBR}}
\RU{\input{patterns/patterns_opt_dbg_RU}}
\THA{\input{patterns/patterns_opt_dbg_THA}}
\DE{\input{patterns/patterns_opt_dbg_DE}}
\FR{\input{patterns/patterns_opt_dbg_FR}}
\PL{\input{patterns/patterns_opt_dbg_PL}}

\RU{\section{Некоторые базовые понятия}}
\EN{\section{Some basics}}
\DE{\section{Einige Grundlagen}}
\FR{\section{Quelques bases}}
\ES{\section{\ESph{}}}
\ITA{\section{Alcune basi teoriche}}
\PTBR{\section{\PTBRph{}}}
\THA{\section{\THAph{}}}
\PL{\section{\PLph{}}}

% sections:
\EN{\input{patterns/intro_CPU_ISA_EN}}
\ES{\input{patterns/intro_CPU_ISA_ES}}
\ITA{\input{patterns/intro_CPU_ISA_ITA}}
\PTBR{\input{patterns/intro_CPU_ISA_PTBR}}
\RU{\input{patterns/intro_CPU_ISA_RU}}
\DE{\input{patterns/intro_CPU_ISA_DE}}
\FR{\input{patterns/intro_CPU_ISA_FR}}
\PL{\input{patterns/intro_CPU_ISA_PL}}

\EN{\input{patterns/numeral_EN}}
\RU{\input{patterns/numeral_RU}}
\ITA{\input{patterns/numeral_ITA}}
\DE{\input{patterns/numeral_DE}}
\FR{\input{patterns/numeral_FR}}
\PL{\input{patterns/numeral_PL}}

% chapters
\input{patterns/00_empty/main}
\input{patterns/011_ret/main}
\input{patterns/01_helloworld/main}
\input{patterns/015_prolog_epilogue/main}
\input{patterns/02_stack/main}
\input{patterns/03_printf/main}
\input{patterns/04_scanf/main}
\input{patterns/05_passing_arguments/main}
\input{patterns/06_return_results/main}
\input{patterns/061_pointers/main}
\input{patterns/065_GOTO/main}
\input{patterns/07_jcc/main}
\input{patterns/08_switch/main}
\input{patterns/09_loops/main}
\input{patterns/10_strings/main}
\input{patterns/11_arith_optimizations/main}
\input{patterns/12_FPU/main}
\input{patterns/13_arrays/main}
\input{patterns/14_bitfields/main}
\EN{\input{patterns/145_LCG/main_EN}}
\RU{\input{patterns/145_LCG/main_RU}}
\input{patterns/15_structs/main}
\input{patterns/17_unions/main}
\input{patterns/18_pointers_to_functions/main}
\input{patterns/185_64bit_in_32_env/main}

\EN{\input{patterns/19_SIMD/main_EN}}
\RU{\input{patterns/19_SIMD/main_RU}}
\DE{\input{patterns/19_SIMD/main_DE}}

\EN{\input{patterns/20_x64/main_EN}}
\RU{\input{patterns/20_x64/main_RU}}

\EN{\input{patterns/205_floating_SIMD/main_EN}}
\RU{\input{patterns/205_floating_SIMD/main_RU}}
\DE{\input{patterns/205_floating_SIMD/main_DE}}

\EN{\input{patterns/ARM/main_EN}}
\RU{\input{patterns/ARM/main_RU}}
\DE{\input{patterns/ARM/main_DE}}

\input{patterns/MIPS/main}

\ifdefined\SPANISH
\chapter{Patrones de código}
\fi % SPANISH

\ifdefined\GERMAN
\chapter{Code-Muster}
\fi % GERMAN

\ifdefined\ENGLISH
\chapter{Code Patterns}
\fi % ENGLISH

\ifdefined\ITALIAN
\chapter{Forme di codice}
\fi % ITALIAN

\ifdefined\RUSSIAN
\chapter{Образцы кода}
\fi % RUSSIAN

\ifdefined\BRAZILIAN
\chapter{Padrões de códigos}
\fi % BRAZILIAN

\ifdefined\THAI
\chapter{รูปแบบของโค้ด}
\fi % THAI

\ifdefined\FRENCH
\chapter{Modèle de code}
\fi % FRENCH

\ifdefined\POLISH
\chapter{\PLph{}}
\fi % POLISH

% sections
\EN{\input{patterns/patterns_opt_dbg_EN}}
\ES{\input{patterns/patterns_opt_dbg_ES}}
\ITA{\input{patterns/patterns_opt_dbg_ITA}}
\PTBR{\input{patterns/patterns_opt_dbg_PTBR}}
\RU{\input{patterns/patterns_opt_dbg_RU}}
\THA{\input{patterns/patterns_opt_dbg_THA}}
\DE{\input{patterns/patterns_opt_dbg_DE}}
\FR{\input{patterns/patterns_opt_dbg_FR}}
\PL{\input{patterns/patterns_opt_dbg_PL}}

\RU{\section{Некоторые базовые понятия}}
\EN{\section{Some basics}}
\DE{\section{Einige Grundlagen}}
\FR{\section{Quelques bases}}
\ES{\section{\ESph{}}}
\ITA{\section{Alcune basi teoriche}}
\PTBR{\section{\PTBRph{}}}
\THA{\section{\THAph{}}}
\PL{\section{\PLph{}}}

% sections:
\EN{\input{patterns/intro_CPU_ISA_EN}}
\ES{\input{patterns/intro_CPU_ISA_ES}}
\ITA{\input{patterns/intro_CPU_ISA_ITA}}
\PTBR{\input{patterns/intro_CPU_ISA_PTBR}}
\RU{\input{patterns/intro_CPU_ISA_RU}}
\DE{\input{patterns/intro_CPU_ISA_DE}}
\FR{\input{patterns/intro_CPU_ISA_FR}}
\PL{\input{patterns/intro_CPU_ISA_PL}}

\EN{\input{patterns/numeral_EN}}
\RU{\input{patterns/numeral_RU}}
\ITA{\input{patterns/numeral_ITA}}
\DE{\input{patterns/numeral_DE}}
\FR{\input{patterns/numeral_FR}}
\PL{\input{patterns/numeral_PL}}

% chapters
\input{patterns/00_empty/main}
\input{patterns/011_ret/main}
\input{patterns/01_helloworld/main}
\input{patterns/015_prolog_epilogue/main}
\input{patterns/02_stack/main}
\input{patterns/03_printf/main}
\input{patterns/04_scanf/main}
\input{patterns/05_passing_arguments/main}
\input{patterns/06_return_results/main}
\input{patterns/061_pointers/main}
\input{patterns/065_GOTO/main}
\input{patterns/07_jcc/main}
\input{patterns/08_switch/main}
\input{patterns/09_loops/main}
\input{patterns/10_strings/main}
\input{patterns/11_arith_optimizations/main}
\input{patterns/12_FPU/main}
\input{patterns/13_arrays/main}
\input{patterns/14_bitfields/main}
\EN{\input{patterns/145_LCG/main_EN}}
\RU{\input{patterns/145_LCG/main_RU}}
\input{patterns/15_structs/main}
\input{patterns/17_unions/main}
\input{patterns/18_pointers_to_functions/main}
\input{patterns/185_64bit_in_32_env/main}

\EN{\input{patterns/19_SIMD/main_EN}}
\RU{\input{patterns/19_SIMD/main_RU}}
\DE{\input{patterns/19_SIMD/main_DE}}

\EN{\input{patterns/20_x64/main_EN}}
\RU{\input{patterns/20_x64/main_RU}}

\EN{\input{patterns/205_floating_SIMD/main_EN}}
\RU{\input{patterns/205_floating_SIMD/main_RU}}
\DE{\input{patterns/205_floating_SIMD/main_DE}}

\EN{\input{patterns/ARM/main_EN}}
\RU{\input{patterns/ARM/main_RU}}
\DE{\input{patterns/ARM/main_DE}}

\input{patterns/MIPS/main}

\ifdefined\SPANISH
\chapter{Patrones de código}
\fi % SPANISH

\ifdefined\GERMAN
\chapter{Code-Muster}
\fi % GERMAN

\ifdefined\ENGLISH
\chapter{Code Patterns}
\fi % ENGLISH

\ifdefined\ITALIAN
\chapter{Forme di codice}
\fi % ITALIAN

\ifdefined\RUSSIAN
\chapter{Образцы кода}
\fi % RUSSIAN

\ifdefined\BRAZILIAN
\chapter{Padrões de códigos}
\fi % BRAZILIAN

\ifdefined\THAI
\chapter{รูปแบบของโค้ด}
\fi % THAI

\ifdefined\FRENCH
\chapter{Modèle de code}
\fi % FRENCH

\ifdefined\POLISH
\chapter{\PLph{}}
\fi % POLISH

% sections
\EN{\input{patterns/patterns_opt_dbg_EN}}
\ES{\input{patterns/patterns_opt_dbg_ES}}
\ITA{\input{patterns/patterns_opt_dbg_ITA}}
\PTBR{\input{patterns/patterns_opt_dbg_PTBR}}
\RU{\input{patterns/patterns_opt_dbg_RU}}
\THA{\input{patterns/patterns_opt_dbg_THA}}
\DE{\input{patterns/patterns_opt_dbg_DE}}
\FR{\input{patterns/patterns_opt_dbg_FR}}
\PL{\input{patterns/patterns_opt_dbg_PL}}

\RU{\section{Некоторые базовые понятия}}
\EN{\section{Some basics}}
\DE{\section{Einige Grundlagen}}
\FR{\section{Quelques bases}}
\ES{\section{\ESph{}}}
\ITA{\section{Alcune basi teoriche}}
\PTBR{\section{\PTBRph{}}}
\THA{\section{\THAph{}}}
\PL{\section{\PLph{}}}

% sections:
\EN{\input{patterns/intro_CPU_ISA_EN}}
\ES{\input{patterns/intro_CPU_ISA_ES}}
\ITA{\input{patterns/intro_CPU_ISA_ITA}}
\PTBR{\input{patterns/intro_CPU_ISA_PTBR}}
\RU{\input{patterns/intro_CPU_ISA_RU}}
\DE{\input{patterns/intro_CPU_ISA_DE}}
\FR{\input{patterns/intro_CPU_ISA_FR}}
\PL{\input{patterns/intro_CPU_ISA_PL}}

\EN{\input{patterns/numeral_EN}}
\RU{\input{patterns/numeral_RU}}
\ITA{\input{patterns/numeral_ITA}}
\DE{\input{patterns/numeral_DE}}
\FR{\input{patterns/numeral_FR}}
\PL{\input{patterns/numeral_PL}}

% chapters
\input{patterns/00_empty/main}
\input{patterns/011_ret/main}
\input{patterns/01_helloworld/main}
\input{patterns/015_prolog_epilogue/main}
\input{patterns/02_stack/main}
\input{patterns/03_printf/main}
\input{patterns/04_scanf/main}
\input{patterns/05_passing_arguments/main}
\input{patterns/06_return_results/main}
\input{patterns/061_pointers/main}
\input{patterns/065_GOTO/main}
\input{patterns/07_jcc/main}
\input{patterns/08_switch/main}
\input{patterns/09_loops/main}
\input{patterns/10_strings/main}
\input{patterns/11_arith_optimizations/main}
\input{patterns/12_FPU/main}
\input{patterns/13_arrays/main}
\input{patterns/14_bitfields/main}
\EN{\input{patterns/145_LCG/main_EN}}
\RU{\input{patterns/145_LCG/main_RU}}
\input{patterns/15_structs/main}
\input{patterns/17_unions/main}
\input{patterns/18_pointers_to_functions/main}
\input{patterns/185_64bit_in_32_env/main}

\EN{\input{patterns/19_SIMD/main_EN}}
\RU{\input{patterns/19_SIMD/main_RU}}
\DE{\input{patterns/19_SIMD/main_DE}}

\EN{\input{patterns/20_x64/main_EN}}
\RU{\input{patterns/20_x64/main_RU}}

\EN{\input{patterns/205_floating_SIMD/main_EN}}
\RU{\input{patterns/205_floating_SIMD/main_RU}}
\DE{\input{patterns/205_floating_SIMD/main_DE}}

\EN{\input{patterns/ARM/main_EN}}
\RU{\input{patterns/ARM/main_RU}}
\DE{\input{patterns/ARM/main_DE}}

\input{patterns/MIPS/main}

\ifdefined\SPANISH
\chapter{Patrones de código}
\fi % SPANISH

\ifdefined\GERMAN
\chapter{Code-Muster}
\fi % GERMAN

\ifdefined\ENGLISH
\chapter{Code Patterns}
\fi % ENGLISH

\ifdefined\ITALIAN
\chapter{Forme di codice}
\fi % ITALIAN

\ifdefined\RUSSIAN
\chapter{Образцы кода}
\fi % RUSSIAN

\ifdefined\BRAZILIAN
\chapter{Padrões de códigos}
\fi % BRAZILIAN

\ifdefined\THAI
\chapter{รูปแบบของโค้ด}
\fi % THAI

\ifdefined\FRENCH
\chapter{Modèle de code}
\fi % FRENCH

\ifdefined\POLISH
\chapter{\PLph{}}
\fi % POLISH

% sections
\EN{\input{patterns/patterns_opt_dbg_EN}}
\ES{\input{patterns/patterns_opt_dbg_ES}}
\ITA{\input{patterns/patterns_opt_dbg_ITA}}
\PTBR{\input{patterns/patterns_opt_dbg_PTBR}}
\RU{\input{patterns/patterns_opt_dbg_RU}}
\THA{\input{patterns/patterns_opt_dbg_THA}}
\DE{\input{patterns/patterns_opt_dbg_DE}}
\FR{\input{patterns/patterns_opt_dbg_FR}}
\PL{\input{patterns/patterns_opt_dbg_PL}}

\RU{\section{Некоторые базовые понятия}}
\EN{\section{Some basics}}
\DE{\section{Einige Grundlagen}}
\FR{\section{Quelques bases}}
\ES{\section{\ESph{}}}
\ITA{\section{Alcune basi teoriche}}
\PTBR{\section{\PTBRph{}}}
\THA{\section{\THAph{}}}
\PL{\section{\PLph{}}}

% sections:
\EN{\input{patterns/intro_CPU_ISA_EN}}
\ES{\input{patterns/intro_CPU_ISA_ES}}
\ITA{\input{patterns/intro_CPU_ISA_ITA}}
\PTBR{\input{patterns/intro_CPU_ISA_PTBR}}
\RU{\input{patterns/intro_CPU_ISA_RU}}
\DE{\input{patterns/intro_CPU_ISA_DE}}
\FR{\input{patterns/intro_CPU_ISA_FR}}
\PL{\input{patterns/intro_CPU_ISA_PL}}

\EN{\input{patterns/numeral_EN}}
\RU{\input{patterns/numeral_RU}}
\ITA{\input{patterns/numeral_ITA}}
\DE{\input{patterns/numeral_DE}}
\FR{\input{patterns/numeral_FR}}
\PL{\input{patterns/numeral_PL}}

% chapters
\input{patterns/00_empty/main}
\input{patterns/011_ret/main}
\input{patterns/01_helloworld/main}
\input{patterns/015_prolog_epilogue/main}
\input{patterns/02_stack/main}
\input{patterns/03_printf/main}
\input{patterns/04_scanf/main}
\input{patterns/05_passing_arguments/main}
\input{patterns/06_return_results/main}
\input{patterns/061_pointers/main}
\input{patterns/065_GOTO/main}
\input{patterns/07_jcc/main}
\input{patterns/08_switch/main}
\input{patterns/09_loops/main}
\input{patterns/10_strings/main}
\input{patterns/11_arith_optimizations/main}
\input{patterns/12_FPU/main}
\input{patterns/13_arrays/main}
\input{patterns/14_bitfields/main}
\EN{\input{patterns/145_LCG/main_EN}}
\RU{\input{patterns/145_LCG/main_RU}}
\input{patterns/15_structs/main}
\input{patterns/17_unions/main}
\input{patterns/18_pointers_to_functions/main}
\input{patterns/185_64bit_in_32_env/main}

\EN{\input{patterns/19_SIMD/main_EN}}
\RU{\input{patterns/19_SIMD/main_RU}}
\DE{\input{patterns/19_SIMD/main_DE}}

\EN{\input{patterns/20_x64/main_EN}}
\RU{\input{patterns/20_x64/main_RU}}

\EN{\input{patterns/205_floating_SIMD/main_EN}}
\RU{\input{patterns/205_floating_SIMD/main_RU}}
\DE{\input{patterns/205_floating_SIMD/main_DE}}

\EN{\input{patterns/ARM/main_EN}}
\RU{\input{patterns/ARM/main_RU}}
\DE{\input{patterns/ARM/main_DE}}

\input{patterns/MIPS/main}

\ifdefined\SPANISH
\chapter{Patrones de código}
\fi % SPANISH

\ifdefined\GERMAN
\chapter{Code-Muster}
\fi % GERMAN

\ifdefined\ENGLISH
\chapter{Code Patterns}
\fi % ENGLISH

\ifdefined\ITALIAN
\chapter{Forme di codice}
\fi % ITALIAN

\ifdefined\RUSSIAN
\chapter{Образцы кода}
\fi % RUSSIAN

\ifdefined\BRAZILIAN
\chapter{Padrões de códigos}
\fi % BRAZILIAN

\ifdefined\THAI
\chapter{รูปแบบของโค้ด}
\fi % THAI

\ifdefined\FRENCH
\chapter{Modèle de code}
\fi % FRENCH

\ifdefined\POLISH
\chapter{\PLph{}}
\fi % POLISH

% sections
\EN{\input{patterns/patterns_opt_dbg_EN}}
\ES{\input{patterns/patterns_opt_dbg_ES}}
\ITA{\input{patterns/patterns_opt_dbg_ITA}}
\PTBR{\input{patterns/patterns_opt_dbg_PTBR}}
\RU{\input{patterns/patterns_opt_dbg_RU}}
\THA{\input{patterns/patterns_opt_dbg_THA}}
\DE{\input{patterns/patterns_opt_dbg_DE}}
\FR{\input{patterns/patterns_opt_dbg_FR}}
\PL{\input{patterns/patterns_opt_dbg_PL}}

\RU{\section{Некоторые базовые понятия}}
\EN{\section{Some basics}}
\DE{\section{Einige Grundlagen}}
\FR{\section{Quelques bases}}
\ES{\section{\ESph{}}}
\ITA{\section{Alcune basi teoriche}}
\PTBR{\section{\PTBRph{}}}
\THA{\section{\THAph{}}}
\PL{\section{\PLph{}}}

% sections:
\EN{\input{patterns/intro_CPU_ISA_EN}}
\ES{\input{patterns/intro_CPU_ISA_ES}}
\ITA{\input{patterns/intro_CPU_ISA_ITA}}
\PTBR{\input{patterns/intro_CPU_ISA_PTBR}}
\RU{\input{patterns/intro_CPU_ISA_RU}}
\DE{\input{patterns/intro_CPU_ISA_DE}}
\FR{\input{patterns/intro_CPU_ISA_FR}}
\PL{\input{patterns/intro_CPU_ISA_PL}}

\EN{\input{patterns/numeral_EN}}
\RU{\input{patterns/numeral_RU}}
\ITA{\input{patterns/numeral_ITA}}
\DE{\input{patterns/numeral_DE}}
\FR{\input{patterns/numeral_FR}}
\PL{\input{patterns/numeral_PL}}

% chapters
\input{patterns/00_empty/main}
\input{patterns/011_ret/main}
\input{patterns/01_helloworld/main}
\input{patterns/015_prolog_epilogue/main}
\input{patterns/02_stack/main}
\input{patterns/03_printf/main}
\input{patterns/04_scanf/main}
\input{patterns/05_passing_arguments/main}
\input{patterns/06_return_results/main}
\input{patterns/061_pointers/main}
\input{patterns/065_GOTO/main}
\input{patterns/07_jcc/main}
\input{patterns/08_switch/main}
\input{patterns/09_loops/main}
\input{patterns/10_strings/main}
\input{patterns/11_arith_optimizations/main}
\input{patterns/12_FPU/main}
\input{patterns/13_arrays/main}
\input{patterns/14_bitfields/main}
\EN{\input{patterns/145_LCG/main_EN}}
\RU{\input{patterns/145_LCG/main_RU}}
\input{patterns/15_structs/main}
\input{patterns/17_unions/main}
\input{patterns/18_pointers_to_functions/main}
\input{patterns/185_64bit_in_32_env/main}

\EN{\input{patterns/19_SIMD/main_EN}}
\RU{\input{patterns/19_SIMD/main_RU}}
\DE{\input{patterns/19_SIMD/main_DE}}

\EN{\input{patterns/20_x64/main_EN}}
\RU{\input{patterns/20_x64/main_RU}}

\EN{\input{patterns/205_floating_SIMD/main_EN}}
\RU{\input{patterns/205_floating_SIMD/main_RU}}
\DE{\input{patterns/205_floating_SIMD/main_DE}}

\EN{\input{patterns/ARM/main_EN}}
\RU{\input{patterns/ARM/main_RU}}
\DE{\input{patterns/ARM/main_DE}}

\input{patterns/MIPS/main}

\ifdefined\SPANISH
\chapter{Patrones de código}
\fi % SPANISH

\ifdefined\GERMAN
\chapter{Code-Muster}
\fi % GERMAN

\ifdefined\ENGLISH
\chapter{Code Patterns}
\fi % ENGLISH

\ifdefined\ITALIAN
\chapter{Forme di codice}
\fi % ITALIAN

\ifdefined\RUSSIAN
\chapter{Образцы кода}
\fi % RUSSIAN

\ifdefined\BRAZILIAN
\chapter{Padrões de códigos}
\fi % BRAZILIAN

\ifdefined\THAI
\chapter{รูปแบบของโค้ด}
\fi % THAI

\ifdefined\FRENCH
\chapter{Modèle de code}
\fi % FRENCH

\ifdefined\POLISH
\chapter{\PLph{}}
\fi % POLISH

% sections
\EN{\input{patterns/patterns_opt_dbg_EN}}
\ES{\input{patterns/patterns_opt_dbg_ES}}
\ITA{\input{patterns/patterns_opt_dbg_ITA}}
\PTBR{\input{patterns/patterns_opt_dbg_PTBR}}
\RU{\input{patterns/patterns_opt_dbg_RU}}
\THA{\input{patterns/patterns_opt_dbg_THA}}
\DE{\input{patterns/patterns_opt_dbg_DE}}
\FR{\input{patterns/patterns_opt_dbg_FR}}
\PL{\input{patterns/patterns_opt_dbg_PL}}

\RU{\section{Некоторые базовые понятия}}
\EN{\section{Some basics}}
\DE{\section{Einige Grundlagen}}
\FR{\section{Quelques bases}}
\ES{\section{\ESph{}}}
\ITA{\section{Alcune basi teoriche}}
\PTBR{\section{\PTBRph{}}}
\THA{\section{\THAph{}}}
\PL{\section{\PLph{}}}

% sections:
\EN{\input{patterns/intro_CPU_ISA_EN}}
\ES{\input{patterns/intro_CPU_ISA_ES}}
\ITA{\input{patterns/intro_CPU_ISA_ITA}}
\PTBR{\input{patterns/intro_CPU_ISA_PTBR}}
\RU{\input{patterns/intro_CPU_ISA_RU}}
\DE{\input{patterns/intro_CPU_ISA_DE}}
\FR{\input{patterns/intro_CPU_ISA_FR}}
\PL{\input{patterns/intro_CPU_ISA_PL}}

\EN{\input{patterns/numeral_EN}}
\RU{\input{patterns/numeral_RU}}
\ITA{\input{patterns/numeral_ITA}}
\DE{\input{patterns/numeral_DE}}
\FR{\input{patterns/numeral_FR}}
\PL{\input{patterns/numeral_PL}}

% chapters
\input{patterns/00_empty/main}
\input{patterns/011_ret/main}
\input{patterns/01_helloworld/main}
\input{patterns/015_prolog_epilogue/main}
\input{patterns/02_stack/main}
\input{patterns/03_printf/main}
\input{patterns/04_scanf/main}
\input{patterns/05_passing_arguments/main}
\input{patterns/06_return_results/main}
\input{patterns/061_pointers/main}
\input{patterns/065_GOTO/main}
\input{patterns/07_jcc/main}
\input{patterns/08_switch/main}
\input{patterns/09_loops/main}
\input{patterns/10_strings/main}
\input{patterns/11_arith_optimizations/main}
\input{patterns/12_FPU/main}
\input{patterns/13_arrays/main}
\input{patterns/14_bitfields/main}
\EN{\input{patterns/145_LCG/main_EN}}
\RU{\input{patterns/145_LCG/main_RU}}
\input{patterns/15_structs/main}
\input{patterns/17_unions/main}
\input{patterns/18_pointers_to_functions/main}
\input{patterns/185_64bit_in_32_env/main}

\EN{\input{patterns/19_SIMD/main_EN}}
\RU{\input{patterns/19_SIMD/main_RU}}
\DE{\input{patterns/19_SIMD/main_DE}}

\EN{\input{patterns/20_x64/main_EN}}
\RU{\input{patterns/20_x64/main_RU}}

\EN{\input{patterns/205_floating_SIMD/main_EN}}
\RU{\input{patterns/205_floating_SIMD/main_RU}}
\DE{\input{patterns/205_floating_SIMD/main_DE}}

\EN{\input{patterns/ARM/main_EN}}
\RU{\input{patterns/ARM/main_RU}}
\DE{\input{patterns/ARM/main_DE}}

\input{patterns/MIPS/main}

\ifdefined\SPANISH
\chapter{Patrones de código}
\fi % SPANISH

\ifdefined\GERMAN
\chapter{Code-Muster}
\fi % GERMAN

\ifdefined\ENGLISH
\chapter{Code Patterns}
\fi % ENGLISH

\ifdefined\ITALIAN
\chapter{Forme di codice}
\fi % ITALIAN

\ifdefined\RUSSIAN
\chapter{Образцы кода}
\fi % RUSSIAN

\ifdefined\BRAZILIAN
\chapter{Padrões de códigos}
\fi % BRAZILIAN

\ifdefined\THAI
\chapter{รูปแบบของโค้ด}
\fi % THAI

\ifdefined\FRENCH
\chapter{Modèle de code}
\fi % FRENCH

\ifdefined\POLISH
\chapter{\PLph{}}
\fi % POLISH

% sections
\EN{\input{patterns/patterns_opt_dbg_EN}}
\ES{\input{patterns/patterns_opt_dbg_ES}}
\ITA{\input{patterns/patterns_opt_dbg_ITA}}
\PTBR{\input{patterns/patterns_opt_dbg_PTBR}}
\RU{\input{patterns/patterns_opt_dbg_RU}}
\THA{\input{patterns/patterns_opt_dbg_THA}}
\DE{\input{patterns/patterns_opt_dbg_DE}}
\FR{\input{patterns/patterns_opt_dbg_FR}}
\PL{\input{patterns/patterns_opt_dbg_PL}}

\RU{\section{Некоторые базовые понятия}}
\EN{\section{Some basics}}
\DE{\section{Einige Grundlagen}}
\FR{\section{Quelques bases}}
\ES{\section{\ESph{}}}
\ITA{\section{Alcune basi teoriche}}
\PTBR{\section{\PTBRph{}}}
\THA{\section{\THAph{}}}
\PL{\section{\PLph{}}}

% sections:
\EN{\input{patterns/intro_CPU_ISA_EN}}
\ES{\input{patterns/intro_CPU_ISA_ES}}
\ITA{\input{patterns/intro_CPU_ISA_ITA}}
\PTBR{\input{patterns/intro_CPU_ISA_PTBR}}
\RU{\input{patterns/intro_CPU_ISA_RU}}
\DE{\input{patterns/intro_CPU_ISA_DE}}
\FR{\input{patterns/intro_CPU_ISA_FR}}
\PL{\input{patterns/intro_CPU_ISA_PL}}

\EN{\input{patterns/numeral_EN}}
\RU{\input{patterns/numeral_RU}}
\ITA{\input{patterns/numeral_ITA}}
\DE{\input{patterns/numeral_DE}}
\FR{\input{patterns/numeral_FR}}
\PL{\input{patterns/numeral_PL}}

% chapters
\input{patterns/00_empty/main}
\input{patterns/011_ret/main}
\input{patterns/01_helloworld/main}
\input{patterns/015_prolog_epilogue/main}
\input{patterns/02_stack/main}
\input{patterns/03_printf/main}
\input{patterns/04_scanf/main}
\input{patterns/05_passing_arguments/main}
\input{patterns/06_return_results/main}
\input{patterns/061_pointers/main}
\input{patterns/065_GOTO/main}
\input{patterns/07_jcc/main}
\input{patterns/08_switch/main}
\input{patterns/09_loops/main}
\input{patterns/10_strings/main}
\input{patterns/11_arith_optimizations/main}
\input{patterns/12_FPU/main}
\input{patterns/13_arrays/main}
\input{patterns/14_bitfields/main}
\EN{\input{patterns/145_LCG/main_EN}}
\RU{\input{patterns/145_LCG/main_RU}}
\input{patterns/15_structs/main}
\input{patterns/17_unions/main}
\input{patterns/18_pointers_to_functions/main}
\input{patterns/185_64bit_in_32_env/main}

\EN{\input{patterns/19_SIMD/main_EN}}
\RU{\input{patterns/19_SIMD/main_RU}}
\DE{\input{patterns/19_SIMD/main_DE}}

\EN{\input{patterns/20_x64/main_EN}}
\RU{\input{patterns/20_x64/main_RU}}

\EN{\input{patterns/205_floating_SIMD/main_EN}}
\RU{\input{patterns/205_floating_SIMD/main_RU}}
\DE{\input{patterns/205_floating_SIMD/main_DE}}

\EN{\input{patterns/ARM/main_EN}}
\RU{\input{patterns/ARM/main_RU}}
\DE{\input{patterns/ARM/main_DE}}

\input{patterns/MIPS/main}

\ifdefined\SPANISH
\chapter{Patrones de código}
\fi % SPANISH

\ifdefined\GERMAN
\chapter{Code-Muster}
\fi % GERMAN

\ifdefined\ENGLISH
\chapter{Code Patterns}
\fi % ENGLISH

\ifdefined\ITALIAN
\chapter{Forme di codice}
\fi % ITALIAN

\ifdefined\RUSSIAN
\chapter{Образцы кода}
\fi % RUSSIAN

\ifdefined\BRAZILIAN
\chapter{Padrões de códigos}
\fi % BRAZILIAN

\ifdefined\THAI
\chapter{รูปแบบของโค้ด}
\fi % THAI

\ifdefined\FRENCH
\chapter{Modèle de code}
\fi % FRENCH

\ifdefined\POLISH
\chapter{\PLph{}}
\fi % POLISH

% sections
\EN{\input{patterns/patterns_opt_dbg_EN}}
\ES{\input{patterns/patterns_opt_dbg_ES}}
\ITA{\input{patterns/patterns_opt_dbg_ITA}}
\PTBR{\input{patterns/patterns_opt_dbg_PTBR}}
\RU{\input{patterns/patterns_opt_dbg_RU}}
\THA{\input{patterns/patterns_opt_dbg_THA}}
\DE{\input{patterns/patterns_opt_dbg_DE}}
\FR{\input{patterns/patterns_opt_dbg_FR}}
\PL{\input{patterns/patterns_opt_dbg_PL}}

\RU{\section{Некоторые базовые понятия}}
\EN{\section{Some basics}}
\DE{\section{Einige Grundlagen}}
\FR{\section{Quelques bases}}
\ES{\section{\ESph{}}}
\ITA{\section{Alcune basi teoriche}}
\PTBR{\section{\PTBRph{}}}
\THA{\section{\THAph{}}}
\PL{\section{\PLph{}}}

% sections:
\EN{\input{patterns/intro_CPU_ISA_EN}}
\ES{\input{patterns/intro_CPU_ISA_ES}}
\ITA{\input{patterns/intro_CPU_ISA_ITA}}
\PTBR{\input{patterns/intro_CPU_ISA_PTBR}}
\RU{\input{patterns/intro_CPU_ISA_RU}}
\DE{\input{patterns/intro_CPU_ISA_DE}}
\FR{\input{patterns/intro_CPU_ISA_FR}}
\PL{\input{patterns/intro_CPU_ISA_PL}}

\EN{\input{patterns/numeral_EN}}
\RU{\input{patterns/numeral_RU}}
\ITA{\input{patterns/numeral_ITA}}
\DE{\input{patterns/numeral_DE}}
\FR{\input{patterns/numeral_FR}}
\PL{\input{patterns/numeral_PL}}

% chapters
\input{patterns/00_empty/main}
\input{patterns/011_ret/main}
\input{patterns/01_helloworld/main}
\input{patterns/015_prolog_epilogue/main}
\input{patterns/02_stack/main}
\input{patterns/03_printf/main}
\input{patterns/04_scanf/main}
\input{patterns/05_passing_arguments/main}
\input{patterns/06_return_results/main}
\input{patterns/061_pointers/main}
\input{patterns/065_GOTO/main}
\input{patterns/07_jcc/main}
\input{patterns/08_switch/main}
\input{patterns/09_loops/main}
\input{patterns/10_strings/main}
\input{patterns/11_arith_optimizations/main}
\input{patterns/12_FPU/main}
\input{patterns/13_arrays/main}
\input{patterns/14_bitfields/main}
\EN{\input{patterns/145_LCG/main_EN}}
\RU{\input{patterns/145_LCG/main_RU}}
\input{patterns/15_structs/main}
\input{patterns/17_unions/main}
\input{patterns/18_pointers_to_functions/main}
\input{patterns/185_64bit_in_32_env/main}

\EN{\input{patterns/19_SIMD/main_EN}}
\RU{\input{patterns/19_SIMD/main_RU}}
\DE{\input{patterns/19_SIMD/main_DE}}

\EN{\input{patterns/20_x64/main_EN}}
\RU{\input{patterns/20_x64/main_RU}}

\EN{\input{patterns/205_floating_SIMD/main_EN}}
\RU{\input{patterns/205_floating_SIMD/main_RU}}
\DE{\input{patterns/205_floating_SIMD/main_DE}}

\EN{\input{patterns/ARM/main_EN}}
\RU{\input{patterns/ARM/main_RU}}
\DE{\input{patterns/ARM/main_DE}}

\input{patterns/MIPS/main}

\ifdefined\SPANISH
\chapter{Patrones de código}
\fi % SPANISH

\ifdefined\GERMAN
\chapter{Code-Muster}
\fi % GERMAN

\ifdefined\ENGLISH
\chapter{Code Patterns}
\fi % ENGLISH

\ifdefined\ITALIAN
\chapter{Forme di codice}
\fi % ITALIAN

\ifdefined\RUSSIAN
\chapter{Образцы кода}
\fi % RUSSIAN

\ifdefined\BRAZILIAN
\chapter{Padrões de códigos}
\fi % BRAZILIAN

\ifdefined\THAI
\chapter{รูปแบบของโค้ด}
\fi % THAI

\ifdefined\FRENCH
\chapter{Modèle de code}
\fi % FRENCH

\ifdefined\POLISH
\chapter{\PLph{}}
\fi % POLISH

% sections
\EN{\input{patterns/patterns_opt_dbg_EN}}
\ES{\input{patterns/patterns_opt_dbg_ES}}
\ITA{\input{patterns/patterns_opt_dbg_ITA}}
\PTBR{\input{patterns/patterns_opt_dbg_PTBR}}
\RU{\input{patterns/patterns_opt_dbg_RU}}
\THA{\input{patterns/patterns_opt_dbg_THA}}
\DE{\input{patterns/patterns_opt_dbg_DE}}
\FR{\input{patterns/patterns_opt_dbg_FR}}
\PL{\input{patterns/patterns_opt_dbg_PL}}

\RU{\section{Некоторые базовые понятия}}
\EN{\section{Some basics}}
\DE{\section{Einige Grundlagen}}
\FR{\section{Quelques bases}}
\ES{\section{\ESph{}}}
\ITA{\section{Alcune basi teoriche}}
\PTBR{\section{\PTBRph{}}}
\THA{\section{\THAph{}}}
\PL{\section{\PLph{}}}

% sections:
\EN{\input{patterns/intro_CPU_ISA_EN}}
\ES{\input{patterns/intro_CPU_ISA_ES}}
\ITA{\input{patterns/intro_CPU_ISA_ITA}}
\PTBR{\input{patterns/intro_CPU_ISA_PTBR}}
\RU{\input{patterns/intro_CPU_ISA_RU}}
\DE{\input{patterns/intro_CPU_ISA_DE}}
\FR{\input{patterns/intro_CPU_ISA_FR}}
\PL{\input{patterns/intro_CPU_ISA_PL}}

\EN{\input{patterns/numeral_EN}}
\RU{\input{patterns/numeral_RU}}
\ITA{\input{patterns/numeral_ITA}}
\DE{\input{patterns/numeral_DE}}
\FR{\input{patterns/numeral_FR}}
\PL{\input{patterns/numeral_PL}}

% chapters
\input{patterns/00_empty/main}
\input{patterns/011_ret/main}
\input{patterns/01_helloworld/main}
\input{patterns/015_prolog_epilogue/main}
\input{patterns/02_stack/main}
\input{patterns/03_printf/main}
\input{patterns/04_scanf/main}
\input{patterns/05_passing_arguments/main}
\input{patterns/06_return_results/main}
\input{patterns/061_pointers/main}
\input{patterns/065_GOTO/main}
\input{patterns/07_jcc/main}
\input{patterns/08_switch/main}
\input{patterns/09_loops/main}
\input{patterns/10_strings/main}
\input{patterns/11_arith_optimizations/main}
\input{patterns/12_FPU/main}
\input{patterns/13_arrays/main}
\input{patterns/14_bitfields/main}
\EN{\input{patterns/145_LCG/main_EN}}
\RU{\input{patterns/145_LCG/main_RU}}
\input{patterns/15_structs/main}
\input{patterns/17_unions/main}
\input{patterns/18_pointers_to_functions/main}
\input{patterns/185_64bit_in_32_env/main}

\EN{\input{patterns/19_SIMD/main_EN}}
\RU{\input{patterns/19_SIMD/main_RU}}
\DE{\input{patterns/19_SIMD/main_DE}}

\EN{\input{patterns/20_x64/main_EN}}
\RU{\input{patterns/20_x64/main_RU}}

\EN{\input{patterns/205_floating_SIMD/main_EN}}
\RU{\input{patterns/205_floating_SIMD/main_RU}}
\DE{\input{patterns/205_floating_SIMD/main_DE}}

\EN{\input{patterns/ARM/main_EN}}
\RU{\input{patterns/ARM/main_RU}}
\DE{\input{patterns/ARM/main_DE}}

\input{patterns/MIPS/main}

\ifdefined\SPANISH
\chapter{Patrones de código}
\fi % SPANISH

\ifdefined\GERMAN
\chapter{Code-Muster}
\fi % GERMAN

\ifdefined\ENGLISH
\chapter{Code Patterns}
\fi % ENGLISH

\ifdefined\ITALIAN
\chapter{Forme di codice}
\fi % ITALIAN

\ifdefined\RUSSIAN
\chapter{Образцы кода}
\fi % RUSSIAN

\ifdefined\BRAZILIAN
\chapter{Padrões de códigos}
\fi % BRAZILIAN

\ifdefined\THAI
\chapter{รูปแบบของโค้ด}
\fi % THAI

\ifdefined\FRENCH
\chapter{Modèle de code}
\fi % FRENCH

\ifdefined\POLISH
\chapter{\PLph{}}
\fi % POLISH

% sections
\EN{\input{patterns/patterns_opt_dbg_EN}}
\ES{\input{patterns/patterns_opt_dbg_ES}}
\ITA{\input{patterns/patterns_opt_dbg_ITA}}
\PTBR{\input{patterns/patterns_opt_dbg_PTBR}}
\RU{\input{patterns/patterns_opt_dbg_RU}}
\THA{\input{patterns/patterns_opt_dbg_THA}}
\DE{\input{patterns/patterns_opt_dbg_DE}}
\FR{\input{patterns/patterns_opt_dbg_FR}}
\PL{\input{patterns/patterns_opt_dbg_PL}}

\RU{\section{Некоторые базовые понятия}}
\EN{\section{Some basics}}
\DE{\section{Einige Grundlagen}}
\FR{\section{Quelques bases}}
\ES{\section{\ESph{}}}
\ITA{\section{Alcune basi teoriche}}
\PTBR{\section{\PTBRph{}}}
\THA{\section{\THAph{}}}
\PL{\section{\PLph{}}}

% sections:
\EN{\input{patterns/intro_CPU_ISA_EN}}
\ES{\input{patterns/intro_CPU_ISA_ES}}
\ITA{\input{patterns/intro_CPU_ISA_ITA}}
\PTBR{\input{patterns/intro_CPU_ISA_PTBR}}
\RU{\input{patterns/intro_CPU_ISA_RU}}
\DE{\input{patterns/intro_CPU_ISA_DE}}
\FR{\input{patterns/intro_CPU_ISA_FR}}
\PL{\input{patterns/intro_CPU_ISA_PL}}

\EN{\input{patterns/numeral_EN}}
\RU{\input{patterns/numeral_RU}}
\ITA{\input{patterns/numeral_ITA}}
\DE{\input{patterns/numeral_DE}}
\FR{\input{patterns/numeral_FR}}
\PL{\input{patterns/numeral_PL}}

% chapters
\input{patterns/00_empty/main}
\input{patterns/011_ret/main}
\input{patterns/01_helloworld/main}
\input{patterns/015_prolog_epilogue/main}
\input{patterns/02_stack/main}
\input{patterns/03_printf/main}
\input{patterns/04_scanf/main}
\input{patterns/05_passing_arguments/main}
\input{patterns/06_return_results/main}
\input{patterns/061_pointers/main}
\input{patterns/065_GOTO/main}
\input{patterns/07_jcc/main}
\input{patterns/08_switch/main}
\input{patterns/09_loops/main}
\input{patterns/10_strings/main}
\input{patterns/11_arith_optimizations/main}
\input{patterns/12_FPU/main}
\input{patterns/13_arrays/main}
\input{patterns/14_bitfields/main}
\EN{\input{patterns/145_LCG/main_EN}}
\RU{\input{patterns/145_LCG/main_RU}}
\input{patterns/15_structs/main}
\input{patterns/17_unions/main}
\input{patterns/18_pointers_to_functions/main}
\input{patterns/185_64bit_in_32_env/main}

\EN{\input{patterns/19_SIMD/main_EN}}
\RU{\input{patterns/19_SIMD/main_RU}}
\DE{\input{patterns/19_SIMD/main_DE}}

\EN{\input{patterns/20_x64/main_EN}}
\RU{\input{patterns/20_x64/main_RU}}

\EN{\input{patterns/205_floating_SIMD/main_EN}}
\RU{\input{patterns/205_floating_SIMD/main_RU}}
\DE{\input{patterns/205_floating_SIMD/main_DE}}

\EN{\input{patterns/ARM/main_EN}}
\RU{\input{patterns/ARM/main_RU}}
\DE{\input{patterns/ARM/main_DE}}

\input{patterns/MIPS/main}

\EN{\input{patterns/12_FPU/main_EN}}
\RU{\input{patterns/12_FPU/main_RU}}
\DE{\input{patterns/12_FPU/main_DE}}
\FR{\input{patterns/12_FPU/main_FR}}


\ifdefined\SPANISH
\chapter{Patrones de código}
\fi % SPANISH

\ifdefined\GERMAN
\chapter{Code-Muster}
\fi % GERMAN

\ifdefined\ENGLISH
\chapter{Code Patterns}
\fi % ENGLISH

\ifdefined\ITALIAN
\chapter{Forme di codice}
\fi % ITALIAN

\ifdefined\RUSSIAN
\chapter{Образцы кода}
\fi % RUSSIAN

\ifdefined\BRAZILIAN
\chapter{Padrões de códigos}
\fi % BRAZILIAN

\ifdefined\THAI
\chapter{รูปแบบของโค้ด}
\fi % THAI

\ifdefined\FRENCH
\chapter{Modèle de code}
\fi % FRENCH

\ifdefined\POLISH
\chapter{\PLph{}}
\fi % POLISH

% sections
\EN{\input{patterns/patterns_opt_dbg_EN}}
\ES{\input{patterns/patterns_opt_dbg_ES}}
\ITA{\input{patterns/patterns_opt_dbg_ITA}}
\PTBR{\input{patterns/patterns_opt_dbg_PTBR}}
\RU{\input{patterns/patterns_opt_dbg_RU}}
\THA{\input{patterns/patterns_opt_dbg_THA}}
\DE{\input{patterns/patterns_opt_dbg_DE}}
\FR{\input{patterns/patterns_opt_dbg_FR}}
\PL{\input{patterns/patterns_opt_dbg_PL}}

\RU{\section{Некоторые базовые понятия}}
\EN{\section{Some basics}}
\DE{\section{Einige Grundlagen}}
\FR{\section{Quelques bases}}
\ES{\section{\ESph{}}}
\ITA{\section{Alcune basi teoriche}}
\PTBR{\section{\PTBRph{}}}
\THA{\section{\THAph{}}}
\PL{\section{\PLph{}}}

% sections:
\EN{\input{patterns/intro_CPU_ISA_EN}}
\ES{\input{patterns/intro_CPU_ISA_ES}}
\ITA{\input{patterns/intro_CPU_ISA_ITA}}
\PTBR{\input{patterns/intro_CPU_ISA_PTBR}}
\RU{\input{patterns/intro_CPU_ISA_RU}}
\DE{\input{patterns/intro_CPU_ISA_DE}}
\FR{\input{patterns/intro_CPU_ISA_FR}}
\PL{\input{patterns/intro_CPU_ISA_PL}}

\EN{\input{patterns/numeral_EN}}
\RU{\input{patterns/numeral_RU}}
\ITA{\input{patterns/numeral_ITA}}
\DE{\input{patterns/numeral_DE}}
\FR{\input{patterns/numeral_FR}}
\PL{\input{patterns/numeral_PL}}

% chapters
\input{patterns/00_empty/main}
\input{patterns/011_ret/main}
\input{patterns/01_helloworld/main}
\input{patterns/015_prolog_epilogue/main}
\input{patterns/02_stack/main}
\input{patterns/03_printf/main}
\input{patterns/04_scanf/main}
\input{patterns/05_passing_arguments/main}
\input{patterns/06_return_results/main}
\input{patterns/061_pointers/main}
\input{patterns/065_GOTO/main}
\input{patterns/07_jcc/main}
\input{patterns/08_switch/main}
\input{patterns/09_loops/main}
\input{patterns/10_strings/main}
\input{patterns/11_arith_optimizations/main}
\input{patterns/12_FPU/main}
\input{patterns/13_arrays/main}
\input{patterns/14_bitfields/main}
\EN{\input{patterns/145_LCG/main_EN}}
\RU{\input{patterns/145_LCG/main_RU}}
\input{patterns/15_structs/main}
\input{patterns/17_unions/main}
\input{patterns/18_pointers_to_functions/main}
\input{patterns/185_64bit_in_32_env/main}

\EN{\input{patterns/19_SIMD/main_EN}}
\RU{\input{patterns/19_SIMD/main_RU}}
\DE{\input{patterns/19_SIMD/main_DE}}

\EN{\input{patterns/20_x64/main_EN}}
\RU{\input{patterns/20_x64/main_RU}}

\EN{\input{patterns/205_floating_SIMD/main_EN}}
\RU{\input{patterns/205_floating_SIMD/main_RU}}
\DE{\input{patterns/205_floating_SIMD/main_DE}}

\EN{\input{patterns/ARM/main_EN}}
\RU{\input{patterns/ARM/main_RU}}
\DE{\input{patterns/ARM/main_DE}}

\input{patterns/MIPS/main}

\ifdefined\SPANISH
\chapter{Patrones de código}
\fi % SPANISH

\ifdefined\GERMAN
\chapter{Code-Muster}
\fi % GERMAN

\ifdefined\ENGLISH
\chapter{Code Patterns}
\fi % ENGLISH

\ifdefined\ITALIAN
\chapter{Forme di codice}
\fi % ITALIAN

\ifdefined\RUSSIAN
\chapter{Образцы кода}
\fi % RUSSIAN

\ifdefined\BRAZILIAN
\chapter{Padrões de códigos}
\fi % BRAZILIAN

\ifdefined\THAI
\chapter{รูปแบบของโค้ด}
\fi % THAI

\ifdefined\FRENCH
\chapter{Modèle de code}
\fi % FRENCH

\ifdefined\POLISH
\chapter{\PLph{}}
\fi % POLISH

% sections
\EN{\input{patterns/patterns_opt_dbg_EN}}
\ES{\input{patterns/patterns_opt_dbg_ES}}
\ITA{\input{patterns/patterns_opt_dbg_ITA}}
\PTBR{\input{patterns/patterns_opt_dbg_PTBR}}
\RU{\input{patterns/patterns_opt_dbg_RU}}
\THA{\input{patterns/patterns_opt_dbg_THA}}
\DE{\input{patterns/patterns_opt_dbg_DE}}
\FR{\input{patterns/patterns_opt_dbg_FR}}
\PL{\input{patterns/patterns_opt_dbg_PL}}

\RU{\section{Некоторые базовые понятия}}
\EN{\section{Some basics}}
\DE{\section{Einige Grundlagen}}
\FR{\section{Quelques bases}}
\ES{\section{\ESph{}}}
\ITA{\section{Alcune basi teoriche}}
\PTBR{\section{\PTBRph{}}}
\THA{\section{\THAph{}}}
\PL{\section{\PLph{}}}

% sections:
\EN{\input{patterns/intro_CPU_ISA_EN}}
\ES{\input{patterns/intro_CPU_ISA_ES}}
\ITA{\input{patterns/intro_CPU_ISA_ITA}}
\PTBR{\input{patterns/intro_CPU_ISA_PTBR}}
\RU{\input{patterns/intro_CPU_ISA_RU}}
\DE{\input{patterns/intro_CPU_ISA_DE}}
\FR{\input{patterns/intro_CPU_ISA_FR}}
\PL{\input{patterns/intro_CPU_ISA_PL}}

\EN{\input{patterns/numeral_EN}}
\RU{\input{patterns/numeral_RU}}
\ITA{\input{patterns/numeral_ITA}}
\DE{\input{patterns/numeral_DE}}
\FR{\input{patterns/numeral_FR}}
\PL{\input{patterns/numeral_PL}}

% chapters
\input{patterns/00_empty/main}
\input{patterns/011_ret/main}
\input{patterns/01_helloworld/main}
\input{patterns/015_prolog_epilogue/main}
\input{patterns/02_stack/main}
\input{patterns/03_printf/main}
\input{patterns/04_scanf/main}
\input{patterns/05_passing_arguments/main}
\input{patterns/06_return_results/main}
\input{patterns/061_pointers/main}
\input{patterns/065_GOTO/main}
\input{patterns/07_jcc/main}
\input{patterns/08_switch/main}
\input{patterns/09_loops/main}
\input{patterns/10_strings/main}
\input{patterns/11_arith_optimizations/main}
\input{patterns/12_FPU/main}
\input{patterns/13_arrays/main}
\input{patterns/14_bitfields/main}
\EN{\input{patterns/145_LCG/main_EN}}
\RU{\input{patterns/145_LCG/main_RU}}
\input{patterns/15_structs/main}
\input{patterns/17_unions/main}
\input{patterns/18_pointers_to_functions/main}
\input{patterns/185_64bit_in_32_env/main}

\EN{\input{patterns/19_SIMD/main_EN}}
\RU{\input{patterns/19_SIMD/main_RU}}
\DE{\input{patterns/19_SIMD/main_DE}}

\EN{\input{patterns/20_x64/main_EN}}
\RU{\input{patterns/20_x64/main_RU}}

\EN{\input{patterns/205_floating_SIMD/main_EN}}
\RU{\input{patterns/205_floating_SIMD/main_RU}}
\DE{\input{patterns/205_floating_SIMD/main_DE}}

\EN{\input{patterns/ARM/main_EN}}
\RU{\input{patterns/ARM/main_RU}}
\DE{\input{patterns/ARM/main_DE}}

\input{patterns/MIPS/main}

\EN{\section{Returning Values}
\label{ret_val_func}

Another simple function is the one that simply returns a constant value:

\lstinputlisting[caption=\EN{\CCpp Code},style=customc]{patterns/011_ret/1.c}

Let's compile it.

\subsection{x86}

Here's what both the GCC and MSVC compilers produce (with optimization) on the x86 platform:

\lstinputlisting[caption=\Optimizing GCC/MSVC (\assemblyOutput),style=customasmx86]{patterns/011_ret/1.s}

\myindex{x86!\Instructions!RET}
There are just two instructions: the first places the value 123 into the \EAX register,
which is used by convention for storing the return
value, and the second one is \RET, which returns execution to the \gls{caller}.

The caller will take the result from the \EAX register.

\subsection{ARM}

There are a few differences on the ARM platform:

\lstinputlisting[caption=\OptimizingKeilVI (\ARMMode) ASM Output,style=customasmARM]{patterns/011_ret/1_Keil_ARM_O3.s}

ARM uses the register \Reg{0} for returning the results of functions, so 123 is copied into \Reg{0}.

\myindex{ARM!\Instructions!MOV}
\myindex{x86!\Instructions!MOV}
It is worth noting that \MOV is a misleading name for the instruction in both the x86 and ARM \ac{ISA}s.

The data is not in fact \IT{moved}, but \IT{copied}.

\subsection{MIPS}

\label{MIPS_leaf_function_ex1}

The GCC assembly output below lists registers by number:

\lstinputlisting[caption=\Optimizing GCC 4.4.5 (\assemblyOutput),style=customasmMIPS]{patterns/011_ret/MIPS.s}

\dots while \IDA does it by their pseudo names:

\lstinputlisting[caption=\Optimizing GCC 4.4.5 (IDA),style=customasmMIPS]{patterns/011_ret/MIPS_IDA.lst}

The \$2 (or \$V0) register is used to store the function's return value.
\myindex{MIPS!\Pseudoinstructions!LI}
\INS{LI} stands for ``Load Immediate'' and is the MIPS equivalent to \MOV.

\myindex{MIPS!\Instructions!J}
The other instruction is the jump instruction (J or JR) which returns the execution flow to the \gls{caller}.

\myindex{MIPS!Branch delay slot}
You might be wondering why the positions of the load instruction (LI) and the jump instruction (J or JR) are swapped. This is due to a \ac{RISC} feature called ``branch delay slot''.

The reason this happens is a quirk in the architecture of some RISC \ac{ISA}s and isn't important for our
purposes---we must simply keep in mind that in MIPS, the instruction following a jump or branch instruction
is executed \IT{before} the jump/branch instruction itself.

As a consequence, branch instructions always swap places with the instruction executed immediately beforehand.


In practice, functions which merely return 1 (\IT{true}) or 0 (\IT{false}) are very frequent.

The smallest ever of the standard UNIX utilities, \IT{/bin/true} and \IT{/bin/false} return 0 and 1 respectively, as an exit code.
(Zero as an exit code usually means success, non-zero means error.)
}
\RU{\subsubsection{std::string}
\myindex{\Cpp!STL!std::string}
\label{std_string}

\myparagraph{Как устроена структура}

Многие строковые библиотеки \InSqBrackets{\CNotes 2.2} обеспечивают структуру содержащую ссылку 
на буфер собственно со строкой, переменная всегда содержащую длину строки 
(что очень удобно для массы функций \InSqBrackets{\CNotes 2.2.1}) и переменную содержащую текущий размер буфера.

Строка в буфере обыкновенно оканчивается нулем: это для того чтобы указатель на буфер можно было
передавать в функции требующие на вход обычную сишную \ac{ASCIIZ}-строку.

Стандарт \Cpp не описывает, как именно нужно реализовывать std::string,
но, как правило, они реализованы как описано выше, с небольшими дополнениями.

Строки в \Cpp это не класс (как, например, QString в Qt), а темплейт (basic\_string), 
это сделано для того чтобы поддерживать 
строки содержащие разного типа символы: как минимум \Tchar и \IT{wchar\_t}.

Так что, std::string это класс с базовым типом \Tchar.

А std::wstring это класс с базовым типом \IT{wchar\_t}.

\mysubparagraph{MSVC}

В реализации MSVC, вместо ссылки на буфер может содержаться сам буфер (если строка короче 16-и символов).

Это означает, что каждая короткая строка будет занимать в памяти по крайней мере $16 + 4 + 4 = 24$ 
байт для 32-битной среды либо $16 + 8 + 8 = 32$ 
байта в 64-битной, а если строка длиннее 16-и символов, то прибавьте еще длину самой строки.

\lstinputlisting[caption=пример для MSVC,style=customc]{\CURPATH/STL/string/MSVC_RU.cpp}

Собственно, из этого исходника почти всё ясно.

Несколько замечаний:

Если строка короче 16-и символов, 
то отдельный буфер для строки в \glslink{heap}{куче} выделяться не будет.

Это удобно потому что на практике, основная часть строк действительно короткие.
Вероятно, разработчики в Microsoft выбрали размер в 16 символов как разумный баланс.

Теперь очень важный момент в конце функции main(): мы не пользуемся методом c\_str(), тем не менее,
если это скомпилировать и запустить, то обе строки появятся в консоли!

Работает это вот почему.

В первом случае строка короче 16-и символов и в начале объекта std::string (его можно рассматривать
просто как структуру) расположен буфер с этой строкой.
\printf трактует указатель как указатель на массив символов оканчивающийся нулем и поэтому всё работает.

Вывод второй строки (длиннее 16-и символов) даже еще опаснее: это вообще типичная программистская ошибка 
(или опечатка), забыть дописать c\_str().
Это работает потому что в это время в начале структуры расположен указатель на буфер.
Это может надолго остаться незамеченным: до тех пока там не появится строка 
короче 16-и символов, тогда процесс упадет.

\mysubparagraph{GCC}

В реализации GCC в структуре есть еще одна переменная --- reference count.

Интересно, что указатель на экземпляр класса std::string в GCC указывает не на начало самой структуры, 
а на указатель на буфера.
В libstdc++-v3\textbackslash{}include\textbackslash{}bits\textbackslash{}basic\_string.h 
мы можем прочитать что это сделано для удобства отладки:

\begin{lstlisting}
   *  The reason you want _M_data pointing to the character %array and
   *  not the _Rep is so that the debugger can see the string
   *  contents. (Probably we should add a non-inline member to get
   *  the _Rep for the debugger to use, so users can check the actual
   *  string length.)
\end{lstlisting}

\href{http://go.yurichev.com/17085}{исходный код basic\_string.h}

В нашем примере мы учитываем это:

\lstinputlisting[caption=пример для GCC,style=customc]{\CURPATH/STL/string/GCC_RU.cpp}

Нужны еще небольшие хаки чтобы сымитировать типичную ошибку, которую мы уже видели выше, из-за
более ужесточенной проверки типов в GCC, тем не менее, printf() работает и здесь без c\_str().

\myparagraph{Чуть более сложный пример}

\lstinputlisting[style=customc]{\CURPATH/STL/string/3.cpp}

\lstinputlisting[caption=MSVC 2012,style=customasmx86]{\CURPATH/STL/string/3_MSVC_RU.asm}

Собственно, компилятор не конструирует строки статически: да в общем-то и как
это возможно, если буфер с ней нужно хранить в \glslink{heap}{куче}?

Вместо этого в сегменте данных хранятся обычные \ac{ASCIIZ}-строки, а позже, во время выполнения, 
при помощи метода \q{assign}, конструируются строки s1 и s2
.
При помощи \TT{operator+}, создается строка s3.

Обратите внимание на то что вызов метода c\_str() отсутствует,
потому что его код достаточно короткий и компилятор вставил его прямо здесь:
если строка короче 16-и байт, то в регистре EAX остается указатель на буфер,
а если длиннее, то из этого же места достается адрес на буфер расположенный в \glslink{heap}{куче}.

Далее следуют вызовы трех деструкторов, причем, они вызываются только если строка длиннее 16-и байт:
тогда нужно освободить буфера в \glslink{heap}{куче}.
В противном случае, так как все три объекта std::string хранятся в стеке,
они освобождаются автоматически после выхода из функции.

Следовательно, работа с короткими строками более быстрая из-за м\'{е}ньшего обращения к \glslink{heap}{куче}.

Код на GCC даже проще (из-за того, что в GCC, как мы уже видели, не реализована возможность хранить короткую
строку прямо в структуре):

% TODO1 comment each function meaning
\lstinputlisting[caption=GCC 4.8.1,style=customasmx86]{\CURPATH/STL/string/3_GCC_RU.s}

Можно заметить, что в деструкторы передается не указатель на объект,
а указатель на место за 12 байт (или 3 слова) перед ним, то есть, на настоящее начало структуры.

\myparagraph{std::string как глобальная переменная}
\label{sec:std_string_as_global_variable}

Опытные программисты на \Cpp знают, что глобальные переменные \ac{STL}-типов вполне можно объявлять.

Да, действительно:

\lstinputlisting[style=customc]{\CURPATH/STL/string/5.cpp}

Но как и где будет вызываться конструктор \TT{std::string}?

На самом деле, эта переменная будет инициализирована даже перед началом \main.

\lstinputlisting[caption=MSVC 2012: здесь конструируется глобальная переменная{,} а также регистрируется её деструктор,style=customasmx86]{\CURPATH/STL/string/5_MSVC_p2.asm}

\lstinputlisting[caption=MSVC 2012: здесь глобальная переменная используется в \main,style=customasmx86]{\CURPATH/STL/string/5_MSVC_p1.asm}

\lstinputlisting[caption=MSVC 2012: эта функция-деструктор вызывается перед выходом,style=customasmx86]{\CURPATH/STL/string/5_MSVC_p3.asm}

\myindex{\CStandardLibrary!atexit()}
В реальности, из \ac{CRT}, еще до вызова main(), вызывается специальная функция,
в которой перечислены все конструкторы подобных переменных.
Более того: при помощи atexit() регистрируется функция, которая будет вызвана в конце работы программы:
в этой функции компилятор собирает вызовы деструкторов всех подобных глобальных переменных.

GCC работает похожим образом:

\lstinputlisting[caption=GCC 4.8.1,style=customasmx86]{\CURPATH/STL/string/5_GCC.s}

Но он не выделяет отдельной функции в которой будут собраны деструкторы: 
каждый деструктор передается в atexit() по одному.

% TODO а если глобальная STL-переменная в другом модуле? надо проверить.

}
\ifdefined\SPANISH
\chapter{Patrones de código}
\fi % SPANISH

\ifdefined\GERMAN
\chapter{Code-Muster}
\fi % GERMAN

\ifdefined\ENGLISH
\chapter{Code Patterns}
\fi % ENGLISH

\ifdefined\ITALIAN
\chapter{Forme di codice}
\fi % ITALIAN

\ifdefined\RUSSIAN
\chapter{Образцы кода}
\fi % RUSSIAN

\ifdefined\BRAZILIAN
\chapter{Padrões de códigos}
\fi % BRAZILIAN

\ifdefined\THAI
\chapter{รูปแบบของโค้ด}
\fi % THAI

\ifdefined\FRENCH
\chapter{Modèle de code}
\fi % FRENCH

\ifdefined\POLISH
\chapter{\PLph{}}
\fi % POLISH

% sections
\EN{\input{patterns/patterns_opt_dbg_EN}}
\ES{\input{patterns/patterns_opt_dbg_ES}}
\ITA{\input{patterns/patterns_opt_dbg_ITA}}
\PTBR{\input{patterns/patterns_opt_dbg_PTBR}}
\RU{\input{patterns/patterns_opt_dbg_RU}}
\THA{\input{patterns/patterns_opt_dbg_THA}}
\DE{\input{patterns/patterns_opt_dbg_DE}}
\FR{\input{patterns/patterns_opt_dbg_FR}}
\PL{\input{patterns/patterns_opt_dbg_PL}}

\RU{\section{Некоторые базовые понятия}}
\EN{\section{Some basics}}
\DE{\section{Einige Grundlagen}}
\FR{\section{Quelques bases}}
\ES{\section{\ESph{}}}
\ITA{\section{Alcune basi teoriche}}
\PTBR{\section{\PTBRph{}}}
\THA{\section{\THAph{}}}
\PL{\section{\PLph{}}}

% sections:
\EN{\input{patterns/intro_CPU_ISA_EN}}
\ES{\input{patterns/intro_CPU_ISA_ES}}
\ITA{\input{patterns/intro_CPU_ISA_ITA}}
\PTBR{\input{patterns/intro_CPU_ISA_PTBR}}
\RU{\input{patterns/intro_CPU_ISA_RU}}
\DE{\input{patterns/intro_CPU_ISA_DE}}
\FR{\input{patterns/intro_CPU_ISA_FR}}
\PL{\input{patterns/intro_CPU_ISA_PL}}

\EN{\input{patterns/numeral_EN}}
\RU{\input{patterns/numeral_RU}}
\ITA{\input{patterns/numeral_ITA}}
\DE{\input{patterns/numeral_DE}}
\FR{\input{patterns/numeral_FR}}
\PL{\input{patterns/numeral_PL}}

% chapters
\input{patterns/00_empty/main}
\input{patterns/011_ret/main}
\input{patterns/01_helloworld/main}
\input{patterns/015_prolog_epilogue/main}
\input{patterns/02_stack/main}
\input{patterns/03_printf/main}
\input{patterns/04_scanf/main}
\input{patterns/05_passing_arguments/main}
\input{patterns/06_return_results/main}
\input{patterns/061_pointers/main}
\input{patterns/065_GOTO/main}
\input{patterns/07_jcc/main}
\input{patterns/08_switch/main}
\input{patterns/09_loops/main}
\input{patterns/10_strings/main}
\input{patterns/11_arith_optimizations/main}
\input{patterns/12_FPU/main}
\input{patterns/13_arrays/main}
\input{patterns/14_bitfields/main}
\EN{\input{patterns/145_LCG/main_EN}}
\RU{\input{patterns/145_LCG/main_RU}}
\input{patterns/15_structs/main}
\input{patterns/17_unions/main}
\input{patterns/18_pointers_to_functions/main}
\input{patterns/185_64bit_in_32_env/main}

\EN{\input{patterns/19_SIMD/main_EN}}
\RU{\input{patterns/19_SIMD/main_RU}}
\DE{\input{patterns/19_SIMD/main_DE}}

\EN{\input{patterns/20_x64/main_EN}}
\RU{\input{patterns/20_x64/main_RU}}

\EN{\input{patterns/205_floating_SIMD/main_EN}}
\RU{\input{patterns/205_floating_SIMD/main_RU}}
\DE{\input{patterns/205_floating_SIMD/main_DE}}

\EN{\input{patterns/ARM/main_EN}}
\RU{\input{patterns/ARM/main_RU}}
\DE{\input{patterns/ARM/main_DE}}

\input{patterns/MIPS/main}

\ifdefined\SPANISH
\chapter{Patrones de código}
\fi % SPANISH

\ifdefined\GERMAN
\chapter{Code-Muster}
\fi % GERMAN

\ifdefined\ENGLISH
\chapter{Code Patterns}
\fi % ENGLISH

\ifdefined\ITALIAN
\chapter{Forme di codice}
\fi % ITALIAN

\ifdefined\RUSSIAN
\chapter{Образцы кода}
\fi % RUSSIAN

\ifdefined\BRAZILIAN
\chapter{Padrões de códigos}
\fi % BRAZILIAN

\ifdefined\THAI
\chapter{รูปแบบของโค้ด}
\fi % THAI

\ifdefined\FRENCH
\chapter{Modèle de code}
\fi % FRENCH

\ifdefined\POLISH
\chapter{\PLph{}}
\fi % POLISH

% sections
\EN{\input{patterns/patterns_opt_dbg_EN}}
\ES{\input{patterns/patterns_opt_dbg_ES}}
\ITA{\input{patterns/patterns_opt_dbg_ITA}}
\PTBR{\input{patterns/patterns_opt_dbg_PTBR}}
\RU{\input{patterns/patterns_opt_dbg_RU}}
\THA{\input{patterns/patterns_opt_dbg_THA}}
\DE{\input{patterns/patterns_opt_dbg_DE}}
\FR{\input{patterns/patterns_opt_dbg_FR}}
\PL{\input{patterns/patterns_opt_dbg_PL}}

\RU{\section{Некоторые базовые понятия}}
\EN{\section{Some basics}}
\DE{\section{Einige Grundlagen}}
\FR{\section{Quelques bases}}
\ES{\section{\ESph{}}}
\ITA{\section{Alcune basi teoriche}}
\PTBR{\section{\PTBRph{}}}
\THA{\section{\THAph{}}}
\PL{\section{\PLph{}}}

% sections:
\EN{\input{patterns/intro_CPU_ISA_EN}}
\ES{\input{patterns/intro_CPU_ISA_ES}}
\ITA{\input{patterns/intro_CPU_ISA_ITA}}
\PTBR{\input{patterns/intro_CPU_ISA_PTBR}}
\RU{\input{patterns/intro_CPU_ISA_RU}}
\DE{\input{patterns/intro_CPU_ISA_DE}}
\FR{\input{patterns/intro_CPU_ISA_FR}}
\PL{\input{patterns/intro_CPU_ISA_PL}}

\EN{\input{patterns/numeral_EN}}
\RU{\input{patterns/numeral_RU}}
\ITA{\input{patterns/numeral_ITA}}
\DE{\input{patterns/numeral_DE}}
\FR{\input{patterns/numeral_FR}}
\PL{\input{patterns/numeral_PL}}

% chapters
\input{patterns/00_empty/main}
\input{patterns/011_ret/main}
\input{patterns/01_helloworld/main}
\input{patterns/015_prolog_epilogue/main}
\input{patterns/02_stack/main}
\input{patterns/03_printf/main}
\input{patterns/04_scanf/main}
\input{patterns/05_passing_arguments/main}
\input{patterns/06_return_results/main}
\input{patterns/061_pointers/main}
\input{patterns/065_GOTO/main}
\input{patterns/07_jcc/main}
\input{patterns/08_switch/main}
\input{patterns/09_loops/main}
\input{patterns/10_strings/main}
\input{patterns/11_arith_optimizations/main}
\input{patterns/12_FPU/main}
\input{patterns/13_arrays/main}
\input{patterns/14_bitfields/main}
\EN{\input{patterns/145_LCG/main_EN}}
\RU{\input{patterns/145_LCG/main_RU}}
\input{patterns/15_structs/main}
\input{patterns/17_unions/main}
\input{patterns/18_pointers_to_functions/main}
\input{patterns/185_64bit_in_32_env/main}

\EN{\input{patterns/19_SIMD/main_EN}}
\RU{\input{patterns/19_SIMD/main_RU}}
\DE{\input{patterns/19_SIMD/main_DE}}

\EN{\input{patterns/20_x64/main_EN}}
\RU{\input{patterns/20_x64/main_RU}}

\EN{\input{patterns/205_floating_SIMD/main_EN}}
\RU{\input{patterns/205_floating_SIMD/main_RU}}
\DE{\input{patterns/205_floating_SIMD/main_DE}}

\EN{\input{patterns/ARM/main_EN}}
\RU{\input{patterns/ARM/main_RU}}
\DE{\input{patterns/ARM/main_DE}}

\input{patterns/MIPS/main}

\ifdefined\SPANISH
\chapter{Patrones de código}
\fi % SPANISH

\ifdefined\GERMAN
\chapter{Code-Muster}
\fi % GERMAN

\ifdefined\ENGLISH
\chapter{Code Patterns}
\fi % ENGLISH

\ifdefined\ITALIAN
\chapter{Forme di codice}
\fi % ITALIAN

\ifdefined\RUSSIAN
\chapter{Образцы кода}
\fi % RUSSIAN

\ifdefined\BRAZILIAN
\chapter{Padrões de códigos}
\fi % BRAZILIAN

\ifdefined\THAI
\chapter{รูปแบบของโค้ด}
\fi % THAI

\ifdefined\FRENCH
\chapter{Modèle de code}
\fi % FRENCH

\ifdefined\POLISH
\chapter{\PLph{}}
\fi % POLISH

% sections
\EN{\input{patterns/patterns_opt_dbg_EN}}
\ES{\input{patterns/patterns_opt_dbg_ES}}
\ITA{\input{patterns/patterns_opt_dbg_ITA}}
\PTBR{\input{patterns/patterns_opt_dbg_PTBR}}
\RU{\input{patterns/patterns_opt_dbg_RU}}
\THA{\input{patterns/patterns_opt_dbg_THA}}
\DE{\input{patterns/patterns_opt_dbg_DE}}
\FR{\input{patterns/patterns_opt_dbg_FR}}
\PL{\input{patterns/patterns_opt_dbg_PL}}

\RU{\section{Некоторые базовые понятия}}
\EN{\section{Some basics}}
\DE{\section{Einige Grundlagen}}
\FR{\section{Quelques bases}}
\ES{\section{\ESph{}}}
\ITA{\section{Alcune basi teoriche}}
\PTBR{\section{\PTBRph{}}}
\THA{\section{\THAph{}}}
\PL{\section{\PLph{}}}

% sections:
\EN{\input{patterns/intro_CPU_ISA_EN}}
\ES{\input{patterns/intro_CPU_ISA_ES}}
\ITA{\input{patterns/intro_CPU_ISA_ITA}}
\PTBR{\input{patterns/intro_CPU_ISA_PTBR}}
\RU{\input{patterns/intro_CPU_ISA_RU}}
\DE{\input{patterns/intro_CPU_ISA_DE}}
\FR{\input{patterns/intro_CPU_ISA_FR}}
\PL{\input{patterns/intro_CPU_ISA_PL}}

\EN{\input{patterns/numeral_EN}}
\RU{\input{patterns/numeral_RU}}
\ITA{\input{patterns/numeral_ITA}}
\DE{\input{patterns/numeral_DE}}
\FR{\input{patterns/numeral_FR}}
\PL{\input{patterns/numeral_PL}}

% chapters
\input{patterns/00_empty/main}
\input{patterns/011_ret/main}
\input{patterns/01_helloworld/main}
\input{patterns/015_prolog_epilogue/main}
\input{patterns/02_stack/main}
\input{patterns/03_printf/main}
\input{patterns/04_scanf/main}
\input{patterns/05_passing_arguments/main}
\input{patterns/06_return_results/main}
\input{patterns/061_pointers/main}
\input{patterns/065_GOTO/main}
\input{patterns/07_jcc/main}
\input{patterns/08_switch/main}
\input{patterns/09_loops/main}
\input{patterns/10_strings/main}
\input{patterns/11_arith_optimizations/main}
\input{patterns/12_FPU/main}
\input{patterns/13_arrays/main}
\input{patterns/14_bitfields/main}
\EN{\input{patterns/145_LCG/main_EN}}
\RU{\input{patterns/145_LCG/main_RU}}
\input{patterns/15_structs/main}
\input{patterns/17_unions/main}
\input{patterns/18_pointers_to_functions/main}
\input{patterns/185_64bit_in_32_env/main}

\EN{\input{patterns/19_SIMD/main_EN}}
\RU{\input{patterns/19_SIMD/main_RU}}
\DE{\input{patterns/19_SIMD/main_DE}}

\EN{\input{patterns/20_x64/main_EN}}
\RU{\input{patterns/20_x64/main_RU}}

\EN{\input{patterns/205_floating_SIMD/main_EN}}
\RU{\input{patterns/205_floating_SIMD/main_RU}}
\DE{\input{patterns/205_floating_SIMD/main_DE}}

\EN{\input{patterns/ARM/main_EN}}
\RU{\input{patterns/ARM/main_RU}}
\DE{\input{patterns/ARM/main_DE}}

\input{patterns/MIPS/main}

\ifdefined\SPANISH
\chapter{Patrones de código}
\fi % SPANISH

\ifdefined\GERMAN
\chapter{Code-Muster}
\fi % GERMAN

\ifdefined\ENGLISH
\chapter{Code Patterns}
\fi % ENGLISH

\ifdefined\ITALIAN
\chapter{Forme di codice}
\fi % ITALIAN

\ifdefined\RUSSIAN
\chapter{Образцы кода}
\fi % RUSSIAN

\ifdefined\BRAZILIAN
\chapter{Padrões de códigos}
\fi % BRAZILIAN

\ifdefined\THAI
\chapter{รูปแบบของโค้ด}
\fi % THAI

\ifdefined\FRENCH
\chapter{Modèle de code}
\fi % FRENCH

\ifdefined\POLISH
\chapter{\PLph{}}
\fi % POLISH

% sections
\EN{\input{patterns/patterns_opt_dbg_EN}}
\ES{\input{patterns/patterns_opt_dbg_ES}}
\ITA{\input{patterns/patterns_opt_dbg_ITA}}
\PTBR{\input{patterns/patterns_opt_dbg_PTBR}}
\RU{\input{patterns/patterns_opt_dbg_RU}}
\THA{\input{patterns/patterns_opt_dbg_THA}}
\DE{\input{patterns/patterns_opt_dbg_DE}}
\FR{\input{patterns/patterns_opt_dbg_FR}}
\PL{\input{patterns/patterns_opt_dbg_PL}}

\RU{\section{Некоторые базовые понятия}}
\EN{\section{Some basics}}
\DE{\section{Einige Grundlagen}}
\FR{\section{Quelques bases}}
\ES{\section{\ESph{}}}
\ITA{\section{Alcune basi teoriche}}
\PTBR{\section{\PTBRph{}}}
\THA{\section{\THAph{}}}
\PL{\section{\PLph{}}}

% sections:
\EN{\input{patterns/intro_CPU_ISA_EN}}
\ES{\input{patterns/intro_CPU_ISA_ES}}
\ITA{\input{patterns/intro_CPU_ISA_ITA}}
\PTBR{\input{patterns/intro_CPU_ISA_PTBR}}
\RU{\input{patterns/intro_CPU_ISA_RU}}
\DE{\input{patterns/intro_CPU_ISA_DE}}
\FR{\input{patterns/intro_CPU_ISA_FR}}
\PL{\input{patterns/intro_CPU_ISA_PL}}

\EN{\input{patterns/numeral_EN}}
\RU{\input{patterns/numeral_RU}}
\ITA{\input{patterns/numeral_ITA}}
\DE{\input{patterns/numeral_DE}}
\FR{\input{patterns/numeral_FR}}
\PL{\input{patterns/numeral_PL}}

% chapters
\input{patterns/00_empty/main}
\input{patterns/011_ret/main}
\input{patterns/01_helloworld/main}
\input{patterns/015_prolog_epilogue/main}
\input{patterns/02_stack/main}
\input{patterns/03_printf/main}
\input{patterns/04_scanf/main}
\input{patterns/05_passing_arguments/main}
\input{patterns/06_return_results/main}
\input{patterns/061_pointers/main}
\input{patterns/065_GOTO/main}
\input{patterns/07_jcc/main}
\input{patterns/08_switch/main}
\input{patterns/09_loops/main}
\input{patterns/10_strings/main}
\input{patterns/11_arith_optimizations/main}
\input{patterns/12_FPU/main}
\input{patterns/13_arrays/main}
\input{patterns/14_bitfields/main}
\EN{\input{patterns/145_LCG/main_EN}}
\RU{\input{patterns/145_LCG/main_RU}}
\input{patterns/15_structs/main}
\input{patterns/17_unions/main}
\input{patterns/18_pointers_to_functions/main}
\input{patterns/185_64bit_in_32_env/main}

\EN{\input{patterns/19_SIMD/main_EN}}
\RU{\input{patterns/19_SIMD/main_RU}}
\DE{\input{patterns/19_SIMD/main_DE}}

\EN{\input{patterns/20_x64/main_EN}}
\RU{\input{patterns/20_x64/main_RU}}

\EN{\input{patterns/205_floating_SIMD/main_EN}}
\RU{\input{patterns/205_floating_SIMD/main_RU}}
\DE{\input{patterns/205_floating_SIMD/main_DE}}

\EN{\input{patterns/ARM/main_EN}}
\RU{\input{patterns/ARM/main_RU}}
\DE{\input{patterns/ARM/main_DE}}

\input{patterns/MIPS/main}


\EN{\section{Returning Values}
\label{ret_val_func}

Another simple function is the one that simply returns a constant value:

\lstinputlisting[caption=\EN{\CCpp Code},style=customc]{patterns/011_ret/1.c}

Let's compile it.

\subsection{x86}

Here's what both the GCC and MSVC compilers produce (with optimization) on the x86 platform:

\lstinputlisting[caption=\Optimizing GCC/MSVC (\assemblyOutput),style=customasmx86]{patterns/011_ret/1.s}

\myindex{x86!\Instructions!RET}
There are just two instructions: the first places the value 123 into the \EAX register,
which is used by convention for storing the return
value, and the second one is \RET, which returns execution to the \gls{caller}.

The caller will take the result from the \EAX register.

\subsection{ARM}

There are a few differences on the ARM platform:

\lstinputlisting[caption=\OptimizingKeilVI (\ARMMode) ASM Output,style=customasmARM]{patterns/011_ret/1_Keil_ARM_O3.s}

ARM uses the register \Reg{0} for returning the results of functions, so 123 is copied into \Reg{0}.

\myindex{ARM!\Instructions!MOV}
\myindex{x86!\Instructions!MOV}
It is worth noting that \MOV is a misleading name for the instruction in both the x86 and ARM \ac{ISA}s.

The data is not in fact \IT{moved}, but \IT{copied}.

\subsection{MIPS}

\label{MIPS_leaf_function_ex1}

The GCC assembly output below lists registers by number:

\lstinputlisting[caption=\Optimizing GCC 4.4.5 (\assemblyOutput),style=customasmMIPS]{patterns/011_ret/MIPS.s}

\dots while \IDA does it by their pseudo names:

\lstinputlisting[caption=\Optimizing GCC 4.4.5 (IDA),style=customasmMIPS]{patterns/011_ret/MIPS_IDA.lst}

The \$2 (or \$V0) register is used to store the function's return value.
\myindex{MIPS!\Pseudoinstructions!LI}
\INS{LI} stands for ``Load Immediate'' and is the MIPS equivalent to \MOV.

\myindex{MIPS!\Instructions!J}
The other instruction is the jump instruction (J or JR) which returns the execution flow to the \gls{caller}.

\myindex{MIPS!Branch delay slot}
You might be wondering why the positions of the load instruction (LI) and the jump instruction (J or JR) are swapped. This is due to a \ac{RISC} feature called ``branch delay slot''.

The reason this happens is a quirk in the architecture of some RISC \ac{ISA}s and isn't important for our
purposes---we must simply keep in mind that in MIPS, the instruction following a jump or branch instruction
is executed \IT{before} the jump/branch instruction itself.

As a consequence, branch instructions always swap places with the instruction executed immediately beforehand.


In practice, functions which merely return 1 (\IT{true}) or 0 (\IT{false}) are very frequent.

The smallest ever of the standard UNIX utilities, \IT{/bin/true} and \IT{/bin/false} return 0 and 1 respectively, as an exit code.
(Zero as an exit code usually means success, non-zero means error.)
}
\RU{\subsubsection{std::string}
\myindex{\Cpp!STL!std::string}
\label{std_string}

\myparagraph{Как устроена структура}

Многие строковые библиотеки \InSqBrackets{\CNotes 2.2} обеспечивают структуру содержащую ссылку 
на буфер собственно со строкой, переменная всегда содержащую длину строки 
(что очень удобно для массы функций \InSqBrackets{\CNotes 2.2.1}) и переменную содержащую текущий размер буфера.

Строка в буфере обыкновенно оканчивается нулем: это для того чтобы указатель на буфер можно было
передавать в функции требующие на вход обычную сишную \ac{ASCIIZ}-строку.

Стандарт \Cpp не описывает, как именно нужно реализовывать std::string,
но, как правило, они реализованы как описано выше, с небольшими дополнениями.

Строки в \Cpp это не класс (как, например, QString в Qt), а темплейт (basic\_string), 
это сделано для того чтобы поддерживать 
строки содержащие разного типа символы: как минимум \Tchar и \IT{wchar\_t}.

Так что, std::string это класс с базовым типом \Tchar.

А std::wstring это класс с базовым типом \IT{wchar\_t}.

\mysubparagraph{MSVC}

В реализации MSVC, вместо ссылки на буфер может содержаться сам буфер (если строка короче 16-и символов).

Это означает, что каждая короткая строка будет занимать в памяти по крайней мере $16 + 4 + 4 = 24$ 
байт для 32-битной среды либо $16 + 8 + 8 = 32$ 
байта в 64-битной, а если строка длиннее 16-и символов, то прибавьте еще длину самой строки.

\lstinputlisting[caption=пример для MSVC,style=customc]{\CURPATH/STL/string/MSVC_RU.cpp}

Собственно, из этого исходника почти всё ясно.

Несколько замечаний:

Если строка короче 16-и символов, 
то отдельный буфер для строки в \glslink{heap}{куче} выделяться не будет.

Это удобно потому что на практике, основная часть строк действительно короткие.
Вероятно, разработчики в Microsoft выбрали размер в 16 символов как разумный баланс.

Теперь очень важный момент в конце функции main(): мы не пользуемся методом c\_str(), тем не менее,
если это скомпилировать и запустить, то обе строки появятся в консоли!

Работает это вот почему.

В первом случае строка короче 16-и символов и в начале объекта std::string (его можно рассматривать
просто как структуру) расположен буфер с этой строкой.
\printf трактует указатель как указатель на массив символов оканчивающийся нулем и поэтому всё работает.

Вывод второй строки (длиннее 16-и символов) даже еще опаснее: это вообще типичная программистская ошибка 
(или опечатка), забыть дописать c\_str().
Это работает потому что в это время в начале структуры расположен указатель на буфер.
Это может надолго остаться незамеченным: до тех пока там не появится строка 
короче 16-и символов, тогда процесс упадет.

\mysubparagraph{GCC}

В реализации GCC в структуре есть еще одна переменная --- reference count.

Интересно, что указатель на экземпляр класса std::string в GCC указывает не на начало самой структуры, 
а на указатель на буфера.
В libstdc++-v3\textbackslash{}include\textbackslash{}bits\textbackslash{}basic\_string.h 
мы можем прочитать что это сделано для удобства отладки:

\begin{lstlisting}
   *  The reason you want _M_data pointing to the character %array and
   *  not the _Rep is so that the debugger can see the string
   *  contents. (Probably we should add a non-inline member to get
   *  the _Rep for the debugger to use, so users can check the actual
   *  string length.)
\end{lstlisting}

\href{http://go.yurichev.com/17085}{исходный код basic\_string.h}

В нашем примере мы учитываем это:

\lstinputlisting[caption=пример для GCC,style=customc]{\CURPATH/STL/string/GCC_RU.cpp}

Нужны еще небольшие хаки чтобы сымитировать типичную ошибку, которую мы уже видели выше, из-за
более ужесточенной проверки типов в GCC, тем не менее, printf() работает и здесь без c\_str().

\myparagraph{Чуть более сложный пример}

\lstinputlisting[style=customc]{\CURPATH/STL/string/3.cpp}

\lstinputlisting[caption=MSVC 2012,style=customasmx86]{\CURPATH/STL/string/3_MSVC_RU.asm}

Собственно, компилятор не конструирует строки статически: да в общем-то и как
это возможно, если буфер с ней нужно хранить в \glslink{heap}{куче}?

Вместо этого в сегменте данных хранятся обычные \ac{ASCIIZ}-строки, а позже, во время выполнения, 
при помощи метода \q{assign}, конструируются строки s1 и s2
.
При помощи \TT{operator+}, создается строка s3.

Обратите внимание на то что вызов метода c\_str() отсутствует,
потому что его код достаточно короткий и компилятор вставил его прямо здесь:
если строка короче 16-и байт, то в регистре EAX остается указатель на буфер,
а если длиннее, то из этого же места достается адрес на буфер расположенный в \glslink{heap}{куче}.

Далее следуют вызовы трех деструкторов, причем, они вызываются только если строка длиннее 16-и байт:
тогда нужно освободить буфера в \glslink{heap}{куче}.
В противном случае, так как все три объекта std::string хранятся в стеке,
они освобождаются автоматически после выхода из функции.

Следовательно, работа с короткими строками более быстрая из-за м\'{е}ньшего обращения к \glslink{heap}{куче}.

Код на GCC даже проще (из-за того, что в GCC, как мы уже видели, не реализована возможность хранить короткую
строку прямо в структуре):

% TODO1 comment each function meaning
\lstinputlisting[caption=GCC 4.8.1,style=customasmx86]{\CURPATH/STL/string/3_GCC_RU.s}

Можно заметить, что в деструкторы передается не указатель на объект,
а указатель на место за 12 байт (или 3 слова) перед ним, то есть, на настоящее начало структуры.

\myparagraph{std::string как глобальная переменная}
\label{sec:std_string_as_global_variable}

Опытные программисты на \Cpp знают, что глобальные переменные \ac{STL}-типов вполне можно объявлять.

Да, действительно:

\lstinputlisting[style=customc]{\CURPATH/STL/string/5.cpp}

Но как и где будет вызываться конструктор \TT{std::string}?

На самом деле, эта переменная будет инициализирована даже перед началом \main.

\lstinputlisting[caption=MSVC 2012: здесь конструируется глобальная переменная{,} а также регистрируется её деструктор,style=customasmx86]{\CURPATH/STL/string/5_MSVC_p2.asm}

\lstinputlisting[caption=MSVC 2012: здесь глобальная переменная используется в \main,style=customasmx86]{\CURPATH/STL/string/5_MSVC_p1.asm}

\lstinputlisting[caption=MSVC 2012: эта функция-деструктор вызывается перед выходом,style=customasmx86]{\CURPATH/STL/string/5_MSVC_p3.asm}

\myindex{\CStandardLibrary!atexit()}
В реальности, из \ac{CRT}, еще до вызова main(), вызывается специальная функция,
в которой перечислены все конструкторы подобных переменных.
Более того: при помощи atexit() регистрируется функция, которая будет вызвана в конце работы программы:
в этой функции компилятор собирает вызовы деструкторов всех подобных глобальных переменных.

GCC работает похожим образом:

\lstinputlisting[caption=GCC 4.8.1,style=customasmx86]{\CURPATH/STL/string/5_GCC.s}

Но он не выделяет отдельной функции в которой будут собраны деструкторы: 
каждый деструктор передается в atexit() по одному.

% TODO а если глобальная STL-переменная в другом модуле? надо проверить.

}
\DE{\subsection{Einfachste XOR-Verschlüsselung überhaupt}

Ich habe einmal eine Software gesehen, bei der alle Debugging-Ausgaben mit XOR mit dem Wert 3
verschlüsselt wurden. Mit anderen Worten, die beiden niedrigsten Bits aller Buchstaben wurden invertiert.

``Hello, world'' wurde zu ``Kfool/\#tlqog'':

\begin{lstlisting}
#!/usr/bin/python

msg="Hello, world!"

print "".join(map(lambda x: chr(ord(x)^3), msg))
\end{lstlisting}

Das ist eine ziemlich interessante Verschlüsselung (oder besser eine Verschleierung),
weil sie zwei wichtige Eigenschaften hat:
1) es ist eine einzige Funktion zum Verschlüsseln und entschlüsseln, sie muss nur wiederholt angewendet werden
2) die entstehenden Buchstaben befinden sich im druckbaren Bereich, also die ganze Zeichenkette kann ohne
Escape-Symbole im Code verwendet werden.

Die zweite Eigenschaft nutzt die Tatsache, dass alle druckbaren Zeichen in Reihen organisiert sind: 0x2x-0x7x,
und wenn die beiden niederwertigsten Bits invertiert werden, wird der Buchstabe um eine oder drei Stellen nach
links oder rechts \IT{verschoben}, aber niemals in eine andere Reihe:

\begin{figure}[H]
\centering
\includegraphics[width=0.7\textwidth]{ascii_clean.png}
\caption{7-Bit \ac{ASCII} Tabelle in Emacs}
\end{figure}

\dots mit dem Zeichen 0x7F als einziger Ausnahme.

Im Folgenden werden also beispielsweise die Zeichen A-Z \IT{verschlüsselt}:

\begin{lstlisting}
#!/usr/bin/python

msg="@ABCDEFGHIJKLMNO"

print "".join(map(lambda x: chr(ord(x)^3), msg))
\end{lstlisting}

Ergebnis:
% FIXME \verb  --  relevant comment for German?
\begin{lstlisting}
CBA@GFEDKJIHONML
\end{lstlisting}

Es sieht so aus als würden die Zeichen ``@'' und ``C'' sowie ``B'' und ``A'' vertauscht werden.

Hier ist noch ein interessantes Beispiel, in dem gezeigt wird, wie die Eigenschaften von XOR
ausgenutzt werden können: Exakt den gleichen Effekt, dass druckbare Zeichen auch druckbar bleiben,
kann man dadurch erzielen, dass irgendeine Kombination der niedrigsten vier Bits invertiert wird.
}

\EN{\section{Returning Values}
\label{ret_val_func}

Another simple function is the one that simply returns a constant value:

\lstinputlisting[caption=\EN{\CCpp Code},style=customc]{patterns/011_ret/1.c}

Let's compile it.

\subsection{x86}

Here's what both the GCC and MSVC compilers produce (with optimization) on the x86 platform:

\lstinputlisting[caption=\Optimizing GCC/MSVC (\assemblyOutput),style=customasmx86]{patterns/011_ret/1.s}

\myindex{x86!\Instructions!RET}
There are just two instructions: the first places the value 123 into the \EAX register,
which is used by convention for storing the return
value, and the second one is \RET, which returns execution to the \gls{caller}.

The caller will take the result from the \EAX register.

\subsection{ARM}

There are a few differences on the ARM platform:

\lstinputlisting[caption=\OptimizingKeilVI (\ARMMode) ASM Output,style=customasmARM]{patterns/011_ret/1_Keil_ARM_O3.s}

ARM uses the register \Reg{0} for returning the results of functions, so 123 is copied into \Reg{0}.

\myindex{ARM!\Instructions!MOV}
\myindex{x86!\Instructions!MOV}
It is worth noting that \MOV is a misleading name for the instruction in both the x86 and ARM \ac{ISA}s.

The data is not in fact \IT{moved}, but \IT{copied}.

\subsection{MIPS}

\label{MIPS_leaf_function_ex1}

The GCC assembly output below lists registers by number:

\lstinputlisting[caption=\Optimizing GCC 4.4.5 (\assemblyOutput),style=customasmMIPS]{patterns/011_ret/MIPS.s}

\dots while \IDA does it by their pseudo names:

\lstinputlisting[caption=\Optimizing GCC 4.4.5 (IDA),style=customasmMIPS]{patterns/011_ret/MIPS_IDA.lst}

The \$2 (or \$V0) register is used to store the function's return value.
\myindex{MIPS!\Pseudoinstructions!LI}
\INS{LI} stands for ``Load Immediate'' and is the MIPS equivalent to \MOV.

\myindex{MIPS!\Instructions!J}
The other instruction is the jump instruction (J or JR) which returns the execution flow to the \gls{caller}.

\myindex{MIPS!Branch delay slot}
You might be wondering why the positions of the load instruction (LI) and the jump instruction (J or JR) are swapped. This is due to a \ac{RISC} feature called ``branch delay slot''.

The reason this happens is a quirk in the architecture of some RISC \ac{ISA}s and isn't important for our
purposes---we must simply keep in mind that in MIPS, the instruction following a jump or branch instruction
is executed \IT{before} the jump/branch instruction itself.

As a consequence, branch instructions always swap places with the instruction executed immediately beforehand.


In practice, functions which merely return 1 (\IT{true}) or 0 (\IT{false}) are very frequent.

The smallest ever of the standard UNIX utilities, \IT{/bin/true} and \IT{/bin/false} return 0 and 1 respectively, as an exit code.
(Zero as an exit code usually means success, non-zero means error.)
}
\RU{\subsubsection{std::string}
\myindex{\Cpp!STL!std::string}
\label{std_string}

\myparagraph{Как устроена структура}

Многие строковые библиотеки \InSqBrackets{\CNotes 2.2} обеспечивают структуру содержащую ссылку 
на буфер собственно со строкой, переменная всегда содержащую длину строки 
(что очень удобно для массы функций \InSqBrackets{\CNotes 2.2.1}) и переменную содержащую текущий размер буфера.

Строка в буфере обыкновенно оканчивается нулем: это для того чтобы указатель на буфер можно было
передавать в функции требующие на вход обычную сишную \ac{ASCIIZ}-строку.

Стандарт \Cpp не описывает, как именно нужно реализовывать std::string,
но, как правило, они реализованы как описано выше, с небольшими дополнениями.

Строки в \Cpp это не класс (как, например, QString в Qt), а темплейт (basic\_string), 
это сделано для того чтобы поддерживать 
строки содержащие разного типа символы: как минимум \Tchar и \IT{wchar\_t}.

Так что, std::string это класс с базовым типом \Tchar.

А std::wstring это класс с базовым типом \IT{wchar\_t}.

\mysubparagraph{MSVC}

В реализации MSVC, вместо ссылки на буфер может содержаться сам буфер (если строка короче 16-и символов).

Это означает, что каждая короткая строка будет занимать в памяти по крайней мере $16 + 4 + 4 = 24$ 
байт для 32-битной среды либо $16 + 8 + 8 = 32$ 
байта в 64-битной, а если строка длиннее 16-и символов, то прибавьте еще длину самой строки.

\lstinputlisting[caption=пример для MSVC,style=customc]{\CURPATH/STL/string/MSVC_RU.cpp}

Собственно, из этого исходника почти всё ясно.

Несколько замечаний:

Если строка короче 16-и символов, 
то отдельный буфер для строки в \glslink{heap}{куче} выделяться не будет.

Это удобно потому что на практике, основная часть строк действительно короткие.
Вероятно, разработчики в Microsoft выбрали размер в 16 символов как разумный баланс.

Теперь очень важный момент в конце функции main(): мы не пользуемся методом c\_str(), тем не менее,
если это скомпилировать и запустить, то обе строки появятся в консоли!

Работает это вот почему.

В первом случае строка короче 16-и символов и в начале объекта std::string (его можно рассматривать
просто как структуру) расположен буфер с этой строкой.
\printf трактует указатель как указатель на массив символов оканчивающийся нулем и поэтому всё работает.

Вывод второй строки (длиннее 16-и символов) даже еще опаснее: это вообще типичная программистская ошибка 
(или опечатка), забыть дописать c\_str().
Это работает потому что в это время в начале структуры расположен указатель на буфер.
Это может надолго остаться незамеченным: до тех пока там не появится строка 
короче 16-и символов, тогда процесс упадет.

\mysubparagraph{GCC}

В реализации GCC в структуре есть еще одна переменная --- reference count.

Интересно, что указатель на экземпляр класса std::string в GCC указывает не на начало самой структуры, 
а на указатель на буфера.
В libstdc++-v3\textbackslash{}include\textbackslash{}bits\textbackslash{}basic\_string.h 
мы можем прочитать что это сделано для удобства отладки:

\begin{lstlisting}
   *  The reason you want _M_data pointing to the character %array and
   *  not the _Rep is so that the debugger can see the string
   *  contents. (Probably we should add a non-inline member to get
   *  the _Rep for the debugger to use, so users can check the actual
   *  string length.)
\end{lstlisting}

\href{http://go.yurichev.com/17085}{исходный код basic\_string.h}

В нашем примере мы учитываем это:

\lstinputlisting[caption=пример для GCC,style=customc]{\CURPATH/STL/string/GCC_RU.cpp}

Нужны еще небольшие хаки чтобы сымитировать типичную ошибку, которую мы уже видели выше, из-за
более ужесточенной проверки типов в GCC, тем не менее, printf() работает и здесь без c\_str().

\myparagraph{Чуть более сложный пример}

\lstinputlisting[style=customc]{\CURPATH/STL/string/3.cpp}

\lstinputlisting[caption=MSVC 2012,style=customasmx86]{\CURPATH/STL/string/3_MSVC_RU.asm}

Собственно, компилятор не конструирует строки статически: да в общем-то и как
это возможно, если буфер с ней нужно хранить в \glslink{heap}{куче}?

Вместо этого в сегменте данных хранятся обычные \ac{ASCIIZ}-строки, а позже, во время выполнения, 
при помощи метода \q{assign}, конструируются строки s1 и s2
.
При помощи \TT{operator+}, создается строка s3.

Обратите внимание на то что вызов метода c\_str() отсутствует,
потому что его код достаточно короткий и компилятор вставил его прямо здесь:
если строка короче 16-и байт, то в регистре EAX остается указатель на буфер,
а если длиннее, то из этого же места достается адрес на буфер расположенный в \glslink{heap}{куче}.

Далее следуют вызовы трех деструкторов, причем, они вызываются только если строка длиннее 16-и байт:
тогда нужно освободить буфера в \glslink{heap}{куче}.
В противном случае, так как все три объекта std::string хранятся в стеке,
они освобождаются автоматически после выхода из функции.

Следовательно, работа с короткими строками более быстрая из-за м\'{е}ньшего обращения к \glslink{heap}{куче}.

Код на GCC даже проще (из-за того, что в GCC, как мы уже видели, не реализована возможность хранить короткую
строку прямо в структуре):

% TODO1 comment each function meaning
\lstinputlisting[caption=GCC 4.8.1,style=customasmx86]{\CURPATH/STL/string/3_GCC_RU.s}

Можно заметить, что в деструкторы передается не указатель на объект,
а указатель на место за 12 байт (или 3 слова) перед ним, то есть, на настоящее начало структуры.

\myparagraph{std::string как глобальная переменная}
\label{sec:std_string_as_global_variable}

Опытные программисты на \Cpp знают, что глобальные переменные \ac{STL}-типов вполне можно объявлять.

Да, действительно:

\lstinputlisting[style=customc]{\CURPATH/STL/string/5.cpp}

Но как и где будет вызываться конструктор \TT{std::string}?

На самом деле, эта переменная будет инициализирована даже перед началом \main.

\lstinputlisting[caption=MSVC 2012: здесь конструируется глобальная переменная{,} а также регистрируется её деструктор,style=customasmx86]{\CURPATH/STL/string/5_MSVC_p2.asm}

\lstinputlisting[caption=MSVC 2012: здесь глобальная переменная используется в \main,style=customasmx86]{\CURPATH/STL/string/5_MSVC_p1.asm}

\lstinputlisting[caption=MSVC 2012: эта функция-деструктор вызывается перед выходом,style=customasmx86]{\CURPATH/STL/string/5_MSVC_p3.asm}

\myindex{\CStandardLibrary!atexit()}
В реальности, из \ac{CRT}, еще до вызова main(), вызывается специальная функция,
в которой перечислены все конструкторы подобных переменных.
Более того: при помощи atexit() регистрируется функция, которая будет вызвана в конце работы программы:
в этой функции компилятор собирает вызовы деструкторов всех подобных глобальных переменных.

GCC работает похожим образом:

\lstinputlisting[caption=GCC 4.8.1,style=customasmx86]{\CURPATH/STL/string/5_GCC.s}

Но он не выделяет отдельной функции в которой будут собраны деструкторы: 
каждый деструктор передается в atexit() по одному.

% TODO а если глобальная STL-переменная в другом модуле? надо проверить.

}

\EN{\section{Returning Values}
\label{ret_val_func}

Another simple function is the one that simply returns a constant value:

\lstinputlisting[caption=\EN{\CCpp Code},style=customc]{patterns/011_ret/1.c}

Let's compile it.

\subsection{x86}

Here's what both the GCC and MSVC compilers produce (with optimization) on the x86 platform:

\lstinputlisting[caption=\Optimizing GCC/MSVC (\assemblyOutput),style=customasmx86]{patterns/011_ret/1.s}

\myindex{x86!\Instructions!RET}
There are just two instructions: the first places the value 123 into the \EAX register,
which is used by convention for storing the return
value, and the second one is \RET, which returns execution to the \gls{caller}.

The caller will take the result from the \EAX register.

\subsection{ARM}

There are a few differences on the ARM platform:

\lstinputlisting[caption=\OptimizingKeilVI (\ARMMode) ASM Output,style=customasmARM]{patterns/011_ret/1_Keil_ARM_O3.s}

ARM uses the register \Reg{0} for returning the results of functions, so 123 is copied into \Reg{0}.

\myindex{ARM!\Instructions!MOV}
\myindex{x86!\Instructions!MOV}
It is worth noting that \MOV is a misleading name for the instruction in both the x86 and ARM \ac{ISA}s.

The data is not in fact \IT{moved}, but \IT{copied}.

\subsection{MIPS}

\label{MIPS_leaf_function_ex1}

The GCC assembly output below lists registers by number:

\lstinputlisting[caption=\Optimizing GCC 4.4.5 (\assemblyOutput),style=customasmMIPS]{patterns/011_ret/MIPS.s}

\dots while \IDA does it by their pseudo names:

\lstinputlisting[caption=\Optimizing GCC 4.4.5 (IDA),style=customasmMIPS]{patterns/011_ret/MIPS_IDA.lst}

The \$2 (or \$V0) register is used to store the function's return value.
\myindex{MIPS!\Pseudoinstructions!LI}
\INS{LI} stands for ``Load Immediate'' and is the MIPS equivalent to \MOV.

\myindex{MIPS!\Instructions!J}
The other instruction is the jump instruction (J or JR) which returns the execution flow to the \gls{caller}.

\myindex{MIPS!Branch delay slot}
You might be wondering why the positions of the load instruction (LI) and the jump instruction (J or JR) are swapped. This is due to a \ac{RISC} feature called ``branch delay slot''.

The reason this happens is a quirk in the architecture of some RISC \ac{ISA}s and isn't important for our
purposes---we must simply keep in mind that in MIPS, the instruction following a jump or branch instruction
is executed \IT{before} the jump/branch instruction itself.

As a consequence, branch instructions always swap places with the instruction executed immediately beforehand.


In practice, functions which merely return 1 (\IT{true}) or 0 (\IT{false}) are very frequent.

The smallest ever of the standard UNIX utilities, \IT{/bin/true} and \IT{/bin/false} return 0 and 1 respectively, as an exit code.
(Zero as an exit code usually means success, non-zero means error.)
}
\RU{\subsubsection{std::string}
\myindex{\Cpp!STL!std::string}
\label{std_string}

\myparagraph{Как устроена структура}

Многие строковые библиотеки \InSqBrackets{\CNotes 2.2} обеспечивают структуру содержащую ссылку 
на буфер собственно со строкой, переменная всегда содержащую длину строки 
(что очень удобно для массы функций \InSqBrackets{\CNotes 2.2.1}) и переменную содержащую текущий размер буфера.

Строка в буфере обыкновенно оканчивается нулем: это для того чтобы указатель на буфер можно было
передавать в функции требующие на вход обычную сишную \ac{ASCIIZ}-строку.

Стандарт \Cpp не описывает, как именно нужно реализовывать std::string,
но, как правило, они реализованы как описано выше, с небольшими дополнениями.

Строки в \Cpp это не класс (как, например, QString в Qt), а темплейт (basic\_string), 
это сделано для того чтобы поддерживать 
строки содержащие разного типа символы: как минимум \Tchar и \IT{wchar\_t}.

Так что, std::string это класс с базовым типом \Tchar.

А std::wstring это класс с базовым типом \IT{wchar\_t}.

\mysubparagraph{MSVC}

В реализации MSVC, вместо ссылки на буфер может содержаться сам буфер (если строка короче 16-и символов).

Это означает, что каждая короткая строка будет занимать в памяти по крайней мере $16 + 4 + 4 = 24$ 
байт для 32-битной среды либо $16 + 8 + 8 = 32$ 
байта в 64-битной, а если строка длиннее 16-и символов, то прибавьте еще длину самой строки.

\lstinputlisting[caption=пример для MSVC,style=customc]{\CURPATH/STL/string/MSVC_RU.cpp}

Собственно, из этого исходника почти всё ясно.

Несколько замечаний:

Если строка короче 16-и символов, 
то отдельный буфер для строки в \glslink{heap}{куче} выделяться не будет.

Это удобно потому что на практике, основная часть строк действительно короткие.
Вероятно, разработчики в Microsoft выбрали размер в 16 символов как разумный баланс.

Теперь очень важный момент в конце функции main(): мы не пользуемся методом c\_str(), тем не менее,
если это скомпилировать и запустить, то обе строки появятся в консоли!

Работает это вот почему.

В первом случае строка короче 16-и символов и в начале объекта std::string (его можно рассматривать
просто как структуру) расположен буфер с этой строкой.
\printf трактует указатель как указатель на массив символов оканчивающийся нулем и поэтому всё работает.

Вывод второй строки (длиннее 16-и символов) даже еще опаснее: это вообще типичная программистская ошибка 
(или опечатка), забыть дописать c\_str().
Это работает потому что в это время в начале структуры расположен указатель на буфер.
Это может надолго остаться незамеченным: до тех пока там не появится строка 
короче 16-и символов, тогда процесс упадет.

\mysubparagraph{GCC}

В реализации GCC в структуре есть еще одна переменная --- reference count.

Интересно, что указатель на экземпляр класса std::string в GCC указывает не на начало самой структуры, 
а на указатель на буфера.
В libstdc++-v3\textbackslash{}include\textbackslash{}bits\textbackslash{}basic\_string.h 
мы можем прочитать что это сделано для удобства отладки:

\begin{lstlisting}
   *  The reason you want _M_data pointing to the character %array and
   *  not the _Rep is so that the debugger can see the string
   *  contents. (Probably we should add a non-inline member to get
   *  the _Rep for the debugger to use, so users can check the actual
   *  string length.)
\end{lstlisting}

\href{http://go.yurichev.com/17085}{исходный код basic\_string.h}

В нашем примере мы учитываем это:

\lstinputlisting[caption=пример для GCC,style=customc]{\CURPATH/STL/string/GCC_RU.cpp}

Нужны еще небольшие хаки чтобы сымитировать типичную ошибку, которую мы уже видели выше, из-за
более ужесточенной проверки типов в GCC, тем не менее, printf() работает и здесь без c\_str().

\myparagraph{Чуть более сложный пример}

\lstinputlisting[style=customc]{\CURPATH/STL/string/3.cpp}

\lstinputlisting[caption=MSVC 2012,style=customasmx86]{\CURPATH/STL/string/3_MSVC_RU.asm}

Собственно, компилятор не конструирует строки статически: да в общем-то и как
это возможно, если буфер с ней нужно хранить в \glslink{heap}{куче}?

Вместо этого в сегменте данных хранятся обычные \ac{ASCIIZ}-строки, а позже, во время выполнения, 
при помощи метода \q{assign}, конструируются строки s1 и s2
.
При помощи \TT{operator+}, создается строка s3.

Обратите внимание на то что вызов метода c\_str() отсутствует,
потому что его код достаточно короткий и компилятор вставил его прямо здесь:
если строка короче 16-и байт, то в регистре EAX остается указатель на буфер,
а если длиннее, то из этого же места достается адрес на буфер расположенный в \glslink{heap}{куче}.

Далее следуют вызовы трех деструкторов, причем, они вызываются только если строка длиннее 16-и байт:
тогда нужно освободить буфера в \glslink{heap}{куче}.
В противном случае, так как все три объекта std::string хранятся в стеке,
они освобождаются автоматически после выхода из функции.

Следовательно, работа с короткими строками более быстрая из-за м\'{е}ньшего обращения к \glslink{heap}{куче}.

Код на GCC даже проще (из-за того, что в GCC, как мы уже видели, не реализована возможность хранить короткую
строку прямо в структуре):

% TODO1 comment each function meaning
\lstinputlisting[caption=GCC 4.8.1,style=customasmx86]{\CURPATH/STL/string/3_GCC_RU.s}

Можно заметить, что в деструкторы передается не указатель на объект,
а указатель на место за 12 байт (или 3 слова) перед ним, то есть, на настоящее начало структуры.

\myparagraph{std::string как глобальная переменная}
\label{sec:std_string_as_global_variable}

Опытные программисты на \Cpp знают, что глобальные переменные \ac{STL}-типов вполне можно объявлять.

Да, действительно:

\lstinputlisting[style=customc]{\CURPATH/STL/string/5.cpp}

Но как и где будет вызываться конструктор \TT{std::string}?

На самом деле, эта переменная будет инициализирована даже перед началом \main.

\lstinputlisting[caption=MSVC 2012: здесь конструируется глобальная переменная{,} а также регистрируется её деструктор,style=customasmx86]{\CURPATH/STL/string/5_MSVC_p2.asm}

\lstinputlisting[caption=MSVC 2012: здесь глобальная переменная используется в \main,style=customasmx86]{\CURPATH/STL/string/5_MSVC_p1.asm}

\lstinputlisting[caption=MSVC 2012: эта функция-деструктор вызывается перед выходом,style=customasmx86]{\CURPATH/STL/string/5_MSVC_p3.asm}

\myindex{\CStandardLibrary!atexit()}
В реальности, из \ac{CRT}, еще до вызова main(), вызывается специальная функция,
в которой перечислены все конструкторы подобных переменных.
Более того: при помощи atexit() регистрируется функция, которая будет вызвана в конце работы программы:
в этой функции компилятор собирает вызовы деструкторов всех подобных глобальных переменных.

GCC работает похожим образом:

\lstinputlisting[caption=GCC 4.8.1,style=customasmx86]{\CURPATH/STL/string/5_GCC.s}

Но он не выделяет отдельной функции в которой будут собраны деструкторы: 
каждый деструктор передается в atexit() по одному.

% TODO а если глобальная STL-переменная в другом модуле? надо проверить.

}
\DE{\subsection{Einfachste XOR-Verschlüsselung überhaupt}

Ich habe einmal eine Software gesehen, bei der alle Debugging-Ausgaben mit XOR mit dem Wert 3
verschlüsselt wurden. Mit anderen Worten, die beiden niedrigsten Bits aller Buchstaben wurden invertiert.

``Hello, world'' wurde zu ``Kfool/\#tlqog'':

\begin{lstlisting}
#!/usr/bin/python

msg="Hello, world!"

print "".join(map(lambda x: chr(ord(x)^3), msg))
\end{lstlisting}

Das ist eine ziemlich interessante Verschlüsselung (oder besser eine Verschleierung),
weil sie zwei wichtige Eigenschaften hat:
1) es ist eine einzige Funktion zum Verschlüsseln und entschlüsseln, sie muss nur wiederholt angewendet werden
2) die entstehenden Buchstaben befinden sich im druckbaren Bereich, also die ganze Zeichenkette kann ohne
Escape-Symbole im Code verwendet werden.

Die zweite Eigenschaft nutzt die Tatsache, dass alle druckbaren Zeichen in Reihen organisiert sind: 0x2x-0x7x,
und wenn die beiden niederwertigsten Bits invertiert werden, wird der Buchstabe um eine oder drei Stellen nach
links oder rechts \IT{verschoben}, aber niemals in eine andere Reihe:

\begin{figure}[H]
\centering
\includegraphics[width=0.7\textwidth]{ascii_clean.png}
\caption{7-Bit \ac{ASCII} Tabelle in Emacs}
\end{figure}

\dots mit dem Zeichen 0x7F als einziger Ausnahme.

Im Folgenden werden also beispielsweise die Zeichen A-Z \IT{verschlüsselt}:

\begin{lstlisting}
#!/usr/bin/python

msg="@ABCDEFGHIJKLMNO"

print "".join(map(lambda x: chr(ord(x)^3), msg))
\end{lstlisting}

Ergebnis:
% FIXME \verb  --  relevant comment for German?
\begin{lstlisting}
CBA@GFEDKJIHONML
\end{lstlisting}

Es sieht so aus als würden die Zeichen ``@'' und ``C'' sowie ``B'' und ``A'' vertauscht werden.

Hier ist noch ein interessantes Beispiel, in dem gezeigt wird, wie die Eigenschaften von XOR
ausgenutzt werden können: Exakt den gleichen Effekt, dass druckbare Zeichen auch druckbar bleiben,
kann man dadurch erzielen, dass irgendeine Kombination der niedrigsten vier Bits invertiert wird.
}

\EN{\section{Returning Values}
\label{ret_val_func}

Another simple function is the one that simply returns a constant value:

\lstinputlisting[caption=\EN{\CCpp Code},style=customc]{patterns/011_ret/1.c}

Let's compile it.

\subsection{x86}

Here's what both the GCC and MSVC compilers produce (with optimization) on the x86 platform:

\lstinputlisting[caption=\Optimizing GCC/MSVC (\assemblyOutput),style=customasmx86]{patterns/011_ret/1.s}

\myindex{x86!\Instructions!RET}
There are just two instructions: the first places the value 123 into the \EAX register,
which is used by convention for storing the return
value, and the second one is \RET, which returns execution to the \gls{caller}.

The caller will take the result from the \EAX register.

\subsection{ARM}

There are a few differences on the ARM platform:

\lstinputlisting[caption=\OptimizingKeilVI (\ARMMode) ASM Output,style=customasmARM]{patterns/011_ret/1_Keil_ARM_O3.s}

ARM uses the register \Reg{0} for returning the results of functions, so 123 is copied into \Reg{0}.

\myindex{ARM!\Instructions!MOV}
\myindex{x86!\Instructions!MOV}
It is worth noting that \MOV is a misleading name for the instruction in both the x86 and ARM \ac{ISA}s.

The data is not in fact \IT{moved}, but \IT{copied}.

\subsection{MIPS}

\label{MIPS_leaf_function_ex1}

The GCC assembly output below lists registers by number:

\lstinputlisting[caption=\Optimizing GCC 4.4.5 (\assemblyOutput),style=customasmMIPS]{patterns/011_ret/MIPS.s}

\dots while \IDA does it by their pseudo names:

\lstinputlisting[caption=\Optimizing GCC 4.4.5 (IDA),style=customasmMIPS]{patterns/011_ret/MIPS_IDA.lst}

The \$2 (or \$V0) register is used to store the function's return value.
\myindex{MIPS!\Pseudoinstructions!LI}
\INS{LI} stands for ``Load Immediate'' and is the MIPS equivalent to \MOV.

\myindex{MIPS!\Instructions!J}
The other instruction is the jump instruction (J or JR) which returns the execution flow to the \gls{caller}.

\myindex{MIPS!Branch delay slot}
You might be wondering why the positions of the load instruction (LI) and the jump instruction (J or JR) are swapped. This is due to a \ac{RISC} feature called ``branch delay slot''.

The reason this happens is a quirk in the architecture of some RISC \ac{ISA}s and isn't important for our
purposes---we must simply keep in mind that in MIPS, the instruction following a jump or branch instruction
is executed \IT{before} the jump/branch instruction itself.

As a consequence, branch instructions always swap places with the instruction executed immediately beforehand.


In practice, functions which merely return 1 (\IT{true}) or 0 (\IT{false}) are very frequent.

The smallest ever of the standard UNIX utilities, \IT{/bin/true} and \IT{/bin/false} return 0 and 1 respectively, as an exit code.
(Zero as an exit code usually means success, non-zero means error.)
}
\RU{\subsubsection{std::string}
\myindex{\Cpp!STL!std::string}
\label{std_string}

\myparagraph{Как устроена структура}

Многие строковые библиотеки \InSqBrackets{\CNotes 2.2} обеспечивают структуру содержащую ссылку 
на буфер собственно со строкой, переменная всегда содержащую длину строки 
(что очень удобно для массы функций \InSqBrackets{\CNotes 2.2.1}) и переменную содержащую текущий размер буфера.

Строка в буфере обыкновенно оканчивается нулем: это для того чтобы указатель на буфер можно было
передавать в функции требующие на вход обычную сишную \ac{ASCIIZ}-строку.

Стандарт \Cpp не описывает, как именно нужно реализовывать std::string,
но, как правило, они реализованы как описано выше, с небольшими дополнениями.

Строки в \Cpp это не класс (как, например, QString в Qt), а темплейт (basic\_string), 
это сделано для того чтобы поддерживать 
строки содержащие разного типа символы: как минимум \Tchar и \IT{wchar\_t}.

Так что, std::string это класс с базовым типом \Tchar.

А std::wstring это класс с базовым типом \IT{wchar\_t}.

\mysubparagraph{MSVC}

В реализации MSVC, вместо ссылки на буфер может содержаться сам буфер (если строка короче 16-и символов).

Это означает, что каждая короткая строка будет занимать в памяти по крайней мере $16 + 4 + 4 = 24$ 
байт для 32-битной среды либо $16 + 8 + 8 = 32$ 
байта в 64-битной, а если строка длиннее 16-и символов, то прибавьте еще длину самой строки.

\lstinputlisting[caption=пример для MSVC,style=customc]{\CURPATH/STL/string/MSVC_RU.cpp}

Собственно, из этого исходника почти всё ясно.

Несколько замечаний:

Если строка короче 16-и символов, 
то отдельный буфер для строки в \glslink{heap}{куче} выделяться не будет.

Это удобно потому что на практике, основная часть строк действительно короткие.
Вероятно, разработчики в Microsoft выбрали размер в 16 символов как разумный баланс.

Теперь очень важный момент в конце функции main(): мы не пользуемся методом c\_str(), тем не менее,
если это скомпилировать и запустить, то обе строки появятся в консоли!

Работает это вот почему.

В первом случае строка короче 16-и символов и в начале объекта std::string (его можно рассматривать
просто как структуру) расположен буфер с этой строкой.
\printf трактует указатель как указатель на массив символов оканчивающийся нулем и поэтому всё работает.

Вывод второй строки (длиннее 16-и символов) даже еще опаснее: это вообще типичная программистская ошибка 
(или опечатка), забыть дописать c\_str().
Это работает потому что в это время в начале структуры расположен указатель на буфер.
Это может надолго остаться незамеченным: до тех пока там не появится строка 
короче 16-и символов, тогда процесс упадет.

\mysubparagraph{GCC}

В реализации GCC в структуре есть еще одна переменная --- reference count.

Интересно, что указатель на экземпляр класса std::string в GCC указывает не на начало самой структуры, 
а на указатель на буфера.
В libstdc++-v3\textbackslash{}include\textbackslash{}bits\textbackslash{}basic\_string.h 
мы можем прочитать что это сделано для удобства отладки:

\begin{lstlisting}
   *  The reason you want _M_data pointing to the character %array and
   *  not the _Rep is so that the debugger can see the string
   *  contents. (Probably we should add a non-inline member to get
   *  the _Rep for the debugger to use, so users can check the actual
   *  string length.)
\end{lstlisting}

\href{http://go.yurichev.com/17085}{исходный код basic\_string.h}

В нашем примере мы учитываем это:

\lstinputlisting[caption=пример для GCC,style=customc]{\CURPATH/STL/string/GCC_RU.cpp}

Нужны еще небольшие хаки чтобы сымитировать типичную ошибку, которую мы уже видели выше, из-за
более ужесточенной проверки типов в GCC, тем не менее, printf() работает и здесь без c\_str().

\myparagraph{Чуть более сложный пример}

\lstinputlisting[style=customc]{\CURPATH/STL/string/3.cpp}

\lstinputlisting[caption=MSVC 2012,style=customasmx86]{\CURPATH/STL/string/3_MSVC_RU.asm}

Собственно, компилятор не конструирует строки статически: да в общем-то и как
это возможно, если буфер с ней нужно хранить в \glslink{heap}{куче}?

Вместо этого в сегменте данных хранятся обычные \ac{ASCIIZ}-строки, а позже, во время выполнения, 
при помощи метода \q{assign}, конструируются строки s1 и s2
.
При помощи \TT{operator+}, создается строка s3.

Обратите внимание на то что вызов метода c\_str() отсутствует,
потому что его код достаточно короткий и компилятор вставил его прямо здесь:
если строка короче 16-и байт, то в регистре EAX остается указатель на буфер,
а если длиннее, то из этого же места достается адрес на буфер расположенный в \glslink{heap}{куче}.

Далее следуют вызовы трех деструкторов, причем, они вызываются только если строка длиннее 16-и байт:
тогда нужно освободить буфера в \glslink{heap}{куче}.
В противном случае, так как все три объекта std::string хранятся в стеке,
они освобождаются автоматически после выхода из функции.

Следовательно, работа с короткими строками более быстрая из-за м\'{е}ньшего обращения к \glslink{heap}{куче}.

Код на GCC даже проще (из-за того, что в GCC, как мы уже видели, не реализована возможность хранить короткую
строку прямо в структуре):

% TODO1 comment each function meaning
\lstinputlisting[caption=GCC 4.8.1,style=customasmx86]{\CURPATH/STL/string/3_GCC_RU.s}

Можно заметить, что в деструкторы передается не указатель на объект,
а указатель на место за 12 байт (или 3 слова) перед ним, то есть, на настоящее начало структуры.

\myparagraph{std::string как глобальная переменная}
\label{sec:std_string_as_global_variable}

Опытные программисты на \Cpp знают, что глобальные переменные \ac{STL}-типов вполне можно объявлять.

Да, действительно:

\lstinputlisting[style=customc]{\CURPATH/STL/string/5.cpp}

Но как и где будет вызываться конструктор \TT{std::string}?

На самом деле, эта переменная будет инициализирована даже перед началом \main.

\lstinputlisting[caption=MSVC 2012: здесь конструируется глобальная переменная{,} а также регистрируется её деструктор,style=customasmx86]{\CURPATH/STL/string/5_MSVC_p2.asm}

\lstinputlisting[caption=MSVC 2012: здесь глобальная переменная используется в \main,style=customasmx86]{\CURPATH/STL/string/5_MSVC_p1.asm}

\lstinputlisting[caption=MSVC 2012: эта функция-деструктор вызывается перед выходом,style=customasmx86]{\CURPATH/STL/string/5_MSVC_p3.asm}

\myindex{\CStandardLibrary!atexit()}
В реальности, из \ac{CRT}, еще до вызова main(), вызывается специальная функция,
в которой перечислены все конструкторы подобных переменных.
Более того: при помощи atexit() регистрируется функция, которая будет вызвана в конце работы программы:
в этой функции компилятор собирает вызовы деструкторов всех подобных глобальных переменных.

GCC работает похожим образом:

\lstinputlisting[caption=GCC 4.8.1,style=customasmx86]{\CURPATH/STL/string/5_GCC.s}

Но он не выделяет отдельной функции в которой будут собраны деструкторы: 
каждый деструктор передается в atexit() по одному.

% TODO а если глобальная STL-переменная в другом модуле? надо проверить.

}
\DE{\subsection{Einfachste XOR-Verschlüsselung überhaupt}

Ich habe einmal eine Software gesehen, bei der alle Debugging-Ausgaben mit XOR mit dem Wert 3
verschlüsselt wurden. Mit anderen Worten, die beiden niedrigsten Bits aller Buchstaben wurden invertiert.

``Hello, world'' wurde zu ``Kfool/\#tlqog'':

\begin{lstlisting}
#!/usr/bin/python

msg="Hello, world!"

print "".join(map(lambda x: chr(ord(x)^3), msg))
\end{lstlisting}

Das ist eine ziemlich interessante Verschlüsselung (oder besser eine Verschleierung),
weil sie zwei wichtige Eigenschaften hat:
1) es ist eine einzige Funktion zum Verschlüsseln und entschlüsseln, sie muss nur wiederholt angewendet werden
2) die entstehenden Buchstaben befinden sich im druckbaren Bereich, also die ganze Zeichenkette kann ohne
Escape-Symbole im Code verwendet werden.

Die zweite Eigenschaft nutzt die Tatsache, dass alle druckbaren Zeichen in Reihen organisiert sind: 0x2x-0x7x,
und wenn die beiden niederwertigsten Bits invertiert werden, wird der Buchstabe um eine oder drei Stellen nach
links oder rechts \IT{verschoben}, aber niemals in eine andere Reihe:

\begin{figure}[H]
\centering
\includegraphics[width=0.7\textwidth]{ascii_clean.png}
\caption{7-Bit \ac{ASCII} Tabelle in Emacs}
\end{figure}

\dots mit dem Zeichen 0x7F als einziger Ausnahme.

Im Folgenden werden also beispielsweise die Zeichen A-Z \IT{verschlüsselt}:

\begin{lstlisting}
#!/usr/bin/python

msg="@ABCDEFGHIJKLMNO"

print "".join(map(lambda x: chr(ord(x)^3), msg))
\end{lstlisting}

Ergebnis:
% FIXME \verb  --  relevant comment for German?
\begin{lstlisting}
CBA@GFEDKJIHONML
\end{lstlisting}

Es sieht so aus als würden die Zeichen ``@'' und ``C'' sowie ``B'' und ``A'' vertauscht werden.

Hier ist noch ein interessantes Beispiel, in dem gezeigt wird, wie die Eigenschaften von XOR
ausgenutzt werden können: Exakt den gleichen Effekt, dass druckbare Zeichen auch druckbar bleiben,
kann man dadurch erzielen, dass irgendeine Kombination der niedrigsten vier Bits invertiert wird.
}

\ifdefined\SPANISH
\chapter{Patrones de código}
\fi % SPANISH

\ifdefined\GERMAN
\chapter{Code-Muster}
\fi % GERMAN

\ifdefined\ENGLISH
\chapter{Code Patterns}
\fi % ENGLISH

\ifdefined\ITALIAN
\chapter{Forme di codice}
\fi % ITALIAN

\ifdefined\RUSSIAN
\chapter{Образцы кода}
\fi % RUSSIAN

\ifdefined\BRAZILIAN
\chapter{Padrões de códigos}
\fi % BRAZILIAN

\ifdefined\THAI
\chapter{รูปแบบของโค้ด}
\fi % THAI

\ifdefined\FRENCH
\chapter{Modèle de code}
\fi % FRENCH

\ifdefined\POLISH
\chapter{\PLph{}}
\fi % POLISH

% sections
\EN{\input{patterns/patterns_opt_dbg_EN}}
\ES{\input{patterns/patterns_opt_dbg_ES}}
\ITA{\input{patterns/patterns_opt_dbg_ITA}}
\PTBR{\input{patterns/patterns_opt_dbg_PTBR}}
\RU{\input{patterns/patterns_opt_dbg_RU}}
\THA{\input{patterns/patterns_opt_dbg_THA}}
\DE{\input{patterns/patterns_opt_dbg_DE}}
\FR{\input{patterns/patterns_opt_dbg_FR}}
\PL{\input{patterns/patterns_opt_dbg_PL}}

\RU{\section{Некоторые базовые понятия}}
\EN{\section{Some basics}}
\DE{\section{Einige Grundlagen}}
\FR{\section{Quelques bases}}
\ES{\section{\ESph{}}}
\ITA{\section{Alcune basi teoriche}}
\PTBR{\section{\PTBRph{}}}
\THA{\section{\THAph{}}}
\PL{\section{\PLph{}}}

% sections:
\EN{\input{patterns/intro_CPU_ISA_EN}}
\ES{\input{patterns/intro_CPU_ISA_ES}}
\ITA{\input{patterns/intro_CPU_ISA_ITA}}
\PTBR{\input{patterns/intro_CPU_ISA_PTBR}}
\RU{\input{patterns/intro_CPU_ISA_RU}}
\DE{\input{patterns/intro_CPU_ISA_DE}}
\FR{\input{patterns/intro_CPU_ISA_FR}}
\PL{\input{patterns/intro_CPU_ISA_PL}}

\EN{\input{patterns/numeral_EN}}
\RU{\input{patterns/numeral_RU}}
\ITA{\input{patterns/numeral_ITA}}
\DE{\input{patterns/numeral_DE}}
\FR{\input{patterns/numeral_FR}}
\PL{\input{patterns/numeral_PL}}

% chapters
\input{patterns/00_empty/main}
\input{patterns/011_ret/main}
\input{patterns/01_helloworld/main}
\input{patterns/015_prolog_epilogue/main}
\input{patterns/02_stack/main}
\input{patterns/03_printf/main}
\input{patterns/04_scanf/main}
\input{patterns/05_passing_arguments/main}
\input{patterns/06_return_results/main}
\input{patterns/061_pointers/main}
\input{patterns/065_GOTO/main}
\input{patterns/07_jcc/main}
\input{patterns/08_switch/main}
\input{patterns/09_loops/main}
\input{patterns/10_strings/main}
\input{patterns/11_arith_optimizations/main}
\input{patterns/12_FPU/main}
\input{patterns/13_arrays/main}
\input{patterns/14_bitfields/main}
\EN{\input{patterns/145_LCG/main_EN}}
\RU{\input{patterns/145_LCG/main_RU}}
\input{patterns/15_structs/main}
\input{patterns/17_unions/main}
\input{patterns/18_pointers_to_functions/main}
\input{patterns/185_64bit_in_32_env/main}

\EN{\input{patterns/19_SIMD/main_EN}}
\RU{\input{patterns/19_SIMD/main_RU}}
\DE{\input{patterns/19_SIMD/main_DE}}

\EN{\input{patterns/20_x64/main_EN}}
\RU{\input{patterns/20_x64/main_RU}}

\EN{\input{patterns/205_floating_SIMD/main_EN}}
\RU{\input{patterns/205_floating_SIMD/main_RU}}
\DE{\input{patterns/205_floating_SIMD/main_DE}}

\EN{\input{patterns/ARM/main_EN}}
\RU{\input{patterns/ARM/main_RU}}
\DE{\input{patterns/ARM/main_DE}}

\input{patterns/MIPS/main}


\ifdefined\SPANISH
\chapter{Patrones de código}
\fi % SPANISH

\ifdefined\GERMAN
\chapter{Code-Muster}
\fi % GERMAN

\ifdefined\ENGLISH
\chapter{Code Patterns}
\fi % ENGLISH

\ifdefined\ITALIAN
\chapter{Forme di codice}
\fi % ITALIAN

\ifdefined\RUSSIAN
\chapter{Образцы кода}
\fi % RUSSIAN

\ifdefined\BRAZILIAN
\chapter{Padrões de códigos}
\fi % BRAZILIAN

\ifdefined\THAI
\chapter{รูปแบบของโค้ด}
\fi % THAI

\ifdefined\FRENCH
\chapter{Modèle de code}
\fi % FRENCH

\ifdefined\POLISH
\chapter{\PLph{}}
\fi % POLISH

% sections
\EN{\section{The method}

When the author of this book first started learning C and, later, \Cpp, he used to write small pieces of code, compile them,
and then look at the assembly language output. This made it very easy for him to understand what was going on in the code that he had written.
\footnote{In fact, he still does this when he can't understand what a particular bit of code does.}.
He did this so many times that the relationship between the \CCpp code and what the compiler produced was imprinted deeply in his mind.
It's now easy for him to imagine instantly a rough outline of a C code's appearance and function.
Perhaps this technique could be helpful for others.

%There are a lot of examples for both x86/x64 and ARM.
%Those who already familiar with one of architectures, may freely skim over pages.

By the way, there is a great website where you can do the same, with various compilers, instead of installing them on your box.
You can use it as well: \url{https://gcc.godbolt.org/}.

\section*{\Exercises}

When the author of this book studied assembly language, he also often compiled small C functions and then rewrote
them gradually to assembly, trying to make their code as short as possible.
This probably is not worth doing in real-world scenarios today,
because it's hard to compete with the latest compilers in terms of efficiency. It is, however, a very good way to gain a better understanding of assembly.
Feel free, therefore, to take any assembly code from this book and try to make it shorter.
However, don't forget to test what you have written.

% rewrote to show that debug\release and optimisations levels are orthogonal concepts.
\section*{Optimization levels and debug information}

Source code can be compiled by different compilers with various optimization levels.
A typical compiler has about three such levels, where level zero means that optimization is completely disabled.
Optimization can also be targeted towards code size or code speed.
A non-optimizing compiler is faster and produces more understandable (albeit verbose) code,
whereas an optimizing compiler is slower and tries to produce code that runs faster (but is not necessarily more compact).
In addition to optimization levels, a compiler can include some debug information in the resulting file,
producing code that is easy to debug.
One of the important features of the ´debug' code is that it might contain links
between each line of the source code and its respective machine code address.
Optimizing compilers, on the other hand, tend to produce output where entire lines of source code
can be optimized away and thus not even be present in the resulting machine code.
Reverse engineers can encounter either version, simply because some developers turn on the compiler's optimization flags and others do not.
Because of this, we'll try to work on examples of both debug and release versions of the code featured in this book, wherever possible.

Sometimes some pretty ancient compilers are used in this book, in order to get the shortest (or simplest) possible code snippet.
}
\ES{\input{patterns/patterns_opt_dbg_ES}}
\ITA{\input{patterns/patterns_opt_dbg_ITA}}
\PTBR{\input{patterns/patterns_opt_dbg_PTBR}}
\RU{\input{patterns/patterns_opt_dbg_RU}}
\THA{\input{patterns/patterns_opt_dbg_THA}}
\DE{\input{patterns/patterns_opt_dbg_DE}}
\FR{\input{patterns/patterns_opt_dbg_FR}}
\PL{\input{patterns/patterns_opt_dbg_PL}}

\RU{\section{Некоторые базовые понятия}}
\EN{\section{Some basics}}
\DE{\section{Einige Grundlagen}}
\FR{\section{Quelques bases}}
\ES{\section{\ESph{}}}
\ITA{\section{Alcune basi teoriche}}
\PTBR{\section{\PTBRph{}}}
\THA{\section{\THAph{}}}
\PL{\section{\PLph{}}}

% sections:
\EN{\input{patterns/intro_CPU_ISA_EN}}
\ES{\input{patterns/intro_CPU_ISA_ES}}
\ITA{\input{patterns/intro_CPU_ISA_ITA}}
\PTBR{\input{patterns/intro_CPU_ISA_PTBR}}
\RU{\input{patterns/intro_CPU_ISA_RU}}
\DE{\input{patterns/intro_CPU_ISA_DE}}
\FR{\input{patterns/intro_CPU_ISA_FR}}
\PL{\input{patterns/intro_CPU_ISA_PL}}

\EN{\subsection{Numeral Systems}

Humans have become accustomed to a decimal numeral system, probably because almost everyone has 10 fingers.
Nevertheless, the number \q{10} has no significant meaning in science and mathematics.
The natural numeral system in digital electronics is binary: 0 is for an absence of current in the wire, and 1 for presence.
10 in binary is 2 in decimal, 100 in binary is 4 in decimal, and so on.

% This sentence is a bit unweildy - maybe try 'Our ten-digit system would be described as having a radix...' - Renaissance
If the numeral system has 10 digits, it has a \IT{radix} (or \IT{base}) of 10.
The binary numeral system has a \IT{radix} of 2.

Important things to recall:

1) A \IT{number} is a number, while a \IT{digit} is a term from writing systems, and is usually one character

% The original is 'number' is not changed; I think the intent is value, and changed it - Renaissance
2) The value of a number does not change when converted to another radix; only the writing notation for that value has changed (and therefore the way of representing it in \ac{RAM}).

\subsection{Converting From One Radix To Another}

Positional notation is used almost every numerical system. This means that a digit has weight relative to where it is placed inside of the larger number.
If 2 is placed at the rightmost place, it's 2, but if it's placed one digit before rightmost, it's 20.

What does $1234$ stand for?

$10^3 \cdot 1 + 10^2 \cdot 2 + 10^1 \cdot 3 + 1 \cdot 4 = 1234$ or
$1000 \cdot 1 + 100 \cdot 2 + 10 \cdot 3 + 4 = 1234$

It's the same story for binary numbers, but the base is 2 instead of 10.
What does 0b101011 stand for?

$2^5 \cdot 1 + 2^4 \cdot 0 + 2^3 \cdot 1 + 2^2 \cdot 0 + 2^1 \cdot 1 + 2^0 \cdot 1 = 43$ or
$32 \cdot 1 + 16 \cdot 0 + 8 \cdot 1 + 4 \cdot 0 + 2 \cdot 1 + 1 = 43$

There is such a thing as non-positional notation, such as the Roman numeral system.
\footnote{About numeric system evolution, see \InSqBrackets{\TAOCPvolII{}, 195--213.}}.
% Maybe add a sentence to fill in that X is always 10, and is therefore non-positional, even though putting an I before subtracts and after adds, and is in that sense positional
Perhaps, humankind switched to positional notation because it's easier to do basic operations (addition, multiplication, etc.) on paper by hand.

Binary numbers can be added, subtracted and so on in the very same as taught in schools, but only 2 digits are available.

Binary numbers are bulky when represented in source code and dumps, so that is where the hexadecimal numeral system can be useful.
A hexadecimal radix uses the digits 0..9, and also 6 Latin characters: A..F.
Each hexadecimal digit takes 4 bits or 4 binary digits, so it's very easy to convert from binary number to hexadecimal and back, even manually, in one's mind.

\begin{center}
\begin{longtable}{ | l | l | l | }
\hline
\HeaderColor hexadecimal & \HeaderColor binary & \HeaderColor decimal \\
\hline
0	&0000	&0 \\
1	&0001	&1 \\
2	&0010	&2 \\
3	&0011	&3 \\
4	&0100	&4 \\
5	&0101	&5 \\
6	&0110	&6 \\
7	&0111	&7 \\
8	&1000	&8 \\
9	&1001	&9 \\
A	&1010	&10 \\
B	&1011	&11 \\
C	&1100	&12 \\
D	&1101	&13 \\
E	&1110	&14 \\
F	&1111	&15 \\
\hline
\end{longtable}
\end{center}

How can one tell which radix is being used in a specific instance?

Decimal numbers are usually written as is, i.e., 1234. Some assemblers allow an identifier on decimal radix numbers, in which the number would be written with a "d" suffix: 1234d.

Binary numbers are sometimes prepended with the "0b" prefix: 0b100110111 (\ac{GCC} has a non-standard language extension for this\footnote{\url{https://gcc.gnu.org/onlinedocs/gcc/Binary-constants.html}}).
There is also another way: using a "b" suffix, for example: 100110111b.
This book tries to use the "0b" prefix consistently throughout the book for binary numbers.

Hexadecimal numbers are prepended with "0x" prefix in \CCpp and other \ac{PL}s: 0x1234ABCD.
Alternatively, they are given a "h" suffix: 1234ABCDh. This is common way of representing them in assemblers and debuggers.
In this convention, if the number is started with a Latin (A..F) digit, a 0 is added at the beginning: 0ABCDEFh.
There was also convention that was popular in 8-bit home computers era, using \$ prefix, like \$ABCD.
The book will try to stick to "0x" prefix throughout the book for hexadecimal numbers.

Should one learn to convert numbers mentally? A table of 1-digit hexadecimal numbers can easily be memorized.
As for larger numbers, it's probably not worth tormenting yourself.

Perhaps the most visible hexadecimal numbers are in \ac{URL}s.
This is the way that non-Latin characters are encoded.
For example:
\url{https://en.wiktionary.org/wiki/na\%C3\%AFvet\%C3\%A9} is the \ac{URL} of Wiktionary article about \q{naïveté} word.

\subsubsection{Octal Radix}

Another numeral system heavily used in the past of computer programming is octal. In octal there are 8 digits (0..7), and each is mapped to 3 bits, so it's easy to convert numbers back and forth.
It has been superseded by the hexadecimal system almost everywhere, but, surprisingly, there is a *NIX utility, used often by many people, which takes octal numbers as argument: \TT{chmod}.

\myindex{UNIX!chmod}
As many *NIX users know, \TT{chmod} argument can be a number of 3 digits. The first digit represents the rights of the owner of the file (read, write and/or execute), the second is the rights for the group to which the file belongs, and the third is for everyone else.
Each digit that \TT{chmod} takes can be represented in binary form:

\begin{center}
\begin{longtable}{ | l | l | l | }
\hline
\HeaderColor decimal & \HeaderColor binary & \HeaderColor meaning \\
\hline
7	&111	&\textbf{rwx} \\
6	&110	&\textbf{rw-} \\
5	&101	&\textbf{r-x} \\
4	&100	&\textbf{r-{}-} \\
3	&011	&\textbf{-wx} \\
2	&010	&\textbf{-w-} \\
1	&001	&\textbf{-{}-x} \\
0	&000	&\textbf{-{}-{}-} \\
\hline
\end{longtable}
\end{center}

So each bit is mapped to a flag: read/write/execute.

The importance of \TT{chmod} here is that the whole number in argument can be represented as octal number.
Let's take, for example, 644.
When you run \TT{chmod 644 file}, you set read/write permissions for owner, read permissions for group and again, read permissions for everyone else.
If we convert the octal number 644 to binary, it would be \TT{110100100}, or, in groups of 3 bits, \TT{110 100 100}.

Now we see that each triplet describe permissions for owner/group/others: first is \TT{rw-}, second is \TT{r--} and third is \TT{r--}.

The octal numeral system was also popular on old computers like PDP-8, because word there could be 12, 24 or 36 bits, and these numbers are all divisible by 3, so the octal system was natural in that environment.
Nowadays, all popular computers employ word/address sizes of 16, 32 or 64 bits, and these numbers are all divisible by 4, so the hexadecimal system is more natural there.

The octal numeral system is supported by all standard \CCpp compilers.
This is a source of confusion sometimes, because octal numbers are encoded with a zero prepended, for example, 0377 is 255.
Sometimes, you might make a typo and write "09" instead of 9, and the compiler would report an error.
GCC might report something like this:\\
\TT{error: invalid digit "9" in octal constant}.

Also, the octal system is somewhat popular in Java. When the IDA shows Java strings with non-printable characters,
they are encoded in the octal system instead of hexadecimal.
\myindex{JAD}
The JAD Java decompiler behaves the same way.

\subsubsection{Divisibility}

When you see a decimal number like 120, you can quickly deduce that it's divisible by 10, because the last digit is zero.
In the same way, 123400 is divisible by 100, because the two last digits are zeros.

Likewise, the hexadecimal number 0x1230 is divisible by 0x10 (or 16), 0x123000 is divisible by 0x1000 (or 4096), etc.

The binary number 0b1000101000 is divisible by 0b1000 (8), etc.

This property can often be used to quickly realize if the size of some block in memory is padded to some boundary.
For example, sections in \ac{PE} files are almost always started at addresses ending with 3 hexadecimal zeros: 0x41000, 0x10001000, etc.
The reason behind this is the fact that almost all \ac{PE} sections are padded to a boundary of 0x1000 (4096) bytes.

\subsubsection{Multi-Precision Arithmetic and Radix}

\index{RSA}
Multi-precision arithmetic can use huge numbers, and each one may be stored in several bytes.
For example, RSA keys, both public and private, span up to 4096 bits, and maybe even more.

% I'm not sure how to change this, but the normal format for quoting would be just to mention the author or book, and footnote to the full reference
In \InSqBrackets{\TAOCPvolII, 265} we find the following idea: when you store a multi-precision number in several bytes,
the whole number can be represented as having a radix of $2^8=256$, and each digit goes to the corresponding byte.
Likewise, if you store a multi-precision number in several 32-bit integer values, each digit goes to each 32-bit slot,
and you may think about this number as stored in radix of $2^{32}$.

\subsubsection{How to Pronounce Non-Decimal Numbers}

Numbers in a non-decimal base are usually pronounced by digit by digit: ``one-zero-zero-one-one-...''.
Words like ``ten'' and ``thousand'' are usually not pronounced, to prevent confusion with the decimal base system.

\subsubsection{Floating point numbers}

To distinguish floating point numbers from integers, they are usually written with ``.0'' at the end,
like $0.0$, $123.0$, etc.
}
\RU{\subsection{Представление чисел}

Люди привыкли к десятичной системе счисления вероятно потому что почти у каждого есть по 10 пальцев.
Тем не менее, число 10 не имеет особого значения в науке и математике.
Двоичная система естествена для цифровой электроники: 0 означает отсутствие тока в проводе и 1 --- его присутствие.
10 в двоичной системе это 2 в десятичной; 100 в двоичной это 4 в десятичной, итд.

Если в системе счисления есть 10 цифр, её \IT{основание} или \IT{radix} это 10.
Двоичная система имеет \IT{основание} 2.

Важные вещи, которые полезно вспомнить:
1) \IT{число} это число, в то время как \IT{цифра} это термин из системы письменности, и это обычно один символ;
2) само число не меняется, когда конвертируется из одного основания в другое: меняется способ его записи (или представления
в памяти).

Как сконвертировать число из одного основания в другое?

Позиционная нотация используется почти везде, это означает, что всякая цифра имеет свой вес, в зависимости от её расположения
внутри числа.
Если 2 расположена в самом последнем месте справа, это 2.
Если она расположена в месте перед последним, это 20.

Что означает $1234$?

$10^3 \cdot 1 + 10^2 \cdot 2 + 10^1 \cdot 3 + 1 \cdot 4$ = 1234 или
$1000 \cdot 1 + 100 \cdot 2 + 10 \cdot 3 + 4 = 1234$

Та же история и для двоичных чисел, только основание там 2 вместо 10.
Что означает 0b101011?

$2^5 \cdot 1 + 2^4 \cdot 0 + 2^3 \cdot 1 + 2^2 \cdot 0 + 2^1 \cdot 1 + 2^0 \cdot 1 = 43$ или
$32 \cdot 1 + 16 \cdot 0 + 8 \cdot 1 + 4 \cdot 0 + 2 \cdot 1 + 1 = 43$

Позиционную нотацию можно противопоставить непозиционной нотации, такой как римская система записи чисел
\footnote{Об эволюции способов записи чисел, см.также: \InSqBrackets{\TAOCPvolII{}, 195--213.}}.
Вероятно, человечество перешло на позиционную нотацию, потому что так проще работать с числами (сложение, умножение, итд)
на бумаге, в ручную.

Действительно, двоичные числа можно складывать, вычитать, итд, точно также, как этому обычно обучают в школах,
только доступны лишь 2 цифры.

Двоичные числа громоздки, когда их используют в исходных кодах и дампах, так что в этих случаях применяется шестнадцатеричная
система.
Используются цифры 0..9 и еще 6 латинских букв: A..F.
Каждая шестнадцатеричная цифра занимает 4 бита или 4 двоичных цифры, так что конвертировать из двоичной системы в
шестнадцатеричную и назад, можно легко вручную, или даже в уме.

\begin{center}
\begin{longtable}{ | l | l | l | }
\hline
\HeaderColor шестнадцатеричная & \HeaderColor двоичная & \HeaderColor десятичная \\
\hline
0	&0000	&0 \\
1	&0001	&1 \\
2	&0010	&2 \\
3	&0011	&3 \\
4	&0100	&4 \\
5	&0101	&5 \\
6	&0110	&6 \\
7	&0111	&7 \\
8	&1000	&8 \\
9	&1001	&9 \\
A	&1010	&10 \\
B	&1011	&11 \\
C	&1100	&12 \\
D	&1101	&13 \\
E	&1110	&14 \\
F	&1111	&15 \\
\hline
\end{longtable}
\end{center}

Как понять, какое основание используется в конкретном месте?

Десятичные числа обычно записываются как есть, т.е., 1234. Но некоторые ассемблеры позволяют подчеркивать
этот факт для ясности, и это число может быть дополнено суффиксом "d": 1234d.

К двоичным числам иногда спереди добавляют префикс "0b": 0b100110111
(В \ac{GCC} для этого есть нестандартное расширение языка
\footnote{\url{https://gcc.gnu.org/onlinedocs/gcc/Binary-constants.html}}).
Есть также еще один способ: суффикс "b", например: 100110111b.
В этой книге я буду пытаться придерживаться префикса "0b" для двоичных чисел.

Шестнадцатеричные числа имеют префикс "0x" в \CCpp и некоторых других \ac{PL}: 0x1234ABCD.
Либо они имеют суффикс "h": 1234ABCDh --- обычно так они представляются в ассемблерах и отладчиках.
Если число начинается с цифры A..F, перед ним добавляется 0: 0ABCDEFh.
Во времена 8-битных домашних компьютеров, был также способ записи чисел используя префикс \$, например, \$ABCD.
В книге я попытаюсь придерживаться префикса "0x" для шестнадцатеричных чисел.

Нужно ли учиться конвертировать числа в уме? Таблицу шестнадцатеричных чисел из одной цифры легко запомнить.
А запоминать б\'{о}льшие числа, наверное, не стоит.

Наверное, чаще всего шестнадцатеричные числа можно увидеть в \ac{URL}-ах.
Так кодируются буквы не из числа латинских.
Например:
\url{https://en.wiktionary.org/wiki/na\%C3\%AFvet\%C3\%A9} это \ac{URL} страницы в Wiktionary о слове \q{naïveté}.

\subsubsection{Восьмеричная система}

Еще одна система, которая в прошлом много использовалась в программировании это восьмеричная: есть 8 цифр (0..7) и каждая
описывает 3 бита, так что легко конвертировать числа туда и назад.
Она почти везде была заменена шестнадцатеричной, но удивительно, в *NIX имеется утилита использующаяся многими людьми,
которая принимает на вход восьмеричное число: \TT{chmod}.

\myindex{UNIX!chmod}
Как знают многие пользователи *NIX, аргумент \TT{chmod} это число из трех цифр. Первая цифра это права владельца файла,
вторая это права группы (которой файл принадлежит), третья для всех остальных.
И каждая цифра может быть представлена в двоичном виде:

\begin{center}
\begin{longtable}{ | l | l | l | }
\hline
\HeaderColor десятичная & \HeaderColor двоичная & \HeaderColor значение \\
\hline
7	&111	&\textbf{rwx} \\
6	&110	&\textbf{rw-} \\
5	&101	&\textbf{r-x} \\
4	&100	&\textbf{r-{}-} \\
3	&011	&\textbf{-wx} \\
2	&010	&\textbf{-w-} \\
1	&001	&\textbf{-{}-x} \\
0	&000	&\textbf{-{}-{}-} \\
\hline
\end{longtable}
\end{center}

Так что каждый бит привязан к флагу: read/write/execute (чтение/запись/исполнение).

И вот почему я вспомнил здесь о \TT{chmod}, это потому что всё число может быть представлено как число в восьмеричной системе.
Для примера возьмем 644.
Когда вы запускаете \TT{chmod 644 file}, вы выставляете права read/write для владельца, права read для группы, и снова,
read для всех остальных.
Сконвертируем число 644 из восьмеричной системы в двоичную, это будет \TT{110100100}, или (в группах по 3 бита) \TT{110 100 100}.

Теперь мы видим, что каждая тройка описывает права для владельца/группы/остальных:
первая это \TT{rw-}, вторая это \TT{r--} и третья это \TT{r--}.

Восьмеричная система была также популярная на старых компьютерах вроде PDP-8, потому что слово там могло содержать 12, 24 или
36 бит, и эти числа делятся на 3, так что выбор восьмеричной системы в той среде был логичен.
Сейчас, все популярные компьютеры имеют размер слова/адреса 16, 32 или 64 бита, и эти числа делятся на 4,
так что шестнадцатеричная система здесь удобнее.

Восьмеричная система поддерживается всеми стандартными компиляторами \CCpp{}.
Это иногда источник недоумения, потому что восьмеричные числа кодируются с нулем вперед, например, 0377 это 255.
И иногда, вы можете сделать опечатку, и написать "09" вместо 9, и компилятор выдаст ошибку.
GCC может выдать что-то вроде:\\
\TT{error: invalid digit "9" in octal constant}.

Также, восьмеричная система популярна в Java: когда IDA показывает строку с непечатаемыми символами,
они кодируются в восьмеричной системе вместо шестнадцатеричной.
\myindex{JAD}
Точно также себя ведет декомпилятор с Java JAD.

\subsubsection{Делимость}

Когда вы видите десятичное число вроде 120, вы можете быстро понять что оно делится на 10, потому что последняя цифра это 0.
Точно также, 123400 делится на 100, потому что две последних цифры это нули.

Точно также, шестнадцатеричное число 0x1230 делится на 0x10 (или 16), 0x123000 делится на 0x1000 (или 4096), итд.

Двоичное число 0b1000101000 делится на 0b1000 (8), итд.

Это свойство можно часто использовать, чтобы быстро понять,
что длина какого-либо блока в памяти выровнена по некоторой границе.
Например, секции в \ac{PE}-файлах почти всегда начинаются с адресов заканчивающихся 3 шестнадцатеричными нулями:
0x41000, 0x10001000, итд.
Причина в том, что почти все секции в \ac{PE} выровнены по границе 0x1000 (4096) байт.

\subsubsection{Арифметика произвольной точности и основание}

\index{RSA}
Арифметика произвольной точности (multi-precision arithmetic) может использовать огромные числа,
которые могут храниться в нескольких байтах.
Например, ключи RSA, и открытые и закрытые, могут занимать до 4096 бит и даже больше.

В \InSqBrackets{\TAOCPvolII, 265} можно найти такую идею: когда вы сохраняете число произвольной точности в нескольких байтах,
всё число может быть представлено как имеющую систему счисления по основанию $2^8=256$, и каждая цифра находится
в соответствующем байте.
Точно также, если вы сохраняете число произвольной точности в нескольких 32-битных целочисленных значениях,
каждая цифра отправляется в каждый 32-битный слот, и вы можете считать что это число записано в системе с основанием $2^{32}$.

\subsubsection{Произношение}

Числа в недесятичных системах счислениях обычно произносятся по одной цифре: ``один-ноль-ноль-один-один-...''.
Слова вроде ``десять'', ``тысяча'', итд, обычно не произносятся, потому что тогда можно спутать с десятичной системой.

\subsubsection{Числа с плавающей запятой}

Чтобы отличать числа с плавающей запятой от целочисленных, часто, в конце добавляют ``.0'',
например $0.0$, $123.0$, итд.

}
\ITA{\input{patterns/numeral_ITA}}
\DE{\input{patterns/numeral_DE}}
\FR{\input{patterns/numeral_FR}}
\PL{\input{patterns/numeral_PL}}

% chapters
\ifdefined\SPANISH
\chapter{Patrones de código}
\fi % SPANISH

\ifdefined\GERMAN
\chapter{Code-Muster}
\fi % GERMAN

\ifdefined\ENGLISH
\chapter{Code Patterns}
\fi % ENGLISH

\ifdefined\ITALIAN
\chapter{Forme di codice}
\fi % ITALIAN

\ifdefined\RUSSIAN
\chapter{Образцы кода}
\fi % RUSSIAN

\ifdefined\BRAZILIAN
\chapter{Padrões de códigos}
\fi % BRAZILIAN

\ifdefined\THAI
\chapter{รูปแบบของโค้ด}
\fi % THAI

\ifdefined\FRENCH
\chapter{Modèle de code}
\fi % FRENCH

\ifdefined\POLISH
\chapter{\PLph{}}
\fi % POLISH

% sections
\EN{\input{patterns/patterns_opt_dbg_EN}}
\ES{\input{patterns/patterns_opt_dbg_ES}}
\ITA{\input{patterns/patterns_opt_dbg_ITA}}
\PTBR{\input{patterns/patterns_opt_dbg_PTBR}}
\RU{\input{patterns/patterns_opt_dbg_RU}}
\THA{\input{patterns/patterns_opt_dbg_THA}}
\DE{\input{patterns/patterns_opt_dbg_DE}}
\FR{\input{patterns/patterns_opt_dbg_FR}}
\PL{\input{patterns/patterns_opt_dbg_PL}}

\RU{\section{Некоторые базовые понятия}}
\EN{\section{Some basics}}
\DE{\section{Einige Grundlagen}}
\FR{\section{Quelques bases}}
\ES{\section{\ESph{}}}
\ITA{\section{Alcune basi teoriche}}
\PTBR{\section{\PTBRph{}}}
\THA{\section{\THAph{}}}
\PL{\section{\PLph{}}}

% sections:
\EN{\input{patterns/intro_CPU_ISA_EN}}
\ES{\input{patterns/intro_CPU_ISA_ES}}
\ITA{\input{patterns/intro_CPU_ISA_ITA}}
\PTBR{\input{patterns/intro_CPU_ISA_PTBR}}
\RU{\input{patterns/intro_CPU_ISA_RU}}
\DE{\input{patterns/intro_CPU_ISA_DE}}
\FR{\input{patterns/intro_CPU_ISA_FR}}
\PL{\input{patterns/intro_CPU_ISA_PL}}

\EN{\input{patterns/numeral_EN}}
\RU{\input{patterns/numeral_RU}}
\ITA{\input{patterns/numeral_ITA}}
\DE{\input{patterns/numeral_DE}}
\FR{\input{patterns/numeral_FR}}
\PL{\input{patterns/numeral_PL}}

% chapters
\input{patterns/00_empty/main}
\input{patterns/011_ret/main}
\input{patterns/01_helloworld/main}
\input{patterns/015_prolog_epilogue/main}
\input{patterns/02_stack/main}
\input{patterns/03_printf/main}
\input{patterns/04_scanf/main}
\input{patterns/05_passing_arguments/main}
\input{patterns/06_return_results/main}
\input{patterns/061_pointers/main}
\input{patterns/065_GOTO/main}
\input{patterns/07_jcc/main}
\input{patterns/08_switch/main}
\input{patterns/09_loops/main}
\input{patterns/10_strings/main}
\input{patterns/11_arith_optimizations/main}
\input{patterns/12_FPU/main}
\input{patterns/13_arrays/main}
\input{patterns/14_bitfields/main}
\EN{\input{patterns/145_LCG/main_EN}}
\RU{\input{patterns/145_LCG/main_RU}}
\input{patterns/15_structs/main}
\input{patterns/17_unions/main}
\input{patterns/18_pointers_to_functions/main}
\input{patterns/185_64bit_in_32_env/main}

\EN{\input{patterns/19_SIMD/main_EN}}
\RU{\input{patterns/19_SIMD/main_RU}}
\DE{\input{patterns/19_SIMD/main_DE}}

\EN{\input{patterns/20_x64/main_EN}}
\RU{\input{patterns/20_x64/main_RU}}

\EN{\input{patterns/205_floating_SIMD/main_EN}}
\RU{\input{patterns/205_floating_SIMD/main_RU}}
\DE{\input{patterns/205_floating_SIMD/main_DE}}

\EN{\input{patterns/ARM/main_EN}}
\RU{\input{patterns/ARM/main_RU}}
\DE{\input{patterns/ARM/main_DE}}

\input{patterns/MIPS/main}

\ifdefined\SPANISH
\chapter{Patrones de código}
\fi % SPANISH

\ifdefined\GERMAN
\chapter{Code-Muster}
\fi % GERMAN

\ifdefined\ENGLISH
\chapter{Code Patterns}
\fi % ENGLISH

\ifdefined\ITALIAN
\chapter{Forme di codice}
\fi % ITALIAN

\ifdefined\RUSSIAN
\chapter{Образцы кода}
\fi % RUSSIAN

\ifdefined\BRAZILIAN
\chapter{Padrões de códigos}
\fi % BRAZILIAN

\ifdefined\THAI
\chapter{รูปแบบของโค้ด}
\fi % THAI

\ifdefined\FRENCH
\chapter{Modèle de code}
\fi % FRENCH

\ifdefined\POLISH
\chapter{\PLph{}}
\fi % POLISH

% sections
\EN{\input{patterns/patterns_opt_dbg_EN}}
\ES{\input{patterns/patterns_opt_dbg_ES}}
\ITA{\input{patterns/patterns_opt_dbg_ITA}}
\PTBR{\input{patterns/patterns_opt_dbg_PTBR}}
\RU{\input{patterns/patterns_opt_dbg_RU}}
\THA{\input{patterns/patterns_opt_dbg_THA}}
\DE{\input{patterns/patterns_opt_dbg_DE}}
\FR{\input{patterns/patterns_opt_dbg_FR}}
\PL{\input{patterns/patterns_opt_dbg_PL}}

\RU{\section{Некоторые базовые понятия}}
\EN{\section{Some basics}}
\DE{\section{Einige Grundlagen}}
\FR{\section{Quelques bases}}
\ES{\section{\ESph{}}}
\ITA{\section{Alcune basi teoriche}}
\PTBR{\section{\PTBRph{}}}
\THA{\section{\THAph{}}}
\PL{\section{\PLph{}}}

% sections:
\EN{\input{patterns/intro_CPU_ISA_EN}}
\ES{\input{patterns/intro_CPU_ISA_ES}}
\ITA{\input{patterns/intro_CPU_ISA_ITA}}
\PTBR{\input{patterns/intro_CPU_ISA_PTBR}}
\RU{\input{patterns/intro_CPU_ISA_RU}}
\DE{\input{patterns/intro_CPU_ISA_DE}}
\FR{\input{patterns/intro_CPU_ISA_FR}}
\PL{\input{patterns/intro_CPU_ISA_PL}}

\EN{\input{patterns/numeral_EN}}
\RU{\input{patterns/numeral_RU}}
\ITA{\input{patterns/numeral_ITA}}
\DE{\input{patterns/numeral_DE}}
\FR{\input{patterns/numeral_FR}}
\PL{\input{patterns/numeral_PL}}

% chapters
\input{patterns/00_empty/main}
\input{patterns/011_ret/main}
\input{patterns/01_helloworld/main}
\input{patterns/015_prolog_epilogue/main}
\input{patterns/02_stack/main}
\input{patterns/03_printf/main}
\input{patterns/04_scanf/main}
\input{patterns/05_passing_arguments/main}
\input{patterns/06_return_results/main}
\input{patterns/061_pointers/main}
\input{patterns/065_GOTO/main}
\input{patterns/07_jcc/main}
\input{patterns/08_switch/main}
\input{patterns/09_loops/main}
\input{patterns/10_strings/main}
\input{patterns/11_arith_optimizations/main}
\input{patterns/12_FPU/main}
\input{patterns/13_arrays/main}
\input{patterns/14_bitfields/main}
\EN{\input{patterns/145_LCG/main_EN}}
\RU{\input{patterns/145_LCG/main_RU}}
\input{patterns/15_structs/main}
\input{patterns/17_unions/main}
\input{patterns/18_pointers_to_functions/main}
\input{patterns/185_64bit_in_32_env/main}

\EN{\input{patterns/19_SIMD/main_EN}}
\RU{\input{patterns/19_SIMD/main_RU}}
\DE{\input{patterns/19_SIMD/main_DE}}

\EN{\input{patterns/20_x64/main_EN}}
\RU{\input{patterns/20_x64/main_RU}}

\EN{\input{patterns/205_floating_SIMD/main_EN}}
\RU{\input{patterns/205_floating_SIMD/main_RU}}
\DE{\input{patterns/205_floating_SIMD/main_DE}}

\EN{\input{patterns/ARM/main_EN}}
\RU{\input{patterns/ARM/main_RU}}
\DE{\input{patterns/ARM/main_DE}}

\input{patterns/MIPS/main}

\ifdefined\SPANISH
\chapter{Patrones de código}
\fi % SPANISH

\ifdefined\GERMAN
\chapter{Code-Muster}
\fi % GERMAN

\ifdefined\ENGLISH
\chapter{Code Patterns}
\fi % ENGLISH

\ifdefined\ITALIAN
\chapter{Forme di codice}
\fi % ITALIAN

\ifdefined\RUSSIAN
\chapter{Образцы кода}
\fi % RUSSIAN

\ifdefined\BRAZILIAN
\chapter{Padrões de códigos}
\fi % BRAZILIAN

\ifdefined\THAI
\chapter{รูปแบบของโค้ด}
\fi % THAI

\ifdefined\FRENCH
\chapter{Modèle de code}
\fi % FRENCH

\ifdefined\POLISH
\chapter{\PLph{}}
\fi % POLISH

% sections
\EN{\input{patterns/patterns_opt_dbg_EN}}
\ES{\input{patterns/patterns_opt_dbg_ES}}
\ITA{\input{patterns/patterns_opt_dbg_ITA}}
\PTBR{\input{patterns/patterns_opt_dbg_PTBR}}
\RU{\input{patterns/patterns_opt_dbg_RU}}
\THA{\input{patterns/patterns_opt_dbg_THA}}
\DE{\input{patterns/patterns_opt_dbg_DE}}
\FR{\input{patterns/patterns_opt_dbg_FR}}
\PL{\input{patterns/patterns_opt_dbg_PL}}

\RU{\section{Некоторые базовые понятия}}
\EN{\section{Some basics}}
\DE{\section{Einige Grundlagen}}
\FR{\section{Quelques bases}}
\ES{\section{\ESph{}}}
\ITA{\section{Alcune basi teoriche}}
\PTBR{\section{\PTBRph{}}}
\THA{\section{\THAph{}}}
\PL{\section{\PLph{}}}

% sections:
\EN{\input{patterns/intro_CPU_ISA_EN}}
\ES{\input{patterns/intro_CPU_ISA_ES}}
\ITA{\input{patterns/intro_CPU_ISA_ITA}}
\PTBR{\input{patterns/intro_CPU_ISA_PTBR}}
\RU{\input{patterns/intro_CPU_ISA_RU}}
\DE{\input{patterns/intro_CPU_ISA_DE}}
\FR{\input{patterns/intro_CPU_ISA_FR}}
\PL{\input{patterns/intro_CPU_ISA_PL}}

\EN{\input{patterns/numeral_EN}}
\RU{\input{patterns/numeral_RU}}
\ITA{\input{patterns/numeral_ITA}}
\DE{\input{patterns/numeral_DE}}
\FR{\input{patterns/numeral_FR}}
\PL{\input{patterns/numeral_PL}}

% chapters
\input{patterns/00_empty/main}
\input{patterns/011_ret/main}
\input{patterns/01_helloworld/main}
\input{patterns/015_prolog_epilogue/main}
\input{patterns/02_stack/main}
\input{patterns/03_printf/main}
\input{patterns/04_scanf/main}
\input{patterns/05_passing_arguments/main}
\input{patterns/06_return_results/main}
\input{patterns/061_pointers/main}
\input{patterns/065_GOTO/main}
\input{patterns/07_jcc/main}
\input{patterns/08_switch/main}
\input{patterns/09_loops/main}
\input{patterns/10_strings/main}
\input{patterns/11_arith_optimizations/main}
\input{patterns/12_FPU/main}
\input{patterns/13_arrays/main}
\input{patterns/14_bitfields/main}
\EN{\input{patterns/145_LCG/main_EN}}
\RU{\input{patterns/145_LCG/main_RU}}
\input{patterns/15_structs/main}
\input{patterns/17_unions/main}
\input{patterns/18_pointers_to_functions/main}
\input{patterns/185_64bit_in_32_env/main}

\EN{\input{patterns/19_SIMD/main_EN}}
\RU{\input{patterns/19_SIMD/main_RU}}
\DE{\input{patterns/19_SIMD/main_DE}}

\EN{\input{patterns/20_x64/main_EN}}
\RU{\input{patterns/20_x64/main_RU}}

\EN{\input{patterns/205_floating_SIMD/main_EN}}
\RU{\input{patterns/205_floating_SIMD/main_RU}}
\DE{\input{patterns/205_floating_SIMD/main_DE}}

\EN{\input{patterns/ARM/main_EN}}
\RU{\input{patterns/ARM/main_RU}}
\DE{\input{patterns/ARM/main_DE}}

\input{patterns/MIPS/main}

\ifdefined\SPANISH
\chapter{Patrones de código}
\fi % SPANISH

\ifdefined\GERMAN
\chapter{Code-Muster}
\fi % GERMAN

\ifdefined\ENGLISH
\chapter{Code Patterns}
\fi % ENGLISH

\ifdefined\ITALIAN
\chapter{Forme di codice}
\fi % ITALIAN

\ifdefined\RUSSIAN
\chapter{Образцы кода}
\fi % RUSSIAN

\ifdefined\BRAZILIAN
\chapter{Padrões de códigos}
\fi % BRAZILIAN

\ifdefined\THAI
\chapter{รูปแบบของโค้ด}
\fi % THAI

\ifdefined\FRENCH
\chapter{Modèle de code}
\fi % FRENCH

\ifdefined\POLISH
\chapter{\PLph{}}
\fi % POLISH

% sections
\EN{\input{patterns/patterns_opt_dbg_EN}}
\ES{\input{patterns/patterns_opt_dbg_ES}}
\ITA{\input{patterns/patterns_opt_dbg_ITA}}
\PTBR{\input{patterns/patterns_opt_dbg_PTBR}}
\RU{\input{patterns/patterns_opt_dbg_RU}}
\THA{\input{patterns/patterns_opt_dbg_THA}}
\DE{\input{patterns/patterns_opt_dbg_DE}}
\FR{\input{patterns/patterns_opt_dbg_FR}}
\PL{\input{patterns/patterns_opt_dbg_PL}}

\RU{\section{Некоторые базовые понятия}}
\EN{\section{Some basics}}
\DE{\section{Einige Grundlagen}}
\FR{\section{Quelques bases}}
\ES{\section{\ESph{}}}
\ITA{\section{Alcune basi teoriche}}
\PTBR{\section{\PTBRph{}}}
\THA{\section{\THAph{}}}
\PL{\section{\PLph{}}}

% sections:
\EN{\input{patterns/intro_CPU_ISA_EN}}
\ES{\input{patterns/intro_CPU_ISA_ES}}
\ITA{\input{patterns/intro_CPU_ISA_ITA}}
\PTBR{\input{patterns/intro_CPU_ISA_PTBR}}
\RU{\input{patterns/intro_CPU_ISA_RU}}
\DE{\input{patterns/intro_CPU_ISA_DE}}
\FR{\input{patterns/intro_CPU_ISA_FR}}
\PL{\input{patterns/intro_CPU_ISA_PL}}

\EN{\input{patterns/numeral_EN}}
\RU{\input{patterns/numeral_RU}}
\ITA{\input{patterns/numeral_ITA}}
\DE{\input{patterns/numeral_DE}}
\FR{\input{patterns/numeral_FR}}
\PL{\input{patterns/numeral_PL}}

% chapters
\input{patterns/00_empty/main}
\input{patterns/011_ret/main}
\input{patterns/01_helloworld/main}
\input{patterns/015_prolog_epilogue/main}
\input{patterns/02_stack/main}
\input{patterns/03_printf/main}
\input{patterns/04_scanf/main}
\input{patterns/05_passing_arguments/main}
\input{patterns/06_return_results/main}
\input{patterns/061_pointers/main}
\input{patterns/065_GOTO/main}
\input{patterns/07_jcc/main}
\input{patterns/08_switch/main}
\input{patterns/09_loops/main}
\input{patterns/10_strings/main}
\input{patterns/11_arith_optimizations/main}
\input{patterns/12_FPU/main}
\input{patterns/13_arrays/main}
\input{patterns/14_bitfields/main}
\EN{\input{patterns/145_LCG/main_EN}}
\RU{\input{patterns/145_LCG/main_RU}}
\input{patterns/15_structs/main}
\input{patterns/17_unions/main}
\input{patterns/18_pointers_to_functions/main}
\input{patterns/185_64bit_in_32_env/main}

\EN{\input{patterns/19_SIMD/main_EN}}
\RU{\input{patterns/19_SIMD/main_RU}}
\DE{\input{patterns/19_SIMD/main_DE}}

\EN{\input{patterns/20_x64/main_EN}}
\RU{\input{patterns/20_x64/main_RU}}

\EN{\input{patterns/205_floating_SIMD/main_EN}}
\RU{\input{patterns/205_floating_SIMD/main_RU}}
\DE{\input{patterns/205_floating_SIMD/main_DE}}

\EN{\input{patterns/ARM/main_EN}}
\RU{\input{patterns/ARM/main_RU}}
\DE{\input{patterns/ARM/main_DE}}

\input{patterns/MIPS/main}

\ifdefined\SPANISH
\chapter{Patrones de código}
\fi % SPANISH

\ifdefined\GERMAN
\chapter{Code-Muster}
\fi % GERMAN

\ifdefined\ENGLISH
\chapter{Code Patterns}
\fi % ENGLISH

\ifdefined\ITALIAN
\chapter{Forme di codice}
\fi % ITALIAN

\ifdefined\RUSSIAN
\chapter{Образцы кода}
\fi % RUSSIAN

\ifdefined\BRAZILIAN
\chapter{Padrões de códigos}
\fi % BRAZILIAN

\ifdefined\THAI
\chapter{รูปแบบของโค้ด}
\fi % THAI

\ifdefined\FRENCH
\chapter{Modèle de code}
\fi % FRENCH

\ifdefined\POLISH
\chapter{\PLph{}}
\fi % POLISH

% sections
\EN{\input{patterns/patterns_opt_dbg_EN}}
\ES{\input{patterns/patterns_opt_dbg_ES}}
\ITA{\input{patterns/patterns_opt_dbg_ITA}}
\PTBR{\input{patterns/patterns_opt_dbg_PTBR}}
\RU{\input{patterns/patterns_opt_dbg_RU}}
\THA{\input{patterns/patterns_opt_dbg_THA}}
\DE{\input{patterns/patterns_opt_dbg_DE}}
\FR{\input{patterns/patterns_opt_dbg_FR}}
\PL{\input{patterns/patterns_opt_dbg_PL}}

\RU{\section{Некоторые базовые понятия}}
\EN{\section{Some basics}}
\DE{\section{Einige Grundlagen}}
\FR{\section{Quelques bases}}
\ES{\section{\ESph{}}}
\ITA{\section{Alcune basi teoriche}}
\PTBR{\section{\PTBRph{}}}
\THA{\section{\THAph{}}}
\PL{\section{\PLph{}}}

% sections:
\EN{\input{patterns/intro_CPU_ISA_EN}}
\ES{\input{patterns/intro_CPU_ISA_ES}}
\ITA{\input{patterns/intro_CPU_ISA_ITA}}
\PTBR{\input{patterns/intro_CPU_ISA_PTBR}}
\RU{\input{patterns/intro_CPU_ISA_RU}}
\DE{\input{patterns/intro_CPU_ISA_DE}}
\FR{\input{patterns/intro_CPU_ISA_FR}}
\PL{\input{patterns/intro_CPU_ISA_PL}}

\EN{\input{patterns/numeral_EN}}
\RU{\input{patterns/numeral_RU}}
\ITA{\input{patterns/numeral_ITA}}
\DE{\input{patterns/numeral_DE}}
\FR{\input{patterns/numeral_FR}}
\PL{\input{patterns/numeral_PL}}

% chapters
\input{patterns/00_empty/main}
\input{patterns/011_ret/main}
\input{patterns/01_helloworld/main}
\input{patterns/015_prolog_epilogue/main}
\input{patterns/02_stack/main}
\input{patterns/03_printf/main}
\input{patterns/04_scanf/main}
\input{patterns/05_passing_arguments/main}
\input{patterns/06_return_results/main}
\input{patterns/061_pointers/main}
\input{patterns/065_GOTO/main}
\input{patterns/07_jcc/main}
\input{patterns/08_switch/main}
\input{patterns/09_loops/main}
\input{patterns/10_strings/main}
\input{patterns/11_arith_optimizations/main}
\input{patterns/12_FPU/main}
\input{patterns/13_arrays/main}
\input{patterns/14_bitfields/main}
\EN{\input{patterns/145_LCG/main_EN}}
\RU{\input{patterns/145_LCG/main_RU}}
\input{patterns/15_structs/main}
\input{patterns/17_unions/main}
\input{patterns/18_pointers_to_functions/main}
\input{patterns/185_64bit_in_32_env/main}

\EN{\input{patterns/19_SIMD/main_EN}}
\RU{\input{patterns/19_SIMD/main_RU}}
\DE{\input{patterns/19_SIMD/main_DE}}

\EN{\input{patterns/20_x64/main_EN}}
\RU{\input{patterns/20_x64/main_RU}}

\EN{\input{patterns/205_floating_SIMD/main_EN}}
\RU{\input{patterns/205_floating_SIMD/main_RU}}
\DE{\input{patterns/205_floating_SIMD/main_DE}}

\EN{\input{patterns/ARM/main_EN}}
\RU{\input{patterns/ARM/main_RU}}
\DE{\input{patterns/ARM/main_DE}}

\input{patterns/MIPS/main}

\ifdefined\SPANISH
\chapter{Patrones de código}
\fi % SPANISH

\ifdefined\GERMAN
\chapter{Code-Muster}
\fi % GERMAN

\ifdefined\ENGLISH
\chapter{Code Patterns}
\fi % ENGLISH

\ifdefined\ITALIAN
\chapter{Forme di codice}
\fi % ITALIAN

\ifdefined\RUSSIAN
\chapter{Образцы кода}
\fi % RUSSIAN

\ifdefined\BRAZILIAN
\chapter{Padrões de códigos}
\fi % BRAZILIAN

\ifdefined\THAI
\chapter{รูปแบบของโค้ด}
\fi % THAI

\ifdefined\FRENCH
\chapter{Modèle de code}
\fi % FRENCH

\ifdefined\POLISH
\chapter{\PLph{}}
\fi % POLISH

% sections
\EN{\input{patterns/patterns_opt_dbg_EN}}
\ES{\input{patterns/patterns_opt_dbg_ES}}
\ITA{\input{patterns/patterns_opt_dbg_ITA}}
\PTBR{\input{patterns/patterns_opt_dbg_PTBR}}
\RU{\input{patterns/patterns_opt_dbg_RU}}
\THA{\input{patterns/patterns_opt_dbg_THA}}
\DE{\input{patterns/patterns_opt_dbg_DE}}
\FR{\input{patterns/patterns_opt_dbg_FR}}
\PL{\input{patterns/patterns_opt_dbg_PL}}

\RU{\section{Некоторые базовые понятия}}
\EN{\section{Some basics}}
\DE{\section{Einige Grundlagen}}
\FR{\section{Quelques bases}}
\ES{\section{\ESph{}}}
\ITA{\section{Alcune basi teoriche}}
\PTBR{\section{\PTBRph{}}}
\THA{\section{\THAph{}}}
\PL{\section{\PLph{}}}

% sections:
\EN{\input{patterns/intro_CPU_ISA_EN}}
\ES{\input{patterns/intro_CPU_ISA_ES}}
\ITA{\input{patterns/intro_CPU_ISA_ITA}}
\PTBR{\input{patterns/intro_CPU_ISA_PTBR}}
\RU{\input{patterns/intro_CPU_ISA_RU}}
\DE{\input{patterns/intro_CPU_ISA_DE}}
\FR{\input{patterns/intro_CPU_ISA_FR}}
\PL{\input{patterns/intro_CPU_ISA_PL}}

\EN{\input{patterns/numeral_EN}}
\RU{\input{patterns/numeral_RU}}
\ITA{\input{patterns/numeral_ITA}}
\DE{\input{patterns/numeral_DE}}
\FR{\input{patterns/numeral_FR}}
\PL{\input{patterns/numeral_PL}}

% chapters
\input{patterns/00_empty/main}
\input{patterns/011_ret/main}
\input{patterns/01_helloworld/main}
\input{patterns/015_prolog_epilogue/main}
\input{patterns/02_stack/main}
\input{patterns/03_printf/main}
\input{patterns/04_scanf/main}
\input{patterns/05_passing_arguments/main}
\input{patterns/06_return_results/main}
\input{patterns/061_pointers/main}
\input{patterns/065_GOTO/main}
\input{patterns/07_jcc/main}
\input{patterns/08_switch/main}
\input{patterns/09_loops/main}
\input{patterns/10_strings/main}
\input{patterns/11_arith_optimizations/main}
\input{patterns/12_FPU/main}
\input{patterns/13_arrays/main}
\input{patterns/14_bitfields/main}
\EN{\input{patterns/145_LCG/main_EN}}
\RU{\input{patterns/145_LCG/main_RU}}
\input{patterns/15_structs/main}
\input{patterns/17_unions/main}
\input{patterns/18_pointers_to_functions/main}
\input{patterns/185_64bit_in_32_env/main}

\EN{\input{patterns/19_SIMD/main_EN}}
\RU{\input{patterns/19_SIMD/main_RU}}
\DE{\input{patterns/19_SIMD/main_DE}}

\EN{\input{patterns/20_x64/main_EN}}
\RU{\input{patterns/20_x64/main_RU}}

\EN{\input{patterns/205_floating_SIMD/main_EN}}
\RU{\input{patterns/205_floating_SIMD/main_RU}}
\DE{\input{patterns/205_floating_SIMD/main_DE}}

\EN{\input{patterns/ARM/main_EN}}
\RU{\input{patterns/ARM/main_RU}}
\DE{\input{patterns/ARM/main_DE}}

\input{patterns/MIPS/main}

\ifdefined\SPANISH
\chapter{Patrones de código}
\fi % SPANISH

\ifdefined\GERMAN
\chapter{Code-Muster}
\fi % GERMAN

\ifdefined\ENGLISH
\chapter{Code Patterns}
\fi % ENGLISH

\ifdefined\ITALIAN
\chapter{Forme di codice}
\fi % ITALIAN

\ifdefined\RUSSIAN
\chapter{Образцы кода}
\fi % RUSSIAN

\ifdefined\BRAZILIAN
\chapter{Padrões de códigos}
\fi % BRAZILIAN

\ifdefined\THAI
\chapter{รูปแบบของโค้ด}
\fi % THAI

\ifdefined\FRENCH
\chapter{Modèle de code}
\fi % FRENCH

\ifdefined\POLISH
\chapter{\PLph{}}
\fi % POLISH

% sections
\EN{\input{patterns/patterns_opt_dbg_EN}}
\ES{\input{patterns/patterns_opt_dbg_ES}}
\ITA{\input{patterns/patterns_opt_dbg_ITA}}
\PTBR{\input{patterns/patterns_opt_dbg_PTBR}}
\RU{\input{patterns/patterns_opt_dbg_RU}}
\THA{\input{patterns/patterns_opt_dbg_THA}}
\DE{\input{patterns/patterns_opt_dbg_DE}}
\FR{\input{patterns/patterns_opt_dbg_FR}}
\PL{\input{patterns/patterns_opt_dbg_PL}}

\RU{\section{Некоторые базовые понятия}}
\EN{\section{Some basics}}
\DE{\section{Einige Grundlagen}}
\FR{\section{Quelques bases}}
\ES{\section{\ESph{}}}
\ITA{\section{Alcune basi teoriche}}
\PTBR{\section{\PTBRph{}}}
\THA{\section{\THAph{}}}
\PL{\section{\PLph{}}}

% sections:
\EN{\input{patterns/intro_CPU_ISA_EN}}
\ES{\input{patterns/intro_CPU_ISA_ES}}
\ITA{\input{patterns/intro_CPU_ISA_ITA}}
\PTBR{\input{patterns/intro_CPU_ISA_PTBR}}
\RU{\input{patterns/intro_CPU_ISA_RU}}
\DE{\input{patterns/intro_CPU_ISA_DE}}
\FR{\input{patterns/intro_CPU_ISA_FR}}
\PL{\input{patterns/intro_CPU_ISA_PL}}

\EN{\input{patterns/numeral_EN}}
\RU{\input{patterns/numeral_RU}}
\ITA{\input{patterns/numeral_ITA}}
\DE{\input{patterns/numeral_DE}}
\FR{\input{patterns/numeral_FR}}
\PL{\input{patterns/numeral_PL}}

% chapters
\input{patterns/00_empty/main}
\input{patterns/011_ret/main}
\input{patterns/01_helloworld/main}
\input{patterns/015_prolog_epilogue/main}
\input{patterns/02_stack/main}
\input{patterns/03_printf/main}
\input{patterns/04_scanf/main}
\input{patterns/05_passing_arguments/main}
\input{patterns/06_return_results/main}
\input{patterns/061_pointers/main}
\input{patterns/065_GOTO/main}
\input{patterns/07_jcc/main}
\input{patterns/08_switch/main}
\input{patterns/09_loops/main}
\input{patterns/10_strings/main}
\input{patterns/11_arith_optimizations/main}
\input{patterns/12_FPU/main}
\input{patterns/13_arrays/main}
\input{patterns/14_bitfields/main}
\EN{\input{patterns/145_LCG/main_EN}}
\RU{\input{patterns/145_LCG/main_RU}}
\input{patterns/15_structs/main}
\input{patterns/17_unions/main}
\input{patterns/18_pointers_to_functions/main}
\input{patterns/185_64bit_in_32_env/main}

\EN{\input{patterns/19_SIMD/main_EN}}
\RU{\input{patterns/19_SIMD/main_RU}}
\DE{\input{patterns/19_SIMD/main_DE}}

\EN{\input{patterns/20_x64/main_EN}}
\RU{\input{patterns/20_x64/main_RU}}

\EN{\input{patterns/205_floating_SIMD/main_EN}}
\RU{\input{patterns/205_floating_SIMD/main_RU}}
\DE{\input{patterns/205_floating_SIMD/main_DE}}

\EN{\input{patterns/ARM/main_EN}}
\RU{\input{patterns/ARM/main_RU}}
\DE{\input{patterns/ARM/main_DE}}

\input{patterns/MIPS/main}

\ifdefined\SPANISH
\chapter{Patrones de código}
\fi % SPANISH

\ifdefined\GERMAN
\chapter{Code-Muster}
\fi % GERMAN

\ifdefined\ENGLISH
\chapter{Code Patterns}
\fi % ENGLISH

\ifdefined\ITALIAN
\chapter{Forme di codice}
\fi % ITALIAN

\ifdefined\RUSSIAN
\chapter{Образцы кода}
\fi % RUSSIAN

\ifdefined\BRAZILIAN
\chapter{Padrões de códigos}
\fi % BRAZILIAN

\ifdefined\THAI
\chapter{รูปแบบของโค้ด}
\fi % THAI

\ifdefined\FRENCH
\chapter{Modèle de code}
\fi % FRENCH

\ifdefined\POLISH
\chapter{\PLph{}}
\fi % POLISH

% sections
\EN{\input{patterns/patterns_opt_dbg_EN}}
\ES{\input{patterns/patterns_opt_dbg_ES}}
\ITA{\input{patterns/patterns_opt_dbg_ITA}}
\PTBR{\input{patterns/patterns_opt_dbg_PTBR}}
\RU{\input{patterns/patterns_opt_dbg_RU}}
\THA{\input{patterns/patterns_opt_dbg_THA}}
\DE{\input{patterns/patterns_opt_dbg_DE}}
\FR{\input{patterns/patterns_opt_dbg_FR}}
\PL{\input{patterns/patterns_opt_dbg_PL}}

\RU{\section{Некоторые базовые понятия}}
\EN{\section{Some basics}}
\DE{\section{Einige Grundlagen}}
\FR{\section{Quelques bases}}
\ES{\section{\ESph{}}}
\ITA{\section{Alcune basi teoriche}}
\PTBR{\section{\PTBRph{}}}
\THA{\section{\THAph{}}}
\PL{\section{\PLph{}}}

% sections:
\EN{\input{patterns/intro_CPU_ISA_EN}}
\ES{\input{patterns/intro_CPU_ISA_ES}}
\ITA{\input{patterns/intro_CPU_ISA_ITA}}
\PTBR{\input{patterns/intro_CPU_ISA_PTBR}}
\RU{\input{patterns/intro_CPU_ISA_RU}}
\DE{\input{patterns/intro_CPU_ISA_DE}}
\FR{\input{patterns/intro_CPU_ISA_FR}}
\PL{\input{patterns/intro_CPU_ISA_PL}}

\EN{\input{patterns/numeral_EN}}
\RU{\input{patterns/numeral_RU}}
\ITA{\input{patterns/numeral_ITA}}
\DE{\input{patterns/numeral_DE}}
\FR{\input{patterns/numeral_FR}}
\PL{\input{patterns/numeral_PL}}

% chapters
\input{patterns/00_empty/main}
\input{patterns/011_ret/main}
\input{patterns/01_helloworld/main}
\input{patterns/015_prolog_epilogue/main}
\input{patterns/02_stack/main}
\input{patterns/03_printf/main}
\input{patterns/04_scanf/main}
\input{patterns/05_passing_arguments/main}
\input{patterns/06_return_results/main}
\input{patterns/061_pointers/main}
\input{patterns/065_GOTO/main}
\input{patterns/07_jcc/main}
\input{patterns/08_switch/main}
\input{patterns/09_loops/main}
\input{patterns/10_strings/main}
\input{patterns/11_arith_optimizations/main}
\input{patterns/12_FPU/main}
\input{patterns/13_arrays/main}
\input{patterns/14_bitfields/main}
\EN{\input{patterns/145_LCG/main_EN}}
\RU{\input{patterns/145_LCG/main_RU}}
\input{patterns/15_structs/main}
\input{patterns/17_unions/main}
\input{patterns/18_pointers_to_functions/main}
\input{patterns/185_64bit_in_32_env/main}

\EN{\input{patterns/19_SIMD/main_EN}}
\RU{\input{patterns/19_SIMD/main_RU}}
\DE{\input{patterns/19_SIMD/main_DE}}

\EN{\input{patterns/20_x64/main_EN}}
\RU{\input{patterns/20_x64/main_RU}}

\EN{\input{patterns/205_floating_SIMD/main_EN}}
\RU{\input{patterns/205_floating_SIMD/main_RU}}
\DE{\input{patterns/205_floating_SIMD/main_DE}}

\EN{\input{patterns/ARM/main_EN}}
\RU{\input{patterns/ARM/main_RU}}
\DE{\input{patterns/ARM/main_DE}}

\input{patterns/MIPS/main}

\ifdefined\SPANISH
\chapter{Patrones de código}
\fi % SPANISH

\ifdefined\GERMAN
\chapter{Code-Muster}
\fi % GERMAN

\ifdefined\ENGLISH
\chapter{Code Patterns}
\fi % ENGLISH

\ifdefined\ITALIAN
\chapter{Forme di codice}
\fi % ITALIAN

\ifdefined\RUSSIAN
\chapter{Образцы кода}
\fi % RUSSIAN

\ifdefined\BRAZILIAN
\chapter{Padrões de códigos}
\fi % BRAZILIAN

\ifdefined\THAI
\chapter{รูปแบบของโค้ด}
\fi % THAI

\ifdefined\FRENCH
\chapter{Modèle de code}
\fi % FRENCH

\ifdefined\POLISH
\chapter{\PLph{}}
\fi % POLISH

% sections
\EN{\input{patterns/patterns_opt_dbg_EN}}
\ES{\input{patterns/patterns_opt_dbg_ES}}
\ITA{\input{patterns/patterns_opt_dbg_ITA}}
\PTBR{\input{patterns/patterns_opt_dbg_PTBR}}
\RU{\input{patterns/patterns_opt_dbg_RU}}
\THA{\input{patterns/patterns_opt_dbg_THA}}
\DE{\input{patterns/patterns_opt_dbg_DE}}
\FR{\input{patterns/patterns_opt_dbg_FR}}
\PL{\input{patterns/patterns_opt_dbg_PL}}

\RU{\section{Некоторые базовые понятия}}
\EN{\section{Some basics}}
\DE{\section{Einige Grundlagen}}
\FR{\section{Quelques bases}}
\ES{\section{\ESph{}}}
\ITA{\section{Alcune basi teoriche}}
\PTBR{\section{\PTBRph{}}}
\THA{\section{\THAph{}}}
\PL{\section{\PLph{}}}

% sections:
\EN{\input{patterns/intro_CPU_ISA_EN}}
\ES{\input{patterns/intro_CPU_ISA_ES}}
\ITA{\input{patterns/intro_CPU_ISA_ITA}}
\PTBR{\input{patterns/intro_CPU_ISA_PTBR}}
\RU{\input{patterns/intro_CPU_ISA_RU}}
\DE{\input{patterns/intro_CPU_ISA_DE}}
\FR{\input{patterns/intro_CPU_ISA_FR}}
\PL{\input{patterns/intro_CPU_ISA_PL}}

\EN{\input{patterns/numeral_EN}}
\RU{\input{patterns/numeral_RU}}
\ITA{\input{patterns/numeral_ITA}}
\DE{\input{patterns/numeral_DE}}
\FR{\input{patterns/numeral_FR}}
\PL{\input{patterns/numeral_PL}}

% chapters
\input{patterns/00_empty/main}
\input{patterns/011_ret/main}
\input{patterns/01_helloworld/main}
\input{patterns/015_prolog_epilogue/main}
\input{patterns/02_stack/main}
\input{patterns/03_printf/main}
\input{patterns/04_scanf/main}
\input{patterns/05_passing_arguments/main}
\input{patterns/06_return_results/main}
\input{patterns/061_pointers/main}
\input{patterns/065_GOTO/main}
\input{patterns/07_jcc/main}
\input{patterns/08_switch/main}
\input{patterns/09_loops/main}
\input{patterns/10_strings/main}
\input{patterns/11_arith_optimizations/main}
\input{patterns/12_FPU/main}
\input{patterns/13_arrays/main}
\input{patterns/14_bitfields/main}
\EN{\input{patterns/145_LCG/main_EN}}
\RU{\input{patterns/145_LCG/main_RU}}
\input{patterns/15_structs/main}
\input{patterns/17_unions/main}
\input{patterns/18_pointers_to_functions/main}
\input{patterns/185_64bit_in_32_env/main}

\EN{\input{patterns/19_SIMD/main_EN}}
\RU{\input{patterns/19_SIMD/main_RU}}
\DE{\input{patterns/19_SIMD/main_DE}}

\EN{\input{patterns/20_x64/main_EN}}
\RU{\input{patterns/20_x64/main_RU}}

\EN{\input{patterns/205_floating_SIMD/main_EN}}
\RU{\input{patterns/205_floating_SIMD/main_RU}}
\DE{\input{patterns/205_floating_SIMD/main_DE}}

\EN{\input{patterns/ARM/main_EN}}
\RU{\input{patterns/ARM/main_RU}}
\DE{\input{patterns/ARM/main_DE}}

\input{patterns/MIPS/main}

\ifdefined\SPANISH
\chapter{Patrones de código}
\fi % SPANISH

\ifdefined\GERMAN
\chapter{Code-Muster}
\fi % GERMAN

\ifdefined\ENGLISH
\chapter{Code Patterns}
\fi % ENGLISH

\ifdefined\ITALIAN
\chapter{Forme di codice}
\fi % ITALIAN

\ifdefined\RUSSIAN
\chapter{Образцы кода}
\fi % RUSSIAN

\ifdefined\BRAZILIAN
\chapter{Padrões de códigos}
\fi % BRAZILIAN

\ifdefined\THAI
\chapter{รูปแบบของโค้ด}
\fi % THAI

\ifdefined\FRENCH
\chapter{Modèle de code}
\fi % FRENCH

\ifdefined\POLISH
\chapter{\PLph{}}
\fi % POLISH

% sections
\EN{\input{patterns/patterns_opt_dbg_EN}}
\ES{\input{patterns/patterns_opt_dbg_ES}}
\ITA{\input{patterns/patterns_opt_dbg_ITA}}
\PTBR{\input{patterns/patterns_opt_dbg_PTBR}}
\RU{\input{patterns/patterns_opt_dbg_RU}}
\THA{\input{patterns/patterns_opt_dbg_THA}}
\DE{\input{patterns/patterns_opt_dbg_DE}}
\FR{\input{patterns/patterns_opt_dbg_FR}}
\PL{\input{patterns/patterns_opt_dbg_PL}}

\RU{\section{Некоторые базовые понятия}}
\EN{\section{Some basics}}
\DE{\section{Einige Grundlagen}}
\FR{\section{Quelques bases}}
\ES{\section{\ESph{}}}
\ITA{\section{Alcune basi teoriche}}
\PTBR{\section{\PTBRph{}}}
\THA{\section{\THAph{}}}
\PL{\section{\PLph{}}}

% sections:
\EN{\input{patterns/intro_CPU_ISA_EN}}
\ES{\input{patterns/intro_CPU_ISA_ES}}
\ITA{\input{patterns/intro_CPU_ISA_ITA}}
\PTBR{\input{patterns/intro_CPU_ISA_PTBR}}
\RU{\input{patterns/intro_CPU_ISA_RU}}
\DE{\input{patterns/intro_CPU_ISA_DE}}
\FR{\input{patterns/intro_CPU_ISA_FR}}
\PL{\input{patterns/intro_CPU_ISA_PL}}

\EN{\input{patterns/numeral_EN}}
\RU{\input{patterns/numeral_RU}}
\ITA{\input{patterns/numeral_ITA}}
\DE{\input{patterns/numeral_DE}}
\FR{\input{patterns/numeral_FR}}
\PL{\input{patterns/numeral_PL}}

% chapters
\input{patterns/00_empty/main}
\input{patterns/011_ret/main}
\input{patterns/01_helloworld/main}
\input{patterns/015_prolog_epilogue/main}
\input{patterns/02_stack/main}
\input{patterns/03_printf/main}
\input{patterns/04_scanf/main}
\input{patterns/05_passing_arguments/main}
\input{patterns/06_return_results/main}
\input{patterns/061_pointers/main}
\input{patterns/065_GOTO/main}
\input{patterns/07_jcc/main}
\input{patterns/08_switch/main}
\input{patterns/09_loops/main}
\input{patterns/10_strings/main}
\input{patterns/11_arith_optimizations/main}
\input{patterns/12_FPU/main}
\input{patterns/13_arrays/main}
\input{patterns/14_bitfields/main}
\EN{\input{patterns/145_LCG/main_EN}}
\RU{\input{patterns/145_LCG/main_RU}}
\input{patterns/15_structs/main}
\input{patterns/17_unions/main}
\input{patterns/18_pointers_to_functions/main}
\input{patterns/185_64bit_in_32_env/main}

\EN{\input{patterns/19_SIMD/main_EN}}
\RU{\input{patterns/19_SIMD/main_RU}}
\DE{\input{patterns/19_SIMD/main_DE}}

\EN{\input{patterns/20_x64/main_EN}}
\RU{\input{patterns/20_x64/main_RU}}

\EN{\input{patterns/205_floating_SIMD/main_EN}}
\RU{\input{patterns/205_floating_SIMD/main_RU}}
\DE{\input{patterns/205_floating_SIMD/main_DE}}

\EN{\input{patterns/ARM/main_EN}}
\RU{\input{patterns/ARM/main_RU}}
\DE{\input{patterns/ARM/main_DE}}

\input{patterns/MIPS/main}

\ifdefined\SPANISH
\chapter{Patrones de código}
\fi % SPANISH

\ifdefined\GERMAN
\chapter{Code-Muster}
\fi % GERMAN

\ifdefined\ENGLISH
\chapter{Code Patterns}
\fi % ENGLISH

\ifdefined\ITALIAN
\chapter{Forme di codice}
\fi % ITALIAN

\ifdefined\RUSSIAN
\chapter{Образцы кода}
\fi % RUSSIAN

\ifdefined\BRAZILIAN
\chapter{Padrões de códigos}
\fi % BRAZILIAN

\ifdefined\THAI
\chapter{รูปแบบของโค้ด}
\fi % THAI

\ifdefined\FRENCH
\chapter{Modèle de code}
\fi % FRENCH

\ifdefined\POLISH
\chapter{\PLph{}}
\fi % POLISH

% sections
\EN{\input{patterns/patterns_opt_dbg_EN}}
\ES{\input{patterns/patterns_opt_dbg_ES}}
\ITA{\input{patterns/patterns_opt_dbg_ITA}}
\PTBR{\input{patterns/patterns_opt_dbg_PTBR}}
\RU{\input{patterns/patterns_opt_dbg_RU}}
\THA{\input{patterns/patterns_opt_dbg_THA}}
\DE{\input{patterns/patterns_opt_dbg_DE}}
\FR{\input{patterns/patterns_opt_dbg_FR}}
\PL{\input{patterns/patterns_opt_dbg_PL}}

\RU{\section{Некоторые базовые понятия}}
\EN{\section{Some basics}}
\DE{\section{Einige Grundlagen}}
\FR{\section{Quelques bases}}
\ES{\section{\ESph{}}}
\ITA{\section{Alcune basi teoriche}}
\PTBR{\section{\PTBRph{}}}
\THA{\section{\THAph{}}}
\PL{\section{\PLph{}}}

% sections:
\EN{\input{patterns/intro_CPU_ISA_EN}}
\ES{\input{patterns/intro_CPU_ISA_ES}}
\ITA{\input{patterns/intro_CPU_ISA_ITA}}
\PTBR{\input{patterns/intro_CPU_ISA_PTBR}}
\RU{\input{patterns/intro_CPU_ISA_RU}}
\DE{\input{patterns/intro_CPU_ISA_DE}}
\FR{\input{patterns/intro_CPU_ISA_FR}}
\PL{\input{patterns/intro_CPU_ISA_PL}}

\EN{\input{patterns/numeral_EN}}
\RU{\input{patterns/numeral_RU}}
\ITA{\input{patterns/numeral_ITA}}
\DE{\input{patterns/numeral_DE}}
\FR{\input{patterns/numeral_FR}}
\PL{\input{patterns/numeral_PL}}

% chapters
\input{patterns/00_empty/main}
\input{patterns/011_ret/main}
\input{patterns/01_helloworld/main}
\input{patterns/015_prolog_epilogue/main}
\input{patterns/02_stack/main}
\input{patterns/03_printf/main}
\input{patterns/04_scanf/main}
\input{patterns/05_passing_arguments/main}
\input{patterns/06_return_results/main}
\input{patterns/061_pointers/main}
\input{patterns/065_GOTO/main}
\input{patterns/07_jcc/main}
\input{patterns/08_switch/main}
\input{patterns/09_loops/main}
\input{patterns/10_strings/main}
\input{patterns/11_arith_optimizations/main}
\input{patterns/12_FPU/main}
\input{patterns/13_arrays/main}
\input{patterns/14_bitfields/main}
\EN{\input{patterns/145_LCG/main_EN}}
\RU{\input{patterns/145_LCG/main_RU}}
\input{patterns/15_structs/main}
\input{patterns/17_unions/main}
\input{patterns/18_pointers_to_functions/main}
\input{patterns/185_64bit_in_32_env/main}

\EN{\input{patterns/19_SIMD/main_EN}}
\RU{\input{patterns/19_SIMD/main_RU}}
\DE{\input{patterns/19_SIMD/main_DE}}

\EN{\input{patterns/20_x64/main_EN}}
\RU{\input{patterns/20_x64/main_RU}}

\EN{\input{patterns/205_floating_SIMD/main_EN}}
\RU{\input{patterns/205_floating_SIMD/main_RU}}
\DE{\input{patterns/205_floating_SIMD/main_DE}}

\EN{\input{patterns/ARM/main_EN}}
\RU{\input{patterns/ARM/main_RU}}
\DE{\input{patterns/ARM/main_DE}}

\input{patterns/MIPS/main}

\ifdefined\SPANISH
\chapter{Patrones de código}
\fi % SPANISH

\ifdefined\GERMAN
\chapter{Code-Muster}
\fi % GERMAN

\ifdefined\ENGLISH
\chapter{Code Patterns}
\fi % ENGLISH

\ifdefined\ITALIAN
\chapter{Forme di codice}
\fi % ITALIAN

\ifdefined\RUSSIAN
\chapter{Образцы кода}
\fi % RUSSIAN

\ifdefined\BRAZILIAN
\chapter{Padrões de códigos}
\fi % BRAZILIAN

\ifdefined\THAI
\chapter{รูปแบบของโค้ด}
\fi % THAI

\ifdefined\FRENCH
\chapter{Modèle de code}
\fi % FRENCH

\ifdefined\POLISH
\chapter{\PLph{}}
\fi % POLISH

% sections
\EN{\input{patterns/patterns_opt_dbg_EN}}
\ES{\input{patterns/patterns_opt_dbg_ES}}
\ITA{\input{patterns/patterns_opt_dbg_ITA}}
\PTBR{\input{patterns/patterns_opt_dbg_PTBR}}
\RU{\input{patterns/patterns_opt_dbg_RU}}
\THA{\input{patterns/patterns_opt_dbg_THA}}
\DE{\input{patterns/patterns_opt_dbg_DE}}
\FR{\input{patterns/patterns_opt_dbg_FR}}
\PL{\input{patterns/patterns_opt_dbg_PL}}

\RU{\section{Некоторые базовые понятия}}
\EN{\section{Some basics}}
\DE{\section{Einige Grundlagen}}
\FR{\section{Quelques bases}}
\ES{\section{\ESph{}}}
\ITA{\section{Alcune basi teoriche}}
\PTBR{\section{\PTBRph{}}}
\THA{\section{\THAph{}}}
\PL{\section{\PLph{}}}

% sections:
\EN{\input{patterns/intro_CPU_ISA_EN}}
\ES{\input{patterns/intro_CPU_ISA_ES}}
\ITA{\input{patterns/intro_CPU_ISA_ITA}}
\PTBR{\input{patterns/intro_CPU_ISA_PTBR}}
\RU{\input{patterns/intro_CPU_ISA_RU}}
\DE{\input{patterns/intro_CPU_ISA_DE}}
\FR{\input{patterns/intro_CPU_ISA_FR}}
\PL{\input{patterns/intro_CPU_ISA_PL}}

\EN{\input{patterns/numeral_EN}}
\RU{\input{patterns/numeral_RU}}
\ITA{\input{patterns/numeral_ITA}}
\DE{\input{patterns/numeral_DE}}
\FR{\input{patterns/numeral_FR}}
\PL{\input{patterns/numeral_PL}}

% chapters
\input{patterns/00_empty/main}
\input{patterns/011_ret/main}
\input{patterns/01_helloworld/main}
\input{patterns/015_prolog_epilogue/main}
\input{patterns/02_stack/main}
\input{patterns/03_printf/main}
\input{patterns/04_scanf/main}
\input{patterns/05_passing_arguments/main}
\input{patterns/06_return_results/main}
\input{patterns/061_pointers/main}
\input{patterns/065_GOTO/main}
\input{patterns/07_jcc/main}
\input{patterns/08_switch/main}
\input{patterns/09_loops/main}
\input{patterns/10_strings/main}
\input{patterns/11_arith_optimizations/main}
\input{patterns/12_FPU/main}
\input{patterns/13_arrays/main}
\input{patterns/14_bitfields/main}
\EN{\input{patterns/145_LCG/main_EN}}
\RU{\input{patterns/145_LCG/main_RU}}
\input{patterns/15_structs/main}
\input{patterns/17_unions/main}
\input{patterns/18_pointers_to_functions/main}
\input{patterns/185_64bit_in_32_env/main}

\EN{\input{patterns/19_SIMD/main_EN}}
\RU{\input{patterns/19_SIMD/main_RU}}
\DE{\input{patterns/19_SIMD/main_DE}}

\EN{\input{patterns/20_x64/main_EN}}
\RU{\input{patterns/20_x64/main_RU}}

\EN{\input{patterns/205_floating_SIMD/main_EN}}
\RU{\input{patterns/205_floating_SIMD/main_RU}}
\DE{\input{patterns/205_floating_SIMD/main_DE}}

\EN{\input{patterns/ARM/main_EN}}
\RU{\input{patterns/ARM/main_RU}}
\DE{\input{patterns/ARM/main_DE}}

\input{patterns/MIPS/main}

\ifdefined\SPANISH
\chapter{Patrones de código}
\fi % SPANISH

\ifdefined\GERMAN
\chapter{Code-Muster}
\fi % GERMAN

\ifdefined\ENGLISH
\chapter{Code Patterns}
\fi % ENGLISH

\ifdefined\ITALIAN
\chapter{Forme di codice}
\fi % ITALIAN

\ifdefined\RUSSIAN
\chapter{Образцы кода}
\fi % RUSSIAN

\ifdefined\BRAZILIAN
\chapter{Padrões de códigos}
\fi % BRAZILIAN

\ifdefined\THAI
\chapter{รูปแบบของโค้ด}
\fi % THAI

\ifdefined\FRENCH
\chapter{Modèle de code}
\fi % FRENCH

\ifdefined\POLISH
\chapter{\PLph{}}
\fi % POLISH

% sections
\EN{\input{patterns/patterns_opt_dbg_EN}}
\ES{\input{patterns/patterns_opt_dbg_ES}}
\ITA{\input{patterns/patterns_opt_dbg_ITA}}
\PTBR{\input{patterns/patterns_opt_dbg_PTBR}}
\RU{\input{patterns/patterns_opt_dbg_RU}}
\THA{\input{patterns/patterns_opt_dbg_THA}}
\DE{\input{patterns/patterns_opt_dbg_DE}}
\FR{\input{patterns/patterns_opt_dbg_FR}}
\PL{\input{patterns/patterns_opt_dbg_PL}}

\RU{\section{Некоторые базовые понятия}}
\EN{\section{Some basics}}
\DE{\section{Einige Grundlagen}}
\FR{\section{Quelques bases}}
\ES{\section{\ESph{}}}
\ITA{\section{Alcune basi teoriche}}
\PTBR{\section{\PTBRph{}}}
\THA{\section{\THAph{}}}
\PL{\section{\PLph{}}}

% sections:
\EN{\input{patterns/intro_CPU_ISA_EN}}
\ES{\input{patterns/intro_CPU_ISA_ES}}
\ITA{\input{patterns/intro_CPU_ISA_ITA}}
\PTBR{\input{patterns/intro_CPU_ISA_PTBR}}
\RU{\input{patterns/intro_CPU_ISA_RU}}
\DE{\input{patterns/intro_CPU_ISA_DE}}
\FR{\input{patterns/intro_CPU_ISA_FR}}
\PL{\input{patterns/intro_CPU_ISA_PL}}

\EN{\input{patterns/numeral_EN}}
\RU{\input{patterns/numeral_RU}}
\ITA{\input{patterns/numeral_ITA}}
\DE{\input{patterns/numeral_DE}}
\FR{\input{patterns/numeral_FR}}
\PL{\input{patterns/numeral_PL}}

% chapters
\input{patterns/00_empty/main}
\input{patterns/011_ret/main}
\input{patterns/01_helloworld/main}
\input{patterns/015_prolog_epilogue/main}
\input{patterns/02_stack/main}
\input{patterns/03_printf/main}
\input{patterns/04_scanf/main}
\input{patterns/05_passing_arguments/main}
\input{patterns/06_return_results/main}
\input{patterns/061_pointers/main}
\input{patterns/065_GOTO/main}
\input{patterns/07_jcc/main}
\input{patterns/08_switch/main}
\input{patterns/09_loops/main}
\input{patterns/10_strings/main}
\input{patterns/11_arith_optimizations/main}
\input{patterns/12_FPU/main}
\input{patterns/13_arrays/main}
\input{patterns/14_bitfields/main}
\EN{\input{patterns/145_LCG/main_EN}}
\RU{\input{patterns/145_LCG/main_RU}}
\input{patterns/15_structs/main}
\input{patterns/17_unions/main}
\input{patterns/18_pointers_to_functions/main}
\input{patterns/185_64bit_in_32_env/main}

\EN{\input{patterns/19_SIMD/main_EN}}
\RU{\input{patterns/19_SIMD/main_RU}}
\DE{\input{patterns/19_SIMD/main_DE}}

\EN{\input{patterns/20_x64/main_EN}}
\RU{\input{patterns/20_x64/main_RU}}

\EN{\input{patterns/205_floating_SIMD/main_EN}}
\RU{\input{patterns/205_floating_SIMD/main_RU}}
\DE{\input{patterns/205_floating_SIMD/main_DE}}

\EN{\input{patterns/ARM/main_EN}}
\RU{\input{patterns/ARM/main_RU}}
\DE{\input{patterns/ARM/main_DE}}

\input{patterns/MIPS/main}

\ifdefined\SPANISH
\chapter{Patrones de código}
\fi % SPANISH

\ifdefined\GERMAN
\chapter{Code-Muster}
\fi % GERMAN

\ifdefined\ENGLISH
\chapter{Code Patterns}
\fi % ENGLISH

\ifdefined\ITALIAN
\chapter{Forme di codice}
\fi % ITALIAN

\ifdefined\RUSSIAN
\chapter{Образцы кода}
\fi % RUSSIAN

\ifdefined\BRAZILIAN
\chapter{Padrões de códigos}
\fi % BRAZILIAN

\ifdefined\THAI
\chapter{รูปแบบของโค้ด}
\fi % THAI

\ifdefined\FRENCH
\chapter{Modèle de code}
\fi % FRENCH

\ifdefined\POLISH
\chapter{\PLph{}}
\fi % POLISH

% sections
\EN{\input{patterns/patterns_opt_dbg_EN}}
\ES{\input{patterns/patterns_opt_dbg_ES}}
\ITA{\input{patterns/patterns_opt_dbg_ITA}}
\PTBR{\input{patterns/patterns_opt_dbg_PTBR}}
\RU{\input{patterns/patterns_opt_dbg_RU}}
\THA{\input{patterns/patterns_opt_dbg_THA}}
\DE{\input{patterns/patterns_opt_dbg_DE}}
\FR{\input{patterns/patterns_opt_dbg_FR}}
\PL{\input{patterns/patterns_opt_dbg_PL}}

\RU{\section{Некоторые базовые понятия}}
\EN{\section{Some basics}}
\DE{\section{Einige Grundlagen}}
\FR{\section{Quelques bases}}
\ES{\section{\ESph{}}}
\ITA{\section{Alcune basi teoriche}}
\PTBR{\section{\PTBRph{}}}
\THA{\section{\THAph{}}}
\PL{\section{\PLph{}}}

% sections:
\EN{\input{patterns/intro_CPU_ISA_EN}}
\ES{\input{patterns/intro_CPU_ISA_ES}}
\ITA{\input{patterns/intro_CPU_ISA_ITA}}
\PTBR{\input{patterns/intro_CPU_ISA_PTBR}}
\RU{\input{patterns/intro_CPU_ISA_RU}}
\DE{\input{patterns/intro_CPU_ISA_DE}}
\FR{\input{patterns/intro_CPU_ISA_FR}}
\PL{\input{patterns/intro_CPU_ISA_PL}}

\EN{\input{patterns/numeral_EN}}
\RU{\input{patterns/numeral_RU}}
\ITA{\input{patterns/numeral_ITA}}
\DE{\input{patterns/numeral_DE}}
\FR{\input{patterns/numeral_FR}}
\PL{\input{patterns/numeral_PL}}

% chapters
\input{patterns/00_empty/main}
\input{patterns/011_ret/main}
\input{patterns/01_helloworld/main}
\input{patterns/015_prolog_epilogue/main}
\input{patterns/02_stack/main}
\input{patterns/03_printf/main}
\input{patterns/04_scanf/main}
\input{patterns/05_passing_arguments/main}
\input{patterns/06_return_results/main}
\input{patterns/061_pointers/main}
\input{patterns/065_GOTO/main}
\input{patterns/07_jcc/main}
\input{patterns/08_switch/main}
\input{patterns/09_loops/main}
\input{patterns/10_strings/main}
\input{patterns/11_arith_optimizations/main}
\input{patterns/12_FPU/main}
\input{patterns/13_arrays/main}
\input{patterns/14_bitfields/main}
\EN{\input{patterns/145_LCG/main_EN}}
\RU{\input{patterns/145_LCG/main_RU}}
\input{patterns/15_structs/main}
\input{patterns/17_unions/main}
\input{patterns/18_pointers_to_functions/main}
\input{patterns/185_64bit_in_32_env/main}

\EN{\input{patterns/19_SIMD/main_EN}}
\RU{\input{patterns/19_SIMD/main_RU}}
\DE{\input{patterns/19_SIMD/main_DE}}

\EN{\input{patterns/20_x64/main_EN}}
\RU{\input{patterns/20_x64/main_RU}}

\EN{\input{patterns/205_floating_SIMD/main_EN}}
\RU{\input{patterns/205_floating_SIMD/main_RU}}
\DE{\input{patterns/205_floating_SIMD/main_DE}}

\EN{\input{patterns/ARM/main_EN}}
\RU{\input{patterns/ARM/main_RU}}
\DE{\input{patterns/ARM/main_DE}}

\input{patterns/MIPS/main}

\ifdefined\SPANISH
\chapter{Patrones de código}
\fi % SPANISH

\ifdefined\GERMAN
\chapter{Code-Muster}
\fi % GERMAN

\ifdefined\ENGLISH
\chapter{Code Patterns}
\fi % ENGLISH

\ifdefined\ITALIAN
\chapter{Forme di codice}
\fi % ITALIAN

\ifdefined\RUSSIAN
\chapter{Образцы кода}
\fi % RUSSIAN

\ifdefined\BRAZILIAN
\chapter{Padrões de códigos}
\fi % BRAZILIAN

\ifdefined\THAI
\chapter{รูปแบบของโค้ด}
\fi % THAI

\ifdefined\FRENCH
\chapter{Modèle de code}
\fi % FRENCH

\ifdefined\POLISH
\chapter{\PLph{}}
\fi % POLISH

% sections
\EN{\input{patterns/patterns_opt_dbg_EN}}
\ES{\input{patterns/patterns_opt_dbg_ES}}
\ITA{\input{patterns/patterns_opt_dbg_ITA}}
\PTBR{\input{patterns/patterns_opt_dbg_PTBR}}
\RU{\input{patterns/patterns_opt_dbg_RU}}
\THA{\input{patterns/patterns_opt_dbg_THA}}
\DE{\input{patterns/patterns_opt_dbg_DE}}
\FR{\input{patterns/patterns_opt_dbg_FR}}
\PL{\input{patterns/patterns_opt_dbg_PL}}

\RU{\section{Некоторые базовые понятия}}
\EN{\section{Some basics}}
\DE{\section{Einige Grundlagen}}
\FR{\section{Quelques bases}}
\ES{\section{\ESph{}}}
\ITA{\section{Alcune basi teoriche}}
\PTBR{\section{\PTBRph{}}}
\THA{\section{\THAph{}}}
\PL{\section{\PLph{}}}

% sections:
\EN{\input{patterns/intro_CPU_ISA_EN}}
\ES{\input{patterns/intro_CPU_ISA_ES}}
\ITA{\input{patterns/intro_CPU_ISA_ITA}}
\PTBR{\input{patterns/intro_CPU_ISA_PTBR}}
\RU{\input{patterns/intro_CPU_ISA_RU}}
\DE{\input{patterns/intro_CPU_ISA_DE}}
\FR{\input{patterns/intro_CPU_ISA_FR}}
\PL{\input{patterns/intro_CPU_ISA_PL}}

\EN{\input{patterns/numeral_EN}}
\RU{\input{patterns/numeral_RU}}
\ITA{\input{patterns/numeral_ITA}}
\DE{\input{patterns/numeral_DE}}
\FR{\input{patterns/numeral_FR}}
\PL{\input{patterns/numeral_PL}}

% chapters
\input{patterns/00_empty/main}
\input{patterns/011_ret/main}
\input{patterns/01_helloworld/main}
\input{patterns/015_prolog_epilogue/main}
\input{patterns/02_stack/main}
\input{patterns/03_printf/main}
\input{patterns/04_scanf/main}
\input{patterns/05_passing_arguments/main}
\input{patterns/06_return_results/main}
\input{patterns/061_pointers/main}
\input{patterns/065_GOTO/main}
\input{patterns/07_jcc/main}
\input{patterns/08_switch/main}
\input{patterns/09_loops/main}
\input{patterns/10_strings/main}
\input{patterns/11_arith_optimizations/main}
\input{patterns/12_FPU/main}
\input{patterns/13_arrays/main}
\input{patterns/14_bitfields/main}
\EN{\input{patterns/145_LCG/main_EN}}
\RU{\input{patterns/145_LCG/main_RU}}
\input{patterns/15_structs/main}
\input{patterns/17_unions/main}
\input{patterns/18_pointers_to_functions/main}
\input{patterns/185_64bit_in_32_env/main}

\EN{\input{patterns/19_SIMD/main_EN}}
\RU{\input{patterns/19_SIMD/main_RU}}
\DE{\input{patterns/19_SIMD/main_DE}}

\EN{\input{patterns/20_x64/main_EN}}
\RU{\input{patterns/20_x64/main_RU}}

\EN{\input{patterns/205_floating_SIMD/main_EN}}
\RU{\input{patterns/205_floating_SIMD/main_RU}}
\DE{\input{patterns/205_floating_SIMD/main_DE}}

\EN{\input{patterns/ARM/main_EN}}
\RU{\input{patterns/ARM/main_RU}}
\DE{\input{patterns/ARM/main_DE}}

\input{patterns/MIPS/main}

\ifdefined\SPANISH
\chapter{Patrones de código}
\fi % SPANISH

\ifdefined\GERMAN
\chapter{Code-Muster}
\fi % GERMAN

\ifdefined\ENGLISH
\chapter{Code Patterns}
\fi % ENGLISH

\ifdefined\ITALIAN
\chapter{Forme di codice}
\fi % ITALIAN

\ifdefined\RUSSIAN
\chapter{Образцы кода}
\fi % RUSSIAN

\ifdefined\BRAZILIAN
\chapter{Padrões de códigos}
\fi % BRAZILIAN

\ifdefined\THAI
\chapter{รูปแบบของโค้ด}
\fi % THAI

\ifdefined\FRENCH
\chapter{Modèle de code}
\fi % FRENCH

\ifdefined\POLISH
\chapter{\PLph{}}
\fi % POLISH

% sections
\EN{\input{patterns/patterns_opt_dbg_EN}}
\ES{\input{patterns/patterns_opt_dbg_ES}}
\ITA{\input{patterns/patterns_opt_dbg_ITA}}
\PTBR{\input{patterns/patterns_opt_dbg_PTBR}}
\RU{\input{patterns/patterns_opt_dbg_RU}}
\THA{\input{patterns/patterns_opt_dbg_THA}}
\DE{\input{patterns/patterns_opt_dbg_DE}}
\FR{\input{patterns/patterns_opt_dbg_FR}}
\PL{\input{patterns/patterns_opt_dbg_PL}}

\RU{\section{Некоторые базовые понятия}}
\EN{\section{Some basics}}
\DE{\section{Einige Grundlagen}}
\FR{\section{Quelques bases}}
\ES{\section{\ESph{}}}
\ITA{\section{Alcune basi teoriche}}
\PTBR{\section{\PTBRph{}}}
\THA{\section{\THAph{}}}
\PL{\section{\PLph{}}}

% sections:
\EN{\input{patterns/intro_CPU_ISA_EN}}
\ES{\input{patterns/intro_CPU_ISA_ES}}
\ITA{\input{patterns/intro_CPU_ISA_ITA}}
\PTBR{\input{patterns/intro_CPU_ISA_PTBR}}
\RU{\input{patterns/intro_CPU_ISA_RU}}
\DE{\input{patterns/intro_CPU_ISA_DE}}
\FR{\input{patterns/intro_CPU_ISA_FR}}
\PL{\input{patterns/intro_CPU_ISA_PL}}

\EN{\input{patterns/numeral_EN}}
\RU{\input{patterns/numeral_RU}}
\ITA{\input{patterns/numeral_ITA}}
\DE{\input{patterns/numeral_DE}}
\FR{\input{patterns/numeral_FR}}
\PL{\input{patterns/numeral_PL}}

% chapters
\input{patterns/00_empty/main}
\input{patterns/011_ret/main}
\input{patterns/01_helloworld/main}
\input{patterns/015_prolog_epilogue/main}
\input{patterns/02_stack/main}
\input{patterns/03_printf/main}
\input{patterns/04_scanf/main}
\input{patterns/05_passing_arguments/main}
\input{patterns/06_return_results/main}
\input{patterns/061_pointers/main}
\input{patterns/065_GOTO/main}
\input{patterns/07_jcc/main}
\input{patterns/08_switch/main}
\input{patterns/09_loops/main}
\input{patterns/10_strings/main}
\input{patterns/11_arith_optimizations/main}
\input{patterns/12_FPU/main}
\input{patterns/13_arrays/main}
\input{patterns/14_bitfields/main}
\EN{\input{patterns/145_LCG/main_EN}}
\RU{\input{patterns/145_LCG/main_RU}}
\input{patterns/15_structs/main}
\input{patterns/17_unions/main}
\input{patterns/18_pointers_to_functions/main}
\input{patterns/185_64bit_in_32_env/main}

\EN{\input{patterns/19_SIMD/main_EN}}
\RU{\input{patterns/19_SIMD/main_RU}}
\DE{\input{patterns/19_SIMD/main_DE}}

\EN{\input{patterns/20_x64/main_EN}}
\RU{\input{patterns/20_x64/main_RU}}

\EN{\input{patterns/205_floating_SIMD/main_EN}}
\RU{\input{patterns/205_floating_SIMD/main_RU}}
\DE{\input{patterns/205_floating_SIMD/main_DE}}

\EN{\input{patterns/ARM/main_EN}}
\RU{\input{patterns/ARM/main_RU}}
\DE{\input{patterns/ARM/main_DE}}

\input{patterns/MIPS/main}

\EN{\input{patterns/12_FPU/main_EN}}
\RU{\input{patterns/12_FPU/main_RU}}
\DE{\input{patterns/12_FPU/main_DE}}
\FR{\input{patterns/12_FPU/main_FR}}


\ifdefined\SPANISH
\chapter{Patrones de código}
\fi % SPANISH

\ifdefined\GERMAN
\chapter{Code-Muster}
\fi % GERMAN

\ifdefined\ENGLISH
\chapter{Code Patterns}
\fi % ENGLISH

\ifdefined\ITALIAN
\chapter{Forme di codice}
\fi % ITALIAN

\ifdefined\RUSSIAN
\chapter{Образцы кода}
\fi % RUSSIAN

\ifdefined\BRAZILIAN
\chapter{Padrões de códigos}
\fi % BRAZILIAN

\ifdefined\THAI
\chapter{รูปแบบของโค้ด}
\fi % THAI

\ifdefined\FRENCH
\chapter{Modèle de code}
\fi % FRENCH

\ifdefined\POLISH
\chapter{\PLph{}}
\fi % POLISH

% sections
\EN{\input{patterns/patterns_opt_dbg_EN}}
\ES{\input{patterns/patterns_opt_dbg_ES}}
\ITA{\input{patterns/patterns_opt_dbg_ITA}}
\PTBR{\input{patterns/patterns_opt_dbg_PTBR}}
\RU{\input{patterns/patterns_opt_dbg_RU}}
\THA{\input{patterns/patterns_opt_dbg_THA}}
\DE{\input{patterns/patterns_opt_dbg_DE}}
\FR{\input{patterns/patterns_opt_dbg_FR}}
\PL{\input{patterns/patterns_opt_dbg_PL}}

\RU{\section{Некоторые базовые понятия}}
\EN{\section{Some basics}}
\DE{\section{Einige Grundlagen}}
\FR{\section{Quelques bases}}
\ES{\section{\ESph{}}}
\ITA{\section{Alcune basi teoriche}}
\PTBR{\section{\PTBRph{}}}
\THA{\section{\THAph{}}}
\PL{\section{\PLph{}}}

% sections:
\EN{\input{patterns/intro_CPU_ISA_EN}}
\ES{\input{patterns/intro_CPU_ISA_ES}}
\ITA{\input{patterns/intro_CPU_ISA_ITA}}
\PTBR{\input{patterns/intro_CPU_ISA_PTBR}}
\RU{\input{patterns/intro_CPU_ISA_RU}}
\DE{\input{patterns/intro_CPU_ISA_DE}}
\FR{\input{patterns/intro_CPU_ISA_FR}}
\PL{\input{patterns/intro_CPU_ISA_PL}}

\EN{\input{patterns/numeral_EN}}
\RU{\input{patterns/numeral_RU}}
\ITA{\input{patterns/numeral_ITA}}
\DE{\input{patterns/numeral_DE}}
\FR{\input{patterns/numeral_FR}}
\PL{\input{patterns/numeral_PL}}

% chapters
\input{patterns/00_empty/main}
\input{patterns/011_ret/main}
\input{patterns/01_helloworld/main}
\input{patterns/015_prolog_epilogue/main}
\input{patterns/02_stack/main}
\input{patterns/03_printf/main}
\input{patterns/04_scanf/main}
\input{patterns/05_passing_arguments/main}
\input{patterns/06_return_results/main}
\input{patterns/061_pointers/main}
\input{patterns/065_GOTO/main}
\input{patterns/07_jcc/main}
\input{patterns/08_switch/main}
\input{patterns/09_loops/main}
\input{patterns/10_strings/main}
\input{patterns/11_arith_optimizations/main}
\input{patterns/12_FPU/main}
\input{patterns/13_arrays/main}
\input{patterns/14_bitfields/main}
\EN{\input{patterns/145_LCG/main_EN}}
\RU{\input{patterns/145_LCG/main_RU}}
\input{patterns/15_structs/main}
\input{patterns/17_unions/main}
\input{patterns/18_pointers_to_functions/main}
\input{patterns/185_64bit_in_32_env/main}

\EN{\input{patterns/19_SIMD/main_EN}}
\RU{\input{patterns/19_SIMD/main_RU}}
\DE{\input{patterns/19_SIMD/main_DE}}

\EN{\input{patterns/20_x64/main_EN}}
\RU{\input{patterns/20_x64/main_RU}}

\EN{\input{patterns/205_floating_SIMD/main_EN}}
\RU{\input{patterns/205_floating_SIMD/main_RU}}
\DE{\input{patterns/205_floating_SIMD/main_DE}}

\EN{\input{patterns/ARM/main_EN}}
\RU{\input{patterns/ARM/main_RU}}
\DE{\input{patterns/ARM/main_DE}}

\input{patterns/MIPS/main}

\ifdefined\SPANISH
\chapter{Patrones de código}
\fi % SPANISH

\ifdefined\GERMAN
\chapter{Code-Muster}
\fi % GERMAN

\ifdefined\ENGLISH
\chapter{Code Patterns}
\fi % ENGLISH

\ifdefined\ITALIAN
\chapter{Forme di codice}
\fi % ITALIAN

\ifdefined\RUSSIAN
\chapter{Образцы кода}
\fi % RUSSIAN

\ifdefined\BRAZILIAN
\chapter{Padrões de códigos}
\fi % BRAZILIAN

\ifdefined\THAI
\chapter{รูปแบบของโค้ด}
\fi % THAI

\ifdefined\FRENCH
\chapter{Modèle de code}
\fi % FRENCH

\ifdefined\POLISH
\chapter{\PLph{}}
\fi % POLISH

% sections
\EN{\input{patterns/patterns_opt_dbg_EN}}
\ES{\input{patterns/patterns_opt_dbg_ES}}
\ITA{\input{patterns/patterns_opt_dbg_ITA}}
\PTBR{\input{patterns/patterns_opt_dbg_PTBR}}
\RU{\input{patterns/patterns_opt_dbg_RU}}
\THA{\input{patterns/patterns_opt_dbg_THA}}
\DE{\input{patterns/patterns_opt_dbg_DE}}
\FR{\input{patterns/patterns_opt_dbg_FR}}
\PL{\input{patterns/patterns_opt_dbg_PL}}

\RU{\section{Некоторые базовые понятия}}
\EN{\section{Some basics}}
\DE{\section{Einige Grundlagen}}
\FR{\section{Quelques bases}}
\ES{\section{\ESph{}}}
\ITA{\section{Alcune basi teoriche}}
\PTBR{\section{\PTBRph{}}}
\THA{\section{\THAph{}}}
\PL{\section{\PLph{}}}

% sections:
\EN{\input{patterns/intro_CPU_ISA_EN}}
\ES{\input{patterns/intro_CPU_ISA_ES}}
\ITA{\input{patterns/intro_CPU_ISA_ITA}}
\PTBR{\input{patterns/intro_CPU_ISA_PTBR}}
\RU{\input{patterns/intro_CPU_ISA_RU}}
\DE{\input{patterns/intro_CPU_ISA_DE}}
\FR{\input{patterns/intro_CPU_ISA_FR}}
\PL{\input{patterns/intro_CPU_ISA_PL}}

\EN{\input{patterns/numeral_EN}}
\RU{\input{patterns/numeral_RU}}
\ITA{\input{patterns/numeral_ITA}}
\DE{\input{patterns/numeral_DE}}
\FR{\input{patterns/numeral_FR}}
\PL{\input{patterns/numeral_PL}}

% chapters
\input{patterns/00_empty/main}
\input{patterns/011_ret/main}
\input{patterns/01_helloworld/main}
\input{patterns/015_prolog_epilogue/main}
\input{patterns/02_stack/main}
\input{patterns/03_printf/main}
\input{patterns/04_scanf/main}
\input{patterns/05_passing_arguments/main}
\input{patterns/06_return_results/main}
\input{patterns/061_pointers/main}
\input{patterns/065_GOTO/main}
\input{patterns/07_jcc/main}
\input{patterns/08_switch/main}
\input{patterns/09_loops/main}
\input{patterns/10_strings/main}
\input{patterns/11_arith_optimizations/main}
\input{patterns/12_FPU/main}
\input{patterns/13_arrays/main}
\input{patterns/14_bitfields/main}
\EN{\input{patterns/145_LCG/main_EN}}
\RU{\input{patterns/145_LCG/main_RU}}
\input{patterns/15_structs/main}
\input{patterns/17_unions/main}
\input{patterns/18_pointers_to_functions/main}
\input{patterns/185_64bit_in_32_env/main}

\EN{\input{patterns/19_SIMD/main_EN}}
\RU{\input{patterns/19_SIMD/main_RU}}
\DE{\input{patterns/19_SIMD/main_DE}}

\EN{\input{patterns/20_x64/main_EN}}
\RU{\input{patterns/20_x64/main_RU}}

\EN{\input{patterns/205_floating_SIMD/main_EN}}
\RU{\input{patterns/205_floating_SIMD/main_RU}}
\DE{\input{patterns/205_floating_SIMD/main_DE}}

\EN{\input{patterns/ARM/main_EN}}
\RU{\input{patterns/ARM/main_RU}}
\DE{\input{patterns/ARM/main_DE}}

\input{patterns/MIPS/main}

\EN{\section{Returning Values}
\label{ret_val_func}

Another simple function is the one that simply returns a constant value:

\lstinputlisting[caption=\EN{\CCpp Code},style=customc]{patterns/011_ret/1.c}

Let's compile it.

\subsection{x86}

Here's what both the GCC and MSVC compilers produce (with optimization) on the x86 platform:

\lstinputlisting[caption=\Optimizing GCC/MSVC (\assemblyOutput),style=customasmx86]{patterns/011_ret/1.s}

\myindex{x86!\Instructions!RET}
There are just two instructions: the first places the value 123 into the \EAX register,
which is used by convention for storing the return
value, and the second one is \RET, which returns execution to the \gls{caller}.

The caller will take the result from the \EAX register.

\subsection{ARM}

There are a few differences on the ARM platform:

\lstinputlisting[caption=\OptimizingKeilVI (\ARMMode) ASM Output,style=customasmARM]{patterns/011_ret/1_Keil_ARM_O3.s}

ARM uses the register \Reg{0} for returning the results of functions, so 123 is copied into \Reg{0}.

\myindex{ARM!\Instructions!MOV}
\myindex{x86!\Instructions!MOV}
It is worth noting that \MOV is a misleading name for the instruction in both the x86 and ARM \ac{ISA}s.

The data is not in fact \IT{moved}, but \IT{copied}.

\subsection{MIPS}

\label{MIPS_leaf_function_ex1}

The GCC assembly output below lists registers by number:

\lstinputlisting[caption=\Optimizing GCC 4.4.5 (\assemblyOutput),style=customasmMIPS]{patterns/011_ret/MIPS.s}

\dots while \IDA does it by their pseudo names:

\lstinputlisting[caption=\Optimizing GCC 4.4.5 (IDA),style=customasmMIPS]{patterns/011_ret/MIPS_IDA.lst}

The \$2 (or \$V0) register is used to store the function's return value.
\myindex{MIPS!\Pseudoinstructions!LI}
\INS{LI} stands for ``Load Immediate'' and is the MIPS equivalent to \MOV.

\myindex{MIPS!\Instructions!J}
The other instruction is the jump instruction (J or JR) which returns the execution flow to the \gls{caller}.

\myindex{MIPS!Branch delay slot}
You might be wondering why the positions of the load instruction (LI) and the jump instruction (J or JR) are swapped. This is due to a \ac{RISC} feature called ``branch delay slot''.

The reason this happens is a quirk in the architecture of some RISC \ac{ISA}s and isn't important for our
purposes---we must simply keep in mind that in MIPS, the instruction following a jump or branch instruction
is executed \IT{before} the jump/branch instruction itself.

As a consequence, branch instructions always swap places with the instruction executed immediately beforehand.


In practice, functions which merely return 1 (\IT{true}) or 0 (\IT{false}) are very frequent.

The smallest ever of the standard UNIX utilities, \IT{/bin/true} and \IT{/bin/false} return 0 and 1 respectively, as an exit code.
(Zero as an exit code usually means success, non-zero means error.)
}
\RU{\subsubsection{std::string}
\myindex{\Cpp!STL!std::string}
\label{std_string}

\myparagraph{Как устроена структура}

Многие строковые библиотеки \InSqBrackets{\CNotes 2.2} обеспечивают структуру содержащую ссылку 
на буфер собственно со строкой, переменная всегда содержащую длину строки 
(что очень удобно для массы функций \InSqBrackets{\CNotes 2.2.1}) и переменную содержащую текущий размер буфера.

Строка в буфере обыкновенно оканчивается нулем: это для того чтобы указатель на буфер можно было
передавать в функции требующие на вход обычную сишную \ac{ASCIIZ}-строку.

Стандарт \Cpp не описывает, как именно нужно реализовывать std::string,
но, как правило, они реализованы как описано выше, с небольшими дополнениями.

Строки в \Cpp это не класс (как, например, QString в Qt), а темплейт (basic\_string), 
это сделано для того чтобы поддерживать 
строки содержащие разного типа символы: как минимум \Tchar и \IT{wchar\_t}.

Так что, std::string это класс с базовым типом \Tchar.

А std::wstring это класс с базовым типом \IT{wchar\_t}.

\mysubparagraph{MSVC}

В реализации MSVC, вместо ссылки на буфер может содержаться сам буфер (если строка короче 16-и символов).

Это означает, что каждая короткая строка будет занимать в памяти по крайней мере $16 + 4 + 4 = 24$ 
байт для 32-битной среды либо $16 + 8 + 8 = 32$ 
байта в 64-битной, а если строка длиннее 16-и символов, то прибавьте еще длину самой строки.

\lstinputlisting[caption=пример для MSVC,style=customc]{\CURPATH/STL/string/MSVC_RU.cpp}

Собственно, из этого исходника почти всё ясно.

Несколько замечаний:

Если строка короче 16-и символов, 
то отдельный буфер для строки в \glslink{heap}{куче} выделяться не будет.

Это удобно потому что на практике, основная часть строк действительно короткие.
Вероятно, разработчики в Microsoft выбрали размер в 16 символов как разумный баланс.

Теперь очень важный момент в конце функции main(): мы не пользуемся методом c\_str(), тем не менее,
если это скомпилировать и запустить, то обе строки появятся в консоли!

Работает это вот почему.

В первом случае строка короче 16-и символов и в начале объекта std::string (его можно рассматривать
просто как структуру) расположен буфер с этой строкой.
\printf трактует указатель как указатель на массив символов оканчивающийся нулем и поэтому всё работает.

Вывод второй строки (длиннее 16-и символов) даже еще опаснее: это вообще типичная программистская ошибка 
(или опечатка), забыть дописать c\_str().
Это работает потому что в это время в начале структуры расположен указатель на буфер.
Это может надолго остаться незамеченным: до тех пока там не появится строка 
короче 16-и символов, тогда процесс упадет.

\mysubparagraph{GCC}

В реализации GCC в структуре есть еще одна переменная --- reference count.

Интересно, что указатель на экземпляр класса std::string в GCC указывает не на начало самой структуры, 
а на указатель на буфера.
В libstdc++-v3\textbackslash{}include\textbackslash{}bits\textbackslash{}basic\_string.h 
мы можем прочитать что это сделано для удобства отладки:

\begin{lstlisting}
   *  The reason you want _M_data pointing to the character %array and
   *  not the _Rep is so that the debugger can see the string
   *  contents. (Probably we should add a non-inline member to get
   *  the _Rep for the debugger to use, so users can check the actual
   *  string length.)
\end{lstlisting}

\href{http://go.yurichev.com/17085}{исходный код basic\_string.h}

В нашем примере мы учитываем это:

\lstinputlisting[caption=пример для GCC,style=customc]{\CURPATH/STL/string/GCC_RU.cpp}

Нужны еще небольшие хаки чтобы сымитировать типичную ошибку, которую мы уже видели выше, из-за
более ужесточенной проверки типов в GCC, тем не менее, printf() работает и здесь без c\_str().

\myparagraph{Чуть более сложный пример}

\lstinputlisting[style=customc]{\CURPATH/STL/string/3.cpp}

\lstinputlisting[caption=MSVC 2012,style=customasmx86]{\CURPATH/STL/string/3_MSVC_RU.asm}

Собственно, компилятор не конструирует строки статически: да в общем-то и как
это возможно, если буфер с ней нужно хранить в \glslink{heap}{куче}?

Вместо этого в сегменте данных хранятся обычные \ac{ASCIIZ}-строки, а позже, во время выполнения, 
при помощи метода \q{assign}, конструируются строки s1 и s2
.
При помощи \TT{operator+}, создается строка s3.

Обратите внимание на то что вызов метода c\_str() отсутствует,
потому что его код достаточно короткий и компилятор вставил его прямо здесь:
если строка короче 16-и байт, то в регистре EAX остается указатель на буфер,
а если длиннее, то из этого же места достается адрес на буфер расположенный в \glslink{heap}{куче}.

Далее следуют вызовы трех деструкторов, причем, они вызываются только если строка длиннее 16-и байт:
тогда нужно освободить буфера в \glslink{heap}{куче}.
В противном случае, так как все три объекта std::string хранятся в стеке,
они освобождаются автоматически после выхода из функции.

Следовательно, работа с короткими строками более быстрая из-за м\'{е}ньшего обращения к \glslink{heap}{куче}.

Код на GCC даже проще (из-за того, что в GCC, как мы уже видели, не реализована возможность хранить короткую
строку прямо в структуре):

% TODO1 comment each function meaning
\lstinputlisting[caption=GCC 4.8.1,style=customasmx86]{\CURPATH/STL/string/3_GCC_RU.s}

Можно заметить, что в деструкторы передается не указатель на объект,
а указатель на место за 12 байт (или 3 слова) перед ним, то есть, на настоящее начало структуры.

\myparagraph{std::string как глобальная переменная}
\label{sec:std_string_as_global_variable}

Опытные программисты на \Cpp знают, что глобальные переменные \ac{STL}-типов вполне можно объявлять.

Да, действительно:

\lstinputlisting[style=customc]{\CURPATH/STL/string/5.cpp}

Но как и где будет вызываться конструктор \TT{std::string}?

На самом деле, эта переменная будет инициализирована даже перед началом \main.

\lstinputlisting[caption=MSVC 2012: здесь конструируется глобальная переменная{,} а также регистрируется её деструктор,style=customasmx86]{\CURPATH/STL/string/5_MSVC_p2.asm}

\lstinputlisting[caption=MSVC 2012: здесь глобальная переменная используется в \main,style=customasmx86]{\CURPATH/STL/string/5_MSVC_p1.asm}

\lstinputlisting[caption=MSVC 2012: эта функция-деструктор вызывается перед выходом,style=customasmx86]{\CURPATH/STL/string/5_MSVC_p3.asm}

\myindex{\CStandardLibrary!atexit()}
В реальности, из \ac{CRT}, еще до вызова main(), вызывается специальная функция,
в которой перечислены все конструкторы подобных переменных.
Более того: при помощи atexit() регистрируется функция, которая будет вызвана в конце работы программы:
в этой функции компилятор собирает вызовы деструкторов всех подобных глобальных переменных.

GCC работает похожим образом:

\lstinputlisting[caption=GCC 4.8.1,style=customasmx86]{\CURPATH/STL/string/5_GCC.s}

Но он не выделяет отдельной функции в которой будут собраны деструкторы: 
каждый деструктор передается в atexit() по одному.

% TODO а если глобальная STL-переменная в другом модуле? надо проверить.

}
\ifdefined\SPANISH
\chapter{Patrones de código}
\fi % SPANISH

\ifdefined\GERMAN
\chapter{Code-Muster}
\fi % GERMAN

\ifdefined\ENGLISH
\chapter{Code Patterns}
\fi % ENGLISH

\ifdefined\ITALIAN
\chapter{Forme di codice}
\fi % ITALIAN

\ifdefined\RUSSIAN
\chapter{Образцы кода}
\fi % RUSSIAN

\ifdefined\BRAZILIAN
\chapter{Padrões de códigos}
\fi % BRAZILIAN

\ifdefined\THAI
\chapter{รูปแบบของโค้ด}
\fi % THAI

\ifdefined\FRENCH
\chapter{Modèle de code}
\fi % FRENCH

\ifdefined\POLISH
\chapter{\PLph{}}
\fi % POLISH

% sections
\EN{\input{patterns/patterns_opt_dbg_EN}}
\ES{\input{patterns/patterns_opt_dbg_ES}}
\ITA{\input{patterns/patterns_opt_dbg_ITA}}
\PTBR{\input{patterns/patterns_opt_dbg_PTBR}}
\RU{\input{patterns/patterns_opt_dbg_RU}}
\THA{\input{patterns/patterns_opt_dbg_THA}}
\DE{\input{patterns/patterns_opt_dbg_DE}}
\FR{\input{patterns/patterns_opt_dbg_FR}}
\PL{\input{patterns/patterns_opt_dbg_PL}}

\RU{\section{Некоторые базовые понятия}}
\EN{\section{Some basics}}
\DE{\section{Einige Grundlagen}}
\FR{\section{Quelques bases}}
\ES{\section{\ESph{}}}
\ITA{\section{Alcune basi teoriche}}
\PTBR{\section{\PTBRph{}}}
\THA{\section{\THAph{}}}
\PL{\section{\PLph{}}}

% sections:
\EN{\input{patterns/intro_CPU_ISA_EN}}
\ES{\input{patterns/intro_CPU_ISA_ES}}
\ITA{\input{patterns/intro_CPU_ISA_ITA}}
\PTBR{\input{patterns/intro_CPU_ISA_PTBR}}
\RU{\input{patterns/intro_CPU_ISA_RU}}
\DE{\input{patterns/intro_CPU_ISA_DE}}
\FR{\input{patterns/intro_CPU_ISA_FR}}
\PL{\input{patterns/intro_CPU_ISA_PL}}

\EN{\input{patterns/numeral_EN}}
\RU{\input{patterns/numeral_RU}}
\ITA{\input{patterns/numeral_ITA}}
\DE{\input{patterns/numeral_DE}}
\FR{\input{patterns/numeral_FR}}
\PL{\input{patterns/numeral_PL}}

% chapters
\input{patterns/00_empty/main}
\input{patterns/011_ret/main}
\input{patterns/01_helloworld/main}
\input{patterns/015_prolog_epilogue/main}
\input{patterns/02_stack/main}
\input{patterns/03_printf/main}
\input{patterns/04_scanf/main}
\input{patterns/05_passing_arguments/main}
\input{patterns/06_return_results/main}
\input{patterns/061_pointers/main}
\input{patterns/065_GOTO/main}
\input{patterns/07_jcc/main}
\input{patterns/08_switch/main}
\input{patterns/09_loops/main}
\input{patterns/10_strings/main}
\input{patterns/11_arith_optimizations/main}
\input{patterns/12_FPU/main}
\input{patterns/13_arrays/main}
\input{patterns/14_bitfields/main}
\EN{\input{patterns/145_LCG/main_EN}}
\RU{\input{patterns/145_LCG/main_RU}}
\input{patterns/15_structs/main}
\input{patterns/17_unions/main}
\input{patterns/18_pointers_to_functions/main}
\input{patterns/185_64bit_in_32_env/main}

\EN{\input{patterns/19_SIMD/main_EN}}
\RU{\input{patterns/19_SIMD/main_RU}}
\DE{\input{patterns/19_SIMD/main_DE}}

\EN{\input{patterns/20_x64/main_EN}}
\RU{\input{patterns/20_x64/main_RU}}

\EN{\input{patterns/205_floating_SIMD/main_EN}}
\RU{\input{patterns/205_floating_SIMD/main_RU}}
\DE{\input{patterns/205_floating_SIMD/main_DE}}

\EN{\input{patterns/ARM/main_EN}}
\RU{\input{patterns/ARM/main_RU}}
\DE{\input{patterns/ARM/main_DE}}

\input{patterns/MIPS/main}

\ifdefined\SPANISH
\chapter{Patrones de código}
\fi % SPANISH

\ifdefined\GERMAN
\chapter{Code-Muster}
\fi % GERMAN

\ifdefined\ENGLISH
\chapter{Code Patterns}
\fi % ENGLISH

\ifdefined\ITALIAN
\chapter{Forme di codice}
\fi % ITALIAN

\ifdefined\RUSSIAN
\chapter{Образцы кода}
\fi % RUSSIAN

\ifdefined\BRAZILIAN
\chapter{Padrões de códigos}
\fi % BRAZILIAN

\ifdefined\THAI
\chapter{รูปแบบของโค้ด}
\fi % THAI

\ifdefined\FRENCH
\chapter{Modèle de code}
\fi % FRENCH

\ifdefined\POLISH
\chapter{\PLph{}}
\fi % POLISH

% sections
\EN{\input{patterns/patterns_opt_dbg_EN}}
\ES{\input{patterns/patterns_opt_dbg_ES}}
\ITA{\input{patterns/patterns_opt_dbg_ITA}}
\PTBR{\input{patterns/patterns_opt_dbg_PTBR}}
\RU{\input{patterns/patterns_opt_dbg_RU}}
\THA{\input{patterns/patterns_opt_dbg_THA}}
\DE{\input{patterns/patterns_opt_dbg_DE}}
\FR{\input{patterns/patterns_opt_dbg_FR}}
\PL{\input{patterns/patterns_opt_dbg_PL}}

\RU{\section{Некоторые базовые понятия}}
\EN{\section{Some basics}}
\DE{\section{Einige Grundlagen}}
\FR{\section{Quelques bases}}
\ES{\section{\ESph{}}}
\ITA{\section{Alcune basi teoriche}}
\PTBR{\section{\PTBRph{}}}
\THA{\section{\THAph{}}}
\PL{\section{\PLph{}}}

% sections:
\EN{\input{patterns/intro_CPU_ISA_EN}}
\ES{\input{patterns/intro_CPU_ISA_ES}}
\ITA{\input{patterns/intro_CPU_ISA_ITA}}
\PTBR{\input{patterns/intro_CPU_ISA_PTBR}}
\RU{\input{patterns/intro_CPU_ISA_RU}}
\DE{\input{patterns/intro_CPU_ISA_DE}}
\FR{\input{patterns/intro_CPU_ISA_FR}}
\PL{\input{patterns/intro_CPU_ISA_PL}}

\EN{\input{patterns/numeral_EN}}
\RU{\input{patterns/numeral_RU}}
\ITA{\input{patterns/numeral_ITA}}
\DE{\input{patterns/numeral_DE}}
\FR{\input{patterns/numeral_FR}}
\PL{\input{patterns/numeral_PL}}

% chapters
\input{patterns/00_empty/main}
\input{patterns/011_ret/main}
\input{patterns/01_helloworld/main}
\input{patterns/015_prolog_epilogue/main}
\input{patterns/02_stack/main}
\input{patterns/03_printf/main}
\input{patterns/04_scanf/main}
\input{patterns/05_passing_arguments/main}
\input{patterns/06_return_results/main}
\input{patterns/061_pointers/main}
\input{patterns/065_GOTO/main}
\input{patterns/07_jcc/main}
\input{patterns/08_switch/main}
\input{patterns/09_loops/main}
\input{patterns/10_strings/main}
\input{patterns/11_arith_optimizations/main}
\input{patterns/12_FPU/main}
\input{patterns/13_arrays/main}
\input{patterns/14_bitfields/main}
\EN{\input{patterns/145_LCG/main_EN}}
\RU{\input{patterns/145_LCG/main_RU}}
\input{patterns/15_structs/main}
\input{patterns/17_unions/main}
\input{patterns/18_pointers_to_functions/main}
\input{patterns/185_64bit_in_32_env/main}

\EN{\input{patterns/19_SIMD/main_EN}}
\RU{\input{patterns/19_SIMD/main_RU}}
\DE{\input{patterns/19_SIMD/main_DE}}

\EN{\input{patterns/20_x64/main_EN}}
\RU{\input{patterns/20_x64/main_RU}}

\EN{\input{patterns/205_floating_SIMD/main_EN}}
\RU{\input{patterns/205_floating_SIMD/main_RU}}
\DE{\input{patterns/205_floating_SIMD/main_DE}}

\EN{\input{patterns/ARM/main_EN}}
\RU{\input{patterns/ARM/main_RU}}
\DE{\input{patterns/ARM/main_DE}}

\input{patterns/MIPS/main}

\ifdefined\SPANISH
\chapter{Patrones de código}
\fi % SPANISH

\ifdefined\GERMAN
\chapter{Code-Muster}
\fi % GERMAN

\ifdefined\ENGLISH
\chapter{Code Patterns}
\fi % ENGLISH

\ifdefined\ITALIAN
\chapter{Forme di codice}
\fi % ITALIAN

\ifdefined\RUSSIAN
\chapter{Образцы кода}
\fi % RUSSIAN

\ifdefined\BRAZILIAN
\chapter{Padrões de códigos}
\fi % BRAZILIAN

\ifdefined\THAI
\chapter{รูปแบบของโค้ด}
\fi % THAI

\ifdefined\FRENCH
\chapter{Modèle de code}
\fi % FRENCH

\ifdefined\POLISH
\chapter{\PLph{}}
\fi % POLISH

% sections
\EN{\input{patterns/patterns_opt_dbg_EN}}
\ES{\input{patterns/patterns_opt_dbg_ES}}
\ITA{\input{patterns/patterns_opt_dbg_ITA}}
\PTBR{\input{patterns/patterns_opt_dbg_PTBR}}
\RU{\input{patterns/patterns_opt_dbg_RU}}
\THA{\input{patterns/patterns_opt_dbg_THA}}
\DE{\input{patterns/patterns_opt_dbg_DE}}
\FR{\input{patterns/patterns_opt_dbg_FR}}
\PL{\input{patterns/patterns_opt_dbg_PL}}

\RU{\section{Некоторые базовые понятия}}
\EN{\section{Some basics}}
\DE{\section{Einige Grundlagen}}
\FR{\section{Quelques bases}}
\ES{\section{\ESph{}}}
\ITA{\section{Alcune basi teoriche}}
\PTBR{\section{\PTBRph{}}}
\THA{\section{\THAph{}}}
\PL{\section{\PLph{}}}

% sections:
\EN{\input{patterns/intro_CPU_ISA_EN}}
\ES{\input{patterns/intro_CPU_ISA_ES}}
\ITA{\input{patterns/intro_CPU_ISA_ITA}}
\PTBR{\input{patterns/intro_CPU_ISA_PTBR}}
\RU{\input{patterns/intro_CPU_ISA_RU}}
\DE{\input{patterns/intro_CPU_ISA_DE}}
\FR{\input{patterns/intro_CPU_ISA_FR}}
\PL{\input{patterns/intro_CPU_ISA_PL}}

\EN{\input{patterns/numeral_EN}}
\RU{\input{patterns/numeral_RU}}
\ITA{\input{patterns/numeral_ITA}}
\DE{\input{patterns/numeral_DE}}
\FR{\input{patterns/numeral_FR}}
\PL{\input{patterns/numeral_PL}}

% chapters
\input{patterns/00_empty/main}
\input{patterns/011_ret/main}
\input{patterns/01_helloworld/main}
\input{patterns/015_prolog_epilogue/main}
\input{patterns/02_stack/main}
\input{patterns/03_printf/main}
\input{patterns/04_scanf/main}
\input{patterns/05_passing_arguments/main}
\input{patterns/06_return_results/main}
\input{patterns/061_pointers/main}
\input{patterns/065_GOTO/main}
\input{patterns/07_jcc/main}
\input{patterns/08_switch/main}
\input{patterns/09_loops/main}
\input{patterns/10_strings/main}
\input{patterns/11_arith_optimizations/main}
\input{patterns/12_FPU/main}
\input{patterns/13_arrays/main}
\input{patterns/14_bitfields/main}
\EN{\input{patterns/145_LCG/main_EN}}
\RU{\input{patterns/145_LCG/main_RU}}
\input{patterns/15_structs/main}
\input{patterns/17_unions/main}
\input{patterns/18_pointers_to_functions/main}
\input{patterns/185_64bit_in_32_env/main}

\EN{\input{patterns/19_SIMD/main_EN}}
\RU{\input{patterns/19_SIMD/main_RU}}
\DE{\input{patterns/19_SIMD/main_DE}}

\EN{\input{patterns/20_x64/main_EN}}
\RU{\input{patterns/20_x64/main_RU}}

\EN{\input{patterns/205_floating_SIMD/main_EN}}
\RU{\input{patterns/205_floating_SIMD/main_RU}}
\DE{\input{patterns/205_floating_SIMD/main_DE}}

\EN{\input{patterns/ARM/main_EN}}
\RU{\input{patterns/ARM/main_RU}}
\DE{\input{patterns/ARM/main_DE}}

\input{patterns/MIPS/main}

\ifdefined\SPANISH
\chapter{Patrones de código}
\fi % SPANISH

\ifdefined\GERMAN
\chapter{Code-Muster}
\fi % GERMAN

\ifdefined\ENGLISH
\chapter{Code Patterns}
\fi % ENGLISH

\ifdefined\ITALIAN
\chapter{Forme di codice}
\fi % ITALIAN

\ifdefined\RUSSIAN
\chapter{Образцы кода}
\fi % RUSSIAN

\ifdefined\BRAZILIAN
\chapter{Padrões de códigos}
\fi % BRAZILIAN

\ifdefined\THAI
\chapter{รูปแบบของโค้ด}
\fi % THAI

\ifdefined\FRENCH
\chapter{Modèle de code}
\fi % FRENCH

\ifdefined\POLISH
\chapter{\PLph{}}
\fi % POLISH

% sections
\EN{\input{patterns/patterns_opt_dbg_EN}}
\ES{\input{patterns/patterns_opt_dbg_ES}}
\ITA{\input{patterns/patterns_opt_dbg_ITA}}
\PTBR{\input{patterns/patterns_opt_dbg_PTBR}}
\RU{\input{patterns/patterns_opt_dbg_RU}}
\THA{\input{patterns/patterns_opt_dbg_THA}}
\DE{\input{patterns/patterns_opt_dbg_DE}}
\FR{\input{patterns/patterns_opt_dbg_FR}}
\PL{\input{patterns/patterns_opt_dbg_PL}}

\RU{\section{Некоторые базовые понятия}}
\EN{\section{Some basics}}
\DE{\section{Einige Grundlagen}}
\FR{\section{Quelques bases}}
\ES{\section{\ESph{}}}
\ITA{\section{Alcune basi teoriche}}
\PTBR{\section{\PTBRph{}}}
\THA{\section{\THAph{}}}
\PL{\section{\PLph{}}}

% sections:
\EN{\input{patterns/intro_CPU_ISA_EN}}
\ES{\input{patterns/intro_CPU_ISA_ES}}
\ITA{\input{patterns/intro_CPU_ISA_ITA}}
\PTBR{\input{patterns/intro_CPU_ISA_PTBR}}
\RU{\input{patterns/intro_CPU_ISA_RU}}
\DE{\input{patterns/intro_CPU_ISA_DE}}
\FR{\input{patterns/intro_CPU_ISA_FR}}
\PL{\input{patterns/intro_CPU_ISA_PL}}

\EN{\input{patterns/numeral_EN}}
\RU{\input{patterns/numeral_RU}}
\ITA{\input{patterns/numeral_ITA}}
\DE{\input{patterns/numeral_DE}}
\FR{\input{patterns/numeral_FR}}
\PL{\input{patterns/numeral_PL}}

% chapters
\input{patterns/00_empty/main}
\input{patterns/011_ret/main}
\input{patterns/01_helloworld/main}
\input{patterns/015_prolog_epilogue/main}
\input{patterns/02_stack/main}
\input{patterns/03_printf/main}
\input{patterns/04_scanf/main}
\input{patterns/05_passing_arguments/main}
\input{patterns/06_return_results/main}
\input{patterns/061_pointers/main}
\input{patterns/065_GOTO/main}
\input{patterns/07_jcc/main}
\input{patterns/08_switch/main}
\input{patterns/09_loops/main}
\input{patterns/10_strings/main}
\input{patterns/11_arith_optimizations/main}
\input{patterns/12_FPU/main}
\input{patterns/13_arrays/main}
\input{patterns/14_bitfields/main}
\EN{\input{patterns/145_LCG/main_EN}}
\RU{\input{patterns/145_LCG/main_RU}}
\input{patterns/15_structs/main}
\input{patterns/17_unions/main}
\input{patterns/18_pointers_to_functions/main}
\input{patterns/185_64bit_in_32_env/main}

\EN{\input{patterns/19_SIMD/main_EN}}
\RU{\input{patterns/19_SIMD/main_RU}}
\DE{\input{patterns/19_SIMD/main_DE}}

\EN{\input{patterns/20_x64/main_EN}}
\RU{\input{patterns/20_x64/main_RU}}

\EN{\input{patterns/205_floating_SIMD/main_EN}}
\RU{\input{patterns/205_floating_SIMD/main_RU}}
\DE{\input{patterns/205_floating_SIMD/main_DE}}

\EN{\input{patterns/ARM/main_EN}}
\RU{\input{patterns/ARM/main_RU}}
\DE{\input{patterns/ARM/main_DE}}

\input{patterns/MIPS/main}


\EN{\section{Returning Values}
\label{ret_val_func}

Another simple function is the one that simply returns a constant value:

\lstinputlisting[caption=\EN{\CCpp Code},style=customc]{patterns/011_ret/1.c}

Let's compile it.

\subsection{x86}

Here's what both the GCC and MSVC compilers produce (with optimization) on the x86 platform:

\lstinputlisting[caption=\Optimizing GCC/MSVC (\assemblyOutput),style=customasmx86]{patterns/011_ret/1.s}

\myindex{x86!\Instructions!RET}
There are just two instructions: the first places the value 123 into the \EAX register,
which is used by convention for storing the return
value, and the second one is \RET, which returns execution to the \gls{caller}.

The caller will take the result from the \EAX register.

\subsection{ARM}

There are a few differences on the ARM platform:

\lstinputlisting[caption=\OptimizingKeilVI (\ARMMode) ASM Output,style=customasmARM]{patterns/011_ret/1_Keil_ARM_O3.s}

ARM uses the register \Reg{0} for returning the results of functions, so 123 is copied into \Reg{0}.

\myindex{ARM!\Instructions!MOV}
\myindex{x86!\Instructions!MOV}
It is worth noting that \MOV is a misleading name for the instruction in both the x86 and ARM \ac{ISA}s.

The data is not in fact \IT{moved}, but \IT{copied}.

\subsection{MIPS}

\label{MIPS_leaf_function_ex1}

The GCC assembly output below lists registers by number:

\lstinputlisting[caption=\Optimizing GCC 4.4.5 (\assemblyOutput),style=customasmMIPS]{patterns/011_ret/MIPS.s}

\dots while \IDA does it by their pseudo names:

\lstinputlisting[caption=\Optimizing GCC 4.4.5 (IDA),style=customasmMIPS]{patterns/011_ret/MIPS_IDA.lst}

The \$2 (or \$V0) register is used to store the function's return value.
\myindex{MIPS!\Pseudoinstructions!LI}
\INS{LI} stands for ``Load Immediate'' and is the MIPS equivalent to \MOV.

\myindex{MIPS!\Instructions!J}
The other instruction is the jump instruction (J or JR) which returns the execution flow to the \gls{caller}.

\myindex{MIPS!Branch delay slot}
You might be wondering why the positions of the load instruction (LI) and the jump instruction (J or JR) are swapped. This is due to a \ac{RISC} feature called ``branch delay slot''.

The reason this happens is a quirk in the architecture of some RISC \ac{ISA}s and isn't important for our
purposes---we must simply keep in mind that in MIPS, the instruction following a jump or branch instruction
is executed \IT{before} the jump/branch instruction itself.

As a consequence, branch instructions always swap places with the instruction executed immediately beforehand.


In practice, functions which merely return 1 (\IT{true}) or 0 (\IT{false}) are very frequent.

The smallest ever of the standard UNIX utilities, \IT{/bin/true} and \IT{/bin/false} return 0 and 1 respectively, as an exit code.
(Zero as an exit code usually means success, non-zero means error.)
}
\RU{\subsubsection{std::string}
\myindex{\Cpp!STL!std::string}
\label{std_string}

\myparagraph{Как устроена структура}

Многие строковые библиотеки \InSqBrackets{\CNotes 2.2} обеспечивают структуру содержащую ссылку 
на буфер собственно со строкой, переменная всегда содержащую длину строки 
(что очень удобно для массы функций \InSqBrackets{\CNotes 2.2.1}) и переменную содержащую текущий размер буфера.

Строка в буфере обыкновенно оканчивается нулем: это для того чтобы указатель на буфер можно было
передавать в функции требующие на вход обычную сишную \ac{ASCIIZ}-строку.

Стандарт \Cpp не описывает, как именно нужно реализовывать std::string,
но, как правило, они реализованы как описано выше, с небольшими дополнениями.

Строки в \Cpp это не класс (как, например, QString в Qt), а темплейт (basic\_string), 
это сделано для того чтобы поддерживать 
строки содержащие разного типа символы: как минимум \Tchar и \IT{wchar\_t}.

Так что, std::string это класс с базовым типом \Tchar.

А std::wstring это класс с базовым типом \IT{wchar\_t}.

\mysubparagraph{MSVC}

В реализации MSVC, вместо ссылки на буфер может содержаться сам буфер (если строка короче 16-и символов).

Это означает, что каждая короткая строка будет занимать в памяти по крайней мере $16 + 4 + 4 = 24$ 
байт для 32-битной среды либо $16 + 8 + 8 = 32$ 
байта в 64-битной, а если строка длиннее 16-и символов, то прибавьте еще длину самой строки.

\lstinputlisting[caption=пример для MSVC,style=customc]{\CURPATH/STL/string/MSVC_RU.cpp}

Собственно, из этого исходника почти всё ясно.

Несколько замечаний:

Если строка короче 16-и символов, 
то отдельный буфер для строки в \glslink{heap}{куче} выделяться не будет.

Это удобно потому что на практике, основная часть строк действительно короткие.
Вероятно, разработчики в Microsoft выбрали размер в 16 символов как разумный баланс.

Теперь очень важный момент в конце функции main(): мы не пользуемся методом c\_str(), тем не менее,
если это скомпилировать и запустить, то обе строки появятся в консоли!

Работает это вот почему.

В первом случае строка короче 16-и символов и в начале объекта std::string (его можно рассматривать
просто как структуру) расположен буфер с этой строкой.
\printf трактует указатель как указатель на массив символов оканчивающийся нулем и поэтому всё работает.

Вывод второй строки (длиннее 16-и символов) даже еще опаснее: это вообще типичная программистская ошибка 
(или опечатка), забыть дописать c\_str().
Это работает потому что в это время в начале структуры расположен указатель на буфер.
Это может надолго остаться незамеченным: до тех пока там не появится строка 
короче 16-и символов, тогда процесс упадет.

\mysubparagraph{GCC}

В реализации GCC в структуре есть еще одна переменная --- reference count.

Интересно, что указатель на экземпляр класса std::string в GCC указывает не на начало самой структуры, 
а на указатель на буфера.
В libstdc++-v3\textbackslash{}include\textbackslash{}bits\textbackslash{}basic\_string.h 
мы можем прочитать что это сделано для удобства отладки:

\begin{lstlisting}
   *  The reason you want _M_data pointing to the character %array and
   *  not the _Rep is so that the debugger can see the string
   *  contents. (Probably we should add a non-inline member to get
   *  the _Rep for the debugger to use, so users can check the actual
   *  string length.)
\end{lstlisting}

\href{http://go.yurichev.com/17085}{исходный код basic\_string.h}

В нашем примере мы учитываем это:

\lstinputlisting[caption=пример для GCC,style=customc]{\CURPATH/STL/string/GCC_RU.cpp}

Нужны еще небольшие хаки чтобы сымитировать типичную ошибку, которую мы уже видели выше, из-за
более ужесточенной проверки типов в GCC, тем не менее, printf() работает и здесь без c\_str().

\myparagraph{Чуть более сложный пример}

\lstinputlisting[style=customc]{\CURPATH/STL/string/3.cpp}

\lstinputlisting[caption=MSVC 2012,style=customasmx86]{\CURPATH/STL/string/3_MSVC_RU.asm}

Собственно, компилятор не конструирует строки статически: да в общем-то и как
это возможно, если буфер с ней нужно хранить в \glslink{heap}{куче}?

Вместо этого в сегменте данных хранятся обычные \ac{ASCIIZ}-строки, а позже, во время выполнения, 
при помощи метода \q{assign}, конструируются строки s1 и s2
.
При помощи \TT{operator+}, создается строка s3.

Обратите внимание на то что вызов метода c\_str() отсутствует,
потому что его код достаточно короткий и компилятор вставил его прямо здесь:
если строка короче 16-и байт, то в регистре EAX остается указатель на буфер,
а если длиннее, то из этого же места достается адрес на буфер расположенный в \glslink{heap}{куче}.

Далее следуют вызовы трех деструкторов, причем, они вызываются только если строка длиннее 16-и байт:
тогда нужно освободить буфера в \glslink{heap}{куче}.
В противном случае, так как все три объекта std::string хранятся в стеке,
они освобождаются автоматически после выхода из функции.

Следовательно, работа с короткими строками более быстрая из-за м\'{е}ньшего обращения к \glslink{heap}{куче}.

Код на GCC даже проще (из-за того, что в GCC, как мы уже видели, не реализована возможность хранить короткую
строку прямо в структуре):

% TODO1 comment each function meaning
\lstinputlisting[caption=GCC 4.8.1,style=customasmx86]{\CURPATH/STL/string/3_GCC_RU.s}

Можно заметить, что в деструкторы передается не указатель на объект,
а указатель на место за 12 байт (или 3 слова) перед ним, то есть, на настоящее начало структуры.

\myparagraph{std::string как глобальная переменная}
\label{sec:std_string_as_global_variable}

Опытные программисты на \Cpp знают, что глобальные переменные \ac{STL}-типов вполне можно объявлять.

Да, действительно:

\lstinputlisting[style=customc]{\CURPATH/STL/string/5.cpp}

Но как и где будет вызываться конструктор \TT{std::string}?

На самом деле, эта переменная будет инициализирована даже перед началом \main.

\lstinputlisting[caption=MSVC 2012: здесь конструируется глобальная переменная{,} а также регистрируется её деструктор,style=customasmx86]{\CURPATH/STL/string/5_MSVC_p2.asm}

\lstinputlisting[caption=MSVC 2012: здесь глобальная переменная используется в \main,style=customasmx86]{\CURPATH/STL/string/5_MSVC_p1.asm}

\lstinputlisting[caption=MSVC 2012: эта функция-деструктор вызывается перед выходом,style=customasmx86]{\CURPATH/STL/string/5_MSVC_p3.asm}

\myindex{\CStandardLibrary!atexit()}
В реальности, из \ac{CRT}, еще до вызова main(), вызывается специальная функция,
в которой перечислены все конструкторы подобных переменных.
Более того: при помощи atexit() регистрируется функция, которая будет вызвана в конце работы программы:
в этой функции компилятор собирает вызовы деструкторов всех подобных глобальных переменных.

GCC работает похожим образом:

\lstinputlisting[caption=GCC 4.8.1,style=customasmx86]{\CURPATH/STL/string/5_GCC.s}

Но он не выделяет отдельной функции в которой будут собраны деструкторы: 
каждый деструктор передается в atexit() по одному.

% TODO а если глобальная STL-переменная в другом модуле? надо проверить.

}
\DE{\subsection{Einfachste XOR-Verschlüsselung überhaupt}

Ich habe einmal eine Software gesehen, bei der alle Debugging-Ausgaben mit XOR mit dem Wert 3
verschlüsselt wurden. Mit anderen Worten, die beiden niedrigsten Bits aller Buchstaben wurden invertiert.

``Hello, world'' wurde zu ``Kfool/\#tlqog'':

\begin{lstlisting}
#!/usr/bin/python

msg="Hello, world!"

print "".join(map(lambda x: chr(ord(x)^3), msg))
\end{lstlisting}

Das ist eine ziemlich interessante Verschlüsselung (oder besser eine Verschleierung),
weil sie zwei wichtige Eigenschaften hat:
1) es ist eine einzige Funktion zum Verschlüsseln und entschlüsseln, sie muss nur wiederholt angewendet werden
2) die entstehenden Buchstaben befinden sich im druckbaren Bereich, also die ganze Zeichenkette kann ohne
Escape-Symbole im Code verwendet werden.

Die zweite Eigenschaft nutzt die Tatsache, dass alle druckbaren Zeichen in Reihen organisiert sind: 0x2x-0x7x,
und wenn die beiden niederwertigsten Bits invertiert werden, wird der Buchstabe um eine oder drei Stellen nach
links oder rechts \IT{verschoben}, aber niemals in eine andere Reihe:

\begin{figure}[H]
\centering
\includegraphics[width=0.7\textwidth]{ascii_clean.png}
\caption{7-Bit \ac{ASCII} Tabelle in Emacs}
\end{figure}

\dots mit dem Zeichen 0x7F als einziger Ausnahme.

Im Folgenden werden also beispielsweise die Zeichen A-Z \IT{verschlüsselt}:

\begin{lstlisting}
#!/usr/bin/python

msg="@ABCDEFGHIJKLMNO"

print "".join(map(lambda x: chr(ord(x)^3), msg))
\end{lstlisting}

Ergebnis:
% FIXME \verb  --  relevant comment for German?
\begin{lstlisting}
CBA@GFEDKJIHONML
\end{lstlisting}

Es sieht so aus als würden die Zeichen ``@'' und ``C'' sowie ``B'' und ``A'' vertauscht werden.

Hier ist noch ein interessantes Beispiel, in dem gezeigt wird, wie die Eigenschaften von XOR
ausgenutzt werden können: Exakt den gleichen Effekt, dass druckbare Zeichen auch druckbar bleiben,
kann man dadurch erzielen, dass irgendeine Kombination der niedrigsten vier Bits invertiert wird.
}

\EN{\section{Returning Values}
\label{ret_val_func}

Another simple function is the one that simply returns a constant value:

\lstinputlisting[caption=\EN{\CCpp Code},style=customc]{patterns/011_ret/1.c}

Let's compile it.

\subsection{x86}

Here's what both the GCC and MSVC compilers produce (with optimization) on the x86 platform:

\lstinputlisting[caption=\Optimizing GCC/MSVC (\assemblyOutput),style=customasmx86]{patterns/011_ret/1.s}

\myindex{x86!\Instructions!RET}
There are just two instructions: the first places the value 123 into the \EAX register,
which is used by convention for storing the return
value, and the second one is \RET, which returns execution to the \gls{caller}.

The caller will take the result from the \EAX register.

\subsection{ARM}

There are a few differences on the ARM platform:

\lstinputlisting[caption=\OptimizingKeilVI (\ARMMode) ASM Output,style=customasmARM]{patterns/011_ret/1_Keil_ARM_O3.s}

ARM uses the register \Reg{0} for returning the results of functions, so 123 is copied into \Reg{0}.

\myindex{ARM!\Instructions!MOV}
\myindex{x86!\Instructions!MOV}
It is worth noting that \MOV is a misleading name for the instruction in both the x86 and ARM \ac{ISA}s.

The data is not in fact \IT{moved}, but \IT{copied}.

\subsection{MIPS}

\label{MIPS_leaf_function_ex1}

The GCC assembly output below lists registers by number:

\lstinputlisting[caption=\Optimizing GCC 4.4.5 (\assemblyOutput),style=customasmMIPS]{patterns/011_ret/MIPS.s}

\dots while \IDA does it by their pseudo names:

\lstinputlisting[caption=\Optimizing GCC 4.4.5 (IDA),style=customasmMIPS]{patterns/011_ret/MIPS_IDA.lst}

The \$2 (or \$V0) register is used to store the function's return value.
\myindex{MIPS!\Pseudoinstructions!LI}
\INS{LI} stands for ``Load Immediate'' and is the MIPS equivalent to \MOV.

\myindex{MIPS!\Instructions!J}
The other instruction is the jump instruction (J or JR) which returns the execution flow to the \gls{caller}.

\myindex{MIPS!Branch delay slot}
You might be wondering why the positions of the load instruction (LI) and the jump instruction (J or JR) are swapped. This is due to a \ac{RISC} feature called ``branch delay slot''.

The reason this happens is a quirk in the architecture of some RISC \ac{ISA}s and isn't important for our
purposes---we must simply keep in mind that in MIPS, the instruction following a jump or branch instruction
is executed \IT{before} the jump/branch instruction itself.

As a consequence, branch instructions always swap places with the instruction executed immediately beforehand.


In practice, functions which merely return 1 (\IT{true}) or 0 (\IT{false}) are very frequent.

The smallest ever of the standard UNIX utilities, \IT{/bin/true} and \IT{/bin/false} return 0 and 1 respectively, as an exit code.
(Zero as an exit code usually means success, non-zero means error.)
}
\RU{\subsubsection{std::string}
\myindex{\Cpp!STL!std::string}
\label{std_string}

\myparagraph{Как устроена структура}

Многие строковые библиотеки \InSqBrackets{\CNotes 2.2} обеспечивают структуру содержащую ссылку 
на буфер собственно со строкой, переменная всегда содержащую длину строки 
(что очень удобно для массы функций \InSqBrackets{\CNotes 2.2.1}) и переменную содержащую текущий размер буфера.

Строка в буфере обыкновенно оканчивается нулем: это для того чтобы указатель на буфер можно было
передавать в функции требующие на вход обычную сишную \ac{ASCIIZ}-строку.

Стандарт \Cpp не описывает, как именно нужно реализовывать std::string,
но, как правило, они реализованы как описано выше, с небольшими дополнениями.

Строки в \Cpp это не класс (как, например, QString в Qt), а темплейт (basic\_string), 
это сделано для того чтобы поддерживать 
строки содержащие разного типа символы: как минимум \Tchar и \IT{wchar\_t}.

Так что, std::string это класс с базовым типом \Tchar.

А std::wstring это класс с базовым типом \IT{wchar\_t}.

\mysubparagraph{MSVC}

В реализации MSVC, вместо ссылки на буфер может содержаться сам буфер (если строка короче 16-и символов).

Это означает, что каждая короткая строка будет занимать в памяти по крайней мере $16 + 4 + 4 = 24$ 
байт для 32-битной среды либо $16 + 8 + 8 = 32$ 
байта в 64-битной, а если строка длиннее 16-и символов, то прибавьте еще длину самой строки.

\lstinputlisting[caption=пример для MSVC,style=customc]{\CURPATH/STL/string/MSVC_RU.cpp}

Собственно, из этого исходника почти всё ясно.

Несколько замечаний:

Если строка короче 16-и символов, 
то отдельный буфер для строки в \glslink{heap}{куче} выделяться не будет.

Это удобно потому что на практике, основная часть строк действительно короткие.
Вероятно, разработчики в Microsoft выбрали размер в 16 символов как разумный баланс.

Теперь очень важный момент в конце функции main(): мы не пользуемся методом c\_str(), тем не менее,
если это скомпилировать и запустить, то обе строки появятся в консоли!

Работает это вот почему.

В первом случае строка короче 16-и символов и в начале объекта std::string (его можно рассматривать
просто как структуру) расположен буфер с этой строкой.
\printf трактует указатель как указатель на массив символов оканчивающийся нулем и поэтому всё работает.

Вывод второй строки (длиннее 16-и символов) даже еще опаснее: это вообще типичная программистская ошибка 
(или опечатка), забыть дописать c\_str().
Это работает потому что в это время в начале структуры расположен указатель на буфер.
Это может надолго остаться незамеченным: до тех пока там не появится строка 
короче 16-и символов, тогда процесс упадет.

\mysubparagraph{GCC}

В реализации GCC в структуре есть еще одна переменная --- reference count.

Интересно, что указатель на экземпляр класса std::string в GCC указывает не на начало самой структуры, 
а на указатель на буфера.
В libstdc++-v3\textbackslash{}include\textbackslash{}bits\textbackslash{}basic\_string.h 
мы можем прочитать что это сделано для удобства отладки:

\begin{lstlisting}
   *  The reason you want _M_data pointing to the character %array and
   *  not the _Rep is so that the debugger can see the string
   *  contents. (Probably we should add a non-inline member to get
   *  the _Rep for the debugger to use, so users can check the actual
   *  string length.)
\end{lstlisting}

\href{http://go.yurichev.com/17085}{исходный код basic\_string.h}

В нашем примере мы учитываем это:

\lstinputlisting[caption=пример для GCC,style=customc]{\CURPATH/STL/string/GCC_RU.cpp}

Нужны еще небольшие хаки чтобы сымитировать типичную ошибку, которую мы уже видели выше, из-за
более ужесточенной проверки типов в GCC, тем не менее, printf() работает и здесь без c\_str().

\myparagraph{Чуть более сложный пример}

\lstinputlisting[style=customc]{\CURPATH/STL/string/3.cpp}

\lstinputlisting[caption=MSVC 2012,style=customasmx86]{\CURPATH/STL/string/3_MSVC_RU.asm}

Собственно, компилятор не конструирует строки статически: да в общем-то и как
это возможно, если буфер с ней нужно хранить в \glslink{heap}{куче}?

Вместо этого в сегменте данных хранятся обычные \ac{ASCIIZ}-строки, а позже, во время выполнения, 
при помощи метода \q{assign}, конструируются строки s1 и s2
.
При помощи \TT{operator+}, создается строка s3.

Обратите внимание на то что вызов метода c\_str() отсутствует,
потому что его код достаточно короткий и компилятор вставил его прямо здесь:
если строка короче 16-и байт, то в регистре EAX остается указатель на буфер,
а если длиннее, то из этого же места достается адрес на буфер расположенный в \glslink{heap}{куче}.

Далее следуют вызовы трех деструкторов, причем, они вызываются только если строка длиннее 16-и байт:
тогда нужно освободить буфера в \glslink{heap}{куче}.
В противном случае, так как все три объекта std::string хранятся в стеке,
они освобождаются автоматически после выхода из функции.

Следовательно, работа с короткими строками более быстрая из-за м\'{е}ньшего обращения к \glslink{heap}{куче}.

Код на GCC даже проще (из-за того, что в GCC, как мы уже видели, не реализована возможность хранить короткую
строку прямо в структуре):

% TODO1 comment each function meaning
\lstinputlisting[caption=GCC 4.8.1,style=customasmx86]{\CURPATH/STL/string/3_GCC_RU.s}

Можно заметить, что в деструкторы передается не указатель на объект,
а указатель на место за 12 байт (или 3 слова) перед ним, то есть, на настоящее начало структуры.

\myparagraph{std::string как глобальная переменная}
\label{sec:std_string_as_global_variable}

Опытные программисты на \Cpp знают, что глобальные переменные \ac{STL}-типов вполне можно объявлять.

Да, действительно:

\lstinputlisting[style=customc]{\CURPATH/STL/string/5.cpp}

Но как и где будет вызываться конструктор \TT{std::string}?

На самом деле, эта переменная будет инициализирована даже перед началом \main.

\lstinputlisting[caption=MSVC 2012: здесь конструируется глобальная переменная{,} а также регистрируется её деструктор,style=customasmx86]{\CURPATH/STL/string/5_MSVC_p2.asm}

\lstinputlisting[caption=MSVC 2012: здесь глобальная переменная используется в \main,style=customasmx86]{\CURPATH/STL/string/5_MSVC_p1.asm}

\lstinputlisting[caption=MSVC 2012: эта функция-деструктор вызывается перед выходом,style=customasmx86]{\CURPATH/STL/string/5_MSVC_p3.asm}

\myindex{\CStandardLibrary!atexit()}
В реальности, из \ac{CRT}, еще до вызова main(), вызывается специальная функция,
в которой перечислены все конструкторы подобных переменных.
Более того: при помощи atexit() регистрируется функция, которая будет вызвана в конце работы программы:
в этой функции компилятор собирает вызовы деструкторов всех подобных глобальных переменных.

GCC работает похожим образом:

\lstinputlisting[caption=GCC 4.8.1,style=customasmx86]{\CURPATH/STL/string/5_GCC.s}

Но он не выделяет отдельной функции в которой будут собраны деструкторы: 
каждый деструктор передается в atexit() по одному.

% TODO а если глобальная STL-переменная в другом модуле? надо проверить.

}

\EN{\section{Returning Values}
\label{ret_val_func}

Another simple function is the one that simply returns a constant value:

\lstinputlisting[caption=\EN{\CCpp Code},style=customc]{patterns/011_ret/1.c}

Let's compile it.

\subsection{x86}

Here's what both the GCC and MSVC compilers produce (with optimization) on the x86 platform:

\lstinputlisting[caption=\Optimizing GCC/MSVC (\assemblyOutput),style=customasmx86]{patterns/011_ret/1.s}

\myindex{x86!\Instructions!RET}
There are just two instructions: the first places the value 123 into the \EAX register,
which is used by convention for storing the return
value, and the second one is \RET, which returns execution to the \gls{caller}.

The caller will take the result from the \EAX register.

\subsection{ARM}

There are a few differences on the ARM platform:

\lstinputlisting[caption=\OptimizingKeilVI (\ARMMode) ASM Output,style=customasmARM]{patterns/011_ret/1_Keil_ARM_O3.s}

ARM uses the register \Reg{0} for returning the results of functions, so 123 is copied into \Reg{0}.

\myindex{ARM!\Instructions!MOV}
\myindex{x86!\Instructions!MOV}
It is worth noting that \MOV is a misleading name for the instruction in both the x86 and ARM \ac{ISA}s.

The data is not in fact \IT{moved}, but \IT{copied}.

\subsection{MIPS}

\label{MIPS_leaf_function_ex1}

The GCC assembly output below lists registers by number:

\lstinputlisting[caption=\Optimizing GCC 4.4.5 (\assemblyOutput),style=customasmMIPS]{patterns/011_ret/MIPS.s}

\dots while \IDA does it by their pseudo names:

\lstinputlisting[caption=\Optimizing GCC 4.4.5 (IDA),style=customasmMIPS]{patterns/011_ret/MIPS_IDA.lst}

The \$2 (or \$V0) register is used to store the function's return value.
\myindex{MIPS!\Pseudoinstructions!LI}
\INS{LI} stands for ``Load Immediate'' and is the MIPS equivalent to \MOV.

\myindex{MIPS!\Instructions!J}
The other instruction is the jump instruction (J or JR) which returns the execution flow to the \gls{caller}.

\myindex{MIPS!Branch delay slot}
You might be wondering why the positions of the load instruction (LI) and the jump instruction (J or JR) are swapped. This is due to a \ac{RISC} feature called ``branch delay slot''.

The reason this happens is a quirk in the architecture of some RISC \ac{ISA}s and isn't important for our
purposes---we must simply keep in mind that in MIPS, the instruction following a jump or branch instruction
is executed \IT{before} the jump/branch instruction itself.

As a consequence, branch instructions always swap places with the instruction executed immediately beforehand.


In practice, functions which merely return 1 (\IT{true}) or 0 (\IT{false}) are very frequent.

The smallest ever of the standard UNIX utilities, \IT{/bin/true} and \IT{/bin/false} return 0 and 1 respectively, as an exit code.
(Zero as an exit code usually means success, non-zero means error.)
}
\RU{\subsubsection{std::string}
\myindex{\Cpp!STL!std::string}
\label{std_string}

\myparagraph{Как устроена структура}

Многие строковые библиотеки \InSqBrackets{\CNotes 2.2} обеспечивают структуру содержащую ссылку 
на буфер собственно со строкой, переменная всегда содержащую длину строки 
(что очень удобно для массы функций \InSqBrackets{\CNotes 2.2.1}) и переменную содержащую текущий размер буфера.

Строка в буфере обыкновенно оканчивается нулем: это для того чтобы указатель на буфер можно было
передавать в функции требующие на вход обычную сишную \ac{ASCIIZ}-строку.

Стандарт \Cpp не описывает, как именно нужно реализовывать std::string,
но, как правило, они реализованы как описано выше, с небольшими дополнениями.

Строки в \Cpp это не класс (как, например, QString в Qt), а темплейт (basic\_string), 
это сделано для того чтобы поддерживать 
строки содержащие разного типа символы: как минимум \Tchar и \IT{wchar\_t}.

Так что, std::string это класс с базовым типом \Tchar.

А std::wstring это класс с базовым типом \IT{wchar\_t}.

\mysubparagraph{MSVC}

В реализации MSVC, вместо ссылки на буфер может содержаться сам буфер (если строка короче 16-и символов).

Это означает, что каждая короткая строка будет занимать в памяти по крайней мере $16 + 4 + 4 = 24$ 
байт для 32-битной среды либо $16 + 8 + 8 = 32$ 
байта в 64-битной, а если строка длиннее 16-и символов, то прибавьте еще длину самой строки.

\lstinputlisting[caption=пример для MSVC,style=customc]{\CURPATH/STL/string/MSVC_RU.cpp}

Собственно, из этого исходника почти всё ясно.

Несколько замечаний:

Если строка короче 16-и символов, 
то отдельный буфер для строки в \glslink{heap}{куче} выделяться не будет.

Это удобно потому что на практике, основная часть строк действительно короткие.
Вероятно, разработчики в Microsoft выбрали размер в 16 символов как разумный баланс.

Теперь очень важный момент в конце функции main(): мы не пользуемся методом c\_str(), тем не менее,
если это скомпилировать и запустить, то обе строки появятся в консоли!

Работает это вот почему.

В первом случае строка короче 16-и символов и в начале объекта std::string (его можно рассматривать
просто как структуру) расположен буфер с этой строкой.
\printf трактует указатель как указатель на массив символов оканчивающийся нулем и поэтому всё работает.

Вывод второй строки (длиннее 16-и символов) даже еще опаснее: это вообще типичная программистская ошибка 
(или опечатка), забыть дописать c\_str().
Это работает потому что в это время в начале структуры расположен указатель на буфер.
Это может надолго остаться незамеченным: до тех пока там не появится строка 
короче 16-и символов, тогда процесс упадет.

\mysubparagraph{GCC}

В реализации GCC в структуре есть еще одна переменная --- reference count.

Интересно, что указатель на экземпляр класса std::string в GCC указывает не на начало самой структуры, 
а на указатель на буфера.
В libstdc++-v3\textbackslash{}include\textbackslash{}bits\textbackslash{}basic\_string.h 
мы можем прочитать что это сделано для удобства отладки:

\begin{lstlisting}
   *  The reason you want _M_data pointing to the character %array and
   *  not the _Rep is so that the debugger can see the string
   *  contents. (Probably we should add a non-inline member to get
   *  the _Rep for the debugger to use, so users can check the actual
   *  string length.)
\end{lstlisting}

\href{http://go.yurichev.com/17085}{исходный код basic\_string.h}

В нашем примере мы учитываем это:

\lstinputlisting[caption=пример для GCC,style=customc]{\CURPATH/STL/string/GCC_RU.cpp}

Нужны еще небольшие хаки чтобы сымитировать типичную ошибку, которую мы уже видели выше, из-за
более ужесточенной проверки типов в GCC, тем не менее, printf() работает и здесь без c\_str().

\myparagraph{Чуть более сложный пример}

\lstinputlisting[style=customc]{\CURPATH/STL/string/3.cpp}

\lstinputlisting[caption=MSVC 2012,style=customasmx86]{\CURPATH/STL/string/3_MSVC_RU.asm}

Собственно, компилятор не конструирует строки статически: да в общем-то и как
это возможно, если буфер с ней нужно хранить в \glslink{heap}{куче}?

Вместо этого в сегменте данных хранятся обычные \ac{ASCIIZ}-строки, а позже, во время выполнения, 
при помощи метода \q{assign}, конструируются строки s1 и s2
.
При помощи \TT{operator+}, создается строка s3.

Обратите внимание на то что вызов метода c\_str() отсутствует,
потому что его код достаточно короткий и компилятор вставил его прямо здесь:
если строка короче 16-и байт, то в регистре EAX остается указатель на буфер,
а если длиннее, то из этого же места достается адрес на буфер расположенный в \glslink{heap}{куче}.

Далее следуют вызовы трех деструкторов, причем, они вызываются только если строка длиннее 16-и байт:
тогда нужно освободить буфера в \glslink{heap}{куче}.
В противном случае, так как все три объекта std::string хранятся в стеке,
они освобождаются автоматически после выхода из функции.

Следовательно, работа с короткими строками более быстрая из-за м\'{е}ньшего обращения к \glslink{heap}{куче}.

Код на GCC даже проще (из-за того, что в GCC, как мы уже видели, не реализована возможность хранить короткую
строку прямо в структуре):

% TODO1 comment each function meaning
\lstinputlisting[caption=GCC 4.8.1,style=customasmx86]{\CURPATH/STL/string/3_GCC_RU.s}

Можно заметить, что в деструкторы передается не указатель на объект,
а указатель на место за 12 байт (или 3 слова) перед ним, то есть, на настоящее начало структуры.

\myparagraph{std::string как глобальная переменная}
\label{sec:std_string_as_global_variable}

Опытные программисты на \Cpp знают, что глобальные переменные \ac{STL}-типов вполне можно объявлять.

Да, действительно:

\lstinputlisting[style=customc]{\CURPATH/STL/string/5.cpp}

Но как и где будет вызываться конструктор \TT{std::string}?

На самом деле, эта переменная будет инициализирована даже перед началом \main.

\lstinputlisting[caption=MSVC 2012: здесь конструируется глобальная переменная{,} а также регистрируется её деструктор,style=customasmx86]{\CURPATH/STL/string/5_MSVC_p2.asm}

\lstinputlisting[caption=MSVC 2012: здесь глобальная переменная используется в \main,style=customasmx86]{\CURPATH/STL/string/5_MSVC_p1.asm}

\lstinputlisting[caption=MSVC 2012: эта функция-деструктор вызывается перед выходом,style=customasmx86]{\CURPATH/STL/string/5_MSVC_p3.asm}

\myindex{\CStandardLibrary!atexit()}
В реальности, из \ac{CRT}, еще до вызова main(), вызывается специальная функция,
в которой перечислены все конструкторы подобных переменных.
Более того: при помощи atexit() регистрируется функция, которая будет вызвана в конце работы программы:
в этой функции компилятор собирает вызовы деструкторов всех подобных глобальных переменных.

GCC работает похожим образом:

\lstinputlisting[caption=GCC 4.8.1,style=customasmx86]{\CURPATH/STL/string/5_GCC.s}

Но он не выделяет отдельной функции в которой будут собраны деструкторы: 
каждый деструктор передается в atexit() по одному.

% TODO а если глобальная STL-переменная в другом модуле? надо проверить.

}
\DE{\subsection{Einfachste XOR-Verschlüsselung überhaupt}

Ich habe einmal eine Software gesehen, bei der alle Debugging-Ausgaben mit XOR mit dem Wert 3
verschlüsselt wurden. Mit anderen Worten, die beiden niedrigsten Bits aller Buchstaben wurden invertiert.

``Hello, world'' wurde zu ``Kfool/\#tlqog'':

\begin{lstlisting}
#!/usr/bin/python

msg="Hello, world!"

print "".join(map(lambda x: chr(ord(x)^3), msg))
\end{lstlisting}

Das ist eine ziemlich interessante Verschlüsselung (oder besser eine Verschleierung),
weil sie zwei wichtige Eigenschaften hat:
1) es ist eine einzige Funktion zum Verschlüsseln und entschlüsseln, sie muss nur wiederholt angewendet werden
2) die entstehenden Buchstaben befinden sich im druckbaren Bereich, also die ganze Zeichenkette kann ohne
Escape-Symbole im Code verwendet werden.

Die zweite Eigenschaft nutzt die Tatsache, dass alle druckbaren Zeichen in Reihen organisiert sind: 0x2x-0x7x,
und wenn die beiden niederwertigsten Bits invertiert werden, wird der Buchstabe um eine oder drei Stellen nach
links oder rechts \IT{verschoben}, aber niemals in eine andere Reihe:

\begin{figure}[H]
\centering
\includegraphics[width=0.7\textwidth]{ascii_clean.png}
\caption{7-Bit \ac{ASCII} Tabelle in Emacs}
\end{figure}

\dots mit dem Zeichen 0x7F als einziger Ausnahme.

Im Folgenden werden also beispielsweise die Zeichen A-Z \IT{verschlüsselt}:

\begin{lstlisting}
#!/usr/bin/python

msg="@ABCDEFGHIJKLMNO"

print "".join(map(lambda x: chr(ord(x)^3), msg))
\end{lstlisting}

Ergebnis:
% FIXME \verb  --  relevant comment for German?
\begin{lstlisting}
CBA@GFEDKJIHONML
\end{lstlisting}

Es sieht so aus als würden die Zeichen ``@'' und ``C'' sowie ``B'' und ``A'' vertauscht werden.

Hier ist noch ein interessantes Beispiel, in dem gezeigt wird, wie die Eigenschaften von XOR
ausgenutzt werden können: Exakt den gleichen Effekt, dass druckbare Zeichen auch druckbar bleiben,
kann man dadurch erzielen, dass irgendeine Kombination der niedrigsten vier Bits invertiert wird.
}

\EN{\section{Returning Values}
\label{ret_val_func}

Another simple function is the one that simply returns a constant value:

\lstinputlisting[caption=\EN{\CCpp Code},style=customc]{patterns/011_ret/1.c}

Let's compile it.

\subsection{x86}

Here's what both the GCC and MSVC compilers produce (with optimization) on the x86 platform:

\lstinputlisting[caption=\Optimizing GCC/MSVC (\assemblyOutput),style=customasmx86]{patterns/011_ret/1.s}

\myindex{x86!\Instructions!RET}
There are just two instructions: the first places the value 123 into the \EAX register,
which is used by convention for storing the return
value, and the second one is \RET, which returns execution to the \gls{caller}.

The caller will take the result from the \EAX register.

\subsection{ARM}

There are a few differences on the ARM platform:

\lstinputlisting[caption=\OptimizingKeilVI (\ARMMode) ASM Output,style=customasmARM]{patterns/011_ret/1_Keil_ARM_O3.s}

ARM uses the register \Reg{0} for returning the results of functions, so 123 is copied into \Reg{0}.

\myindex{ARM!\Instructions!MOV}
\myindex{x86!\Instructions!MOV}
It is worth noting that \MOV is a misleading name for the instruction in both the x86 and ARM \ac{ISA}s.

The data is not in fact \IT{moved}, but \IT{copied}.

\subsection{MIPS}

\label{MIPS_leaf_function_ex1}

The GCC assembly output below lists registers by number:

\lstinputlisting[caption=\Optimizing GCC 4.4.5 (\assemblyOutput),style=customasmMIPS]{patterns/011_ret/MIPS.s}

\dots while \IDA does it by their pseudo names:

\lstinputlisting[caption=\Optimizing GCC 4.4.5 (IDA),style=customasmMIPS]{patterns/011_ret/MIPS_IDA.lst}

The \$2 (or \$V0) register is used to store the function's return value.
\myindex{MIPS!\Pseudoinstructions!LI}
\INS{LI} stands for ``Load Immediate'' and is the MIPS equivalent to \MOV.

\myindex{MIPS!\Instructions!J}
The other instruction is the jump instruction (J or JR) which returns the execution flow to the \gls{caller}.

\myindex{MIPS!Branch delay slot}
You might be wondering why the positions of the load instruction (LI) and the jump instruction (J or JR) are swapped. This is due to a \ac{RISC} feature called ``branch delay slot''.

The reason this happens is a quirk in the architecture of some RISC \ac{ISA}s and isn't important for our
purposes---we must simply keep in mind that in MIPS, the instruction following a jump or branch instruction
is executed \IT{before} the jump/branch instruction itself.

As a consequence, branch instructions always swap places with the instruction executed immediately beforehand.


In practice, functions which merely return 1 (\IT{true}) or 0 (\IT{false}) are very frequent.

The smallest ever of the standard UNIX utilities, \IT{/bin/true} and \IT{/bin/false} return 0 and 1 respectively, as an exit code.
(Zero as an exit code usually means success, non-zero means error.)
}
\RU{\subsubsection{std::string}
\myindex{\Cpp!STL!std::string}
\label{std_string}

\myparagraph{Как устроена структура}

Многие строковые библиотеки \InSqBrackets{\CNotes 2.2} обеспечивают структуру содержащую ссылку 
на буфер собственно со строкой, переменная всегда содержащую длину строки 
(что очень удобно для массы функций \InSqBrackets{\CNotes 2.2.1}) и переменную содержащую текущий размер буфера.

Строка в буфере обыкновенно оканчивается нулем: это для того чтобы указатель на буфер можно было
передавать в функции требующие на вход обычную сишную \ac{ASCIIZ}-строку.

Стандарт \Cpp не описывает, как именно нужно реализовывать std::string,
но, как правило, они реализованы как описано выше, с небольшими дополнениями.

Строки в \Cpp это не класс (как, например, QString в Qt), а темплейт (basic\_string), 
это сделано для того чтобы поддерживать 
строки содержащие разного типа символы: как минимум \Tchar и \IT{wchar\_t}.

Так что, std::string это класс с базовым типом \Tchar.

А std::wstring это класс с базовым типом \IT{wchar\_t}.

\mysubparagraph{MSVC}

В реализации MSVC, вместо ссылки на буфер может содержаться сам буфер (если строка короче 16-и символов).

Это означает, что каждая короткая строка будет занимать в памяти по крайней мере $16 + 4 + 4 = 24$ 
байт для 32-битной среды либо $16 + 8 + 8 = 32$ 
байта в 64-битной, а если строка длиннее 16-и символов, то прибавьте еще длину самой строки.

\lstinputlisting[caption=пример для MSVC,style=customc]{\CURPATH/STL/string/MSVC_RU.cpp}

Собственно, из этого исходника почти всё ясно.

Несколько замечаний:

Если строка короче 16-и символов, 
то отдельный буфер для строки в \glslink{heap}{куче} выделяться не будет.

Это удобно потому что на практике, основная часть строк действительно короткие.
Вероятно, разработчики в Microsoft выбрали размер в 16 символов как разумный баланс.

Теперь очень важный момент в конце функции main(): мы не пользуемся методом c\_str(), тем не менее,
если это скомпилировать и запустить, то обе строки появятся в консоли!

Работает это вот почему.

В первом случае строка короче 16-и символов и в начале объекта std::string (его можно рассматривать
просто как структуру) расположен буфер с этой строкой.
\printf трактует указатель как указатель на массив символов оканчивающийся нулем и поэтому всё работает.

Вывод второй строки (длиннее 16-и символов) даже еще опаснее: это вообще типичная программистская ошибка 
(или опечатка), забыть дописать c\_str().
Это работает потому что в это время в начале структуры расположен указатель на буфер.
Это может надолго остаться незамеченным: до тех пока там не появится строка 
короче 16-и символов, тогда процесс упадет.

\mysubparagraph{GCC}

В реализации GCC в структуре есть еще одна переменная --- reference count.

Интересно, что указатель на экземпляр класса std::string в GCC указывает не на начало самой структуры, 
а на указатель на буфера.
В libstdc++-v3\textbackslash{}include\textbackslash{}bits\textbackslash{}basic\_string.h 
мы можем прочитать что это сделано для удобства отладки:

\begin{lstlisting}
   *  The reason you want _M_data pointing to the character %array and
   *  not the _Rep is so that the debugger can see the string
   *  contents. (Probably we should add a non-inline member to get
   *  the _Rep for the debugger to use, so users can check the actual
   *  string length.)
\end{lstlisting}

\href{http://go.yurichev.com/17085}{исходный код basic\_string.h}

В нашем примере мы учитываем это:

\lstinputlisting[caption=пример для GCC,style=customc]{\CURPATH/STL/string/GCC_RU.cpp}

Нужны еще небольшие хаки чтобы сымитировать типичную ошибку, которую мы уже видели выше, из-за
более ужесточенной проверки типов в GCC, тем не менее, printf() работает и здесь без c\_str().

\myparagraph{Чуть более сложный пример}

\lstinputlisting[style=customc]{\CURPATH/STL/string/3.cpp}

\lstinputlisting[caption=MSVC 2012,style=customasmx86]{\CURPATH/STL/string/3_MSVC_RU.asm}

Собственно, компилятор не конструирует строки статически: да в общем-то и как
это возможно, если буфер с ней нужно хранить в \glslink{heap}{куче}?

Вместо этого в сегменте данных хранятся обычные \ac{ASCIIZ}-строки, а позже, во время выполнения, 
при помощи метода \q{assign}, конструируются строки s1 и s2
.
При помощи \TT{operator+}, создается строка s3.

Обратите внимание на то что вызов метода c\_str() отсутствует,
потому что его код достаточно короткий и компилятор вставил его прямо здесь:
если строка короче 16-и байт, то в регистре EAX остается указатель на буфер,
а если длиннее, то из этого же места достается адрес на буфер расположенный в \glslink{heap}{куче}.

Далее следуют вызовы трех деструкторов, причем, они вызываются только если строка длиннее 16-и байт:
тогда нужно освободить буфера в \glslink{heap}{куче}.
В противном случае, так как все три объекта std::string хранятся в стеке,
они освобождаются автоматически после выхода из функции.

Следовательно, работа с короткими строками более быстрая из-за м\'{е}ньшего обращения к \glslink{heap}{куче}.

Код на GCC даже проще (из-за того, что в GCC, как мы уже видели, не реализована возможность хранить короткую
строку прямо в структуре):

% TODO1 comment each function meaning
\lstinputlisting[caption=GCC 4.8.1,style=customasmx86]{\CURPATH/STL/string/3_GCC_RU.s}

Можно заметить, что в деструкторы передается не указатель на объект,
а указатель на место за 12 байт (или 3 слова) перед ним, то есть, на настоящее начало структуры.

\myparagraph{std::string как глобальная переменная}
\label{sec:std_string_as_global_variable}

Опытные программисты на \Cpp знают, что глобальные переменные \ac{STL}-типов вполне можно объявлять.

Да, действительно:

\lstinputlisting[style=customc]{\CURPATH/STL/string/5.cpp}

Но как и где будет вызываться конструктор \TT{std::string}?

На самом деле, эта переменная будет инициализирована даже перед началом \main.

\lstinputlisting[caption=MSVC 2012: здесь конструируется глобальная переменная{,} а также регистрируется её деструктор,style=customasmx86]{\CURPATH/STL/string/5_MSVC_p2.asm}

\lstinputlisting[caption=MSVC 2012: здесь глобальная переменная используется в \main,style=customasmx86]{\CURPATH/STL/string/5_MSVC_p1.asm}

\lstinputlisting[caption=MSVC 2012: эта функция-деструктор вызывается перед выходом,style=customasmx86]{\CURPATH/STL/string/5_MSVC_p3.asm}

\myindex{\CStandardLibrary!atexit()}
В реальности, из \ac{CRT}, еще до вызова main(), вызывается специальная функция,
в которой перечислены все конструкторы подобных переменных.
Более того: при помощи atexit() регистрируется функция, которая будет вызвана в конце работы программы:
в этой функции компилятор собирает вызовы деструкторов всех подобных глобальных переменных.

GCC работает похожим образом:

\lstinputlisting[caption=GCC 4.8.1,style=customasmx86]{\CURPATH/STL/string/5_GCC.s}

Но он не выделяет отдельной функции в которой будут собраны деструкторы: 
каждый деструктор передается в atexit() по одному.

% TODO а если глобальная STL-переменная в другом модуле? надо проверить.

}
\DE{\subsection{Einfachste XOR-Verschlüsselung überhaupt}

Ich habe einmal eine Software gesehen, bei der alle Debugging-Ausgaben mit XOR mit dem Wert 3
verschlüsselt wurden. Mit anderen Worten, die beiden niedrigsten Bits aller Buchstaben wurden invertiert.

``Hello, world'' wurde zu ``Kfool/\#tlqog'':

\begin{lstlisting}
#!/usr/bin/python

msg="Hello, world!"

print "".join(map(lambda x: chr(ord(x)^3), msg))
\end{lstlisting}

Das ist eine ziemlich interessante Verschlüsselung (oder besser eine Verschleierung),
weil sie zwei wichtige Eigenschaften hat:
1) es ist eine einzige Funktion zum Verschlüsseln und entschlüsseln, sie muss nur wiederholt angewendet werden
2) die entstehenden Buchstaben befinden sich im druckbaren Bereich, also die ganze Zeichenkette kann ohne
Escape-Symbole im Code verwendet werden.

Die zweite Eigenschaft nutzt die Tatsache, dass alle druckbaren Zeichen in Reihen organisiert sind: 0x2x-0x7x,
und wenn die beiden niederwertigsten Bits invertiert werden, wird der Buchstabe um eine oder drei Stellen nach
links oder rechts \IT{verschoben}, aber niemals in eine andere Reihe:

\begin{figure}[H]
\centering
\includegraphics[width=0.7\textwidth]{ascii_clean.png}
\caption{7-Bit \ac{ASCII} Tabelle in Emacs}
\end{figure}

\dots mit dem Zeichen 0x7F als einziger Ausnahme.

Im Folgenden werden also beispielsweise die Zeichen A-Z \IT{verschlüsselt}:

\begin{lstlisting}
#!/usr/bin/python

msg="@ABCDEFGHIJKLMNO"

print "".join(map(lambda x: chr(ord(x)^3), msg))
\end{lstlisting}

Ergebnis:
% FIXME \verb  --  relevant comment for German?
\begin{lstlisting}
CBA@GFEDKJIHONML
\end{lstlisting}

Es sieht so aus als würden die Zeichen ``@'' und ``C'' sowie ``B'' und ``A'' vertauscht werden.

Hier ist noch ein interessantes Beispiel, in dem gezeigt wird, wie die Eigenschaften von XOR
ausgenutzt werden können: Exakt den gleichen Effekt, dass druckbare Zeichen auch druckbar bleiben,
kann man dadurch erzielen, dass irgendeine Kombination der niedrigsten vier Bits invertiert wird.
}

\ifdefined\SPANISH
\chapter{Patrones de código}
\fi % SPANISH

\ifdefined\GERMAN
\chapter{Code-Muster}
\fi % GERMAN

\ifdefined\ENGLISH
\chapter{Code Patterns}
\fi % ENGLISH

\ifdefined\ITALIAN
\chapter{Forme di codice}
\fi % ITALIAN

\ifdefined\RUSSIAN
\chapter{Образцы кода}
\fi % RUSSIAN

\ifdefined\BRAZILIAN
\chapter{Padrões de códigos}
\fi % BRAZILIAN

\ifdefined\THAI
\chapter{รูปแบบของโค้ด}
\fi % THAI

\ifdefined\FRENCH
\chapter{Modèle de code}
\fi % FRENCH

\ifdefined\POLISH
\chapter{\PLph{}}
\fi % POLISH

% sections
\EN{\input{patterns/patterns_opt_dbg_EN}}
\ES{\input{patterns/patterns_opt_dbg_ES}}
\ITA{\input{patterns/patterns_opt_dbg_ITA}}
\PTBR{\input{patterns/patterns_opt_dbg_PTBR}}
\RU{\input{patterns/patterns_opt_dbg_RU}}
\THA{\input{patterns/patterns_opt_dbg_THA}}
\DE{\input{patterns/patterns_opt_dbg_DE}}
\FR{\input{patterns/patterns_opt_dbg_FR}}
\PL{\input{patterns/patterns_opt_dbg_PL}}

\RU{\section{Некоторые базовые понятия}}
\EN{\section{Some basics}}
\DE{\section{Einige Grundlagen}}
\FR{\section{Quelques bases}}
\ES{\section{\ESph{}}}
\ITA{\section{Alcune basi teoriche}}
\PTBR{\section{\PTBRph{}}}
\THA{\section{\THAph{}}}
\PL{\section{\PLph{}}}

% sections:
\EN{\input{patterns/intro_CPU_ISA_EN}}
\ES{\input{patterns/intro_CPU_ISA_ES}}
\ITA{\input{patterns/intro_CPU_ISA_ITA}}
\PTBR{\input{patterns/intro_CPU_ISA_PTBR}}
\RU{\input{patterns/intro_CPU_ISA_RU}}
\DE{\input{patterns/intro_CPU_ISA_DE}}
\FR{\input{patterns/intro_CPU_ISA_FR}}
\PL{\input{patterns/intro_CPU_ISA_PL}}

\EN{\input{patterns/numeral_EN}}
\RU{\input{patterns/numeral_RU}}
\ITA{\input{patterns/numeral_ITA}}
\DE{\input{patterns/numeral_DE}}
\FR{\input{patterns/numeral_FR}}
\PL{\input{patterns/numeral_PL}}

% chapters
\input{patterns/00_empty/main}
\input{patterns/011_ret/main}
\input{patterns/01_helloworld/main}
\input{patterns/015_prolog_epilogue/main}
\input{patterns/02_stack/main}
\input{patterns/03_printf/main}
\input{patterns/04_scanf/main}
\input{patterns/05_passing_arguments/main}
\input{patterns/06_return_results/main}
\input{patterns/061_pointers/main}
\input{patterns/065_GOTO/main}
\input{patterns/07_jcc/main}
\input{patterns/08_switch/main}
\input{patterns/09_loops/main}
\input{patterns/10_strings/main}
\input{patterns/11_arith_optimizations/main}
\input{patterns/12_FPU/main}
\input{patterns/13_arrays/main}
\input{patterns/14_bitfields/main}
\EN{\input{patterns/145_LCG/main_EN}}
\RU{\input{patterns/145_LCG/main_RU}}
\input{patterns/15_structs/main}
\input{patterns/17_unions/main}
\input{patterns/18_pointers_to_functions/main}
\input{patterns/185_64bit_in_32_env/main}

\EN{\input{patterns/19_SIMD/main_EN}}
\RU{\input{patterns/19_SIMD/main_RU}}
\DE{\input{patterns/19_SIMD/main_DE}}

\EN{\input{patterns/20_x64/main_EN}}
\RU{\input{patterns/20_x64/main_RU}}

\EN{\input{patterns/205_floating_SIMD/main_EN}}
\RU{\input{patterns/205_floating_SIMD/main_RU}}
\DE{\input{patterns/205_floating_SIMD/main_DE}}

\EN{\input{patterns/ARM/main_EN}}
\RU{\input{patterns/ARM/main_RU}}
\DE{\input{patterns/ARM/main_DE}}

\input{patterns/MIPS/main}


\ifdefined\SPANISH
\chapter{Patrones de código}
\fi % SPANISH

\ifdefined\GERMAN
\chapter{Code-Muster}
\fi % GERMAN

\ifdefined\ENGLISH
\chapter{Code Patterns}
\fi % ENGLISH

\ifdefined\ITALIAN
\chapter{Forme di codice}
\fi % ITALIAN

\ifdefined\RUSSIAN
\chapter{Образцы кода}
\fi % RUSSIAN

\ifdefined\BRAZILIAN
\chapter{Padrões de códigos}
\fi % BRAZILIAN

\ifdefined\THAI
\chapter{รูปแบบของโค้ด}
\fi % THAI

\ifdefined\FRENCH
\chapter{Modèle de code}
\fi % FRENCH

\ifdefined\POLISH
\chapter{\PLph{}}
\fi % POLISH

% sections
\EN{\section{The method}

When the author of this book first started learning C and, later, \Cpp, he used to write small pieces of code, compile them,
and then look at the assembly language output. This made it very easy for him to understand what was going on in the code that he had written.
\footnote{In fact, he still does this when he can't understand what a particular bit of code does.}.
He did this so many times that the relationship between the \CCpp code and what the compiler produced was imprinted deeply in his mind.
It's now easy for him to imagine instantly a rough outline of a C code's appearance and function.
Perhaps this technique could be helpful for others.

%There are a lot of examples for both x86/x64 and ARM.
%Those who already familiar with one of architectures, may freely skim over pages.

By the way, there is a great website where you can do the same, with various compilers, instead of installing them on your box.
You can use it as well: \url{https://gcc.godbolt.org/}.

\section*{\Exercises}

When the author of this book studied assembly language, he also often compiled small C functions and then rewrote
them gradually to assembly, trying to make their code as short as possible.
This probably is not worth doing in real-world scenarios today,
because it's hard to compete with the latest compilers in terms of efficiency. It is, however, a very good way to gain a better understanding of assembly.
Feel free, therefore, to take any assembly code from this book and try to make it shorter.
However, don't forget to test what you have written.

% rewrote to show that debug\release and optimisations levels are orthogonal concepts.
\section*{Optimization levels and debug information}

Source code can be compiled by different compilers with various optimization levels.
A typical compiler has about three such levels, where level zero means that optimization is completely disabled.
Optimization can also be targeted towards code size or code speed.
A non-optimizing compiler is faster and produces more understandable (albeit verbose) code,
whereas an optimizing compiler is slower and tries to produce code that runs faster (but is not necessarily more compact).
In addition to optimization levels, a compiler can include some debug information in the resulting file,
producing code that is easy to debug.
One of the important features of the ´debug' code is that it might contain links
between each line of the source code and its respective machine code address.
Optimizing compilers, on the other hand, tend to produce output where entire lines of source code
can be optimized away and thus not even be present in the resulting machine code.
Reverse engineers can encounter either version, simply because some developers turn on the compiler's optimization flags and others do not.
Because of this, we'll try to work on examples of both debug and release versions of the code featured in this book, wherever possible.

Sometimes some pretty ancient compilers are used in this book, in order to get the shortest (or simplest) possible code snippet.
}
\ES{\input{patterns/patterns_opt_dbg_ES}}
\ITA{\input{patterns/patterns_opt_dbg_ITA}}
\PTBR{\input{patterns/patterns_opt_dbg_PTBR}}
\RU{\input{patterns/patterns_opt_dbg_RU}}
\THA{\input{patterns/patterns_opt_dbg_THA}}
\DE{\input{patterns/patterns_opt_dbg_DE}}
\FR{\input{patterns/patterns_opt_dbg_FR}}
\PL{\input{patterns/patterns_opt_dbg_PL}}

\RU{\section{Некоторые базовые понятия}}
\EN{\section{Some basics}}
\DE{\section{Einige Grundlagen}}
\FR{\section{Quelques bases}}
\ES{\section{\ESph{}}}
\ITA{\section{Alcune basi teoriche}}
\PTBR{\section{\PTBRph{}}}
\THA{\section{\THAph{}}}
\PL{\section{\PLph{}}}

% sections:
\EN{\input{patterns/intro_CPU_ISA_EN}}
\ES{\input{patterns/intro_CPU_ISA_ES}}
\ITA{\input{patterns/intro_CPU_ISA_ITA}}
\PTBR{\input{patterns/intro_CPU_ISA_PTBR}}
\RU{\input{patterns/intro_CPU_ISA_RU}}
\DE{\input{patterns/intro_CPU_ISA_DE}}
\FR{\input{patterns/intro_CPU_ISA_FR}}
\PL{\input{patterns/intro_CPU_ISA_PL}}

\EN{\subsection{Numeral Systems}

Humans have become accustomed to a decimal numeral system, probably because almost everyone has 10 fingers.
Nevertheless, the number \q{10} has no significant meaning in science and mathematics.
The natural numeral system in digital electronics is binary: 0 is for an absence of current in the wire, and 1 for presence.
10 in binary is 2 in decimal, 100 in binary is 4 in decimal, and so on.

% This sentence is a bit unweildy - maybe try 'Our ten-digit system would be described as having a radix...' - Renaissance
If the numeral system has 10 digits, it has a \IT{radix} (or \IT{base}) of 10.
The binary numeral system has a \IT{radix} of 2.

Important things to recall:

1) A \IT{number} is a number, while a \IT{digit} is a term from writing systems, and is usually one character

% The original is 'number' is not changed; I think the intent is value, and changed it - Renaissance
2) The value of a number does not change when converted to another radix; only the writing notation for that value has changed (and therefore the way of representing it in \ac{RAM}).

\subsection{Converting From One Radix To Another}

Positional notation is used almost every numerical system. This means that a digit has weight relative to where it is placed inside of the larger number.
If 2 is placed at the rightmost place, it's 2, but if it's placed one digit before rightmost, it's 20.

What does $1234$ stand for?

$10^3 \cdot 1 + 10^2 \cdot 2 + 10^1 \cdot 3 + 1 \cdot 4 = 1234$ or
$1000 \cdot 1 + 100 \cdot 2 + 10 \cdot 3 + 4 = 1234$

It's the same story for binary numbers, but the base is 2 instead of 10.
What does 0b101011 stand for?

$2^5 \cdot 1 + 2^4 \cdot 0 + 2^3 \cdot 1 + 2^2 \cdot 0 + 2^1 \cdot 1 + 2^0 \cdot 1 = 43$ or
$32 \cdot 1 + 16 \cdot 0 + 8 \cdot 1 + 4 \cdot 0 + 2 \cdot 1 + 1 = 43$

There is such a thing as non-positional notation, such as the Roman numeral system.
\footnote{About numeric system evolution, see \InSqBrackets{\TAOCPvolII{}, 195--213.}}.
% Maybe add a sentence to fill in that X is always 10, and is therefore non-positional, even though putting an I before subtracts and after adds, and is in that sense positional
Perhaps, humankind switched to positional notation because it's easier to do basic operations (addition, multiplication, etc.) on paper by hand.

Binary numbers can be added, subtracted and so on in the very same as taught in schools, but only 2 digits are available.

Binary numbers are bulky when represented in source code and dumps, so that is where the hexadecimal numeral system can be useful.
A hexadecimal radix uses the digits 0..9, and also 6 Latin characters: A..F.
Each hexadecimal digit takes 4 bits or 4 binary digits, so it's very easy to convert from binary number to hexadecimal and back, even manually, in one's mind.

\begin{center}
\begin{longtable}{ | l | l | l | }
\hline
\HeaderColor hexadecimal & \HeaderColor binary & \HeaderColor decimal \\
\hline
0	&0000	&0 \\
1	&0001	&1 \\
2	&0010	&2 \\
3	&0011	&3 \\
4	&0100	&4 \\
5	&0101	&5 \\
6	&0110	&6 \\
7	&0111	&7 \\
8	&1000	&8 \\
9	&1001	&9 \\
A	&1010	&10 \\
B	&1011	&11 \\
C	&1100	&12 \\
D	&1101	&13 \\
E	&1110	&14 \\
F	&1111	&15 \\
\hline
\end{longtable}
\end{center}

How can one tell which radix is being used in a specific instance?

Decimal numbers are usually written as is, i.e., 1234. Some assemblers allow an identifier on decimal radix numbers, in which the number would be written with a "d" suffix: 1234d.

Binary numbers are sometimes prepended with the "0b" prefix: 0b100110111 (\ac{GCC} has a non-standard language extension for this\footnote{\url{https://gcc.gnu.org/onlinedocs/gcc/Binary-constants.html}}).
There is also another way: using a "b" suffix, for example: 100110111b.
This book tries to use the "0b" prefix consistently throughout the book for binary numbers.

Hexadecimal numbers are prepended with "0x" prefix in \CCpp and other \ac{PL}s: 0x1234ABCD.
Alternatively, they are given a "h" suffix: 1234ABCDh. This is common way of representing them in assemblers and debuggers.
In this convention, if the number is started with a Latin (A..F) digit, a 0 is added at the beginning: 0ABCDEFh.
There was also convention that was popular in 8-bit home computers era, using \$ prefix, like \$ABCD.
The book will try to stick to "0x" prefix throughout the book for hexadecimal numbers.

Should one learn to convert numbers mentally? A table of 1-digit hexadecimal numbers can easily be memorized.
As for larger numbers, it's probably not worth tormenting yourself.

Perhaps the most visible hexadecimal numbers are in \ac{URL}s.
This is the way that non-Latin characters are encoded.
For example:
\url{https://en.wiktionary.org/wiki/na\%C3\%AFvet\%C3\%A9} is the \ac{URL} of Wiktionary article about \q{naïveté} word.

\subsubsection{Octal Radix}

Another numeral system heavily used in the past of computer programming is octal. In octal there are 8 digits (0..7), and each is mapped to 3 bits, so it's easy to convert numbers back and forth.
It has been superseded by the hexadecimal system almost everywhere, but, surprisingly, there is a *NIX utility, used often by many people, which takes octal numbers as argument: \TT{chmod}.

\myindex{UNIX!chmod}
As many *NIX users know, \TT{chmod} argument can be a number of 3 digits. The first digit represents the rights of the owner of the file (read, write and/or execute), the second is the rights for the group to which the file belongs, and the third is for everyone else.
Each digit that \TT{chmod} takes can be represented in binary form:

\begin{center}
\begin{longtable}{ | l | l | l | }
\hline
\HeaderColor decimal & \HeaderColor binary & \HeaderColor meaning \\
\hline
7	&111	&\textbf{rwx} \\
6	&110	&\textbf{rw-} \\
5	&101	&\textbf{r-x} \\
4	&100	&\textbf{r-{}-} \\
3	&011	&\textbf{-wx} \\
2	&010	&\textbf{-w-} \\
1	&001	&\textbf{-{}-x} \\
0	&000	&\textbf{-{}-{}-} \\
\hline
\end{longtable}
\end{center}

So each bit is mapped to a flag: read/write/execute.

The importance of \TT{chmod} here is that the whole number in argument can be represented as octal number.
Let's take, for example, 644.
When you run \TT{chmod 644 file}, you set read/write permissions for owner, read permissions for group and again, read permissions for everyone else.
If we convert the octal number 644 to binary, it would be \TT{110100100}, or, in groups of 3 bits, \TT{110 100 100}.

Now we see that each triplet describe permissions for owner/group/others: first is \TT{rw-}, second is \TT{r--} and third is \TT{r--}.

The octal numeral system was also popular on old computers like PDP-8, because word there could be 12, 24 or 36 bits, and these numbers are all divisible by 3, so the octal system was natural in that environment.
Nowadays, all popular computers employ word/address sizes of 16, 32 or 64 bits, and these numbers are all divisible by 4, so the hexadecimal system is more natural there.

The octal numeral system is supported by all standard \CCpp compilers.
This is a source of confusion sometimes, because octal numbers are encoded with a zero prepended, for example, 0377 is 255.
Sometimes, you might make a typo and write "09" instead of 9, and the compiler would report an error.
GCC might report something like this:\\
\TT{error: invalid digit "9" in octal constant}.

Also, the octal system is somewhat popular in Java. When the IDA shows Java strings with non-printable characters,
they are encoded in the octal system instead of hexadecimal.
\myindex{JAD}
The JAD Java decompiler behaves the same way.

\subsubsection{Divisibility}

When you see a decimal number like 120, you can quickly deduce that it's divisible by 10, because the last digit is zero.
In the same way, 123400 is divisible by 100, because the two last digits are zeros.

Likewise, the hexadecimal number 0x1230 is divisible by 0x10 (or 16), 0x123000 is divisible by 0x1000 (or 4096), etc.

The binary number 0b1000101000 is divisible by 0b1000 (8), etc.

This property can often be used to quickly realize if the size of some block in memory is padded to some boundary.
For example, sections in \ac{PE} files are almost always started at addresses ending with 3 hexadecimal zeros: 0x41000, 0x10001000, etc.
The reason behind this is the fact that almost all \ac{PE} sections are padded to a boundary of 0x1000 (4096) bytes.

\subsubsection{Multi-Precision Arithmetic and Radix}

\index{RSA}
Multi-precision arithmetic can use huge numbers, and each one may be stored in several bytes.
For example, RSA keys, both public and private, span up to 4096 bits, and maybe even more.

% I'm not sure how to change this, but the normal format for quoting would be just to mention the author or book, and footnote to the full reference
In \InSqBrackets{\TAOCPvolII, 265} we find the following idea: when you store a multi-precision number in several bytes,
the whole number can be represented as having a radix of $2^8=256$, and each digit goes to the corresponding byte.
Likewise, if you store a multi-precision number in several 32-bit integer values, each digit goes to each 32-bit slot,
and you may think about this number as stored in radix of $2^{32}$.

\subsubsection{How to Pronounce Non-Decimal Numbers}

Numbers in a non-decimal base are usually pronounced by digit by digit: ``one-zero-zero-one-one-...''.
Words like ``ten'' and ``thousand'' are usually not pronounced, to prevent confusion with the decimal base system.

\subsubsection{Floating point numbers}

To distinguish floating point numbers from integers, they are usually written with ``.0'' at the end,
like $0.0$, $123.0$, etc.
}
\RU{\subsection{Представление чисел}

Люди привыкли к десятичной системе счисления вероятно потому что почти у каждого есть по 10 пальцев.
Тем не менее, число 10 не имеет особого значения в науке и математике.
Двоичная система естествена для цифровой электроники: 0 означает отсутствие тока в проводе и 1 --- его присутствие.
10 в двоичной системе это 2 в десятичной; 100 в двоичной это 4 в десятичной, итд.

Если в системе счисления есть 10 цифр, её \IT{основание} или \IT{radix} это 10.
Двоичная система имеет \IT{основание} 2.

Важные вещи, которые полезно вспомнить:
1) \IT{число} это число, в то время как \IT{цифра} это термин из системы письменности, и это обычно один символ;
2) само число не меняется, когда конвертируется из одного основания в другое: меняется способ его записи (или представления
в памяти).

Как сконвертировать число из одного основания в другое?

Позиционная нотация используется почти везде, это означает, что всякая цифра имеет свой вес, в зависимости от её расположения
внутри числа.
Если 2 расположена в самом последнем месте справа, это 2.
Если она расположена в месте перед последним, это 20.

Что означает $1234$?

$10^3 \cdot 1 + 10^2 \cdot 2 + 10^1 \cdot 3 + 1 \cdot 4$ = 1234 или
$1000 \cdot 1 + 100 \cdot 2 + 10 \cdot 3 + 4 = 1234$

Та же история и для двоичных чисел, только основание там 2 вместо 10.
Что означает 0b101011?

$2^5 \cdot 1 + 2^4 \cdot 0 + 2^3 \cdot 1 + 2^2 \cdot 0 + 2^1 \cdot 1 + 2^0 \cdot 1 = 43$ или
$32 \cdot 1 + 16 \cdot 0 + 8 \cdot 1 + 4 \cdot 0 + 2 \cdot 1 + 1 = 43$

Позиционную нотацию можно противопоставить непозиционной нотации, такой как римская система записи чисел
\footnote{Об эволюции способов записи чисел, см.также: \InSqBrackets{\TAOCPvolII{}, 195--213.}}.
Вероятно, человечество перешло на позиционную нотацию, потому что так проще работать с числами (сложение, умножение, итд)
на бумаге, в ручную.

Действительно, двоичные числа можно складывать, вычитать, итд, точно также, как этому обычно обучают в школах,
только доступны лишь 2 цифры.

Двоичные числа громоздки, когда их используют в исходных кодах и дампах, так что в этих случаях применяется шестнадцатеричная
система.
Используются цифры 0..9 и еще 6 латинских букв: A..F.
Каждая шестнадцатеричная цифра занимает 4 бита или 4 двоичных цифры, так что конвертировать из двоичной системы в
шестнадцатеричную и назад, можно легко вручную, или даже в уме.

\begin{center}
\begin{longtable}{ | l | l | l | }
\hline
\HeaderColor шестнадцатеричная & \HeaderColor двоичная & \HeaderColor десятичная \\
\hline
0	&0000	&0 \\
1	&0001	&1 \\
2	&0010	&2 \\
3	&0011	&3 \\
4	&0100	&4 \\
5	&0101	&5 \\
6	&0110	&6 \\
7	&0111	&7 \\
8	&1000	&8 \\
9	&1001	&9 \\
A	&1010	&10 \\
B	&1011	&11 \\
C	&1100	&12 \\
D	&1101	&13 \\
E	&1110	&14 \\
F	&1111	&15 \\
\hline
\end{longtable}
\end{center}

Как понять, какое основание используется в конкретном месте?

Десятичные числа обычно записываются как есть, т.е., 1234. Но некоторые ассемблеры позволяют подчеркивать
этот факт для ясности, и это число может быть дополнено суффиксом "d": 1234d.

К двоичным числам иногда спереди добавляют префикс "0b": 0b100110111
(В \ac{GCC} для этого есть нестандартное расширение языка
\footnote{\url{https://gcc.gnu.org/onlinedocs/gcc/Binary-constants.html}}).
Есть также еще один способ: суффикс "b", например: 100110111b.
В этой книге я буду пытаться придерживаться префикса "0b" для двоичных чисел.

Шестнадцатеричные числа имеют префикс "0x" в \CCpp и некоторых других \ac{PL}: 0x1234ABCD.
Либо они имеют суффикс "h": 1234ABCDh --- обычно так они представляются в ассемблерах и отладчиках.
Если число начинается с цифры A..F, перед ним добавляется 0: 0ABCDEFh.
Во времена 8-битных домашних компьютеров, был также способ записи чисел используя префикс \$, например, \$ABCD.
В книге я попытаюсь придерживаться префикса "0x" для шестнадцатеричных чисел.

Нужно ли учиться конвертировать числа в уме? Таблицу шестнадцатеричных чисел из одной цифры легко запомнить.
А запоминать б\'{о}льшие числа, наверное, не стоит.

Наверное, чаще всего шестнадцатеричные числа можно увидеть в \ac{URL}-ах.
Так кодируются буквы не из числа латинских.
Например:
\url{https://en.wiktionary.org/wiki/na\%C3\%AFvet\%C3\%A9} это \ac{URL} страницы в Wiktionary о слове \q{naïveté}.

\subsubsection{Восьмеричная система}

Еще одна система, которая в прошлом много использовалась в программировании это восьмеричная: есть 8 цифр (0..7) и каждая
описывает 3 бита, так что легко конвертировать числа туда и назад.
Она почти везде была заменена шестнадцатеричной, но удивительно, в *NIX имеется утилита использующаяся многими людьми,
которая принимает на вход восьмеричное число: \TT{chmod}.

\myindex{UNIX!chmod}
Как знают многие пользователи *NIX, аргумент \TT{chmod} это число из трех цифр. Первая цифра это права владельца файла,
вторая это права группы (которой файл принадлежит), третья для всех остальных.
И каждая цифра может быть представлена в двоичном виде:

\begin{center}
\begin{longtable}{ | l | l | l | }
\hline
\HeaderColor десятичная & \HeaderColor двоичная & \HeaderColor значение \\
\hline
7	&111	&\textbf{rwx} \\
6	&110	&\textbf{rw-} \\
5	&101	&\textbf{r-x} \\
4	&100	&\textbf{r-{}-} \\
3	&011	&\textbf{-wx} \\
2	&010	&\textbf{-w-} \\
1	&001	&\textbf{-{}-x} \\
0	&000	&\textbf{-{}-{}-} \\
\hline
\end{longtable}
\end{center}

Так что каждый бит привязан к флагу: read/write/execute (чтение/запись/исполнение).

И вот почему я вспомнил здесь о \TT{chmod}, это потому что всё число может быть представлено как число в восьмеричной системе.
Для примера возьмем 644.
Когда вы запускаете \TT{chmod 644 file}, вы выставляете права read/write для владельца, права read для группы, и снова,
read для всех остальных.
Сконвертируем число 644 из восьмеричной системы в двоичную, это будет \TT{110100100}, или (в группах по 3 бита) \TT{110 100 100}.

Теперь мы видим, что каждая тройка описывает права для владельца/группы/остальных:
первая это \TT{rw-}, вторая это \TT{r--} и третья это \TT{r--}.

Восьмеричная система была также популярная на старых компьютерах вроде PDP-8, потому что слово там могло содержать 12, 24 или
36 бит, и эти числа делятся на 3, так что выбор восьмеричной системы в той среде был логичен.
Сейчас, все популярные компьютеры имеют размер слова/адреса 16, 32 или 64 бита, и эти числа делятся на 4,
так что шестнадцатеричная система здесь удобнее.

Восьмеричная система поддерживается всеми стандартными компиляторами \CCpp{}.
Это иногда источник недоумения, потому что восьмеричные числа кодируются с нулем вперед, например, 0377 это 255.
И иногда, вы можете сделать опечатку, и написать "09" вместо 9, и компилятор выдаст ошибку.
GCC может выдать что-то вроде:\\
\TT{error: invalid digit "9" in octal constant}.

Также, восьмеричная система популярна в Java: когда IDA показывает строку с непечатаемыми символами,
они кодируются в восьмеричной системе вместо шестнадцатеричной.
\myindex{JAD}
Точно также себя ведет декомпилятор с Java JAD.

\subsubsection{Делимость}

Когда вы видите десятичное число вроде 120, вы можете быстро понять что оно делится на 10, потому что последняя цифра это 0.
Точно также, 123400 делится на 100, потому что две последних цифры это нули.

Точно также, шестнадцатеричное число 0x1230 делится на 0x10 (или 16), 0x123000 делится на 0x1000 (или 4096), итд.

Двоичное число 0b1000101000 делится на 0b1000 (8), итд.

Это свойство можно часто использовать, чтобы быстро понять,
что длина какого-либо блока в памяти выровнена по некоторой границе.
Например, секции в \ac{PE}-файлах почти всегда начинаются с адресов заканчивающихся 3 шестнадцатеричными нулями:
0x41000, 0x10001000, итд.
Причина в том, что почти все секции в \ac{PE} выровнены по границе 0x1000 (4096) байт.

\subsubsection{Арифметика произвольной точности и основание}

\index{RSA}
Арифметика произвольной точности (multi-precision arithmetic) может использовать огромные числа,
которые могут храниться в нескольких байтах.
Например, ключи RSA, и открытые и закрытые, могут занимать до 4096 бит и даже больше.

В \InSqBrackets{\TAOCPvolII, 265} можно найти такую идею: когда вы сохраняете число произвольной точности в нескольких байтах,
всё число может быть представлено как имеющую систему счисления по основанию $2^8=256$, и каждая цифра находится
в соответствующем байте.
Точно также, если вы сохраняете число произвольной точности в нескольких 32-битных целочисленных значениях,
каждая цифра отправляется в каждый 32-битный слот, и вы можете считать что это число записано в системе с основанием $2^{32}$.

\subsubsection{Произношение}

Числа в недесятичных системах счислениях обычно произносятся по одной цифре: ``один-ноль-ноль-один-один-...''.
Слова вроде ``десять'', ``тысяча'', итд, обычно не произносятся, потому что тогда можно спутать с десятичной системой.

\subsubsection{Числа с плавающей запятой}

Чтобы отличать числа с плавающей запятой от целочисленных, часто, в конце добавляют ``.0'',
например $0.0$, $123.0$, итд.

}
\ITA{\input{patterns/numeral_ITA}}
\DE{\input{patterns/numeral_DE}}
\FR{\input{patterns/numeral_FR}}
\PL{\input{patterns/numeral_PL}}

% chapters
\ifdefined\SPANISH
\chapter{Patrones de código}
\fi % SPANISH

\ifdefined\GERMAN
\chapter{Code-Muster}
\fi % GERMAN

\ifdefined\ENGLISH
\chapter{Code Patterns}
\fi % ENGLISH

\ifdefined\ITALIAN
\chapter{Forme di codice}
\fi % ITALIAN

\ifdefined\RUSSIAN
\chapter{Образцы кода}
\fi % RUSSIAN

\ifdefined\BRAZILIAN
\chapter{Padrões de códigos}
\fi % BRAZILIAN

\ifdefined\THAI
\chapter{รูปแบบของโค้ด}
\fi % THAI

\ifdefined\FRENCH
\chapter{Modèle de code}
\fi % FRENCH

\ifdefined\POLISH
\chapter{\PLph{}}
\fi % POLISH

% sections
\EN{\input{patterns/patterns_opt_dbg_EN}}
\ES{\input{patterns/patterns_opt_dbg_ES}}
\ITA{\input{patterns/patterns_opt_dbg_ITA}}
\PTBR{\input{patterns/patterns_opt_dbg_PTBR}}
\RU{\input{patterns/patterns_opt_dbg_RU}}
\THA{\input{patterns/patterns_opt_dbg_THA}}
\DE{\input{patterns/patterns_opt_dbg_DE}}
\FR{\input{patterns/patterns_opt_dbg_FR}}
\PL{\input{patterns/patterns_opt_dbg_PL}}

\RU{\section{Некоторые базовые понятия}}
\EN{\section{Some basics}}
\DE{\section{Einige Grundlagen}}
\FR{\section{Quelques bases}}
\ES{\section{\ESph{}}}
\ITA{\section{Alcune basi teoriche}}
\PTBR{\section{\PTBRph{}}}
\THA{\section{\THAph{}}}
\PL{\section{\PLph{}}}

% sections:
\EN{\input{patterns/intro_CPU_ISA_EN}}
\ES{\input{patterns/intro_CPU_ISA_ES}}
\ITA{\input{patterns/intro_CPU_ISA_ITA}}
\PTBR{\input{patterns/intro_CPU_ISA_PTBR}}
\RU{\input{patterns/intro_CPU_ISA_RU}}
\DE{\input{patterns/intro_CPU_ISA_DE}}
\FR{\input{patterns/intro_CPU_ISA_FR}}
\PL{\input{patterns/intro_CPU_ISA_PL}}

\EN{\input{patterns/numeral_EN}}
\RU{\input{patterns/numeral_RU}}
\ITA{\input{patterns/numeral_ITA}}
\DE{\input{patterns/numeral_DE}}
\FR{\input{patterns/numeral_FR}}
\PL{\input{patterns/numeral_PL}}

% chapters
\input{patterns/00_empty/main}
\input{patterns/011_ret/main}
\input{patterns/01_helloworld/main}
\input{patterns/015_prolog_epilogue/main}
\input{patterns/02_stack/main}
\input{patterns/03_printf/main}
\input{patterns/04_scanf/main}
\input{patterns/05_passing_arguments/main}
\input{patterns/06_return_results/main}
\input{patterns/061_pointers/main}
\input{patterns/065_GOTO/main}
\input{patterns/07_jcc/main}
\input{patterns/08_switch/main}
\input{patterns/09_loops/main}
\input{patterns/10_strings/main}
\input{patterns/11_arith_optimizations/main}
\input{patterns/12_FPU/main}
\input{patterns/13_arrays/main}
\input{patterns/14_bitfields/main}
\EN{\input{patterns/145_LCG/main_EN}}
\RU{\input{patterns/145_LCG/main_RU}}
\input{patterns/15_structs/main}
\input{patterns/17_unions/main}
\input{patterns/18_pointers_to_functions/main}
\input{patterns/185_64bit_in_32_env/main}

\EN{\input{patterns/19_SIMD/main_EN}}
\RU{\input{patterns/19_SIMD/main_RU}}
\DE{\input{patterns/19_SIMD/main_DE}}

\EN{\input{patterns/20_x64/main_EN}}
\RU{\input{patterns/20_x64/main_RU}}

\EN{\input{patterns/205_floating_SIMD/main_EN}}
\RU{\input{patterns/205_floating_SIMD/main_RU}}
\DE{\input{patterns/205_floating_SIMD/main_DE}}

\EN{\input{patterns/ARM/main_EN}}
\RU{\input{patterns/ARM/main_RU}}
\DE{\input{patterns/ARM/main_DE}}

\input{patterns/MIPS/main}

\ifdefined\SPANISH
\chapter{Patrones de código}
\fi % SPANISH

\ifdefined\GERMAN
\chapter{Code-Muster}
\fi % GERMAN

\ifdefined\ENGLISH
\chapter{Code Patterns}
\fi % ENGLISH

\ifdefined\ITALIAN
\chapter{Forme di codice}
\fi % ITALIAN

\ifdefined\RUSSIAN
\chapter{Образцы кода}
\fi % RUSSIAN

\ifdefined\BRAZILIAN
\chapter{Padrões de códigos}
\fi % BRAZILIAN

\ifdefined\THAI
\chapter{รูปแบบของโค้ด}
\fi % THAI

\ifdefined\FRENCH
\chapter{Modèle de code}
\fi % FRENCH

\ifdefined\POLISH
\chapter{\PLph{}}
\fi % POLISH

% sections
\EN{\input{patterns/patterns_opt_dbg_EN}}
\ES{\input{patterns/patterns_opt_dbg_ES}}
\ITA{\input{patterns/patterns_opt_dbg_ITA}}
\PTBR{\input{patterns/patterns_opt_dbg_PTBR}}
\RU{\input{patterns/patterns_opt_dbg_RU}}
\THA{\input{patterns/patterns_opt_dbg_THA}}
\DE{\input{patterns/patterns_opt_dbg_DE}}
\FR{\input{patterns/patterns_opt_dbg_FR}}
\PL{\input{patterns/patterns_opt_dbg_PL}}

\RU{\section{Некоторые базовые понятия}}
\EN{\section{Some basics}}
\DE{\section{Einige Grundlagen}}
\FR{\section{Quelques bases}}
\ES{\section{\ESph{}}}
\ITA{\section{Alcune basi teoriche}}
\PTBR{\section{\PTBRph{}}}
\THA{\section{\THAph{}}}
\PL{\section{\PLph{}}}

% sections:
\EN{\input{patterns/intro_CPU_ISA_EN}}
\ES{\input{patterns/intro_CPU_ISA_ES}}
\ITA{\input{patterns/intro_CPU_ISA_ITA}}
\PTBR{\input{patterns/intro_CPU_ISA_PTBR}}
\RU{\input{patterns/intro_CPU_ISA_RU}}
\DE{\input{patterns/intro_CPU_ISA_DE}}
\FR{\input{patterns/intro_CPU_ISA_FR}}
\PL{\input{patterns/intro_CPU_ISA_PL}}

\EN{\input{patterns/numeral_EN}}
\RU{\input{patterns/numeral_RU}}
\ITA{\input{patterns/numeral_ITA}}
\DE{\input{patterns/numeral_DE}}
\FR{\input{patterns/numeral_FR}}
\PL{\input{patterns/numeral_PL}}

% chapters
\input{patterns/00_empty/main}
\input{patterns/011_ret/main}
\input{patterns/01_helloworld/main}
\input{patterns/015_prolog_epilogue/main}
\input{patterns/02_stack/main}
\input{patterns/03_printf/main}
\input{patterns/04_scanf/main}
\input{patterns/05_passing_arguments/main}
\input{patterns/06_return_results/main}
\input{patterns/061_pointers/main}
\input{patterns/065_GOTO/main}
\input{patterns/07_jcc/main}
\input{patterns/08_switch/main}
\input{patterns/09_loops/main}
\input{patterns/10_strings/main}
\input{patterns/11_arith_optimizations/main}
\input{patterns/12_FPU/main}
\input{patterns/13_arrays/main}
\input{patterns/14_bitfields/main}
\EN{\input{patterns/145_LCG/main_EN}}
\RU{\input{patterns/145_LCG/main_RU}}
\input{patterns/15_structs/main}
\input{patterns/17_unions/main}
\input{patterns/18_pointers_to_functions/main}
\input{patterns/185_64bit_in_32_env/main}

\EN{\input{patterns/19_SIMD/main_EN}}
\RU{\input{patterns/19_SIMD/main_RU}}
\DE{\input{patterns/19_SIMD/main_DE}}

\EN{\input{patterns/20_x64/main_EN}}
\RU{\input{patterns/20_x64/main_RU}}

\EN{\input{patterns/205_floating_SIMD/main_EN}}
\RU{\input{patterns/205_floating_SIMD/main_RU}}
\DE{\input{patterns/205_floating_SIMD/main_DE}}

\EN{\input{patterns/ARM/main_EN}}
\RU{\input{patterns/ARM/main_RU}}
\DE{\input{patterns/ARM/main_DE}}

\input{patterns/MIPS/main}

\ifdefined\SPANISH
\chapter{Patrones de código}
\fi % SPANISH

\ifdefined\GERMAN
\chapter{Code-Muster}
\fi % GERMAN

\ifdefined\ENGLISH
\chapter{Code Patterns}
\fi % ENGLISH

\ifdefined\ITALIAN
\chapter{Forme di codice}
\fi % ITALIAN

\ifdefined\RUSSIAN
\chapter{Образцы кода}
\fi % RUSSIAN

\ifdefined\BRAZILIAN
\chapter{Padrões de códigos}
\fi % BRAZILIAN

\ifdefined\THAI
\chapter{รูปแบบของโค้ด}
\fi % THAI

\ifdefined\FRENCH
\chapter{Modèle de code}
\fi % FRENCH

\ifdefined\POLISH
\chapter{\PLph{}}
\fi % POLISH

% sections
\EN{\input{patterns/patterns_opt_dbg_EN}}
\ES{\input{patterns/patterns_opt_dbg_ES}}
\ITA{\input{patterns/patterns_opt_dbg_ITA}}
\PTBR{\input{patterns/patterns_opt_dbg_PTBR}}
\RU{\input{patterns/patterns_opt_dbg_RU}}
\THA{\input{patterns/patterns_opt_dbg_THA}}
\DE{\input{patterns/patterns_opt_dbg_DE}}
\FR{\input{patterns/patterns_opt_dbg_FR}}
\PL{\input{patterns/patterns_opt_dbg_PL}}

\RU{\section{Некоторые базовые понятия}}
\EN{\section{Some basics}}
\DE{\section{Einige Grundlagen}}
\FR{\section{Quelques bases}}
\ES{\section{\ESph{}}}
\ITA{\section{Alcune basi teoriche}}
\PTBR{\section{\PTBRph{}}}
\THA{\section{\THAph{}}}
\PL{\section{\PLph{}}}

% sections:
\EN{\input{patterns/intro_CPU_ISA_EN}}
\ES{\input{patterns/intro_CPU_ISA_ES}}
\ITA{\input{patterns/intro_CPU_ISA_ITA}}
\PTBR{\input{patterns/intro_CPU_ISA_PTBR}}
\RU{\input{patterns/intro_CPU_ISA_RU}}
\DE{\input{patterns/intro_CPU_ISA_DE}}
\FR{\input{patterns/intro_CPU_ISA_FR}}
\PL{\input{patterns/intro_CPU_ISA_PL}}

\EN{\input{patterns/numeral_EN}}
\RU{\input{patterns/numeral_RU}}
\ITA{\input{patterns/numeral_ITA}}
\DE{\input{patterns/numeral_DE}}
\FR{\input{patterns/numeral_FR}}
\PL{\input{patterns/numeral_PL}}

% chapters
\input{patterns/00_empty/main}
\input{patterns/011_ret/main}
\input{patterns/01_helloworld/main}
\input{patterns/015_prolog_epilogue/main}
\input{patterns/02_stack/main}
\input{patterns/03_printf/main}
\input{patterns/04_scanf/main}
\input{patterns/05_passing_arguments/main}
\input{patterns/06_return_results/main}
\input{patterns/061_pointers/main}
\input{patterns/065_GOTO/main}
\input{patterns/07_jcc/main}
\input{patterns/08_switch/main}
\input{patterns/09_loops/main}
\input{patterns/10_strings/main}
\input{patterns/11_arith_optimizations/main}
\input{patterns/12_FPU/main}
\input{patterns/13_arrays/main}
\input{patterns/14_bitfields/main}
\EN{\input{patterns/145_LCG/main_EN}}
\RU{\input{patterns/145_LCG/main_RU}}
\input{patterns/15_structs/main}
\input{patterns/17_unions/main}
\input{patterns/18_pointers_to_functions/main}
\input{patterns/185_64bit_in_32_env/main}

\EN{\input{patterns/19_SIMD/main_EN}}
\RU{\input{patterns/19_SIMD/main_RU}}
\DE{\input{patterns/19_SIMD/main_DE}}

\EN{\input{patterns/20_x64/main_EN}}
\RU{\input{patterns/20_x64/main_RU}}

\EN{\input{patterns/205_floating_SIMD/main_EN}}
\RU{\input{patterns/205_floating_SIMD/main_RU}}
\DE{\input{patterns/205_floating_SIMD/main_DE}}

\EN{\input{patterns/ARM/main_EN}}
\RU{\input{patterns/ARM/main_RU}}
\DE{\input{patterns/ARM/main_DE}}

\input{patterns/MIPS/main}

\ifdefined\SPANISH
\chapter{Patrones de código}
\fi % SPANISH

\ifdefined\GERMAN
\chapter{Code-Muster}
\fi % GERMAN

\ifdefined\ENGLISH
\chapter{Code Patterns}
\fi % ENGLISH

\ifdefined\ITALIAN
\chapter{Forme di codice}
\fi % ITALIAN

\ifdefined\RUSSIAN
\chapter{Образцы кода}
\fi % RUSSIAN

\ifdefined\BRAZILIAN
\chapter{Padrões de códigos}
\fi % BRAZILIAN

\ifdefined\THAI
\chapter{รูปแบบของโค้ด}
\fi % THAI

\ifdefined\FRENCH
\chapter{Modèle de code}
\fi % FRENCH

\ifdefined\POLISH
\chapter{\PLph{}}
\fi % POLISH

% sections
\EN{\input{patterns/patterns_opt_dbg_EN}}
\ES{\input{patterns/patterns_opt_dbg_ES}}
\ITA{\input{patterns/patterns_opt_dbg_ITA}}
\PTBR{\input{patterns/patterns_opt_dbg_PTBR}}
\RU{\input{patterns/patterns_opt_dbg_RU}}
\THA{\input{patterns/patterns_opt_dbg_THA}}
\DE{\input{patterns/patterns_opt_dbg_DE}}
\FR{\input{patterns/patterns_opt_dbg_FR}}
\PL{\input{patterns/patterns_opt_dbg_PL}}

\RU{\section{Некоторые базовые понятия}}
\EN{\section{Some basics}}
\DE{\section{Einige Grundlagen}}
\FR{\section{Quelques bases}}
\ES{\section{\ESph{}}}
\ITA{\section{Alcune basi teoriche}}
\PTBR{\section{\PTBRph{}}}
\THA{\section{\THAph{}}}
\PL{\section{\PLph{}}}

% sections:
\EN{\input{patterns/intro_CPU_ISA_EN}}
\ES{\input{patterns/intro_CPU_ISA_ES}}
\ITA{\input{patterns/intro_CPU_ISA_ITA}}
\PTBR{\input{patterns/intro_CPU_ISA_PTBR}}
\RU{\input{patterns/intro_CPU_ISA_RU}}
\DE{\input{patterns/intro_CPU_ISA_DE}}
\FR{\input{patterns/intro_CPU_ISA_FR}}
\PL{\input{patterns/intro_CPU_ISA_PL}}

\EN{\input{patterns/numeral_EN}}
\RU{\input{patterns/numeral_RU}}
\ITA{\input{patterns/numeral_ITA}}
\DE{\input{patterns/numeral_DE}}
\FR{\input{patterns/numeral_FR}}
\PL{\input{patterns/numeral_PL}}

% chapters
\input{patterns/00_empty/main}
\input{patterns/011_ret/main}
\input{patterns/01_helloworld/main}
\input{patterns/015_prolog_epilogue/main}
\input{patterns/02_stack/main}
\input{patterns/03_printf/main}
\input{patterns/04_scanf/main}
\input{patterns/05_passing_arguments/main}
\input{patterns/06_return_results/main}
\input{patterns/061_pointers/main}
\input{patterns/065_GOTO/main}
\input{patterns/07_jcc/main}
\input{patterns/08_switch/main}
\input{patterns/09_loops/main}
\input{patterns/10_strings/main}
\input{patterns/11_arith_optimizations/main}
\input{patterns/12_FPU/main}
\input{patterns/13_arrays/main}
\input{patterns/14_bitfields/main}
\EN{\input{patterns/145_LCG/main_EN}}
\RU{\input{patterns/145_LCG/main_RU}}
\input{patterns/15_structs/main}
\input{patterns/17_unions/main}
\input{patterns/18_pointers_to_functions/main}
\input{patterns/185_64bit_in_32_env/main}

\EN{\input{patterns/19_SIMD/main_EN}}
\RU{\input{patterns/19_SIMD/main_RU}}
\DE{\input{patterns/19_SIMD/main_DE}}

\EN{\input{patterns/20_x64/main_EN}}
\RU{\input{patterns/20_x64/main_RU}}

\EN{\input{patterns/205_floating_SIMD/main_EN}}
\RU{\input{patterns/205_floating_SIMD/main_RU}}
\DE{\input{patterns/205_floating_SIMD/main_DE}}

\EN{\input{patterns/ARM/main_EN}}
\RU{\input{patterns/ARM/main_RU}}
\DE{\input{patterns/ARM/main_DE}}

\input{patterns/MIPS/main}

\ifdefined\SPANISH
\chapter{Patrones de código}
\fi % SPANISH

\ifdefined\GERMAN
\chapter{Code-Muster}
\fi % GERMAN

\ifdefined\ENGLISH
\chapter{Code Patterns}
\fi % ENGLISH

\ifdefined\ITALIAN
\chapter{Forme di codice}
\fi % ITALIAN

\ifdefined\RUSSIAN
\chapter{Образцы кода}
\fi % RUSSIAN

\ifdefined\BRAZILIAN
\chapter{Padrões de códigos}
\fi % BRAZILIAN

\ifdefined\THAI
\chapter{รูปแบบของโค้ด}
\fi % THAI

\ifdefined\FRENCH
\chapter{Modèle de code}
\fi % FRENCH

\ifdefined\POLISH
\chapter{\PLph{}}
\fi % POLISH

% sections
\EN{\input{patterns/patterns_opt_dbg_EN}}
\ES{\input{patterns/patterns_opt_dbg_ES}}
\ITA{\input{patterns/patterns_opt_dbg_ITA}}
\PTBR{\input{patterns/patterns_opt_dbg_PTBR}}
\RU{\input{patterns/patterns_opt_dbg_RU}}
\THA{\input{patterns/patterns_opt_dbg_THA}}
\DE{\input{patterns/patterns_opt_dbg_DE}}
\FR{\input{patterns/patterns_opt_dbg_FR}}
\PL{\input{patterns/patterns_opt_dbg_PL}}

\RU{\section{Некоторые базовые понятия}}
\EN{\section{Some basics}}
\DE{\section{Einige Grundlagen}}
\FR{\section{Quelques bases}}
\ES{\section{\ESph{}}}
\ITA{\section{Alcune basi teoriche}}
\PTBR{\section{\PTBRph{}}}
\THA{\section{\THAph{}}}
\PL{\section{\PLph{}}}

% sections:
\EN{\input{patterns/intro_CPU_ISA_EN}}
\ES{\input{patterns/intro_CPU_ISA_ES}}
\ITA{\input{patterns/intro_CPU_ISA_ITA}}
\PTBR{\input{patterns/intro_CPU_ISA_PTBR}}
\RU{\input{patterns/intro_CPU_ISA_RU}}
\DE{\input{patterns/intro_CPU_ISA_DE}}
\FR{\input{patterns/intro_CPU_ISA_FR}}
\PL{\input{patterns/intro_CPU_ISA_PL}}

\EN{\input{patterns/numeral_EN}}
\RU{\input{patterns/numeral_RU}}
\ITA{\input{patterns/numeral_ITA}}
\DE{\input{patterns/numeral_DE}}
\FR{\input{patterns/numeral_FR}}
\PL{\input{patterns/numeral_PL}}

% chapters
\input{patterns/00_empty/main}
\input{patterns/011_ret/main}
\input{patterns/01_helloworld/main}
\input{patterns/015_prolog_epilogue/main}
\input{patterns/02_stack/main}
\input{patterns/03_printf/main}
\input{patterns/04_scanf/main}
\input{patterns/05_passing_arguments/main}
\input{patterns/06_return_results/main}
\input{patterns/061_pointers/main}
\input{patterns/065_GOTO/main}
\input{patterns/07_jcc/main}
\input{patterns/08_switch/main}
\input{patterns/09_loops/main}
\input{patterns/10_strings/main}
\input{patterns/11_arith_optimizations/main}
\input{patterns/12_FPU/main}
\input{patterns/13_arrays/main}
\input{patterns/14_bitfields/main}
\EN{\input{patterns/145_LCG/main_EN}}
\RU{\input{patterns/145_LCG/main_RU}}
\input{patterns/15_structs/main}
\input{patterns/17_unions/main}
\input{patterns/18_pointers_to_functions/main}
\input{patterns/185_64bit_in_32_env/main}

\EN{\input{patterns/19_SIMD/main_EN}}
\RU{\input{patterns/19_SIMD/main_RU}}
\DE{\input{patterns/19_SIMD/main_DE}}

\EN{\input{patterns/20_x64/main_EN}}
\RU{\input{patterns/20_x64/main_RU}}

\EN{\input{patterns/205_floating_SIMD/main_EN}}
\RU{\input{patterns/205_floating_SIMD/main_RU}}
\DE{\input{patterns/205_floating_SIMD/main_DE}}

\EN{\input{patterns/ARM/main_EN}}
\RU{\input{patterns/ARM/main_RU}}
\DE{\input{patterns/ARM/main_DE}}

\input{patterns/MIPS/main}

\ifdefined\SPANISH
\chapter{Patrones de código}
\fi % SPANISH

\ifdefined\GERMAN
\chapter{Code-Muster}
\fi % GERMAN

\ifdefined\ENGLISH
\chapter{Code Patterns}
\fi % ENGLISH

\ifdefined\ITALIAN
\chapter{Forme di codice}
\fi % ITALIAN

\ifdefined\RUSSIAN
\chapter{Образцы кода}
\fi % RUSSIAN

\ifdefined\BRAZILIAN
\chapter{Padrões de códigos}
\fi % BRAZILIAN

\ifdefined\THAI
\chapter{รูปแบบของโค้ด}
\fi % THAI

\ifdefined\FRENCH
\chapter{Modèle de code}
\fi % FRENCH

\ifdefined\POLISH
\chapter{\PLph{}}
\fi % POLISH

% sections
\EN{\input{patterns/patterns_opt_dbg_EN}}
\ES{\input{patterns/patterns_opt_dbg_ES}}
\ITA{\input{patterns/patterns_opt_dbg_ITA}}
\PTBR{\input{patterns/patterns_opt_dbg_PTBR}}
\RU{\input{patterns/patterns_opt_dbg_RU}}
\THA{\input{patterns/patterns_opt_dbg_THA}}
\DE{\input{patterns/patterns_opt_dbg_DE}}
\FR{\input{patterns/patterns_opt_dbg_FR}}
\PL{\input{patterns/patterns_opt_dbg_PL}}

\RU{\section{Некоторые базовые понятия}}
\EN{\section{Some basics}}
\DE{\section{Einige Grundlagen}}
\FR{\section{Quelques bases}}
\ES{\section{\ESph{}}}
\ITA{\section{Alcune basi teoriche}}
\PTBR{\section{\PTBRph{}}}
\THA{\section{\THAph{}}}
\PL{\section{\PLph{}}}

% sections:
\EN{\input{patterns/intro_CPU_ISA_EN}}
\ES{\input{patterns/intro_CPU_ISA_ES}}
\ITA{\input{patterns/intro_CPU_ISA_ITA}}
\PTBR{\input{patterns/intro_CPU_ISA_PTBR}}
\RU{\input{patterns/intro_CPU_ISA_RU}}
\DE{\input{patterns/intro_CPU_ISA_DE}}
\FR{\input{patterns/intro_CPU_ISA_FR}}
\PL{\input{patterns/intro_CPU_ISA_PL}}

\EN{\input{patterns/numeral_EN}}
\RU{\input{patterns/numeral_RU}}
\ITA{\input{patterns/numeral_ITA}}
\DE{\input{patterns/numeral_DE}}
\FR{\input{patterns/numeral_FR}}
\PL{\input{patterns/numeral_PL}}

% chapters
\input{patterns/00_empty/main}
\input{patterns/011_ret/main}
\input{patterns/01_helloworld/main}
\input{patterns/015_prolog_epilogue/main}
\input{patterns/02_stack/main}
\input{patterns/03_printf/main}
\input{patterns/04_scanf/main}
\input{patterns/05_passing_arguments/main}
\input{patterns/06_return_results/main}
\input{patterns/061_pointers/main}
\input{patterns/065_GOTO/main}
\input{patterns/07_jcc/main}
\input{patterns/08_switch/main}
\input{patterns/09_loops/main}
\input{patterns/10_strings/main}
\input{patterns/11_arith_optimizations/main}
\input{patterns/12_FPU/main}
\input{patterns/13_arrays/main}
\input{patterns/14_bitfields/main}
\EN{\input{patterns/145_LCG/main_EN}}
\RU{\input{patterns/145_LCG/main_RU}}
\input{patterns/15_structs/main}
\input{patterns/17_unions/main}
\input{patterns/18_pointers_to_functions/main}
\input{patterns/185_64bit_in_32_env/main}

\EN{\input{patterns/19_SIMD/main_EN}}
\RU{\input{patterns/19_SIMD/main_RU}}
\DE{\input{patterns/19_SIMD/main_DE}}

\EN{\input{patterns/20_x64/main_EN}}
\RU{\input{patterns/20_x64/main_RU}}

\EN{\input{patterns/205_floating_SIMD/main_EN}}
\RU{\input{patterns/205_floating_SIMD/main_RU}}
\DE{\input{patterns/205_floating_SIMD/main_DE}}

\EN{\input{patterns/ARM/main_EN}}
\RU{\input{patterns/ARM/main_RU}}
\DE{\input{patterns/ARM/main_DE}}

\input{patterns/MIPS/main}

\ifdefined\SPANISH
\chapter{Patrones de código}
\fi % SPANISH

\ifdefined\GERMAN
\chapter{Code-Muster}
\fi % GERMAN

\ifdefined\ENGLISH
\chapter{Code Patterns}
\fi % ENGLISH

\ifdefined\ITALIAN
\chapter{Forme di codice}
\fi % ITALIAN

\ifdefined\RUSSIAN
\chapter{Образцы кода}
\fi % RUSSIAN

\ifdefined\BRAZILIAN
\chapter{Padrões de códigos}
\fi % BRAZILIAN

\ifdefined\THAI
\chapter{รูปแบบของโค้ด}
\fi % THAI

\ifdefined\FRENCH
\chapter{Modèle de code}
\fi % FRENCH

\ifdefined\POLISH
\chapter{\PLph{}}
\fi % POLISH

% sections
\EN{\input{patterns/patterns_opt_dbg_EN}}
\ES{\input{patterns/patterns_opt_dbg_ES}}
\ITA{\input{patterns/patterns_opt_dbg_ITA}}
\PTBR{\input{patterns/patterns_opt_dbg_PTBR}}
\RU{\input{patterns/patterns_opt_dbg_RU}}
\THA{\input{patterns/patterns_opt_dbg_THA}}
\DE{\input{patterns/patterns_opt_dbg_DE}}
\FR{\input{patterns/patterns_opt_dbg_FR}}
\PL{\input{patterns/patterns_opt_dbg_PL}}

\RU{\section{Некоторые базовые понятия}}
\EN{\section{Some basics}}
\DE{\section{Einige Grundlagen}}
\FR{\section{Quelques bases}}
\ES{\section{\ESph{}}}
\ITA{\section{Alcune basi teoriche}}
\PTBR{\section{\PTBRph{}}}
\THA{\section{\THAph{}}}
\PL{\section{\PLph{}}}

% sections:
\EN{\input{patterns/intro_CPU_ISA_EN}}
\ES{\input{patterns/intro_CPU_ISA_ES}}
\ITA{\input{patterns/intro_CPU_ISA_ITA}}
\PTBR{\input{patterns/intro_CPU_ISA_PTBR}}
\RU{\input{patterns/intro_CPU_ISA_RU}}
\DE{\input{patterns/intro_CPU_ISA_DE}}
\FR{\input{patterns/intro_CPU_ISA_FR}}
\PL{\input{patterns/intro_CPU_ISA_PL}}

\EN{\input{patterns/numeral_EN}}
\RU{\input{patterns/numeral_RU}}
\ITA{\input{patterns/numeral_ITA}}
\DE{\input{patterns/numeral_DE}}
\FR{\input{patterns/numeral_FR}}
\PL{\input{patterns/numeral_PL}}

% chapters
\input{patterns/00_empty/main}
\input{patterns/011_ret/main}
\input{patterns/01_helloworld/main}
\input{patterns/015_prolog_epilogue/main}
\input{patterns/02_stack/main}
\input{patterns/03_printf/main}
\input{patterns/04_scanf/main}
\input{patterns/05_passing_arguments/main}
\input{patterns/06_return_results/main}
\input{patterns/061_pointers/main}
\input{patterns/065_GOTO/main}
\input{patterns/07_jcc/main}
\input{patterns/08_switch/main}
\input{patterns/09_loops/main}
\input{patterns/10_strings/main}
\input{patterns/11_arith_optimizations/main}
\input{patterns/12_FPU/main}
\input{patterns/13_arrays/main}
\input{patterns/14_bitfields/main}
\EN{\input{patterns/145_LCG/main_EN}}
\RU{\input{patterns/145_LCG/main_RU}}
\input{patterns/15_structs/main}
\input{patterns/17_unions/main}
\input{patterns/18_pointers_to_functions/main}
\input{patterns/185_64bit_in_32_env/main}

\EN{\input{patterns/19_SIMD/main_EN}}
\RU{\input{patterns/19_SIMD/main_RU}}
\DE{\input{patterns/19_SIMD/main_DE}}

\EN{\input{patterns/20_x64/main_EN}}
\RU{\input{patterns/20_x64/main_RU}}

\EN{\input{patterns/205_floating_SIMD/main_EN}}
\RU{\input{patterns/205_floating_SIMD/main_RU}}
\DE{\input{patterns/205_floating_SIMD/main_DE}}

\EN{\input{patterns/ARM/main_EN}}
\RU{\input{patterns/ARM/main_RU}}
\DE{\input{patterns/ARM/main_DE}}

\input{patterns/MIPS/main}

\ifdefined\SPANISH
\chapter{Patrones de código}
\fi % SPANISH

\ifdefined\GERMAN
\chapter{Code-Muster}
\fi % GERMAN

\ifdefined\ENGLISH
\chapter{Code Patterns}
\fi % ENGLISH

\ifdefined\ITALIAN
\chapter{Forme di codice}
\fi % ITALIAN

\ifdefined\RUSSIAN
\chapter{Образцы кода}
\fi % RUSSIAN

\ifdefined\BRAZILIAN
\chapter{Padrões de códigos}
\fi % BRAZILIAN

\ifdefined\THAI
\chapter{รูปแบบของโค้ด}
\fi % THAI

\ifdefined\FRENCH
\chapter{Modèle de code}
\fi % FRENCH

\ifdefined\POLISH
\chapter{\PLph{}}
\fi % POLISH

% sections
\EN{\input{patterns/patterns_opt_dbg_EN}}
\ES{\input{patterns/patterns_opt_dbg_ES}}
\ITA{\input{patterns/patterns_opt_dbg_ITA}}
\PTBR{\input{patterns/patterns_opt_dbg_PTBR}}
\RU{\input{patterns/patterns_opt_dbg_RU}}
\THA{\input{patterns/patterns_opt_dbg_THA}}
\DE{\input{patterns/patterns_opt_dbg_DE}}
\FR{\input{patterns/patterns_opt_dbg_FR}}
\PL{\input{patterns/patterns_opt_dbg_PL}}

\RU{\section{Некоторые базовые понятия}}
\EN{\section{Some basics}}
\DE{\section{Einige Grundlagen}}
\FR{\section{Quelques bases}}
\ES{\section{\ESph{}}}
\ITA{\section{Alcune basi teoriche}}
\PTBR{\section{\PTBRph{}}}
\THA{\section{\THAph{}}}
\PL{\section{\PLph{}}}

% sections:
\EN{\input{patterns/intro_CPU_ISA_EN}}
\ES{\input{patterns/intro_CPU_ISA_ES}}
\ITA{\input{patterns/intro_CPU_ISA_ITA}}
\PTBR{\input{patterns/intro_CPU_ISA_PTBR}}
\RU{\input{patterns/intro_CPU_ISA_RU}}
\DE{\input{patterns/intro_CPU_ISA_DE}}
\FR{\input{patterns/intro_CPU_ISA_FR}}
\PL{\input{patterns/intro_CPU_ISA_PL}}

\EN{\input{patterns/numeral_EN}}
\RU{\input{patterns/numeral_RU}}
\ITA{\input{patterns/numeral_ITA}}
\DE{\input{patterns/numeral_DE}}
\FR{\input{patterns/numeral_FR}}
\PL{\input{patterns/numeral_PL}}

% chapters
\input{patterns/00_empty/main}
\input{patterns/011_ret/main}
\input{patterns/01_helloworld/main}
\input{patterns/015_prolog_epilogue/main}
\input{patterns/02_stack/main}
\input{patterns/03_printf/main}
\input{patterns/04_scanf/main}
\input{patterns/05_passing_arguments/main}
\input{patterns/06_return_results/main}
\input{patterns/061_pointers/main}
\input{patterns/065_GOTO/main}
\input{patterns/07_jcc/main}
\input{patterns/08_switch/main}
\input{patterns/09_loops/main}
\input{patterns/10_strings/main}
\input{patterns/11_arith_optimizations/main}
\input{patterns/12_FPU/main}
\input{patterns/13_arrays/main}
\input{patterns/14_bitfields/main}
\EN{\input{patterns/145_LCG/main_EN}}
\RU{\input{patterns/145_LCG/main_RU}}
\input{patterns/15_structs/main}
\input{patterns/17_unions/main}
\input{patterns/18_pointers_to_functions/main}
\input{patterns/185_64bit_in_32_env/main}

\EN{\input{patterns/19_SIMD/main_EN}}
\RU{\input{patterns/19_SIMD/main_RU}}
\DE{\input{patterns/19_SIMD/main_DE}}

\EN{\input{patterns/20_x64/main_EN}}
\RU{\input{patterns/20_x64/main_RU}}

\EN{\input{patterns/205_floating_SIMD/main_EN}}
\RU{\input{patterns/205_floating_SIMD/main_RU}}
\DE{\input{patterns/205_floating_SIMD/main_DE}}

\EN{\input{patterns/ARM/main_EN}}
\RU{\input{patterns/ARM/main_RU}}
\DE{\input{patterns/ARM/main_DE}}

\input{patterns/MIPS/main}

\ifdefined\SPANISH
\chapter{Patrones de código}
\fi % SPANISH

\ifdefined\GERMAN
\chapter{Code-Muster}
\fi % GERMAN

\ifdefined\ENGLISH
\chapter{Code Patterns}
\fi % ENGLISH

\ifdefined\ITALIAN
\chapter{Forme di codice}
\fi % ITALIAN

\ifdefined\RUSSIAN
\chapter{Образцы кода}
\fi % RUSSIAN

\ifdefined\BRAZILIAN
\chapter{Padrões de códigos}
\fi % BRAZILIAN

\ifdefined\THAI
\chapter{รูปแบบของโค้ด}
\fi % THAI

\ifdefined\FRENCH
\chapter{Modèle de code}
\fi % FRENCH

\ifdefined\POLISH
\chapter{\PLph{}}
\fi % POLISH

% sections
\EN{\input{patterns/patterns_opt_dbg_EN}}
\ES{\input{patterns/patterns_opt_dbg_ES}}
\ITA{\input{patterns/patterns_opt_dbg_ITA}}
\PTBR{\input{patterns/patterns_opt_dbg_PTBR}}
\RU{\input{patterns/patterns_opt_dbg_RU}}
\THA{\input{patterns/patterns_opt_dbg_THA}}
\DE{\input{patterns/patterns_opt_dbg_DE}}
\FR{\input{patterns/patterns_opt_dbg_FR}}
\PL{\input{patterns/patterns_opt_dbg_PL}}

\RU{\section{Некоторые базовые понятия}}
\EN{\section{Some basics}}
\DE{\section{Einige Grundlagen}}
\FR{\section{Quelques bases}}
\ES{\section{\ESph{}}}
\ITA{\section{Alcune basi teoriche}}
\PTBR{\section{\PTBRph{}}}
\THA{\section{\THAph{}}}
\PL{\section{\PLph{}}}

% sections:
\EN{\input{patterns/intro_CPU_ISA_EN}}
\ES{\input{patterns/intro_CPU_ISA_ES}}
\ITA{\input{patterns/intro_CPU_ISA_ITA}}
\PTBR{\input{patterns/intro_CPU_ISA_PTBR}}
\RU{\input{patterns/intro_CPU_ISA_RU}}
\DE{\input{patterns/intro_CPU_ISA_DE}}
\FR{\input{patterns/intro_CPU_ISA_FR}}
\PL{\input{patterns/intro_CPU_ISA_PL}}

\EN{\input{patterns/numeral_EN}}
\RU{\input{patterns/numeral_RU}}
\ITA{\input{patterns/numeral_ITA}}
\DE{\input{patterns/numeral_DE}}
\FR{\input{patterns/numeral_FR}}
\PL{\input{patterns/numeral_PL}}

% chapters
\input{patterns/00_empty/main}
\input{patterns/011_ret/main}
\input{patterns/01_helloworld/main}
\input{patterns/015_prolog_epilogue/main}
\input{patterns/02_stack/main}
\input{patterns/03_printf/main}
\input{patterns/04_scanf/main}
\input{patterns/05_passing_arguments/main}
\input{patterns/06_return_results/main}
\input{patterns/061_pointers/main}
\input{patterns/065_GOTO/main}
\input{patterns/07_jcc/main}
\input{patterns/08_switch/main}
\input{patterns/09_loops/main}
\input{patterns/10_strings/main}
\input{patterns/11_arith_optimizations/main}
\input{patterns/12_FPU/main}
\input{patterns/13_arrays/main}
\input{patterns/14_bitfields/main}
\EN{\input{patterns/145_LCG/main_EN}}
\RU{\input{patterns/145_LCG/main_RU}}
\input{patterns/15_structs/main}
\input{patterns/17_unions/main}
\input{patterns/18_pointers_to_functions/main}
\input{patterns/185_64bit_in_32_env/main}

\EN{\input{patterns/19_SIMD/main_EN}}
\RU{\input{patterns/19_SIMD/main_RU}}
\DE{\input{patterns/19_SIMD/main_DE}}

\EN{\input{patterns/20_x64/main_EN}}
\RU{\input{patterns/20_x64/main_RU}}

\EN{\input{patterns/205_floating_SIMD/main_EN}}
\RU{\input{patterns/205_floating_SIMD/main_RU}}
\DE{\input{patterns/205_floating_SIMD/main_DE}}

\EN{\input{patterns/ARM/main_EN}}
\RU{\input{patterns/ARM/main_RU}}
\DE{\input{patterns/ARM/main_DE}}

\input{patterns/MIPS/main}

\ifdefined\SPANISH
\chapter{Patrones de código}
\fi % SPANISH

\ifdefined\GERMAN
\chapter{Code-Muster}
\fi % GERMAN

\ifdefined\ENGLISH
\chapter{Code Patterns}
\fi % ENGLISH

\ifdefined\ITALIAN
\chapter{Forme di codice}
\fi % ITALIAN

\ifdefined\RUSSIAN
\chapter{Образцы кода}
\fi % RUSSIAN

\ifdefined\BRAZILIAN
\chapter{Padrões de códigos}
\fi % BRAZILIAN

\ifdefined\THAI
\chapter{รูปแบบของโค้ด}
\fi % THAI

\ifdefined\FRENCH
\chapter{Modèle de code}
\fi % FRENCH

\ifdefined\POLISH
\chapter{\PLph{}}
\fi % POLISH

% sections
\EN{\input{patterns/patterns_opt_dbg_EN}}
\ES{\input{patterns/patterns_opt_dbg_ES}}
\ITA{\input{patterns/patterns_opt_dbg_ITA}}
\PTBR{\input{patterns/patterns_opt_dbg_PTBR}}
\RU{\input{patterns/patterns_opt_dbg_RU}}
\THA{\input{patterns/patterns_opt_dbg_THA}}
\DE{\input{patterns/patterns_opt_dbg_DE}}
\FR{\input{patterns/patterns_opt_dbg_FR}}
\PL{\input{patterns/patterns_opt_dbg_PL}}

\RU{\section{Некоторые базовые понятия}}
\EN{\section{Some basics}}
\DE{\section{Einige Grundlagen}}
\FR{\section{Quelques bases}}
\ES{\section{\ESph{}}}
\ITA{\section{Alcune basi teoriche}}
\PTBR{\section{\PTBRph{}}}
\THA{\section{\THAph{}}}
\PL{\section{\PLph{}}}

% sections:
\EN{\input{patterns/intro_CPU_ISA_EN}}
\ES{\input{patterns/intro_CPU_ISA_ES}}
\ITA{\input{patterns/intro_CPU_ISA_ITA}}
\PTBR{\input{patterns/intro_CPU_ISA_PTBR}}
\RU{\input{patterns/intro_CPU_ISA_RU}}
\DE{\input{patterns/intro_CPU_ISA_DE}}
\FR{\input{patterns/intro_CPU_ISA_FR}}
\PL{\input{patterns/intro_CPU_ISA_PL}}

\EN{\input{patterns/numeral_EN}}
\RU{\input{patterns/numeral_RU}}
\ITA{\input{patterns/numeral_ITA}}
\DE{\input{patterns/numeral_DE}}
\FR{\input{patterns/numeral_FR}}
\PL{\input{patterns/numeral_PL}}

% chapters
\input{patterns/00_empty/main}
\input{patterns/011_ret/main}
\input{patterns/01_helloworld/main}
\input{patterns/015_prolog_epilogue/main}
\input{patterns/02_stack/main}
\input{patterns/03_printf/main}
\input{patterns/04_scanf/main}
\input{patterns/05_passing_arguments/main}
\input{patterns/06_return_results/main}
\input{patterns/061_pointers/main}
\input{patterns/065_GOTO/main}
\input{patterns/07_jcc/main}
\input{patterns/08_switch/main}
\input{patterns/09_loops/main}
\input{patterns/10_strings/main}
\input{patterns/11_arith_optimizations/main}
\input{patterns/12_FPU/main}
\input{patterns/13_arrays/main}
\input{patterns/14_bitfields/main}
\EN{\input{patterns/145_LCG/main_EN}}
\RU{\input{patterns/145_LCG/main_RU}}
\input{patterns/15_structs/main}
\input{patterns/17_unions/main}
\input{patterns/18_pointers_to_functions/main}
\input{patterns/185_64bit_in_32_env/main}

\EN{\input{patterns/19_SIMD/main_EN}}
\RU{\input{patterns/19_SIMD/main_RU}}
\DE{\input{patterns/19_SIMD/main_DE}}

\EN{\input{patterns/20_x64/main_EN}}
\RU{\input{patterns/20_x64/main_RU}}

\EN{\input{patterns/205_floating_SIMD/main_EN}}
\RU{\input{patterns/205_floating_SIMD/main_RU}}
\DE{\input{patterns/205_floating_SIMD/main_DE}}

\EN{\input{patterns/ARM/main_EN}}
\RU{\input{patterns/ARM/main_RU}}
\DE{\input{patterns/ARM/main_DE}}

\input{patterns/MIPS/main}

\ifdefined\SPANISH
\chapter{Patrones de código}
\fi % SPANISH

\ifdefined\GERMAN
\chapter{Code-Muster}
\fi % GERMAN

\ifdefined\ENGLISH
\chapter{Code Patterns}
\fi % ENGLISH

\ifdefined\ITALIAN
\chapter{Forme di codice}
\fi % ITALIAN

\ifdefined\RUSSIAN
\chapter{Образцы кода}
\fi % RUSSIAN

\ifdefined\BRAZILIAN
\chapter{Padrões de códigos}
\fi % BRAZILIAN

\ifdefined\THAI
\chapter{รูปแบบของโค้ด}
\fi % THAI

\ifdefined\FRENCH
\chapter{Modèle de code}
\fi % FRENCH

\ifdefined\POLISH
\chapter{\PLph{}}
\fi % POLISH

% sections
\EN{\input{patterns/patterns_opt_dbg_EN}}
\ES{\input{patterns/patterns_opt_dbg_ES}}
\ITA{\input{patterns/patterns_opt_dbg_ITA}}
\PTBR{\input{patterns/patterns_opt_dbg_PTBR}}
\RU{\input{patterns/patterns_opt_dbg_RU}}
\THA{\input{patterns/patterns_opt_dbg_THA}}
\DE{\input{patterns/patterns_opt_dbg_DE}}
\FR{\input{patterns/patterns_opt_dbg_FR}}
\PL{\input{patterns/patterns_opt_dbg_PL}}

\RU{\section{Некоторые базовые понятия}}
\EN{\section{Some basics}}
\DE{\section{Einige Grundlagen}}
\FR{\section{Quelques bases}}
\ES{\section{\ESph{}}}
\ITA{\section{Alcune basi teoriche}}
\PTBR{\section{\PTBRph{}}}
\THA{\section{\THAph{}}}
\PL{\section{\PLph{}}}

% sections:
\EN{\input{patterns/intro_CPU_ISA_EN}}
\ES{\input{patterns/intro_CPU_ISA_ES}}
\ITA{\input{patterns/intro_CPU_ISA_ITA}}
\PTBR{\input{patterns/intro_CPU_ISA_PTBR}}
\RU{\input{patterns/intro_CPU_ISA_RU}}
\DE{\input{patterns/intro_CPU_ISA_DE}}
\FR{\input{patterns/intro_CPU_ISA_FR}}
\PL{\input{patterns/intro_CPU_ISA_PL}}

\EN{\input{patterns/numeral_EN}}
\RU{\input{patterns/numeral_RU}}
\ITA{\input{patterns/numeral_ITA}}
\DE{\input{patterns/numeral_DE}}
\FR{\input{patterns/numeral_FR}}
\PL{\input{patterns/numeral_PL}}

% chapters
\input{patterns/00_empty/main}
\input{patterns/011_ret/main}
\input{patterns/01_helloworld/main}
\input{patterns/015_prolog_epilogue/main}
\input{patterns/02_stack/main}
\input{patterns/03_printf/main}
\input{patterns/04_scanf/main}
\input{patterns/05_passing_arguments/main}
\input{patterns/06_return_results/main}
\input{patterns/061_pointers/main}
\input{patterns/065_GOTO/main}
\input{patterns/07_jcc/main}
\input{patterns/08_switch/main}
\input{patterns/09_loops/main}
\input{patterns/10_strings/main}
\input{patterns/11_arith_optimizations/main}
\input{patterns/12_FPU/main}
\input{patterns/13_arrays/main}
\input{patterns/14_bitfields/main}
\EN{\input{patterns/145_LCG/main_EN}}
\RU{\input{patterns/145_LCG/main_RU}}
\input{patterns/15_structs/main}
\input{patterns/17_unions/main}
\input{patterns/18_pointers_to_functions/main}
\input{patterns/185_64bit_in_32_env/main}

\EN{\input{patterns/19_SIMD/main_EN}}
\RU{\input{patterns/19_SIMD/main_RU}}
\DE{\input{patterns/19_SIMD/main_DE}}

\EN{\input{patterns/20_x64/main_EN}}
\RU{\input{patterns/20_x64/main_RU}}

\EN{\input{patterns/205_floating_SIMD/main_EN}}
\RU{\input{patterns/205_floating_SIMD/main_RU}}
\DE{\input{patterns/205_floating_SIMD/main_DE}}

\EN{\input{patterns/ARM/main_EN}}
\RU{\input{patterns/ARM/main_RU}}
\DE{\input{patterns/ARM/main_DE}}

\input{patterns/MIPS/main}

\ifdefined\SPANISH
\chapter{Patrones de código}
\fi % SPANISH

\ifdefined\GERMAN
\chapter{Code-Muster}
\fi % GERMAN

\ifdefined\ENGLISH
\chapter{Code Patterns}
\fi % ENGLISH

\ifdefined\ITALIAN
\chapter{Forme di codice}
\fi % ITALIAN

\ifdefined\RUSSIAN
\chapter{Образцы кода}
\fi % RUSSIAN

\ifdefined\BRAZILIAN
\chapter{Padrões de códigos}
\fi % BRAZILIAN

\ifdefined\THAI
\chapter{รูปแบบของโค้ด}
\fi % THAI

\ifdefined\FRENCH
\chapter{Modèle de code}
\fi % FRENCH

\ifdefined\POLISH
\chapter{\PLph{}}
\fi % POLISH

% sections
\EN{\input{patterns/patterns_opt_dbg_EN}}
\ES{\input{patterns/patterns_opt_dbg_ES}}
\ITA{\input{patterns/patterns_opt_dbg_ITA}}
\PTBR{\input{patterns/patterns_opt_dbg_PTBR}}
\RU{\input{patterns/patterns_opt_dbg_RU}}
\THA{\input{patterns/patterns_opt_dbg_THA}}
\DE{\input{patterns/patterns_opt_dbg_DE}}
\FR{\input{patterns/patterns_opt_dbg_FR}}
\PL{\input{patterns/patterns_opt_dbg_PL}}

\RU{\section{Некоторые базовые понятия}}
\EN{\section{Some basics}}
\DE{\section{Einige Grundlagen}}
\FR{\section{Quelques bases}}
\ES{\section{\ESph{}}}
\ITA{\section{Alcune basi teoriche}}
\PTBR{\section{\PTBRph{}}}
\THA{\section{\THAph{}}}
\PL{\section{\PLph{}}}

% sections:
\EN{\input{patterns/intro_CPU_ISA_EN}}
\ES{\input{patterns/intro_CPU_ISA_ES}}
\ITA{\input{patterns/intro_CPU_ISA_ITA}}
\PTBR{\input{patterns/intro_CPU_ISA_PTBR}}
\RU{\input{patterns/intro_CPU_ISA_RU}}
\DE{\input{patterns/intro_CPU_ISA_DE}}
\FR{\input{patterns/intro_CPU_ISA_FR}}
\PL{\input{patterns/intro_CPU_ISA_PL}}

\EN{\input{patterns/numeral_EN}}
\RU{\input{patterns/numeral_RU}}
\ITA{\input{patterns/numeral_ITA}}
\DE{\input{patterns/numeral_DE}}
\FR{\input{patterns/numeral_FR}}
\PL{\input{patterns/numeral_PL}}

% chapters
\input{patterns/00_empty/main}
\input{patterns/011_ret/main}
\input{patterns/01_helloworld/main}
\input{patterns/015_prolog_epilogue/main}
\input{patterns/02_stack/main}
\input{patterns/03_printf/main}
\input{patterns/04_scanf/main}
\input{patterns/05_passing_arguments/main}
\input{patterns/06_return_results/main}
\input{patterns/061_pointers/main}
\input{patterns/065_GOTO/main}
\input{patterns/07_jcc/main}
\input{patterns/08_switch/main}
\input{patterns/09_loops/main}
\input{patterns/10_strings/main}
\input{patterns/11_arith_optimizations/main}
\input{patterns/12_FPU/main}
\input{patterns/13_arrays/main}
\input{patterns/14_bitfields/main}
\EN{\input{patterns/145_LCG/main_EN}}
\RU{\input{patterns/145_LCG/main_RU}}
\input{patterns/15_structs/main}
\input{patterns/17_unions/main}
\input{patterns/18_pointers_to_functions/main}
\input{patterns/185_64bit_in_32_env/main}

\EN{\input{patterns/19_SIMD/main_EN}}
\RU{\input{patterns/19_SIMD/main_RU}}
\DE{\input{patterns/19_SIMD/main_DE}}

\EN{\input{patterns/20_x64/main_EN}}
\RU{\input{patterns/20_x64/main_RU}}

\EN{\input{patterns/205_floating_SIMD/main_EN}}
\RU{\input{patterns/205_floating_SIMD/main_RU}}
\DE{\input{patterns/205_floating_SIMD/main_DE}}

\EN{\input{patterns/ARM/main_EN}}
\RU{\input{patterns/ARM/main_RU}}
\DE{\input{patterns/ARM/main_DE}}

\input{patterns/MIPS/main}

\ifdefined\SPANISH
\chapter{Patrones de código}
\fi % SPANISH

\ifdefined\GERMAN
\chapter{Code-Muster}
\fi % GERMAN

\ifdefined\ENGLISH
\chapter{Code Patterns}
\fi % ENGLISH

\ifdefined\ITALIAN
\chapter{Forme di codice}
\fi % ITALIAN

\ifdefined\RUSSIAN
\chapter{Образцы кода}
\fi % RUSSIAN

\ifdefined\BRAZILIAN
\chapter{Padrões de códigos}
\fi % BRAZILIAN

\ifdefined\THAI
\chapter{รูปแบบของโค้ด}
\fi % THAI

\ifdefined\FRENCH
\chapter{Modèle de code}
\fi % FRENCH

\ifdefined\POLISH
\chapter{\PLph{}}
\fi % POLISH

% sections
\EN{\input{patterns/patterns_opt_dbg_EN}}
\ES{\input{patterns/patterns_opt_dbg_ES}}
\ITA{\input{patterns/patterns_opt_dbg_ITA}}
\PTBR{\input{patterns/patterns_opt_dbg_PTBR}}
\RU{\input{patterns/patterns_opt_dbg_RU}}
\THA{\input{patterns/patterns_opt_dbg_THA}}
\DE{\input{patterns/patterns_opt_dbg_DE}}
\FR{\input{patterns/patterns_opt_dbg_FR}}
\PL{\input{patterns/patterns_opt_dbg_PL}}

\RU{\section{Некоторые базовые понятия}}
\EN{\section{Some basics}}
\DE{\section{Einige Grundlagen}}
\FR{\section{Quelques bases}}
\ES{\section{\ESph{}}}
\ITA{\section{Alcune basi teoriche}}
\PTBR{\section{\PTBRph{}}}
\THA{\section{\THAph{}}}
\PL{\section{\PLph{}}}

% sections:
\EN{\input{patterns/intro_CPU_ISA_EN}}
\ES{\input{patterns/intro_CPU_ISA_ES}}
\ITA{\input{patterns/intro_CPU_ISA_ITA}}
\PTBR{\input{patterns/intro_CPU_ISA_PTBR}}
\RU{\input{patterns/intro_CPU_ISA_RU}}
\DE{\input{patterns/intro_CPU_ISA_DE}}
\FR{\input{patterns/intro_CPU_ISA_FR}}
\PL{\input{patterns/intro_CPU_ISA_PL}}

\EN{\input{patterns/numeral_EN}}
\RU{\input{patterns/numeral_RU}}
\ITA{\input{patterns/numeral_ITA}}
\DE{\input{patterns/numeral_DE}}
\FR{\input{patterns/numeral_FR}}
\PL{\input{patterns/numeral_PL}}

% chapters
\input{patterns/00_empty/main}
\input{patterns/011_ret/main}
\input{patterns/01_helloworld/main}
\input{patterns/015_prolog_epilogue/main}
\input{patterns/02_stack/main}
\input{patterns/03_printf/main}
\input{patterns/04_scanf/main}
\input{patterns/05_passing_arguments/main}
\input{patterns/06_return_results/main}
\input{patterns/061_pointers/main}
\input{patterns/065_GOTO/main}
\input{patterns/07_jcc/main}
\input{patterns/08_switch/main}
\input{patterns/09_loops/main}
\input{patterns/10_strings/main}
\input{patterns/11_arith_optimizations/main}
\input{patterns/12_FPU/main}
\input{patterns/13_arrays/main}
\input{patterns/14_bitfields/main}
\EN{\input{patterns/145_LCG/main_EN}}
\RU{\input{patterns/145_LCG/main_RU}}
\input{patterns/15_structs/main}
\input{patterns/17_unions/main}
\input{patterns/18_pointers_to_functions/main}
\input{patterns/185_64bit_in_32_env/main}

\EN{\input{patterns/19_SIMD/main_EN}}
\RU{\input{patterns/19_SIMD/main_RU}}
\DE{\input{patterns/19_SIMD/main_DE}}

\EN{\input{patterns/20_x64/main_EN}}
\RU{\input{patterns/20_x64/main_RU}}

\EN{\input{patterns/205_floating_SIMD/main_EN}}
\RU{\input{patterns/205_floating_SIMD/main_RU}}
\DE{\input{patterns/205_floating_SIMD/main_DE}}

\EN{\input{patterns/ARM/main_EN}}
\RU{\input{patterns/ARM/main_RU}}
\DE{\input{patterns/ARM/main_DE}}

\input{patterns/MIPS/main}

\ifdefined\SPANISH
\chapter{Patrones de código}
\fi % SPANISH

\ifdefined\GERMAN
\chapter{Code-Muster}
\fi % GERMAN

\ifdefined\ENGLISH
\chapter{Code Patterns}
\fi % ENGLISH

\ifdefined\ITALIAN
\chapter{Forme di codice}
\fi % ITALIAN

\ifdefined\RUSSIAN
\chapter{Образцы кода}
\fi % RUSSIAN

\ifdefined\BRAZILIAN
\chapter{Padrões de códigos}
\fi % BRAZILIAN

\ifdefined\THAI
\chapter{รูปแบบของโค้ด}
\fi % THAI

\ifdefined\FRENCH
\chapter{Modèle de code}
\fi % FRENCH

\ifdefined\POLISH
\chapter{\PLph{}}
\fi % POLISH

% sections
\EN{\input{patterns/patterns_opt_dbg_EN}}
\ES{\input{patterns/patterns_opt_dbg_ES}}
\ITA{\input{patterns/patterns_opt_dbg_ITA}}
\PTBR{\input{patterns/patterns_opt_dbg_PTBR}}
\RU{\input{patterns/patterns_opt_dbg_RU}}
\THA{\input{patterns/patterns_opt_dbg_THA}}
\DE{\input{patterns/patterns_opt_dbg_DE}}
\FR{\input{patterns/patterns_opt_dbg_FR}}
\PL{\input{patterns/patterns_opt_dbg_PL}}

\RU{\section{Некоторые базовые понятия}}
\EN{\section{Some basics}}
\DE{\section{Einige Grundlagen}}
\FR{\section{Quelques bases}}
\ES{\section{\ESph{}}}
\ITA{\section{Alcune basi teoriche}}
\PTBR{\section{\PTBRph{}}}
\THA{\section{\THAph{}}}
\PL{\section{\PLph{}}}

% sections:
\EN{\input{patterns/intro_CPU_ISA_EN}}
\ES{\input{patterns/intro_CPU_ISA_ES}}
\ITA{\input{patterns/intro_CPU_ISA_ITA}}
\PTBR{\input{patterns/intro_CPU_ISA_PTBR}}
\RU{\input{patterns/intro_CPU_ISA_RU}}
\DE{\input{patterns/intro_CPU_ISA_DE}}
\FR{\input{patterns/intro_CPU_ISA_FR}}
\PL{\input{patterns/intro_CPU_ISA_PL}}

\EN{\input{patterns/numeral_EN}}
\RU{\input{patterns/numeral_RU}}
\ITA{\input{patterns/numeral_ITA}}
\DE{\input{patterns/numeral_DE}}
\FR{\input{patterns/numeral_FR}}
\PL{\input{patterns/numeral_PL}}

% chapters
\input{patterns/00_empty/main}
\input{patterns/011_ret/main}
\input{patterns/01_helloworld/main}
\input{patterns/015_prolog_epilogue/main}
\input{patterns/02_stack/main}
\input{patterns/03_printf/main}
\input{patterns/04_scanf/main}
\input{patterns/05_passing_arguments/main}
\input{patterns/06_return_results/main}
\input{patterns/061_pointers/main}
\input{patterns/065_GOTO/main}
\input{patterns/07_jcc/main}
\input{patterns/08_switch/main}
\input{patterns/09_loops/main}
\input{patterns/10_strings/main}
\input{patterns/11_arith_optimizations/main}
\input{patterns/12_FPU/main}
\input{patterns/13_arrays/main}
\input{patterns/14_bitfields/main}
\EN{\input{patterns/145_LCG/main_EN}}
\RU{\input{patterns/145_LCG/main_RU}}
\input{patterns/15_structs/main}
\input{patterns/17_unions/main}
\input{patterns/18_pointers_to_functions/main}
\input{patterns/185_64bit_in_32_env/main}

\EN{\input{patterns/19_SIMD/main_EN}}
\RU{\input{patterns/19_SIMD/main_RU}}
\DE{\input{patterns/19_SIMD/main_DE}}

\EN{\input{patterns/20_x64/main_EN}}
\RU{\input{patterns/20_x64/main_RU}}

\EN{\input{patterns/205_floating_SIMD/main_EN}}
\RU{\input{patterns/205_floating_SIMD/main_RU}}
\DE{\input{patterns/205_floating_SIMD/main_DE}}

\EN{\input{patterns/ARM/main_EN}}
\RU{\input{patterns/ARM/main_RU}}
\DE{\input{patterns/ARM/main_DE}}

\input{patterns/MIPS/main}

\ifdefined\SPANISH
\chapter{Patrones de código}
\fi % SPANISH

\ifdefined\GERMAN
\chapter{Code-Muster}
\fi % GERMAN

\ifdefined\ENGLISH
\chapter{Code Patterns}
\fi % ENGLISH

\ifdefined\ITALIAN
\chapter{Forme di codice}
\fi % ITALIAN

\ifdefined\RUSSIAN
\chapter{Образцы кода}
\fi % RUSSIAN

\ifdefined\BRAZILIAN
\chapter{Padrões de códigos}
\fi % BRAZILIAN

\ifdefined\THAI
\chapter{รูปแบบของโค้ด}
\fi % THAI

\ifdefined\FRENCH
\chapter{Modèle de code}
\fi % FRENCH

\ifdefined\POLISH
\chapter{\PLph{}}
\fi % POLISH

% sections
\EN{\input{patterns/patterns_opt_dbg_EN}}
\ES{\input{patterns/patterns_opt_dbg_ES}}
\ITA{\input{patterns/patterns_opt_dbg_ITA}}
\PTBR{\input{patterns/patterns_opt_dbg_PTBR}}
\RU{\input{patterns/patterns_opt_dbg_RU}}
\THA{\input{patterns/patterns_opt_dbg_THA}}
\DE{\input{patterns/patterns_opt_dbg_DE}}
\FR{\input{patterns/patterns_opt_dbg_FR}}
\PL{\input{patterns/patterns_opt_dbg_PL}}

\RU{\section{Некоторые базовые понятия}}
\EN{\section{Some basics}}
\DE{\section{Einige Grundlagen}}
\FR{\section{Quelques bases}}
\ES{\section{\ESph{}}}
\ITA{\section{Alcune basi teoriche}}
\PTBR{\section{\PTBRph{}}}
\THA{\section{\THAph{}}}
\PL{\section{\PLph{}}}

% sections:
\EN{\input{patterns/intro_CPU_ISA_EN}}
\ES{\input{patterns/intro_CPU_ISA_ES}}
\ITA{\input{patterns/intro_CPU_ISA_ITA}}
\PTBR{\input{patterns/intro_CPU_ISA_PTBR}}
\RU{\input{patterns/intro_CPU_ISA_RU}}
\DE{\input{patterns/intro_CPU_ISA_DE}}
\FR{\input{patterns/intro_CPU_ISA_FR}}
\PL{\input{patterns/intro_CPU_ISA_PL}}

\EN{\input{patterns/numeral_EN}}
\RU{\input{patterns/numeral_RU}}
\ITA{\input{patterns/numeral_ITA}}
\DE{\input{patterns/numeral_DE}}
\FR{\input{patterns/numeral_FR}}
\PL{\input{patterns/numeral_PL}}

% chapters
\input{patterns/00_empty/main}
\input{patterns/011_ret/main}
\input{patterns/01_helloworld/main}
\input{patterns/015_prolog_epilogue/main}
\input{patterns/02_stack/main}
\input{patterns/03_printf/main}
\input{patterns/04_scanf/main}
\input{patterns/05_passing_arguments/main}
\input{patterns/06_return_results/main}
\input{patterns/061_pointers/main}
\input{patterns/065_GOTO/main}
\input{patterns/07_jcc/main}
\input{patterns/08_switch/main}
\input{patterns/09_loops/main}
\input{patterns/10_strings/main}
\input{patterns/11_arith_optimizations/main}
\input{patterns/12_FPU/main}
\input{patterns/13_arrays/main}
\input{patterns/14_bitfields/main}
\EN{\input{patterns/145_LCG/main_EN}}
\RU{\input{patterns/145_LCG/main_RU}}
\input{patterns/15_structs/main}
\input{patterns/17_unions/main}
\input{patterns/18_pointers_to_functions/main}
\input{patterns/185_64bit_in_32_env/main}

\EN{\input{patterns/19_SIMD/main_EN}}
\RU{\input{patterns/19_SIMD/main_RU}}
\DE{\input{patterns/19_SIMD/main_DE}}

\EN{\input{patterns/20_x64/main_EN}}
\RU{\input{patterns/20_x64/main_RU}}

\EN{\input{patterns/205_floating_SIMD/main_EN}}
\RU{\input{patterns/205_floating_SIMD/main_RU}}
\DE{\input{patterns/205_floating_SIMD/main_DE}}

\EN{\input{patterns/ARM/main_EN}}
\RU{\input{patterns/ARM/main_RU}}
\DE{\input{patterns/ARM/main_DE}}

\input{patterns/MIPS/main}

\ifdefined\SPANISH
\chapter{Patrones de código}
\fi % SPANISH

\ifdefined\GERMAN
\chapter{Code-Muster}
\fi % GERMAN

\ifdefined\ENGLISH
\chapter{Code Patterns}
\fi % ENGLISH

\ifdefined\ITALIAN
\chapter{Forme di codice}
\fi % ITALIAN

\ifdefined\RUSSIAN
\chapter{Образцы кода}
\fi % RUSSIAN

\ifdefined\BRAZILIAN
\chapter{Padrões de códigos}
\fi % BRAZILIAN

\ifdefined\THAI
\chapter{รูปแบบของโค้ด}
\fi % THAI

\ifdefined\FRENCH
\chapter{Modèle de code}
\fi % FRENCH

\ifdefined\POLISH
\chapter{\PLph{}}
\fi % POLISH

% sections
\EN{\input{patterns/patterns_opt_dbg_EN}}
\ES{\input{patterns/patterns_opt_dbg_ES}}
\ITA{\input{patterns/patterns_opt_dbg_ITA}}
\PTBR{\input{patterns/patterns_opt_dbg_PTBR}}
\RU{\input{patterns/patterns_opt_dbg_RU}}
\THA{\input{patterns/patterns_opt_dbg_THA}}
\DE{\input{patterns/patterns_opt_dbg_DE}}
\FR{\input{patterns/patterns_opt_dbg_FR}}
\PL{\input{patterns/patterns_opt_dbg_PL}}

\RU{\section{Некоторые базовые понятия}}
\EN{\section{Some basics}}
\DE{\section{Einige Grundlagen}}
\FR{\section{Quelques bases}}
\ES{\section{\ESph{}}}
\ITA{\section{Alcune basi teoriche}}
\PTBR{\section{\PTBRph{}}}
\THA{\section{\THAph{}}}
\PL{\section{\PLph{}}}

% sections:
\EN{\input{patterns/intro_CPU_ISA_EN}}
\ES{\input{patterns/intro_CPU_ISA_ES}}
\ITA{\input{patterns/intro_CPU_ISA_ITA}}
\PTBR{\input{patterns/intro_CPU_ISA_PTBR}}
\RU{\input{patterns/intro_CPU_ISA_RU}}
\DE{\input{patterns/intro_CPU_ISA_DE}}
\FR{\input{patterns/intro_CPU_ISA_FR}}
\PL{\input{patterns/intro_CPU_ISA_PL}}

\EN{\input{patterns/numeral_EN}}
\RU{\input{patterns/numeral_RU}}
\ITA{\input{patterns/numeral_ITA}}
\DE{\input{patterns/numeral_DE}}
\FR{\input{patterns/numeral_FR}}
\PL{\input{patterns/numeral_PL}}

% chapters
\input{patterns/00_empty/main}
\input{patterns/011_ret/main}
\input{patterns/01_helloworld/main}
\input{patterns/015_prolog_epilogue/main}
\input{patterns/02_stack/main}
\input{patterns/03_printf/main}
\input{patterns/04_scanf/main}
\input{patterns/05_passing_arguments/main}
\input{patterns/06_return_results/main}
\input{patterns/061_pointers/main}
\input{patterns/065_GOTO/main}
\input{patterns/07_jcc/main}
\input{patterns/08_switch/main}
\input{patterns/09_loops/main}
\input{patterns/10_strings/main}
\input{patterns/11_arith_optimizations/main}
\input{patterns/12_FPU/main}
\input{patterns/13_arrays/main}
\input{patterns/14_bitfields/main}
\EN{\input{patterns/145_LCG/main_EN}}
\RU{\input{patterns/145_LCG/main_RU}}
\input{patterns/15_structs/main}
\input{patterns/17_unions/main}
\input{patterns/18_pointers_to_functions/main}
\input{patterns/185_64bit_in_32_env/main}

\EN{\input{patterns/19_SIMD/main_EN}}
\RU{\input{patterns/19_SIMD/main_RU}}
\DE{\input{patterns/19_SIMD/main_DE}}

\EN{\input{patterns/20_x64/main_EN}}
\RU{\input{patterns/20_x64/main_RU}}

\EN{\input{patterns/205_floating_SIMD/main_EN}}
\RU{\input{patterns/205_floating_SIMD/main_RU}}
\DE{\input{patterns/205_floating_SIMD/main_DE}}

\EN{\input{patterns/ARM/main_EN}}
\RU{\input{patterns/ARM/main_RU}}
\DE{\input{patterns/ARM/main_DE}}

\input{patterns/MIPS/main}

\EN{\input{patterns/12_FPU/main_EN}}
\RU{\input{patterns/12_FPU/main_RU}}
\DE{\input{patterns/12_FPU/main_DE}}
\FR{\input{patterns/12_FPU/main_FR}}


\ifdefined\SPANISH
\chapter{Patrones de código}
\fi % SPANISH

\ifdefined\GERMAN
\chapter{Code-Muster}
\fi % GERMAN

\ifdefined\ENGLISH
\chapter{Code Patterns}
\fi % ENGLISH

\ifdefined\ITALIAN
\chapter{Forme di codice}
\fi % ITALIAN

\ifdefined\RUSSIAN
\chapter{Образцы кода}
\fi % RUSSIAN

\ifdefined\BRAZILIAN
\chapter{Padrões de códigos}
\fi % BRAZILIAN

\ifdefined\THAI
\chapter{รูปแบบของโค้ด}
\fi % THAI

\ifdefined\FRENCH
\chapter{Modèle de code}
\fi % FRENCH

\ifdefined\POLISH
\chapter{\PLph{}}
\fi % POLISH

% sections
\EN{\input{patterns/patterns_opt_dbg_EN}}
\ES{\input{patterns/patterns_opt_dbg_ES}}
\ITA{\input{patterns/patterns_opt_dbg_ITA}}
\PTBR{\input{patterns/patterns_opt_dbg_PTBR}}
\RU{\input{patterns/patterns_opt_dbg_RU}}
\THA{\input{patterns/patterns_opt_dbg_THA}}
\DE{\input{patterns/patterns_opt_dbg_DE}}
\FR{\input{patterns/patterns_opt_dbg_FR}}
\PL{\input{patterns/patterns_opt_dbg_PL}}

\RU{\section{Некоторые базовые понятия}}
\EN{\section{Some basics}}
\DE{\section{Einige Grundlagen}}
\FR{\section{Quelques bases}}
\ES{\section{\ESph{}}}
\ITA{\section{Alcune basi teoriche}}
\PTBR{\section{\PTBRph{}}}
\THA{\section{\THAph{}}}
\PL{\section{\PLph{}}}

% sections:
\EN{\input{patterns/intro_CPU_ISA_EN}}
\ES{\input{patterns/intro_CPU_ISA_ES}}
\ITA{\input{patterns/intro_CPU_ISA_ITA}}
\PTBR{\input{patterns/intro_CPU_ISA_PTBR}}
\RU{\input{patterns/intro_CPU_ISA_RU}}
\DE{\input{patterns/intro_CPU_ISA_DE}}
\FR{\input{patterns/intro_CPU_ISA_FR}}
\PL{\input{patterns/intro_CPU_ISA_PL}}

\EN{\input{patterns/numeral_EN}}
\RU{\input{patterns/numeral_RU}}
\ITA{\input{patterns/numeral_ITA}}
\DE{\input{patterns/numeral_DE}}
\FR{\input{patterns/numeral_FR}}
\PL{\input{patterns/numeral_PL}}

% chapters
\input{patterns/00_empty/main}
\input{patterns/011_ret/main}
\input{patterns/01_helloworld/main}
\input{patterns/015_prolog_epilogue/main}
\input{patterns/02_stack/main}
\input{patterns/03_printf/main}
\input{patterns/04_scanf/main}
\input{patterns/05_passing_arguments/main}
\input{patterns/06_return_results/main}
\input{patterns/061_pointers/main}
\input{patterns/065_GOTO/main}
\input{patterns/07_jcc/main}
\input{patterns/08_switch/main}
\input{patterns/09_loops/main}
\input{patterns/10_strings/main}
\input{patterns/11_arith_optimizations/main}
\input{patterns/12_FPU/main}
\input{patterns/13_arrays/main}
\input{patterns/14_bitfields/main}
\EN{\input{patterns/145_LCG/main_EN}}
\RU{\input{patterns/145_LCG/main_RU}}
\input{patterns/15_structs/main}
\input{patterns/17_unions/main}
\input{patterns/18_pointers_to_functions/main}
\input{patterns/185_64bit_in_32_env/main}

\EN{\input{patterns/19_SIMD/main_EN}}
\RU{\input{patterns/19_SIMD/main_RU}}
\DE{\input{patterns/19_SIMD/main_DE}}

\EN{\input{patterns/20_x64/main_EN}}
\RU{\input{patterns/20_x64/main_RU}}

\EN{\input{patterns/205_floating_SIMD/main_EN}}
\RU{\input{patterns/205_floating_SIMD/main_RU}}
\DE{\input{patterns/205_floating_SIMD/main_DE}}

\EN{\input{patterns/ARM/main_EN}}
\RU{\input{patterns/ARM/main_RU}}
\DE{\input{patterns/ARM/main_DE}}

\input{patterns/MIPS/main}

\ifdefined\SPANISH
\chapter{Patrones de código}
\fi % SPANISH

\ifdefined\GERMAN
\chapter{Code-Muster}
\fi % GERMAN

\ifdefined\ENGLISH
\chapter{Code Patterns}
\fi % ENGLISH

\ifdefined\ITALIAN
\chapter{Forme di codice}
\fi % ITALIAN

\ifdefined\RUSSIAN
\chapter{Образцы кода}
\fi % RUSSIAN

\ifdefined\BRAZILIAN
\chapter{Padrões de códigos}
\fi % BRAZILIAN

\ifdefined\THAI
\chapter{รูปแบบของโค้ด}
\fi % THAI

\ifdefined\FRENCH
\chapter{Modèle de code}
\fi % FRENCH

\ifdefined\POLISH
\chapter{\PLph{}}
\fi % POLISH

% sections
\EN{\input{patterns/patterns_opt_dbg_EN}}
\ES{\input{patterns/patterns_opt_dbg_ES}}
\ITA{\input{patterns/patterns_opt_dbg_ITA}}
\PTBR{\input{patterns/patterns_opt_dbg_PTBR}}
\RU{\input{patterns/patterns_opt_dbg_RU}}
\THA{\input{patterns/patterns_opt_dbg_THA}}
\DE{\input{patterns/patterns_opt_dbg_DE}}
\FR{\input{patterns/patterns_opt_dbg_FR}}
\PL{\input{patterns/patterns_opt_dbg_PL}}

\RU{\section{Некоторые базовые понятия}}
\EN{\section{Some basics}}
\DE{\section{Einige Grundlagen}}
\FR{\section{Quelques bases}}
\ES{\section{\ESph{}}}
\ITA{\section{Alcune basi teoriche}}
\PTBR{\section{\PTBRph{}}}
\THA{\section{\THAph{}}}
\PL{\section{\PLph{}}}

% sections:
\EN{\input{patterns/intro_CPU_ISA_EN}}
\ES{\input{patterns/intro_CPU_ISA_ES}}
\ITA{\input{patterns/intro_CPU_ISA_ITA}}
\PTBR{\input{patterns/intro_CPU_ISA_PTBR}}
\RU{\input{patterns/intro_CPU_ISA_RU}}
\DE{\input{patterns/intro_CPU_ISA_DE}}
\FR{\input{patterns/intro_CPU_ISA_FR}}
\PL{\input{patterns/intro_CPU_ISA_PL}}

\EN{\input{patterns/numeral_EN}}
\RU{\input{patterns/numeral_RU}}
\ITA{\input{patterns/numeral_ITA}}
\DE{\input{patterns/numeral_DE}}
\FR{\input{patterns/numeral_FR}}
\PL{\input{patterns/numeral_PL}}

% chapters
\input{patterns/00_empty/main}
\input{patterns/011_ret/main}
\input{patterns/01_helloworld/main}
\input{patterns/015_prolog_epilogue/main}
\input{patterns/02_stack/main}
\input{patterns/03_printf/main}
\input{patterns/04_scanf/main}
\input{patterns/05_passing_arguments/main}
\input{patterns/06_return_results/main}
\input{patterns/061_pointers/main}
\input{patterns/065_GOTO/main}
\input{patterns/07_jcc/main}
\input{patterns/08_switch/main}
\input{patterns/09_loops/main}
\input{patterns/10_strings/main}
\input{patterns/11_arith_optimizations/main}
\input{patterns/12_FPU/main}
\input{patterns/13_arrays/main}
\input{patterns/14_bitfields/main}
\EN{\input{patterns/145_LCG/main_EN}}
\RU{\input{patterns/145_LCG/main_RU}}
\input{patterns/15_structs/main}
\input{patterns/17_unions/main}
\input{patterns/18_pointers_to_functions/main}
\input{patterns/185_64bit_in_32_env/main}

\EN{\input{patterns/19_SIMD/main_EN}}
\RU{\input{patterns/19_SIMD/main_RU}}
\DE{\input{patterns/19_SIMD/main_DE}}

\EN{\input{patterns/20_x64/main_EN}}
\RU{\input{patterns/20_x64/main_RU}}

\EN{\input{patterns/205_floating_SIMD/main_EN}}
\RU{\input{patterns/205_floating_SIMD/main_RU}}
\DE{\input{patterns/205_floating_SIMD/main_DE}}

\EN{\input{patterns/ARM/main_EN}}
\RU{\input{patterns/ARM/main_RU}}
\DE{\input{patterns/ARM/main_DE}}

\input{patterns/MIPS/main}

\EN{\section{Returning Values}
\label{ret_val_func}

Another simple function is the one that simply returns a constant value:

\lstinputlisting[caption=\EN{\CCpp Code},style=customc]{patterns/011_ret/1.c}

Let's compile it.

\subsection{x86}

Here's what both the GCC and MSVC compilers produce (with optimization) on the x86 platform:

\lstinputlisting[caption=\Optimizing GCC/MSVC (\assemblyOutput),style=customasmx86]{patterns/011_ret/1.s}

\myindex{x86!\Instructions!RET}
There are just two instructions: the first places the value 123 into the \EAX register,
which is used by convention for storing the return
value, and the second one is \RET, which returns execution to the \gls{caller}.

The caller will take the result from the \EAX register.

\subsection{ARM}

There are a few differences on the ARM platform:

\lstinputlisting[caption=\OptimizingKeilVI (\ARMMode) ASM Output,style=customasmARM]{patterns/011_ret/1_Keil_ARM_O3.s}

ARM uses the register \Reg{0} for returning the results of functions, so 123 is copied into \Reg{0}.

\myindex{ARM!\Instructions!MOV}
\myindex{x86!\Instructions!MOV}
It is worth noting that \MOV is a misleading name for the instruction in both the x86 and ARM \ac{ISA}s.

The data is not in fact \IT{moved}, but \IT{copied}.

\subsection{MIPS}

\label{MIPS_leaf_function_ex1}

The GCC assembly output below lists registers by number:

\lstinputlisting[caption=\Optimizing GCC 4.4.5 (\assemblyOutput),style=customasmMIPS]{patterns/011_ret/MIPS.s}

\dots while \IDA does it by their pseudo names:

\lstinputlisting[caption=\Optimizing GCC 4.4.5 (IDA),style=customasmMIPS]{patterns/011_ret/MIPS_IDA.lst}

The \$2 (or \$V0) register is used to store the function's return value.
\myindex{MIPS!\Pseudoinstructions!LI}
\INS{LI} stands for ``Load Immediate'' and is the MIPS equivalent to \MOV.

\myindex{MIPS!\Instructions!J}
The other instruction is the jump instruction (J or JR) which returns the execution flow to the \gls{caller}.

\myindex{MIPS!Branch delay slot}
You might be wondering why the positions of the load instruction (LI) and the jump instruction (J or JR) are swapped. This is due to a \ac{RISC} feature called ``branch delay slot''.

The reason this happens is a quirk in the architecture of some RISC \ac{ISA}s and isn't important for our
purposes---we must simply keep in mind that in MIPS, the instruction following a jump or branch instruction
is executed \IT{before} the jump/branch instruction itself.

As a consequence, branch instructions always swap places with the instruction executed immediately beforehand.


In practice, functions which merely return 1 (\IT{true}) or 0 (\IT{false}) are very frequent.

The smallest ever of the standard UNIX utilities, \IT{/bin/true} and \IT{/bin/false} return 0 and 1 respectively, as an exit code.
(Zero as an exit code usually means success, non-zero means error.)
}
\RU{\subsubsection{std::string}
\myindex{\Cpp!STL!std::string}
\label{std_string}

\myparagraph{Как устроена структура}

Многие строковые библиотеки \InSqBrackets{\CNotes 2.2} обеспечивают структуру содержащую ссылку 
на буфер собственно со строкой, переменная всегда содержащую длину строки 
(что очень удобно для массы функций \InSqBrackets{\CNotes 2.2.1}) и переменную содержащую текущий размер буфера.

Строка в буфере обыкновенно оканчивается нулем: это для того чтобы указатель на буфер можно было
передавать в функции требующие на вход обычную сишную \ac{ASCIIZ}-строку.

Стандарт \Cpp не описывает, как именно нужно реализовывать std::string,
но, как правило, они реализованы как описано выше, с небольшими дополнениями.

Строки в \Cpp это не класс (как, например, QString в Qt), а темплейт (basic\_string), 
это сделано для того чтобы поддерживать 
строки содержащие разного типа символы: как минимум \Tchar и \IT{wchar\_t}.

Так что, std::string это класс с базовым типом \Tchar.

А std::wstring это класс с базовым типом \IT{wchar\_t}.

\mysubparagraph{MSVC}

В реализации MSVC, вместо ссылки на буфер может содержаться сам буфер (если строка короче 16-и символов).

Это означает, что каждая короткая строка будет занимать в памяти по крайней мере $16 + 4 + 4 = 24$ 
байт для 32-битной среды либо $16 + 8 + 8 = 32$ 
байта в 64-битной, а если строка длиннее 16-и символов, то прибавьте еще длину самой строки.

\lstinputlisting[caption=пример для MSVC,style=customc]{\CURPATH/STL/string/MSVC_RU.cpp}

Собственно, из этого исходника почти всё ясно.

Несколько замечаний:

Если строка короче 16-и символов, 
то отдельный буфер для строки в \glslink{heap}{куче} выделяться не будет.

Это удобно потому что на практике, основная часть строк действительно короткие.
Вероятно, разработчики в Microsoft выбрали размер в 16 символов как разумный баланс.

Теперь очень важный момент в конце функции main(): мы не пользуемся методом c\_str(), тем не менее,
если это скомпилировать и запустить, то обе строки появятся в консоли!

Работает это вот почему.

В первом случае строка короче 16-и символов и в начале объекта std::string (его можно рассматривать
просто как структуру) расположен буфер с этой строкой.
\printf трактует указатель как указатель на массив символов оканчивающийся нулем и поэтому всё работает.

Вывод второй строки (длиннее 16-и символов) даже еще опаснее: это вообще типичная программистская ошибка 
(или опечатка), забыть дописать c\_str().
Это работает потому что в это время в начале структуры расположен указатель на буфер.
Это может надолго остаться незамеченным: до тех пока там не появится строка 
короче 16-и символов, тогда процесс упадет.

\mysubparagraph{GCC}

В реализации GCC в структуре есть еще одна переменная --- reference count.

Интересно, что указатель на экземпляр класса std::string в GCC указывает не на начало самой структуры, 
а на указатель на буфера.
В libstdc++-v3\textbackslash{}include\textbackslash{}bits\textbackslash{}basic\_string.h 
мы можем прочитать что это сделано для удобства отладки:

\begin{lstlisting}
   *  The reason you want _M_data pointing to the character %array and
   *  not the _Rep is so that the debugger can see the string
   *  contents. (Probably we should add a non-inline member to get
   *  the _Rep for the debugger to use, so users can check the actual
   *  string length.)
\end{lstlisting}

\href{http://go.yurichev.com/17085}{исходный код basic\_string.h}

В нашем примере мы учитываем это:

\lstinputlisting[caption=пример для GCC,style=customc]{\CURPATH/STL/string/GCC_RU.cpp}

Нужны еще небольшие хаки чтобы сымитировать типичную ошибку, которую мы уже видели выше, из-за
более ужесточенной проверки типов в GCC, тем не менее, printf() работает и здесь без c\_str().

\myparagraph{Чуть более сложный пример}

\lstinputlisting[style=customc]{\CURPATH/STL/string/3.cpp}

\lstinputlisting[caption=MSVC 2012,style=customasmx86]{\CURPATH/STL/string/3_MSVC_RU.asm}

Собственно, компилятор не конструирует строки статически: да в общем-то и как
это возможно, если буфер с ней нужно хранить в \glslink{heap}{куче}?

Вместо этого в сегменте данных хранятся обычные \ac{ASCIIZ}-строки, а позже, во время выполнения, 
при помощи метода \q{assign}, конструируются строки s1 и s2
.
При помощи \TT{operator+}, создается строка s3.

Обратите внимание на то что вызов метода c\_str() отсутствует,
потому что его код достаточно короткий и компилятор вставил его прямо здесь:
если строка короче 16-и байт, то в регистре EAX остается указатель на буфер,
а если длиннее, то из этого же места достается адрес на буфер расположенный в \glslink{heap}{куче}.

Далее следуют вызовы трех деструкторов, причем, они вызываются только если строка длиннее 16-и байт:
тогда нужно освободить буфера в \glslink{heap}{куче}.
В противном случае, так как все три объекта std::string хранятся в стеке,
они освобождаются автоматически после выхода из функции.

Следовательно, работа с короткими строками более быстрая из-за м\'{е}ньшего обращения к \glslink{heap}{куче}.

Код на GCC даже проще (из-за того, что в GCC, как мы уже видели, не реализована возможность хранить короткую
строку прямо в структуре):

% TODO1 comment each function meaning
\lstinputlisting[caption=GCC 4.8.1,style=customasmx86]{\CURPATH/STL/string/3_GCC_RU.s}

Можно заметить, что в деструкторы передается не указатель на объект,
а указатель на место за 12 байт (или 3 слова) перед ним, то есть, на настоящее начало структуры.

\myparagraph{std::string как глобальная переменная}
\label{sec:std_string_as_global_variable}

Опытные программисты на \Cpp знают, что глобальные переменные \ac{STL}-типов вполне можно объявлять.

Да, действительно:

\lstinputlisting[style=customc]{\CURPATH/STL/string/5.cpp}

Но как и где будет вызываться конструктор \TT{std::string}?

На самом деле, эта переменная будет инициализирована даже перед началом \main.

\lstinputlisting[caption=MSVC 2012: здесь конструируется глобальная переменная{,} а также регистрируется её деструктор,style=customasmx86]{\CURPATH/STL/string/5_MSVC_p2.asm}

\lstinputlisting[caption=MSVC 2012: здесь глобальная переменная используется в \main,style=customasmx86]{\CURPATH/STL/string/5_MSVC_p1.asm}

\lstinputlisting[caption=MSVC 2012: эта функция-деструктор вызывается перед выходом,style=customasmx86]{\CURPATH/STL/string/5_MSVC_p3.asm}

\myindex{\CStandardLibrary!atexit()}
В реальности, из \ac{CRT}, еще до вызова main(), вызывается специальная функция,
в которой перечислены все конструкторы подобных переменных.
Более того: при помощи atexit() регистрируется функция, которая будет вызвана в конце работы программы:
в этой функции компилятор собирает вызовы деструкторов всех подобных глобальных переменных.

GCC работает похожим образом:

\lstinputlisting[caption=GCC 4.8.1,style=customasmx86]{\CURPATH/STL/string/5_GCC.s}

Но он не выделяет отдельной функции в которой будут собраны деструкторы: 
каждый деструктор передается в atexit() по одному.

% TODO а если глобальная STL-переменная в другом модуле? надо проверить.

}
\ifdefined\SPANISH
\chapter{Patrones de código}
\fi % SPANISH

\ifdefined\GERMAN
\chapter{Code-Muster}
\fi % GERMAN

\ifdefined\ENGLISH
\chapter{Code Patterns}
\fi % ENGLISH

\ifdefined\ITALIAN
\chapter{Forme di codice}
\fi % ITALIAN

\ifdefined\RUSSIAN
\chapter{Образцы кода}
\fi % RUSSIAN

\ifdefined\BRAZILIAN
\chapter{Padrões de códigos}
\fi % BRAZILIAN

\ifdefined\THAI
\chapter{รูปแบบของโค้ด}
\fi % THAI

\ifdefined\FRENCH
\chapter{Modèle de code}
\fi % FRENCH

\ifdefined\POLISH
\chapter{\PLph{}}
\fi % POLISH

% sections
\EN{\input{patterns/patterns_opt_dbg_EN}}
\ES{\input{patterns/patterns_opt_dbg_ES}}
\ITA{\input{patterns/patterns_opt_dbg_ITA}}
\PTBR{\input{patterns/patterns_opt_dbg_PTBR}}
\RU{\input{patterns/patterns_opt_dbg_RU}}
\THA{\input{patterns/patterns_opt_dbg_THA}}
\DE{\input{patterns/patterns_opt_dbg_DE}}
\FR{\input{patterns/patterns_opt_dbg_FR}}
\PL{\input{patterns/patterns_opt_dbg_PL}}

\RU{\section{Некоторые базовые понятия}}
\EN{\section{Some basics}}
\DE{\section{Einige Grundlagen}}
\FR{\section{Quelques bases}}
\ES{\section{\ESph{}}}
\ITA{\section{Alcune basi teoriche}}
\PTBR{\section{\PTBRph{}}}
\THA{\section{\THAph{}}}
\PL{\section{\PLph{}}}

% sections:
\EN{\input{patterns/intro_CPU_ISA_EN}}
\ES{\input{patterns/intro_CPU_ISA_ES}}
\ITA{\input{patterns/intro_CPU_ISA_ITA}}
\PTBR{\input{patterns/intro_CPU_ISA_PTBR}}
\RU{\input{patterns/intro_CPU_ISA_RU}}
\DE{\input{patterns/intro_CPU_ISA_DE}}
\FR{\input{patterns/intro_CPU_ISA_FR}}
\PL{\input{patterns/intro_CPU_ISA_PL}}

\EN{\input{patterns/numeral_EN}}
\RU{\input{patterns/numeral_RU}}
\ITA{\input{patterns/numeral_ITA}}
\DE{\input{patterns/numeral_DE}}
\FR{\input{patterns/numeral_FR}}
\PL{\input{patterns/numeral_PL}}

% chapters
\input{patterns/00_empty/main}
\input{patterns/011_ret/main}
\input{patterns/01_helloworld/main}
\input{patterns/015_prolog_epilogue/main}
\input{patterns/02_stack/main}
\input{patterns/03_printf/main}
\input{patterns/04_scanf/main}
\input{patterns/05_passing_arguments/main}
\input{patterns/06_return_results/main}
\input{patterns/061_pointers/main}
\input{patterns/065_GOTO/main}
\input{patterns/07_jcc/main}
\input{patterns/08_switch/main}
\input{patterns/09_loops/main}
\input{patterns/10_strings/main}
\input{patterns/11_arith_optimizations/main}
\input{patterns/12_FPU/main}
\input{patterns/13_arrays/main}
\input{patterns/14_bitfields/main}
\EN{\input{patterns/145_LCG/main_EN}}
\RU{\input{patterns/145_LCG/main_RU}}
\input{patterns/15_structs/main}
\input{patterns/17_unions/main}
\input{patterns/18_pointers_to_functions/main}
\input{patterns/185_64bit_in_32_env/main}

\EN{\input{patterns/19_SIMD/main_EN}}
\RU{\input{patterns/19_SIMD/main_RU}}
\DE{\input{patterns/19_SIMD/main_DE}}

\EN{\input{patterns/20_x64/main_EN}}
\RU{\input{patterns/20_x64/main_RU}}

\EN{\input{patterns/205_floating_SIMD/main_EN}}
\RU{\input{patterns/205_floating_SIMD/main_RU}}
\DE{\input{patterns/205_floating_SIMD/main_DE}}

\EN{\input{patterns/ARM/main_EN}}
\RU{\input{patterns/ARM/main_RU}}
\DE{\input{patterns/ARM/main_DE}}

\input{patterns/MIPS/main}

\ifdefined\SPANISH
\chapter{Patrones de código}
\fi % SPANISH

\ifdefined\GERMAN
\chapter{Code-Muster}
\fi % GERMAN

\ifdefined\ENGLISH
\chapter{Code Patterns}
\fi % ENGLISH

\ifdefined\ITALIAN
\chapter{Forme di codice}
\fi % ITALIAN

\ifdefined\RUSSIAN
\chapter{Образцы кода}
\fi % RUSSIAN

\ifdefined\BRAZILIAN
\chapter{Padrões de códigos}
\fi % BRAZILIAN

\ifdefined\THAI
\chapter{รูปแบบของโค้ด}
\fi % THAI

\ifdefined\FRENCH
\chapter{Modèle de code}
\fi % FRENCH

\ifdefined\POLISH
\chapter{\PLph{}}
\fi % POLISH

% sections
\EN{\input{patterns/patterns_opt_dbg_EN}}
\ES{\input{patterns/patterns_opt_dbg_ES}}
\ITA{\input{patterns/patterns_opt_dbg_ITA}}
\PTBR{\input{patterns/patterns_opt_dbg_PTBR}}
\RU{\input{patterns/patterns_opt_dbg_RU}}
\THA{\input{patterns/patterns_opt_dbg_THA}}
\DE{\input{patterns/patterns_opt_dbg_DE}}
\FR{\input{patterns/patterns_opt_dbg_FR}}
\PL{\input{patterns/patterns_opt_dbg_PL}}

\RU{\section{Некоторые базовые понятия}}
\EN{\section{Some basics}}
\DE{\section{Einige Grundlagen}}
\FR{\section{Quelques bases}}
\ES{\section{\ESph{}}}
\ITA{\section{Alcune basi teoriche}}
\PTBR{\section{\PTBRph{}}}
\THA{\section{\THAph{}}}
\PL{\section{\PLph{}}}

% sections:
\EN{\input{patterns/intro_CPU_ISA_EN}}
\ES{\input{patterns/intro_CPU_ISA_ES}}
\ITA{\input{patterns/intro_CPU_ISA_ITA}}
\PTBR{\input{patterns/intro_CPU_ISA_PTBR}}
\RU{\input{patterns/intro_CPU_ISA_RU}}
\DE{\input{patterns/intro_CPU_ISA_DE}}
\FR{\input{patterns/intro_CPU_ISA_FR}}
\PL{\input{patterns/intro_CPU_ISA_PL}}

\EN{\input{patterns/numeral_EN}}
\RU{\input{patterns/numeral_RU}}
\ITA{\input{patterns/numeral_ITA}}
\DE{\input{patterns/numeral_DE}}
\FR{\input{patterns/numeral_FR}}
\PL{\input{patterns/numeral_PL}}

% chapters
\input{patterns/00_empty/main}
\input{patterns/011_ret/main}
\input{patterns/01_helloworld/main}
\input{patterns/015_prolog_epilogue/main}
\input{patterns/02_stack/main}
\input{patterns/03_printf/main}
\input{patterns/04_scanf/main}
\input{patterns/05_passing_arguments/main}
\input{patterns/06_return_results/main}
\input{patterns/061_pointers/main}
\input{patterns/065_GOTO/main}
\input{patterns/07_jcc/main}
\input{patterns/08_switch/main}
\input{patterns/09_loops/main}
\input{patterns/10_strings/main}
\input{patterns/11_arith_optimizations/main}
\input{patterns/12_FPU/main}
\input{patterns/13_arrays/main}
\input{patterns/14_bitfields/main}
\EN{\input{patterns/145_LCG/main_EN}}
\RU{\input{patterns/145_LCG/main_RU}}
\input{patterns/15_structs/main}
\input{patterns/17_unions/main}
\input{patterns/18_pointers_to_functions/main}
\input{patterns/185_64bit_in_32_env/main}

\EN{\input{patterns/19_SIMD/main_EN}}
\RU{\input{patterns/19_SIMD/main_RU}}
\DE{\input{patterns/19_SIMD/main_DE}}

\EN{\input{patterns/20_x64/main_EN}}
\RU{\input{patterns/20_x64/main_RU}}

\EN{\input{patterns/205_floating_SIMD/main_EN}}
\RU{\input{patterns/205_floating_SIMD/main_RU}}
\DE{\input{patterns/205_floating_SIMD/main_DE}}

\EN{\input{patterns/ARM/main_EN}}
\RU{\input{patterns/ARM/main_RU}}
\DE{\input{patterns/ARM/main_DE}}

\input{patterns/MIPS/main}

\ifdefined\SPANISH
\chapter{Patrones de código}
\fi % SPANISH

\ifdefined\GERMAN
\chapter{Code-Muster}
\fi % GERMAN

\ifdefined\ENGLISH
\chapter{Code Patterns}
\fi % ENGLISH

\ifdefined\ITALIAN
\chapter{Forme di codice}
\fi % ITALIAN

\ifdefined\RUSSIAN
\chapter{Образцы кода}
\fi % RUSSIAN

\ifdefined\BRAZILIAN
\chapter{Padrões de códigos}
\fi % BRAZILIAN

\ifdefined\THAI
\chapter{รูปแบบของโค้ด}
\fi % THAI

\ifdefined\FRENCH
\chapter{Modèle de code}
\fi % FRENCH

\ifdefined\POLISH
\chapter{\PLph{}}
\fi % POLISH

% sections
\EN{\input{patterns/patterns_opt_dbg_EN}}
\ES{\input{patterns/patterns_opt_dbg_ES}}
\ITA{\input{patterns/patterns_opt_dbg_ITA}}
\PTBR{\input{patterns/patterns_opt_dbg_PTBR}}
\RU{\input{patterns/patterns_opt_dbg_RU}}
\THA{\input{patterns/patterns_opt_dbg_THA}}
\DE{\input{patterns/patterns_opt_dbg_DE}}
\FR{\input{patterns/patterns_opt_dbg_FR}}
\PL{\input{patterns/patterns_opt_dbg_PL}}

\RU{\section{Некоторые базовые понятия}}
\EN{\section{Some basics}}
\DE{\section{Einige Grundlagen}}
\FR{\section{Quelques bases}}
\ES{\section{\ESph{}}}
\ITA{\section{Alcune basi teoriche}}
\PTBR{\section{\PTBRph{}}}
\THA{\section{\THAph{}}}
\PL{\section{\PLph{}}}

% sections:
\EN{\input{patterns/intro_CPU_ISA_EN}}
\ES{\input{patterns/intro_CPU_ISA_ES}}
\ITA{\input{patterns/intro_CPU_ISA_ITA}}
\PTBR{\input{patterns/intro_CPU_ISA_PTBR}}
\RU{\input{patterns/intro_CPU_ISA_RU}}
\DE{\input{patterns/intro_CPU_ISA_DE}}
\FR{\input{patterns/intro_CPU_ISA_FR}}
\PL{\input{patterns/intro_CPU_ISA_PL}}

\EN{\input{patterns/numeral_EN}}
\RU{\input{patterns/numeral_RU}}
\ITA{\input{patterns/numeral_ITA}}
\DE{\input{patterns/numeral_DE}}
\FR{\input{patterns/numeral_FR}}
\PL{\input{patterns/numeral_PL}}

% chapters
\input{patterns/00_empty/main}
\input{patterns/011_ret/main}
\input{patterns/01_helloworld/main}
\input{patterns/015_prolog_epilogue/main}
\input{patterns/02_stack/main}
\input{patterns/03_printf/main}
\input{patterns/04_scanf/main}
\input{patterns/05_passing_arguments/main}
\input{patterns/06_return_results/main}
\input{patterns/061_pointers/main}
\input{patterns/065_GOTO/main}
\input{patterns/07_jcc/main}
\input{patterns/08_switch/main}
\input{patterns/09_loops/main}
\input{patterns/10_strings/main}
\input{patterns/11_arith_optimizations/main}
\input{patterns/12_FPU/main}
\input{patterns/13_arrays/main}
\input{patterns/14_bitfields/main}
\EN{\input{patterns/145_LCG/main_EN}}
\RU{\input{patterns/145_LCG/main_RU}}
\input{patterns/15_structs/main}
\input{patterns/17_unions/main}
\input{patterns/18_pointers_to_functions/main}
\input{patterns/185_64bit_in_32_env/main}

\EN{\input{patterns/19_SIMD/main_EN}}
\RU{\input{patterns/19_SIMD/main_RU}}
\DE{\input{patterns/19_SIMD/main_DE}}

\EN{\input{patterns/20_x64/main_EN}}
\RU{\input{patterns/20_x64/main_RU}}

\EN{\input{patterns/205_floating_SIMD/main_EN}}
\RU{\input{patterns/205_floating_SIMD/main_RU}}
\DE{\input{patterns/205_floating_SIMD/main_DE}}

\EN{\input{patterns/ARM/main_EN}}
\RU{\input{patterns/ARM/main_RU}}
\DE{\input{patterns/ARM/main_DE}}

\input{patterns/MIPS/main}

\ifdefined\SPANISH
\chapter{Patrones de código}
\fi % SPANISH

\ifdefined\GERMAN
\chapter{Code-Muster}
\fi % GERMAN

\ifdefined\ENGLISH
\chapter{Code Patterns}
\fi % ENGLISH

\ifdefined\ITALIAN
\chapter{Forme di codice}
\fi % ITALIAN

\ifdefined\RUSSIAN
\chapter{Образцы кода}
\fi % RUSSIAN

\ifdefined\BRAZILIAN
\chapter{Padrões de códigos}
\fi % BRAZILIAN

\ifdefined\THAI
\chapter{รูปแบบของโค้ด}
\fi % THAI

\ifdefined\FRENCH
\chapter{Modèle de code}
\fi % FRENCH

\ifdefined\POLISH
\chapter{\PLph{}}
\fi % POLISH

% sections
\EN{\input{patterns/patterns_opt_dbg_EN}}
\ES{\input{patterns/patterns_opt_dbg_ES}}
\ITA{\input{patterns/patterns_opt_dbg_ITA}}
\PTBR{\input{patterns/patterns_opt_dbg_PTBR}}
\RU{\input{patterns/patterns_opt_dbg_RU}}
\THA{\input{patterns/patterns_opt_dbg_THA}}
\DE{\input{patterns/patterns_opt_dbg_DE}}
\FR{\input{patterns/patterns_opt_dbg_FR}}
\PL{\input{patterns/patterns_opt_dbg_PL}}

\RU{\section{Некоторые базовые понятия}}
\EN{\section{Some basics}}
\DE{\section{Einige Grundlagen}}
\FR{\section{Quelques bases}}
\ES{\section{\ESph{}}}
\ITA{\section{Alcune basi teoriche}}
\PTBR{\section{\PTBRph{}}}
\THA{\section{\THAph{}}}
\PL{\section{\PLph{}}}

% sections:
\EN{\input{patterns/intro_CPU_ISA_EN}}
\ES{\input{patterns/intro_CPU_ISA_ES}}
\ITA{\input{patterns/intro_CPU_ISA_ITA}}
\PTBR{\input{patterns/intro_CPU_ISA_PTBR}}
\RU{\input{patterns/intro_CPU_ISA_RU}}
\DE{\input{patterns/intro_CPU_ISA_DE}}
\FR{\input{patterns/intro_CPU_ISA_FR}}
\PL{\input{patterns/intro_CPU_ISA_PL}}

\EN{\input{patterns/numeral_EN}}
\RU{\input{patterns/numeral_RU}}
\ITA{\input{patterns/numeral_ITA}}
\DE{\input{patterns/numeral_DE}}
\FR{\input{patterns/numeral_FR}}
\PL{\input{patterns/numeral_PL}}

% chapters
\input{patterns/00_empty/main}
\input{patterns/011_ret/main}
\input{patterns/01_helloworld/main}
\input{patterns/015_prolog_epilogue/main}
\input{patterns/02_stack/main}
\input{patterns/03_printf/main}
\input{patterns/04_scanf/main}
\input{patterns/05_passing_arguments/main}
\input{patterns/06_return_results/main}
\input{patterns/061_pointers/main}
\input{patterns/065_GOTO/main}
\input{patterns/07_jcc/main}
\input{patterns/08_switch/main}
\input{patterns/09_loops/main}
\input{patterns/10_strings/main}
\input{patterns/11_arith_optimizations/main}
\input{patterns/12_FPU/main}
\input{patterns/13_arrays/main}
\input{patterns/14_bitfields/main}
\EN{\input{patterns/145_LCG/main_EN}}
\RU{\input{patterns/145_LCG/main_RU}}
\input{patterns/15_structs/main}
\input{patterns/17_unions/main}
\input{patterns/18_pointers_to_functions/main}
\input{patterns/185_64bit_in_32_env/main}

\EN{\input{patterns/19_SIMD/main_EN}}
\RU{\input{patterns/19_SIMD/main_RU}}
\DE{\input{patterns/19_SIMD/main_DE}}

\EN{\input{patterns/20_x64/main_EN}}
\RU{\input{patterns/20_x64/main_RU}}

\EN{\input{patterns/205_floating_SIMD/main_EN}}
\RU{\input{patterns/205_floating_SIMD/main_RU}}
\DE{\input{patterns/205_floating_SIMD/main_DE}}

\EN{\input{patterns/ARM/main_EN}}
\RU{\input{patterns/ARM/main_RU}}
\DE{\input{patterns/ARM/main_DE}}

\input{patterns/MIPS/main}


\EN{\section{Returning Values}
\label{ret_val_func}

Another simple function is the one that simply returns a constant value:

\lstinputlisting[caption=\EN{\CCpp Code},style=customc]{patterns/011_ret/1.c}

Let's compile it.

\subsection{x86}

Here's what both the GCC and MSVC compilers produce (with optimization) on the x86 platform:

\lstinputlisting[caption=\Optimizing GCC/MSVC (\assemblyOutput),style=customasmx86]{patterns/011_ret/1.s}

\myindex{x86!\Instructions!RET}
There are just two instructions: the first places the value 123 into the \EAX register,
which is used by convention for storing the return
value, and the second one is \RET, which returns execution to the \gls{caller}.

The caller will take the result from the \EAX register.

\subsection{ARM}

There are a few differences on the ARM platform:

\lstinputlisting[caption=\OptimizingKeilVI (\ARMMode) ASM Output,style=customasmARM]{patterns/011_ret/1_Keil_ARM_O3.s}

ARM uses the register \Reg{0} for returning the results of functions, so 123 is copied into \Reg{0}.

\myindex{ARM!\Instructions!MOV}
\myindex{x86!\Instructions!MOV}
It is worth noting that \MOV is a misleading name for the instruction in both the x86 and ARM \ac{ISA}s.

The data is not in fact \IT{moved}, but \IT{copied}.

\subsection{MIPS}

\label{MIPS_leaf_function_ex1}

The GCC assembly output below lists registers by number:

\lstinputlisting[caption=\Optimizing GCC 4.4.5 (\assemblyOutput),style=customasmMIPS]{patterns/011_ret/MIPS.s}

\dots while \IDA does it by their pseudo names:

\lstinputlisting[caption=\Optimizing GCC 4.4.5 (IDA),style=customasmMIPS]{patterns/011_ret/MIPS_IDA.lst}

The \$2 (or \$V0) register is used to store the function's return value.
\myindex{MIPS!\Pseudoinstructions!LI}
\INS{LI} stands for ``Load Immediate'' and is the MIPS equivalent to \MOV.

\myindex{MIPS!\Instructions!J}
The other instruction is the jump instruction (J or JR) which returns the execution flow to the \gls{caller}.

\myindex{MIPS!Branch delay slot}
You might be wondering why the positions of the load instruction (LI) and the jump instruction (J or JR) are swapped. This is due to a \ac{RISC} feature called ``branch delay slot''.

The reason this happens is a quirk in the architecture of some RISC \ac{ISA}s and isn't important for our
purposes---we must simply keep in mind that in MIPS, the instruction following a jump or branch instruction
is executed \IT{before} the jump/branch instruction itself.

As a consequence, branch instructions always swap places with the instruction executed immediately beforehand.


In practice, functions which merely return 1 (\IT{true}) or 0 (\IT{false}) are very frequent.

The smallest ever of the standard UNIX utilities, \IT{/bin/true} and \IT{/bin/false} return 0 and 1 respectively, as an exit code.
(Zero as an exit code usually means success, non-zero means error.)
}
\RU{\subsubsection{std::string}
\myindex{\Cpp!STL!std::string}
\label{std_string}

\myparagraph{Как устроена структура}

Многие строковые библиотеки \InSqBrackets{\CNotes 2.2} обеспечивают структуру содержащую ссылку 
на буфер собственно со строкой, переменная всегда содержащую длину строки 
(что очень удобно для массы функций \InSqBrackets{\CNotes 2.2.1}) и переменную содержащую текущий размер буфера.

Строка в буфере обыкновенно оканчивается нулем: это для того чтобы указатель на буфер можно было
передавать в функции требующие на вход обычную сишную \ac{ASCIIZ}-строку.

Стандарт \Cpp не описывает, как именно нужно реализовывать std::string,
но, как правило, они реализованы как описано выше, с небольшими дополнениями.

Строки в \Cpp это не класс (как, например, QString в Qt), а темплейт (basic\_string), 
это сделано для того чтобы поддерживать 
строки содержащие разного типа символы: как минимум \Tchar и \IT{wchar\_t}.

Так что, std::string это класс с базовым типом \Tchar.

А std::wstring это класс с базовым типом \IT{wchar\_t}.

\mysubparagraph{MSVC}

В реализации MSVC, вместо ссылки на буфер может содержаться сам буфер (если строка короче 16-и символов).

Это означает, что каждая короткая строка будет занимать в памяти по крайней мере $16 + 4 + 4 = 24$ 
байт для 32-битной среды либо $16 + 8 + 8 = 32$ 
байта в 64-битной, а если строка длиннее 16-и символов, то прибавьте еще длину самой строки.

\lstinputlisting[caption=пример для MSVC,style=customc]{\CURPATH/STL/string/MSVC_RU.cpp}

Собственно, из этого исходника почти всё ясно.

Несколько замечаний:

Если строка короче 16-и символов, 
то отдельный буфер для строки в \glslink{heap}{куче} выделяться не будет.

Это удобно потому что на практике, основная часть строк действительно короткие.
Вероятно, разработчики в Microsoft выбрали размер в 16 символов как разумный баланс.

Теперь очень важный момент в конце функции main(): мы не пользуемся методом c\_str(), тем не менее,
если это скомпилировать и запустить, то обе строки появятся в консоли!

Работает это вот почему.

В первом случае строка короче 16-и символов и в начале объекта std::string (его можно рассматривать
просто как структуру) расположен буфер с этой строкой.
\printf трактует указатель как указатель на массив символов оканчивающийся нулем и поэтому всё работает.

Вывод второй строки (длиннее 16-и символов) даже еще опаснее: это вообще типичная программистская ошибка 
(или опечатка), забыть дописать c\_str().
Это работает потому что в это время в начале структуры расположен указатель на буфер.
Это может надолго остаться незамеченным: до тех пока там не появится строка 
короче 16-и символов, тогда процесс упадет.

\mysubparagraph{GCC}

В реализации GCC в структуре есть еще одна переменная --- reference count.

Интересно, что указатель на экземпляр класса std::string в GCC указывает не на начало самой структуры, 
а на указатель на буфера.
В libstdc++-v3\textbackslash{}include\textbackslash{}bits\textbackslash{}basic\_string.h 
мы можем прочитать что это сделано для удобства отладки:

\begin{lstlisting}
   *  The reason you want _M_data pointing to the character %array and
   *  not the _Rep is so that the debugger can see the string
   *  contents. (Probably we should add a non-inline member to get
   *  the _Rep for the debugger to use, so users can check the actual
   *  string length.)
\end{lstlisting}

\href{http://go.yurichev.com/17085}{исходный код basic\_string.h}

В нашем примере мы учитываем это:

\lstinputlisting[caption=пример для GCC,style=customc]{\CURPATH/STL/string/GCC_RU.cpp}

Нужны еще небольшие хаки чтобы сымитировать типичную ошибку, которую мы уже видели выше, из-за
более ужесточенной проверки типов в GCC, тем не менее, printf() работает и здесь без c\_str().

\myparagraph{Чуть более сложный пример}

\lstinputlisting[style=customc]{\CURPATH/STL/string/3.cpp}

\lstinputlisting[caption=MSVC 2012,style=customasmx86]{\CURPATH/STL/string/3_MSVC_RU.asm}

Собственно, компилятор не конструирует строки статически: да в общем-то и как
это возможно, если буфер с ней нужно хранить в \glslink{heap}{куче}?

Вместо этого в сегменте данных хранятся обычные \ac{ASCIIZ}-строки, а позже, во время выполнения, 
при помощи метода \q{assign}, конструируются строки s1 и s2
.
При помощи \TT{operator+}, создается строка s3.

Обратите внимание на то что вызов метода c\_str() отсутствует,
потому что его код достаточно короткий и компилятор вставил его прямо здесь:
если строка короче 16-и байт, то в регистре EAX остается указатель на буфер,
а если длиннее, то из этого же места достается адрес на буфер расположенный в \glslink{heap}{куче}.

Далее следуют вызовы трех деструкторов, причем, они вызываются только если строка длиннее 16-и байт:
тогда нужно освободить буфера в \glslink{heap}{куче}.
В противном случае, так как все три объекта std::string хранятся в стеке,
они освобождаются автоматически после выхода из функции.

Следовательно, работа с короткими строками более быстрая из-за м\'{е}ньшего обращения к \glslink{heap}{куче}.

Код на GCC даже проще (из-за того, что в GCC, как мы уже видели, не реализована возможность хранить короткую
строку прямо в структуре):

% TODO1 comment each function meaning
\lstinputlisting[caption=GCC 4.8.1,style=customasmx86]{\CURPATH/STL/string/3_GCC_RU.s}

Можно заметить, что в деструкторы передается не указатель на объект,
а указатель на место за 12 байт (или 3 слова) перед ним, то есть, на настоящее начало структуры.

\myparagraph{std::string как глобальная переменная}
\label{sec:std_string_as_global_variable}

Опытные программисты на \Cpp знают, что глобальные переменные \ac{STL}-типов вполне можно объявлять.

Да, действительно:

\lstinputlisting[style=customc]{\CURPATH/STL/string/5.cpp}

Но как и где будет вызываться конструктор \TT{std::string}?

На самом деле, эта переменная будет инициализирована даже перед началом \main.

\lstinputlisting[caption=MSVC 2012: здесь конструируется глобальная переменная{,} а также регистрируется её деструктор,style=customasmx86]{\CURPATH/STL/string/5_MSVC_p2.asm}

\lstinputlisting[caption=MSVC 2012: здесь глобальная переменная используется в \main,style=customasmx86]{\CURPATH/STL/string/5_MSVC_p1.asm}

\lstinputlisting[caption=MSVC 2012: эта функция-деструктор вызывается перед выходом,style=customasmx86]{\CURPATH/STL/string/5_MSVC_p3.asm}

\myindex{\CStandardLibrary!atexit()}
В реальности, из \ac{CRT}, еще до вызова main(), вызывается специальная функция,
в которой перечислены все конструкторы подобных переменных.
Более того: при помощи atexit() регистрируется функция, которая будет вызвана в конце работы программы:
в этой функции компилятор собирает вызовы деструкторов всех подобных глобальных переменных.

GCC работает похожим образом:

\lstinputlisting[caption=GCC 4.8.1,style=customasmx86]{\CURPATH/STL/string/5_GCC.s}

Но он не выделяет отдельной функции в которой будут собраны деструкторы: 
каждый деструктор передается в atexit() по одному.

% TODO а если глобальная STL-переменная в другом модуле? надо проверить.

}
\DE{\subsection{Einfachste XOR-Verschlüsselung überhaupt}

Ich habe einmal eine Software gesehen, bei der alle Debugging-Ausgaben mit XOR mit dem Wert 3
verschlüsselt wurden. Mit anderen Worten, die beiden niedrigsten Bits aller Buchstaben wurden invertiert.

``Hello, world'' wurde zu ``Kfool/\#tlqog'':

\begin{lstlisting}
#!/usr/bin/python

msg="Hello, world!"

print "".join(map(lambda x: chr(ord(x)^3), msg))
\end{lstlisting}

Das ist eine ziemlich interessante Verschlüsselung (oder besser eine Verschleierung),
weil sie zwei wichtige Eigenschaften hat:
1) es ist eine einzige Funktion zum Verschlüsseln und entschlüsseln, sie muss nur wiederholt angewendet werden
2) die entstehenden Buchstaben befinden sich im druckbaren Bereich, also die ganze Zeichenkette kann ohne
Escape-Symbole im Code verwendet werden.

Die zweite Eigenschaft nutzt die Tatsache, dass alle druckbaren Zeichen in Reihen organisiert sind: 0x2x-0x7x,
und wenn die beiden niederwertigsten Bits invertiert werden, wird der Buchstabe um eine oder drei Stellen nach
links oder rechts \IT{verschoben}, aber niemals in eine andere Reihe:

\begin{figure}[H]
\centering
\includegraphics[width=0.7\textwidth]{ascii_clean.png}
\caption{7-Bit \ac{ASCII} Tabelle in Emacs}
\end{figure}

\dots mit dem Zeichen 0x7F als einziger Ausnahme.

Im Folgenden werden also beispielsweise die Zeichen A-Z \IT{verschlüsselt}:

\begin{lstlisting}
#!/usr/bin/python

msg="@ABCDEFGHIJKLMNO"

print "".join(map(lambda x: chr(ord(x)^3), msg))
\end{lstlisting}

Ergebnis:
% FIXME \verb  --  relevant comment for German?
\begin{lstlisting}
CBA@GFEDKJIHONML
\end{lstlisting}

Es sieht so aus als würden die Zeichen ``@'' und ``C'' sowie ``B'' und ``A'' vertauscht werden.

Hier ist noch ein interessantes Beispiel, in dem gezeigt wird, wie die Eigenschaften von XOR
ausgenutzt werden können: Exakt den gleichen Effekt, dass druckbare Zeichen auch druckbar bleiben,
kann man dadurch erzielen, dass irgendeine Kombination der niedrigsten vier Bits invertiert wird.
}

\EN{\section{Returning Values}
\label{ret_val_func}

Another simple function is the one that simply returns a constant value:

\lstinputlisting[caption=\EN{\CCpp Code},style=customc]{patterns/011_ret/1.c}

Let's compile it.

\subsection{x86}

Here's what both the GCC and MSVC compilers produce (with optimization) on the x86 platform:

\lstinputlisting[caption=\Optimizing GCC/MSVC (\assemblyOutput),style=customasmx86]{patterns/011_ret/1.s}

\myindex{x86!\Instructions!RET}
There are just two instructions: the first places the value 123 into the \EAX register,
which is used by convention for storing the return
value, and the second one is \RET, which returns execution to the \gls{caller}.

The caller will take the result from the \EAX register.

\subsection{ARM}

There are a few differences on the ARM platform:

\lstinputlisting[caption=\OptimizingKeilVI (\ARMMode) ASM Output,style=customasmARM]{patterns/011_ret/1_Keil_ARM_O3.s}

ARM uses the register \Reg{0} for returning the results of functions, so 123 is copied into \Reg{0}.

\myindex{ARM!\Instructions!MOV}
\myindex{x86!\Instructions!MOV}
It is worth noting that \MOV is a misleading name for the instruction in both the x86 and ARM \ac{ISA}s.

The data is not in fact \IT{moved}, but \IT{copied}.

\subsection{MIPS}

\label{MIPS_leaf_function_ex1}

The GCC assembly output below lists registers by number:

\lstinputlisting[caption=\Optimizing GCC 4.4.5 (\assemblyOutput),style=customasmMIPS]{patterns/011_ret/MIPS.s}

\dots while \IDA does it by their pseudo names:

\lstinputlisting[caption=\Optimizing GCC 4.4.5 (IDA),style=customasmMIPS]{patterns/011_ret/MIPS_IDA.lst}

The \$2 (or \$V0) register is used to store the function's return value.
\myindex{MIPS!\Pseudoinstructions!LI}
\INS{LI} stands for ``Load Immediate'' and is the MIPS equivalent to \MOV.

\myindex{MIPS!\Instructions!J}
The other instruction is the jump instruction (J or JR) which returns the execution flow to the \gls{caller}.

\myindex{MIPS!Branch delay slot}
You might be wondering why the positions of the load instruction (LI) and the jump instruction (J or JR) are swapped. This is due to a \ac{RISC} feature called ``branch delay slot''.

The reason this happens is a quirk in the architecture of some RISC \ac{ISA}s and isn't important for our
purposes---we must simply keep in mind that in MIPS, the instruction following a jump or branch instruction
is executed \IT{before} the jump/branch instruction itself.

As a consequence, branch instructions always swap places with the instruction executed immediately beforehand.


In practice, functions which merely return 1 (\IT{true}) or 0 (\IT{false}) are very frequent.

The smallest ever of the standard UNIX utilities, \IT{/bin/true} and \IT{/bin/false} return 0 and 1 respectively, as an exit code.
(Zero as an exit code usually means success, non-zero means error.)
}
\RU{\subsubsection{std::string}
\myindex{\Cpp!STL!std::string}
\label{std_string}

\myparagraph{Как устроена структура}

Многие строковые библиотеки \InSqBrackets{\CNotes 2.2} обеспечивают структуру содержащую ссылку 
на буфер собственно со строкой, переменная всегда содержащую длину строки 
(что очень удобно для массы функций \InSqBrackets{\CNotes 2.2.1}) и переменную содержащую текущий размер буфера.

Строка в буфере обыкновенно оканчивается нулем: это для того чтобы указатель на буфер можно было
передавать в функции требующие на вход обычную сишную \ac{ASCIIZ}-строку.

Стандарт \Cpp не описывает, как именно нужно реализовывать std::string,
но, как правило, они реализованы как описано выше, с небольшими дополнениями.

Строки в \Cpp это не класс (как, например, QString в Qt), а темплейт (basic\_string), 
это сделано для того чтобы поддерживать 
строки содержащие разного типа символы: как минимум \Tchar и \IT{wchar\_t}.

Так что, std::string это класс с базовым типом \Tchar.

А std::wstring это класс с базовым типом \IT{wchar\_t}.

\mysubparagraph{MSVC}

В реализации MSVC, вместо ссылки на буфер может содержаться сам буфер (если строка короче 16-и символов).

Это означает, что каждая короткая строка будет занимать в памяти по крайней мере $16 + 4 + 4 = 24$ 
байт для 32-битной среды либо $16 + 8 + 8 = 32$ 
байта в 64-битной, а если строка длиннее 16-и символов, то прибавьте еще длину самой строки.

\lstinputlisting[caption=пример для MSVC,style=customc]{\CURPATH/STL/string/MSVC_RU.cpp}

Собственно, из этого исходника почти всё ясно.

Несколько замечаний:

Если строка короче 16-и символов, 
то отдельный буфер для строки в \glslink{heap}{куче} выделяться не будет.

Это удобно потому что на практике, основная часть строк действительно короткие.
Вероятно, разработчики в Microsoft выбрали размер в 16 символов как разумный баланс.

Теперь очень важный момент в конце функции main(): мы не пользуемся методом c\_str(), тем не менее,
если это скомпилировать и запустить, то обе строки появятся в консоли!

Работает это вот почему.

В первом случае строка короче 16-и символов и в начале объекта std::string (его можно рассматривать
просто как структуру) расположен буфер с этой строкой.
\printf трактует указатель как указатель на массив символов оканчивающийся нулем и поэтому всё работает.

Вывод второй строки (длиннее 16-и символов) даже еще опаснее: это вообще типичная программистская ошибка 
(или опечатка), забыть дописать c\_str().
Это работает потому что в это время в начале структуры расположен указатель на буфер.
Это может надолго остаться незамеченным: до тех пока там не появится строка 
короче 16-и символов, тогда процесс упадет.

\mysubparagraph{GCC}

В реализации GCC в структуре есть еще одна переменная --- reference count.

Интересно, что указатель на экземпляр класса std::string в GCC указывает не на начало самой структуры, 
а на указатель на буфера.
В libstdc++-v3\textbackslash{}include\textbackslash{}bits\textbackslash{}basic\_string.h 
мы можем прочитать что это сделано для удобства отладки:

\begin{lstlisting}
   *  The reason you want _M_data pointing to the character %array and
   *  not the _Rep is so that the debugger can see the string
   *  contents. (Probably we should add a non-inline member to get
   *  the _Rep for the debugger to use, so users can check the actual
   *  string length.)
\end{lstlisting}

\href{http://go.yurichev.com/17085}{исходный код basic\_string.h}

В нашем примере мы учитываем это:

\lstinputlisting[caption=пример для GCC,style=customc]{\CURPATH/STL/string/GCC_RU.cpp}

Нужны еще небольшие хаки чтобы сымитировать типичную ошибку, которую мы уже видели выше, из-за
более ужесточенной проверки типов в GCC, тем не менее, printf() работает и здесь без c\_str().

\myparagraph{Чуть более сложный пример}

\lstinputlisting[style=customc]{\CURPATH/STL/string/3.cpp}

\lstinputlisting[caption=MSVC 2012,style=customasmx86]{\CURPATH/STL/string/3_MSVC_RU.asm}

Собственно, компилятор не конструирует строки статически: да в общем-то и как
это возможно, если буфер с ней нужно хранить в \glslink{heap}{куче}?

Вместо этого в сегменте данных хранятся обычные \ac{ASCIIZ}-строки, а позже, во время выполнения, 
при помощи метода \q{assign}, конструируются строки s1 и s2
.
При помощи \TT{operator+}, создается строка s3.

Обратите внимание на то что вызов метода c\_str() отсутствует,
потому что его код достаточно короткий и компилятор вставил его прямо здесь:
если строка короче 16-и байт, то в регистре EAX остается указатель на буфер,
а если длиннее, то из этого же места достается адрес на буфер расположенный в \glslink{heap}{куче}.

Далее следуют вызовы трех деструкторов, причем, они вызываются только если строка длиннее 16-и байт:
тогда нужно освободить буфера в \glslink{heap}{куче}.
В противном случае, так как все три объекта std::string хранятся в стеке,
они освобождаются автоматически после выхода из функции.

Следовательно, работа с короткими строками более быстрая из-за м\'{е}ньшего обращения к \glslink{heap}{куче}.

Код на GCC даже проще (из-за того, что в GCC, как мы уже видели, не реализована возможность хранить короткую
строку прямо в структуре):

% TODO1 comment each function meaning
\lstinputlisting[caption=GCC 4.8.1,style=customasmx86]{\CURPATH/STL/string/3_GCC_RU.s}

Можно заметить, что в деструкторы передается не указатель на объект,
а указатель на место за 12 байт (или 3 слова) перед ним, то есть, на настоящее начало структуры.

\myparagraph{std::string как глобальная переменная}
\label{sec:std_string_as_global_variable}

Опытные программисты на \Cpp знают, что глобальные переменные \ac{STL}-типов вполне можно объявлять.

Да, действительно:

\lstinputlisting[style=customc]{\CURPATH/STL/string/5.cpp}

Но как и где будет вызываться конструктор \TT{std::string}?

На самом деле, эта переменная будет инициализирована даже перед началом \main.

\lstinputlisting[caption=MSVC 2012: здесь конструируется глобальная переменная{,} а также регистрируется её деструктор,style=customasmx86]{\CURPATH/STL/string/5_MSVC_p2.asm}

\lstinputlisting[caption=MSVC 2012: здесь глобальная переменная используется в \main,style=customasmx86]{\CURPATH/STL/string/5_MSVC_p1.asm}

\lstinputlisting[caption=MSVC 2012: эта функция-деструктор вызывается перед выходом,style=customasmx86]{\CURPATH/STL/string/5_MSVC_p3.asm}

\myindex{\CStandardLibrary!atexit()}
В реальности, из \ac{CRT}, еще до вызова main(), вызывается специальная функция,
в которой перечислены все конструкторы подобных переменных.
Более того: при помощи atexit() регистрируется функция, которая будет вызвана в конце работы программы:
в этой функции компилятор собирает вызовы деструкторов всех подобных глобальных переменных.

GCC работает похожим образом:

\lstinputlisting[caption=GCC 4.8.1,style=customasmx86]{\CURPATH/STL/string/5_GCC.s}

Но он не выделяет отдельной функции в которой будут собраны деструкторы: 
каждый деструктор передается в atexit() по одному.

% TODO а если глобальная STL-переменная в другом модуле? надо проверить.

}

\EN{\section{Returning Values}
\label{ret_val_func}

Another simple function is the one that simply returns a constant value:

\lstinputlisting[caption=\EN{\CCpp Code},style=customc]{patterns/011_ret/1.c}

Let's compile it.

\subsection{x86}

Here's what both the GCC and MSVC compilers produce (with optimization) on the x86 platform:

\lstinputlisting[caption=\Optimizing GCC/MSVC (\assemblyOutput),style=customasmx86]{patterns/011_ret/1.s}

\myindex{x86!\Instructions!RET}
There are just two instructions: the first places the value 123 into the \EAX register,
which is used by convention for storing the return
value, and the second one is \RET, which returns execution to the \gls{caller}.

The caller will take the result from the \EAX register.

\subsection{ARM}

There are a few differences on the ARM platform:

\lstinputlisting[caption=\OptimizingKeilVI (\ARMMode) ASM Output,style=customasmARM]{patterns/011_ret/1_Keil_ARM_O3.s}

ARM uses the register \Reg{0} for returning the results of functions, so 123 is copied into \Reg{0}.

\myindex{ARM!\Instructions!MOV}
\myindex{x86!\Instructions!MOV}
It is worth noting that \MOV is a misleading name for the instruction in both the x86 and ARM \ac{ISA}s.

The data is not in fact \IT{moved}, but \IT{copied}.

\subsection{MIPS}

\label{MIPS_leaf_function_ex1}

The GCC assembly output below lists registers by number:

\lstinputlisting[caption=\Optimizing GCC 4.4.5 (\assemblyOutput),style=customasmMIPS]{patterns/011_ret/MIPS.s}

\dots while \IDA does it by their pseudo names:

\lstinputlisting[caption=\Optimizing GCC 4.4.5 (IDA),style=customasmMIPS]{patterns/011_ret/MIPS_IDA.lst}

The \$2 (or \$V0) register is used to store the function's return value.
\myindex{MIPS!\Pseudoinstructions!LI}
\INS{LI} stands for ``Load Immediate'' and is the MIPS equivalent to \MOV.

\myindex{MIPS!\Instructions!J}
The other instruction is the jump instruction (J or JR) which returns the execution flow to the \gls{caller}.

\myindex{MIPS!Branch delay slot}
You might be wondering why the positions of the load instruction (LI) and the jump instruction (J or JR) are swapped. This is due to a \ac{RISC} feature called ``branch delay slot''.

The reason this happens is a quirk in the architecture of some RISC \ac{ISA}s and isn't important for our
purposes---we must simply keep in mind that in MIPS, the instruction following a jump or branch instruction
is executed \IT{before} the jump/branch instruction itself.

As a consequence, branch instructions always swap places with the instruction executed immediately beforehand.


In practice, functions which merely return 1 (\IT{true}) or 0 (\IT{false}) are very frequent.

The smallest ever of the standard UNIX utilities, \IT{/bin/true} and \IT{/bin/false} return 0 and 1 respectively, as an exit code.
(Zero as an exit code usually means success, non-zero means error.)
}
\RU{\subsubsection{std::string}
\myindex{\Cpp!STL!std::string}
\label{std_string}

\myparagraph{Как устроена структура}

Многие строковые библиотеки \InSqBrackets{\CNotes 2.2} обеспечивают структуру содержащую ссылку 
на буфер собственно со строкой, переменная всегда содержащую длину строки 
(что очень удобно для массы функций \InSqBrackets{\CNotes 2.2.1}) и переменную содержащую текущий размер буфера.

Строка в буфере обыкновенно оканчивается нулем: это для того чтобы указатель на буфер можно было
передавать в функции требующие на вход обычную сишную \ac{ASCIIZ}-строку.

Стандарт \Cpp не описывает, как именно нужно реализовывать std::string,
но, как правило, они реализованы как описано выше, с небольшими дополнениями.

Строки в \Cpp это не класс (как, например, QString в Qt), а темплейт (basic\_string), 
это сделано для того чтобы поддерживать 
строки содержащие разного типа символы: как минимум \Tchar и \IT{wchar\_t}.

Так что, std::string это класс с базовым типом \Tchar.

А std::wstring это класс с базовым типом \IT{wchar\_t}.

\mysubparagraph{MSVC}

В реализации MSVC, вместо ссылки на буфер может содержаться сам буфер (если строка короче 16-и символов).

Это означает, что каждая короткая строка будет занимать в памяти по крайней мере $16 + 4 + 4 = 24$ 
байт для 32-битной среды либо $16 + 8 + 8 = 32$ 
байта в 64-битной, а если строка длиннее 16-и символов, то прибавьте еще длину самой строки.

\lstinputlisting[caption=пример для MSVC,style=customc]{\CURPATH/STL/string/MSVC_RU.cpp}

Собственно, из этого исходника почти всё ясно.

Несколько замечаний:

Если строка короче 16-и символов, 
то отдельный буфер для строки в \glslink{heap}{куче} выделяться не будет.

Это удобно потому что на практике, основная часть строк действительно короткие.
Вероятно, разработчики в Microsoft выбрали размер в 16 символов как разумный баланс.

Теперь очень важный момент в конце функции main(): мы не пользуемся методом c\_str(), тем не менее,
если это скомпилировать и запустить, то обе строки появятся в консоли!

Работает это вот почему.

В первом случае строка короче 16-и символов и в начале объекта std::string (его можно рассматривать
просто как структуру) расположен буфер с этой строкой.
\printf трактует указатель как указатель на массив символов оканчивающийся нулем и поэтому всё работает.

Вывод второй строки (длиннее 16-и символов) даже еще опаснее: это вообще типичная программистская ошибка 
(или опечатка), забыть дописать c\_str().
Это работает потому что в это время в начале структуры расположен указатель на буфер.
Это может надолго остаться незамеченным: до тех пока там не появится строка 
короче 16-и символов, тогда процесс упадет.

\mysubparagraph{GCC}

В реализации GCC в структуре есть еще одна переменная --- reference count.

Интересно, что указатель на экземпляр класса std::string в GCC указывает не на начало самой структуры, 
а на указатель на буфера.
В libstdc++-v3\textbackslash{}include\textbackslash{}bits\textbackslash{}basic\_string.h 
мы можем прочитать что это сделано для удобства отладки:

\begin{lstlisting}
   *  The reason you want _M_data pointing to the character %array and
   *  not the _Rep is so that the debugger can see the string
   *  contents. (Probably we should add a non-inline member to get
   *  the _Rep for the debugger to use, so users can check the actual
   *  string length.)
\end{lstlisting}

\href{http://go.yurichev.com/17085}{исходный код basic\_string.h}

В нашем примере мы учитываем это:

\lstinputlisting[caption=пример для GCC,style=customc]{\CURPATH/STL/string/GCC_RU.cpp}

Нужны еще небольшие хаки чтобы сымитировать типичную ошибку, которую мы уже видели выше, из-за
более ужесточенной проверки типов в GCC, тем не менее, printf() работает и здесь без c\_str().

\myparagraph{Чуть более сложный пример}

\lstinputlisting[style=customc]{\CURPATH/STL/string/3.cpp}

\lstinputlisting[caption=MSVC 2012,style=customasmx86]{\CURPATH/STL/string/3_MSVC_RU.asm}

Собственно, компилятор не конструирует строки статически: да в общем-то и как
это возможно, если буфер с ней нужно хранить в \glslink{heap}{куче}?

Вместо этого в сегменте данных хранятся обычные \ac{ASCIIZ}-строки, а позже, во время выполнения, 
при помощи метода \q{assign}, конструируются строки s1 и s2
.
При помощи \TT{operator+}, создается строка s3.

Обратите внимание на то что вызов метода c\_str() отсутствует,
потому что его код достаточно короткий и компилятор вставил его прямо здесь:
если строка короче 16-и байт, то в регистре EAX остается указатель на буфер,
а если длиннее, то из этого же места достается адрес на буфер расположенный в \glslink{heap}{куче}.

Далее следуют вызовы трех деструкторов, причем, они вызываются только если строка длиннее 16-и байт:
тогда нужно освободить буфера в \glslink{heap}{куче}.
В противном случае, так как все три объекта std::string хранятся в стеке,
они освобождаются автоматически после выхода из функции.

Следовательно, работа с короткими строками более быстрая из-за м\'{е}ньшего обращения к \glslink{heap}{куче}.

Код на GCC даже проще (из-за того, что в GCC, как мы уже видели, не реализована возможность хранить короткую
строку прямо в структуре):

% TODO1 comment each function meaning
\lstinputlisting[caption=GCC 4.8.1,style=customasmx86]{\CURPATH/STL/string/3_GCC_RU.s}

Можно заметить, что в деструкторы передается не указатель на объект,
а указатель на место за 12 байт (или 3 слова) перед ним, то есть, на настоящее начало структуры.

\myparagraph{std::string как глобальная переменная}
\label{sec:std_string_as_global_variable}

Опытные программисты на \Cpp знают, что глобальные переменные \ac{STL}-типов вполне можно объявлять.

Да, действительно:

\lstinputlisting[style=customc]{\CURPATH/STL/string/5.cpp}

Но как и где будет вызываться конструктор \TT{std::string}?

На самом деле, эта переменная будет инициализирована даже перед началом \main.

\lstinputlisting[caption=MSVC 2012: здесь конструируется глобальная переменная{,} а также регистрируется её деструктор,style=customasmx86]{\CURPATH/STL/string/5_MSVC_p2.asm}

\lstinputlisting[caption=MSVC 2012: здесь глобальная переменная используется в \main,style=customasmx86]{\CURPATH/STL/string/5_MSVC_p1.asm}

\lstinputlisting[caption=MSVC 2012: эта функция-деструктор вызывается перед выходом,style=customasmx86]{\CURPATH/STL/string/5_MSVC_p3.asm}

\myindex{\CStandardLibrary!atexit()}
В реальности, из \ac{CRT}, еще до вызова main(), вызывается специальная функция,
в которой перечислены все конструкторы подобных переменных.
Более того: при помощи atexit() регистрируется функция, которая будет вызвана в конце работы программы:
в этой функции компилятор собирает вызовы деструкторов всех подобных глобальных переменных.

GCC работает похожим образом:

\lstinputlisting[caption=GCC 4.8.1,style=customasmx86]{\CURPATH/STL/string/5_GCC.s}

Но он не выделяет отдельной функции в которой будут собраны деструкторы: 
каждый деструктор передается в atexit() по одному.

% TODO а если глобальная STL-переменная в другом модуле? надо проверить.

}
\DE{\subsection{Einfachste XOR-Verschlüsselung überhaupt}

Ich habe einmal eine Software gesehen, bei der alle Debugging-Ausgaben mit XOR mit dem Wert 3
verschlüsselt wurden. Mit anderen Worten, die beiden niedrigsten Bits aller Buchstaben wurden invertiert.

``Hello, world'' wurde zu ``Kfool/\#tlqog'':

\begin{lstlisting}
#!/usr/bin/python

msg="Hello, world!"

print "".join(map(lambda x: chr(ord(x)^3), msg))
\end{lstlisting}

Das ist eine ziemlich interessante Verschlüsselung (oder besser eine Verschleierung),
weil sie zwei wichtige Eigenschaften hat:
1) es ist eine einzige Funktion zum Verschlüsseln und entschlüsseln, sie muss nur wiederholt angewendet werden
2) die entstehenden Buchstaben befinden sich im druckbaren Bereich, also die ganze Zeichenkette kann ohne
Escape-Symbole im Code verwendet werden.

Die zweite Eigenschaft nutzt die Tatsache, dass alle druckbaren Zeichen in Reihen organisiert sind: 0x2x-0x7x,
und wenn die beiden niederwertigsten Bits invertiert werden, wird der Buchstabe um eine oder drei Stellen nach
links oder rechts \IT{verschoben}, aber niemals in eine andere Reihe:

\begin{figure}[H]
\centering
\includegraphics[width=0.7\textwidth]{ascii_clean.png}
\caption{7-Bit \ac{ASCII} Tabelle in Emacs}
\end{figure}

\dots mit dem Zeichen 0x7F als einziger Ausnahme.

Im Folgenden werden also beispielsweise die Zeichen A-Z \IT{verschlüsselt}:

\begin{lstlisting}
#!/usr/bin/python

msg="@ABCDEFGHIJKLMNO"

print "".join(map(lambda x: chr(ord(x)^3), msg))
\end{lstlisting}

Ergebnis:
% FIXME \verb  --  relevant comment for German?
\begin{lstlisting}
CBA@GFEDKJIHONML
\end{lstlisting}

Es sieht so aus als würden die Zeichen ``@'' und ``C'' sowie ``B'' und ``A'' vertauscht werden.

Hier ist noch ein interessantes Beispiel, in dem gezeigt wird, wie die Eigenschaften von XOR
ausgenutzt werden können: Exakt den gleichen Effekt, dass druckbare Zeichen auch druckbar bleiben,
kann man dadurch erzielen, dass irgendeine Kombination der niedrigsten vier Bits invertiert wird.
}

\EN{\section{Returning Values}
\label{ret_val_func}

Another simple function is the one that simply returns a constant value:

\lstinputlisting[caption=\EN{\CCpp Code},style=customc]{patterns/011_ret/1.c}

Let's compile it.

\subsection{x86}

Here's what both the GCC and MSVC compilers produce (with optimization) on the x86 platform:

\lstinputlisting[caption=\Optimizing GCC/MSVC (\assemblyOutput),style=customasmx86]{patterns/011_ret/1.s}

\myindex{x86!\Instructions!RET}
There are just two instructions: the first places the value 123 into the \EAX register,
which is used by convention for storing the return
value, and the second one is \RET, which returns execution to the \gls{caller}.

The caller will take the result from the \EAX register.

\subsection{ARM}

There are a few differences on the ARM platform:

\lstinputlisting[caption=\OptimizingKeilVI (\ARMMode) ASM Output,style=customasmARM]{patterns/011_ret/1_Keil_ARM_O3.s}

ARM uses the register \Reg{0} for returning the results of functions, so 123 is copied into \Reg{0}.

\myindex{ARM!\Instructions!MOV}
\myindex{x86!\Instructions!MOV}
It is worth noting that \MOV is a misleading name for the instruction in both the x86 and ARM \ac{ISA}s.

The data is not in fact \IT{moved}, but \IT{copied}.

\subsection{MIPS}

\label{MIPS_leaf_function_ex1}

The GCC assembly output below lists registers by number:

\lstinputlisting[caption=\Optimizing GCC 4.4.5 (\assemblyOutput),style=customasmMIPS]{patterns/011_ret/MIPS.s}

\dots while \IDA does it by their pseudo names:

\lstinputlisting[caption=\Optimizing GCC 4.4.5 (IDA),style=customasmMIPS]{patterns/011_ret/MIPS_IDA.lst}

The \$2 (or \$V0) register is used to store the function's return value.
\myindex{MIPS!\Pseudoinstructions!LI}
\INS{LI} stands for ``Load Immediate'' and is the MIPS equivalent to \MOV.

\myindex{MIPS!\Instructions!J}
The other instruction is the jump instruction (J or JR) which returns the execution flow to the \gls{caller}.

\myindex{MIPS!Branch delay slot}
You might be wondering why the positions of the load instruction (LI) and the jump instruction (J or JR) are swapped. This is due to a \ac{RISC} feature called ``branch delay slot''.

The reason this happens is a quirk in the architecture of some RISC \ac{ISA}s and isn't important for our
purposes---we must simply keep in mind that in MIPS, the instruction following a jump or branch instruction
is executed \IT{before} the jump/branch instruction itself.

As a consequence, branch instructions always swap places with the instruction executed immediately beforehand.


In practice, functions which merely return 1 (\IT{true}) or 0 (\IT{false}) are very frequent.

The smallest ever of the standard UNIX utilities, \IT{/bin/true} and \IT{/bin/false} return 0 and 1 respectively, as an exit code.
(Zero as an exit code usually means success, non-zero means error.)
}
\RU{\subsubsection{std::string}
\myindex{\Cpp!STL!std::string}
\label{std_string}

\myparagraph{Как устроена структура}

Многие строковые библиотеки \InSqBrackets{\CNotes 2.2} обеспечивают структуру содержащую ссылку 
на буфер собственно со строкой, переменная всегда содержащую длину строки 
(что очень удобно для массы функций \InSqBrackets{\CNotes 2.2.1}) и переменную содержащую текущий размер буфера.

Строка в буфере обыкновенно оканчивается нулем: это для того чтобы указатель на буфер можно было
передавать в функции требующие на вход обычную сишную \ac{ASCIIZ}-строку.

Стандарт \Cpp не описывает, как именно нужно реализовывать std::string,
но, как правило, они реализованы как описано выше, с небольшими дополнениями.

Строки в \Cpp это не класс (как, например, QString в Qt), а темплейт (basic\_string), 
это сделано для того чтобы поддерживать 
строки содержащие разного типа символы: как минимум \Tchar и \IT{wchar\_t}.

Так что, std::string это класс с базовым типом \Tchar.

А std::wstring это класс с базовым типом \IT{wchar\_t}.

\mysubparagraph{MSVC}

В реализации MSVC, вместо ссылки на буфер может содержаться сам буфер (если строка короче 16-и символов).

Это означает, что каждая короткая строка будет занимать в памяти по крайней мере $16 + 4 + 4 = 24$ 
байт для 32-битной среды либо $16 + 8 + 8 = 32$ 
байта в 64-битной, а если строка длиннее 16-и символов, то прибавьте еще длину самой строки.

\lstinputlisting[caption=пример для MSVC,style=customc]{\CURPATH/STL/string/MSVC_RU.cpp}

Собственно, из этого исходника почти всё ясно.

Несколько замечаний:

Если строка короче 16-и символов, 
то отдельный буфер для строки в \glslink{heap}{куче} выделяться не будет.

Это удобно потому что на практике, основная часть строк действительно короткие.
Вероятно, разработчики в Microsoft выбрали размер в 16 символов как разумный баланс.

Теперь очень важный момент в конце функции main(): мы не пользуемся методом c\_str(), тем не менее,
если это скомпилировать и запустить, то обе строки появятся в консоли!

Работает это вот почему.

В первом случае строка короче 16-и символов и в начале объекта std::string (его можно рассматривать
просто как структуру) расположен буфер с этой строкой.
\printf трактует указатель как указатель на массив символов оканчивающийся нулем и поэтому всё работает.

Вывод второй строки (длиннее 16-и символов) даже еще опаснее: это вообще типичная программистская ошибка 
(или опечатка), забыть дописать c\_str().
Это работает потому что в это время в начале структуры расположен указатель на буфер.
Это может надолго остаться незамеченным: до тех пока там не появится строка 
короче 16-и символов, тогда процесс упадет.

\mysubparagraph{GCC}

В реализации GCC в структуре есть еще одна переменная --- reference count.

Интересно, что указатель на экземпляр класса std::string в GCC указывает не на начало самой структуры, 
а на указатель на буфера.
В libstdc++-v3\textbackslash{}include\textbackslash{}bits\textbackslash{}basic\_string.h 
мы можем прочитать что это сделано для удобства отладки:

\begin{lstlisting}
   *  The reason you want _M_data pointing to the character %array and
   *  not the _Rep is so that the debugger can see the string
   *  contents. (Probably we should add a non-inline member to get
   *  the _Rep for the debugger to use, so users can check the actual
   *  string length.)
\end{lstlisting}

\href{http://go.yurichev.com/17085}{исходный код basic\_string.h}

В нашем примере мы учитываем это:

\lstinputlisting[caption=пример для GCC,style=customc]{\CURPATH/STL/string/GCC_RU.cpp}

Нужны еще небольшие хаки чтобы сымитировать типичную ошибку, которую мы уже видели выше, из-за
более ужесточенной проверки типов в GCC, тем не менее, printf() работает и здесь без c\_str().

\myparagraph{Чуть более сложный пример}

\lstinputlisting[style=customc]{\CURPATH/STL/string/3.cpp}

\lstinputlisting[caption=MSVC 2012,style=customasmx86]{\CURPATH/STL/string/3_MSVC_RU.asm}

Собственно, компилятор не конструирует строки статически: да в общем-то и как
это возможно, если буфер с ней нужно хранить в \glslink{heap}{куче}?

Вместо этого в сегменте данных хранятся обычные \ac{ASCIIZ}-строки, а позже, во время выполнения, 
при помощи метода \q{assign}, конструируются строки s1 и s2
.
При помощи \TT{operator+}, создается строка s3.

Обратите внимание на то что вызов метода c\_str() отсутствует,
потому что его код достаточно короткий и компилятор вставил его прямо здесь:
если строка короче 16-и байт, то в регистре EAX остается указатель на буфер,
а если длиннее, то из этого же места достается адрес на буфер расположенный в \glslink{heap}{куче}.

Далее следуют вызовы трех деструкторов, причем, они вызываются только если строка длиннее 16-и байт:
тогда нужно освободить буфера в \glslink{heap}{куче}.
В противном случае, так как все три объекта std::string хранятся в стеке,
они освобождаются автоматически после выхода из функции.

Следовательно, работа с короткими строками более быстрая из-за м\'{е}ньшего обращения к \glslink{heap}{куче}.

Код на GCC даже проще (из-за того, что в GCC, как мы уже видели, не реализована возможность хранить короткую
строку прямо в структуре):

% TODO1 comment each function meaning
\lstinputlisting[caption=GCC 4.8.1,style=customasmx86]{\CURPATH/STL/string/3_GCC_RU.s}

Можно заметить, что в деструкторы передается не указатель на объект,
а указатель на место за 12 байт (или 3 слова) перед ним, то есть, на настоящее начало структуры.

\myparagraph{std::string как глобальная переменная}
\label{sec:std_string_as_global_variable}

Опытные программисты на \Cpp знают, что глобальные переменные \ac{STL}-типов вполне можно объявлять.

Да, действительно:

\lstinputlisting[style=customc]{\CURPATH/STL/string/5.cpp}

Но как и где будет вызываться конструктор \TT{std::string}?

На самом деле, эта переменная будет инициализирована даже перед началом \main.

\lstinputlisting[caption=MSVC 2012: здесь конструируется глобальная переменная{,} а также регистрируется её деструктор,style=customasmx86]{\CURPATH/STL/string/5_MSVC_p2.asm}

\lstinputlisting[caption=MSVC 2012: здесь глобальная переменная используется в \main,style=customasmx86]{\CURPATH/STL/string/5_MSVC_p1.asm}

\lstinputlisting[caption=MSVC 2012: эта функция-деструктор вызывается перед выходом,style=customasmx86]{\CURPATH/STL/string/5_MSVC_p3.asm}

\myindex{\CStandardLibrary!atexit()}
В реальности, из \ac{CRT}, еще до вызова main(), вызывается специальная функция,
в которой перечислены все конструкторы подобных переменных.
Более того: при помощи atexit() регистрируется функция, которая будет вызвана в конце работы программы:
в этой функции компилятор собирает вызовы деструкторов всех подобных глобальных переменных.

GCC работает похожим образом:

\lstinputlisting[caption=GCC 4.8.1,style=customasmx86]{\CURPATH/STL/string/5_GCC.s}

Но он не выделяет отдельной функции в которой будут собраны деструкторы: 
каждый деструктор передается в atexit() по одному.

% TODO а если глобальная STL-переменная в другом модуле? надо проверить.

}
\DE{\subsection{Einfachste XOR-Verschlüsselung überhaupt}

Ich habe einmal eine Software gesehen, bei der alle Debugging-Ausgaben mit XOR mit dem Wert 3
verschlüsselt wurden. Mit anderen Worten, die beiden niedrigsten Bits aller Buchstaben wurden invertiert.

``Hello, world'' wurde zu ``Kfool/\#tlqog'':

\begin{lstlisting}
#!/usr/bin/python

msg="Hello, world!"

print "".join(map(lambda x: chr(ord(x)^3), msg))
\end{lstlisting}

Das ist eine ziemlich interessante Verschlüsselung (oder besser eine Verschleierung),
weil sie zwei wichtige Eigenschaften hat:
1) es ist eine einzige Funktion zum Verschlüsseln und entschlüsseln, sie muss nur wiederholt angewendet werden
2) die entstehenden Buchstaben befinden sich im druckbaren Bereich, also die ganze Zeichenkette kann ohne
Escape-Symbole im Code verwendet werden.

Die zweite Eigenschaft nutzt die Tatsache, dass alle druckbaren Zeichen in Reihen organisiert sind: 0x2x-0x7x,
und wenn die beiden niederwertigsten Bits invertiert werden, wird der Buchstabe um eine oder drei Stellen nach
links oder rechts \IT{verschoben}, aber niemals in eine andere Reihe:

\begin{figure}[H]
\centering
\includegraphics[width=0.7\textwidth]{ascii_clean.png}
\caption{7-Bit \ac{ASCII} Tabelle in Emacs}
\end{figure}

\dots mit dem Zeichen 0x7F als einziger Ausnahme.

Im Folgenden werden also beispielsweise die Zeichen A-Z \IT{verschlüsselt}:

\begin{lstlisting}
#!/usr/bin/python

msg="@ABCDEFGHIJKLMNO"

print "".join(map(lambda x: chr(ord(x)^3), msg))
\end{lstlisting}

Ergebnis:
% FIXME \verb  --  relevant comment for German?
\begin{lstlisting}
CBA@GFEDKJIHONML
\end{lstlisting}

Es sieht so aus als würden die Zeichen ``@'' und ``C'' sowie ``B'' und ``A'' vertauscht werden.

Hier ist noch ein interessantes Beispiel, in dem gezeigt wird, wie die Eigenschaften von XOR
ausgenutzt werden können: Exakt den gleichen Effekt, dass druckbare Zeichen auch druckbar bleiben,
kann man dadurch erzielen, dass irgendeine Kombination der niedrigsten vier Bits invertiert wird.
}

\ifdefined\SPANISH
\chapter{Patrones de código}
\fi % SPANISH

\ifdefined\GERMAN
\chapter{Code-Muster}
\fi % GERMAN

\ifdefined\ENGLISH
\chapter{Code Patterns}
\fi % ENGLISH

\ifdefined\ITALIAN
\chapter{Forme di codice}
\fi % ITALIAN

\ifdefined\RUSSIAN
\chapter{Образцы кода}
\fi % RUSSIAN

\ifdefined\BRAZILIAN
\chapter{Padrões de códigos}
\fi % BRAZILIAN

\ifdefined\THAI
\chapter{รูปแบบของโค้ด}
\fi % THAI

\ifdefined\FRENCH
\chapter{Modèle de code}
\fi % FRENCH

\ifdefined\POLISH
\chapter{\PLph{}}
\fi % POLISH

% sections
\EN{\input{patterns/patterns_opt_dbg_EN}}
\ES{\input{patterns/patterns_opt_dbg_ES}}
\ITA{\input{patterns/patterns_opt_dbg_ITA}}
\PTBR{\input{patterns/patterns_opt_dbg_PTBR}}
\RU{\input{patterns/patterns_opt_dbg_RU}}
\THA{\input{patterns/patterns_opt_dbg_THA}}
\DE{\input{patterns/patterns_opt_dbg_DE}}
\FR{\input{patterns/patterns_opt_dbg_FR}}
\PL{\input{patterns/patterns_opt_dbg_PL}}

\RU{\section{Некоторые базовые понятия}}
\EN{\section{Some basics}}
\DE{\section{Einige Grundlagen}}
\FR{\section{Quelques bases}}
\ES{\section{\ESph{}}}
\ITA{\section{Alcune basi teoriche}}
\PTBR{\section{\PTBRph{}}}
\THA{\section{\THAph{}}}
\PL{\section{\PLph{}}}

% sections:
\EN{\input{patterns/intro_CPU_ISA_EN}}
\ES{\input{patterns/intro_CPU_ISA_ES}}
\ITA{\input{patterns/intro_CPU_ISA_ITA}}
\PTBR{\input{patterns/intro_CPU_ISA_PTBR}}
\RU{\input{patterns/intro_CPU_ISA_RU}}
\DE{\input{patterns/intro_CPU_ISA_DE}}
\FR{\input{patterns/intro_CPU_ISA_FR}}
\PL{\input{patterns/intro_CPU_ISA_PL}}

\EN{\input{patterns/numeral_EN}}
\RU{\input{patterns/numeral_RU}}
\ITA{\input{patterns/numeral_ITA}}
\DE{\input{patterns/numeral_DE}}
\FR{\input{patterns/numeral_FR}}
\PL{\input{patterns/numeral_PL}}

% chapters
\input{patterns/00_empty/main}
\input{patterns/011_ret/main}
\input{patterns/01_helloworld/main}
\input{patterns/015_prolog_epilogue/main}
\input{patterns/02_stack/main}
\input{patterns/03_printf/main}
\input{patterns/04_scanf/main}
\input{patterns/05_passing_arguments/main}
\input{patterns/06_return_results/main}
\input{patterns/061_pointers/main}
\input{patterns/065_GOTO/main}
\input{patterns/07_jcc/main}
\input{patterns/08_switch/main}
\input{patterns/09_loops/main}
\input{patterns/10_strings/main}
\input{patterns/11_arith_optimizations/main}
\input{patterns/12_FPU/main}
\input{patterns/13_arrays/main}
\input{patterns/14_bitfields/main}
\EN{\input{patterns/145_LCG/main_EN}}
\RU{\input{patterns/145_LCG/main_RU}}
\input{patterns/15_structs/main}
\input{patterns/17_unions/main}
\input{patterns/18_pointers_to_functions/main}
\input{patterns/185_64bit_in_32_env/main}

\EN{\input{patterns/19_SIMD/main_EN}}
\RU{\input{patterns/19_SIMD/main_RU}}
\DE{\input{patterns/19_SIMD/main_DE}}

\EN{\input{patterns/20_x64/main_EN}}
\RU{\input{patterns/20_x64/main_RU}}

\EN{\input{patterns/205_floating_SIMD/main_EN}}
\RU{\input{patterns/205_floating_SIMD/main_RU}}
\DE{\input{patterns/205_floating_SIMD/main_DE}}

\EN{\input{patterns/ARM/main_EN}}
\RU{\input{patterns/ARM/main_RU}}
\DE{\input{patterns/ARM/main_DE}}

\input{patterns/MIPS/main}


\ifdefined\SPANISH
\chapter{Patrones de código}
\fi % SPANISH

\ifdefined\GERMAN
\chapter{Code-Muster}
\fi % GERMAN

\ifdefined\ENGLISH
\chapter{Code Patterns}
\fi % ENGLISH

\ifdefined\ITALIAN
\chapter{Forme di codice}
\fi % ITALIAN

\ifdefined\RUSSIAN
\chapter{Образцы кода}
\fi % RUSSIAN

\ifdefined\BRAZILIAN
\chapter{Padrões de códigos}
\fi % BRAZILIAN

\ifdefined\THAI
\chapter{รูปแบบของโค้ด}
\fi % THAI

\ifdefined\FRENCH
\chapter{Modèle de code}
\fi % FRENCH

\ifdefined\POLISH
\chapter{\PLph{}}
\fi % POLISH

% sections
\EN{\section{The method}

When the author of this book first started learning C and, later, \Cpp, he used to write small pieces of code, compile them,
and then look at the assembly language output. This made it very easy for him to understand what was going on in the code that he had written.
\footnote{In fact, he still does this when he can't understand what a particular bit of code does.}.
He did this so many times that the relationship between the \CCpp code and what the compiler produced was imprinted deeply in his mind.
It's now easy for him to imagine instantly a rough outline of a C code's appearance and function.
Perhaps this technique could be helpful for others.

%There are a lot of examples for both x86/x64 and ARM.
%Those who already familiar with one of architectures, may freely skim over pages.

By the way, there is a great website where you can do the same, with various compilers, instead of installing them on your box.
You can use it as well: \url{https://gcc.godbolt.org/}.

\section*{\Exercises}

When the author of this book studied assembly language, he also often compiled small C functions and then rewrote
them gradually to assembly, trying to make their code as short as possible.
This probably is not worth doing in real-world scenarios today,
because it's hard to compete with the latest compilers in terms of efficiency. It is, however, a very good way to gain a better understanding of assembly.
Feel free, therefore, to take any assembly code from this book and try to make it shorter.
However, don't forget to test what you have written.

% rewrote to show that debug\release and optimisations levels are orthogonal concepts.
\section*{Optimization levels and debug information}

Source code can be compiled by different compilers with various optimization levels.
A typical compiler has about three such levels, where level zero means that optimization is completely disabled.
Optimization can also be targeted towards code size or code speed.
A non-optimizing compiler is faster and produces more understandable (albeit verbose) code,
whereas an optimizing compiler is slower and tries to produce code that runs faster (but is not necessarily more compact).
In addition to optimization levels, a compiler can include some debug information in the resulting file,
producing code that is easy to debug.
One of the important features of the ´debug' code is that it might contain links
between each line of the source code and its respective machine code address.
Optimizing compilers, on the other hand, tend to produce output where entire lines of source code
can be optimized away and thus not even be present in the resulting machine code.
Reverse engineers can encounter either version, simply because some developers turn on the compiler's optimization flags and others do not.
Because of this, we'll try to work on examples of both debug and release versions of the code featured in this book, wherever possible.

Sometimes some pretty ancient compilers are used in this book, in order to get the shortest (or simplest) possible code snippet.
}
\ES{\input{patterns/patterns_opt_dbg_ES}}
\ITA{\input{patterns/patterns_opt_dbg_ITA}}
\PTBR{\input{patterns/patterns_opt_dbg_PTBR}}
\RU{\input{patterns/patterns_opt_dbg_RU}}
\THA{\input{patterns/patterns_opt_dbg_THA}}
\DE{\input{patterns/patterns_opt_dbg_DE}}
\FR{\input{patterns/patterns_opt_dbg_FR}}
\PL{\input{patterns/patterns_opt_dbg_PL}}

\RU{\section{Некоторые базовые понятия}}
\EN{\section{Some basics}}
\DE{\section{Einige Grundlagen}}
\FR{\section{Quelques bases}}
\ES{\section{\ESph{}}}
\ITA{\section{Alcune basi teoriche}}
\PTBR{\section{\PTBRph{}}}
\THA{\section{\THAph{}}}
\PL{\section{\PLph{}}}

% sections:
\EN{\input{patterns/intro_CPU_ISA_EN}}
\ES{\input{patterns/intro_CPU_ISA_ES}}
\ITA{\input{patterns/intro_CPU_ISA_ITA}}
\PTBR{\input{patterns/intro_CPU_ISA_PTBR}}
\RU{\input{patterns/intro_CPU_ISA_RU}}
\DE{\input{patterns/intro_CPU_ISA_DE}}
\FR{\input{patterns/intro_CPU_ISA_FR}}
\PL{\input{patterns/intro_CPU_ISA_PL}}

\EN{\subsection{Numeral Systems}

Humans have become accustomed to a decimal numeral system, probably because almost everyone has 10 fingers.
Nevertheless, the number \q{10} has no significant meaning in science and mathematics.
The natural numeral system in digital electronics is binary: 0 is for an absence of current in the wire, and 1 for presence.
10 in binary is 2 in decimal, 100 in binary is 4 in decimal, and so on.

% This sentence is a bit unweildy - maybe try 'Our ten-digit system would be described as having a radix...' - Renaissance
If the numeral system has 10 digits, it has a \IT{radix} (or \IT{base}) of 10.
The binary numeral system has a \IT{radix} of 2.

Important things to recall:

1) A \IT{number} is a number, while a \IT{digit} is a term from writing systems, and is usually one character

% The original is 'number' is not changed; I think the intent is value, and changed it - Renaissance
2) The value of a number does not change when converted to another radix; only the writing notation for that value has changed (and therefore the way of representing it in \ac{RAM}).

\subsection{Converting From One Radix To Another}

Positional notation is used almost every numerical system. This means that a digit has weight relative to where it is placed inside of the larger number.
If 2 is placed at the rightmost place, it's 2, but if it's placed one digit before rightmost, it's 20.

What does $1234$ stand for?

$10^3 \cdot 1 + 10^2 \cdot 2 + 10^1 \cdot 3 + 1 \cdot 4 = 1234$ or
$1000 \cdot 1 + 100 \cdot 2 + 10 \cdot 3 + 4 = 1234$

It's the same story for binary numbers, but the base is 2 instead of 10.
What does 0b101011 stand for?

$2^5 \cdot 1 + 2^4 \cdot 0 + 2^3 \cdot 1 + 2^2 \cdot 0 + 2^1 \cdot 1 + 2^0 \cdot 1 = 43$ or
$32 \cdot 1 + 16 \cdot 0 + 8 \cdot 1 + 4 \cdot 0 + 2 \cdot 1 + 1 = 43$

There is such a thing as non-positional notation, such as the Roman numeral system.
\footnote{About numeric system evolution, see \InSqBrackets{\TAOCPvolII{}, 195--213.}}.
% Maybe add a sentence to fill in that X is always 10, and is therefore non-positional, even though putting an I before subtracts and after adds, and is in that sense positional
Perhaps, humankind switched to positional notation because it's easier to do basic operations (addition, multiplication, etc.) on paper by hand.

Binary numbers can be added, subtracted and so on in the very same as taught in schools, but only 2 digits are available.

Binary numbers are bulky when represented in source code and dumps, so that is where the hexadecimal numeral system can be useful.
A hexadecimal radix uses the digits 0..9, and also 6 Latin characters: A..F.
Each hexadecimal digit takes 4 bits or 4 binary digits, so it's very easy to convert from binary number to hexadecimal and back, even manually, in one's mind.

\begin{center}
\begin{longtable}{ | l | l | l | }
\hline
\HeaderColor hexadecimal & \HeaderColor binary & \HeaderColor decimal \\
\hline
0	&0000	&0 \\
1	&0001	&1 \\
2	&0010	&2 \\
3	&0011	&3 \\
4	&0100	&4 \\
5	&0101	&5 \\
6	&0110	&6 \\
7	&0111	&7 \\
8	&1000	&8 \\
9	&1001	&9 \\
A	&1010	&10 \\
B	&1011	&11 \\
C	&1100	&12 \\
D	&1101	&13 \\
E	&1110	&14 \\
F	&1111	&15 \\
\hline
\end{longtable}
\end{center}

How can one tell which radix is being used in a specific instance?

Decimal numbers are usually written as is, i.e., 1234. Some assemblers allow an identifier on decimal radix numbers, in which the number would be written with a "d" suffix: 1234d.

Binary numbers are sometimes prepended with the "0b" prefix: 0b100110111 (\ac{GCC} has a non-standard language extension for this\footnote{\url{https://gcc.gnu.org/onlinedocs/gcc/Binary-constants.html}}).
There is also another way: using a "b" suffix, for example: 100110111b.
This book tries to use the "0b" prefix consistently throughout the book for binary numbers.

Hexadecimal numbers are prepended with "0x" prefix in \CCpp and other \ac{PL}s: 0x1234ABCD.
Alternatively, they are given a "h" suffix: 1234ABCDh. This is common way of representing them in assemblers and debuggers.
In this convention, if the number is started with a Latin (A..F) digit, a 0 is added at the beginning: 0ABCDEFh.
There was also convention that was popular in 8-bit home computers era, using \$ prefix, like \$ABCD.
The book will try to stick to "0x" prefix throughout the book for hexadecimal numbers.

Should one learn to convert numbers mentally? A table of 1-digit hexadecimal numbers can easily be memorized.
As for larger numbers, it's probably not worth tormenting yourself.

Perhaps the most visible hexadecimal numbers are in \ac{URL}s.
This is the way that non-Latin characters are encoded.
For example:
\url{https://en.wiktionary.org/wiki/na\%C3\%AFvet\%C3\%A9} is the \ac{URL} of Wiktionary article about \q{naïveté} word.

\subsubsection{Octal Radix}

Another numeral system heavily used in the past of computer programming is octal. In octal there are 8 digits (0..7), and each is mapped to 3 bits, so it's easy to convert numbers back and forth.
It has been superseded by the hexadecimal system almost everywhere, but, surprisingly, there is a *NIX utility, used often by many people, which takes octal numbers as argument: \TT{chmod}.

\myindex{UNIX!chmod}
As many *NIX users know, \TT{chmod} argument can be a number of 3 digits. The first digit represents the rights of the owner of the file (read, write and/or execute), the second is the rights for the group to which the file belongs, and the third is for everyone else.
Each digit that \TT{chmod} takes can be represented in binary form:

\begin{center}
\begin{longtable}{ | l | l | l | }
\hline
\HeaderColor decimal & \HeaderColor binary & \HeaderColor meaning \\
\hline
7	&111	&\textbf{rwx} \\
6	&110	&\textbf{rw-} \\
5	&101	&\textbf{r-x} \\
4	&100	&\textbf{r-{}-} \\
3	&011	&\textbf{-wx} \\
2	&010	&\textbf{-w-} \\
1	&001	&\textbf{-{}-x} \\
0	&000	&\textbf{-{}-{}-} \\
\hline
\end{longtable}
\end{center}

So each bit is mapped to a flag: read/write/execute.

The importance of \TT{chmod} here is that the whole number in argument can be represented as octal number.
Let's take, for example, 644.
When you run \TT{chmod 644 file}, you set read/write permissions for owner, read permissions for group and again, read permissions for everyone else.
If we convert the octal number 644 to binary, it would be \TT{110100100}, or, in groups of 3 bits, \TT{110 100 100}.

Now we see that each triplet describe permissions for owner/group/others: first is \TT{rw-}, second is \TT{r--} and third is \TT{r--}.

The octal numeral system was also popular on old computers like PDP-8, because word there could be 12, 24 or 36 bits, and these numbers are all divisible by 3, so the octal system was natural in that environment.
Nowadays, all popular computers employ word/address sizes of 16, 32 or 64 bits, and these numbers are all divisible by 4, so the hexadecimal system is more natural there.

The octal numeral system is supported by all standard \CCpp compilers.
This is a source of confusion sometimes, because octal numbers are encoded with a zero prepended, for example, 0377 is 255.
Sometimes, you might make a typo and write "09" instead of 9, and the compiler would report an error.
GCC might report something like this:\\
\TT{error: invalid digit "9" in octal constant}.

Also, the octal system is somewhat popular in Java. When the IDA shows Java strings with non-printable characters,
they are encoded in the octal system instead of hexadecimal.
\myindex{JAD}
The JAD Java decompiler behaves the same way.

\subsubsection{Divisibility}

When you see a decimal number like 120, you can quickly deduce that it's divisible by 10, because the last digit is zero.
In the same way, 123400 is divisible by 100, because the two last digits are zeros.

Likewise, the hexadecimal number 0x1230 is divisible by 0x10 (or 16), 0x123000 is divisible by 0x1000 (or 4096), etc.

The binary number 0b1000101000 is divisible by 0b1000 (8), etc.

This property can often be used to quickly realize if the size of some block in memory is padded to some boundary.
For example, sections in \ac{PE} files are almost always started at addresses ending with 3 hexadecimal zeros: 0x41000, 0x10001000, etc.
The reason behind this is the fact that almost all \ac{PE} sections are padded to a boundary of 0x1000 (4096) bytes.

\subsubsection{Multi-Precision Arithmetic and Radix}

\index{RSA}
Multi-precision arithmetic can use huge numbers, and each one may be stored in several bytes.
For example, RSA keys, both public and private, span up to 4096 bits, and maybe even more.

% I'm not sure how to change this, but the normal format for quoting would be just to mention the author or book, and footnote to the full reference
In \InSqBrackets{\TAOCPvolII, 265} we find the following idea: when you store a multi-precision number in several bytes,
the whole number can be represented as having a radix of $2^8=256$, and each digit goes to the corresponding byte.
Likewise, if you store a multi-precision number in several 32-bit integer values, each digit goes to each 32-bit slot,
and you may think about this number as stored in radix of $2^{32}$.

\subsubsection{How to Pronounce Non-Decimal Numbers}

Numbers in a non-decimal base are usually pronounced by digit by digit: ``one-zero-zero-one-one-...''.
Words like ``ten'' and ``thousand'' are usually not pronounced, to prevent confusion with the decimal base system.

\subsubsection{Floating point numbers}

To distinguish floating point numbers from integers, they are usually written with ``.0'' at the end,
like $0.0$, $123.0$, etc.
}
\RU{\subsection{Представление чисел}

Люди привыкли к десятичной системе счисления вероятно потому что почти у каждого есть по 10 пальцев.
Тем не менее, число 10 не имеет особого значения в науке и математике.
Двоичная система естествена для цифровой электроники: 0 означает отсутствие тока в проводе и 1 --- его присутствие.
10 в двоичной системе это 2 в десятичной; 100 в двоичной это 4 в десятичной, итд.

Если в системе счисления есть 10 цифр, её \IT{основание} или \IT{radix} это 10.
Двоичная система имеет \IT{основание} 2.

Важные вещи, которые полезно вспомнить:
1) \IT{число} это число, в то время как \IT{цифра} это термин из системы письменности, и это обычно один символ;
2) само число не меняется, когда конвертируется из одного основания в другое: меняется способ его записи (или представления
в памяти).

Как сконвертировать число из одного основания в другое?

Позиционная нотация используется почти везде, это означает, что всякая цифра имеет свой вес, в зависимости от её расположения
внутри числа.
Если 2 расположена в самом последнем месте справа, это 2.
Если она расположена в месте перед последним, это 20.

Что означает $1234$?

$10^3 \cdot 1 + 10^2 \cdot 2 + 10^1 \cdot 3 + 1 \cdot 4$ = 1234 или
$1000 \cdot 1 + 100 \cdot 2 + 10 \cdot 3 + 4 = 1234$

Та же история и для двоичных чисел, только основание там 2 вместо 10.
Что означает 0b101011?

$2^5 \cdot 1 + 2^4 \cdot 0 + 2^3 \cdot 1 + 2^2 \cdot 0 + 2^1 \cdot 1 + 2^0 \cdot 1 = 43$ или
$32 \cdot 1 + 16 \cdot 0 + 8 \cdot 1 + 4 \cdot 0 + 2 \cdot 1 + 1 = 43$

Позиционную нотацию можно противопоставить непозиционной нотации, такой как римская система записи чисел
\footnote{Об эволюции способов записи чисел, см.также: \InSqBrackets{\TAOCPvolII{}, 195--213.}}.
Вероятно, человечество перешло на позиционную нотацию, потому что так проще работать с числами (сложение, умножение, итд)
на бумаге, в ручную.

Действительно, двоичные числа можно складывать, вычитать, итд, точно также, как этому обычно обучают в школах,
только доступны лишь 2 цифры.

Двоичные числа громоздки, когда их используют в исходных кодах и дампах, так что в этих случаях применяется шестнадцатеричная
система.
Используются цифры 0..9 и еще 6 латинских букв: A..F.
Каждая шестнадцатеричная цифра занимает 4 бита или 4 двоичных цифры, так что конвертировать из двоичной системы в
шестнадцатеричную и назад, можно легко вручную, или даже в уме.

\begin{center}
\begin{longtable}{ | l | l | l | }
\hline
\HeaderColor шестнадцатеричная & \HeaderColor двоичная & \HeaderColor десятичная \\
\hline
0	&0000	&0 \\
1	&0001	&1 \\
2	&0010	&2 \\
3	&0011	&3 \\
4	&0100	&4 \\
5	&0101	&5 \\
6	&0110	&6 \\
7	&0111	&7 \\
8	&1000	&8 \\
9	&1001	&9 \\
A	&1010	&10 \\
B	&1011	&11 \\
C	&1100	&12 \\
D	&1101	&13 \\
E	&1110	&14 \\
F	&1111	&15 \\
\hline
\end{longtable}
\end{center}

Как понять, какое основание используется в конкретном месте?

Десятичные числа обычно записываются как есть, т.е., 1234. Но некоторые ассемблеры позволяют подчеркивать
этот факт для ясности, и это число может быть дополнено суффиксом "d": 1234d.

К двоичным числам иногда спереди добавляют префикс "0b": 0b100110111
(В \ac{GCC} для этого есть нестандартное расширение языка
\footnote{\url{https://gcc.gnu.org/onlinedocs/gcc/Binary-constants.html}}).
Есть также еще один способ: суффикс "b", например: 100110111b.
В этой книге я буду пытаться придерживаться префикса "0b" для двоичных чисел.

Шестнадцатеричные числа имеют префикс "0x" в \CCpp и некоторых других \ac{PL}: 0x1234ABCD.
Либо они имеют суффикс "h": 1234ABCDh --- обычно так они представляются в ассемблерах и отладчиках.
Если число начинается с цифры A..F, перед ним добавляется 0: 0ABCDEFh.
Во времена 8-битных домашних компьютеров, был также способ записи чисел используя префикс \$, например, \$ABCD.
В книге я попытаюсь придерживаться префикса "0x" для шестнадцатеричных чисел.

Нужно ли учиться конвертировать числа в уме? Таблицу шестнадцатеричных чисел из одной цифры легко запомнить.
А запоминать б\'{о}льшие числа, наверное, не стоит.

Наверное, чаще всего шестнадцатеричные числа можно увидеть в \ac{URL}-ах.
Так кодируются буквы не из числа латинских.
Например:
\url{https://en.wiktionary.org/wiki/na\%C3\%AFvet\%C3\%A9} это \ac{URL} страницы в Wiktionary о слове \q{naïveté}.

\subsubsection{Восьмеричная система}

Еще одна система, которая в прошлом много использовалась в программировании это восьмеричная: есть 8 цифр (0..7) и каждая
описывает 3 бита, так что легко конвертировать числа туда и назад.
Она почти везде была заменена шестнадцатеричной, но удивительно, в *NIX имеется утилита использующаяся многими людьми,
которая принимает на вход восьмеричное число: \TT{chmod}.

\myindex{UNIX!chmod}
Как знают многие пользователи *NIX, аргумент \TT{chmod} это число из трех цифр. Первая цифра это права владельца файла,
вторая это права группы (которой файл принадлежит), третья для всех остальных.
И каждая цифра может быть представлена в двоичном виде:

\begin{center}
\begin{longtable}{ | l | l | l | }
\hline
\HeaderColor десятичная & \HeaderColor двоичная & \HeaderColor значение \\
\hline
7	&111	&\textbf{rwx} \\
6	&110	&\textbf{rw-} \\
5	&101	&\textbf{r-x} \\
4	&100	&\textbf{r-{}-} \\
3	&011	&\textbf{-wx} \\
2	&010	&\textbf{-w-} \\
1	&001	&\textbf{-{}-x} \\
0	&000	&\textbf{-{}-{}-} \\
\hline
\end{longtable}
\end{center}

Так что каждый бит привязан к флагу: read/write/execute (чтение/запись/исполнение).

И вот почему я вспомнил здесь о \TT{chmod}, это потому что всё число может быть представлено как число в восьмеричной системе.
Для примера возьмем 644.
Когда вы запускаете \TT{chmod 644 file}, вы выставляете права read/write для владельца, права read для группы, и снова,
read для всех остальных.
Сконвертируем число 644 из восьмеричной системы в двоичную, это будет \TT{110100100}, или (в группах по 3 бита) \TT{110 100 100}.

Теперь мы видим, что каждая тройка описывает права для владельца/группы/остальных:
первая это \TT{rw-}, вторая это \TT{r--} и третья это \TT{r--}.

Восьмеричная система была также популярная на старых компьютерах вроде PDP-8, потому что слово там могло содержать 12, 24 или
36 бит, и эти числа делятся на 3, так что выбор восьмеричной системы в той среде был логичен.
Сейчас, все популярные компьютеры имеют размер слова/адреса 16, 32 или 64 бита, и эти числа делятся на 4,
так что шестнадцатеричная система здесь удобнее.

Восьмеричная система поддерживается всеми стандартными компиляторами \CCpp{}.
Это иногда источник недоумения, потому что восьмеричные числа кодируются с нулем вперед, например, 0377 это 255.
И иногда, вы можете сделать опечатку, и написать "09" вместо 9, и компилятор выдаст ошибку.
GCC может выдать что-то вроде:\\
\TT{error: invalid digit "9" in octal constant}.

Также, восьмеричная система популярна в Java: когда IDA показывает строку с непечатаемыми символами,
они кодируются в восьмеричной системе вместо шестнадцатеричной.
\myindex{JAD}
Точно также себя ведет декомпилятор с Java JAD.

\subsubsection{Делимость}

Когда вы видите десятичное число вроде 120, вы можете быстро понять что оно делится на 10, потому что последняя цифра это 0.
Точно также, 123400 делится на 100, потому что две последних цифры это нули.

Точно также, шестнадцатеричное число 0x1230 делится на 0x10 (или 16), 0x123000 делится на 0x1000 (или 4096), итд.

Двоичное число 0b1000101000 делится на 0b1000 (8), итд.

Это свойство можно часто использовать, чтобы быстро понять,
что длина какого-либо блока в памяти выровнена по некоторой границе.
Например, секции в \ac{PE}-файлах почти всегда начинаются с адресов заканчивающихся 3 шестнадцатеричными нулями:
0x41000, 0x10001000, итд.
Причина в том, что почти все секции в \ac{PE} выровнены по границе 0x1000 (4096) байт.

\subsubsection{Арифметика произвольной точности и основание}

\index{RSA}
Арифметика произвольной точности (multi-precision arithmetic) может использовать огромные числа,
которые могут храниться в нескольких байтах.
Например, ключи RSA, и открытые и закрытые, могут занимать до 4096 бит и даже больше.

В \InSqBrackets{\TAOCPvolII, 265} можно найти такую идею: когда вы сохраняете число произвольной точности в нескольких байтах,
всё число может быть представлено как имеющую систему счисления по основанию $2^8=256$, и каждая цифра находится
в соответствующем байте.
Точно также, если вы сохраняете число произвольной точности в нескольких 32-битных целочисленных значениях,
каждая цифра отправляется в каждый 32-битный слот, и вы можете считать что это число записано в системе с основанием $2^{32}$.

\subsubsection{Произношение}

Числа в недесятичных системах счислениях обычно произносятся по одной цифре: ``один-ноль-ноль-один-один-...''.
Слова вроде ``десять'', ``тысяча'', итд, обычно не произносятся, потому что тогда можно спутать с десятичной системой.

\subsubsection{Числа с плавающей запятой}

Чтобы отличать числа с плавающей запятой от целочисленных, часто, в конце добавляют ``.0'',
например $0.0$, $123.0$, итд.

}
\ITA{\input{patterns/numeral_ITA}}
\DE{\input{patterns/numeral_DE}}
\FR{\input{patterns/numeral_FR}}
\PL{\input{patterns/numeral_PL}}

% chapters
\ifdefined\SPANISH
\chapter{Patrones de código}
\fi % SPANISH

\ifdefined\GERMAN
\chapter{Code-Muster}
\fi % GERMAN

\ifdefined\ENGLISH
\chapter{Code Patterns}
\fi % ENGLISH

\ifdefined\ITALIAN
\chapter{Forme di codice}
\fi % ITALIAN

\ifdefined\RUSSIAN
\chapter{Образцы кода}
\fi % RUSSIAN

\ifdefined\BRAZILIAN
\chapter{Padrões de códigos}
\fi % BRAZILIAN

\ifdefined\THAI
\chapter{รูปแบบของโค้ด}
\fi % THAI

\ifdefined\FRENCH
\chapter{Modèle de code}
\fi % FRENCH

\ifdefined\POLISH
\chapter{\PLph{}}
\fi % POLISH

% sections
\EN{\input{patterns/patterns_opt_dbg_EN}}
\ES{\input{patterns/patterns_opt_dbg_ES}}
\ITA{\input{patterns/patterns_opt_dbg_ITA}}
\PTBR{\input{patterns/patterns_opt_dbg_PTBR}}
\RU{\input{patterns/patterns_opt_dbg_RU}}
\THA{\input{patterns/patterns_opt_dbg_THA}}
\DE{\input{patterns/patterns_opt_dbg_DE}}
\FR{\input{patterns/patterns_opt_dbg_FR}}
\PL{\input{patterns/patterns_opt_dbg_PL}}

\RU{\section{Некоторые базовые понятия}}
\EN{\section{Some basics}}
\DE{\section{Einige Grundlagen}}
\FR{\section{Quelques bases}}
\ES{\section{\ESph{}}}
\ITA{\section{Alcune basi teoriche}}
\PTBR{\section{\PTBRph{}}}
\THA{\section{\THAph{}}}
\PL{\section{\PLph{}}}

% sections:
\EN{\input{patterns/intro_CPU_ISA_EN}}
\ES{\input{patterns/intro_CPU_ISA_ES}}
\ITA{\input{patterns/intro_CPU_ISA_ITA}}
\PTBR{\input{patterns/intro_CPU_ISA_PTBR}}
\RU{\input{patterns/intro_CPU_ISA_RU}}
\DE{\input{patterns/intro_CPU_ISA_DE}}
\FR{\input{patterns/intro_CPU_ISA_FR}}
\PL{\input{patterns/intro_CPU_ISA_PL}}

\EN{\input{patterns/numeral_EN}}
\RU{\input{patterns/numeral_RU}}
\ITA{\input{patterns/numeral_ITA}}
\DE{\input{patterns/numeral_DE}}
\FR{\input{patterns/numeral_FR}}
\PL{\input{patterns/numeral_PL}}

% chapters
\input{patterns/00_empty/main}
\input{patterns/011_ret/main}
\input{patterns/01_helloworld/main}
\input{patterns/015_prolog_epilogue/main}
\input{patterns/02_stack/main}
\input{patterns/03_printf/main}
\input{patterns/04_scanf/main}
\input{patterns/05_passing_arguments/main}
\input{patterns/06_return_results/main}
\input{patterns/061_pointers/main}
\input{patterns/065_GOTO/main}
\input{patterns/07_jcc/main}
\input{patterns/08_switch/main}
\input{patterns/09_loops/main}
\input{patterns/10_strings/main}
\input{patterns/11_arith_optimizations/main}
\input{patterns/12_FPU/main}
\input{patterns/13_arrays/main}
\input{patterns/14_bitfields/main}
\EN{\input{patterns/145_LCG/main_EN}}
\RU{\input{patterns/145_LCG/main_RU}}
\input{patterns/15_structs/main}
\input{patterns/17_unions/main}
\input{patterns/18_pointers_to_functions/main}
\input{patterns/185_64bit_in_32_env/main}

\EN{\input{patterns/19_SIMD/main_EN}}
\RU{\input{patterns/19_SIMD/main_RU}}
\DE{\input{patterns/19_SIMD/main_DE}}

\EN{\input{patterns/20_x64/main_EN}}
\RU{\input{patterns/20_x64/main_RU}}

\EN{\input{patterns/205_floating_SIMD/main_EN}}
\RU{\input{patterns/205_floating_SIMD/main_RU}}
\DE{\input{patterns/205_floating_SIMD/main_DE}}

\EN{\input{patterns/ARM/main_EN}}
\RU{\input{patterns/ARM/main_RU}}
\DE{\input{patterns/ARM/main_DE}}

\input{patterns/MIPS/main}

\ifdefined\SPANISH
\chapter{Patrones de código}
\fi % SPANISH

\ifdefined\GERMAN
\chapter{Code-Muster}
\fi % GERMAN

\ifdefined\ENGLISH
\chapter{Code Patterns}
\fi % ENGLISH

\ifdefined\ITALIAN
\chapter{Forme di codice}
\fi % ITALIAN

\ifdefined\RUSSIAN
\chapter{Образцы кода}
\fi % RUSSIAN

\ifdefined\BRAZILIAN
\chapter{Padrões de códigos}
\fi % BRAZILIAN

\ifdefined\THAI
\chapter{รูปแบบของโค้ด}
\fi % THAI

\ifdefined\FRENCH
\chapter{Modèle de code}
\fi % FRENCH

\ifdefined\POLISH
\chapter{\PLph{}}
\fi % POLISH

% sections
\EN{\input{patterns/patterns_opt_dbg_EN}}
\ES{\input{patterns/patterns_opt_dbg_ES}}
\ITA{\input{patterns/patterns_opt_dbg_ITA}}
\PTBR{\input{patterns/patterns_opt_dbg_PTBR}}
\RU{\input{patterns/patterns_opt_dbg_RU}}
\THA{\input{patterns/patterns_opt_dbg_THA}}
\DE{\input{patterns/patterns_opt_dbg_DE}}
\FR{\input{patterns/patterns_opt_dbg_FR}}
\PL{\input{patterns/patterns_opt_dbg_PL}}

\RU{\section{Некоторые базовые понятия}}
\EN{\section{Some basics}}
\DE{\section{Einige Grundlagen}}
\FR{\section{Quelques bases}}
\ES{\section{\ESph{}}}
\ITA{\section{Alcune basi teoriche}}
\PTBR{\section{\PTBRph{}}}
\THA{\section{\THAph{}}}
\PL{\section{\PLph{}}}

% sections:
\EN{\input{patterns/intro_CPU_ISA_EN}}
\ES{\input{patterns/intro_CPU_ISA_ES}}
\ITA{\input{patterns/intro_CPU_ISA_ITA}}
\PTBR{\input{patterns/intro_CPU_ISA_PTBR}}
\RU{\input{patterns/intro_CPU_ISA_RU}}
\DE{\input{patterns/intro_CPU_ISA_DE}}
\FR{\input{patterns/intro_CPU_ISA_FR}}
\PL{\input{patterns/intro_CPU_ISA_PL}}

\EN{\input{patterns/numeral_EN}}
\RU{\input{patterns/numeral_RU}}
\ITA{\input{patterns/numeral_ITA}}
\DE{\input{patterns/numeral_DE}}
\FR{\input{patterns/numeral_FR}}
\PL{\input{patterns/numeral_PL}}

% chapters
\input{patterns/00_empty/main}
\input{patterns/011_ret/main}
\input{patterns/01_helloworld/main}
\input{patterns/015_prolog_epilogue/main}
\input{patterns/02_stack/main}
\input{patterns/03_printf/main}
\input{patterns/04_scanf/main}
\input{patterns/05_passing_arguments/main}
\input{patterns/06_return_results/main}
\input{patterns/061_pointers/main}
\input{patterns/065_GOTO/main}
\input{patterns/07_jcc/main}
\input{patterns/08_switch/main}
\input{patterns/09_loops/main}
\input{patterns/10_strings/main}
\input{patterns/11_arith_optimizations/main}
\input{patterns/12_FPU/main}
\input{patterns/13_arrays/main}
\input{patterns/14_bitfields/main}
\EN{\input{patterns/145_LCG/main_EN}}
\RU{\input{patterns/145_LCG/main_RU}}
\input{patterns/15_structs/main}
\input{patterns/17_unions/main}
\input{patterns/18_pointers_to_functions/main}
\input{patterns/185_64bit_in_32_env/main}

\EN{\input{patterns/19_SIMD/main_EN}}
\RU{\input{patterns/19_SIMD/main_RU}}
\DE{\input{patterns/19_SIMD/main_DE}}

\EN{\input{patterns/20_x64/main_EN}}
\RU{\input{patterns/20_x64/main_RU}}

\EN{\input{patterns/205_floating_SIMD/main_EN}}
\RU{\input{patterns/205_floating_SIMD/main_RU}}
\DE{\input{patterns/205_floating_SIMD/main_DE}}

\EN{\input{patterns/ARM/main_EN}}
\RU{\input{patterns/ARM/main_RU}}
\DE{\input{patterns/ARM/main_DE}}

\input{patterns/MIPS/main}

\ifdefined\SPANISH
\chapter{Patrones de código}
\fi % SPANISH

\ifdefined\GERMAN
\chapter{Code-Muster}
\fi % GERMAN

\ifdefined\ENGLISH
\chapter{Code Patterns}
\fi % ENGLISH

\ifdefined\ITALIAN
\chapter{Forme di codice}
\fi % ITALIAN

\ifdefined\RUSSIAN
\chapter{Образцы кода}
\fi % RUSSIAN

\ifdefined\BRAZILIAN
\chapter{Padrões de códigos}
\fi % BRAZILIAN

\ifdefined\THAI
\chapter{รูปแบบของโค้ด}
\fi % THAI

\ifdefined\FRENCH
\chapter{Modèle de code}
\fi % FRENCH

\ifdefined\POLISH
\chapter{\PLph{}}
\fi % POLISH

% sections
\EN{\input{patterns/patterns_opt_dbg_EN}}
\ES{\input{patterns/patterns_opt_dbg_ES}}
\ITA{\input{patterns/patterns_opt_dbg_ITA}}
\PTBR{\input{patterns/patterns_opt_dbg_PTBR}}
\RU{\input{patterns/patterns_opt_dbg_RU}}
\THA{\input{patterns/patterns_opt_dbg_THA}}
\DE{\input{patterns/patterns_opt_dbg_DE}}
\FR{\input{patterns/patterns_opt_dbg_FR}}
\PL{\input{patterns/patterns_opt_dbg_PL}}

\RU{\section{Некоторые базовые понятия}}
\EN{\section{Some basics}}
\DE{\section{Einige Grundlagen}}
\FR{\section{Quelques bases}}
\ES{\section{\ESph{}}}
\ITA{\section{Alcune basi teoriche}}
\PTBR{\section{\PTBRph{}}}
\THA{\section{\THAph{}}}
\PL{\section{\PLph{}}}

% sections:
\EN{\input{patterns/intro_CPU_ISA_EN}}
\ES{\input{patterns/intro_CPU_ISA_ES}}
\ITA{\input{patterns/intro_CPU_ISA_ITA}}
\PTBR{\input{patterns/intro_CPU_ISA_PTBR}}
\RU{\input{patterns/intro_CPU_ISA_RU}}
\DE{\input{patterns/intro_CPU_ISA_DE}}
\FR{\input{patterns/intro_CPU_ISA_FR}}
\PL{\input{patterns/intro_CPU_ISA_PL}}

\EN{\input{patterns/numeral_EN}}
\RU{\input{patterns/numeral_RU}}
\ITA{\input{patterns/numeral_ITA}}
\DE{\input{patterns/numeral_DE}}
\FR{\input{patterns/numeral_FR}}
\PL{\input{patterns/numeral_PL}}

% chapters
\input{patterns/00_empty/main}
\input{patterns/011_ret/main}
\input{patterns/01_helloworld/main}
\input{patterns/015_prolog_epilogue/main}
\input{patterns/02_stack/main}
\input{patterns/03_printf/main}
\input{patterns/04_scanf/main}
\input{patterns/05_passing_arguments/main}
\input{patterns/06_return_results/main}
\input{patterns/061_pointers/main}
\input{patterns/065_GOTO/main}
\input{patterns/07_jcc/main}
\input{patterns/08_switch/main}
\input{patterns/09_loops/main}
\input{patterns/10_strings/main}
\input{patterns/11_arith_optimizations/main}
\input{patterns/12_FPU/main}
\input{patterns/13_arrays/main}
\input{patterns/14_bitfields/main}
\EN{\input{patterns/145_LCG/main_EN}}
\RU{\input{patterns/145_LCG/main_RU}}
\input{patterns/15_structs/main}
\input{patterns/17_unions/main}
\input{patterns/18_pointers_to_functions/main}
\input{patterns/185_64bit_in_32_env/main}

\EN{\input{patterns/19_SIMD/main_EN}}
\RU{\input{patterns/19_SIMD/main_RU}}
\DE{\input{patterns/19_SIMD/main_DE}}

\EN{\input{patterns/20_x64/main_EN}}
\RU{\input{patterns/20_x64/main_RU}}

\EN{\input{patterns/205_floating_SIMD/main_EN}}
\RU{\input{patterns/205_floating_SIMD/main_RU}}
\DE{\input{patterns/205_floating_SIMD/main_DE}}

\EN{\input{patterns/ARM/main_EN}}
\RU{\input{patterns/ARM/main_RU}}
\DE{\input{patterns/ARM/main_DE}}

\input{patterns/MIPS/main}

\ifdefined\SPANISH
\chapter{Patrones de código}
\fi % SPANISH

\ifdefined\GERMAN
\chapter{Code-Muster}
\fi % GERMAN

\ifdefined\ENGLISH
\chapter{Code Patterns}
\fi % ENGLISH

\ifdefined\ITALIAN
\chapter{Forme di codice}
\fi % ITALIAN

\ifdefined\RUSSIAN
\chapter{Образцы кода}
\fi % RUSSIAN

\ifdefined\BRAZILIAN
\chapter{Padrões de códigos}
\fi % BRAZILIAN

\ifdefined\THAI
\chapter{รูปแบบของโค้ด}
\fi % THAI

\ifdefined\FRENCH
\chapter{Modèle de code}
\fi % FRENCH

\ifdefined\POLISH
\chapter{\PLph{}}
\fi % POLISH

% sections
\EN{\input{patterns/patterns_opt_dbg_EN}}
\ES{\input{patterns/patterns_opt_dbg_ES}}
\ITA{\input{patterns/patterns_opt_dbg_ITA}}
\PTBR{\input{patterns/patterns_opt_dbg_PTBR}}
\RU{\input{patterns/patterns_opt_dbg_RU}}
\THA{\input{patterns/patterns_opt_dbg_THA}}
\DE{\input{patterns/patterns_opt_dbg_DE}}
\FR{\input{patterns/patterns_opt_dbg_FR}}
\PL{\input{patterns/patterns_opt_dbg_PL}}

\RU{\section{Некоторые базовые понятия}}
\EN{\section{Some basics}}
\DE{\section{Einige Grundlagen}}
\FR{\section{Quelques bases}}
\ES{\section{\ESph{}}}
\ITA{\section{Alcune basi teoriche}}
\PTBR{\section{\PTBRph{}}}
\THA{\section{\THAph{}}}
\PL{\section{\PLph{}}}

% sections:
\EN{\input{patterns/intro_CPU_ISA_EN}}
\ES{\input{patterns/intro_CPU_ISA_ES}}
\ITA{\input{patterns/intro_CPU_ISA_ITA}}
\PTBR{\input{patterns/intro_CPU_ISA_PTBR}}
\RU{\input{patterns/intro_CPU_ISA_RU}}
\DE{\input{patterns/intro_CPU_ISA_DE}}
\FR{\input{patterns/intro_CPU_ISA_FR}}
\PL{\input{patterns/intro_CPU_ISA_PL}}

\EN{\input{patterns/numeral_EN}}
\RU{\input{patterns/numeral_RU}}
\ITA{\input{patterns/numeral_ITA}}
\DE{\input{patterns/numeral_DE}}
\FR{\input{patterns/numeral_FR}}
\PL{\input{patterns/numeral_PL}}

% chapters
\input{patterns/00_empty/main}
\input{patterns/011_ret/main}
\input{patterns/01_helloworld/main}
\input{patterns/015_prolog_epilogue/main}
\input{patterns/02_stack/main}
\input{patterns/03_printf/main}
\input{patterns/04_scanf/main}
\input{patterns/05_passing_arguments/main}
\input{patterns/06_return_results/main}
\input{patterns/061_pointers/main}
\input{patterns/065_GOTO/main}
\input{patterns/07_jcc/main}
\input{patterns/08_switch/main}
\input{patterns/09_loops/main}
\input{patterns/10_strings/main}
\input{patterns/11_arith_optimizations/main}
\input{patterns/12_FPU/main}
\input{patterns/13_arrays/main}
\input{patterns/14_bitfields/main}
\EN{\input{patterns/145_LCG/main_EN}}
\RU{\input{patterns/145_LCG/main_RU}}
\input{patterns/15_structs/main}
\input{patterns/17_unions/main}
\input{patterns/18_pointers_to_functions/main}
\input{patterns/185_64bit_in_32_env/main}

\EN{\input{patterns/19_SIMD/main_EN}}
\RU{\input{patterns/19_SIMD/main_RU}}
\DE{\input{patterns/19_SIMD/main_DE}}

\EN{\input{patterns/20_x64/main_EN}}
\RU{\input{patterns/20_x64/main_RU}}

\EN{\input{patterns/205_floating_SIMD/main_EN}}
\RU{\input{patterns/205_floating_SIMD/main_RU}}
\DE{\input{patterns/205_floating_SIMD/main_DE}}

\EN{\input{patterns/ARM/main_EN}}
\RU{\input{patterns/ARM/main_RU}}
\DE{\input{patterns/ARM/main_DE}}

\input{patterns/MIPS/main}

\ifdefined\SPANISH
\chapter{Patrones de código}
\fi % SPANISH

\ifdefined\GERMAN
\chapter{Code-Muster}
\fi % GERMAN

\ifdefined\ENGLISH
\chapter{Code Patterns}
\fi % ENGLISH

\ifdefined\ITALIAN
\chapter{Forme di codice}
\fi % ITALIAN

\ifdefined\RUSSIAN
\chapter{Образцы кода}
\fi % RUSSIAN

\ifdefined\BRAZILIAN
\chapter{Padrões de códigos}
\fi % BRAZILIAN

\ifdefined\THAI
\chapter{รูปแบบของโค้ด}
\fi % THAI

\ifdefined\FRENCH
\chapter{Modèle de code}
\fi % FRENCH

\ifdefined\POLISH
\chapter{\PLph{}}
\fi % POLISH

% sections
\EN{\input{patterns/patterns_opt_dbg_EN}}
\ES{\input{patterns/patterns_opt_dbg_ES}}
\ITA{\input{patterns/patterns_opt_dbg_ITA}}
\PTBR{\input{patterns/patterns_opt_dbg_PTBR}}
\RU{\input{patterns/patterns_opt_dbg_RU}}
\THA{\input{patterns/patterns_opt_dbg_THA}}
\DE{\input{patterns/patterns_opt_dbg_DE}}
\FR{\input{patterns/patterns_opt_dbg_FR}}
\PL{\input{patterns/patterns_opt_dbg_PL}}

\RU{\section{Некоторые базовые понятия}}
\EN{\section{Some basics}}
\DE{\section{Einige Grundlagen}}
\FR{\section{Quelques bases}}
\ES{\section{\ESph{}}}
\ITA{\section{Alcune basi teoriche}}
\PTBR{\section{\PTBRph{}}}
\THA{\section{\THAph{}}}
\PL{\section{\PLph{}}}

% sections:
\EN{\input{patterns/intro_CPU_ISA_EN}}
\ES{\input{patterns/intro_CPU_ISA_ES}}
\ITA{\input{patterns/intro_CPU_ISA_ITA}}
\PTBR{\input{patterns/intro_CPU_ISA_PTBR}}
\RU{\input{patterns/intro_CPU_ISA_RU}}
\DE{\input{patterns/intro_CPU_ISA_DE}}
\FR{\input{patterns/intro_CPU_ISA_FR}}
\PL{\input{patterns/intro_CPU_ISA_PL}}

\EN{\input{patterns/numeral_EN}}
\RU{\input{patterns/numeral_RU}}
\ITA{\input{patterns/numeral_ITA}}
\DE{\input{patterns/numeral_DE}}
\FR{\input{patterns/numeral_FR}}
\PL{\input{patterns/numeral_PL}}

% chapters
\input{patterns/00_empty/main}
\input{patterns/011_ret/main}
\input{patterns/01_helloworld/main}
\input{patterns/015_prolog_epilogue/main}
\input{patterns/02_stack/main}
\input{patterns/03_printf/main}
\input{patterns/04_scanf/main}
\input{patterns/05_passing_arguments/main}
\input{patterns/06_return_results/main}
\input{patterns/061_pointers/main}
\input{patterns/065_GOTO/main}
\input{patterns/07_jcc/main}
\input{patterns/08_switch/main}
\input{patterns/09_loops/main}
\input{patterns/10_strings/main}
\input{patterns/11_arith_optimizations/main}
\input{patterns/12_FPU/main}
\input{patterns/13_arrays/main}
\input{patterns/14_bitfields/main}
\EN{\input{patterns/145_LCG/main_EN}}
\RU{\input{patterns/145_LCG/main_RU}}
\input{patterns/15_structs/main}
\input{patterns/17_unions/main}
\input{patterns/18_pointers_to_functions/main}
\input{patterns/185_64bit_in_32_env/main}

\EN{\input{patterns/19_SIMD/main_EN}}
\RU{\input{patterns/19_SIMD/main_RU}}
\DE{\input{patterns/19_SIMD/main_DE}}

\EN{\input{patterns/20_x64/main_EN}}
\RU{\input{patterns/20_x64/main_RU}}

\EN{\input{patterns/205_floating_SIMD/main_EN}}
\RU{\input{patterns/205_floating_SIMD/main_RU}}
\DE{\input{patterns/205_floating_SIMD/main_DE}}

\EN{\input{patterns/ARM/main_EN}}
\RU{\input{patterns/ARM/main_RU}}
\DE{\input{patterns/ARM/main_DE}}

\input{patterns/MIPS/main}

\ifdefined\SPANISH
\chapter{Patrones de código}
\fi % SPANISH

\ifdefined\GERMAN
\chapter{Code-Muster}
\fi % GERMAN

\ifdefined\ENGLISH
\chapter{Code Patterns}
\fi % ENGLISH

\ifdefined\ITALIAN
\chapter{Forme di codice}
\fi % ITALIAN

\ifdefined\RUSSIAN
\chapter{Образцы кода}
\fi % RUSSIAN

\ifdefined\BRAZILIAN
\chapter{Padrões de códigos}
\fi % BRAZILIAN

\ifdefined\THAI
\chapter{รูปแบบของโค้ด}
\fi % THAI

\ifdefined\FRENCH
\chapter{Modèle de code}
\fi % FRENCH

\ifdefined\POLISH
\chapter{\PLph{}}
\fi % POLISH

% sections
\EN{\input{patterns/patterns_opt_dbg_EN}}
\ES{\input{patterns/patterns_opt_dbg_ES}}
\ITA{\input{patterns/patterns_opt_dbg_ITA}}
\PTBR{\input{patterns/patterns_opt_dbg_PTBR}}
\RU{\input{patterns/patterns_opt_dbg_RU}}
\THA{\input{patterns/patterns_opt_dbg_THA}}
\DE{\input{patterns/patterns_opt_dbg_DE}}
\FR{\input{patterns/patterns_opt_dbg_FR}}
\PL{\input{patterns/patterns_opt_dbg_PL}}

\RU{\section{Некоторые базовые понятия}}
\EN{\section{Some basics}}
\DE{\section{Einige Grundlagen}}
\FR{\section{Quelques bases}}
\ES{\section{\ESph{}}}
\ITA{\section{Alcune basi teoriche}}
\PTBR{\section{\PTBRph{}}}
\THA{\section{\THAph{}}}
\PL{\section{\PLph{}}}

% sections:
\EN{\input{patterns/intro_CPU_ISA_EN}}
\ES{\input{patterns/intro_CPU_ISA_ES}}
\ITA{\input{patterns/intro_CPU_ISA_ITA}}
\PTBR{\input{patterns/intro_CPU_ISA_PTBR}}
\RU{\input{patterns/intro_CPU_ISA_RU}}
\DE{\input{patterns/intro_CPU_ISA_DE}}
\FR{\input{patterns/intro_CPU_ISA_FR}}
\PL{\input{patterns/intro_CPU_ISA_PL}}

\EN{\input{patterns/numeral_EN}}
\RU{\input{patterns/numeral_RU}}
\ITA{\input{patterns/numeral_ITA}}
\DE{\input{patterns/numeral_DE}}
\FR{\input{patterns/numeral_FR}}
\PL{\input{patterns/numeral_PL}}

% chapters
\input{patterns/00_empty/main}
\input{patterns/011_ret/main}
\input{patterns/01_helloworld/main}
\input{patterns/015_prolog_epilogue/main}
\input{patterns/02_stack/main}
\input{patterns/03_printf/main}
\input{patterns/04_scanf/main}
\input{patterns/05_passing_arguments/main}
\input{patterns/06_return_results/main}
\input{patterns/061_pointers/main}
\input{patterns/065_GOTO/main}
\input{patterns/07_jcc/main}
\input{patterns/08_switch/main}
\input{patterns/09_loops/main}
\input{patterns/10_strings/main}
\input{patterns/11_arith_optimizations/main}
\input{patterns/12_FPU/main}
\input{patterns/13_arrays/main}
\input{patterns/14_bitfields/main}
\EN{\input{patterns/145_LCG/main_EN}}
\RU{\input{patterns/145_LCG/main_RU}}
\input{patterns/15_structs/main}
\input{patterns/17_unions/main}
\input{patterns/18_pointers_to_functions/main}
\input{patterns/185_64bit_in_32_env/main}

\EN{\input{patterns/19_SIMD/main_EN}}
\RU{\input{patterns/19_SIMD/main_RU}}
\DE{\input{patterns/19_SIMD/main_DE}}

\EN{\input{patterns/20_x64/main_EN}}
\RU{\input{patterns/20_x64/main_RU}}

\EN{\input{patterns/205_floating_SIMD/main_EN}}
\RU{\input{patterns/205_floating_SIMD/main_RU}}
\DE{\input{patterns/205_floating_SIMD/main_DE}}

\EN{\input{patterns/ARM/main_EN}}
\RU{\input{patterns/ARM/main_RU}}
\DE{\input{patterns/ARM/main_DE}}

\input{patterns/MIPS/main}

\ifdefined\SPANISH
\chapter{Patrones de código}
\fi % SPANISH

\ifdefined\GERMAN
\chapter{Code-Muster}
\fi % GERMAN

\ifdefined\ENGLISH
\chapter{Code Patterns}
\fi % ENGLISH

\ifdefined\ITALIAN
\chapter{Forme di codice}
\fi % ITALIAN

\ifdefined\RUSSIAN
\chapter{Образцы кода}
\fi % RUSSIAN

\ifdefined\BRAZILIAN
\chapter{Padrões de códigos}
\fi % BRAZILIAN

\ifdefined\THAI
\chapter{รูปแบบของโค้ด}
\fi % THAI

\ifdefined\FRENCH
\chapter{Modèle de code}
\fi % FRENCH

\ifdefined\POLISH
\chapter{\PLph{}}
\fi % POLISH

% sections
\EN{\input{patterns/patterns_opt_dbg_EN}}
\ES{\input{patterns/patterns_opt_dbg_ES}}
\ITA{\input{patterns/patterns_opt_dbg_ITA}}
\PTBR{\input{patterns/patterns_opt_dbg_PTBR}}
\RU{\input{patterns/patterns_opt_dbg_RU}}
\THA{\input{patterns/patterns_opt_dbg_THA}}
\DE{\input{patterns/patterns_opt_dbg_DE}}
\FR{\input{patterns/patterns_opt_dbg_FR}}
\PL{\input{patterns/patterns_opt_dbg_PL}}

\RU{\section{Некоторые базовые понятия}}
\EN{\section{Some basics}}
\DE{\section{Einige Grundlagen}}
\FR{\section{Quelques bases}}
\ES{\section{\ESph{}}}
\ITA{\section{Alcune basi teoriche}}
\PTBR{\section{\PTBRph{}}}
\THA{\section{\THAph{}}}
\PL{\section{\PLph{}}}

% sections:
\EN{\input{patterns/intro_CPU_ISA_EN}}
\ES{\input{patterns/intro_CPU_ISA_ES}}
\ITA{\input{patterns/intro_CPU_ISA_ITA}}
\PTBR{\input{patterns/intro_CPU_ISA_PTBR}}
\RU{\input{patterns/intro_CPU_ISA_RU}}
\DE{\input{patterns/intro_CPU_ISA_DE}}
\FR{\input{patterns/intro_CPU_ISA_FR}}
\PL{\input{patterns/intro_CPU_ISA_PL}}

\EN{\input{patterns/numeral_EN}}
\RU{\input{patterns/numeral_RU}}
\ITA{\input{patterns/numeral_ITA}}
\DE{\input{patterns/numeral_DE}}
\FR{\input{patterns/numeral_FR}}
\PL{\input{patterns/numeral_PL}}

% chapters
\input{patterns/00_empty/main}
\input{patterns/011_ret/main}
\input{patterns/01_helloworld/main}
\input{patterns/015_prolog_epilogue/main}
\input{patterns/02_stack/main}
\input{patterns/03_printf/main}
\input{patterns/04_scanf/main}
\input{patterns/05_passing_arguments/main}
\input{patterns/06_return_results/main}
\input{patterns/061_pointers/main}
\input{patterns/065_GOTO/main}
\input{patterns/07_jcc/main}
\input{patterns/08_switch/main}
\input{patterns/09_loops/main}
\input{patterns/10_strings/main}
\input{patterns/11_arith_optimizations/main}
\input{patterns/12_FPU/main}
\input{patterns/13_arrays/main}
\input{patterns/14_bitfields/main}
\EN{\input{patterns/145_LCG/main_EN}}
\RU{\input{patterns/145_LCG/main_RU}}
\input{patterns/15_structs/main}
\input{patterns/17_unions/main}
\input{patterns/18_pointers_to_functions/main}
\input{patterns/185_64bit_in_32_env/main}

\EN{\input{patterns/19_SIMD/main_EN}}
\RU{\input{patterns/19_SIMD/main_RU}}
\DE{\input{patterns/19_SIMD/main_DE}}

\EN{\input{patterns/20_x64/main_EN}}
\RU{\input{patterns/20_x64/main_RU}}

\EN{\input{patterns/205_floating_SIMD/main_EN}}
\RU{\input{patterns/205_floating_SIMD/main_RU}}
\DE{\input{patterns/205_floating_SIMD/main_DE}}

\EN{\input{patterns/ARM/main_EN}}
\RU{\input{patterns/ARM/main_RU}}
\DE{\input{patterns/ARM/main_DE}}

\input{patterns/MIPS/main}

\ifdefined\SPANISH
\chapter{Patrones de código}
\fi % SPANISH

\ifdefined\GERMAN
\chapter{Code-Muster}
\fi % GERMAN

\ifdefined\ENGLISH
\chapter{Code Patterns}
\fi % ENGLISH

\ifdefined\ITALIAN
\chapter{Forme di codice}
\fi % ITALIAN

\ifdefined\RUSSIAN
\chapter{Образцы кода}
\fi % RUSSIAN

\ifdefined\BRAZILIAN
\chapter{Padrões de códigos}
\fi % BRAZILIAN

\ifdefined\THAI
\chapter{รูปแบบของโค้ด}
\fi % THAI

\ifdefined\FRENCH
\chapter{Modèle de code}
\fi % FRENCH

\ifdefined\POLISH
\chapter{\PLph{}}
\fi % POLISH

% sections
\EN{\input{patterns/patterns_opt_dbg_EN}}
\ES{\input{patterns/patterns_opt_dbg_ES}}
\ITA{\input{patterns/patterns_opt_dbg_ITA}}
\PTBR{\input{patterns/patterns_opt_dbg_PTBR}}
\RU{\input{patterns/patterns_opt_dbg_RU}}
\THA{\input{patterns/patterns_opt_dbg_THA}}
\DE{\input{patterns/patterns_opt_dbg_DE}}
\FR{\input{patterns/patterns_opt_dbg_FR}}
\PL{\input{patterns/patterns_opt_dbg_PL}}

\RU{\section{Некоторые базовые понятия}}
\EN{\section{Some basics}}
\DE{\section{Einige Grundlagen}}
\FR{\section{Quelques bases}}
\ES{\section{\ESph{}}}
\ITA{\section{Alcune basi teoriche}}
\PTBR{\section{\PTBRph{}}}
\THA{\section{\THAph{}}}
\PL{\section{\PLph{}}}

% sections:
\EN{\input{patterns/intro_CPU_ISA_EN}}
\ES{\input{patterns/intro_CPU_ISA_ES}}
\ITA{\input{patterns/intro_CPU_ISA_ITA}}
\PTBR{\input{patterns/intro_CPU_ISA_PTBR}}
\RU{\input{patterns/intro_CPU_ISA_RU}}
\DE{\input{patterns/intro_CPU_ISA_DE}}
\FR{\input{patterns/intro_CPU_ISA_FR}}
\PL{\input{patterns/intro_CPU_ISA_PL}}

\EN{\input{patterns/numeral_EN}}
\RU{\input{patterns/numeral_RU}}
\ITA{\input{patterns/numeral_ITA}}
\DE{\input{patterns/numeral_DE}}
\FR{\input{patterns/numeral_FR}}
\PL{\input{patterns/numeral_PL}}

% chapters
\input{patterns/00_empty/main}
\input{patterns/011_ret/main}
\input{patterns/01_helloworld/main}
\input{patterns/015_prolog_epilogue/main}
\input{patterns/02_stack/main}
\input{patterns/03_printf/main}
\input{patterns/04_scanf/main}
\input{patterns/05_passing_arguments/main}
\input{patterns/06_return_results/main}
\input{patterns/061_pointers/main}
\input{patterns/065_GOTO/main}
\input{patterns/07_jcc/main}
\input{patterns/08_switch/main}
\input{patterns/09_loops/main}
\input{patterns/10_strings/main}
\input{patterns/11_arith_optimizations/main}
\input{patterns/12_FPU/main}
\input{patterns/13_arrays/main}
\input{patterns/14_bitfields/main}
\EN{\input{patterns/145_LCG/main_EN}}
\RU{\input{patterns/145_LCG/main_RU}}
\input{patterns/15_structs/main}
\input{patterns/17_unions/main}
\input{patterns/18_pointers_to_functions/main}
\input{patterns/185_64bit_in_32_env/main}

\EN{\input{patterns/19_SIMD/main_EN}}
\RU{\input{patterns/19_SIMD/main_RU}}
\DE{\input{patterns/19_SIMD/main_DE}}

\EN{\input{patterns/20_x64/main_EN}}
\RU{\input{patterns/20_x64/main_RU}}

\EN{\input{patterns/205_floating_SIMD/main_EN}}
\RU{\input{patterns/205_floating_SIMD/main_RU}}
\DE{\input{patterns/205_floating_SIMD/main_DE}}

\EN{\input{patterns/ARM/main_EN}}
\RU{\input{patterns/ARM/main_RU}}
\DE{\input{patterns/ARM/main_DE}}

\input{patterns/MIPS/main}

\ifdefined\SPANISH
\chapter{Patrones de código}
\fi % SPANISH

\ifdefined\GERMAN
\chapter{Code-Muster}
\fi % GERMAN

\ifdefined\ENGLISH
\chapter{Code Patterns}
\fi % ENGLISH

\ifdefined\ITALIAN
\chapter{Forme di codice}
\fi % ITALIAN

\ifdefined\RUSSIAN
\chapter{Образцы кода}
\fi % RUSSIAN

\ifdefined\BRAZILIAN
\chapter{Padrões de códigos}
\fi % BRAZILIAN

\ifdefined\THAI
\chapter{รูปแบบของโค้ด}
\fi % THAI

\ifdefined\FRENCH
\chapter{Modèle de code}
\fi % FRENCH

\ifdefined\POLISH
\chapter{\PLph{}}
\fi % POLISH

% sections
\EN{\input{patterns/patterns_opt_dbg_EN}}
\ES{\input{patterns/patterns_opt_dbg_ES}}
\ITA{\input{patterns/patterns_opt_dbg_ITA}}
\PTBR{\input{patterns/patterns_opt_dbg_PTBR}}
\RU{\input{patterns/patterns_opt_dbg_RU}}
\THA{\input{patterns/patterns_opt_dbg_THA}}
\DE{\input{patterns/patterns_opt_dbg_DE}}
\FR{\input{patterns/patterns_opt_dbg_FR}}
\PL{\input{patterns/patterns_opt_dbg_PL}}

\RU{\section{Некоторые базовые понятия}}
\EN{\section{Some basics}}
\DE{\section{Einige Grundlagen}}
\FR{\section{Quelques bases}}
\ES{\section{\ESph{}}}
\ITA{\section{Alcune basi teoriche}}
\PTBR{\section{\PTBRph{}}}
\THA{\section{\THAph{}}}
\PL{\section{\PLph{}}}

% sections:
\EN{\input{patterns/intro_CPU_ISA_EN}}
\ES{\input{patterns/intro_CPU_ISA_ES}}
\ITA{\input{patterns/intro_CPU_ISA_ITA}}
\PTBR{\input{patterns/intro_CPU_ISA_PTBR}}
\RU{\input{patterns/intro_CPU_ISA_RU}}
\DE{\input{patterns/intro_CPU_ISA_DE}}
\FR{\input{patterns/intro_CPU_ISA_FR}}
\PL{\input{patterns/intro_CPU_ISA_PL}}

\EN{\input{patterns/numeral_EN}}
\RU{\input{patterns/numeral_RU}}
\ITA{\input{patterns/numeral_ITA}}
\DE{\input{patterns/numeral_DE}}
\FR{\input{patterns/numeral_FR}}
\PL{\input{patterns/numeral_PL}}

% chapters
\input{patterns/00_empty/main}
\input{patterns/011_ret/main}
\input{patterns/01_helloworld/main}
\input{patterns/015_prolog_epilogue/main}
\input{patterns/02_stack/main}
\input{patterns/03_printf/main}
\input{patterns/04_scanf/main}
\input{patterns/05_passing_arguments/main}
\input{patterns/06_return_results/main}
\input{patterns/061_pointers/main}
\input{patterns/065_GOTO/main}
\input{patterns/07_jcc/main}
\input{patterns/08_switch/main}
\input{patterns/09_loops/main}
\input{patterns/10_strings/main}
\input{patterns/11_arith_optimizations/main}
\input{patterns/12_FPU/main}
\input{patterns/13_arrays/main}
\input{patterns/14_bitfields/main}
\EN{\input{patterns/145_LCG/main_EN}}
\RU{\input{patterns/145_LCG/main_RU}}
\input{patterns/15_structs/main}
\input{patterns/17_unions/main}
\input{patterns/18_pointers_to_functions/main}
\input{patterns/185_64bit_in_32_env/main}

\EN{\input{patterns/19_SIMD/main_EN}}
\RU{\input{patterns/19_SIMD/main_RU}}
\DE{\input{patterns/19_SIMD/main_DE}}

\EN{\input{patterns/20_x64/main_EN}}
\RU{\input{patterns/20_x64/main_RU}}

\EN{\input{patterns/205_floating_SIMD/main_EN}}
\RU{\input{patterns/205_floating_SIMD/main_RU}}
\DE{\input{patterns/205_floating_SIMD/main_DE}}

\EN{\input{patterns/ARM/main_EN}}
\RU{\input{patterns/ARM/main_RU}}
\DE{\input{patterns/ARM/main_DE}}

\input{patterns/MIPS/main}

\ifdefined\SPANISH
\chapter{Patrones de código}
\fi % SPANISH

\ifdefined\GERMAN
\chapter{Code-Muster}
\fi % GERMAN

\ifdefined\ENGLISH
\chapter{Code Patterns}
\fi % ENGLISH

\ifdefined\ITALIAN
\chapter{Forme di codice}
\fi % ITALIAN

\ifdefined\RUSSIAN
\chapter{Образцы кода}
\fi % RUSSIAN

\ifdefined\BRAZILIAN
\chapter{Padrões de códigos}
\fi % BRAZILIAN

\ifdefined\THAI
\chapter{รูปแบบของโค้ด}
\fi % THAI

\ifdefined\FRENCH
\chapter{Modèle de code}
\fi % FRENCH

\ifdefined\POLISH
\chapter{\PLph{}}
\fi % POLISH

% sections
\EN{\input{patterns/patterns_opt_dbg_EN}}
\ES{\input{patterns/patterns_opt_dbg_ES}}
\ITA{\input{patterns/patterns_opt_dbg_ITA}}
\PTBR{\input{patterns/patterns_opt_dbg_PTBR}}
\RU{\input{patterns/patterns_opt_dbg_RU}}
\THA{\input{patterns/patterns_opt_dbg_THA}}
\DE{\input{patterns/patterns_opt_dbg_DE}}
\FR{\input{patterns/patterns_opt_dbg_FR}}
\PL{\input{patterns/patterns_opt_dbg_PL}}

\RU{\section{Некоторые базовые понятия}}
\EN{\section{Some basics}}
\DE{\section{Einige Grundlagen}}
\FR{\section{Quelques bases}}
\ES{\section{\ESph{}}}
\ITA{\section{Alcune basi teoriche}}
\PTBR{\section{\PTBRph{}}}
\THA{\section{\THAph{}}}
\PL{\section{\PLph{}}}

% sections:
\EN{\input{patterns/intro_CPU_ISA_EN}}
\ES{\input{patterns/intro_CPU_ISA_ES}}
\ITA{\input{patterns/intro_CPU_ISA_ITA}}
\PTBR{\input{patterns/intro_CPU_ISA_PTBR}}
\RU{\input{patterns/intro_CPU_ISA_RU}}
\DE{\input{patterns/intro_CPU_ISA_DE}}
\FR{\input{patterns/intro_CPU_ISA_FR}}
\PL{\input{patterns/intro_CPU_ISA_PL}}

\EN{\input{patterns/numeral_EN}}
\RU{\input{patterns/numeral_RU}}
\ITA{\input{patterns/numeral_ITA}}
\DE{\input{patterns/numeral_DE}}
\FR{\input{patterns/numeral_FR}}
\PL{\input{patterns/numeral_PL}}

% chapters
\input{patterns/00_empty/main}
\input{patterns/011_ret/main}
\input{patterns/01_helloworld/main}
\input{patterns/015_prolog_epilogue/main}
\input{patterns/02_stack/main}
\input{patterns/03_printf/main}
\input{patterns/04_scanf/main}
\input{patterns/05_passing_arguments/main}
\input{patterns/06_return_results/main}
\input{patterns/061_pointers/main}
\input{patterns/065_GOTO/main}
\input{patterns/07_jcc/main}
\input{patterns/08_switch/main}
\input{patterns/09_loops/main}
\input{patterns/10_strings/main}
\input{patterns/11_arith_optimizations/main}
\input{patterns/12_FPU/main}
\input{patterns/13_arrays/main}
\input{patterns/14_bitfields/main}
\EN{\input{patterns/145_LCG/main_EN}}
\RU{\input{patterns/145_LCG/main_RU}}
\input{patterns/15_structs/main}
\input{patterns/17_unions/main}
\input{patterns/18_pointers_to_functions/main}
\input{patterns/185_64bit_in_32_env/main}

\EN{\input{patterns/19_SIMD/main_EN}}
\RU{\input{patterns/19_SIMD/main_RU}}
\DE{\input{patterns/19_SIMD/main_DE}}

\EN{\input{patterns/20_x64/main_EN}}
\RU{\input{patterns/20_x64/main_RU}}

\EN{\input{patterns/205_floating_SIMD/main_EN}}
\RU{\input{patterns/205_floating_SIMD/main_RU}}
\DE{\input{patterns/205_floating_SIMD/main_DE}}

\EN{\input{patterns/ARM/main_EN}}
\RU{\input{patterns/ARM/main_RU}}
\DE{\input{patterns/ARM/main_DE}}

\input{patterns/MIPS/main}

\ifdefined\SPANISH
\chapter{Patrones de código}
\fi % SPANISH

\ifdefined\GERMAN
\chapter{Code-Muster}
\fi % GERMAN

\ifdefined\ENGLISH
\chapter{Code Patterns}
\fi % ENGLISH

\ifdefined\ITALIAN
\chapter{Forme di codice}
\fi % ITALIAN

\ifdefined\RUSSIAN
\chapter{Образцы кода}
\fi % RUSSIAN

\ifdefined\BRAZILIAN
\chapter{Padrões de códigos}
\fi % BRAZILIAN

\ifdefined\THAI
\chapter{รูปแบบของโค้ด}
\fi % THAI

\ifdefined\FRENCH
\chapter{Modèle de code}
\fi % FRENCH

\ifdefined\POLISH
\chapter{\PLph{}}
\fi % POLISH

% sections
\EN{\input{patterns/patterns_opt_dbg_EN}}
\ES{\input{patterns/patterns_opt_dbg_ES}}
\ITA{\input{patterns/patterns_opt_dbg_ITA}}
\PTBR{\input{patterns/patterns_opt_dbg_PTBR}}
\RU{\input{patterns/patterns_opt_dbg_RU}}
\THA{\input{patterns/patterns_opt_dbg_THA}}
\DE{\input{patterns/patterns_opt_dbg_DE}}
\FR{\input{patterns/patterns_opt_dbg_FR}}
\PL{\input{patterns/patterns_opt_dbg_PL}}

\RU{\section{Некоторые базовые понятия}}
\EN{\section{Some basics}}
\DE{\section{Einige Grundlagen}}
\FR{\section{Quelques bases}}
\ES{\section{\ESph{}}}
\ITA{\section{Alcune basi teoriche}}
\PTBR{\section{\PTBRph{}}}
\THA{\section{\THAph{}}}
\PL{\section{\PLph{}}}

% sections:
\EN{\input{patterns/intro_CPU_ISA_EN}}
\ES{\input{patterns/intro_CPU_ISA_ES}}
\ITA{\input{patterns/intro_CPU_ISA_ITA}}
\PTBR{\input{patterns/intro_CPU_ISA_PTBR}}
\RU{\input{patterns/intro_CPU_ISA_RU}}
\DE{\input{patterns/intro_CPU_ISA_DE}}
\FR{\input{patterns/intro_CPU_ISA_FR}}
\PL{\input{patterns/intro_CPU_ISA_PL}}

\EN{\input{patterns/numeral_EN}}
\RU{\input{patterns/numeral_RU}}
\ITA{\input{patterns/numeral_ITA}}
\DE{\input{patterns/numeral_DE}}
\FR{\input{patterns/numeral_FR}}
\PL{\input{patterns/numeral_PL}}

% chapters
\input{patterns/00_empty/main}
\input{patterns/011_ret/main}
\input{patterns/01_helloworld/main}
\input{patterns/015_prolog_epilogue/main}
\input{patterns/02_stack/main}
\input{patterns/03_printf/main}
\input{patterns/04_scanf/main}
\input{patterns/05_passing_arguments/main}
\input{patterns/06_return_results/main}
\input{patterns/061_pointers/main}
\input{patterns/065_GOTO/main}
\input{patterns/07_jcc/main}
\input{patterns/08_switch/main}
\input{patterns/09_loops/main}
\input{patterns/10_strings/main}
\input{patterns/11_arith_optimizations/main}
\input{patterns/12_FPU/main}
\input{patterns/13_arrays/main}
\input{patterns/14_bitfields/main}
\EN{\input{patterns/145_LCG/main_EN}}
\RU{\input{patterns/145_LCG/main_RU}}
\input{patterns/15_structs/main}
\input{patterns/17_unions/main}
\input{patterns/18_pointers_to_functions/main}
\input{patterns/185_64bit_in_32_env/main}

\EN{\input{patterns/19_SIMD/main_EN}}
\RU{\input{patterns/19_SIMD/main_RU}}
\DE{\input{patterns/19_SIMD/main_DE}}

\EN{\input{patterns/20_x64/main_EN}}
\RU{\input{patterns/20_x64/main_RU}}

\EN{\input{patterns/205_floating_SIMD/main_EN}}
\RU{\input{patterns/205_floating_SIMD/main_RU}}
\DE{\input{patterns/205_floating_SIMD/main_DE}}

\EN{\input{patterns/ARM/main_EN}}
\RU{\input{patterns/ARM/main_RU}}
\DE{\input{patterns/ARM/main_DE}}

\input{patterns/MIPS/main}

\ifdefined\SPANISH
\chapter{Patrones de código}
\fi % SPANISH

\ifdefined\GERMAN
\chapter{Code-Muster}
\fi % GERMAN

\ifdefined\ENGLISH
\chapter{Code Patterns}
\fi % ENGLISH

\ifdefined\ITALIAN
\chapter{Forme di codice}
\fi % ITALIAN

\ifdefined\RUSSIAN
\chapter{Образцы кода}
\fi % RUSSIAN

\ifdefined\BRAZILIAN
\chapter{Padrões de códigos}
\fi % BRAZILIAN

\ifdefined\THAI
\chapter{รูปแบบของโค้ด}
\fi % THAI

\ifdefined\FRENCH
\chapter{Modèle de code}
\fi % FRENCH

\ifdefined\POLISH
\chapter{\PLph{}}
\fi % POLISH

% sections
\EN{\input{patterns/patterns_opt_dbg_EN}}
\ES{\input{patterns/patterns_opt_dbg_ES}}
\ITA{\input{patterns/patterns_opt_dbg_ITA}}
\PTBR{\input{patterns/patterns_opt_dbg_PTBR}}
\RU{\input{patterns/patterns_opt_dbg_RU}}
\THA{\input{patterns/patterns_opt_dbg_THA}}
\DE{\input{patterns/patterns_opt_dbg_DE}}
\FR{\input{patterns/patterns_opt_dbg_FR}}
\PL{\input{patterns/patterns_opt_dbg_PL}}

\RU{\section{Некоторые базовые понятия}}
\EN{\section{Some basics}}
\DE{\section{Einige Grundlagen}}
\FR{\section{Quelques bases}}
\ES{\section{\ESph{}}}
\ITA{\section{Alcune basi teoriche}}
\PTBR{\section{\PTBRph{}}}
\THA{\section{\THAph{}}}
\PL{\section{\PLph{}}}

% sections:
\EN{\input{patterns/intro_CPU_ISA_EN}}
\ES{\input{patterns/intro_CPU_ISA_ES}}
\ITA{\input{patterns/intro_CPU_ISA_ITA}}
\PTBR{\input{patterns/intro_CPU_ISA_PTBR}}
\RU{\input{patterns/intro_CPU_ISA_RU}}
\DE{\input{patterns/intro_CPU_ISA_DE}}
\FR{\input{patterns/intro_CPU_ISA_FR}}
\PL{\input{patterns/intro_CPU_ISA_PL}}

\EN{\input{patterns/numeral_EN}}
\RU{\input{patterns/numeral_RU}}
\ITA{\input{patterns/numeral_ITA}}
\DE{\input{patterns/numeral_DE}}
\FR{\input{patterns/numeral_FR}}
\PL{\input{patterns/numeral_PL}}

% chapters
\input{patterns/00_empty/main}
\input{patterns/011_ret/main}
\input{patterns/01_helloworld/main}
\input{patterns/015_prolog_epilogue/main}
\input{patterns/02_stack/main}
\input{patterns/03_printf/main}
\input{patterns/04_scanf/main}
\input{patterns/05_passing_arguments/main}
\input{patterns/06_return_results/main}
\input{patterns/061_pointers/main}
\input{patterns/065_GOTO/main}
\input{patterns/07_jcc/main}
\input{patterns/08_switch/main}
\input{patterns/09_loops/main}
\input{patterns/10_strings/main}
\input{patterns/11_arith_optimizations/main}
\input{patterns/12_FPU/main}
\input{patterns/13_arrays/main}
\input{patterns/14_bitfields/main}
\EN{\input{patterns/145_LCG/main_EN}}
\RU{\input{patterns/145_LCG/main_RU}}
\input{patterns/15_structs/main}
\input{patterns/17_unions/main}
\input{patterns/18_pointers_to_functions/main}
\input{patterns/185_64bit_in_32_env/main}

\EN{\input{patterns/19_SIMD/main_EN}}
\RU{\input{patterns/19_SIMD/main_RU}}
\DE{\input{patterns/19_SIMD/main_DE}}

\EN{\input{patterns/20_x64/main_EN}}
\RU{\input{patterns/20_x64/main_RU}}

\EN{\input{patterns/205_floating_SIMD/main_EN}}
\RU{\input{patterns/205_floating_SIMD/main_RU}}
\DE{\input{patterns/205_floating_SIMD/main_DE}}

\EN{\input{patterns/ARM/main_EN}}
\RU{\input{patterns/ARM/main_RU}}
\DE{\input{patterns/ARM/main_DE}}

\input{patterns/MIPS/main}

\ifdefined\SPANISH
\chapter{Patrones de código}
\fi % SPANISH

\ifdefined\GERMAN
\chapter{Code-Muster}
\fi % GERMAN

\ifdefined\ENGLISH
\chapter{Code Patterns}
\fi % ENGLISH

\ifdefined\ITALIAN
\chapter{Forme di codice}
\fi % ITALIAN

\ifdefined\RUSSIAN
\chapter{Образцы кода}
\fi % RUSSIAN

\ifdefined\BRAZILIAN
\chapter{Padrões de códigos}
\fi % BRAZILIAN

\ifdefined\THAI
\chapter{รูปแบบของโค้ด}
\fi % THAI

\ifdefined\FRENCH
\chapter{Modèle de code}
\fi % FRENCH

\ifdefined\POLISH
\chapter{\PLph{}}
\fi % POLISH

% sections
\EN{\input{patterns/patterns_opt_dbg_EN}}
\ES{\input{patterns/patterns_opt_dbg_ES}}
\ITA{\input{patterns/patterns_opt_dbg_ITA}}
\PTBR{\input{patterns/patterns_opt_dbg_PTBR}}
\RU{\input{patterns/patterns_opt_dbg_RU}}
\THA{\input{patterns/patterns_opt_dbg_THA}}
\DE{\input{patterns/patterns_opt_dbg_DE}}
\FR{\input{patterns/patterns_opt_dbg_FR}}
\PL{\input{patterns/patterns_opt_dbg_PL}}

\RU{\section{Некоторые базовые понятия}}
\EN{\section{Some basics}}
\DE{\section{Einige Grundlagen}}
\FR{\section{Quelques bases}}
\ES{\section{\ESph{}}}
\ITA{\section{Alcune basi teoriche}}
\PTBR{\section{\PTBRph{}}}
\THA{\section{\THAph{}}}
\PL{\section{\PLph{}}}

% sections:
\EN{\input{patterns/intro_CPU_ISA_EN}}
\ES{\input{patterns/intro_CPU_ISA_ES}}
\ITA{\input{patterns/intro_CPU_ISA_ITA}}
\PTBR{\input{patterns/intro_CPU_ISA_PTBR}}
\RU{\input{patterns/intro_CPU_ISA_RU}}
\DE{\input{patterns/intro_CPU_ISA_DE}}
\FR{\input{patterns/intro_CPU_ISA_FR}}
\PL{\input{patterns/intro_CPU_ISA_PL}}

\EN{\input{patterns/numeral_EN}}
\RU{\input{patterns/numeral_RU}}
\ITA{\input{patterns/numeral_ITA}}
\DE{\input{patterns/numeral_DE}}
\FR{\input{patterns/numeral_FR}}
\PL{\input{patterns/numeral_PL}}

% chapters
\input{patterns/00_empty/main}
\input{patterns/011_ret/main}
\input{patterns/01_helloworld/main}
\input{patterns/015_prolog_epilogue/main}
\input{patterns/02_stack/main}
\input{patterns/03_printf/main}
\input{patterns/04_scanf/main}
\input{patterns/05_passing_arguments/main}
\input{patterns/06_return_results/main}
\input{patterns/061_pointers/main}
\input{patterns/065_GOTO/main}
\input{patterns/07_jcc/main}
\input{patterns/08_switch/main}
\input{patterns/09_loops/main}
\input{patterns/10_strings/main}
\input{patterns/11_arith_optimizations/main}
\input{patterns/12_FPU/main}
\input{patterns/13_arrays/main}
\input{patterns/14_bitfields/main}
\EN{\input{patterns/145_LCG/main_EN}}
\RU{\input{patterns/145_LCG/main_RU}}
\input{patterns/15_structs/main}
\input{patterns/17_unions/main}
\input{patterns/18_pointers_to_functions/main}
\input{patterns/185_64bit_in_32_env/main}

\EN{\input{patterns/19_SIMD/main_EN}}
\RU{\input{patterns/19_SIMD/main_RU}}
\DE{\input{patterns/19_SIMD/main_DE}}

\EN{\input{patterns/20_x64/main_EN}}
\RU{\input{patterns/20_x64/main_RU}}

\EN{\input{patterns/205_floating_SIMD/main_EN}}
\RU{\input{patterns/205_floating_SIMD/main_RU}}
\DE{\input{patterns/205_floating_SIMD/main_DE}}

\EN{\input{patterns/ARM/main_EN}}
\RU{\input{patterns/ARM/main_RU}}
\DE{\input{patterns/ARM/main_DE}}

\input{patterns/MIPS/main}

\ifdefined\SPANISH
\chapter{Patrones de código}
\fi % SPANISH

\ifdefined\GERMAN
\chapter{Code-Muster}
\fi % GERMAN

\ifdefined\ENGLISH
\chapter{Code Patterns}
\fi % ENGLISH

\ifdefined\ITALIAN
\chapter{Forme di codice}
\fi % ITALIAN

\ifdefined\RUSSIAN
\chapter{Образцы кода}
\fi % RUSSIAN

\ifdefined\BRAZILIAN
\chapter{Padrões de códigos}
\fi % BRAZILIAN

\ifdefined\THAI
\chapter{รูปแบบของโค้ด}
\fi % THAI

\ifdefined\FRENCH
\chapter{Modèle de code}
\fi % FRENCH

\ifdefined\POLISH
\chapter{\PLph{}}
\fi % POLISH

% sections
\EN{\input{patterns/patterns_opt_dbg_EN}}
\ES{\input{patterns/patterns_opt_dbg_ES}}
\ITA{\input{patterns/patterns_opt_dbg_ITA}}
\PTBR{\input{patterns/patterns_opt_dbg_PTBR}}
\RU{\input{patterns/patterns_opt_dbg_RU}}
\THA{\input{patterns/patterns_opt_dbg_THA}}
\DE{\input{patterns/patterns_opt_dbg_DE}}
\FR{\input{patterns/patterns_opt_dbg_FR}}
\PL{\input{patterns/patterns_opt_dbg_PL}}

\RU{\section{Некоторые базовые понятия}}
\EN{\section{Some basics}}
\DE{\section{Einige Grundlagen}}
\FR{\section{Quelques bases}}
\ES{\section{\ESph{}}}
\ITA{\section{Alcune basi teoriche}}
\PTBR{\section{\PTBRph{}}}
\THA{\section{\THAph{}}}
\PL{\section{\PLph{}}}

% sections:
\EN{\input{patterns/intro_CPU_ISA_EN}}
\ES{\input{patterns/intro_CPU_ISA_ES}}
\ITA{\input{patterns/intro_CPU_ISA_ITA}}
\PTBR{\input{patterns/intro_CPU_ISA_PTBR}}
\RU{\input{patterns/intro_CPU_ISA_RU}}
\DE{\input{patterns/intro_CPU_ISA_DE}}
\FR{\input{patterns/intro_CPU_ISA_FR}}
\PL{\input{patterns/intro_CPU_ISA_PL}}

\EN{\input{patterns/numeral_EN}}
\RU{\input{patterns/numeral_RU}}
\ITA{\input{patterns/numeral_ITA}}
\DE{\input{patterns/numeral_DE}}
\FR{\input{patterns/numeral_FR}}
\PL{\input{patterns/numeral_PL}}

% chapters
\input{patterns/00_empty/main}
\input{patterns/011_ret/main}
\input{patterns/01_helloworld/main}
\input{patterns/015_prolog_epilogue/main}
\input{patterns/02_stack/main}
\input{patterns/03_printf/main}
\input{patterns/04_scanf/main}
\input{patterns/05_passing_arguments/main}
\input{patterns/06_return_results/main}
\input{patterns/061_pointers/main}
\input{patterns/065_GOTO/main}
\input{patterns/07_jcc/main}
\input{patterns/08_switch/main}
\input{patterns/09_loops/main}
\input{patterns/10_strings/main}
\input{patterns/11_arith_optimizations/main}
\input{patterns/12_FPU/main}
\input{patterns/13_arrays/main}
\input{patterns/14_bitfields/main}
\EN{\input{patterns/145_LCG/main_EN}}
\RU{\input{patterns/145_LCG/main_RU}}
\input{patterns/15_structs/main}
\input{patterns/17_unions/main}
\input{patterns/18_pointers_to_functions/main}
\input{patterns/185_64bit_in_32_env/main}

\EN{\input{patterns/19_SIMD/main_EN}}
\RU{\input{patterns/19_SIMD/main_RU}}
\DE{\input{patterns/19_SIMD/main_DE}}

\EN{\input{patterns/20_x64/main_EN}}
\RU{\input{patterns/20_x64/main_RU}}

\EN{\input{patterns/205_floating_SIMD/main_EN}}
\RU{\input{patterns/205_floating_SIMD/main_RU}}
\DE{\input{patterns/205_floating_SIMD/main_DE}}

\EN{\input{patterns/ARM/main_EN}}
\RU{\input{patterns/ARM/main_RU}}
\DE{\input{patterns/ARM/main_DE}}

\input{patterns/MIPS/main}

\ifdefined\SPANISH
\chapter{Patrones de código}
\fi % SPANISH

\ifdefined\GERMAN
\chapter{Code-Muster}
\fi % GERMAN

\ifdefined\ENGLISH
\chapter{Code Patterns}
\fi % ENGLISH

\ifdefined\ITALIAN
\chapter{Forme di codice}
\fi % ITALIAN

\ifdefined\RUSSIAN
\chapter{Образцы кода}
\fi % RUSSIAN

\ifdefined\BRAZILIAN
\chapter{Padrões de códigos}
\fi % BRAZILIAN

\ifdefined\THAI
\chapter{รูปแบบของโค้ด}
\fi % THAI

\ifdefined\FRENCH
\chapter{Modèle de code}
\fi % FRENCH

\ifdefined\POLISH
\chapter{\PLph{}}
\fi % POLISH

% sections
\EN{\input{patterns/patterns_opt_dbg_EN}}
\ES{\input{patterns/patterns_opt_dbg_ES}}
\ITA{\input{patterns/patterns_opt_dbg_ITA}}
\PTBR{\input{patterns/patterns_opt_dbg_PTBR}}
\RU{\input{patterns/patterns_opt_dbg_RU}}
\THA{\input{patterns/patterns_opt_dbg_THA}}
\DE{\input{patterns/patterns_opt_dbg_DE}}
\FR{\input{patterns/patterns_opt_dbg_FR}}
\PL{\input{patterns/patterns_opt_dbg_PL}}

\RU{\section{Некоторые базовые понятия}}
\EN{\section{Some basics}}
\DE{\section{Einige Grundlagen}}
\FR{\section{Quelques bases}}
\ES{\section{\ESph{}}}
\ITA{\section{Alcune basi teoriche}}
\PTBR{\section{\PTBRph{}}}
\THA{\section{\THAph{}}}
\PL{\section{\PLph{}}}

% sections:
\EN{\input{patterns/intro_CPU_ISA_EN}}
\ES{\input{patterns/intro_CPU_ISA_ES}}
\ITA{\input{patterns/intro_CPU_ISA_ITA}}
\PTBR{\input{patterns/intro_CPU_ISA_PTBR}}
\RU{\input{patterns/intro_CPU_ISA_RU}}
\DE{\input{patterns/intro_CPU_ISA_DE}}
\FR{\input{patterns/intro_CPU_ISA_FR}}
\PL{\input{patterns/intro_CPU_ISA_PL}}

\EN{\input{patterns/numeral_EN}}
\RU{\input{patterns/numeral_RU}}
\ITA{\input{patterns/numeral_ITA}}
\DE{\input{patterns/numeral_DE}}
\FR{\input{patterns/numeral_FR}}
\PL{\input{patterns/numeral_PL}}

% chapters
\input{patterns/00_empty/main}
\input{patterns/011_ret/main}
\input{patterns/01_helloworld/main}
\input{patterns/015_prolog_epilogue/main}
\input{patterns/02_stack/main}
\input{patterns/03_printf/main}
\input{patterns/04_scanf/main}
\input{patterns/05_passing_arguments/main}
\input{patterns/06_return_results/main}
\input{patterns/061_pointers/main}
\input{patterns/065_GOTO/main}
\input{patterns/07_jcc/main}
\input{patterns/08_switch/main}
\input{patterns/09_loops/main}
\input{patterns/10_strings/main}
\input{patterns/11_arith_optimizations/main}
\input{patterns/12_FPU/main}
\input{patterns/13_arrays/main}
\input{patterns/14_bitfields/main}
\EN{\input{patterns/145_LCG/main_EN}}
\RU{\input{patterns/145_LCG/main_RU}}
\input{patterns/15_structs/main}
\input{patterns/17_unions/main}
\input{patterns/18_pointers_to_functions/main}
\input{patterns/185_64bit_in_32_env/main}

\EN{\input{patterns/19_SIMD/main_EN}}
\RU{\input{patterns/19_SIMD/main_RU}}
\DE{\input{patterns/19_SIMD/main_DE}}

\EN{\input{patterns/20_x64/main_EN}}
\RU{\input{patterns/20_x64/main_RU}}

\EN{\input{patterns/205_floating_SIMD/main_EN}}
\RU{\input{patterns/205_floating_SIMD/main_RU}}
\DE{\input{patterns/205_floating_SIMD/main_DE}}

\EN{\input{patterns/ARM/main_EN}}
\RU{\input{patterns/ARM/main_RU}}
\DE{\input{patterns/ARM/main_DE}}

\input{patterns/MIPS/main}

\ifdefined\SPANISH
\chapter{Patrones de código}
\fi % SPANISH

\ifdefined\GERMAN
\chapter{Code-Muster}
\fi % GERMAN

\ifdefined\ENGLISH
\chapter{Code Patterns}
\fi % ENGLISH

\ifdefined\ITALIAN
\chapter{Forme di codice}
\fi % ITALIAN

\ifdefined\RUSSIAN
\chapter{Образцы кода}
\fi % RUSSIAN

\ifdefined\BRAZILIAN
\chapter{Padrões de códigos}
\fi % BRAZILIAN

\ifdefined\THAI
\chapter{รูปแบบของโค้ด}
\fi % THAI

\ifdefined\FRENCH
\chapter{Modèle de code}
\fi % FRENCH

\ifdefined\POLISH
\chapter{\PLph{}}
\fi % POLISH

% sections
\EN{\input{patterns/patterns_opt_dbg_EN}}
\ES{\input{patterns/patterns_opt_dbg_ES}}
\ITA{\input{patterns/patterns_opt_dbg_ITA}}
\PTBR{\input{patterns/patterns_opt_dbg_PTBR}}
\RU{\input{patterns/patterns_opt_dbg_RU}}
\THA{\input{patterns/patterns_opt_dbg_THA}}
\DE{\input{patterns/patterns_opt_dbg_DE}}
\FR{\input{patterns/patterns_opt_dbg_FR}}
\PL{\input{patterns/patterns_opt_dbg_PL}}

\RU{\section{Некоторые базовые понятия}}
\EN{\section{Some basics}}
\DE{\section{Einige Grundlagen}}
\FR{\section{Quelques bases}}
\ES{\section{\ESph{}}}
\ITA{\section{Alcune basi teoriche}}
\PTBR{\section{\PTBRph{}}}
\THA{\section{\THAph{}}}
\PL{\section{\PLph{}}}

% sections:
\EN{\input{patterns/intro_CPU_ISA_EN}}
\ES{\input{patterns/intro_CPU_ISA_ES}}
\ITA{\input{patterns/intro_CPU_ISA_ITA}}
\PTBR{\input{patterns/intro_CPU_ISA_PTBR}}
\RU{\input{patterns/intro_CPU_ISA_RU}}
\DE{\input{patterns/intro_CPU_ISA_DE}}
\FR{\input{patterns/intro_CPU_ISA_FR}}
\PL{\input{patterns/intro_CPU_ISA_PL}}

\EN{\input{patterns/numeral_EN}}
\RU{\input{patterns/numeral_RU}}
\ITA{\input{patterns/numeral_ITA}}
\DE{\input{patterns/numeral_DE}}
\FR{\input{patterns/numeral_FR}}
\PL{\input{patterns/numeral_PL}}

% chapters
\input{patterns/00_empty/main}
\input{patterns/011_ret/main}
\input{patterns/01_helloworld/main}
\input{patterns/015_prolog_epilogue/main}
\input{patterns/02_stack/main}
\input{patterns/03_printf/main}
\input{patterns/04_scanf/main}
\input{patterns/05_passing_arguments/main}
\input{patterns/06_return_results/main}
\input{patterns/061_pointers/main}
\input{patterns/065_GOTO/main}
\input{patterns/07_jcc/main}
\input{patterns/08_switch/main}
\input{patterns/09_loops/main}
\input{patterns/10_strings/main}
\input{patterns/11_arith_optimizations/main}
\input{patterns/12_FPU/main}
\input{patterns/13_arrays/main}
\input{patterns/14_bitfields/main}
\EN{\input{patterns/145_LCG/main_EN}}
\RU{\input{patterns/145_LCG/main_RU}}
\input{patterns/15_structs/main}
\input{patterns/17_unions/main}
\input{patterns/18_pointers_to_functions/main}
\input{patterns/185_64bit_in_32_env/main}

\EN{\input{patterns/19_SIMD/main_EN}}
\RU{\input{patterns/19_SIMD/main_RU}}
\DE{\input{patterns/19_SIMD/main_DE}}

\EN{\input{patterns/20_x64/main_EN}}
\RU{\input{patterns/20_x64/main_RU}}

\EN{\input{patterns/205_floating_SIMD/main_EN}}
\RU{\input{patterns/205_floating_SIMD/main_RU}}
\DE{\input{patterns/205_floating_SIMD/main_DE}}

\EN{\input{patterns/ARM/main_EN}}
\RU{\input{patterns/ARM/main_RU}}
\DE{\input{patterns/ARM/main_DE}}

\input{patterns/MIPS/main}

\EN{\input{patterns/12_FPU/main_EN}}
\RU{\input{patterns/12_FPU/main_RU}}
\DE{\input{patterns/12_FPU/main_DE}}
\FR{\input{patterns/12_FPU/main_FR}}


\ifdefined\SPANISH
\chapter{Patrones de código}
\fi % SPANISH

\ifdefined\GERMAN
\chapter{Code-Muster}
\fi % GERMAN

\ifdefined\ENGLISH
\chapter{Code Patterns}
\fi % ENGLISH

\ifdefined\ITALIAN
\chapter{Forme di codice}
\fi % ITALIAN

\ifdefined\RUSSIAN
\chapter{Образцы кода}
\fi % RUSSIAN

\ifdefined\BRAZILIAN
\chapter{Padrões de códigos}
\fi % BRAZILIAN

\ifdefined\THAI
\chapter{รูปแบบของโค้ด}
\fi % THAI

\ifdefined\FRENCH
\chapter{Modèle de code}
\fi % FRENCH

\ifdefined\POLISH
\chapter{\PLph{}}
\fi % POLISH

% sections
\EN{\input{patterns/patterns_opt_dbg_EN}}
\ES{\input{patterns/patterns_opt_dbg_ES}}
\ITA{\input{patterns/patterns_opt_dbg_ITA}}
\PTBR{\input{patterns/patterns_opt_dbg_PTBR}}
\RU{\input{patterns/patterns_opt_dbg_RU}}
\THA{\input{patterns/patterns_opt_dbg_THA}}
\DE{\input{patterns/patterns_opt_dbg_DE}}
\FR{\input{patterns/patterns_opt_dbg_FR}}
\PL{\input{patterns/patterns_opt_dbg_PL}}

\RU{\section{Некоторые базовые понятия}}
\EN{\section{Some basics}}
\DE{\section{Einige Grundlagen}}
\FR{\section{Quelques bases}}
\ES{\section{\ESph{}}}
\ITA{\section{Alcune basi teoriche}}
\PTBR{\section{\PTBRph{}}}
\THA{\section{\THAph{}}}
\PL{\section{\PLph{}}}

% sections:
\EN{\input{patterns/intro_CPU_ISA_EN}}
\ES{\input{patterns/intro_CPU_ISA_ES}}
\ITA{\input{patterns/intro_CPU_ISA_ITA}}
\PTBR{\input{patterns/intro_CPU_ISA_PTBR}}
\RU{\input{patterns/intro_CPU_ISA_RU}}
\DE{\input{patterns/intro_CPU_ISA_DE}}
\FR{\input{patterns/intro_CPU_ISA_FR}}
\PL{\input{patterns/intro_CPU_ISA_PL}}

\EN{\input{patterns/numeral_EN}}
\RU{\input{patterns/numeral_RU}}
\ITA{\input{patterns/numeral_ITA}}
\DE{\input{patterns/numeral_DE}}
\FR{\input{patterns/numeral_FR}}
\PL{\input{patterns/numeral_PL}}

% chapters
\input{patterns/00_empty/main}
\input{patterns/011_ret/main}
\input{patterns/01_helloworld/main}
\input{patterns/015_prolog_epilogue/main}
\input{patterns/02_stack/main}
\input{patterns/03_printf/main}
\input{patterns/04_scanf/main}
\input{patterns/05_passing_arguments/main}
\input{patterns/06_return_results/main}
\input{patterns/061_pointers/main}
\input{patterns/065_GOTO/main}
\input{patterns/07_jcc/main}
\input{patterns/08_switch/main}
\input{patterns/09_loops/main}
\input{patterns/10_strings/main}
\input{patterns/11_arith_optimizations/main}
\input{patterns/12_FPU/main}
\input{patterns/13_arrays/main}
\input{patterns/14_bitfields/main}
\EN{\input{patterns/145_LCG/main_EN}}
\RU{\input{patterns/145_LCG/main_RU}}
\input{patterns/15_structs/main}
\input{patterns/17_unions/main}
\input{patterns/18_pointers_to_functions/main}
\input{patterns/185_64bit_in_32_env/main}

\EN{\input{patterns/19_SIMD/main_EN}}
\RU{\input{patterns/19_SIMD/main_RU}}
\DE{\input{patterns/19_SIMD/main_DE}}

\EN{\input{patterns/20_x64/main_EN}}
\RU{\input{patterns/20_x64/main_RU}}

\EN{\input{patterns/205_floating_SIMD/main_EN}}
\RU{\input{patterns/205_floating_SIMD/main_RU}}
\DE{\input{patterns/205_floating_SIMD/main_DE}}

\EN{\input{patterns/ARM/main_EN}}
\RU{\input{patterns/ARM/main_RU}}
\DE{\input{patterns/ARM/main_DE}}

\input{patterns/MIPS/main}

\ifdefined\SPANISH
\chapter{Patrones de código}
\fi % SPANISH

\ifdefined\GERMAN
\chapter{Code-Muster}
\fi % GERMAN

\ifdefined\ENGLISH
\chapter{Code Patterns}
\fi % ENGLISH

\ifdefined\ITALIAN
\chapter{Forme di codice}
\fi % ITALIAN

\ifdefined\RUSSIAN
\chapter{Образцы кода}
\fi % RUSSIAN

\ifdefined\BRAZILIAN
\chapter{Padrões de códigos}
\fi % BRAZILIAN

\ifdefined\THAI
\chapter{รูปแบบของโค้ด}
\fi % THAI

\ifdefined\FRENCH
\chapter{Modèle de code}
\fi % FRENCH

\ifdefined\POLISH
\chapter{\PLph{}}
\fi % POLISH

% sections
\EN{\input{patterns/patterns_opt_dbg_EN}}
\ES{\input{patterns/patterns_opt_dbg_ES}}
\ITA{\input{patterns/patterns_opt_dbg_ITA}}
\PTBR{\input{patterns/patterns_opt_dbg_PTBR}}
\RU{\input{patterns/patterns_opt_dbg_RU}}
\THA{\input{patterns/patterns_opt_dbg_THA}}
\DE{\input{patterns/patterns_opt_dbg_DE}}
\FR{\input{patterns/patterns_opt_dbg_FR}}
\PL{\input{patterns/patterns_opt_dbg_PL}}

\RU{\section{Некоторые базовые понятия}}
\EN{\section{Some basics}}
\DE{\section{Einige Grundlagen}}
\FR{\section{Quelques bases}}
\ES{\section{\ESph{}}}
\ITA{\section{Alcune basi teoriche}}
\PTBR{\section{\PTBRph{}}}
\THA{\section{\THAph{}}}
\PL{\section{\PLph{}}}

% sections:
\EN{\input{patterns/intro_CPU_ISA_EN}}
\ES{\input{patterns/intro_CPU_ISA_ES}}
\ITA{\input{patterns/intro_CPU_ISA_ITA}}
\PTBR{\input{patterns/intro_CPU_ISA_PTBR}}
\RU{\input{patterns/intro_CPU_ISA_RU}}
\DE{\input{patterns/intro_CPU_ISA_DE}}
\FR{\input{patterns/intro_CPU_ISA_FR}}
\PL{\input{patterns/intro_CPU_ISA_PL}}

\EN{\input{patterns/numeral_EN}}
\RU{\input{patterns/numeral_RU}}
\ITA{\input{patterns/numeral_ITA}}
\DE{\input{patterns/numeral_DE}}
\FR{\input{patterns/numeral_FR}}
\PL{\input{patterns/numeral_PL}}

% chapters
\input{patterns/00_empty/main}
\input{patterns/011_ret/main}
\input{patterns/01_helloworld/main}
\input{patterns/015_prolog_epilogue/main}
\input{patterns/02_stack/main}
\input{patterns/03_printf/main}
\input{patterns/04_scanf/main}
\input{patterns/05_passing_arguments/main}
\input{patterns/06_return_results/main}
\input{patterns/061_pointers/main}
\input{patterns/065_GOTO/main}
\input{patterns/07_jcc/main}
\input{patterns/08_switch/main}
\input{patterns/09_loops/main}
\input{patterns/10_strings/main}
\input{patterns/11_arith_optimizations/main}
\input{patterns/12_FPU/main}
\input{patterns/13_arrays/main}
\input{patterns/14_bitfields/main}
\EN{\input{patterns/145_LCG/main_EN}}
\RU{\input{patterns/145_LCG/main_RU}}
\input{patterns/15_structs/main}
\input{patterns/17_unions/main}
\input{patterns/18_pointers_to_functions/main}
\input{patterns/185_64bit_in_32_env/main}

\EN{\input{patterns/19_SIMD/main_EN}}
\RU{\input{patterns/19_SIMD/main_RU}}
\DE{\input{patterns/19_SIMD/main_DE}}

\EN{\input{patterns/20_x64/main_EN}}
\RU{\input{patterns/20_x64/main_RU}}

\EN{\input{patterns/205_floating_SIMD/main_EN}}
\RU{\input{patterns/205_floating_SIMD/main_RU}}
\DE{\input{patterns/205_floating_SIMD/main_DE}}

\EN{\input{patterns/ARM/main_EN}}
\RU{\input{patterns/ARM/main_RU}}
\DE{\input{patterns/ARM/main_DE}}

\input{patterns/MIPS/main}

\EN{\section{Returning Values}
\label{ret_val_func}

Another simple function is the one that simply returns a constant value:

\lstinputlisting[caption=\EN{\CCpp Code},style=customc]{patterns/011_ret/1.c}

Let's compile it.

\subsection{x86}

Here's what both the GCC and MSVC compilers produce (with optimization) on the x86 platform:

\lstinputlisting[caption=\Optimizing GCC/MSVC (\assemblyOutput),style=customasmx86]{patterns/011_ret/1.s}

\myindex{x86!\Instructions!RET}
There are just two instructions: the first places the value 123 into the \EAX register,
which is used by convention for storing the return
value, and the second one is \RET, which returns execution to the \gls{caller}.

The caller will take the result from the \EAX register.

\subsection{ARM}

There are a few differences on the ARM platform:

\lstinputlisting[caption=\OptimizingKeilVI (\ARMMode) ASM Output,style=customasmARM]{patterns/011_ret/1_Keil_ARM_O3.s}

ARM uses the register \Reg{0} for returning the results of functions, so 123 is copied into \Reg{0}.

\myindex{ARM!\Instructions!MOV}
\myindex{x86!\Instructions!MOV}
It is worth noting that \MOV is a misleading name for the instruction in both the x86 and ARM \ac{ISA}s.

The data is not in fact \IT{moved}, but \IT{copied}.

\subsection{MIPS}

\label{MIPS_leaf_function_ex1}

The GCC assembly output below lists registers by number:

\lstinputlisting[caption=\Optimizing GCC 4.4.5 (\assemblyOutput),style=customasmMIPS]{patterns/011_ret/MIPS.s}

\dots while \IDA does it by their pseudo names:

\lstinputlisting[caption=\Optimizing GCC 4.4.5 (IDA),style=customasmMIPS]{patterns/011_ret/MIPS_IDA.lst}

The \$2 (or \$V0) register is used to store the function's return value.
\myindex{MIPS!\Pseudoinstructions!LI}
\INS{LI} stands for ``Load Immediate'' and is the MIPS equivalent to \MOV.

\myindex{MIPS!\Instructions!J}
The other instruction is the jump instruction (J or JR) which returns the execution flow to the \gls{caller}.

\myindex{MIPS!Branch delay slot}
You might be wondering why the positions of the load instruction (LI) and the jump instruction (J or JR) are swapped. This is due to a \ac{RISC} feature called ``branch delay slot''.

The reason this happens is a quirk in the architecture of some RISC \ac{ISA}s and isn't important for our
purposes---we must simply keep in mind that in MIPS, the instruction following a jump or branch instruction
is executed \IT{before} the jump/branch instruction itself.

As a consequence, branch instructions always swap places with the instruction executed immediately beforehand.


In practice, functions which merely return 1 (\IT{true}) or 0 (\IT{false}) are very frequent.

The smallest ever of the standard UNIX utilities, \IT{/bin/true} and \IT{/bin/false} return 0 and 1 respectively, as an exit code.
(Zero as an exit code usually means success, non-zero means error.)
}
\RU{\subsubsection{std::string}
\myindex{\Cpp!STL!std::string}
\label{std_string}

\myparagraph{Как устроена структура}

Многие строковые библиотеки \InSqBrackets{\CNotes 2.2} обеспечивают структуру содержащую ссылку 
на буфер собственно со строкой, переменная всегда содержащую длину строки 
(что очень удобно для массы функций \InSqBrackets{\CNotes 2.2.1}) и переменную содержащую текущий размер буфера.

Строка в буфере обыкновенно оканчивается нулем: это для того чтобы указатель на буфер можно было
передавать в функции требующие на вход обычную сишную \ac{ASCIIZ}-строку.

Стандарт \Cpp не описывает, как именно нужно реализовывать std::string,
но, как правило, они реализованы как описано выше, с небольшими дополнениями.

Строки в \Cpp это не класс (как, например, QString в Qt), а темплейт (basic\_string), 
это сделано для того чтобы поддерживать 
строки содержащие разного типа символы: как минимум \Tchar и \IT{wchar\_t}.

Так что, std::string это класс с базовым типом \Tchar.

А std::wstring это класс с базовым типом \IT{wchar\_t}.

\mysubparagraph{MSVC}

В реализации MSVC, вместо ссылки на буфер может содержаться сам буфер (если строка короче 16-и символов).

Это означает, что каждая короткая строка будет занимать в памяти по крайней мере $16 + 4 + 4 = 24$ 
байт для 32-битной среды либо $16 + 8 + 8 = 32$ 
байта в 64-битной, а если строка длиннее 16-и символов, то прибавьте еще длину самой строки.

\lstinputlisting[caption=пример для MSVC,style=customc]{\CURPATH/STL/string/MSVC_RU.cpp}

Собственно, из этого исходника почти всё ясно.

Несколько замечаний:

Если строка короче 16-и символов, 
то отдельный буфер для строки в \glslink{heap}{куче} выделяться не будет.

Это удобно потому что на практике, основная часть строк действительно короткие.
Вероятно, разработчики в Microsoft выбрали размер в 16 символов как разумный баланс.

Теперь очень важный момент в конце функции main(): мы не пользуемся методом c\_str(), тем не менее,
если это скомпилировать и запустить, то обе строки появятся в консоли!

Работает это вот почему.

В первом случае строка короче 16-и символов и в начале объекта std::string (его можно рассматривать
просто как структуру) расположен буфер с этой строкой.
\printf трактует указатель как указатель на массив символов оканчивающийся нулем и поэтому всё работает.

Вывод второй строки (длиннее 16-и символов) даже еще опаснее: это вообще типичная программистская ошибка 
(или опечатка), забыть дописать c\_str().
Это работает потому что в это время в начале структуры расположен указатель на буфер.
Это может надолго остаться незамеченным: до тех пока там не появится строка 
короче 16-и символов, тогда процесс упадет.

\mysubparagraph{GCC}

В реализации GCC в структуре есть еще одна переменная --- reference count.

Интересно, что указатель на экземпляр класса std::string в GCC указывает не на начало самой структуры, 
а на указатель на буфера.
В libstdc++-v3\textbackslash{}include\textbackslash{}bits\textbackslash{}basic\_string.h 
мы можем прочитать что это сделано для удобства отладки:

\begin{lstlisting}
   *  The reason you want _M_data pointing to the character %array and
   *  not the _Rep is so that the debugger can see the string
   *  contents. (Probably we should add a non-inline member to get
   *  the _Rep for the debugger to use, so users can check the actual
   *  string length.)
\end{lstlisting}

\href{http://go.yurichev.com/17085}{исходный код basic\_string.h}

В нашем примере мы учитываем это:

\lstinputlisting[caption=пример для GCC,style=customc]{\CURPATH/STL/string/GCC_RU.cpp}

Нужны еще небольшие хаки чтобы сымитировать типичную ошибку, которую мы уже видели выше, из-за
более ужесточенной проверки типов в GCC, тем не менее, printf() работает и здесь без c\_str().

\myparagraph{Чуть более сложный пример}

\lstinputlisting[style=customc]{\CURPATH/STL/string/3.cpp}

\lstinputlisting[caption=MSVC 2012,style=customasmx86]{\CURPATH/STL/string/3_MSVC_RU.asm}

Собственно, компилятор не конструирует строки статически: да в общем-то и как
это возможно, если буфер с ней нужно хранить в \glslink{heap}{куче}?

Вместо этого в сегменте данных хранятся обычные \ac{ASCIIZ}-строки, а позже, во время выполнения, 
при помощи метода \q{assign}, конструируются строки s1 и s2
.
При помощи \TT{operator+}, создается строка s3.

Обратите внимание на то что вызов метода c\_str() отсутствует,
потому что его код достаточно короткий и компилятор вставил его прямо здесь:
если строка короче 16-и байт, то в регистре EAX остается указатель на буфер,
а если длиннее, то из этого же места достается адрес на буфер расположенный в \glslink{heap}{куче}.

Далее следуют вызовы трех деструкторов, причем, они вызываются только если строка длиннее 16-и байт:
тогда нужно освободить буфера в \glslink{heap}{куче}.
В противном случае, так как все три объекта std::string хранятся в стеке,
они освобождаются автоматически после выхода из функции.

Следовательно, работа с короткими строками более быстрая из-за м\'{е}ньшего обращения к \glslink{heap}{куче}.

Код на GCC даже проще (из-за того, что в GCC, как мы уже видели, не реализована возможность хранить короткую
строку прямо в структуре):

% TODO1 comment each function meaning
\lstinputlisting[caption=GCC 4.8.1,style=customasmx86]{\CURPATH/STL/string/3_GCC_RU.s}

Можно заметить, что в деструкторы передается не указатель на объект,
а указатель на место за 12 байт (или 3 слова) перед ним, то есть, на настоящее начало структуры.

\myparagraph{std::string как глобальная переменная}
\label{sec:std_string_as_global_variable}

Опытные программисты на \Cpp знают, что глобальные переменные \ac{STL}-типов вполне можно объявлять.

Да, действительно:

\lstinputlisting[style=customc]{\CURPATH/STL/string/5.cpp}

Но как и где будет вызываться конструктор \TT{std::string}?

На самом деле, эта переменная будет инициализирована даже перед началом \main.

\lstinputlisting[caption=MSVC 2012: здесь конструируется глобальная переменная{,} а также регистрируется её деструктор,style=customasmx86]{\CURPATH/STL/string/5_MSVC_p2.asm}

\lstinputlisting[caption=MSVC 2012: здесь глобальная переменная используется в \main,style=customasmx86]{\CURPATH/STL/string/5_MSVC_p1.asm}

\lstinputlisting[caption=MSVC 2012: эта функция-деструктор вызывается перед выходом,style=customasmx86]{\CURPATH/STL/string/5_MSVC_p3.asm}

\myindex{\CStandardLibrary!atexit()}
В реальности, из \ac{CRT}, еще до вызова main(), вызывается специальная функция,
в которой перечислены все конструкторы подобных переменных.
Более того: при помощи atexit() регистрируется функция, которая будет вызвана в конце работы программы:
в этой функции компилятор собирает вызовы деструкторов всех подобных глобальных переменных.

GCC работает похожим образом:

\lstinputlisting[caption=GCC 4.8.1,style=customasmx86]{\CURPATH/STL/string/5_GCC.s}

Но он не выделяет отдельной функции в которой будут собраны деструкторы: 
каждый деструктор передается в atexit() по одному.

% TODO а если глобальная STL-переменная в другом модуле? надо проверить.

}
\ifdefined\SPANISH
\chapter{Patrones de código}
\fi % SPANISH

\ifdefined\GERMAN
\chapter{Code-Muster}
\fi % GERMAN

\ifdefined\ENGLISH
\chapter{Code Patterns}
\fi % ENGLISH

\ifdefined\ITALIAN
\chapter{Forme di codice}
\fi % ITALIAN

\ifdefined\RUSSIAN
\chapter{Образцы кода}
\fi % RUSSIAN

\ifdefined\BRAZILIAN
\chapter{Padrões de códigos}
\fi % BRAZILIAN

\ifdefined\THAI
\chapter{รูปแบบของโค้ด}
\fi % THAI

\ifdefined\FRENCH
\chapter{Modèle de code}
\fi % FRENCH

\ifdefined\POLISH
\chapter{\PLph{}}
\fi % POLISH

% sections
\EN{\input{patterns/patterns_opt_dbg_EN}}
\ES{\input{patterns/patterns_opt_dbg_ES}}
\ITA{\input{patterns/patterns_opt_dbg_ITA}}
\PTBR{\input{patterns/patterns_opt_dbg_PTBR}}
\RU{\input{patterns/patterns_opt_dbg_RU}}
\THA{\input{patterns/patterns_opt_dbg_THA}}
\DE{\input{patterns/patterns_opt_dbg_DE}}
\FR{\input{patterns/patterns_opt_dbg_FR}}
\PL{\input{patterns/patterns_opt_dbg_PL}}

\RU{\section{Некоторые базовые понятия}}
\EN{\section{Some basics}}
\DE{\section{Einige Grundlagen}}
\FR{\section{Quelques bases}}
\ES{\section{\ESph{}}}
\ITA{\section{Alcune basi teoriche}}
\PTBR{\section{\PTBRph{}}}
\THA{\section{\THAph{}}}
\PL{\section{\PLph{}}}

% sections:
\EN{\input{patterns/intro_CPU_ISA_EN}}
\ES{\input{patterns/intro_CPU_ISA_ES}}
\ITA{\input{patterns/intro_CPU_ISA_ITA}}
\PTBR{\input{patterns/intro_CPU_ISA_PTBR}}
\RU{\input{patterns/intro_CPU_ISA_RU}}
\DE{\input{patterns/intro_CPU_ISA_DE}}
\FR{\input{patterns/intro_CPU_ISA_FR}}
\PL{\input{patterns/intro_CPU_ISA_PL}}

\EN{\input{patterns/numeral_EN}}
\RU{\input{patterns/numeral_RU}}
\ITA{\input{patterns/numeral_ITA}}
\DE{\input{patterns/numeral_DE}}
\FR{\input{patterns/numeral_FR}}
\PL{\input{patterns/numeral_PL}}

% chapters
\input{patterns/00_empty/main}
\input{patterns/011_ret/main}
\input{patterns/01_helloworld/main}
\input{patterns/015_prolog_epilogue/main}
\input{patterns/02_stack/main}
\input{patterns/03_printf/main}
\input{patterns/04_scanf/main}
\input{patterns/05_passing_arguments/main}
\input{patterns/06_return_results/main}
\input{patterns/061_pointers/main}
\input{patterns/065_GOTO/main}
\input{patterns/07_jcc/main}
\input{patterns/08_switch/main}
\input{patterns/09_loops/main}
\input{patterns/10_strings/main}
\input{patterns/11_arith_optimizations/main}
\input{patterns/12_FPU/main}
\input{patterns/13_arrays/main}
\input{patterns/14_bitfields/main}
\EN{\input{patterns/145_LCG/main_EN}}
\RU{\input{patterns/145_LCG/main_RU}}
\input{patterns/15_structs/main}
\input{patterns/17_unions/main}
\input{patterns/18_pointers_to_functions/main}
\input{patterns/185_64bit_in_32_env/main}

\EN{\input{patterns/19_SIMD/main_EN}}
\RU{\input{patterns/19_SIMD/main_RU}}
\DE{\input{patterns/19_SIMD/main_DE}}

\EN{\input{patterns/20_x64/main_EN}}
\RU{\input{patterns/20_x64/main_RU}}

\EN{\input{patterns/205_floating_SIMD/main_EN}}
\RU{\input{patterns/205_floating_SIMD/main_RU}}
\DE{\input{patterns/205_floating_SIMD/main_DE}}

\EN{\input{patterns/ARM/main_EN}}
\RU{\input{patterns/ARM/main_RU}}
\DE{\input{patterns/ARM/main_DE}}

\input{patterns/MIPS/main}

\ifdefined\SPANISH
\chapter{Patrones de código}
\fi % SPANISH

\ifdefined\GERMAN
\chapter{Code-Muster}
\fi % GERMAN

\ifdefined\ENGLISH
\chapter{Code Patterns}
\fi % ENGLISH

\ifdefined\ITALIAN
\chapter{Forme di codice}
\fi % ITALIAN

\ifdefined\RUSSIAN
\chapter{Образцы кода}
\fi % RUSSIAN

\ifdefined\BRAZILIAN
\chapter{Padrões de códigos}
\fi % BRAZILIAN

\ifdefined\THAI
\chapter{รูปแบบของโค้ด}
\fi % THAI

\ifdefined\FRENCH
\chapter{Modèle de code}
\fi % FRENCH

\ifdefined\POLISH
\chapter{\PLph{}}
\fi % POLISH

% sections
\EN{\input{patterns/patterns_opt_dbg_EN}}
\ES{\input{patterns/patterns_opt_dbg_ES}}
\ITA{\input{patterns/patterns_opt_dbg_ITA}}
\PTBR{\input{patterns/patterns_opt_dbg_PTBR}}
\RU{\input{patterns/patterns_opt_dbg_RU}}
\THA{\input{patterns/patterns_opt_dbg_THA}}
\DE{\input{patterns/patterns_opt_dbg_DE}}
\FR{\input{patterns/patterns_opt_dbg_FR}}
\PL{\input{patterns/patterns_opt_dbg_PL}}

\RU{\section{Некоторые базовые понятия}}
\EN{\section{Some basics}}
\DE{\section{Einige Grundlagen}}
\FR{\section{Quelques bases}}
\ES{\section{\ESph{}}}
\ITA{\section{Alcune basi teoriche}}
\PTBR{\section{\PTBRph{}}}
\THA{\section{\THAph{}}}
\PL{\section{\PLph{}}}

% sections:
\EN{\input{patterns/intro_CPU_ISA_EN}}
\ES{\input{patterns/intro_CPU_ISA_ES}}
\ITA{\input{patterns/intro_CPU_ISA_ITA}}
\PTBR{\input{patterns/intro_CPU_ISA_PTBR}}
\RU{\input{patterns/intro_CPU_ISA_RU}}
\DE{\input{patterns/intro_CPU_ISA_DE}}
\FR{\input{patterns/intro_CPU_ISA_FR}}
\PL{\input{patterns/intro_CPU_ISA_PL}}

\EN{\input{patterns/numeral_EN}}
\RU{\input{patterns/numeral_RU}}
\ITA{\input{patterns/numeral_ITA}}
\DE{\input{patterns/numeral_DE}}
\FR{\input{patterns/numeral_FR}}
\PL{\input{patterns/numeral_PL}}

% chapters
\input{patterns/00_empty/main}
\input{patterns/011_ret/main}
\input{patterns/01_helloworld/main}
\input{patterns/015_prolog_epilogue/main}
\input{patterns/02_stack/main}
\input{patterns/03_printf/main}
\input{patterns/04_scanf/main}
\input{patterns/05_passing_arguments/main}
\input{patterns/06_return_results/main}
\input{patterns/061_pointers/main}
\input{patterns/065_GOTO/main}
\input{patterns/07_jcc/main}
\input{patterns/08_switch/main}
\input{patterns/09_loops/main}
\input{patterns/10_strings/main}
\input{patterns/11_arith_optimizations/main}
\input{patterns/12_FPU/main}
\input{patterns/13_arrays/main}
\input{patterns/14_bitfields/main}
\EN{\input{patterns/145_LCG/main_EN}}
\RU{\input{patterns/145_LCG/main_RU}}
\input{patterns/15_structs/main}
\input{patterns/17_unions/main}
\input{patterns/18_pointers_to_functions/main}
\input{patterns/185_64bit_in_32_env/main}

\EN{\input{patterns/19_SIMD/main_EN}}
\RU{\input{patterns/19_SIMD/main_RU}}
\DE{\input{patterns/19_SIMD/main_DE}}

\EN{\input{patterns/20_x64/main_EN}}
\RU{\input{patterns/20_x64/main_RU}}

\EN{\input{patterns/205_floating_SIMD/main_EN}}
\RU{\input{patterns/205_floating_SIMD/main_RU}}
\DE{\input{patterns/205_floating_SIMD/main_DE}}

\EN{\input{patterns/ARM/main_EN}}
\RU{\input{patterns/ARM/main_RU}}
\DE{\input{patterns/ARM/main_DE}}

\input{patterns/MIPS/main}

\ifdefined\SPANISH
\chapter{Patrones de código}
\fi % SPANISH

\ifdefined\GERMAN
\chapter{Code-Muster}
\fi % GERMAN

\ifdefined\ENGLISH
\chapter{Code Patterns}
\fi % ENGLISH

\ifdefined\ITALIAN
\chapter{Forme di codice}
\fi % ITALIAN

\ifdefined\RUSSIAN
\chapter{Образцы кода}
\fi % RUSSIAN

\ifdefined\BRAZILIAN
\chapter{Padrões de códigos}
\fi % BRAZILIAN

\ifdefined\THAI
\chapter{รูปแบบของโค้ด}
\fi % THAI

\ifdefined\FRENCH
\chapter{Modèle de code}
\fi % FRENCH

\ifdefined\POLISH
\chapter{\PLph{}}
\fi % POLISH

% sections
\EN{\input{patterns/patterns_opt_dbg_EN}}
\ES{\input{patterns/patterns_opt_dbg_ES}}
\ITA{\input{patterns/patterns_opt_dbg_ITA}}
\PTBR{\input{patterns/patterns_opt_dbg_PTBR}}
\RU{\input{patterns/patterns_opt_dbg_RU}}
\THA{\input{patterns/patterns_opt_dbg_THA}}
\DE{\input{patterns/patterns_opt_dbg_DE}}
\FR{\input{patterns/patterns_opt_dbg_FR}}
\PL{\input{patterns/patterns_opt_dbg_PL}}

\RU{\section{Некоторые базовые понятия}}
\EN{\section{Some basics}}
\DE{\section{Einige Grundlagen}}
\FR{\section{Quelques bases}}
\ES{\section{\ESph{}}}
\ITA{\section{Alcune basi teoriche}}
\PTBR{\section{\PTBRph{}}}
\THA{\section{\THAph{}}}
\PL{\section{\PLph{}}}

% sections:
\EN{\input{patterns/intro_CPU_ISA_EN}}
\ES{\input{patterns/intro_CPU_ISA_ES}}
\ITA{\input{patterns/intro_CPU_ISA_ITA}}
\PTBR{\input{patterns/intro_CPU_ISA_PTBR}}
\RU{\input{patterns/intro_CPU_ISA_RU}}
\DE{\input{patterns/intro_CPU_ISA_DE}}
\FR{\input{patterns/intro_CPU_ISA_FR}}
\PL{\input{patterns/intro_CPU_ISA_PL}}

\EN{\input{patterns/numeral_EN}}
\RU{\input{patterns/numeral_RU}}
\ITA{\input{patterns/numeral_ITA}}
\DE{\input{patterns/numeral_DE}}
\FR{\input{patterns/numeral_FR}}
\PL{\input{patterns/numeral_PL}}

% chapters
\input{patterns/00_empty/main}
\input{patterns/011_ret/main}
\input{patterns/01_helloworld/main}
\input{patterns/015_prolog_epilogue/main}
\input{patterns/02_stack/main}
\input{patterns/03_printf/main}
\input{patterns/04_scanf/main}
\input{patterns/05_passing_arguments/main}
\input{patterns/06_return_results/main}
\input{patterns/061_pointers/main}
\input{patterns/065_GOTO/main}
\input{patterns/07_jcc/main}
\input{patterns/08_switch/main}
\input{patterns/09_loops/main}
\input{patterns/10_strings/main}
\input{patterns/11_arith_optimizations/main}
\input{patterns/12_FPU/main}
\input{patterns/13_arrays/main}
\input{patterns/14_bitfields/main}
\EN{\input{patterns/145_LCG/main_EN}}
\RU{\input{patterns/145_LCG/main_RU}}
\input{patterns/15_structs/main}
\input{patterns/17_unions/main}
\input{patterns/18_pointers_to_functions/main}
\input{patterns/185_64bit_in_32_env/main}

\EN{\input{patterns/19_SIMD/main_EN}}
\RU{\input{patterns/19_SIMD/main_RU}}
\DE{\input{patterns/19_SIMD/main_DE}}

\EN{\input{patterns/20_x64/main_EN}}
\RU{\input{patterns/20_x64/main_RU}}

\EN{\input{patterns/205_floating_SIMD/main_EN}}
\RU{\input{patterns/205_floating_SIMD/main_RU}}
\DE{\input{patterns/205_floating_SIMD/main_DE}}

\EN{\input{patterns/ARM/main_EN}}
\RU{\input{patterns/ARM/main_RU}}
\DE{\input{patterns/ARM/main_DE}}

\input{patterns/MIPS/main}

\ifdefined\SPANISH
\chapter{Patrones de código}
\fi % SPANISH

\ifdefined\GERMAN
\chapter{Code-Muster}
\fi % GERMAN

\ifdefined\ENGLISH
\chapter{Code Patterns}
\fi % ENGLISH

\ifdefined\ITALIAN
\chapter{Forme di codice}
\fi % ITALIAN

\ifdefined\RUSSIAN
\chapter{Образцы кода}
\fi % RUSSIAN

\ifdefined\BRAZILIAN
\chapter{Padrões de códigos}
\fi % BRAZILIAN

\ifdefined\THAI
\chapter{รูปแบบของโค้ด}
\fi % THAI

\ifdefined\FRENCH
\chapter{Modèle de code}
\fi % FRENCH

\ifdefined\POLISH
\chapter{\PLph{}}
\fi % POLISH

% sections
\EN{\input{patterns/patterns_opt_dbg_EN}}
\ES{\input{patterns/patterns_opt_dbg_ES}}
\ITA{\input{patterns/patterns_opt_dbg_ITA}}
\PTBR{\input{patterns/patterns_opt_dbg_PTBR}}
\RU{\input{patterns/patterns_opt_dbg_RU}}
\THA{\input{patterns/patterns_opt_dbg_THA}}
\DE{\input{patterns/patterns_opt_dbg_DE}}
\FR{\input{patterns/patterns_opt_dbg_FR}}
\PL{\input{patterns/patterns_opt_dbg_PL}}

\RU{\section{Некоторые базовые понятия}}
\EN{\section{Some basics}}
\DE{\section{Einige Grundlagen}}
\FR{\section{Quelques bases}}
\ES{\section{\ESph{}}}
\ITA{\section{Alcune basi teoriche}}
\PTBR{\section{\PTBRph{}}}
\THA{\section{\THAph{}}}
\PL{\section{\PLph{}}}

% sections:
\EN{\input{patterns/intro_CPU_ISA_EN}}
\ES{\input{patterns/intro_CPU_ISA_ES}}
\ITA{\input{patterns/intro_CPU_ISA_ITA}}
\PTBR{\input{patterns/intro_CPU_ISA_PTBR}}
\RU{\input{patterns/intro_CPU_ISA_RU}}
\DE{\input{patterns/intro_CPU_ISA_DE}}
\FR{\input{patterns/intro_CPU_ISA_FR}}
\PL{\input{patterns/intro_CPU_ISA_PL}}

\EN{\input{patterns/numeral_EN}}
\RU{\input{patterns/numeral_RU}}
\ITA{\input{patterns/numeral_ITA}}
\DE{\input{patterns/numeral_DE}}
\FR{\input{patterns/numeral_FR}}
\PL{\input{patterns/numeral_PL}}

% chapters
\input{patterns/00_empty/main}
\input{patterns/011_ret/main}
\input{patterns/01_helloworld/main}
\input{patterns/015_prolog_epilogue/main}
\input{patterns/02_stack/main}
\input{patterns/03_printf/main}
\input{patterns/04_scanf/main}
\input{patterns/05_passing_arguments/main}
\input{patterns/06_return_results/main}
\input{patterns/061_pointers/main}
\input{patterns/065_GOTO/main}
\input{patterns/07_jcc/main}
\input{patterns/08_switch/main}
\input{patterns/09_loops/main}
\input{patterns/10_strings/main}
\input{patterns/11_arith_optimizations/main}
\input{patterns/12_FPU/main}
\input{patterns/13_arrays/main}
\input{patterns/14_bitfields/main}
\EN{\input{patterns/145_LCG/main_EN}}
\RU{\input{patterns/145_LCG/main_RU}}
\input{patterns/15_structs/main}
\input{patterns/17_unions/main}
\input{patterns/18_pointers_to_functions/main}
\input{patterns/185_64bit_in_32_env/main}

\EN{\input{patterns/19_SIMD/main_EN}}
\RU{\input{patterns/19_SIMD/main_RU}}
\DE{\input{patterns/19_SIMD/main_DE}}

\EN{\input{patterns/20_x64/main_EN}}
\RU{\input{patterns/20_x64/main_RU}}

\EN{\input{patterns/205_floating_SIMD/main_EN}}
\RU{\input{patterns/205_floating_SIMD/main_RU}}
\DE{\input{patterns/205_floating_SIMD/main_DE}}

\EN{\input{patterns/ARM/main_EN}}
\RU{\input{patterns/ARM/main_RU}}
\DE{\input{patterns/ARM/main_DE}}

\input{patterns/MIPS/main}


\EN{\section{Returning Values}
\label{ret_val_func}

Another simple function is the one that simply returns a constant value:

\lstinputlisting[caption=\EN{\CCpp Code},style=customc]{patterns/011_ret/1.c}

Let's compile it.

\subsection{x86}

Here's what both the GCC and MSVC compilers produce (with optimization) on the x86 platform:

\lstinputlisting[caption=\Optimizing GCC/MSVC (\assemblyOutput),style=customasmx86]{patterns/011_ret/1.s}

\myindex{x86!\Instructions!RET}
There are just two instructions: the first places the value 123 into the \EAX register,
which is used by convention for storing the return
value, and the second one is \RET, which returns execution to the \gls{caller}.

The caller will take the result from the \EAX register.

\subsection{ARM}

There are a few differences on the ARM platform:

\lstinputlisting[caption=\OptimizingKeilVI (\ARMMode) ASM Output,style=customasmARM]{patterns/011_ret/1_Keil_ARM_O3.s}

ARM uses the register \Reg{0} for returning the results of functions, so 123 is copied into \Reg{0}.

\myindex{ARM!\Instructions!MOV}
\myindex{x86!\Instructions!MOV}
It is worth noting that \MOV is a misleading name for the instruction in both the x86 and ARM \ac{ISA}s.

The data is not in fact \IT{moved}, but \IT{copied}.

\subsection{MIPS}

\label{MIPS_leaf_function_ex1}

The GCC assembly output below lists registers by number:

\lstinputlisting[caption=\Optimizing GCC 4.4.5 (\assemblyOutput),style=customasmMIPS]{patterns/011_ret/MIPS.s}

\dots while \IDA does it by their pseudo names:

\lstinputlisting[caption=\Optimizing GCC 4.4.5 (IDA),style=customasmMIPS]{patterns/011_ret/MIPS_IDA.lst}

The \$2 (or \$V0) register is used to store the function's return value.
\myindex{MIPS!\Pseudoinstructions!LI}
\INS{LI} stands for ``Load Immediate'' and is the MIPS equivalent to \MOV.

\myindex{MIPS!\Instructions!J}
The other instruction is the jump instruction (J or JR) which returns the execution flow to the \gls{caller}.

\myindex{MIPS!Branch delay slot}
You might be wondering why the positions of the load instruction (LI) and the jump instruction (J or JR) are swapped. This is due to a \ac{RISC} feature called ``branch delay slot''.

The reason this happens is a quirk in the architecture of some RISC \ac{ISA}s and isn't important for our
purposes---we must simply keep in mind that in MIPS, the instruction following a jump or branch instruction
is executed \IT{before} the jump/branch instruction itself.

As a consequence, branch instructions always swap places with the instruction executed immediately beforehand.


In practice, functions which merely return 1 (\IT{true}) or 0 (\IT{false}) are very frequent.

The smallest ever of the standard UNIX utilities, \IT{/bin/true} and \IT{/bin/false} return 0 and 1 respectively, as an exit code.
(Zero as an exit code usually means success, non-zero means error.)
}
\RU{\subsubsection{std::string}
\myindex{\Cpp!STL!std::string}
\label{std_string}

\myparagraph{Как устроена структура}

Многие строковые библиотеки \InSqBrackets{\CNotes 2.2} обеспечивают структуру содержащую ссылку 
на буфер собственно со строкой, переменная всегда содержащую длину строки 
(что очень удобно для массы функций \InSqBrackets{\CNotes 2.2.1}) и переменную содержащую текущий размер буфера.

Строка в буфере обыкновенно оканчивается нулем: это для того чтобы указатель на буфер можно было
передавать в функции требующие на вход обычную сишную \ac{ASCIIZ}-строку.

Стандарт \Cpp не описывает, как именно нужно реализовывать std::string,
но, как правило, они реализованы как описано выше, с небольшими дополнениями.

Строки в \Cpp это не класс (как, например, QString в Qt), а темплейт (basic\_string), 
это сделано для того чтобы поддерживать 
строки содержащие разного типа символы: как минимум \Tchar и \IT{wchar\_t}.

Так что, std::string это класс с базовым типом \Tchar.

А std::wstring это класс с базовым типом \IT{wchar\_t}.

\mysubparagraph{MSVC}

В реализации MSVC, вместо ссылки на буфер может содержаться сам буфер (если строка короче 16-и символов).

Это означает, что каждая короткая строка будет занимать в памяти по крайней мере $16 + 4 + 4 = 24$ 
байт для 32-битной среды либо $16 + 8 + 8 = 32$ 
байта в 64-битной, а если строка длиннее 16-и символов, то прибавьте еще длину самой строки.

\lstinputlisting[caption=пример для MSVC,style=customc]{\CURPATH/STL/string/MSVC_RU.cpp}

Собственно, из этого исходника почти всё ясно.

Несколько замечаний:

Если строка короче 16-и символов, 
то отдельный буфер для строки в \glslink{heap}{куче} выделяться не будет.

Это удобно потому что на практике, основная часть строк действительно короткие.
Вероятно, разработчики в Microsoft выбрали размер в 16 символов как разумный баланс.

Теперь очень важный момент в конце функции main(): мы не пользуемся методом c\_str(), тем не менее,
если это скомпилировать и запустить, то обе строки появятся в консоли!

Работает это вот почему.

В первом случае строка короче 16-и символов и в начале объекта std::string (его можно рассматривать
просто как структуру) расположен буфер с этой строкой.
\printf трактует указатель как указатель на массив символов оканчивающийся нулем и поэтому всё работает.

Вывод второй строки (длиннее 16-и символов) даже еще опаснее: это вообще типичная программистская ошибка 
(или опечатка), забыть дописать c\_str().
Это работает потому что в это время в начале структуры расположен указатель на буфер.
Это может надолго остаться незамеченным: до тех пока там не появится строка 
короче 16-и символов, тогда процесс упадет.

\mysubparagraph{GCC}

В реализации GCC в структуре есть еще одна переменная --- reference count.

Интересно, что указатель на экземпляр класса std::string в GCC указывает не на начало самой структуры, 
а на указатель на буфера.
В libstdc++-v3\textbackslash{}include\textbackslash{}bits\textbackslash{}basic\_string.h 
мы можем прочитать что это сделано для удобства отладки:

\begin{lstlisting}
   *  The reason you want _M_data pointing to the character %array and
   *  not the _Rep is so that the debugger can see the string
   *  contents. (Probably we should add a non-inline member to get
   *  the _Rep for the debugger to use, so users can check the actual
   *  string length.)
\end{lstlisting}

\href{http://go.yurichev.com/17085}{исходный код basic\_string.h}

В нашем примере мы учитываем это:

\lstinputlisting[caption=пример для GCC,style=customc]{\CURPATH/STL/string/GCC_RU.cpp}

Нужны еще небольшие хаки чтобы сымитировать типичную ошибку, которую мы уже видели выше, из-за
более ужесточенной проверки типов в GCC, тем не менее, printf() работает и здесь без c\_str().

\myparagraph{Чуть более сложный пример}

\lstinputlisting[style=customc]{\CURPATH/STL/string/3.cpp}

\lstinputlisting[caption=MSVC 2012,style=customasmx86]{\CURPATH/STL/string/3_MSVC_RU.asm}

Собственно, компилятор не конструирует строки статически: да в общем-то и как
это возможно, если буфер с ней нужно хранить в \glslink{heap}{куче}?

Вместо этого в сегменте данных хранятся обычные \ac{ASCIIZ}-строки, а позже, во время выполнения, 
при помощи метода \q{assign}, конструируются строки s1 и s2
.
При помощи \TT{operator+}, создается строка s3.

Обратите внимание на то что вызов метода c\_str() отсутствует,
потому что его код достаточно короткий и компилятор вставил его прямо здесь:
если строка короче 16-и байт, то в регистре EAX остается указатель на буфер,
а если длиннее, то из этого же места достается адрес на буфер расположенный в \glslink{heap}{куче}.

Далее следуют вызовы трех деструкторов, причем, они вызываются только если строка длиннее 16-и байт:
тогда нужно освободить буфера в \glslink{heap}{куче}.
В противном случае, так как все три объекта std::string хранятся в стеке,
они освобождаются автоматически после выхода из функции.

Следовательно, работа с короткими строками более быстрая из-за м\'{е}ньшего обращения к \glslink{heap}{куче}.

Код на GCC даже проще (из-за того, что в GCC, как мы уже видели, не реализована возможность хранить короткую
строку прямо в структуре):

% TODO1 comment each function meaning
\lstinputlisting[caption=GCC 4.8.1,style=customasmx86]{\CURPATH/STL/string/3_GCC_RU.s}

Можно заметить, что в деструкторы передается не указатель на объект,
а указатель на место за 12 байт (или 3 слова) перед ним, то есть, на настоящее начало структуры.

\myparagraph{std::string как глобальная переменная}
\label{sec:std_string_as_global_variable}

Опытные программисты на \Cpp знают, что глобальные переменные \ac{STL}-типов вполне можно объявлять.

Да, действительно:

\lstinputlisting[style=customc]{\CURPATH/STL/string/5.cpp}

Но как и где будет вызываться конструктор \TT{std::string}?

На самом деле, эта переменная будет инициализирована даже перед началом \main.

\lstinputlisting[caption=MSVC 2012: здесь конструируется глобальная переменная{,} а также регистрируется её деструктор,style=customasmx86]{\CURPATH/STL/string/5_MSVC_p2.asm}

\lstinputlisting[caption=MSVC 2012: здесь глобальная переменная используется в \main,style=customasmx86]{\CURPATH/STL/string/5_MSVC_p1.asm}

\lstinputlisting[caption=MSVC 2012: эта функция-деструктор вызывается перед выходом,style=customasmx86]{\CURPATH/STL/string/5_MSVC_p3.asm}

\myindex{\CStandardLibrary!atexit()}
В реальности, из \ac{CRT}, еще до вызова main(), вызывается специальная функция,
в которой перечислены все конструкторы подобных переменных.
Более того: при помощи atexit() регистрируется функция, которая будет вызвана в конце работы программы:
в этой функции компилятор собирает вызовы деструкторов всех подобных глобальных переменных.

GCC работает похожим образом:

\lstinputlisting[caption=GCC 4.8.1,style=customasmx86]{\CURPATH/STL/string/5_GCC.s}

Но он не выделяет отдельной функции в которой будут собраны деструкторы: 
каждый деструктор передается в atexit() по одному.

% TODO а если глобальная STL-переменная в другом модуле? надо проверить.

}
\DE{\subsection{Einfachste XOR-Verschlüsselung überhaupt}

Ich habe einmal eine Software gesehen, bei der alle Debugging-Ausgaben mit XOR mit dem Wert 3
verschlüsselt wurden. Mit anderen Worten, die beiden niedrigsten Bits aller Buchstaben wurden invertiert.

``Hello, world'' wurde zu ``Kfool/\#tlqog'':

\begin{lstlisting}
#!/usr/bin/python

msg="Hello, world!"

print "".join(map(lambda x: chr(ord(x)^3), msg))
\end{lstlisting}

Das ist eine ziemlich interessante Verschlüsselung (oder besser eine Verschleierung),
weil sie zwei wichtige Eigenschaften hat:
1) es ist eine einzige Funktion zum Verschlüsseln und entschlüsseln, sie muss nur wiederholt angewendet werden
2) die entstehenden Buchstaben befinden sich im druckbaren Bereich, also die ganze Zeichenkette kann ohne
Escape-Symbole im Code verwendet werden.

Die zweite Eigenschaft nutzt die Tatsache, dass alle druckbaren Zeichen in Reihen organisiert sind: 0x2x-0x7x,
und wenn die beiden niederwertigsten Bits invertiert werden, wird der Buchstabe um eine oder drei Stellen nach
links oder rechts \IT{verschoben}, aber niemals in eine andere Reihe:

\begin{figure}[H]
\centering
\includegraphics[width=0.7\textwidth]{ascii_clean.png}
\caption{7-Bit \ac{ASCII} Tabelle in Emacs}
\end{figure}

\dots mit dem Zeichen 0x7F als einziger Ausnahme.

Im Folgenden werden also beispielsweise die Zeichen A-Z \IT{verschlüsselt}:

\begin{lstlisting}
#!/usr/bin/python

msg="@ABCDEFGHIJKLMNO"

print "".join(map(lambda x: chr(ord(x)^3), msg))
\end{lstlisting}

Ergebnis:
% FIXME \verb  --  relevant comment for German?
\begin{lstlisting}
CBA@GFEDKJIHONML
\end{lstlisting}

Es sieht so aus als würden die Zeichen ``@'' und ``C'' sowie ``B'' und ``A'' vertauscht werden.

Hier ist noch ein interessantes Beispiel, in dem gezeigt wird, wie die Eigenschaften von XOR
ausgenutzt werden können: Exakt den gleichen Effekt, dass druckbare Zeichen auch druckbar bleiben,
kann man dadurch erzielen, dass irgendeine Kombination der niedrigsten vier Bits invertiert wird.
}

\EN{\section{Returning Values}
\label{ret_val_func}

Another simple function is the one that simply returns a constant value:

\lstinputlisting[caption=\EN{\CCpp Code},style=customc]{patterns/011_ret/1.c}

Let's compile it.

\subsection{x86}

Here's what both the GCC and MSVC compilers produce (with optimization) on the x86 platform:

\lstinputlisting[caption=\Optimizing GCC/MSVC (\assemblyOutput),style=customasmx86]{patterns/011_ret/1.s}

\myindex{x86!\Instructions!RET}
There are just two instructions: the first places the value 123 into the \EAX register,
which is used by convention for storing the return
value, and the second one is \RET, which returns execution to the \gls{caller}.

The caller will take the result from the \EAX register.

\subsection{ARM}

There are a few differences on the ARM platform:

\lstinputlisting[caption=\OptimizingKeilVI (\ARMMode) ASM Output,style=customasmARM]{patterns/011_ret/1_Keil_ARM_O3.s}

ARM uses the register \Reg{0} for returning the results of functions, so 123 is copied into \Reg{0}.

\myindex{ARM!\Instructions!MOV}
\myindex{x86!\Instructions!MOV}
It is worth noting that \MOV is a misleading name for the instruction in both the x86 and ARM \ac{ISA}s.

The data is not in fact \IT{moved}, but \IT{copied}.

\subsection{MIPS}

\label{MIPS_leaf_function_ex1}

The GCC assembly output below lists registers by number:

\lstinputlisting[caption=\Optimizing GCC 4.4.5 (\assemblyOutput),style=customasmMIPS]{patterns/011_ret/MIPS.s}

\dots while \IDA does it by their pseudo names:

\lstinputlisting[caption=\Optimizing GCC 4.4.5 (IDA),style=customasmMIPS]{patterns/011_ret/MIPS_IDA.lst}

The \$2 (or \$V0) register is used to store the function's return value.
\myindex{MIPS!\Pseudoinstructions!LI}
\INS{LI} stands for ``Load Immediate'' and is the MIPS equivalent to \MOV.

\myindex{MIPS!\Instructions!J}
The other instruction is the jump instruction (J or JR) which returns the execution flow to the \gls{caller}.

\myindex{MIPS!Branch delay slot}
You might be wondering why the positions of the load instruction (LI) and the jump instruction (J or JR) are swapped. This is due to a \ac{RISC} feature called ``branch delay slot''.

The reason this happens is a quirk in the architecture of some RISC \ac{ISA}s and isn't important for our
purposes---we must simply keep in mind that in MIPS, the instruction following a jump or branch instruction
is executed \IT{before} the jump/branch instruction itself.

As a consequence, branch instructions always swap places with the instruction executed immediately beforehand.


In practice, functions which merely return 1 (\IT{true}) or 0 (\IT{false}) are very frequent.

The smallest ever of the standard UNIX utilities, \IT{/bin/true} and \IT{/bin/false} return 0 and 1 respectively, as an exit code.
(Zero as an exit code usually means success, non-zero means error.)
}
\RU{\subsubsection{std::string}
\myindex{\Cpp!STL!std::string}
\label{std_string}

\myparagraph{Как устроена структура}

Многие строковые библиотеки \InSqBrackets{\CNotes 2.2} обеспечивают структуру содержащую ссылку 
на буфер собственно со строкой, переменная всегда содержащую длину строки 
(что очень удобно для массы функций \InSqBrackets{\CNotes 2.2.1}) и переменную содержащую текущий размер буфера.

Строка в буфере обыкновенно оканчивается нулем: это для того чтобы указатель на буфер можно было
передавать в функции требующие на вход обычную сишную \ac{ASCIIZ}-строку.

Стандарт \Cpp не описывает, как именно нужно реализовывать std::string,
но, как правило, они реализованы как описано выше, с небольшими дополнениями.

Строки в \Cpp это не класс (как, например, QString в Qt), а темплейт (basic\_string), 
это сделано для того чтобы поддерживать 
строки содержащие разного типа символы: как минимум \Tchar и \IT{wchar\_t}.

Так что, std::string это класс с базовым типом \Tchar.

А std::wstring это класс с базовым типом \IT{wchar\_t}.

\mysubparagraph{MSVC}

В реализации MSVC, вместо ссылки на буфер может содержаться сам буфер (если строка короче 16-и символов).

Это означает, что каждая короткая строка будет занимать в памяти по крайней мере $16 + 4 + 4 = 24$ 
байт для 32-битной среды либо $16 + 8 + 8 = 32$ 
байта в 64-битной, а если строка длиннее 16-и символов, то прибавьте еще длину самой строки.

\lstinputlisting[caption=пример для MSVC,style=customc]{\CURPATH/STL/string/MSVC_RU.cpp}

Собственно, из этого исходника почти всё ясно.

Несколько замечаний:

Если строка короче 16-и символов, 
то отдельный буфер для строки в \glslink{heap}{куче} выделяться не будет.

Это удобно потому что на практике, основная часть строк действительно короткие.
Вероятно, разработчики в Microsoft выбрали размер в 16 символов как разумный баланс.

Теперь очень важный момент в конце функции main(): мы не пользуемся методом c\_str(), тем не менее,
если это скомпилировать и запустить, то обе строки появятся в консоли!

Работает это вот почему.

В первом случае строка короче 16-и символов и в начале объекта std::string (его можно рассматривать
просто как структуру) расположен буфер с этой строкой.
\printf трактует указатель как указатель на массив символов оканчивающийся нулем и поэтому всё работает.

Вывод второй строки (длиннее 16-и символов) даже еще опаснее: это вообще типичная программистская ошибка 
(или опечатка), забыть дописать c\_str().
Это работает потому что в это время в начале структуры расположен указатель на буфер.
Это может надолго остаться незамеченным: до тех пока там не появится строка 
короче 16-и символов, тогда процесс упадет.

\mysubparagraph{GCC}

В реализации GCC в структуре есть еще одна переменная --- reference count.

Интересно, что указатель на экземпляр класса std::string в GCC указывает не на начало самой структуры, 
а на указатель на буфера.
В libstdc++-v3\textbackslash{}include\textbackslash{}bits\textbackslash{}basic\_string.h 
мы можем прочитать что это сделано для удобства отладки:

\begin{lstlisting}
   *  The reason you want _M_data pointing to the character %array and
   *  not the _Rep is so that the debugger can see the string
   *  contents. (Probably we should add a non-inline member to get
   *  the _Rep for the debugger to use, so users can check the actual
   *  string length.)
\end{lstlisting}

\href{http://go.yurichev.com/17085}{исходный код basic\_string.h}

В нашем примере мы учитываем это:

\lstinputlisting[caption=пример для GCC,style=customc]{\CURPATH/STL/string/GCC_RU.cpp}

Нужны еще небольшие хаки чтобы сымитировать типичную ошибку, которую мы уже видели выше, из-за
более ужесточенной проверки типов в GCC, тем не менее, printf() работает и здесь без c\_str().

\myparagraph{Чуть более сложный пример}

\lstinputlisting[style=customc]{\CURPATH/STL/string/3.cpp}

\lstinputlisting[caption=MSVC 2012,style=customasmx86]{\CURPATH/STL/string/3_MSVC_RU.asm}

Собственно, компилятор не конструирует строки статически: да в общем-то и как
это возможно, если буфер с ней нужно хранить в \glslink{heap}{куче}?

Вместо этого в сегменте данных хранятся обычные \ac{ASCIIZ}-строки, а позже, во время выполнения, 
при помощи метода \q{assign}, конструируются строки s1 и s2
.
При помощи \TT{operator+}, создается строка s3.

Обратите внимание на то что вызов метода c\_str() отсутствует,
потому что его код достаточно короткий и компилятор вставил его прямо здесь:
если строка короче 16-и байт, то в регистре EAX остается указатель на буфер,
а если длиннее, то из этого же места достается адрес на буфер расположенный в \glslink{heap}{куче}.

Далее следуют вызовы трех деструкторов, причем, они вызываются только если строка длиннее 16-и байт:
тогда нужно освободить буфера в \glslink{heap}{куче}.
В противном случае, так как все три объекта std::string хранятся в стеке,
они освобождаются автоматически после выхода из функции.

Следовательно, работа с короткими строками более быстрая из-за м\'{е}ньшего обращения к \glslink{heap}{куче}.

Код на GCC даже проще (из-за того, что в GCC, как мы уже видели, не реализована возможность хранить короткую
строку прямо в структуре):

% TODO1 comment each function meaning
\lstinputlisting[caption=GCC 4.8.1,style=customasmx86]{\CURPATH/STL/string/3_GCC_RU.s}

Можно заметить, что в деструкторы передается не указатель на объект,
а указатель на место за 12 байт (или 3 слова) перед ним, то есть, на настоящее начало структуры.

\myparagraph{std::string как глобальная переменная}
\label{sec:std_string_as_global_variable}

Опытные программисты на \Cpp знают, что глобальные переменные \ac{STL}-типов вполне можно объявлять.

Да, действительно:

\lstinputlisting[style=customc]{\CURPATH/STL/string/5.cpp}

Но как и где будет вызываться конструктор \TT{std::string}?

На самом деле, эта переменная будет инициализирована даже перед началом \main.

\lstinputlisting[caption=MSVC 2012: здесь конструируется глобальная переменная{,} а также регистрируется её деструктор,style=customasmx86]{\CURPATH/STL/string/5_MSVC_p2.asm}

\lstinputlisting[caption=MSVC 2012: здесь глобальная переменная используется в \main,style=customasmx86]{\CURPATH/STL/string/5_MSVC_p1.asm}

\lstinputlisting[caption=MSVC 2012: эта функция-деструктор вызывается перед выходом,style=customasmx86]{\CURPATH/STL/string/5_MSVC_p3.asm}

\myindex{\CStandardLibrary!atexit()}
В реальности, из \ac{CRT}, еще до вызова main(), вызывается специальная функция,
в которой перечислены все конструкторы подобных переменных.
Более того: при помощи atexit() регистрируется функция, которая будет вызвана в конце работы программы:
в этой функции компилятор собирает вызовы деструкторов всех подобных глобальных переменных.

GCC работает похожим образом:

\lstinputlisting[caption=GCC 4.8.1,style=customasmx86]{\CURPATH/STL/string/5_GCC.s}

Но он не выделяет отдельной функции в которой будут собраны деструкторы: 
каждый деструктор передается в atexit() по одному.

% TODO а если глобальная STL-переменная в другом модуле? надо проверить.

}

\EN{\section{Returning Values}
\label{ret_val_func}

Another simple function is the one that simply returns a constant value:

\lstinputlisting[caption=\EN{\CCpp Code},style=customc]{patterns/011_ret/1.c}

Let's compile it.

\subsection{x86}

Here's what both the GCC and MSVC compilers produce (with optimization) on the x86 platform:

\lstinputlisting[caption=\Optimizing GCC/MSVC (\assemblyOutput),style=customasmx86]{patterns/011_ret/1.s}

\myindex{x86!\Instructions!RET}
There are just two instructions: the first places the value 123 into the \EAX register,
which is used by convention for storing the return
value, and the second one is \RET, which returns execution to the \gls{caller}.

The caller will take the result from the \EAX register.

\subsection{ARM}

There are a few differences on the ARM platform:

\lstinputlisting[caption=\OptimizingKeilVI (\ARMMode) ASM Output,style=customasmARM]{patterns/011_ret/1_Keil_ARM_O3.s}

ARM uses the register \Reg{0} for returning the results of functions, so 123 is copied into \Reg{0}.

\myindex{ARM!\Instructions!MOV}
\myindex{x86!\Instructions!MOV}
It is worth noting that \MOV is a misleading name for the instruction in both the x86 and ARM \ac{ISA}s.

The data is not in fact \IT{moved}, but \IT{copied}.

\subsection{MIPS}

\label{MIPS_leaf_function_ex1}

The GCC assembly output below lists registers by number:

\lstinputlisting[caption=\Optimizing GCC 4.4.5 (\assemblyOutput),style=customasmMIPS]{patterns/011_ret/MIPS.s}

\dots while \IDA does it by their pseudo names:

\lstinputlisting[caption=\Optimizing GCC 4.4.5 (IDA),style=customasmMIPS]{patterns/011_ret/MIPS_IDA.lst}

The \$2 (or \$V0) register is used to store the function's return value.
\myindex{MIPS!\Pseudoinstructions!LI}
\INS{LI} stands for ``Load Immediate'' and is the MIPS equivalent to \MOV.

\myindex{MIPS!\Instructions!J}
The other instruction is the jump instruction (J or JR) which returns the execution flow to the \gls{caller}.

\myindex{MIPS!Branch delay slot}
You might be wondering why the positions of the load instruction (LI) and the jump instruction (J or JR) are swapped. This is due to a \ac{RISC} feature called ``branch delay slot''.

The reason this happens is a quirk in the architecture of some RISC \ac{ISA}s and isn't important for our
purposes---we must simply keep in mind that in MIPS, the instruction following a jump or branch instruction
is executed \IT{before} the jump/branch instruction itself.

As a consequence, branch instructions always swap places with the instruction executed immediately beforehand.


In practice, functions which merely return 1 (\IT{true}) or 0 (\IT{false}) are very frequent.

The smallest ever of the standard UNIX utilities, \IT{/bin/true} and \IT{/bin/false} return 0 and 1 respectively, as an exit code.
(Zero as an exit code usually means success, non-zero means error.)
}
\RU{\subsubsection{std::string}
\myindex{\Cpp!STL!std::string}
\label{std_string}

\myparagraph{Как устроена структура}

Многие строковые библиотеки \InSqBrackets{\CNotes 2.2} обеспечивают структуру содержащую ссылку 
на буфер собственно со строкой, переменная всегда содержащую длину строки 
(что очень удобно для массы функций \InSqBrackets{\CNotes 2.2.1}) и переменную содержащую текущий размер буфера.

Строка в буфере обыкновенно оканчивается нулем: это для того чтобы указатель на буфер можно было
передавать в функции требующие на вход обычную сишную \ac{ASCIIZ}-строку.

Стандарт \Cpp не описывает, как именно нужно реализовывать std::string,
но, как правило, они реализованы как описано выше, с небольшими дополнениями.

Строки в \Cpp это не класс (как, например, QString в Qt), а темплейт (basic\_string), 
это сделано для того чтобы поддерживать 
строки содержащие разного типа символы: как минимум \Tchar и \IT{wchar\_t}.

Так что, std::string это класс с базовым типом \Tchar.

А std::wstring это класс с базовым типом \IT{wchar\_t}.

\mysubparagraph{MSVC}

В реализации MSVC, вместо ссылки на буфер может содержаться сам буфер (если строка короче 16-и символов).

Это означает, что каждая короткая строка будет занимать в памяти по крайней мере $16 + 4 + 4 = 24$ 
байт для 32-битной среды либо $16 + 8 + 8 = 32$ 
байта в 64-битной, а если строка длиннее 16-и символов, то прибавьте еще длину самой строки.

\lstinputlisting[caption=пример для MSVC,style=customc]{\CURPATH/STL/string/MSVC_RU.cpp}

Собственно, из этого исходника почти всё ясно.

Несколько замечаний:

Если строка короче 16-и символов, 
то отдельный буфер для строки в \glslink{heap}{куче} выделяться не будет.

Это удобно потому что на практике, основная часть строк действительно короткие.
Вероятно, разработчики в Microsoft выбрали размер в 16 символов как разумный баланс.

Теперь очень важный момент в конце функции main(): мы не пользуемся методом c\_str(), тем не менее,
если это скомпилировать и запустить, то обе строки появятся в консоли!

Работает это вот почему.

В первом случае строка короче 16-и символов и в начале объекта std::string (его можно рассматривать
просто как структуру) расположен буфер с этой строкой.
\printf трактует указатель как указатель на массив символов оканчивающийся нулем и поэтому всё работает.

Вывод второй строки (длиннее 16-и символов) даже еще опаснее: это вообще типичная программистская ошибка 
(или опечатка), забыть дописать c\_str().
Это работает потому что в это время в начале структуры расположен указатель на буфер.
Это может надолго остаться незамеченным: до тех пока там не появится строка 
короче 16-и символов, тогда процесс упадет.

\mysubparagraph{GCC}

В реализации GCC в структуре есть еще одна переменная --- reference count.

Интересно, что указатель на экземпляр класса std::string в GCC указывает не на начало самой структуры, 
а на указатель на буфера.
В libstdc++-v3\textbackslash{}include\textbackslash{}bits\textbackslash{}basic\_string.h 
мы можем прочитать что это сделано для удобства отладки:

\begin{lstlisting}
   *  The reason you want _M_data pointing to the character %array and
   *  not the _Rep is so that the debugger can see the string
   *  contents. (Probably we should add a non-inline member to get
   *  the _Rep for the debugger to use, so users can check the actual
   *  string length.)
\end{lstlisting}

\href{http://go.yurichev.com/17085}{исходный код basic\_string.h}

В нашем примере мы учитываем это:

\lstinputlisting[caption=пример для GCC,style=customc]{\CURPATH/STL/string/GCC_RU.cpp}

Нужны еще небольшие хаки чтобы сымитировать типичную ошибку, которую мы уже видели выше, из-за
более ужесточенной проверки типов в GCC, тем не менее, printf() работает и здесь без c\_str().

\myparagraph{Чуть более сложный пример}

\lstinputlisting[style=customc]{\CURPATH/STL/string/3.cpp}

\lstinputlisting[caption=MSVC 2012,style=customasmx86]{\CURPATH/STL/string/3_MSVC_RU.asm}

Собственно, компилятор не конструирует строки статически: да в общем-то и как
это возможно, если буфер с ней нужно хранить в \glslink{heap}{куче}?

Вместо этого в сегменте данных хранятся обычные \ac{ASCIIZ}-строки, а позже, во время выполнения, 
при помощи метода \q{assign}, конструируются строки s1 и s2
.
При помощи \TT{operator+}, создается строка s3.

Обратите внимание на то что вызов метода c\_str() отсутствует,
потому что его код достаточно короткий и компилятор вставил его прямо здесь:
если строка короче 16-и байт, то в регистре EAX остается указатель на буфер,
а если длиннее, то из этого же места достается адрес на буфер расположенный в \glslink{heap}{куче}.

Далее следуют вызовы трех деструкторов, причем, они вызываются только если строка длиннее 16-и байт:
тогда нужно освободить буфера в \glslink{heap}{куче}.
В противном случае, так как все три объекта std::string хранятся в стеке,
они освобождаются автоматически после выхода из функции.

Следовательно, работа с короткими строками более быстрая из-за м\'{е}ньшего обращения к \glslink{heap}{куче}.

Код на GCC даже проще (из-за того, что в GCC, как мы уже видели, не реализована возможность хранить короткую
строку прямо в структуре):

% TODO1 comment each function meaning
\lstinputlisting[caption=GCC 4.8.1,style=customasmx86]{\CURPATH/STL/string/3_GCC_RU.s}

Можно заметить, что в деструкторы передается не указатель на объект,
а указатель на место за 12 байт (или 3 слова) перед ним, то есть, на настоящее начало структуры.

\myparagraph{std::string как глобальная переменная}
\label{sec:std_string_as_global_variable}

Опытные программисты на \Cpp знают, что глобальные переменные \ac{STL}-типов вполне можно объявлять.

Да, действительно:

\lstinputlisting[style=customc]{\CURPATH/STL/string/5.cpp}

Но как и где будет вызываться конструктор \TT{std::string}?

На самом деле, эта переменная будет инициализирована даже перед началом \main.

\lstinputlisting[caption=MSVC 2012: здесь конструируется глобальная переменная{,} а также регистрируется её деструктор,style=customasmx86]{\CURPATH/STL/string/5_MSVC_p2.asm}

\lstinputlisting[caption=MSVC 2012: здесь глобальная переменная используется в \main,style=customasmx86]{\CURPATH/STL/string/5_MSVC_p1.asm}

\lstinputlisting[caption=MSVC 2012: эта функция-деструктор вызывается перед выходом,style=customasmx86]{\CURPATH/STL/string/5_MSVC_p3.asm}

\myindex{\CStandardLibrary!atexit()}
В реальности, из \ac{CRT}, еще до вызова main(), вызывается специальная функция,
в которой перечислены все конструкторы подобных переменных.
Более того: при помощи atexit() регистрируется функция, которая будет вызвана в конце работы программы:
в этой функции компилятор собирает вызовы деструкторов всех подобных глобальных переменных.

GCC работает похожим образом:

\lstinputlisting[caption=GCC 4.8.1,style=customasmx86]{\CURPATH/STL/string/5_GCC.s}

Но он не выделяет отдельной функции в которой будут собраны деструкторы: 
каждый деструктор передается в atexit() по одному.

% TODO а если глобальная STL-переменная в другом модуле? надо проверить.

}
\DE{\subsection{Einfachste XOR-Verschlüsselung überhaupt}

Ich habe einmal eine Software gesehen, bei der alle Debugging-Ausgaben mit XOR mit dem Wert 3
verschlüsselt wurden. Mit anderen Worten, die beiden niedrigsten Bits aller Buchstaben wurden invertiert.

``Hello, world'' wurde zu ``Kfool/\#tlqog'':

\begin{lstlisting}
#!/usr/bin/python

msg="Hello, world!"

print "".join(map(lambda x: chr(ord(x)^3), msg))
\end{lstlisting}

Das ist eine ziemlich interessante Verschlüsselung (oder besser eine Verschleierung),
weil sie zwei wichtige Eigenschaften hat:
1) es ist eine einzige Funktion zum Verschlüsseln und entschlüsseln, sie muss nur wiederholt angewendet werden
2) die entstehenden Buchstaben befinden sich im druckbaren Bereich, also die ganze Zeichenkette kann ohne
Escape-Symbole im Code verwendet werden.

Die zweite Eigenschaft nutzt die Tatsache, dass alle druckbaren Zeichen in Reihen organisiert sind: 0x2x-0x7x,
und wenn die beiden niederwertigsten Bits invertiert werden, wird der Buchstabe um eine oder drei Stellen nach
links oder rechts \IT{verschoben}, aber niemals in eine andere Reihe:

\begin{figure}[H]
\centering
\includegraphics[width=0.7\textwidth]{ascii_clean.png}
\caption{7-Bit \ac{ASCII} Tabelle in Emacs}
\end{figure}

\dots mit dem Zeichen 0x7F als einziger Ausnahme.

Im Folgenden werden also beispielsweise die Zeichen A-Z \IT{verschlüsselt}:

\begin{lstlisting}
#!/usr/bin/python

msg="@ABCDEFGHIJKLMNO"

print "".join(map(lambda x: chr(ord(x)^3), msg))
\end{lstlisting}

Ergebnis:
% FIXME \verb  --  relevant comment for German?
\begin{lstlisting}
CBA@GFEDKJIHONML
\end{lstlisting}

Es sieht so aus als würden die Zeichen ``@'' und ``C'' sowie ``B'' und ``A'' vertauscht werden.

Hier ist noch ein interessantes Beispiel, in dem gezeigt wird, wie die Eigenschaften von XOR
ausgenutzt werden können: Exakt den gleichen Effekt, dass druckbare Zeichen auch druckbar bleiben,
kann man dadurch erzielen, dass irgendeine Kombination der niedrigsten vier Bits invertiert wird.
}

\EN{\section{Returning Values}
\label{ret_val_func}

Another simple function is the one that simply returns a constant value:

\lstinputlisting[caption=\EN{\CCpp Code},style=customc]{patterns/011_ret/1.c}

Let's compile it.

\subsection{x86}

Here's what both the GCC and MSVC compilers produce (with optimization) on the x86 platform:

\lstinputlisting[caption=\Optimizing GCC/MSVC (\assemblyOutput),style=customasmx86]{patterns/011_ret/1.s}

\myindex{x86!\Instructions!RET}
There are just two instructions: the first places the value 123 into the \EAX register,
which is used by convention for storing the return
value, and the second one is \RET, which returns execution to the \gls{caller}.

The caller will take the result from the \EAX register.

\subsection{ARM}

There are a few differences on the ARM platform:

\lstinputlisting[caption=\OptimizingKeilVI (\ARMMode) ASM Output,style=customasmARM]{patterns/011_ret/1_Keil_ARM_O3.s}

ARM uses the register \Reg{0} for returning the results of functions, so 123 is copied into \Reg{0}.

\myindex{ARM!\Instructions!MOV}
\myindex{x86!\Instructions!MOV}
It is worth noting that \MOV is a misleading name for the instruction in both the x86 and ARM \ac{ISA}s.

The data is not in fact \IT{moved}, but \IT{copied}.

\subsection{MIPS}

\label{MIPS_leaf_function_ex1}

The GCC assembly output below lists registers by number:

\lstinputlisting[caption=\Optimizing GCC 4.4.5 (\assemblyOutput),style=customasmMIPS]{patterns/011_ret/MIPS.s}

\dots while \IDA does it by their pseudo names:

\lstinputlisting[caption=\Optimizing GCC 4.4.5 (IDA),style=customasmMIPS]{patterns/011_ret/MIPS_IDA.lst}

The \$2 (or \$V0) register is used to store the function's return value.
\myindex{MIPS!\Pseudoinstructions!LI}
\INS{LI} stands for ``Load Immediate'' and is the MIPS equivalent to \MOV.

\myindex{MIPS!\Instructions!J}
The other instruction is the jump instruction (J or JR) which returns the execution flow to the \gls{caller}.

\myindex{MIPS!Branch delay slot}
You might be wondering why the positions of the load instruction (LI) and the jump instruction (J or JR) are swapped. This is due to a \ac{RISC} feature called ``branch delay slot''.

The reason this happens is a quirk in the architecture of some RISC \ac{ISA}s and isn't important for our
purposes---we must simply keep in mind that in MIPS, the instruction following a jump or branch instruction
is executed \IT{before} the jump/branch instruction itself.

As a consequence, branch instructions always swap places with the instruction executed immediately beforehand.


In practice, functions which merely return 1 (\IT{true}) or 0 (\IT{false}) are very frequent.

The smallest ever of the standard UNIX utilities, \IT{/bin/true} and \IT{/bin/false} return 0 and 1 respectively, as an exit code.
(Zero as an exit code usually means success, non-zero means error.)
}
\RU{\subsubsection{std::string}
\myindex{\Cpp!STL!std::string}
\label{std_string}

\myparagraph{Как устроена структура}

Многие строковые библиотеки \InSqBrackets{\CNotes 2.2} обеспечивают структуру содержащую ссылку 
на буфер собственно со строкой, переменная всегда содержащую длину строки 
(что очень удобно для массы функций \InSqBrackets{\CNotes 2.2.1}) и переменную содержащую текущий размер буфера.

Строка в буфере обыкновенно оканчивается нулем: это для того чтобы указатель на буфер можно было
передавать в функции требующие на вход обычную сишную \ac{ASCIIZ}-строку.

Стандарт \Cpp не описывает, как именно нужно реализовывать std::string,
но, как правило, они реализованы как описано выше, с небольшими дополнениями.

Строки в \Cpp это не класс (как, например, QString в Qt), а темплейт (basic\_string), 
это сделано для того чтобы поддерживать 
строки содержащие разного типа символы: как минимум \Tchar и \IT{wchar\_t}.

Так что, std::string это класс с базовым типом \Tchar.

А std::wstring это класс с базовым типом \IT{wchar\_t}.

\mysubparagraph{MSVC}

В реализации MSVC, вместо ссылки на буфер может содержаться сам буфер (если строка короче 16-и символов).

Это означает, что каждая короткая строка будет занимать в памяти по крайней мере $16 + 4 + 4 = 24$ 
байт для 32-битной среды либо $16 + 8 + 8 = 32$ 
байта в 64-битной, а если строка длиннее 16-и символов, то прибавьте еще длину самой строки.

\lstinputlisting[caption=пример для MSVC,style=customc]{\CURPATH/STL/string/MSVC_RU.cpp}

Собственно, из этого исходника почти всё ясно.

Несколько замечаний:

Если строка короче 16-и символов, 
то отдельный буфер для строки в \glslink{heap}{куче} выделяться не будет.

Это удобно потому что на практике, основная часть строк действительно короткие.
Вероятно, разработчики в Microsoft выбрали размер в 16 символов как разумный баланс.

Теперь очень важный момент в конце функции main(): мы не пользуемся методом c\_str(), тем не менее,
если это скомпилировать и запустить, то обе строки появятся в консоли!

Работает это вот почему.

В первом случае строка короче 16-и символов и в начале объекта std::string (его можно рассматривать
просто как структуру) расположен буфер с этой строкой.
\printf трактует указатель как указатель на массив символов оканчивающийся нулем и поэтому всё работает.

Вывод второй строки (длиннее 16-и символов) даже еще опаснее: это вообще типичная программистская ошибка 
(или опечатка), забыть дописать c\_str().
Это работает потому что в это время в начале структуры расположен указатель на буфер.
Это может надолго остаться незамеченным: до тех пока там не появится строка 
короче 16-и символов, тогда процесс упадет.

\mysubparagraph{GCC}

В реализации GCC в структуре есть еще одна переменная --- reference count.

Интересно, что указатель на экземпляр класса std::string в GCC указывает не на начало самой структуры, 
а на указатель на буфера.
В libstdc++-v3\textbackslash{}include\textbackslash{}bits\textbackslash{}basic\_string.h 
мы можем прочитать что это сделано для удобства отладки:

\begin{lstlisting}
   *  The reason you want _M_data pointing to the character %array and
   *  not the _Rep is so that the debugger can see the string
   *  contents. (Probably we should add a non-inline member to get
   *  the _Rep for the debugger to use, so users can check the actual
   *  string length.)
\end{lstlisting}

\href{http://go.yurichev.com/17085}{исходный код basic\_string.h}

В нашем примере мы учитываем это:

\lstinputlisting[caption=пример для GCC,style=customc]{\CURPATH/STL/string/GCC_RU.cpp}

Нужны еще небольшие хаки чтобы сымитировать типичную ошибку, которую мы уже видели выше, из-за
более ужесточенной проверки типов в GCC, тем не менее, printf() работает и здесь без c\_str().

\myparagraph{Чуть более сложный пример}

\lstinputlisting[style=customc]{\CURPATH/STL/string/3.cpp}

\lstinputlisting[caption=MSVC 2012,style=customasmx86]{\CURPATH/STL/string/3_MSVC_RU.asm}

Собственно, компилятор не конструирует строки статически: да в общем-то и как
это возможно, если буфер с ней нужно хранить в \glslink{heap}{куче}?

Вместо этого в сегменте данных хранятся обычные \ac{ASCIIZ}-строки, а позже, во время выполнения, 
при помощи метода \q{assign}, конструируются строки s1 и s2
.
При помощи \TT{operator+}, создается строка s3.

Обратите внимание на то что вызов метода c\_str() отсутствует,
потому что его код достаточно короткий и компилятор вставил его прямо здесь:
если строка короче 16-и байт, то в регистре EAX остается указатель на буфер,
а если длиннее, то из этого же места достается адрес на буфер расположенный в \glslink{heap}{куче}.

Далее следуют вызовы трех деструкторов, причем, они вызываются только если строка длиннее 16-и байт:
тогда нужно освободить буфера в \glslink{heap}{куче}.
В противном случае, так как все три объекта std::string хранятся в стеке,
они освобождаются автоматически после выхода из функции.

Следовательно, работа с короткими строками более быстрая из-за м\'{е}ньшего обращения к \glslink{heap}{куче}.

Код на GCC даже проще (из-за того, что в GCC, как мы уже видели, не реализована возможность хранить короткую
строку прямо в структуре):

% TODO1 comment each function meaning
\lstinputlisting[caption=GCC 4.8.1,style=customasmx86]{\CURPATH/STL/string/3_GCC_RU.s}

Можно заметить, что в деструкторы передается не указатель на объект,
а указатель на место за 12 байт (или 3 слова) перед ним, то есть, на настоящее начало структуры.

\myparagraph{std::string как глобальная переменная}
\label{sec:std_string_as_global_variable}

Опытные программисты на \Cpp знают, что глобальные переменные \ac{STL}-типов вполне можно объявлять.

Да, действительно:

\lstinputlisting[style=customc]{\CURPATH/STL/string/5.cpp}

Но как и где будет вызываться конструктор \TT{std::string}?

На самом деле, эта переменная будет инициализирована даже перед началом \main.

\lstinputlisting[caption=MSVC 2012: здесь конструируется глобальная переменная{,} а также регистрируется её деструктор,style=customasmx86]{\CURPATH/STL/string/5_MSVC_p2.asm}

\lstinputlisting[caption=MSVC 2012: здесь глобальная переменная используется в \main,style=customasmx86]{\CURPATH/STL/string/5_MSVC_p1.asm}

\lstinputlisting[caption=MSVC 2012: эта функция-деструктор вызывается перед выходом,style=customasmx86]{\CURPATH/STL/string/5_MSVC_p3.asm}

\myindex{\CStandardLibrary!atexit()}
В реальности, из \ac{CRT}, еще до вызова main(), вызывается специальная функция,
в которой перечислены все конструкторы подобных переменных.
Более того: при помощи atexit() регистрируется функция, которая будет вызвана в конце работы программы:
в этой функции компилятор собирает вызовы деструкторов всех подобных глобальных переменных.

GCC работает похожим образом:

\lstinputlisting[caption=GCC 4.8.1,style=customasmx86]{\CURPATH/STL/string/5_GCC.s}

Но он не выделяет отдельной функции в которой будут собраны деструкторы: 
каждый деструктор передается в atexit() по одному.

% TODO а если глобальная STL-переменная в другом модуле? надо проверить.

}
\DE{\subsection{Einfachste XOR-Verschlüsselung überhaupt}

Ich habe einmal eine Software gesehen, bei der alle Debugging-Ausgaben mit XOR mit dem Wert 3
verschlüsselt wurden. Mit anderen Worten, die beiden niedrigsten Bits aller Buchstaben wurden invertiert.

``Hello, world'' wurde zu ``Kfool/\#tlqog'':

\begin{lstlisting}
#!/usr/bin/python

msg="Hello, world!"

print "".join(map(lambda x: chr(ord(x)^3), msg))
\end{lstlisting}

Das ist eine ziemlich interessante Verschlüsselung (oder besser eine Verschleierung),
weil sie zwei wichtige Eigenschaften hat:
1) es ist eine einzige Funktion zum Verschlüsseln und entschlüsseln, sie muss nur wiederholt angewendet werden
2) die entstehenden Buchstaben befinden sich im druckbaren Bereich, also die ganze Zeichenkette kann ohne
Escape-Symbole im Code verwendet werden.

Die zweite Eigenschaft nutzt die Tatsache, dass alle druckbaren Zeichen in Reihen organisiert sind: 0x2x-0x7x,
und wenn die beiden niederwertigsten Bits invertiert werden, wird der Buchstabe um eine oder drei Stellen nach
links oder rechts \IT{verschoben}, aber niemals in eine andere Reihe:

\begin{figure}[H]
\centering
\includegraphics[width=0.7\textwidth]{ascii_clean.png}
\caption{7-Bit \ac{ASCII} Tabelle in Emacs}
\end{figure}

\dots mit dem Zeichen 0x7F als einziger Ausnahme.

Im Folgenden werden also beispielsweise die Zeichen A-Z \IT{verschlüsselt}:

\begin{lstlisting}
#!/usr/bin/python

msg="@ABCDEFGHIJKLMNO"

print "".join(map(lambda x: chr(ord(x)^3), msg))
\end{lstlisting}

Ergebnis:
% FIXME \verb  --  relevant comment for German?
\begin{lstlisting}
CBA@GFEDKJIHONML
\end{lstlisting}

Es sieht so aus als würden die Zeichen ``@'' und ``C'' sowie ``B'' und ``A'' vertauscht werden.

Hier ist noch ein interessantes Beispiel, in dem gezeigt wird, wie die Eigenschaften von XOR
ausgenutzt werden können: Exakt den gleichen Effekt, dass druckbare Zeichen auch druckbar bleiben,
kann man dadurch erzielen, dass irgendeine Kombination der niedrigsten vier Bits invertiert wird.
}

\ifdefined\SPANISH
\chapter{Patrones de código}
\fi % SPANISH

\ifdefined\GERMAN
\chapter{Code-Muster}
\fi % GERMAN

\ifdefined\ENGLISH
\chapter{Code Patterns}
\fi % ENGLISH

\ifdefined\ITALIAN
\chapter{Forme di codice}
\fi % ITALIAN

\ifdefined\RUSSIAN
\chapter{Образцы кода}
\fi % RUSSIAN

\ifdefined\BRAZILIAN
\chapter{Padrões de códigos}
\fi % BRAZILIAN

\ifdefined\THAI
\chapter{รูปแบบของโค้ด}
\fi % THAI

\ifdefined\FRENCH
\chapter{Modèle de code}
\fi % FRENCH

\ifdefined\POLISH
\chapter{\PLph{}}
\fi % POLISH

% sections
\EN{\input{patterns/patterns_opt_dbg_EN}}
\ES{\input{patterns/patterns_opt_dbg_ES}}
\ITA{\input{patterns/patterns_opt_dbg_ITA}}
\PTBR{\input{patterns/patterns_opt_dbg_PTBR}}
\RU{\input{patterns/patterns_opt_dbg_RU}}
\THA{\input{patterns/patterns_opt_dbg_THA}}
\DE{\input{patterns/patterns_opt_dbg_DE}}
\FR{\input{patterns/patterns_opt_dbg_FR}}
\PL{\input{patterns/patterns_opt_dbg_PL}}

\RU{\section{Некоторые базовые понятия}}
\EN{\section{Some basics}}
\DE{\section{Einige Grundlagen}}
\FR{\section{Quelques bases}}
\ES{\section{\ESph{}}}
\ITA{\section{Alcune basi teoriche}}
\PTBR{\section{\PTBRph{}}}
\THA{\section{\THAph{}}}
\PL{\section{\PLph{}}}

% sections:
\EN{\input{patterns/intro_CPU_ISA_EN}}
\ES{\input{patterns/intro_CPU_ISA_ES}}
\ITA{\input{patterns/intro_CPU_ISA_ITA}}
\PTBR{\input{patterns/intro_CPU_ISA_PTBR}}
\RU{\input{patterns/intro_CPU_ISA_RU}}
\DE{\input{patterns/intro_CPU_ISA_DE}}
\FR{\input{patterns/intro_CPU_ISA_FR}}
\PL{\input{patterns/intro_CPU_ISA_PL}}

\EN{\input{patterns/numeral_EN}}
\RU{\input{patterns/numeral_RU}}
\ITA{\input{patterns/numeral_ITA}}
\DE{\input{patterns/numeral_DE}}
\FR{\input{patterns/numeral_FR}}
\PL{\input{patterns/numeral_PL}}

% chapters
\input{patterns/00_empty/main}
\input{patterns/011_ret/main}
\input{patterns/01_helloworld/main}
\input{patterns/015_prolog_epilogue/main}
\input{patterns/02_stack/main}
\input{patterns/03_printf/main}
\input{patterns/04_scanf/main}
\input{patterns/05_passing_arguments/main}
\input{patterns/06_return_results/main}
\input{patterns/061_pointers/main}
\input{patterns/065_GOTO/main}
\input{patterns/07_jcc/main}
\input{patterns/08_switch/main}
\input{patterns/09_loops/main}
\input{patterns/10_strings/main}
\input{patterns/11_arith_optimizations/main}
\input{patterns/12_FPU/main}
\input{patterns/13_arrays/main}
\input{patterns/14_bitfields/main}
\EN{\input{patterns/145_LCG/main_EN}}
\RU{\input{patterns/145_LCG/main_RU}}
\input{patterns/15_structs/main}
\input{patterns/17_unions/main}
\input{patterns/18_pointers_to_functions/main}
\input{patterns/185_64bit_in_32_env/main}

\EN{\input{patterns/19_SIMD/main_EN}}
\RU{\input{patterns/19_SIMD/main_RU}}
\DE{\input{patterns/19_SIMD/main_DE}}

\EN{\input{patterns/20_x64/main_EN}}
\RU{\input{patterns/20_x64/main_RU}}

\EN{\input{patterns/205_floating_SIMD/main_EN}}
\RU{\input{patterns/205_floating_SIMD/main_RU}}
\DE{\input{patterns/205_floating_SIMD/main_DE}}

\EN{\input{patterns/ARM/main_EN}}
\RU{\input{patterns/ARM/main_RU}}
\DE{\input{patterns/ARM/main_DE}}

\input{patterns/MIPS/main}


\ifdefined\SPANISH
\chapter{Patrones de código}
\fi % SPANISH

\ifdefined\GERMAN
\chapter{Code-Muster}
\fi % GERMAN

\ifdefined\ENGLISH
\chapter{Code Patterns}
\fi % ENGLISH

\ifdefined\ITALIAN
\chapter{Forme di codice}
\fi % ITALIAN

\ifdefined\RUSSIAN
\chapter{Образцы кода}
\fi % RUSSIAN

\ifdefined\BRAZILIAN
\chapter{Padrões de códigos}
\fi % BRAZILIAN

\ifdefined\THAI
\chapter{รูปแบบของโค้ด}
\fi % THAI

\ifdefined\FRENCH
\chapter{Modèle de code}
\fi % FRENCH

\ifdefined\POLISH
\chapter{\PLph{}}
\fi % POLISH

% sections
\EN{\section{The method}

When the author of this book first started learning C and, later, \Cpp, he used to write small pieces of code, compile them,
and then look at the assembly language output. This made it very easy for him to understand what was going on in the code that he had written.
\footnote{In fact, he still does this when he can't understand what a particular bit of code does.}.
He did this so many times that the relationship between the \CCpp code and what the compiler produced was imprinted deeply in his mind.
It's now easy for him to imagine instantly a rough outline of a C code's appearance and function.
Perhaps this technique could be helpful for others.

%There are a lot of examples for both x86/x64 and ARM.
%Those who already familiar with one of architectures, may freely skim over pages.

By the way, there is a great website where you can do the same, with various compilers, instead of installing them on your box.
You can use it as well: \url{https://gcc.godbolt.org/}.

\section*{\Exercises}

When the author of this book studied assembly language, he also often compiled small C functions and then rewrote
them gradually to assembly, trying to make their code as short as possible.
This probably is not worth doing in real-world scenarios today,
because it's hard to compete with the latest compilers in terms of efficiency. It is, however, a very good way to gain a better understanding of assembly.
Feel free, therefore, to take any assembly code from this book and try to make it shorter.
However, don't forget to test what you have written.

% rewrote to show that debug\release and optimisations levels are orthogonal concepts.
\section*{Optimization levels and debug information}

Source code can be compiled by different compilers with various optimization levels.
A typical compiler has about three such levels, where level zero means that optimization is completely disabled.
Optimization can also be targeted towards code size or code speed.
A non-optimizing compiler is faster and produces more understandable (albeit verbose) code,
whereas an optimizing compiler is slower and tries to produce code that runs faster (but is not necessarily more compact).
In addition to optimization levels, a compiler can include some debug information in the resulting file,
producing code that is easy to debug.
One of the important features of the ´debug' code is that it might contain links
between each line of the source code and its respective machine code address.
Optimizing compilers, on the other hand, tend to produce output where entire lines of source code
can be optimized away and thus not even be present in the resulting machine code.
Reverse engineers can encounter either version, simply because some developers turn on the compiler's optimization flags and others do not.
Because of this, we'll try to work on examples of both debug and release versions of the code featured in this book, wherever possible.

Sometimes some pretty ancient compilers are used in this book, in order to get the shortest (or simplest) possible code snippet.
}
\ES{\input{patterns/patterns_opt_dbg_ES}}
\ITA{\input{patterns/patterns_opt_dbg_ITA}}
\PTBR{\input{patterns/patterns_opt_dbg_PTBR}}
\RU{\input{patterns/patterns_opt_dbg_RU}}
\THA{\input{patterns/patterns_opt_dbg_THA}}
\DE{\input{patterns/patterns_opt_dbg_DE}}
\FR{\input{patterns/patterns_opt_dbg_FR}}
\PL{\input{patterns/patterns_opt_dbg_PL}}

\RU{\section{Некоторые базовые понятия}}
\EN{\section{Some basics}}
\DE{\section{Einige Grundlagen}}
\FR{\section{Quelques bases}}
\ES{\section{\ESph{}}}
\ITA{\section{Alcune basi teoriche}}
\PTBR{\section{\PTBRph{}}}
\THA{\section{\THAph{}}}
\PL{\section{\PLph{}}}

% sections:
\EN{\input{patterns/intro_CPU_ISA_EN}}
\ES{\input{patterns/intro_CPU_ISA_ES}}
\ITA{\input{patterns/intro_CPU_ISA_ITA}}
\PTBR{\input{patterns/intro_CPU_ISA_PTBR}}
\RU{\input{patterns/intro_CPU_ISA_RU}}
\DE{\input{patterns/intro_CPU_ISA_DE}}
\FR{\input{patterns/intro_CPU_ISA_FR}}
\PL{\input{patterns/intro_CPU_ISA_PL}}

\EN{\subsection{Numeral Systems}

Humans have become accustomed to a decimal numeral system, probably because almost everyone has 10 fingers.
Nevertheless, the number \q{10} has no significant meaning in science and mathematics.
The natural numeral system in digital electronics is binary: 0 is for an absence of current in the wire, and 1 for presence.
10 in binary is 2 in decimal, 100 in binary is 4 in decimal, and so on.

% This sentence is a bit unweildy - maybe try 'Our ten-digit system would be described as having a radix...' - Renaissance
If the numeral system has 10 digits, it has a \IT{radix} (or \IT{base}) of 10.
The binary numeral system has a \IT{radix} of 2.

Important things to recall:

1) A \IT{number} is a number, while a \IT{digit} is a term from writing systems, and is usually one character

% The original is 'number' is not changed; I think the intent is value, and changed it - Renaissance
2) The value of a number does not change when converted to another radix; only the writing notation for that value has changed (and therefore the way of representing it in \ac{RAM}).

\subsection{Converting From One Radix To Another}

Positional notation is used almost every numerical system. This means that a digit has weight relative to where it is placed inside of the larger number.
If 2 is placed at the rightmost place, it's 2, but if it's placed one digit before rightmost, it's 20.

What does $1234$ stand for?

$10^3 \cdot 1 + 10^2 \cdot 2 + 10^1 \cdot 3 + 1 \cdot 4 = 1234$ or
$1000 \cdot 1 + 100 \cdot 2 + 10 \cdot 3 + 4 = 1234$

It's the same story for binary numbers, but the base is 2 instead of 10.
What does 0b101011 stand for?

$2^5 \cdot 1 + 2^4 \cdot 0 + 2^3 \cdot 1 + 2^2 \cdot 0 + 2^1 \cdot 1 + 2^0 \cdot 1 = 43$ or
$32 \cdot 1 + 16 \cdot 0 + 8 \cdot 1 + 4 \cdot 0 + 2 \cdot 1 + 1 = 43$

There is such a thing as non-positional notation, such as the Roman numeral system.
\footnote{About numeric system evolution, see \InSqBrackets{\TAOCPvolII{}, 195--213.}}.
% Maybe add a sentence to fill in that X is always 10, and is therefore non-positional, even though putting an I before subtracts and after adds, and is in that sense positional
Perhaps, humankind switched to positional notation because it's easier to do basic operations (addition, multiplication, etc.) on paper by hand.

Binary numbers can be added, subtracted and so on in the very same as taught in schools, but only 2 digits are available.

Binary numbers are bulky when represented in source code and dumps, so that is where the hexadecimal numeral system can be useful.
A hexadecimal radix uses the digits 0..9, and also 6 Latin characters: A..F.
Each hexadecimal digit takes 4 bits or 4 binary digits, so it's very easy to convert from binary number to hexadecimal and back, even manually, in one's mind.

\begin{center}
\begin{longtable}{ | l | l | l | }
\hline
\HeaderColor hexadecimal & \HeaderColor binary & \HeaderColor decimal \\
\hline
0	&0000	&0 \\
1	&0001	&1 \\
2	&0010	&2 \\
3	&0011	&3 \\
4	&0100	&4 \\
5	&0101	&5 \\
6	&0110	&6 \\
7	&0111	&7 \\
8	&1000	&8 \\
9	&1001	&9 \\
A	&1010	&10 \\
B	&1011	&11 \\
C	&1100	&12 \\
D	&1101	&13 \\
E	&1110	&14 \\
F	&1111	&15 \\
\hline
\end{longtable}
\end{center}

How can one tell which radix is being used in a specific instance?

Decimal numbers are usually written as is, i.e., 1234. Some assemblers allow an identifier on decimal radix numbers, in which the number would be written with a "d" suffix: 1234d.

Binary numbers are sometimes prepended with the "0b" prefix: 0b100110111 (\ac{GCC} has a non-standard language extension for this\footnote{\url{https://gcc.gnu.org/onlinedocs/gcc/Binary-constants.html}}).
There is also another way: using a "b" suffix, for example: 100110111b.
This book tries to use the "0b" prefix consistently throughout the book for binary numbers.

Hexadecimal numbers are prepended with "0x" prefix in \CCpp and other \ac{PL}s: 0x1234ABCD.
Alternatively, they are given a "h" suffix: 1234ABCDh. This is common way of representing them in assemblers and debuggers.
In this convention, if the number is started with a Latin (A..F) digit, a 0 is added at the beginning: 0ABCDEFh.
There was also convention that was popular in 8-bit home computers era, using \$ prefix, like \$ABCD.
The book will try to stick to "0x" prefix throughout the book for hexadecimal numbers.

Should one learn to convert numbers mentally? A table of 1-digit hexadecimal numbers can easily be memorized.
As for larger numbers, it's probably not worth tormenting yourself.

Perhaps the most visible hexadecimal numbers are in \ac{URL}s.
This is the way that non-Latin characters are encoded.
For example:
\url{https://en.wiktionary.org/wiki/na\%C3\%AFvet\%C3\%A9} is the \ac{URL} of Wiktionary article about \q{naïveté} word.

\subsubsection{Octal Radix}

Another numeral system heavily used in the past of computer programming is octal. In octal there are 8 digits (0..7), and each is mapped to 3 bits, so it's easy to convert numbers back and forth.
It has been superseded by the hexadecimal system almost everywhere, but, surprisingly, there is a *NIX utility, used often by many people, which takes octal numbers as argument: \TT{chmod}.

\myindex{UNIX!chmod}
As many *NIX users know, \TT{chmod} argument can be a number of 3 digits. The first digit represents the rights of the owner of the file (read, write and/or execute), the second is the rights for the group to which the file belongs, and the third is for everyone else.
Each digit that \TT{chmod} takes can be represented in binary form:

\begin{center}
\begin{longtable}{ | l | l | l | }
\hline
\HeaderColor decimal & \HeaderColor binary & \HeaderColor meaning \\
\hline
7	&111	&\textbf{rwx} \\
6	&110	&\textbf{rw-} \\
5	&101	&\textbf{r-x} \\
4	&100	&\textbf{r-{}-} \\
3	&011	&\textbf{-wx} \\
2	&010	&\textbf{-w-} \\
1	&001	&\textbf{-{}-x} \\
0	&000	&\textbf{-{}-{}-} \\
\hline
\end{longtable}
\end{center}

So each bit is mapped to a flag: read/write/execute.

The importance of \TT{chmod} here is that the whole number in argument can be represented as octal number.
Let's take, for example, 644.
When you run \TT{chmod 644 file}, you set read/write permissions for owner, read permissions for group and again, read permissions for everyone else.
If we convert the octal number 644 to binary, it would be \TT{110100100}, or, in groups of 3 bits, \TT{110 100 100}.

Now we see that each triplet describe permissions for owner/group/others: first is \TT{rw-}, second is \TT{r--} and third is \TT{r--}.

The octal numeral system was also popular on old computers like PDP-8, because word there could be 12, 24 or 36 bits, and these numbers are all divisible by 3, so the octal system was natural in that environment.
Nowadays, all popular computers employ word/address sizes of 16, 32 or 64 bits, and these numbers are all divisible by 4, so the hexadecimal system is more natural there.

The octal numeral system is supported by all standard \CCpp compilers.
This is a source of confusion sometimes, because octal numbers are encoded with a zero prepended, for example, 0377 is 255.
Sometimes, you might make a typo and write "09" instead of 9, and the compiler would report an error.
GCC might report something like this:\\
\TT{error: invalid digit "9" in octal constant}.

Also, the octal system is somewhat popular in Java. When the IDA shows Java strings with non-printable characters,
they are encoded in the octal system instead of hexadecimal.
\myindex{JAD}
The JAD Java decompiler behaves the same way.

\subsubsection{Divisibility}

When you see a decimal number like 120, you can quickly deduce that it's divisible by 10, because the last digit is zero.
In the same way, 123400 is divisible by 100, because the two last digits are zeros.

Likewise, the hexadecimal number 0x1230 is divisible by 0x10 (or 16), 0x123000 is divisible by 0x1000 (or 4096), etc.

The binary number 0b1000101000 is divisible by 0b1000 (8), etc.

This property can often be used to quickly realize if the size of some block in memory is padded to some boundary.
For example, sections in \ac{PE} files are almost always started at addresses ending with 3 hexadecimal zeros: 0x41000, 0x10001000, etc.
The reason behind this is the fact that almost all \ac{PE} sections are padded to a boundary of 0x1000 (4096) bytes.

\subsubsection{Multi-Precision Arithmetic and Radix}

\index{RSA}
Multi-precision arithmetic can use huge numbers, and each one may be stored in several bytes.
For example, RSA keys, both public and private, span up to 4096 bits, and maybe even more.

% I'm not sure how to change this, but the normal format for quoting would be just to mention the author or book, and footnote to the full reference
In \InSqBrackets{\TAOCPvolII, 265} we find the following idea: when you store a multi-precision number in several bytes,
the whole number can be represented as having a radix of $2^8=256$, and each digit goes to the corresponding byte.
Likewise, if you store a multi-precision number in several 32-bit integer values, each digit goes to each 32-bit slot,
and you may think about this number as stored in radix of $2^{32}$.

\subsubsection{How to Pronounce Non-Decimal Numbers}

Numbers in a non-decimal base are usually pronounced by digit by digit: ``one-zero-zero-one-one-...''.
Words like ``ten'' and ``thousand'' are usually not pronounced, to prevent confusion with the decimal base system.

\subsubsection{Floating point numbers}

To distinguish floating point numbers from integers, they are usually written with ``.0'' at the end,
like $0.0$, $123.0$, etc.
}
\RU{\subsection{Представление чисел}

Люди привыкли к десятичной системе счисления вероятно потому что почти у каждого есть по 10 пальцев.
Тем не менее, число 10 не имеет особого значения в науке и математике.
Двоичная система естествена для цифровой электроники: 0 означает отсутствие тока в проводе и 1 --- его присутствие.
10 в двоичной системе это 2 в десятичной; 100 в двоичной это 4 в десятичной, итд.

Если в системе счисления есть 10 цифр, её \IT{основание} или \IT{radix} это 10.
Двоичная система имеет \IT{основание} 2.

Важные вещи, которые полезно вспомнить:
1) \IT{число} это число, в то время как \IT{цифра} это термин из системы письменности, и это обычно один символ;
2) само число не меняется, когда конвертируется из одного основания в другое: меняется способ его записи (или представления
в памяти).

Как сконвертировать число из одного основания в другое?

Позиционная нотация используется почти везде, это означает, что всякая цифра имеет свой вес, в зависимости от её расположения
внутри числа.
Если 2 расположена в самом последнем месте справа, это 2.
Если она расположена в месте перед последним, это 20.

Что означает $1234$?

$10^3 \cdot 1 + 10^2 \cdot 2 + 10^1 \cdot 3 + 1 \cdot 4$ = 1234 или
$1000 \cdot 1 + 100 \cdot 2 + 10 \cdot 3 + 4 = 1234$

Та же история и для двоичных чисел, только основание там 2 вместо 10.
Что означает 0b101011?

$2^5 \cdot 1 + 2^4 \cdot 0 + 2^3 \cdot 1 + 2^2 \cdot 0 + 2^1 \cdot 1 + 2^0 \cdot 1 = 43$ или
$32 \cdot 1 + 16 \cdot 0 + 8 \cdot 1 + 4 \cdot 0 + 2 \cdot 1 + 1 = 43$

Позиционную нотацию можно противопоставить непозиционной нотации, такой как римская система записи чисел
\footnote{Об эволюции способов записи чисел, см.также: \InSqBrackets{\TAOCPvolII{}, 195--213.}}.
Вероятно, человечество перешло на позиционную нотацию, потому что так проще работать с числами (сложение, умножение, итд)
на бумаге, в ручную.

Действительно, двоичные числа можно складывать, вычитать, итд, точно также, как этому обычно обучают в школах,
только доступны лишь 2 цифры.

Двоичные числа громоздки, когда их используют в исходных кодах и дампах, так что в этих случаях применяется шестнадцатеричная
система.
Используются цифры 0..9 и еще 6 латинских букв: A..F.
Каждая шестнадцатеричная цифра занимает 4 бита или 4 двоичных цифры, так что конвертировать из двоичной системы в
шестнадцатеричную и назад, можно легко вручную, или даже в уме.

\begin{center}
\begin{longtable}{ | l | l | l | }
\hline
\HeaderColor шестнадцатеричная & \HeaderColor двоичная & \HeaderColor десятичная \\
\hline
0	&0000	&0 \\
1	&0001	&1 \\
2	&0010	&2 \\
3	&0011	&3 \\
4	&0100	&4 \\
5	&0101	&5 \\
6	&0110	&6 \\
7	&0111	&7 \\
8	&1000	&8 \\
9	&1001	&9 \\
A	&1010	&10 \\
B	&1011	&11 \\
C	&1100	&12 \\
D	&1101	&13 \\
E	&1110	&14 \\
F	&1111	&15 \\
\hline
\end{longtable}
\end{center}

Как понять, какое основание используется в конкретном месте?

Десятичные числа обычно записываются как есть, т.е., 1234. Но некоторые ассемблеры позволяют подчеркивать
этот факт для ясности, и это число может быть дополнено суффиксом "d": 1234d.

К двоичным числам иногда спереди добавляют префикс "0b": 0b100110111
(В \ac{GCC} для этого есть нестандартное расширение языка
\footnote{\url{https://gcc.gnu.org/onlinedocs/gcc/Binary-constants.html}}).
Есть также еще один способ: суффикс "b", например: 100110111b.
В этой книге я буду пытаться придерживаться префикса "0b" для двоичных чисел.

Шестнадцатеричные числа имеют префикс "0x" в \CCpp и некоторых других \ac{PL}: 0x1234ABCD.
Либо они имеют суффикс "h": 1234ABCDh --- обычно так они представляются в ассемблерах и отладчиках.
Если число начинается с цифры A..F, перед ним добавляется 0: 0ABCDEFh.
Во времена 8-битных домашних компьютеров, был также способ записи чисел используя префикс \$, например, \$ABCD.
В книге я попытаюсь придерживаться префикса "0x" для шестнадцатеричных чисел.

Нужно ли учиться конвертировать числа в уме? Таблицу шестнадцатеричных чисел из одной цифры легко запомнить.
А запоминать б\'{о}льшие числа, наверное, не стоит.

Наверное, чаще всего шестнадцатеричные числа можно увидеть в \ac{URL}-ах.
Так кодируются буквы не из числа латинских.
Например:
\url{https://en.wiktionary.org/wiki/na\%C3\%AFvet\%C3\%A9} это \ac{URL} страницы в Wiktionary о слове \q{naïveté}.

\subsubsection{Восьмеричная система}

Еще одна система, которая в прошлом много использовалась в программировании это восьмеричная: есть 8 цифр (0..7) и каждая
описывает 3 бита, так что легко конвертировать числа туда и назад.
Она почти везде была заменена шестнадцатеричной, но удивительно, в *NIX имеется утилита использующаяся многими людьми,
которая принимает на вход восьмеричное число: \TT{chmod}.

\myindex{UNIX!chmod}
Как знают многие пользователи *NIX, аргумент \TT{chmod} это число из трех цифр. Первая цифра это права владельца файла,
вторая это права группы (которой файл принадлежит), третья для всех остальных.
И каждая цифра может быть представлена в двоичном виде:

\begin{center}
\begin{longtable}{ | l | l | l | }
\hline
\HeaderColor десятичная & \HeaderColor двоичная & \HeaderColor значение \\
\hline
7	&111	&\textbf{rwx} \\
6	&110	&\textbf{rw-} \\
5	&101	&\textbf{r-x} \\
4	&100	&\textbf{r-{}-} \\
3	&011	&\textbf{-wx} \\
2	&010	&\textbf{-w-} \\
1	&001	&\textbf{-{}-x} \\
0	&000	&\textbf{-{}-{}-} \\
\hline
\end{longtable}
\end{center}

Так что каждый бит привязан к флагу: read/write/execute (чтение/запись/исполнение).

И вот почему я вспомнил здесь о \TT{chmod}, это потому что всё число может быть представлено как число в восьмеричной системе.
Для примера возьмем 644.
Когда вы запускаете \TT{chmod 644 file}, вы выставляете права read/write для владельца, права read для группы, и снова,
read для всех остальных.
Сконвертируем число 644 из восьмеричной системы в двоичную, это будет \TT{110100100}, или (в группах по 3 бита) \TT{110 100 100}.

Теперь мы видим, что каждая тройка описывает права для владельца/группы/остальных:
первая это \TT{rw-}, вторая это \TT{r--} и третья это \TT{r--}.

Восьмеричная система была также популярная на старых компьютерах вроде PDP-8, потому что слово там могло содержать 12, 24 или
36 бит, и эти числа делятся на 3, так что выбор восьмеричной системы в той среде был логичен.
Сейчас, все популярные компьютеры имеют размер слова/адреса 16, 32 или 64 бита, и эти числа делятся на 4,
так что шестнадцатеричная система здесь удобнее.

Восьмеричная система поддерживается всеми стандартными компиляторами \CCpp{}.
Это иногда источник недоумения, потому что восьмеричные числа кодируются с нулем вперед, например, 0377 это 255.
И иногда, вы можете сделать опечатку, и написать "09" вместо 9, и компилятор выдаст ошибку.
GCC может выдать что-то вроде:\\
\TT{error: invalid digit "9" in octal constant}.

Также, восьмеричная система популярна в Java: когда IDA показывает строку с непечатаемыми символами,
они кодируются в восьмеричной системе вместо шестнадцатеричной.
\myindex{JAD}
Точно также себя ведет декомпилятор с Java JAD.

\subsubsection{Делимость}

Когда вы видите десятичное число вроде 120, вы можете быстро понять что оно делится на 10, потому что последняя цифра это 0.
Точно также, 123400 делится на 100, потому что две последних цифры это нули.

Точно также, шестнадцатеричное число 0x1230 делится на 0x10 (или 16), 0x123000 делится на 0x1000 (или 4096), итд.

Двоичное число 0b1000101000 делится на 0b1000 (8), итд.

Это свойство можно часто использовать, чтобы быстро понять,
что длина какого-либо блока в памяти выровнена по некоторой границе.
Например, секции в \ac{PE}-файлах почти всегда начинаются с адресов заканчивающихся 3 шестнадцатеричными нулями:
0x41000, 0x10001000, итд.
Причина в том, что почти все секции в \ac{PE} выровнены по границе 0x1000 (4096) байт.

\subsubsection{Арифметика произвольной точности и основание}

\index{RSA}
Арифметика произвольной точности (multi-precision arithmetic) может использовать огромные числа,
которые могут храниться в нескольких байтах.
Например, ключи RSA, и открытые и закрытые, могут занимать до 4096 бит и даже больше.

В \InSqBrackets{\TAOCPvolII, 265} можно найти такую идею: когда вы сохраняете число произвольной точности в нескольких байтах,
всё число может быть представлено как имеющую систему счисления по основанию $2^8=256$, и каждая цифра находится
в соответствующем байте.
Точно также, если вы сохраняете число произвольной точности в нескольких 32-битных целочисленных значениях,
каждая цифра отправляется в каждый 32-битный слот, и вы можете считать что это число записано в системе с основанием $2^{32}$.

\subsubsection{Произношение}

Числа в недесятичных системах счислениях обычно произносятся по одной цифре: ``один-ноль-ноль-один-один-...''.
Слова вроде ``десять'', ``тысяча'', итд, обычно не произносятся, потому что тогда можно спутать с десятичной системой.

\subsubsection{Числа с плавающей запятой}

Чтобы отличать числа с плавающей запятой от целочисленных, часто, в конце добавляют ``.0'',
например $0.0$, $123.0$, итд.

}
\ITA{\input{patterns/numeral_ITA}}
\DE{\input{patterns/numeral_DE}}
\FR{\input{patterns/numeral_FR}}
\PL{\input{patterns/numeral_PL}}

% chapters
\ifdefined\SPANISH
\chapter{Patrones de código}
\fi % SPANISH

\ifdefined\GERMAN
\chapter{Code-Muster}
\fi % GERMAN

\ifdefined\ENGLISH
\chapter{Code Patterns}
\fi % ENGLISH

\ifdefined\ITALIAN
\chapter{Forme di codice}
\fi % ITALIAN

\ifdefined\RUSSIAN
\chapter{Образцы кода}
\fi % RUSSIAN

\ifdefined\BRAZILIAN
\chapter{Padrões de códigos}
\fi % BRAZILIAN

\ifdefined\THAI
\chapter{รูปแบบของโค้ด}
\fi % THAI

\ifdefined\FRENCH
\chapter{Modèle de code}
\fi % FRENCH

\ifdefined\POLISH
\chapter{\PLph{}}
\fi % POLISH

% sections
\EN{\input{patterns/patterns_opt_dbg_EN}}
\ES{\input{patterns/patterns_opt_dbg_ES}}
\ITA{\input{patterns/patterns_opt_dbg_ITA}}
\PTBR{\input{patterns/patterns_opt_dbg_PTBR}}
\RU{\input{patterns/patterns_opt_dbg_RU}}
\THA{\input{patterns/patterns_opt_dbg_THA}}
\DE{\input{patterns/patterns_opt_dbg_DE}}
\FR{\input{patterns/patterns_opt_dbg_FR}}
\PL{\input{patterns/patterns_opt_dbg_PL}}

\RU{\section{Некоторые базовые понятия}}
\EN{\section{Some basics}}
\DE{\section{Einige Grundlagen}}
\FR{\section{Quelques bases}}
\ES{\section{\ESph{}}}
\ITA{\section{Alcune basi teoriche}}
\PTBR{\section{\PTBRph{}}}
\THA{\section{\THAph{}}}
\PL{\section{\PLph{}}}

% sections:
\EN{\input{patterns/intro_CPU_ISA_EN}}
\ES{\input{patterns/intro_CPU_ISA_ES}}
\ITA{\input{patterns/intro_CPU_ISA_ITA}}
\PTBR{\input{patterns/intro_CPU_ISA_PTBR}}
\RU{\input{patterns/intro_CPU_ISA_RU}}
\DE{\input{patterns/intro_CPU_ISA_DE}}
\FR{\input{patterns/intro_CPU_ISA_FR}}
\PL{\input{patterns/intro_CPU_ISA_PL}}

\EN{\input{patterns/numeral_EN}}
\RU{\input{patterns/numeral_RU}}
\ITA{\input{patterns/numeral_ITA}}
\DE{\input{patterns/numeral_DE}}
\FR{\input{patterns/numeral_FR}}
\PL{\input{patterns/numeral_PL}}

% chapters
\input{patterns/00_empty/main}
\input{patterns/011_ret/main}
\input{patterns/01_helloworld/main}
\input{patterns/015_prolog_epilogue/main}
\input{patterns/02_stack/main}
\input{patterns/03_printf/main}
\input{patterns/04_scanf/main}
\input{patterns/05_passing_arguments/main}
\input{patterns/06_return_results/main}
\input{patterns/061_pointers/main}
\input{patterns/065_GOTO/main}
\input{patterns/07_jcc/main}
\input{patterns/08_switch/main}
\input{patterns/09_loops/main}
\input{patterns/10_strings/main}
\input{patterns/11_arith_optimizations/main}
\input{patterns/12_FPU/main}
\input{patterns/13_arrays/main}
\input{patterns/14_bitfields/main}
\EN{\input{patterns/145_LCG/main_EN}}
\RU{\input{patterns/145_LCG/main_RU}}
\input{patterns/15_structs/main}
\input{patterns/17_unions/main}
\input{patterns/18_pointers_to_functions/main}
\input{patterns/185_64bit_in_32_env/main}

\EN{\input{patterns/19_SIMD/main_EN}}
\RU{\input{patterns/19_SIMD/main_RU}}
\DE{\input{patterns/19_SIMD/main_DE}}

\EN{\input{patterns/20_x64/main_EN}}
\RU{\input{patterns/20_x64/main_RU}}

\EN{\input{patterns/205_floating_SIMD/main_EN}}
\RU{\input{patterns/205_floating_SIMD/main_RU}}
\DE{\input{patterns/205_floating_SIMD/main_DE}}

\EN{\input{patterns/ARM/main_EN}}
\RU{\input{patterns/ARM/main_RU}}
\DE{\input{patterns/ARM/main_DE}}

\input{patterns/MIPS/main}

\ifdefined\SPANISH
\chapter{Patrones de código}
\fi % SPANISH

\ifdefined\GERMAN
\chapter{Code-Muster}
\fi % GERMAN

\ifdefined\ENGLISH
\chapter{Code Patterns}
\fi % ENGLISH

\ifdefined\ITALIAN
\chapter{Forme di codice}
\fi % ITALIAN

\ifdefined\RUSSIAN
\chapter{Образцы кода}
\fi % RUSSIAN

\ifdefined\BRAZILIAN
\chapter{Padrões de códigos}
\fi % BRAZILIAN

\ifdefined\THAI
\chapter{รูปแบบของโค้ด}
\fi % THAI

\ifdefined\FRENCH
\chapter{Modèle de code}
\fi % FRENCH

\ifdefined\POLISH
\chapter{\PLph{}}
\fi % POLISH

% sections
\EN{\input{patterns/patterns_opt_dbg_EN}}
\ES{\input{patterns/patterns_opt_dbg_ES}}
\ITA{\input{patterns/patterns_opt_dbg_ITA}}
\PTBR{\input{patterns/patterns_opt_dbg_PTBR}}
\RU{\input{patterns/patterns_opt_dbg_RU}}
\THA{\input{patterns/patterns_opt_dbg_THA}}
\DE{\input{patterns/patterns_opt_dbg_DE}}
\FR{\input{patterns/patterns_opt_dbg_FR}}
\PL{\input{patterns/patterns_opt_dbg_PL}}

\RU{\section{Некоторые базовые понятия}}
\EN{\section{Some basics}}
\DE{\section{Einige Grundlagen}}
\FR{\section{Quelques bases}}
\ES{\section{\ESph{}}}
\ITA{\section{Alcune basi teoriche}}
\PTBR{\section{\PTBRph{}}}
\THA{\section{\THAph{}}}
\PL{\section{\PLph{}}}

% sections:
\EN{\input{patterns/intro_CPU_ISA_EN}}
\ES{\input{patterns/intro_CPU_ISA_ES}}
\ITA{\input{patterns/intro_CPU_ISA_ITA}}
\PTBR{\input{patterns/intro_CPU_ISA_PTBR}}
\RU{\input{patterns/intro_CPU_ISA_RU}}
\DE{\input{patterns/intro_CPU_ISA_DE}}
\FR{\input{patterns/intro_CPU_ISA_FR}}
\PL{\input{patterns/intro_CPU_ISA_PL}}

\EN{\input{patterns/numeral_EN}}
\RU{\input{patterns/numeral_RU}}
\ITA{\input{patterns/numeral_ITA}}
\DE{\input{patterns/numeral_DE}}
\FR{\input{patterns/numeral_FR}}
\PL{\input{patterns/numeral_PL}}

% chapters
\input{patterns/00_empty/main}
\input{patterns/011_ret/main}
\input{patterns/01_helloworld/main}
\input{patterns/015_prolog_epilogue/main}
\input{patterns/02_stack/main}
\input{patterns/03_printf/main}
\input{patterns/04_scanf/main}
\input{patterns/05_passing_arguments/main}
\input{patterns/06_return_results/main}
\input{patterns/061_pointers/main}
\input{patterns/065_GOTO/main}
\input{patterns/07_jcc/main}
\input{patterns/08_switch/main}
\input{patterns/09_loops/main}
\input{patterns/10_strings/main}
\input{patterns/11_arith_optimizations/main}
\input{patterns/12_FPU/main}
\input{patterns/13_arrays/main}
\input{patterns/14_bitfields/main}
\EN{\input{patterns/145_LCG/main_EN}}
\RU{\input{patterns/145_LCG/main_RU}}
\input{patterns/15_structs/main}
\input{patterns/17_unions/main}
\input{patterns/18_pointers_to_functions/main}
\input{patterns/185_64bit_in_32_env/main}

\EN{\input{patterns/19_SIMD/main_EN}}
\RU{\input{patterns/19_SIMD/main_RU}}
\DE{\input{patterns/19_SIMD/main_DE}}

\EN{\input{patterns/20_x64/main_EN}}
\RU{\input{patterns/20_x64/main_RU}}

\EN{\input{patterns/205_floating_SIMD/main_EN}}
\RU{\input{patterns/205_floating_SIMD/main_RU}}
\DE{\input{patterns/205_floating_SIMD/main_DE}}

\EN{\input{patterns/ARM/main_EN}}
\RU{\input{patterns/ARM/main_RU}}
\DE{\input{patterns/ARM/main_DE}}

\input{patterns/MIPS/main}

\ifdefined\SPANISH
\chapter{Patrones de código}
\fi % SPANISH

\ifdefined\GERMAN
\chapter{Code-Muster}
\fi % GERMAN

\ifdefined\ENGLISH
\chapter{Code Patterns}
\fi % ENGLISH

\ifdefined\ITALIAN
\chapter{Forme di codice}
\fi % ITALIAN

\ifdefined\RUSSIAN
\chapter{Образцы кода}
\fi % RUSSIAN

\ifdefined\BRAZILIAN
\chapter{Padrões de códigos}
\fi % BRAZILIAN

\ifdefined\THAI
\chapter{รูปแบบของโค้ด}
\fi % THAI

\ifdefined\FRENCH
\chapter{Modèle de code}
\fi % FRENCH

\ifdefined\POLISH
\chapter{\PLph{}}
\fi % POLISH

% sections
\EN{\input{patterns/patterns_opt_dbg_EN}}
\ES{\input{patterns/patterns_opt_dbg_ES}}
\ITA{\input{patterns/patterns_opt_dbg_ITA}}
\PTBR{\input{patterns/patterns_opt_dbg_PTBR}}
\RU{\input{patterns/patterns_opt_dbg_RU}}
\THA{\input{patterns/patterns_opt_dbg_THA}}
\DE{\input{patterns/patterns_opt_dbg_DE}}
\FR{\input{patterns/patterns_opt_dbg_FR}}
\PL{\input{patterns/patterns_opt_dbg_PL}}

\RU{\section{Некоторые базовые понятия}}
\EN{\section{Some basics}}
\DE{\section{Einige Grundlagen}}
\FR{\section{Quelques bases}}
\ES{\section{\ESph{}}}
\ITA{\section{Alcune basi teoriche}}
\PTBR{\section{\PTBRph{}}}
\THA{\section{\THAph{}}}
\PL{\section{\PLph{}}}

% sections:
\EN{\input{patterns/intro_CPU_ISA_EN}}
\ES{\input{patterns/intro_CPU_ISA_ES}}
\ITA{\input{patterns/intro_CPU_ISA_ITA}}
\PTBR{\input{patterns/intro_CPU_ISA_PTBR}}
\RU{\input{patterns/intro_CPU_ISA_RU}}
\DE{\input{patterns/intro_CPU_ISA_DE}}
\FR{\input{patterns/intro_CPU_ISA_FR}}
\PL{\input{patterns/intro_CPU_ISA_PL}}

\EN{\input{patterns/numeral_EN}}
\RU{\input{patterns/numeral_RU}}
\ITA{\input{patterns/numeral_ITA}}
\DE{\input{patterns/numeral_DE}}
\FR{\input{patterns/numeral_FR}}
\PL{\input{patterns/numeral_PL}}

% chapters
\input{patterns/00_empty/main}
\input{patterns/011_ret/main}
\input{patterns/01_helloworld/main}
\input{patterns/015_prolog_epilogue/main}
\input{patterns/02_stack/main}
\input{patterns/03_printf/main}
\input{patterns/04_scanf/main}
\input{patterns/05_passing_arguments/main}
\input{patterns/06_return_results/main}
\input{patterns/061_pointers/main}
\input{patterns/065_GOTO/main}
\input{patterns/07_jcc/main}
\input{patterns/08_switch/main}
\input{patterns/09_loops/main}
\input{patterns/10_strings/main}
\input{patterns/11_arith_optimizations/main}
\input{patterns/12_FPU/main}
\input{patterns/13_arrays/main}
\input{patterns/14_bitfields/main}
\EN{\input{patterns/145_LCG/main_EN}}
\RU{\input{patterns/145_LCG/main_RU}}
\input{patterns/15_structs/main}
\input{patterns/17_unions/main}
\input{patterns/18_pointers_to_functions/main}
\input{patterns/185_64bit_in_32_env/main}

\EN{\input{patterns/19_SIMD/main_EN}}
\RU{\input{patterns/19_SIMD/main_RU}}
\DE{\input{patterns/19_SIMD/main_DE}}

\EN{\input{patterns/20_x64/main_EN}}
\RU{\input{patterns/20_x64/main_RU}}

\EN{\input{patterns/205_floating_SIMD/main_EN}}
\RU{\input{patterns/205_floating_SIMD/main_RU}}
\DE{\input{patterns/205_floating_SIMD/main_DE}}

\EN{\input{patterns/ARM/main_EN}}
\RU{\input{patterns/ARM/main_RU}}
\DE{\input{patterns/ARM/main_DE}}

\input{patterns/MIPS/main}

\ifdefined\SPANISH
\chapter{Patrones de código}
\fi % SPANISH

\ifdefined\GERMAN
\chapter{Code-Muster}
\fi % GERMAN

\ifdefined\ENGLISH
\chapter{Code Patterns}
\fi % ENGLISH

\ifdefined\ITALIAN
\chapter{Forme di codice}
\fi % ITALIAN

\ifdefined\RUSSIAN
\chapter{Образцы кода}
\fi % RUSSIAN

\ifdefined\BRAZILIAN
\chapter{Padrões de códigos}
\fi % BRAZILIAN

\ifdefined\THAI
\chapter{รูปแบบของโค้ด}
\fi % THAI

\ifdefined\FRENCH
\chapter{Modèle de code}
\fi % FRENCH

\ifdefined\POLISH
\chapter{\PLph{}}
\fi % POLISH

% sections
\EN{\input{patterns/patterns_opt_dbg_EN}}
\ES{\input{patterns/patterns_opt_dbg_ES}}
\ITA{\input{patterns/patterns_opt_dbg_ITA}}
\PTBR{\input{patterns/patterns_opt_dbg_PTBR}}
\RU{\input{patterns/patterns_opt_dbg_RU}}
\THA{\input{patterns/patterns_opt_dbg_THA}}
\DE{\input{patterns/patterns_opt_dbg_DE}}
\FR{\input{patterns/patterns_opt_dbg_FR}}
\PL{\input{patterns/patterns_opt_dbg_PL}}

\RU{\section{Некоторые базовые понятия}}
\EN{\section{Some basics}}
\DE{\section{Einige Grundlagen}}
\FR{\section{Quelques bases}}
\ES{\section{\ESph{}}}
\ITA{\section{Alcune basi teoriche}}
\PTBR{\section{\PTBRph{}}}
\THA{\section{\THAph{}}}
\PL{\section{\PLph{}}}

% sections:
\EN{\input{patterns/intro_CPU_ISA_EN}}
\ES{\input{patterns/intro_CPU_ISA_ES}}
\ITA{\input{patterns/intro_CPU_ISA_ITA}}
\PTBR{\input{patterns/intro_CPU_ISA_PTBR}}
\RU{\input{patterns/intro_CPU_ISA_RU}}
\DE{\input{patterns/intro_CPU_ISA_DE}}
\FR{\input{patterns/intro_CPU_ISA_FR}}
\PL{\input{patterns/intro_CPU_ISA_PL}}

\EN{\input{patterns/numeral_EN}}
\RU{\input{patterns/numeral_RU}}
\ITA{\input{patterns/numeral_ITA}}
\DE{\input{patterns/numeral_DE}}
\FR{\input{patterns/numeral_FR}}
\PL{\input{patterns/numeral_PL}}

% chapters
\input{patterns/00_empty/main}
\input{patterns/011_ret/main}
\input{patterns/01_helloworld/main}
\input{patterns/015_prolog_epilogue/main}
\input{patterns/02_stack/main}
\input{patterns/03_printf/main}
\input{patterns/04_scanf/main}
\input{patterns/05_passing_arguments/main}
\input{patterns/06_return_results/main}
\input{patterns/061_pointers/main}
\input{patterns/065_GOTO/main}
\input{patterns/07_jcc/main}
\input{patterns/08_switch/main}
\input{patterns/09_loops/main}
\input{patterns/10_strings/main}
\input{patterns/11_arith_optimizations/main}
\input{patterns/12_FPU/main}
\input{patterns/13_arrays/main}
\input{patterns/14_bitfields/main}
\EN{\input{patterns/145_LCG/main_EN}}
\RU{\input{patterns/145_LCG/main_RU}}
\input{patterns/15_structs/main}
\input{patterns/17_unions/main}
\input{patterns/18_pointers_to_functions/main}
\input{patterns/185_64bit_in_32_env/main}

\EN{\input{patterns/19_SIMD/main_EN}}
\RU{\input{patterns/19_SIMD/main_RU}}
\DE{\input{patterns/19_SIMD/main_DE}}

\EN{\input{patterns/20_x64/main_EN}}
\RU{\input{patterns/20_x64/main_RU}}

\EN{\input{patterns/205_floating_SIMD/main_EN}}
\RU{\input{patterns/205_floating_SIMD/main_RU}}
\DE{\input{patterns/205_floating_SIMD/main_DE}}

\EN{\input{patterns/ARM/main_EN}}
\RU{\input{patterns/ARM/main_RU}}
\DE{\input{patterns/ARM/main_DE}}

\input{patterns/MIPS/main}

\ifdefined\SPANISH
\chapter{Patrones de código}
\fi % SPANISH

\ifdefined\GERMAN
\chapter{Code-Muster}
\fi % GERMAN

\ifdefined\ENGLISH
\chapter{Code Patterns}
\fi % ENGLISH

\ifdefined\ITALIAN
\chapter{Forme di codice}
\fi % ITALIAN

\ifdefined\RUSSIAN
\chapter{Образцы кода}
\fi % RUSSIAN

\ifdefined\BRAZILIAN
\chapter{Padrões de códigos}
\fi % BRAZILIAN

\ifdefined\THAI
\chapter{รูปแบบของโค้ด}
\fi % THAI

\ifdefined\FRENCH
\chapter{Modèle de code}
\fi % FRENCH

\ifdefined\POLISH
\chapter{\PLph{}}
\fi % POLISH

% sections
\EN{\input{patterns/patterns_opt_dbg_EN}}
\ES{\input{patterns/patterns_opt_dbg_ES}}
\ITA{\input{patterns/patterns_opt_dbg_ITA}}
\PTBR{\input{patterns/patterns_opt_dbg_PTBR}}
\RU{\input{patterns/patterns_opt_dbg_RU}}
\THA{\input{patterns/patterns_opt_dbg_THA}}
\DE{\input{patterns/patterns_opt_dbg_DE}}
\FR{\input{patterns/patterns_opt_dbg_FR}}
\PL{\input{patterns/patterns_opt_dbg_PL}}

\RU{\section{Некоторые базовые понятия}}
\EN{\section{Some basics}}
\DE{\section{Einige Grundlagen}}
\FR{\section{Quelques bases}}
\ES{\section{\ESph{}}}
\ITA{\section{Alcune basi teoriche}}
\PTBR{\section{\PTBRph{}}}
\THA{\section{\THAph{}}}
\PL{\section{\PLph{}}}

% sections:
\EN{\input{patterns/intro_CPU_ISA_EN}}
\ES{\input{patterns/intro_CPU_ISA_ES}}
\ITA{\input{patterns/intro_CPU_ISA_ITA}}
\PTBR{\input{patterns/intro_CPU_ISA_PTBR}}
\RU{\input{patterns/intro_CPU_ISA_RU}}
\DE{\input{patterns/intro_CPU_ISA_DE}}
\FR{\input{patterns/intro_CPU_ISA_FR}}
\PL{\input{patterns/intro_CPU_ISA_PL}}

\EN{\input{patterns/numeral_EN}}
\RU{\input{patterns/numeral_RU}}
\ITA{\input{patterns/numeral_ITA}}
\DE{\input{patterns/numeral_DE}}
\FR{\input{patterns/numeral_FR}}
\PL{\input{patterns/numeral_PL}}

% chapters
\input{patterns/00_empty/main}
\input{patterns/011_ret/main}
\input{patterns/01_helloworld/main}
\input{patterns/015_prolog_epilogue/main}
\input{patterns/02_stack/main}
\input{patterns/03_printf/main}
\input{patterns/04_scanf/main}
\input{patterns/05_passing_arguments/main}
\input{patterns/06_return_results/main}
\input{patterns/061_pointers/main}
\input{patterns/065_GOTO/main}
\input{patterns/07_jcc/main}
\input{patterns/08_switch/main}
\input{patterns/09_loops/main}
\input{patterns/10_strings/main}
\input{patterns/11_arith_optimizations/main}
\input{patterns/12_FPU/main}
\input{patterns/13_arrays/main}
\input{patterns/14_bitfields/main}
\EN{\input{patterns/145_LCG/main_EN}}
\RU{\input{patterns/145_LCG/main_RU}}
\input{patterns/15_structs/main}
\input{patterns/17_unions/main}
\input{patterns/18_pointers_to_functions/main}
\input{patterns/185_64bit_in_32_env/main}

\EN{\input{patterns/19_SIMD/main_EN}}
\RU{\input{patterns/19_SIMD/main_RU}}
\DE{\input{patterns/19_SIMD/main_DE}}

\EN{\input{patterns/20_x64/main_EN}}
\RU{\input{patterns/20_x64/main_RU}}

\EN{\input{patterns/205_floating_SIMD/main_EN}}
\RU{\input{patterns/205_floating_SIMD/main_RU}}
\DE{\input{patterns/205_floating_SIMD/main_DE}}

\EN{\input{patterns/ARM/main_EN}}
\RU{\input{patterns/ARM/main_RU}}
\DE{\input{patterns/ARM/main_DE}}

\input{patterns/MIPS/main}

\ifdefined\SPANISH
\chapter{Patrones de código}
\fi % SPANISH

\ifdefined\GERMAN
\chapter{Code-Muster}
\fi % GERMAN

\ifdefined\ENGLISH
\chapter{Code Patterns}
\fi % ENGLISH

\ifdefined\ITALIAN
\chapter{Forme di codice}
\fi % ITALIAN

\ifdefined\RUSSIAN
\chapter{Образцы кода}
\fi % RUSSIAN

\ifdefined\BRAZILIAN
\chapter{Padrões de códigos}
\fi % BRAZILIAN

\ifdefined\THAI
\chapter{รูปแบบของโค้ด}
\fi % THAI

\ifdefined\FRENCH
\chapter{Modèle de code}
\fi % FRENCH

\ifdefined\POLISH
\chapter{\PLph{}}
\fi % POLISH

% sections
\EN{\input{patterns/patterns_opt_dbg_EN}}
\ES{\input{patterns/patterns_opt_dbg_ES}}
\ITA{\input{patterns/patterns_opt_dbg_ITA}}
\PTBR{\input{patterns/patterns_opt_dbg_PTBR}}
\RU{\input{patterns/patterns_opt_dbg_RU}}
\THA{\input{patterns/patterns_opt_dbg_THA}}
\DE{\input{patterns/patterns_opt_dbg_DE}}
\FR{\input{patterns/patterns_opt_dbg_FR}}
\PL{\input{patterns/patterns_opt_dbg_PL}}

\RU{\section{Некоторые базовые понятия}}
\EN{\section{Some basics}}
\DE{\section{Einige Grundlagen}}
\FR{\section{Quelques bases}}
\ES{\section{\ESph{}}}
\ITA{\section{Alcune basi teoriche}}
\PTBR{\section{\PTBRph{}}}
\THA{\section{\THAph{}}}
\PL{\section{\PLph{}}}

% sections:
\EN{\input{patterns/intro_CPU_ISA_EN}}
\ES{\input{patterns/intro_CPU_ISA_ES}}
\ITA{\input{patterns/intro_CPU_ISA_ITA}}
\PTBR{\input{patterns/intro_CPU_ISA_PTBR}}
\RU{\input{patterns/intro_CPU_ISA_RU}}
\DE{\input{patterns/intro_CPU_ISA_DE}}
\FR{\input{patterns/intro_CPU_ISA_FR}}
\PL{\input{patterns/intro_CPU_ISA_PL}}

\EN{\input{patterns/numeral_EN}}
\RU{\input{patterns/numeral_RU}}
\ITA{\input{patterns/numeral_ITA}}
\DE{\input{patterns/numeral_DE}}
\FR{\input{patterns/numeral_FR}}
\PL{\input{patterns/numeral_PL}}

% chapters
\input{patterns/00_empty/main}
\input{patterns/011_ret/main}
\input{patterns/01_helloworld/main}
\input{patterns/015_prolog_epilogue/main}
\input{patterns/02_stack/main}
\input{patterns/03_printf/main}
\input{patterns/04_scanf/main}
\input{patterns/05_passing_arguments/main}
\input{patterns/06_return_results/main}
\input{patterns/061_pointers/main}
\input{patterns/065_GOTO/main}
\input{patterns/07_jcc/main}
\input{patterns/08_switch/main}
\input{patterns/09_loops/main}
\input{patterns/10_strings/main}
\input{patterns/11_arith_optimizations/main}
\input{patterns/12_FPU/main}
\input{patterns/13_arrays/main}
\input{patterns/14_bitfields/main}
\EN{\input{patterns/145_LCG/main_EN}}
\RU{\input{patterns/145_LCG/main_RU}}
\input{patterns/15_structs/main}
\input{patterns/17_unions/main}
\input{patterns/18_pointers_to_functions/main}
\input{patterns/185_64bit_in_32_env/main}

\EN{\input{patterns/19_SIMD/main_EN}}
\RU{\input{patterns/19_SIMD/main_RU}}
\DE{\input{patterns/19_SIMD/main_DE}}

\EN{\input{patterns/20_x64/main_EN}}
\RU{\input{patterns/20_x64/main_RU}}

\EN{\input{patterns/205_floating_SIMD/main_EN}}
\RU{\input{patterns/205_floating_SIMD/main_RU}}
\DE{\input{patterns/205_floating_SIMD/main_DE}}

\EN{\input{patterns/ARM/main_EN}}
\RU{\input{patterns/ARM/main_RU}}
\DE{\input{patterns/ARM/main_DE}}

\input{patterns/MIPS/main}

\ifdefined\SPANISH
\chapter{Patrones de código}
\fi % SPANISH

\ifdefined\GERMAN
\chapter{Code-Muster}
\fi % GERMAN

\ifdefined\ENGLISH
\chapter{Code Patterns}
\fi % ENGLISH

\ifdefined\ITALIAN
\chapter{Forme di codice}
\fi % ITALIAN

\ifdefined\RUSSIAN
\chapter{Образцы кода}
\fi % RUSSIAN

\ifdefined\BRAZILIAN
\chapter{Padrões de códigos}
\fi % BRAZILIAN

\ifdefined\THAI
\chapter{รูปแบบของโค้ด}
\fi % THAI

\ifdefined\FRENCH
\chapter{Modèle de code}
\fi % FRENCH

\ifdefined\POLISH
\chapter{\PLph{}}
\fi % POLISH

% sections
\EN{\input{patterns/patterns_opt_dbg_EN}}
\ES{\input{patterns/patterns_opt_dbg_ES}}
\ITA{\input{patterns/patterns_opt_dbg_ITA}}
\PTBR{\input{patterns/patterns_opt_dbg_PTBR}}
\RU{\input{patterns/patterns_opt_dbg_RU}}
\THA{\input{patterns/patterns_opt_dbg_THA}}
\DE{\input{patterns/patterns_opt_dbg_DE}}
\FR{\input{patterns/patterns_opt_dbg_FR}}
\PL{\input{patterns/patterns_opt_dbg_PL}}

\RU{\section{Некоторые базовые понятия}}
\EN{\section{Some basics}}
\DE{\section{Einige Grundlagen}}
\FR{\section{Quelques bases}}
\ES{\section{\ESph{}}}
\ITA{\section{Alcune basi teoriche}}
\PTBR{\section{\PTBRph{}}}
\THA{\section{\THAph{}}}
\PL{\section{\PLph{}}}

% sections:
\EN{\input{patterns/intro_CPU_ISA_EN}}
\ES{\input{patterns/intro_CPU_ISA_ES}}
\ITA{\input{patterns/intro_CPU_ISA_ITA}}
\PTBR{\input{patterns/intro_CPU_ISA_PTBR}}
\RU{\input{patterns/intro_CPU_ISA_RU}}
\DE{\input{patterns/intro_CPU_ISA_DE}}
\FR{\input{patterns/intro_CPU_ISA_FR}}
\PL{\input{patterns/intro_CPU_ISA_PL}}

\EN{\input{patterns/numeral_EN}}
\RU{\input{patterns/numeral_RU}}
\ITA{\input{patterns/numeral_ITA}}
\DE{\input{patterns/numeral_DE}}
\FR{\input{patterns/numeral_FR}}
\PL{\input{patterns/numeral_PL}}

% chapters
\input{patterns/00_empty/main}
\input{patterns/011_ret/main}
\input{patterns/01_helloworld/main}
\input{patterns/015_prolog_epilogue/main}
\input{patterns/02_stack/main}
\input{patterns/03_printf/main}
\input{patterns/04_scanf/main}
\input{patterns/05_passing_arguments/main}
\input{patterns/06_return_results/main}
\input{patterns/061_pointers/main}
\input{patterns/065_GOTO/main}
\input{patterns/07_jcc/main}
\input{patterns/08_switch/main}
\input{patterns/09_loops/main}
\input{patterns/10_strings/main}
\input{patterns/11_arith_optimizations/main}
\input{patterns/12_FPU/main}
\input{patterns/13_arrays/main}
\input{patterns/14_bitfields/main}
\EN{\input{patterns/145_LCG/main_EN}}
\RU{\input{patterns/145_LCG/main_RU}}
\input{patterns/15_structs/main}
\input{patterns/17_unions/main}
\input{patterns/18_pointers_to_functions/main}
\input{patterns/185_64bit_in_32_env/main}

\EN{\input{patterns/19_SIMD/main_EN}}
\RU{\input{patterns/19_SIMD/main_RU}}
\DE{\input{patterns/19_SIMD/main_DE}}

\EN{\input{patterns/20_x64/main_EN}}
\RU{\input{patterns/20_x64/main_RU}}

\EN{\input{patterns/205_floating_SIMD/main_EN}}
\RU{\input{patterns/205_floating_SIMD/main_RU}}
\DE{\input{patterns/205_floating_SIMD/main_DE}}

\EN{\input{patterns/ARM/main_EN}}
\RU{\input{patterns/ARM/main_RU}}
\DE{\input{patterns/ARM/main_DE}}

\input{patterns/MIPS/main}

\ifdefined\SPANISH
\chapter{Patrones de código}
\fi % SPANISH

\ifdefined\GERMAN
\chapter{Code-Muster}
\fi % GERMAN

\ifdefined\ENGLISH
\chapter{Code Patterns}
\fi % ENGLISH

\ifdefined\ITALIAN
\chapter{Forme di codice}
\fi % ITALIAN

\ifdefined\RUSSIAN
\chapter{Образцы кода}
\fi % RUSSIAN

\ifdefined\BRAZILIAN
\chapter{Padrões de códigos}
\fi % BRAZILIAN

\ifdefined\THAI
\chapter{รูปแบบของโค้ด}
\fi % THAI

\ifdefined\FRENCH
\chapter{Modèle de code}
\fi % FRENCH

\ifdefined\POLISH
\chapter{\PLph{}}
\fi % POLISH

% sections
\EN{\input{patterns/patterns_opt_dbg_EN}}
\ES{\input{patterns/patterns_opt_dbg_ES}}
\ITA{\input{patterns/patterns_opt_dbg_ITA}}
\PTBR{\input{patterns/patterns_opt_dbg_PTBR}}
\RU{\input{patterns/patterns_opt_dbg_RU}}
\THA{\input{patterns/patterns_opt_dbg_THA}}
\DE{\input{patterns/patterns_opt_dbg_DE}}
\FR{\input{patterns/patterns_opt_dbg_FR}}
\PL{\input{patterns/patterns_opt_dbg_PL}}

\RU{\section{Некоторые базовые понятия}}
\EN{\section{Some basics}}
\DE{\section{Einige Grundlagen}}
\FR{\section{Quelques bases}}
\ES{\section{\ESph{}}}
\ITA{\section{Alcune basi teoriche}}
\PTBR{\section{\PTBRph{}}}
\THA{\section{\THAph{}}}
\PL{\section{\PLph{}}}

% sections:
\EN{\input{patterns/intro_CPU_ISA_EN}}
\ES{\input{patterns/intro_CPU_ISA_ES}}
\ITA{\input{patterns/intro_CPU_ISA_ITA}}
\PTBR{\input{patterns/intro_CPU_ISA_PTBR}}
\RU{\input{patterns/intro_CPU_ISA_RU}}
\DE{\input{patterns/intro_CPU_ISA_DE}}
\FR{\input{patterns/intro_CPU_ISA_FR}}
\PL{\input{patterns/intro_CPU_ISA_PL}}

\EN{\input{patterns/numeral_EN}}
\RU{\input{patterns/numeral_RU}}
\ITA{\input{patterns/numeral_ITA}}
\DE{\input{patterns/numeral_DE}}
\FR{\input{patterns/numeral_FR}}
\PL{\input{patterns/numeral_PL}}

% chapters
\input{patterns/00_empty/main}
\input{patterns/011_ret/main}
\input{patterns/01_helloworld/main}
\input{patterns/015_prolog_epilogue/main}
\input{patterns/02_stack/main}
\input{patterns/03_printf/main}
\input{patterns/04_scanf/main}
\input{patterns/05_passing_arguments/main}
\input{patterns/06_return_results/main}
\input{patterns/061_pointers/main}
\input{patterns/065_GOTO/main}
\input{patterns/07_jcc/main}
\input{patterns/08_switch/main}
\input{patterns/09_loops/main}
\input{patterns/10_strings/main}
\input{patterns/11_arith_optimizations/main}
\input{patterns/12_FPU/main}
\input{patterns/13_arrays/main}
\input{patterns/14_bitfields/main}
\EN{\input{patterns/145_LCG/main_EN}}
\RU{\input{patterns/145_LCG/main_RU}}
\input{patterns/15_structs/main}
\input{patterns/17_unions/main}
\input{patterns/18_pointers_to_functions/main}
\input{patterns/185_64bit_in_32_env/main}

\EN{\input{patterns/19_SIMD/main_EN}}
\RU{\input{patterns/19_SIMD/main_RU}}
\DE{\input{patterns/19_SIMD/main_DE}}

\EN{\input{patterns/20_x64/main_EN}}
\RU{\input{patterns/20_x64/main_RU}}

\EN{\input{patterns/205_floating_SIMD/main_EN}}
\RU{\input{patterns/205_floating_SIMD/main_RU}}
\DE{\input{patterns/205_floating_SIMD/main_DE}}

\EN{\input{patterns/ARM/main_EN}}
\RU{\input{patterns/ARM/main_RU}}
\DE{\input{patterns/ARM/main_DE}}

\input{patterns/MIPS/main}

\ifdefined\SPANISH
\chapter{Patrones de código}
\fi % SPANISH

\ifdefined\GERMAN
\chapter{Code-Muster}
\fi % GERMAN

\ifdefined\ENGLISH
\chapter{Code Patterns}
\fi % ENGLISH

\ifdefined\ITALIAN
\chapter{Forme di codice}
\fi % ITALIAN

\ifdefined\RUSSIAN
\chapter{Образцы кода}
\fi % RUSSIAN

\ifdefined\BRAZILIAN
\chapter{Padrões de códigos}
\fi % BRAZILIAN

\ifdefined\THAI
\chapter{รูปแบบของโค้ด}
\fi % THAI

\ifdefined\FRENCH
\chapter{Modèle de code}
\fi % FRENCH

\ifdefined\POLISH
\chapter{\PLph{}}
\fi % POLISH

% sections
\EN{\input{patterns/patterns_opt_dbg_EN}}
\ES{\input{patterns/patterns_opt_dbg_ES}}
\ITA{\input{patterns/patterns_opt_dbg_ITA}}
\PTBR{\input{patterns/patterns_opt_dbg_PTBR}}
\RU{\input{patterns/patterns_opt_dbg_RU}}
\THA{\input{patterns/patterns_opt_dbg_THA}}
\DE{\input{patterns/patterns_opt_dbg_DE}}
\FR{\input{patterns/patterns_opt_dbg_FR}}
\PL{\input{patterns/patterns_opt_dbg_PL}}

\RU{\section{Некоторые базовые понятия}}
\EN{\section{Some basics}}
\DE{\section{Einige Grundlagen}}
\FR{\section{Quelques bases}}
\ES{\section{\ESph{}}}
\ITA{\section{Alcune basi teoriche}}
\PTBR{\section{\PTBRph{}}}
\THA{\section{\THAph{}}}
\PL{\section{\PLph{}}}

% sections:
\EN{\input{patterns/intro_CPU_ISA_EN}}
\ES{\input{patterns/intro_CPU_ISA_ES}}
\ITA{\input{patterns/intro_CPU_ISA_ITA}}
\PTBR{\input{patterns/intro_CPU_ISA_PTBR}}
\RU{\input{patterns/intro_CPU_ISA_RU}}
\DE{\input{patterns/intro_CPU_ISA_DE}}
\FR{\input{patterns/intro_CPU_ISA_FR}}
\PL{\input{patterns/intro_CPU_ISA_PL}}

\EN{\input{patterns/numeral_EN}}
\RU{\input{patterns/numeral_RU}}
\ITA{\input{patterns/numeral_ITA}}
\DE{\input{patterns/numeral_DE}}
\FR{\input{patterns/numeral_FR}}
\PL{\input{patterns/numeral_PL}}

% chapters
\input{patterns/00_empty/main}
\input{patterns/011_ret/main}
\input{patterns/01_helloworld/main}
\input{patterns/015_prolog_epilogue/main}
\input{patterns/02_stack/main}
\input{patterns/03_printf/main}
\input{patterns/04_scanf/main}
\input{patterns/05_passing_arguments/main}
\input{patterns/06_return_results/main}
\input{patterns/061_pointers/main}
\input{patterns/065_GOTO/main}
\input{patterns/07_jcc/main}
\input{patterns/08_switch/main}
\input{patterns/09_loops/main}
\input{patterns/10_strings/main}
\input{patterns/11_arith_optimizations/main}
\input{patterns/12_FPU/main}
\input{patterns/13_arrays/main}
\input{patterns/14_bitfields/main}
\EN{\input{patterns/145_LCG/main_EN}}
\RU{\input{patterns/145_LCG/main_RU}}
\input{patterns/15_structs/main}
\input{patterns/17_unions/main}
\input{patterns/18_pointers_to_functions/main}
\input{patterns/185_64bit_in_32_env/main}

\EN{\input{patterns/19_SIMD/main_EN}}
\RU{\input{patterns/19_SIMD/main_RU}}
\DE{\input{patterns/19_SIMD/main_DE}}

\EN{\input{patterns/20_x64/main_EN}}
\RU{\input{patterns/20_x64/main_RU}}

\EN{\input{patterns/205_floating_SIMD/main_EN}}
\RU{\input{patterns/205_floating_SIMD/main_RU}}
\DE{\input{patterns/205_floating_SIMD/main_DE}}

\EN{\input{patterns/ARM/main_EN}}
\RU{\input{patterns/ARM/main_RU}}
\DE{\input{patterns/ARM/main_DE}}

\input{patterns/MIPS/main}

\ifdefined\SPANISH
\chapter{Patrones de código}
\fi % SPANISH

\ifdefined\GERMAN
\chapter{Code-Muster}
\fi % GERMAN

\ifdefined\ENGLISH
\chapter{Code Patterns}
\fi % ENGLISH

\ifdefined\ITALIAN
\chapter{Forme di codice}
\fi % ITALIAN

\ifdefined\RUSSIAN
\chapter{Образцы кода}
\fi % RUSSIAN

\ifdefined\BRAZILIAN
\chapter{Padrões de códigos}
\fi % BRAZILIAN

\ifdefined\THAI
\chapter{รูปแบบของโค้ด}
\fi % THAI

\ifdefined\FRENCH
\chapter{Modèle de code}
\fi % FRENCH

\ifdefined\POLISH
\chapter{\PLph{}}
\fi % POLISH

% sections
\EN{\input{patterns/patterns_opt_dbg_EN}}
\ES{\input{patterns/patterns_opt_dbg_ES}}
\ITA{\input{patterns/patterns_opt_dbg_ITA}}
\PTBR{\input{patterns/patterns_opt_dbg_PTBR}}
\RU{\input{patterns/patterns_opt_dbg_RU}}
\THA{\input{patterns/patterns_opt_dbg_THA}}
\DE{\input{patterns/patterns_opt_dbg_DE}}
\FR{\input{patterns/patterns_opt_dbg_FR}}
\PL{\input{patterns/patterns_opt_dbg_PL}}

\RU{\section{Некоторые базовые понятия}}
\EN{\section{Some basics}}
\DE{\section{Einige Grundlagen}}
\FR{\section{Quelques bases}}
\ES{\section{\ESph{}}}
\ITA{\section{Alcune basi teoriche}}
\PTBR{\section{\PTBRph{}}}
\THA{\section{\THAph{}}}
\PL{\section{\PLph{}}}

% sections:
\EN{\input{patterns/intro_CPU_ISA_EN}}
\ES{\input{patterns/intro_CPU_ISA_ES}}
\ITA{\input{patterns/intro_CPU_ISA_ITA}}
\PTBR{\input{patterns/intro_CPU_ISA_PTBR}}
\RU{\input{patterns/intro_CPU_ISA_RU}}
\DE{\input{patterns/intro_CPU_ISA_DE}}
\FR{\input{patterns/intro_CPU_ISA_FR}}
\PL{\input{patterns/intro_CPU_ISA_PL}}

\EN{\input{patterns/numeral_EN}}
\RU{\input{patterns/numeral_RU}}
\ITA{\input{patterns/numeral_ITA}}
\DE{\input{patterns/numeral_DE}}
\FR{\input{patterns/numeral_FR}}
\PL{\input{patterns/numeral_PL}}

% chapters
\input{patterns/00_empty/main}
\input{patterns/011_ret/main}
\input{patterns/01_helloworld/main}
\input{patterns/015_prolog_epilogue/main}
\input{patterns/02_stack/main}
\input{patterns/03_printf/main}
\input{patterns/04_scanf/main}
\input{patterns/05_passing_arguments/main}
\input{patterns/06_return_results/main}
\input{patterns/061_pointers/main}
\input{patterns/065_GOTO/main}
\input{patterns/07_jcc/main}
\input{patterns/08_switch/main}
\input{patterns/09_loops/main}
\input{patterns/10_strings/main}
\input{patterns/11_arith_optimizations/main}
\input{patterns/12_FPU/main}
\input{patterns/13_arrays/main}
\input{patterns/14_bitfields/main}
\EN{\input{patterns/145_LCG/main_EN}}
\RU{\input{patterns/145_LCG/main_RU}}
\input{patterns/15_structs/main}
\input{patterns/17_unions/main}
\input{patterns/18_pointers_to_functions/main}
\input{patterns/185_64bit_in_32_env/main}

\EN{\input{patterns/19_SIMD/main_EN}}
\RU{\input{patterns/19_SIMD/main_RU}}
\DE{\input{patterns/19_SIMD/main_DE}}

\EN{\input{patterns/20_x64/main_EN}}
\RU{\input{patterns/20_x64/main_RU}}

\EN{\input{patterns/205_floating_SIMD/main_EN}}
\RU{\input{patterns/205_floating_SIMD/main_RU}}
\DE{\input{patterns/205_floating_SIMD/main_DE}}

\EN{\input{patterns/ARM/main_EN}}
\RU{\input{patterns/ARM/main_RU}}
\DE{\input{patterns/ARM/main_DE}}

\input{patterns/MIPS/main}

\ifdefined\SPANISH
\chapter{Patrones de código}
\fi % SPANISH

\ifdefined\GERMAN
\chapter{Code-Muster}
\fi % GERMAN

\ifdefined\ENGLISH
\chapter{Code Patterns}
\fi % ENGLISH

\ifdefined\ITALIAN
\chapter{Forme di codice}
\fi % ITALIAN

\ifdefined\RUSSIAN
\chapter{Образцы кода}
\fi % RUSSIAN

\ifdefined\BRAZILIAN
\chapter{Padrões de códigos}
\fi % BRAZILIAN

\ifdefined\THAI
\chapter{รูปแบบของโค้ด}
\fi % THAI

\ifdefined\FRENCH
\chapter{Modèle de code}
\fi % FRENCH

\ifdefined\POLISH
\chapter{\PLph{}}
\fi % POLISH

% sections
\EN{\input{patterns/patterns_opt_dbg_EN}}
\ES{\input{patterns/patterns_opt_dbg_ES}}
\ITA{\input{patterns/patterns_opt_dbg_ITA}}
\PTBR{\input{patterns/patterns_opt_dbg_PTBR}}
\RU{\input{patterns/patterns_opt_dbg_RU}}
\THA{\input{patterns/patterns_opt_dbg_THA}}
\DE{\input{patterns/patterns_opt_dbg_DE}}
\FR{\input{patterns/patterns_opt_dbg_FR}}
\PL{\input{patterns/patterns_opt_dbg_PL}}

\RU{\section{Некоторые базовые понятия}}
\EN{\section{Some basics}}
\DE{\section{Einige Grundlagen}}
\FR{\section{Quelques bases}}
\ES{\section{\ESph{}}}
\ITA{\section{Alcune basi teoriche}}
\PTBR{\section{\PTBRph{}}}
\THA{\section{\THAph{}}}
\PL{\section{\PLph{}}}

% sections:
\EN{\input{patterns/intro_CPU_ISA_EN}}
\ES{\input{patterns/intro_CPU_ISA_ES}}
\ITA{\input{patterns/intro_CPU_ISA_ITA}}
\PTBR{\input{patterns/intro_CPU_ISA_PTBR}}
\RU{\input{patterns/intro_CPU_ISA_RU}}
\DE{\input{patterns/intro_CPU_ISA_DE}}
\FR{\input{patterns/intro_CPU_ISA_FR}}
\PL{\input{patterns/intro_CPU_ISA_PL}}

\EN{\input{patterns/numeral_EN}}
\RU{\input{patterns/numeral_RU}}
\ITA{\input{patterns/numeral_ITA}}
\DE{\input{patterns/numeral_DE}}
\FR{\input{patterns/numeral_FR}}
\PL{\input{patterns/numeral_PL}}

% chapters
\input{patterns/00_empty/main}
\input{patterns/011_ret/main}
\input{patterns/01_helloworld/main}
\input{patterns/015_prolog_epilogue/main}
\input{patterns/02_stack/main}
\input{patterns/03_printf/main}
\input{patterns/04_scanf/main}
\input{patterns/05_passing_arguments/main}
\input{patterns/06_return_results/main}
\input{patterns/061_pointers/main}
\input{patterns/065_GOTO/main}
\input{patterns/07_jcc/main}
\input{patterns/08_switch/main}
\input{patterns/09_loops/main}
\input{patterns/10_strings/main}
\input{patterns/11_arith_optimizations/main}
\input{patterns/12_FPU/main}
\input{patterns/13_arrays/main}
\input{patterns/14_bitfields/main}
\EN{\input{patterns/145_LCG/main_EN}}
\RU{\input{patterns/145_LCG/main_RU}}
\input{patterns/15_structs/main}
\input{patterns/17_unions/main}
\input{patterns/18_pointers_to_functions/main}
\input{patterns/185_64bit_in_32_env/main}

\EN{\input{patterns/19_SIMD/main_EN}}
\RU{\input{patterns/19_SIMD/main_RU}}
\DE{\input{patterns/19_SIMD/main_DE}}

\EN{\input{patterns/20_x64/main_EN}}
\RU{\input{patterns/20_x64/main_RU}}

\EN{\input{patterns/205_floating_SIMD/main_EN}}
\RU{\input{patterns/205_floating_SIMD/main_RU}}
\DE{\input{patterns/205_floating_SIMD/main_DE}}

\EN{\input{patterns/ARM/main_EN}}
\RU{\input{patterns/ARM/main_RU}}
\DE{\input{patterns/ARM/main_DE}}

\input{patterns/MIPS/main}

\ifdefined\SPANISH
\chapter{Patrones de código}
\fi % SPANISH

\ifdefined\GERMAN
\chapter{Code-Muster}
\fi % GERMAN

\ifdefined\ENGLISH
\chapter{Code Patterns}
\fi % ENGLISH

\ifdefined\ITALIAN
\chapter{Forme di codice}
\fi % ITALIAN

\ifdefined\RUSSIAN
\chapter{Образцы кода}
\fi % RUSSIAN

\ifdefined\BRAZILIAN
\chapter{Padrões de códigos}
\fi % BRAZILIAN

\ifdefined\THAI
\chapter{รูปแบบของโค้ด}
\fi % THAI

\ifdefined\FRENCH
\chapter{Modèle de code}
\fi % FRENCH

\ifdefined\POLISH
\chapter{\PLph{}}
\fi % POLISH

% sections
\EN{\input{patterns/patterns_opt_dbg_EN}}
\ES{\input{patterns/patterns_opt_dbg_ES}}
\ITA{\input{patterns/patterns_opt_dbg_ITA}}
\PTBR{\input{patterns/patterns_opt_dbg_PTBR}}
\RU{\input{patterns/patterns_opt_dbg_RU}}
\THA{\input{patterns/patterns_opt_dbg_THA}}
\DE{\input{patterns/patterns_opt_dbg_DE}}
\FR{\input{patterns/patterns_opt_dbg_FR}}
\PL{\input{patterns/patterns_opt_dbg_PL}}

\RU{\section{Некоторые базовые понятия}}
\EN{\section{Some basics}}
\DE{\section{Einige Grundlagen}}
\FR{\section{Quelques bases}}
\ES{\section{\ESph{}}}
\ITA{\section{Alcune basi teoriche}}
\PTBR{\section{\PTBRph{}}}
\THA{\section{\THAph{}}}
\PL{\section{\PLph{}}}

% sections:
\EN{\input{patterns/intro_CPU_ISA_EN}}
\ES{\input{patterns/intro_CPU_ISA_ES}}
\ITA{\input{patterns/intro_CPU_ISA_ITA}}
\PTBR{\input{patterns/intro_CPU_ISA_PTBR}}
\RU{\input{patterns/intro_CPU_ISA_RU}}
\DE{\input{patterns/intro_CPU_ISA_DE}}
\FR{\input{patterns/intro_CPU_ISA_FR}}
\PL{\input{patterns/intro_CPU_ISA_PL}}

\EN{\input{patterns/numeral_EN}}
\RU{\input{patterns/numeral_RU}}
\ITA{\input{patterns/numeral_ITA}}
\DE{\input{patterns/numeral_DE}}
\FR{\input{patterns/numeral_FR}}
\PL{\input{patterns/numeral_PL}}

% chapters
\input{patterns/00_empty/main}
\input{patterns/011_ret/main}
\input{patterns/01_helloworld/main}
\input{patterns/015_prolog_epilogue/main}
\input{patterns/02_stack/main}
\input{patterns/03_printf/main}
\input{patterns/04_scanf/main}
\input{patterns/05_passing_arguments/main}
\input{patterns/06_return_results/main}
\input{patterns/061_pointers/main}
\input{patterns/065_GOTO/main}
\input{patterns/07_jcc/main}
\input{patterns/08_switch/main}
\input{patterns/09_loops/main}
\input{patterns/10_strings/main}
\input{patterns/11_arith_optimizations/main}
\input{patterns/12_FPU/main}
\input{patterns/13_arrays/main}
\input{patterns/14_bitfields/main}
\EN{\input{patterns/145_LCG/main_EN}}
\RU{\input{patterns/145_LCG/main_RU}}
\input{patterns/15_structs/main}
\input{patterns/17_unions/main}
\input{patterns/18_pointers_to_functions/main}
\input{patterns/185_64bit_in_32_env/main}

\EN{\input{patterns/19_SIMD/main_EN}}
\RU{\input{patterns/19_SIMD/main_RU}}
\DE{\input{patterns/19_SIMD/main_DE}}

\EN{\input{patterns/20_x64/main_EN}}
\RU{\input{patterns/20_x64/main_RU}}

\EN{\input{patterns/205_floating_SIMD/main_EN}}
\RU{\input{patterns/205_floating_SIMD/main_RU}}
\DE{\input{patterns/205_floating_SIMD/main_DE}}

\EN{\input{patterns/ARM/main_EN}}
\RU{\input{patterns/ARM/main_RU}}
\DE{\input{patterns/ARM/main_DE}}

\input{patterns/MIPS/main}

\ifdefined\SPANISH
\chapter{Patrones de código}
\fi % SPANISH

\ifdefined\GERMAN
\chapter{Code-Muster}
\fi % GERMAN

\ifdefined\ENGLISH
\chapter{Code Patterns}
\fi % ENGLISH

\ifdefined\ITALIAN
\chapter{Forme di codice}
\fi % ITALIAN

\ifdefined\RUSSIAN
\chapter{Образцы кода}
\fi % RUSSIAN

\ifdefined\BRAZILIAN
\chapter{Padrões de códigos}
\fi % BRAZILIAN

\ifdefined\THAI
\chapter{รูปแบบของโค้ด}
\fi % THAI

\ifdefined\FRENCH
\chapter{Modèle de code}
\fi % FRENCH

\ifdefined\POLISH
\chapter{\PLph{}}
\fi % POLISH

% sections
\EN{\input{patterns/patterns_opt_dbg_EN}}
\ES{\input{patterns/patterns_opt_dbg_ES}}
\ITA{\input{patterns/patterns_opt_dbg_ITA}}
\PTBR{\input{patterns/patterns_opt_dbg_PTBR}}
\RU{\input{patterns/patterns_opt_dbg_RU}}
\THA{\input{patterns/patterns_opt_dbg_THA}}
\DE{\input{patterns/patterns_opt_dbg_DE}}
\FR{\input{patterns/patterns_opt_dbg_FR}}
\PL{\input{patterns/patterns_opt_dbg_PL}}

\RU{\section{Некоторые базовые понятия}}
\EN{\section{Some basics}}
\DE{\section{Einige Grundlagen}}
\FR{\section{Quelques bases}}
\ES{\section{\ESph{}}}
\ITA{\section{Alcune basi teoriche}}
\PTBR{\section{\PTBRph{}}}
\THA{\section{\THAph{}}}
\PL{\section{\PLph{}}}

% sections:
\EN{\input{patterns/intro_CPU_ISA_EN}}
\ES{\input{patterns/intro_CPU_ISA_ES}}
\ITA{\input{patterns/intro_CPU_ISA_ITA}}
\PTBR{\input{patterns/intro_CPU_ISA_PTBR}}
\RU{\input{patterns/intro_CPU_ISA_RU}}
\DE{\input{patterns/intro_CPU_ISA_DE}}
\FR{\input{patterns/intro_CPU_ISA_FR}}
\PL{\input{patterns/intro_CPU_ISA_PL}}

\EN{\input{patterns/numeral_EN}}
\RU{\input{patterns/numeral_RU}}
\ITA{\input{patterns/numeral_ITA}}
\DE{\input{patterns/numeral_DE}}
\FR{\input{patterns/numeral_FR}}
\PL{\input{patterns/numeral_PL}}

% chapters
\input{patterns/00_empty/main}
\input{patterns/011_ret/main}
\input{patterns/01_helloworld/main}
\input{patterns/015_prolog_epilogue/main}
\input{patterns/02_stack/main}
\input{patterns/03_printf/main}
\input{patterns/04_scanf/main}
\input{patterns/05_passing_arguments/main}
\input{patterns/06_return_results/main}
\input{patterns/061_pointers/main}
\input{patterns/065_GOTO/main}
\input{patterns/07_jcc/main}
\input{patterns/08_switch/main}
\input{patterns/09_loops/main}
\input{patterns/10_strings/main}
\input{patterns/11_arith_optimizations/main}
\input{patterns/12_FPU/main}
\input{patterns/13_arrays/main}
\input{patterns/14_bitfields/main}
\EN{\input{patterns/145_LCG/main_EN}}
\RU{\input{patterns/145_LCG/main_RU}}
\input{patterns/15_structs/main}
\input{patterns/17_unions/main}
\input{patterns/18_pointers_to_functions/main}
\input{patterns/185_64bit_in_32_env/main}

\EN{\input{patterns/19_SIMD/main_EN}}
\RU{\input{patterns/19_SIMD/main_RU}}
\DE{\input{patterns/19_SIMD/main_DE}}

\EN{\input{patterns/20_x64/main_EN}}
\RU{\input{patterns/20_x64/main_RU}}

\EN{\input{patterns/205_floating_SIMD/main_EN}}
\RU{\input{patterns/205_floating_SIMD/main_RU}}
\DE{\input{patterns/205_floating_SIMD/main_DE}}

\EN{\input{patterns/ARM/main_EN}}
\RU{\input{patterns/ARM/main_RU}}
\DE{\input{patterns/ARM/main_DE}}

\input{patterns/MIPS/main}

\ifdefined\SPANISH
\chapter{Patrones de código}
\fi % SPANISH

\ifdefined\GERMAN
\chapter{Code-Muster}
\fi % GERMAN

\ifdefined\ENGLISH
\chapter{Code Patterns}
\fi % ENGLISH

\ifdefined\ITALIAN
\chapter{Forme di codice}
\fi % ITALIAN

\ifdefined\RUSSIAN
\chapter{Образцы кода}
\fi % RUSSIAN

\ifdefined\BRAZILIAN
\chapter{Padrões de códigos}
\fi % BRAZILIAN

\ifdefined\THAI
\chapter{รูปแบบของโค้ด}
\fi % THAI

\ifdefined\FRENCH
\chapter{Modèle de code}
\fi % FRENCH

\ifdefined\POLISH
\chapter{\PLph{}}
\fi % POLISH

% sections
\EN{\input{patterns/patterns_opt_dbg_EN}}
\ES{\input{patterns/patterns_opt_dbg_ES}}
\ITA{\input{patterns/patterns_opt_dbg_ITA}}
\PTBR{\input{patterns/patterns_opt_dbg_PTBR}}
\RU{\input{patterns/patterns_opt_dbg_RU}}
\THA{\input{patterns/patterns_opt_dbg_THA}}
\DE{\input{patterns/patterns_opt_dbg_DE}}
\FR{\input{patterns/patterns_opt_dbg_FR}}
\PL{\input{patterns/patterns_opt_dbg_PL}}

\RU{\section{Некоторые базовые понятия}}
\EN{\section{Some basics}}
\DE{\section{Einige Grundlagen}}
\FR{\section{Quelques bases}}
\ES{\section{\ESph{}}}
\ITA{\section{Alcune basi teoriche}}
\PTBR{\section{\PTBRph{}}}
\THA{\section{\THAph{}}}
\PL{\section{\PLph{}}}

% sections:
\EN{\input{patterns/intro_CPU_ISA_EN}}
\ES{\input{patterns/intro_CPU_ISA_ES}}
\ITA{\input{patterns/intro_CPU_ISA_ITA}}
\PTBR{\input{patterns/intro_CPU_ISA_PTBR}}
\RU{\input{patterns/intro_CPU_ISA_RU}}
\DE{\input{patterns/intro_CPU_ISA_DE}}
\FR{\input{patterns/intro_CPU_ISA_FR}}
\PL{\input{patterns/intro_CPU_ISA_PL}}

\EN{\input{patterns/numeral_EN}}
\RU{\input{patterns/numeral_RU}}
\ITA{\input{patterns/numeral_ITA}}
\DE{\input{patterns/numeral_DE}}
\FR{\input{patterns/numeral_FR}}
\PL{\input{patterns/numeral_PL}}

% chapters
\input{patterns/00_empty/main}
\input{patterns/011_ret/main}
\input{patterns/01_helloworld/main}
\input{patterns/015_prolog_epilogue/main}
\input{patterns/02_stack/main}
\input{patterns/03_printf/main}
\input{patterns/04_scanf/main}
\input{patterns/05_passing_arguments/main}
\input{patterns/06_return_results/main}
\input{patterns/061_pointers/main}
\input{patterns/065_GOTO/main}
\input{patterns/07_jcc/main}
\input{patterns/08_switch/main}
\input{patterns/09_loops/main}
\input{patterns/10_strings/main}
\input{patterns/11_arith_optimizations/main}
\input{patterns/12_FPU/main}
\input{patterns/13_arrays/main}
\input{patterns/14_bitfields/main}
\EN{\input{patterns/145_LCG/main_EN}}
\RU{\input{patterns/145_LCG/main_RU}}
\input{patterns/15_structs/main}
\input{patterns/17_unions/main}
\input{patterns/18_pointers_to_functions/main}
\input{patterns/185_64bit_in_32_env/main}

\EN{\input{patterns/19_SIMD/main_EN}}
\RU{\input{patterns/19_SIMD/main_RU}}
\DE{\input{patterns/19_SIMD/main_DE}}

\EN{\input{patterns/20_x64/main_EN}}
\RU{\input{patterns/20_x64/main_RU}}

\EN{\input{patterns/205_floating_SIMD/main_EN}}
\RU{\input{patterns/205_floating_SIMD/main_RU}}
\DE{\input{patterns/205_floating_SIMD/main_DE}}

\EN{\input{patterns/ARM/main_EN}}
\RU{\input{patterns/ARM/main_RU}}
\DE{\input{patterns/ARM/main_DE}}

\input{patterns/MIPS/main}

\ifdefined\SPANISH
\chapter{Patrones de código}
\fi % SPANISH

\ifdefined\GERMAN
\chapter{Code-Muster}
\fi % GERMAN

\ifdefined\ENGLISH
\chapter{Code Patterns}
\fi % ENGLISH

\ifdefined\ITALIAN
\chapter{Forme di codice}
\fi % ITALIAN

\ifdefined\RUSSIAN
\chapter{Образцы кода}
\fi % RUSSIAN

\ifdefined\BRAZILIAN
\chapter{Padrões de códigos}
\fi % BRAZILIAN

\ifdefined\THAI
\chapter{รูปแบบของโค้ด}
\fi % THAI

\ifdefined\FRENCH
\chapter{Modèle de code}
\fi % FRENCH

\ifdefined\POLISH
\chapter{\PLph{}}
\fi % POLISH

% sections
\EN{\input{patterns/patterns_opt_dbg_EN}}
\ES{\input{patterns/patterns_opt_dbg_ES}}
\ITA{\input{patterns/patterns_opt_dbg_ITA}}
\PTBR{\input{patterns/patterns_opt_dbg_PTBR}}
\RU{\input{patterns/patterns_opt_dbg_RU}}
\THA{\input{patterns/patterns_opt_dbg_THA}}
\DE{\input{patterns/patterns_opt_dbg_DE}}
\FR{\input{patterns/patterns_opt_dbg_FR}}
\PL{\input{patterns/patterns_opt_dbg_PL}}

\RU{\section{Некоторые базовые понятия}}
\EN{\section{Some basics}}
\DE{\section{Einige Grundlagen}}
\FR{\section{Quelques bases}}
\ES{\section{\ESph{}}}
\ITA{\section{Alcune basi teoriche}}
\PTBR{\section{\PTBRph{}}}
\THA{\section{\THAph{}}}
\PL{\section{\PLph{}}}

% sections:
\EN{\input{patterns/intro_CPU_ISA_EN}}
\ES{\input{patterns/intro_CPU_ISA_ES}}
\ITA{\input{patterns/intro_CPU_ISA_ITA}}
\PTBR{\input{patterns/intro_CPU_ISA_PTBR}}
\RU{\input{patterns/intro_CPU_ISA_RU}}
\DE{\input{patterns/intro_CPU_ISA_DE}}
\FR{\input{patterns/intro_CPU_ISA_FR}}
\PL{\input{patterns/intro_CPU_ISA_PL}}

\EN{\input{patterns/numeral_EN}}
\RU{\input{patterns/numeral_RU}}
\ITA{\input{patterns/numeral_ITA}}
\DE{\input{patterns/numeral_DE}}
\FR{\input{patterns/numeral_FR}}
\PL{\input{patterns/numeral_PL}}

% chapters
\input{patterns/00_empty/main}
\input{patterns/011_ret/main}
\input{patterns/01_helloworld/main}
\input{patterns/015_prolog_epilogue/main}
\input{patterns/02_stack/main}
\input{patterns/03_printf/main}
\input{patterns/04_scanf/main}
\input{patterns/05_passing_arguments/main}
\input{patterns/06_return_results/main}
\input{patterns/061_pointers/main}
\input{patterns/065_GOTO/main}
\input{patterns/07_jcc/main}
\input{patterns/08_switch/main}
\input{patterns/09_loops/main}
\input{patterns/10_strings/main}
\input{patterns/11_arith_optimizations/main}
\input{patterns/12_FPU/main}
\input{patterns/13_arrays/main}
\input{patterns/14_bitfields/main}
\EN{\input{patterns/145_LCG/main_EN}}
\RU{\input{patterns/145_LCG/main_RU}}
\input{patterns/15_structs/main}
\input{patterns/17_unions/main}
\input{patterns/18_pointers_to_functions/main}
\input{patterns/185_64bit_in_32_env/main}

\EN{\input{patterns/19_SIMD/main_EN}}
\RU{\input{patterns/19_SIMD/main_RU}}
\DE{\input{patterns/19_SIMD/main_DE}}

\EN{\input{patterns/20_x64/main_EN}}
\RU{\input{patterns/20_x64/main_RU}}

\EN{\input{patterns/205_floating_SIMD/main_EN}}
\RU{\input{patterns/205_floating_SIMD/main_RU}}
\DE{\input{patterns/205_floating_SIMD/main_DE}}

\EN{\input{patterns/ARM/main_EN}}
\RU{\input{patterns/ARM/main_RU}}
\DE{\input{patterns/ARM/main_DE}}

\input{patterns/MIPS/main}

\ifdefined\SPANISH
\chapter{Patrones de código}
\fi % SPANISH

\ifdefined\GERMAN
\chapter{Code-Muster}
\fi % GERMAN

\ifdefined\ENGLISH
\chapter{Code Patterns}
\fi % ENGLISH

\ifdefined\ITALIAN
\chapter{Forme di codice}
\fi % ITALIAN

\ifdefined\RUSSIAN
\chapter{Образцы кода}
\fi % RUSSIAN

\ifdefined\BRAZILIAN
\chapter{Padrões de códigos}
\fi % BRAZILIAN

\ifdefined\THAI
\chapter{รูปแบบของโค้ด}
\fi % THAI

\ifdefined\FRENCH
\chapter{Modèle de code}
\fi % FRENCH

\ifdefined\POLISH
\chapter{\PLph{}}
\fi % POLISH

% sections
\EN{\input{patterns/patterns_opt_dbg_EN}}
\ES{\input{patterns/patterns_opt_dbg_ES}}
\ITA{\input{patterns/patterns_opt_dbg_ITA}}
\PTBR{\input{patterns/patterns_opt_dbg_PTBR}}
\RU{\input{patterns/patterns_opt_dbg_RU}}
\THA{\input{patterns/patterns_opt_dbg_THA}}
\DE{\input{patterns/patterns_opt_dbg_DE}}
\FR{\input{patterns/patterns_opt_dbg_FR}}
\PL{\input{patterns/patterns_opt_dbg_PL}}

\RU{\section{Некоторые базовые понятия}}
\EN{\section{Some basics}}
\DE{\section{Einige Grundlagen}}
\FR{\section{Quelques bases}}
\ES{\section{\ESph{}}}
\ITA{\section{Alcune basi teoriche}}
\PTBR{\section{\PTBRph{}}}
\THA{\section{\THAph{}}}
\PL{\section{\PLph{}}}

% sections:
\EN{\input{patterns/intro_CPU_ISA_EN}}
\ES{\input{patterns/intro_CPU_ISA_ES}}
\ITA{\input{patterns/intro_CPU_ISA_ITA}}
\PTBR{\input{patterns/intro_CPU_ISA_PTBR}}
\RU{\input{patterns/intro_CPU_ISA_RU}}
\DE{\input{patterns/intro_CPU_ISA_DE}}
\FR{\input{patterns/intro_CPU_ISA_FR}}
\PL{\input{patterns/intro_CPU_ISA_PL}}

\EN{\input{patterns/numeral_EN}}
\RU{\input{patterns/numeral_RU}}
\ITA{\input{patterns/numeral_ITA}}
\DE{\input{patterns/numeral_DE}}
\FR{\input{patterns/numeral_FR}}
\PL{\input{patterns/numeral_PL}}

% chapters
\input{patterns/00_empty/main}
\input{patterns/011_ret/main}
\input{patterns/01_helloworld/main}
\input{patterns/015_prolog_epilogue/main}
\input{patterns/02_stack/main}
\input{patterns/03_printf/main}
\input{patterns/04_scanf/main}
\input{patterns/05_passing_arguments/main}
\input{patterns/06_return_results/main}
\input{patterns/061_pointers/main}
\input{patterns/065_GOTO/main}
\input{patterns/07_jcc/main}
\input{patterns/08_switch/main}
\input{patterns/09_loops/main}
\input{patterns/10_strings/main}
\input{patterns/11_arith_optimizations/main}
\input{patterns/12_FPU/main}
\input{patterns/13_arrays/main}
\input{patterns/14_bitfields/main}
\EN{\input{patterns/145_LCG/main_EN}}
\RU{\input{patterns/145_LCG/main_RU}}
\input{patterns/15_structs/main}
\input{patterns/17_unions/main}
\input{patterns/18_pointers_to_functions/main}
\input{patterns/185_64bit_in_32_env/main}

\EN{\input{patterns/19_SIMD/main_EN}}
\RU{\input{patterns/19_SIMD/main_RU}}
\DE{\input{patterns/19_SIMD/main_DE}}

\EN{\input{patterns/20_x64/main_EN}}
\RU{\input{patterns/20_x64/main_RU}}

\EN{\input{patterns/205_floating_SIMD/main_EN}}
\RU{\input{patterns/205_floating_SIMD/main_RU}}
\DE{\input{patterns/205_floating_SIMD/main_DE}}

\EN{\input{patterns/ARM/main_EN}}
\RU{\input{patterns/ARM/main_RU}}
\DE{\input{patterns/ARM/main_DE}}

\input{patterns/MIPS/main}

\EN{\input{patterns/12_FPU/main_EN}}
\RU{\input{patterns/12_FPU/main_RU}}
\DE{\input{patterns/12_FPU/main_DE}}
\FR{\input{patterns/12_FPU/main_FR}}


\ifdefined\SPANISH
\chapter{Patrones de código}
\fi % SPANISH

\ifdefined\GERMAN
\chapter{Code-Muster}
\fi % GERMAN

\ifdefined\ENGLISH
\chapter{Code Patterns}
\fi % ENGLISH

\ifdefined\ITALIAN
\chapter{Forme di codice}
\fi % ITALIAN

\ifdefined\RUSSIAN
\chapter{Образцы кода}
\fi % RUSSIAN

\ifdefined\BRAZILIAN
\chapter{Padrões de códigos}
\fi % BRAZILIAN

\ifdefined\THAI
\chapter{รูปแบบของโค้ด}
\fi % THAI

\ifdefined\FRENCH
\chapter{Modèle de code}
\fi % FRENCH

\ifdefined\POLISH
\chapter{\PLph{}}
\fi % POLISH

% sections
\EN{\input{patterns/patterns_opt_dbg_EN}}
\ES{\input{patterns/patterns_opt_dbg_ES}}
\ITA{\input{patterns/patterns_opt_dbg_ITA}}
\PTBR{\input{patterns/patterns_opt_dbg_PTBR}}
\RU{\input{patterns/patterns_opt_dbg_RU}}
\THA{\input{patterns/patterns_opt_dbg_THA}}
\DE{\input{patterns/patterns_opt_dbg_DE}}
\FR{\input{patterns/patterns_opt_dbg_FR}}
\PL{\input{patterns/patterns_opt_dbg_PL}}

\RU{\section{Некоторые базовые понятия}}
\EN{\section{Some basics}}
\DE{\section{Einige Grundlagen}}
\FR{\section{Quelques bases}}
\ES{\section{\ESph{}}}
\ITA{\section{Alcune basi teoriche}}
\PTBR{\section{\PTBRph{}}}
\THA{\section{\THAph{}}}
\PL{\section{\PLph{}}}

% sections:
\EN{\input{patterns/intro_CPU_ISA_EN}}
\ES{\input{patterns/intro_CPU_ISA_ES}}
\ITA{\input{patterns/intro_CPU_ISA_ITA}}
\PTBR{\input{patterns/intro_CPU_ISA_PTBR}}
\RU{\input{patterns/intro_CPU_ISA_RU}}
\DE{\input{patterns/intro_CPU_ISA_DE}}
\FR{\input{patterns/intro_CPU_ISA_FR}}
\PL{\input{patterns/intro_CPU_ISA_PL}}

\EN{\input{patterns/numeral_EN}}
\RU{\input{patterns/numeral_RU}}
\ITA{\input{patterns/numeral_ITA}}
\DE{\input{patterns/numeral_DE}}
\FR{\input{patterns/numeral_FR}}
\PL{\input{patterns/numeral_PL}}

% chapters
\input{patterns/00_empty/main}
\input{patterns/011_ret/main}
\input{patterns/01_helloworld/main}
\input{patterns/015_prolog_epilogue/main}
\input{patterns/02_stack/main}
\input{patterns/03_printf/main}
\input{patterns/04_scanf/main}
\input{patterns/05_passing_arguments/main}
\input{patterns/06_return_results/main}
\input{patterns/061_pointers/main}
\input{patterns/065_GOTO/main}
\input{patterns/07_jcc/main}
\input{patterns/08_switch/main}
\input{patterns/09_loops/main}
\input{patterns/10_strings/main}
\input{patterns/11_arith_optimizations/main}
\input{patterns/12_FPU/main}
\input{patterns/13_arrays/main}
\input{patterns/14_bitfields/main}
\EN{\input{patterns/145_LCG/main_EN}}
\RU{\input{patterns/145_LCG/main_RU}}
\input{patterns/15_structs/main}
\input{patterns/17_unions/main}
\input{patterns/18_pointers_to_functions/main}
\input{patterns/185_64bit_in_32_env/main}

\EN{\input{patterns/19_SIMD/main_EN}}
\RU{\input{patterns/19_SIMD/main_RU}}
\DE{\input{patterns/19_SIMD/main_DE}}

\EN{\input{patterns/20_x64/main_EN}}
\RU{\input{patterns/20_x64/main_RU}}

\EN{\input{patterns/205_floating_SIMD/main_EN}}
\RU{\input{patterns/205_floating_SIMD/main_RU}}
\DE{\input{patterns/205_floating_SIMD/main_DE}}

\EN{\input{patterns/ARM/main_EN}}
\RU{\input{patterns/ARM/main_RU}}
\DE{\input{patterns/ARM/main_DE}}

\input{patterns/MIPS/main}

\ifdefined\SPANISH
\chapter{Patrones de código}
\fi % SPANISH

\ifdefined\GERMAN
\chapter{Code-Muster}
\fi % GERMAN

\ifdefined\ENGLISH
\chapter{Code Patterns}
\fi % ENGLISH

\ifdefined\ITALIAN
\chapter{Forme di codice}
\fi % ITALIAN

\ifdefined\RUSSIAN
\chapter{Образцы кода}
\fi % RUSSIAN

\ifdefined\BRAZILIAN
\chapter{Padrões de códigos}
\fi % BRAZILIAN

\ifdefined\THAI
\chapter{รูปแบบของโค้ด}
\fi % THAI

\ifdefined\FRENCH
\chapter{Modèle de code}
\fi % FRENCH

\ifdefined\POLISH
\chapter{\PLph{}}
\fi % POLISH

% sections
\EN{\input{patterns/patterns_opt_dbg_EN}}
\ES{\input{patterns/patterns_opt_dbg_ES}}
\ITA{\input{patterns/patterns_opt_dbg_ITA}}
\PTBR{\input{patterns/patterns_opt_dbg_PTBR}}
\RU{\input{patterns/patterns_opt_dbg_RU}}
\THA{\input{patterns/patterns_opt_dbg_THA}}
\DE{\input{patterns/patterns_opt_dbg_DE}}
\FR{\input{patterns/patterns_opt_dbg_FR}}
\PL{\input{patterns/patterns_opt_dbg_PL}}

\RU{\section{Некоторые базовые понятия}}
\EN{\section{Some basics}}
\DE{\section{Einige Grundlagen}}
\FR{\section{Quelques bases}}
\ES{\section{\ESph{}}}
\ITA{\section{Alcune basi teoriche}}
\PTBR{\section{\PTBRph{}}}
\THA{\section{\THAph{}}}
\PL{\section{\PLph{}}}

% sections:
\EN{\input{patterns/intro_CPU_ISA_EN}}
\ES{\input{patterns/intro_CPU_ISA_ES}}
\ITA{\input{patterns/intro_CPU_ISA_ITA}}
\PTBR{\input{patterns/intro_CPU_ISA_PTBR}}
\RU{\input{patterns/intro_CPU_ISA_RU}}
\DE{\input{patterns/intro_CPU_ISA_DE}}
\FR{\input{patterns/intro_CPU_ISA_FR}}
\PL{\input{patterns/intro_CPU_ISA_PL}}

\EN{\input{patterns/numeral_EN}}
\RU{\input{patterns/numeral_RU}}
\ITA{\input{patterns/numeral_ITA}}
\DE{\input{patterns/numeral_DE}}
\FR{\input{patterns/numeral_FR}}
\PL{\input{patterns/numeral_PL}}

% chapters
\input{patterns/00_empty/main}
\input{patterns/011_ret/main}
\input{patterns/01_helloworld/main}
\input{patterns/015_prolog_epilogue/main}
\input{patterns/02_stack/main}
\input{patterns/03_printf/main}
\input{patterns/04_scanf/main}
\input{patterns/05_passing_arguments/main}
\input{patterns/06_return_results/main}
\input{patterns/061_pointers/main}
\input{patterns/065_GOTO/main}
\input{patterns/07_jcc/main}
\input{patterns/08_switch/main}
\input{patterns/09_loops/main}
\input{patterns/10_strings/main}
\input{patterns/11_arith_optimizations/main}
\input{patterns/12_FPU/main}
\input{patterns/13_arrays/main}
\input{patterns/14_bitfields/main}
\EN{\input{patterns/145_LCG/main_EN}}
\RU{\input{patterns/145_LCG/main_RU}}
\input{patterns/15_structs/main}
\input{patterns/17_unions/main}
\input{patterns/18_pointers_to_functions/main}
\input{patterns/185_64bit_in_32_env/main}

\EN{\input{patterns/19_SIMD/main_EN}}
\RU{\input{patterns/19_SIMD/main_RU}}
\DE{\input{patterns/19_SIMD/main_DE}}

\EN{\input{patterns/20_x64/main_EN}}
\RU{\input{patterns/20_x64/main_RU}}

\EN{\input{patterns/205_floating_SIMD/main_EN}}
\RU{\input{patterns/205_floating_SIMD/main_RU}}
\DE{\input{patterns/205_floating_SIMD/main_DE}}

\EN{\input{patterns/ARM/main_EN}}
\RU{\input{patterns/ARM/main_RU}}
\DE{\input{patterns/ARM/main_DE}}

\input{patterns/MIPS/main}

\EN{\section{Returning Values}
\label{ret_val_func}

Another simple function is the one that simply returns a constant value:

\lstinputlisting[caption=\EN{\CCpp Code},style=customc]{patterns/011_ret/1.c}

Let's compile it.

\subsection{x86}

Here's what both the GCC and MSVC compilers produce (with optimization) on the x86 platform:

\lstinputlisting[caption=\Optimizing GCC/MSVC (\assemblyOutput),style=customasmx86]{patterns/011_ret/1.s}

\myindex{x86!\Instructions!RET}
There are just two instructions: the first places the value 123 into the \EAX register,
which is used by convention for storing the return
value, and the second one is \RET, which returns execution to the \gls{caller}.

The caller will take the result from the \EAX register.

\subsection{ARM}

There are a few differences on the ARM platform:

\lstinputlisting[caption=\OptimizingKeilVI (\ARMMode) ASM Output,style=customasmARM]{patterns/011_ret/1_Keil_ARM_O3.s}

ARM uses the register \Reg{0} for returning the results of functions, so 123 is copied into \Reg{0}.

\myindex{ARM!\Instructions!MOV}
\myindex{x86!\Instructions!MOV}
It is worth noting that \MOV is a misleading name for the instruction in both the x86 and ARM \ac{ISA}s.

The data is not in fact \IT{moved}, but \IT{copied}.

\subsection{MIPS}

\label{MIPS_leaf_function_ex1}

The GCC assembly output below lists registers by number:

\lstinputlisting[caption=\Optimizing GCC 4.4.5 (\assemblyOutput),style=customasmMIPS]{patterns/011_ret/MIPS.s}

\dots while \IDA does it by their pseudo names:

\lstinputlisting[caption=\Optimizing GCC 4.4.5 (IDA),style=customasmMIPS]{patterns/011_ret/MIPS_IDA.lst}

The \$2 (or \$V0) register is used to store the function's return value.
\myindex{MIPS!\Pseudoinstructions!LI}
\INS{LI} stands for ``Load Immediate'' and is the MIPS equivalent to \MOV.

\myindex{MIPS!\Instructions!J}
The other instruction is the jump instruction (J or JR) which returns the execution flow to the \gls{caller}.

\myindex{MIPS!Branch delay slot}
You might be wondering why the positions of the load instruction (LI) and the jump instruction (J or JR) are swapped. This is due to a \ac{RISC} feature called ``branch delay slot''.

The reason this happens is a quirk in the architecture of some RISC \ac{ISA}s and isn't important for our
purposes---we must simply keep in mind that in MIPS, the instruction following a jump or branch instruction
is executed \IT{before} the jump/branch instruction itself.

As a consequence, branch instructions always swap places with the instruction executed immediately beforehand.


In practice, functions which merely return 1 (\IT{true}) or 0 (\IT{false}) are very frequent.

The smallest ever of the standard UNIX utilities, \IT{/bin/true} and \IT{/bin/false} return 0 and 1 respectively, as an exit code.
(Zero as an exit code usually means success, non-zero means error.)
}
\RU{\subsubsection{std::string}
\myindex{\Cpp!STL!std::string}
\label{std_string}

\myparagraph{Как устроена структура}

Многие строковые библиотеки \InSqBrackets{\CNotes 2.2} обеспечивают структуру содержащую ссылку 
на буфер собственно со строкой, переменная всегда содержащую длину строки 
(что очень удобно для массы функций \InSqBrackets{\CNotes 2.2.1}) и переменную содержащую текущий размер буфера.

Строка в буфере обыкновенно оканчивается нулем: это для того чтобы указатель на буфер можно было
передавать в функции требующие на вход обычную сишную \ac{ASCIIZ}-строку.

Стандарт \Cpp не описывает, как именно нужно реализовывать std::string,
но, как правило, они реализованы как описано выше, с небольшими дополнениями.

Строки в \Cpp это не класс (как, например, QString в Qt), а темплейт (basic\_string), 
это сделано для того чтобы поддерживать 
строки содержащие разного типа символы: как минимум \Tchar и \IT{wchar\_t}.

Так что, std::string это класс с базовым типом \Tchar.

А std::wstring это класс с базовым типом \IT{wchar\_t}.

\mysubparagraph{MSVC}

В реализации MSVC, вместо ссылки на буфер может содержаться сам буфер (если строка короче 16-и символов).

Это означает, что каждая короткая строка будет занимать в памяти по крайней мере $16 + 4 + 4 = 24$ 
байт для 32-битной среды либо $16 + 8 + 8 = 32$ 
байта в 64-битной, а если строка длиннее 16-и символов, то прибавьте еще длину самой строки.

\lstinputlisting[caption=пример для MSVC,style=customc]{\CURPATH/STL/string/MSVC_RU.cpp}

Собственно, из этого исходника почти всё ясно.

Несколько замечаний:

Если строка короче 16-и символов, 
то отдельный буфер для строки в \glslink{heap}{куче} выделяться не будет.

Это удобно потому что на практике, основная часть строк действительно короткие.
Вероятно, разработчики в Microsoft выбрали размер в 16 символов как разумный баланс.

Теперь очень важный момент в конце функции main(): мы не пользуемся методом c\_str(), тем не менее,
если это скомпилировать и запустить, то обе строки появятся в консоли!

Работает это вот почему.

В первом случае строка короче 16-и символов и в начале объекта std::string (его можно рассматривать
просто как структуру) расположен буфер с этой строкой.
\printf трактует указатель как указатель на массив символов оканчивающийся нулем и поэтому всё работает.

Вывод второй строки (длиннее 16-и символов) даже еще опаснее: это вообще типичная программистская ошибка 
(или опечатка), забыть дописать c\_str().
Это работает потому что в это время в начале структуры расположен указатель на буфер.
Это может надолго остаться незамеченным: до тех пока там не появится строка 
короче 16-и символов, тогда процесс упадет.

\mysubparagraph{GCC}

В реализации GCC в структуре есть еще одна переменная --- reference count.

Интересно, что указатель на экземпляр класса std::string в GCC указывает не на начало самой структуры, 
а на указатель на буфера.
В libstdc++-v3\textbackslash{}include\textbackslash{}bits\textbackslash{}basic\_string.h 
мы можем прочитать что это сделано для удобства отладки:

\begin{lstlisting}
   *  The reason you want _M_data pointing to the character %array and
   *  not the _Rep is so that the debugger can see the string
   *  contents. (Probably we should add a non-inline member to get
   *  the _Rep for the debugger to use, so users can check the actual
   *  string length.)
\end{lstlisting}

\href{http://go.yurichev.com/17085}{исходный код basic\_string.h}

В нашем примере мы учитываем это:

\lstinputlisting[caption=пример для GCC,style=customc]{\CURPATH/STL/string/GCC_RU.cpp}

Нужны еще небольшие хаки чтобы сымитировать типичную ошибку, которую мы уже видели выше, из-за
более ужесточенной проверки типов в GCC, тем не менее, printf() работает и здесь без c\_str().

\myparagraph{Чуть более сложный пример}

\lstinputlisting[style=customc]{\CURPATH/STL/string/3.cpp}

\lstinputlisting[caption=MSVC 2012,style=customasmx86]{\CURPATH/STL/string/3_MSVC_RU.asm}

Собственно, компилятор не конструирует строки статически: да в общем-то и как
это возможно, если буфер с ней нужно хранить в \glslink{heap}{куче}?

Вместо этого в сегменте данных хранятся обычные \ac{ASCIIZ}-строки, а позже, во время выполнения, 
при помощи метода \q{assign}, конструируются строки s1 и s2
.
При помощи \TT{operator+}, создается строка s3.

Обратите внимание на то что вызов метода c\_str() отсутствует,
потому что его код достаточно короткий и компилятор вставил его прямо здесь:
если строка короче 16-и байт, то в регистре EAX остается указатель на буфер,
а если длиннее, то из этого же места достается адрес на буфер расположенный в \glslink{heap}{куче}.

Далее следуют вызовы трех деструкторов, причем, они вызываются только если строка длиннее 16-и байт:
тогда нужно освободить буфера в \glslink{heap}{куче}.
В противном случае, так как все три объекта std::string хранятся в стеке,
они освобождаются автоматически после выхода из функции.

Следовательно, работа с короткими строками более быстрая из-за м\'{е}ньшего обращения к \glslink{heap}{куче}.

Код на GCC даже проще (из-за того, что в GCC, как мы уже видели, не реализована возможность хранить короткую
строку прямо в структуре):

% TODO1 comment each function meaning
\lstinputlisting[caption=GCC 4.8.1,style=customasmx86]{\CURPATH/STL/string/3_GCC_RU.s}

Можно заметить, что в деструкторы передается не указатель на объект,
а указатель на место за 12 байт (или 3 слова) перед ним, то есть, на настоящее начало структуры.

\myparagraph{std::string как глобальная переменная}
\label{sec:std_string_as_global_variable}

Опытные программисты на \Cpp знают, что глобальные переменные \ac{STL}-типов вполне можно объявлять.

Да, действительно:

\lstinputlisting[style=customc]{\CURPATH/STL/string/5.cpp}

Но как и где будет вызываться конструктор \TT{std::string}?

На самом деле, эта переменная будет инициализирована даже перед началом \main.

\lstinputlisting[caption=MSVC 2012: здесь конструируется глобальная переменная{,} а также регистрируется её деструктор,style=customasmx86]{\CURPATH/STL/string/5_MSVC_p2.asm}

\lstinputlisting[caption=MSVC 2012: здесь глобальная переменная используется в \main,style=customasmx86]{\CURPATH/STL/string/5_MSVC_p1.asm}

\lstinputlisting[caption=MSVC 2012: эта функция-деструктор вызывается перед выходом,style=customasmx86]{\CURPATH/STL/string/5_MSVC_p3.asm}

\myindex{\CStandardLibrary!atexit()}
В реальности, из \ac{CRT}, еще до вызова main(), вызывается специальная функция,
в которой перечислены все конструкторы подобных переменных.
Более того: при помощи atexit() регистрируется функция, которая будет вызвана в конце работы программы:
в этой функции компилятор собирает вызовы деструкторов всех подобных глобальных переменных.

GCC работает похожим образом:

\lstinputlisting[caption=GCC 4.8.1,style=customasmx86]{\CURPATH/STL/string/5_GCC.s}

Но он не выделяет отдельной функции в которой будут собраны деструкторы: 
каждый деструктор передается в atexit() по одному.

% TODO а если глобальная STL-переменная в другом модуле? надо проверить.

}
\ifdefined\SPANISH
\chapter{Patrones de código}
\fi % SPANISH

\ifdefined\GERMAN
\chapter{Code-Muster}
\fi % GERMAN

\ifdefined\ENGLISH
\chapter{Code Patterns}
\fi % ENGLISH

\ifdefined\ITALIAN
\chapter{Forme di codice}
\fi % ITALIAN

\ifdefined\RUSSIAN
\chapter{Образцы кода}
\fi % RUSSIAN

\ifdefined\BRAZILIAN
\chapter{Padrões de códigos}
\fi % BRAZILIAN

\ifdefined\THAI
\chapter{รูปแบบของโค้ด}
\fi % THAI

\ifdefined\FRENCH
\chapter{Modèle de code}
\fi % FRENCH

\ifdefined\POLISH
\chapter{\PLph{}}
\fi % POLISH

% sections
\EN{\input{patterns/patterns_opt_dbg_EN}}
\ES{\input{patterns/patterns_opt_dbg_ES}}
\ITA{\input{patterns/patterns_opt_dbg_ITA}}
\PTBR{\input{patterns/patterns_opt_dbg_PTBR}}
\RU{\input{patterns/patterns_opt_dbg_RU}}
\THA{\input{patterns/patterns_opt_dbg_THA}}
\DE{\input{patterns/patterns_opt_dbg_DE}}
\FR{\input{patterns/patterns_opt_dbg_FR}}
\PL{\input{patterns/patterns_opt_dbg_PL}}

\RU{\section{Некоторые базовые понятия}}
\EN{\section{Some basics}}
\DE{\section{Einige Grundlagen}}
\FR{\section{Quelques bases}}
\ES{\section{\ESph{}}}
\ITA{\section{Alcune basi teoriche}}
\PTBR{\section{\PTBRph{}}}
\THA{\section{\THAph{}}}
\PL{\section{\PLph{}}}

% sections:
\EN{\input{patterns/intro_CPU_ISA_EN}}
\ES{\input{patterns/intro_CPU_ISA_ES}}
\ITA{\input{patterns/intro_CPU_ISA_ITA}}
\PTBR{\input{patterns/intro_CPU_ISA_PTBR}}
\RU{\input{patterns/intro_CPU_ISA_RU}}
\DE{\input{patterns/intro_CPU_ISA_DE}}
\FR{\input{patterns/intro_CPU_ISA_FR}}
\PL{\input{patterns/intro_CPU_ISA_PL}}

\EN{\input{patterns/numeral_EN}}
\RU{\input{patterns/numeral_RU}}
\ITA{\input{patterns/numeral_ITA}}
\DE{\input{patterns/numeral_DE}}
\FR{\input{patterns/numeral_FR}}
\PL{\input{patterns/numeral_PL}}

% chapters
\input{patterns/00_empty/main}
\input{patterns/011_ret/main}
\input{patterns/01_helloworld/main}
\input{patterns/015_prolog_epilogue/main}
\input{patterns/02_stack/main}
\input{patterns/03_printf/main}
\input{patterns/04_scanf/main}
\input{patterns/05_passing_arguments/main}
\input{patterns/06_return_results/main}
\input{patterns/061_pointers/main}
\input{patterns/065_GOTO/main}
\input{patterns/07_jcc/main}
\input{patterns/08_switch/main}
\input{patterns/09_loops/main}
\input{patterns/10_strings/main}
\input{patterns/11_arith_optimizations/main}
\input{patterns/12_FPU/main}
\input{patterns/13_arrays/main}
\input{patterns/14_bitfields/main}
\EN{\input{patterns/145_LCG/main_EN}}
\RU{\input{patterns/145_LCG/main_RU}}
\input{patterns/15_structs/main}
\input{patterns/17_unions/main}
\input{patterns/18_pointers_to_functions/main}
\input{patterns/185_64bit_in_32_env/main}

\EN{\input{patterns/19_SIMD/main_EN}}
\RU{\input{patterns/19_SIMD/main_RU}}
\DE{\input{patterns/19_SIMD/main_DE}}

\EN{\input{patterns/20_x64/main_EN}}
\RU{\input{patterns/20_x64/main_RU}}

\EN{\input{patterns/205_floating_SIMD/main_EN}}
\RU{\input{patterns/205_floating_SIMD/main_RU}}
\DE{\input{patterns/205_floating_SIMD/main_DE}}

\EN{\input{patterns/ARM/main_EN}}
\RU{\input{patterns/ARM/main_RU}}
\DE{\input{patterns/ARM/main_DE}}

\input{patterns/MIPS/main}

\ifdefined\SPANISH
\chapter{Patrones de código}
\fi % SPANISH

\ifdefined\GERMAN
\chapter{Code-Muster}
\fi % GERMAN

\ifdefined\ENGLISH
\chapter{Code Patterns}
\fi % ENGLISH

\ifdefined\ITALIAN
\chapter{Forme di codice}
\fi % ITALIAN

\ifdefined\RUSSIAN
\chapter{Образцы кода}
\fi % RUSSIAN

\ifdefined\BRAZILIAN
\chapter{Padrões de códigos}
\fi % BRAZILIAN

\ifdefined\THAI
\chapter{รูปแบบของโค้ด}
\fi % THAI

\ifdefined\FRENCH
\chapter{Modèle de code}
\fi % FRENCH

\ifdefined\POLISH
\chapter{\PLph{}}
\fi % POLISH

% sections
\EN{\input{patterns/patterns_opt_dbg_EN}}
\ES{\input{patterns/patterns_opt_dbg_ES}}
\ITA{\input{patterns/patterns_opt_dbg_ITA}}
\PTBR{\input{patterns/patterns_opt_dbg_PTBR}}
\RU{\input{patterns/patterns_opt_dbg_RU}}
\THA{\input{patterns/patterns_opt_dbg_THA}}
\DE{\input{patterns/patterns_opt_dbg_DE}}
\FR{\input{patterns/patterns_opt_dbg_FR}}
\PL{\input{patterns/patterns_opt_dbg_PL}}

\RU{\section{Некоторые базовые понятия}}
\EN{\section{Some basics}}
\DE{\section{Einige Grundlagen}}
\FR{\section{Quelques bases}}
\ES{\section{\ESph{}}}
\ITA{\section{Alcune basi teoriche}}
\PTBR{\section{\PTBRph{}}}
\THA{\section{\THAph{}}}
\PL{\section{\PLph{}}}

% sections:
\EN{\input{patterns/intro_CPU_ISA_EN}}
\ES{\input{patterns/intro_CPU_ISA_ES}}
\ITA{\input{patterns/intro_CPU_ISA_ITA}}
\PTBR{\input{patterns/intro_CPU_ISA_PTBR}}
\RU{\input{patterns/intro_CPU_ISA_RU}}
\DE{\input{patterns/intro_CPU_ISA_DE}}
\FR{\input{patterns/intro_CPU_ISA_FR}}
\PL{\input{patterns/intro_CPU_ISA_PL}}

\EN{\input{patterns/numeral_EN}}
\RU{\input{patterns/numeral_RU}}
\ITA{\input{patterns/numeral_ITA}}
\DE{\input{patterns/numeral_DE}}
\FR{\input{patterns/numeral_FR}}
\PL{\input{patterns/numeral_PL}}

% chapters
\input{patterns/00_empty/main}
\input{patterns/011_ret/main}
\input{patterns/01_helloworld/main}
\input{patterns/015_prolog_epilogue/main}
\input{patterns/02_stack/main}
\input{patterns/03_printf/main}
\input{patterns/04_scanf/main}
\input{patterns/05_passing_arguments/main}
\input{patterns/06_return_results/main}
\input{patterns/061_pointers/main}
\input{patterns/065_GOTO/main}
\input{patterns/07_jcc/main}
\input{patterns/08_switch/main}
\input{patterns/09_loops/main}
\input{patterns/10_strings/main}
\input{patterns/11_arith_optimizations/main}
\input{patterns/12_FPU/main}
\input{patterns/13_arrays/main}
\input{patterns/14_bitfields/main}
\EN{\input{patterns/145_LCG/main_EN}}
\RU{\input{patterns/145_LCG/main_RU}}
\input{patterns/15_structs/main}
\input{patterns/17_unions/main}
\input{patterns/18_pointers_to_functions/main}
\input{patterns/185_64bit_in_32_env/main}

\EN{\input{patterns/19_SIMD/main_EN}}
\RU{\input{patterns/19_SIMD/main_RU}}
\DE{\input{patterns/19_SIMD/main_DE}}

\EN{\input{patterns/20_x64/main_EN}}
\RU{\input{patterns/20_x64/main_RU}}

\EN{\input{patterns/205_floating_SIMD/main_EN}}
\RU{\input{patterns/205_floating_SIMD/main_RU}}
\DE{\input{patterns/205_floating_SIMD/main_DE}}

\EN{\input{patterns/ARM/main_EN}}
\RU{\input{patterns/ARM/main_RU}}
\DE{\input{patterns/ARM/main_DE}}

\input{patterns/MIPS/main}

\ifdefined\SPANISH
\chapter{Patrones de código}
\fi % SPANISH

\ifdefined\GERMAN
\chapter{Code-Muster}
\fi % GERMAN

\ifdefined\ENGLISH
\chapter{Code Patterns}
\fi % ENGLISH

\ifdefined\ITALIAN
\chapter{Forme di codice}
\fi % ITALIAN

\ifdefined\RUSSIAN
\chapter{Образцы кода}
\fi % RUSSIAN

\ifdefined\BRAZILIAN
\chapter{Padrões de códigos}
\fi % BRAZILIAN

\ifdefined\THAI
\chapter{รูปแบบของโค้ด}
\fi % THAI

\ifdefined\FRENCH
\chapter{Modèle de code}
\fi % FRENCH

\ifdefined\POLISH
\chapter{\PLph{}}
\fi % POLISH

% sections
\EN{\input{patterns/patterns_opt_dbg_EN}}
\ES{\input{patterns/patterns_opt_dbg_ES}}
\ITA{\input{patterns/patterns_opt_dbg_ITA}}
\PTBR{\input{patterns/patterns_opt_dbg_PTBR}}
\RU{\input{patterns/patterns_opt_dbg_RU}}
\THA{\input{patterns/patterns_opt_dbg_THA}}
\DE{\input{patterns/patterns_opt_dbg_DE}}
\FR{\input{patterns/patterns_opt_dbg_FR}}
\PL{\input{patterns/patterns_opt_dbg_PL}}

\RU{\section{Некоторые базовые понятия}}
\EN{\section{Some basics}}
\DE{\section{Einige Grundlagen}}
\FR{\section{Quelques bases}}
\ES{\section{\ESph{}}}
\ITA{\section{Alcune basi teoriche}}
\PTBR{\section{\PTBRph{}}}
\THA{\section{\THAph{}}}
\PL{\section{\PLph{}}}

% sections:
\EN{\input{patterns/intro_CPU_ISA_EN}}
\ES{\input{patterns/intro_CPU_ISA_ES}}
\ITA{\input{patterns/intro_CPU_ISA_ITA}}
\PTBR{\input{patterns/intro_CPU_ISA_PTBR}}
\RU{\input{patterns/intro_CPU_ISA_RU}}
\DE{\input{patterns/intro_CPU_ISA_DE}}
\FR{\input{patterns/intro_CPU_ISA_FR}}
\PL{\input{patterns/intro_CPU_ISA_PL}}

\EN{\input{patterns/numeral_EN}}
\RU{\input{patterns/numeral_RU}}
\ITA{\input{patterns/numeral_ITA}}
\DE{\input{patterns/numeral_DE}}
\FR{\input{patterns/numeral_FR}}
\PL{\input{patterns/numeral_PL}}

% chapters
\input{patterns/00_empty/main}
\input{patterns/011_ret/main}
\input{patterns/01_helloworld/main}
\input{patterns/015_prolog_epilogue/main}
\input{patterns/02_stack/main}
\input{patterns/03_printf/main}
\input{patterns/04_scanf/main}
\input{patterns/05_passing_arguments/main}
\input{patterns/06_return_results/main}
\input{patterns/061_pointers/main}
\input{patterns/065_GOTO/main}
\input{patterns/07_jcc/main}
\input{patterns/08_switch/main}
\input{patterns/09_loops/main}
\input{patterns/10_strings/main}
\input{patterns/11_arith_optimizations/main}
\input{patterns/12_FPU/main}
\input{patterns/13_arrays/main}
\input{patterns/14_bitfields/main}
\EN{\input{patterns/145_LCG/main_EN}}
\RU{\input{patterns/145_LCG/main_RU}}
\input{patterns/15_structs/main}
\input{patterns/17_unions/main}
\input{patterns/18_pointers_to_functions/main}
\input{patterns/185_64bit_in_32_env/main}

\EN{\input{patterns/19_SIMD/main_EN}}
\RU{\input{patterns/19_SIMD/main_RU}}
\DE{\input{patterns/19_SIMD/main_DE}}

\EN{\input{patterns/20_x64/main_EN}}
\RU{\input{patterns/20_x64/main_RU}}

\EN{\input{patterns/205_floating_SIMD/main_EN}}
\RU{\input{patterns/205_floating_SIMD/main_RU}}
\DE{\input{patterns/205_floating_SIMD/main_DE}}

\EN{\input{patterns/ARM/main_EN}}
\RU{\input{patterns/ARM/main_RU}}
\DE{\input{patterns/ARM/main_DE}}

\input{patterns/MIPS/main}

\ifdefined\SPANISH
\chapter{Patrones de código}
\fi % SPANISH

\ifdefined\GERMAN
\chapter{Code-Muster}
\fi % GERMAN

\ifdefined\ENGLISH
\chapter{Code Patterns}
\fi % ENGLISH

\ifdefined\ITALIAN
\chapter{Forme di codice}
\fi % ITALIAN

\ifdefined\RUSSIAN
\chapter{Образцы кода}
\fi % RUSSIAN

\ifdefined\BRAZILIAN
\chapter{Padrões de códigos}
\fi % BRAZILIAN

\ifdefined\THAI
\chapter{รูปแบบของโค้ด}
\fi % THAI

\ifdefined\FRENCH
\chapter{Modèle de code}
\fi % FRENCH

\ifdefined\POLISH
\chapter{\PLph{}}
\fi % POLISH

% sections
\EN{\input{patterns/patterns_opt_dbg_EN}}
\ES{\input{patterns/patterns_opt_dbg_ES}}
\ITA{\input{patterns/patterns_opt_dbg_ITA}}
\PTBR{\input{patterns/patterns_opt_dbg_PTBR}}
\RU{\input{patterns/patterns_opt_dbg_RU}}
\THA{\input{patterns/patterns_opt_dbg_THA}}
\DE{\input{patterns/patterns_opt_dbg_DE}}
\FR{\input{patterns/patterns_opt_dbg_FR}}
\PL{\input{patterns/patterns_opt_dbg_PL}}

\RU{\section{Некоторые базовые понятия}}
\EN{\section{Some basics}}
\DE{\section{Einige Grundlagen}}
\FR{\section{Quelques bases}}
\ES{\section{\ESph{}}}
\ITA{\section{Alcune basi teoriche}}
\PTBR{\section{\PTBRph{}}}
\THA{\section{\THAph{}}}
\PL{\section{\PLph{}}}

% sections:
\EN{\input{patterns/intro_CPU_ISA_EN}}
\ES{\input{patterns/intro_CPU_ISA_ES}}
\ITA{\input{patterns/intro_CPU_ISA_ITA}}
\PTBR{\input{patterns/intro_CPU_ISA_PTBR}}
\RU{\input{patterns/intro_CPU_ISA_RU}}
\DE{\input{patterns/intro_CPU_ISA_DE}}
\FR{\input{patterns/intro_CPU_ISA_FR}}
\PL{\input{patterns/intro_CPU_ISA_PL}}

\EN{\input{patterns/numeral_EN}}
\RU{\input{patterns/numeral_RU}}
\ITA{\input{patterns/numeral_ITA}}
\DE{\input{patterns/numeral_DE}}
\FR{\input{patterns/numeral_FR}}
\PL{\input{patterns/numeral_PL}}

% chapters
\input{patterns/00_empty/main}
\input{patterns/011_ret/main}
\input{patterns/01_helloworld/main}
\input{patterns/015_prolog_epilogue/main}
\input{patterns/02_stack/main}
\input{patterns/03_printf/main}
\input{patterns/04_scanf/main}
\input{patterns/05_passing_arguments/main}
\input{patterns/06_return_results/main}
\input{patterns/061_pointers/main}
\input{patterns/065_GOTO/main}
\input{patterns/07_jcc/main}
\input{patterns/08_switch/main}
\input{patterns/09_loops/main}
\input{patterns/10_strings/main}
\input{patterns/11_arith_optimizations/main}
\input{patterns/12_FPU/main}
\input{patterns/13_arrays/main}
\input{patterns/14_bitfields/main}
\EN{\input{patterns/145_LCG/main_EN}}
\RU{\input{patterns/145_LCG/main_RU}}
\input{patterns/15_structs/main}
\input{patterns/17_unions/main}
\input{patterns/18_pointers_to_functions/main}
\input{patterns/185_64bit_in_32_env/main}

\EN{\input{patterns/19_SIMD/main_EN}}
\RU{\input{patterns/19_SIMD/main_RU}}
\DE{\input{patterns/19_SIMD/main_DE}}

\EN{\input{patterns/20_x64/main_EN}}
\RU{\input{patterns/20_x64/main_RU}}

\EN{\input{patterns/205_floating_SIMD/main_EN}}
\RU{\input{patterns/205_floating_SIMD/main_RU}}
\DE{\input{patterns/205_floating_SIMD/main_DE}}

\EN{\input{patterns/ARM/main_EN}}
\RU{\input{patterns/ARM/main_RU}}
\DE{\input{patterns/ARM/main_DE}}

\input{patterns/MIPS/main}


\EN{\section{Returning Values}
\label{ret_val_func}

Another simple function is the one that simply returns a constant value:

\lstinputlisting[caption=\EN{\CCpp Code},style=customc]{patterns/011_ret/1.c}

Let's compile it.

\subsection{x86}

Here's what both the GCC and MSVC compilers produce (with optimization) on the x86 platform:

\lstinputlisting[caption=\Optimizing GCC/MSVC (\assemblyOutput),style=customasmx86]{patterns/011_ret/1.s}

\myindex{x86!\Instructions!RET}
There are just two instructions: the first places the value 123 into the \EAX register,
which is used by convention for storing the return
value, and the second one is \RET, which returns execution to the \gls{caller}.

The caller will take the result from the \EAX register.

\subsection{ARM}

There are a few differences on the ARM platform:

\lstinputlisting[caption=\OptimizingKeilVI (\ARMMode) ASM Output,style=customasmARM]{patterns/011_ret/1_Keil_ARM_O3.s}

ARM uses the register \Reg{0} for returning the results of functions, so 123 is copied into \Reg{0}.

\myindex{ARM!\Instructions!MOV}
\myindex{x86!\Instructions!MOV}
It is worth noting that \MOV is a misleading name for the instruction in both the x86 and ARM \ac{ISA}s.

The data is not in fact \IT{moved}, but \IT{copied}.

\subsection{MIPS}

\label{MIPS_leaf_function_ex1}

The GCC assembly output below lists registers by number:

\lstinputlisting[caption=\Optimizing GCC 4.4.5 (\assemblyOutput),style=customasmMIPS]{patterns/011_ret/MIPS.s}

\dots while \IDA does it by their pseudo names:

\lstinputlisting[caption=\Optimizing GCC 4.4.5 (IDA),style=customasmMIPS]{patterns/011_ret/MIPS_IDA.lst}

The \$2 (or \$V0) register is used to store the function's return value.
\myindex{MIPS!\Pseudoinstructions!LI}
\INS{LI} stands for ``Load Immediate'' and is the MIPS equivalent to \MOV.

\myindex{MIPS!\Instructions!J}
The other instruction is the jump instruction (J or JR) which returns the execution flow to the \gls{caller}.

\myindex{MIPS!Branch delay slot}
You might be wondering why the positions of the load instruction (LI) and the jump instruction (J or JR) are swapped. This is due to a \ac{RISC} feature called ``branch delay slot''.

The reason this happens is a quirk in the architecture of some RISC \ac{ISA}s and isn't important for our
purposes---we must simply keep in mind that in MIPS, the instruction following a jump or branch instruction
is executed \IT{before} the jump/branch instruction itself.

As a consequence, branch instructions always swap places with the instruction executed immediately beforehand.


In practice, functions which merely return 1 (\IT{true}) or 0 (\IT{false}) are very frequent.

The smallest ever of the standard UNIX utilities, \IT{/bin/true} and \IT{/bin/false} return 0 and 1 respectively, as an exit code.
(Zero as an exit code usually means success, non-zero means error.)
}
\RU{\subsubsection{std::string}
\myindex{\Cpp!STL!std::string}
\label{std_string}

\myparagraph{Как устроена структура}

Многие строковые библиотеки \InSqBrackets{\CNotes 2.2} обеспечивают структуру содержащую ссылку 
на буфер собственно со строкой, переменная всегда содержащую длину строки 
(что очень удобно для массы функций \InSqBrackets{\CNotes 2.2.1}) и переменную содержащую текущий размер буфера.

Строка в буфере обыкновенно оканчивается нулем: это для того чтобы указатель на буфер можно было
передавать в функции требующие на вход обычную сишную \ac{ASCIIZ}-строку.

Стандарт \Cpp не описывает, как именно нужно реализовывать std::string,
но, как правило, они реализованы как описано выше, с небольшими дополнениями.

Строки в \Cpp это не класс (как, например, QString в Qt), а темплейт (basic\_string), 
это сделано для того чтобы поддерживать 
строки содержащие разного типа символы: как минимум \Tchar и \IT{wchar\_t}.

Так что, std::string это класс с базовым типом \Tchar.

А std::wstring это класс с базовым типом \IT{wchar\_t}.

\mysubparagraph{MSVC}

В реализации MSVC, вместо ссылки на буфер может содержаться сам буфер (если строка короче 16-и символов).

Это означает, что каждая короткая строка будет занимать в памяти по крайней мере $16 + 4 + 4 = 24$ 
байт для 32-битной среды либо $16 + 8 + 8 = 32$ 
байта в 64-битной, а если строка длиннее 16-и символов, то прибавьте еще длину самой строки.

\lstinputlisting[caption=пример для MSVC,style=customc]{\CURPATH/STL/string/MSVC_RU.cpp}

Собственно, из этого исходника почти всё ясно.

Несколько замечаний:

Если строка короче 16-и символов, 
то отдельный буфер для строки в \glslink{heap}{куче} выделяться не будет.

Это удобно потому что на практике, основная часть строк действительно короткие.
Вероятно, разработчики в Microsoft выбрали размер в 16 символов как разумный баланс.

Теперь очень важный момент в конце функции main(): мы не пользуемся методом c\_str(), тем не менее,
если это скомпилировать и запустить, то обе строки появятся в консоли!

Работает это вот почему.

В первом случае строка короче 16-и символов и в начале объекта std::string (его можно рассматривать
просто как структуру) расположен буфер с этой строкой.
\printf трактует указатель как указатель на массив символов оканчивающийся нулем и поэтому всё работает.

Вывод второй строки (длиннее 16-и символов) даже еще опаснее: это вообще типичная программистская ошибка 
(или опечатка), забыть дописать c\_str().
Это работает потому что в это время в начале структуры расположен указатель на буфер.
Это может надолго остаться незамеченным: до тех пока там не появится строка 
короче 16-и символов, тогда процесс упадет.

\mysubparagraph{GCC}

В реализации GCC в структуре есть еще одна переменная --- reference count.

Интересно, что указатель на экземпляр класса std::string в GCC указывает не на начало самой структуры, 
а на указатель на буфера.
В libstdc++-v3\textbackslash{}include\textbackslash{}bits\textbackslash{}basic\_string.h 
мы можем прочитать что это сделано для удобства отладки:

\begin{lstlisting}
   *  The reason you want _M_data pointing to the character %array and
   *  not the _Rep is so that the debugger can see the string
   *  contents. (Probably we should add a non-inline member to get
   *  the _Rep for the debugger to use, so users can check the actual
   *  string length.)
\end{lstlisting}

\href{http://go.yurichev.com/17085}{исходный код basic\_string.h}

В нашем примере мы учитываем это:

\lstinputlisting[caption=пример для GCC,style=customc]{\CURPATH/STL/string/GCC_RU.cpp}

Нужны еще небольшие хаки чтобы сымитировать типичную ошибку, которую мы уже видели выше, из-за
более ужесточенной проверки типов в GCC, тем не менее, printf() работает и здесь без c\_str().

\myparagraph{Чуть более сложный пример}

\lstinputlisting[style=customc]{\CURPATH/STL/string/3.cpp}

\lstinputlisting[caption=MSVC 2012,style=customasmx86]{\CURPATH/STL/string/3_MSVC_RU.asm}

Собственно, компилятор не конструирует строки статически: да в общем-то и как
это возможно, если буфер с ней нужно хранить в \glslink{heap}{куче}?

Вместо этого в сегменте данных хранятся обычные \ac{ASCIIZ}-строки, а позже, во время выполнения, 
при помощи метода \q{assign}, конструируются строки s1 и s2
.
При помощи \TT{operator+}, создается строка s3.

Обратите внимание на то что вызов метода c\_str() отсутствует,
потому что его код достаточно короткий и компилятор вставил его прямо здесь:
если строка короче 16-и байт, то в регистре EAX остается указатель на буфер,
а если длиннее, то из этого же места достается адрес на буфер расположенный в \glslink{heap}{куче}.

Далее следуют вызовы трех деструкторов, причем, они вызываются только если строка длиннее 16-и байт:
тогда нужно освободить буфера в \glslink{heap}{куче}.
В противном случае, так как все три объекта std::string хранятся в стеке,
они освобождаются автоматически после выхода из функции.

Следовательно, работа с короткими строками более быстрая из-за м\'{е}ньшего обращения к \glslink{heap}{куче}.

Код на GCC даже проще (из-за того, что в GCC, как мы уже видели, не реализована возможность хранить короткую
строку прямо в структуре):

% TODO1 comment each function meaning
\lstinputlisting[caption=GCC 4.8.1,style=customasmx86]{\CURPATH/STL/string/3_GCC_RU.s}

Можно заметить, что в деструкторы передается не указатель на объект,
а указатель на место за 12 байт (или 3 слова) перед ним, то есть, на настоящее начало структуры.

\myparagraph{std::string как глобальная переменная}
\label{sec:std_string_as_global_variable}

Опытные программисты на \Cpp знают, что глобальные переменные \ac{STL}-типов вполне можно объявлять.

Да, действительно:

\lstinputlisting[style=customc]{\CURPATH/STL/string/5.cpp}

Но как и где будет вызываться конструктор \TT{std::string}?

На самом деле, эта переменная будет инициализирована даже перед началом \main.

\lstinputlisting[caption=MSVC 2012: здесь конструируется глобальная переменная{,} а также регистрируется её деструктор,style=customasmx86]{\CURPATH/STL/string/5_MSVC_p2.asm}

\lstinputlisting[caption=MSVC 2012: здесь глобальная переменная используется в \main,style=customasmx86]{\CURPATH/STL/string/5_MSVC_p1.asm}

\lstinputlisting[caption=MSVC 2012: эта функция-деструктор вызывается перед выходом,style=customasmx86]{\CURPATH/STL/string/5_MSVC_p3.asm}

\myindex{\CStandardLibrary!atexit()}
В реальности, из \ac{CRT}, еще до вызова main(), вызывается специальная функция,
в которой перечислены все конструкторы подобных переменных.
Более того: при помощи atexit() регистрируется функция, которая будет вызвана в конце работы программы:
в этой функции компилятор собирает вызовы деструкторов всех подобных глобальных переменных.

GCC работает похожим образом:

\lstinputlisting[caption=GCC 4.8.1,style=customasmx86]{\CURPATH/STL/string/5_GCC.s}

Но он не выделяет отдельной функции в которой будут собраны деструкторы: 
каждый деструктор передается в atexit() по одному.

% TODO а если глобальная STL-переменная в другом модуле? надо проверить.

}
\DE{\subsection{Einfachste XOR-Verschlüsselung überhaupt}

Ich habe einmal eine Software gesehen, bei der alle Debugging-Ausgaben mit XOR mit dem Wert 3
verschlüsselt wurden. Mit anderen Worten, die beiden niedrigsten Bits aller Buchstaben wurden invertiert.

``Hello, world'' wurde zu ``Kfool/\#tlqog'':

\begin{lstlisting}
#!/usr/bin/python

msg="Hello, world!"

print "".join(map(lambda x: chr(ord(x)^3), msg))
\end{lstlisting}

Das ist eine ziemlich interessante Verschlüsselung (oder besser eine Verschleierung),
weil sie zwei wichtige Eigenschaften hat:
1) es ist eine einzige Funktion zum Verschlüsseln und entschlüsseln, sie muss nur wiederholt angewendet werden
2) die entstehenden Buchstaben befinden sich im druckbaren Bereich, also die ganze Zeichenkette kann ohne
Escape-Symbole im Code verwendet werden.

Die zweite Eigenschaft nutzt die Tatsache, dass alle druckbaren Zeichen in Reihen organisiert sind: 0x2x-0x7x,
und wenn die beiden niederwertigsten Bits invertiert werden, wird der Buchstabe um eine oder drei Stellen nach
links oder rechts \IT{verschoben}, aber niemals in eine andere Reihe:

\begin{figure}[H]
\centering
\includegraphics[width=0.7\textwidth]{ascii_clean.png}
\caption{7-Bit \ac{ASCII} Tabelle in Emacs}
\end{figure}

\dots mit dem Zeichen 0x7F als einziger Ausnahme.

Im Folgenden werden also beispielsweise die Zeichen A-Z \IT{verschlüsselt}:

\begin{lstlisting}
#!/usr/bin/python

msg="@ABCDEFGHIJKLMNO"

print "".join(map(lambda x: chr(ord(x)^3), msg))
\end{lstlisting}

Ergebnis:
% FIXME \verb  --  relevant comment for German?
\begin{lstlisting}
CBA@GFEDKJIHONML
\end{lstlisting}

Es sieht so aus als würden die Zeichen ``@'' und ``C'' sowie ``B'' und ``A'' vertauscht werden.

Hier ist noch ein interessantes Beispiel, in dem gezeigt wird, wie die Eigenschaften von XOR
ausgenutzt werden können: Exakt den gleichen Effekt, dass druckbare Zeichen auch druckbar bleiben,
kann man dadurch erzielen, dass irgendeine Kombination der niedrigsten vier Bits invertiert wird.
}

\EN{\section{Returning Values}
\label{ret_val_func}

Another simple function is the one that simply returns a constant value:

\lstinputlisting[caption=\EN{\CCpp Code},style=customc]{patterns/011_ret/1.c}

Let's compile it.

\subsection{x86}

Here's what both the GCC and MSVC compilers produce (with optimization) on the x86 platform:

\lstinputlisting[caption=\Optimizing GCC/MSVC (\assemblyOutput),style=customasmx86]{patterns/011_ret/1.s}

\myindex{x86!\Instructions!RET}
There are just two instructions: the first places the value 123 into the \EAX register,
which is used by convention for storing the return
value, and the second one is \RET, which returns execution to the \gls{caller}.

The caller will take the result from the \EAX register.

\subsection{ARM}

There are a few differences on the ARM platform:

\lstinputlisting[caption=\OptimizingKeilVI (\ARMMode) ASM Output,style=customasmARM]{patterns/011_ret/1_Keil_ARM_O3.s}

ARM uses the register \Reg{0} for returning the results of functions, so 123 is copied into \Reg{0}.

\myindex{ARM!\Instructions!MOV}
\myindex{x86!\Instructions!MOV}
It is worth noting that \MOV is a misleading name for the instruction in both the x86 and ARM \ac{ISA}s.

The data is not in fact \IT{moved}, but \IT{copied}.

\subsection{MIPS}

\label{MIPS_leaf_function_ex1}

The GCC assembly output below lists registers by number:

\lstinputlisting[caption=\Optimizing GCC 4.4.5 (\assemblyOutput),style=customasmMIPS]{patterns/011_ret/MIPS.s}

\dots while \IDA does it by their pseudo names:

\lstinputlisting[caption=\Optimizing GCC 4.4.5 (IDA),style=customasmMIPS]{patterns/011_ret/MIPS_IDA.lst}

The \$2 (or \$V0) register is used to store the function's return value.
\myindex{MIPS!\Pseudoinstructions!LI}
\INS{LI} stands for ``Load Immediate'' and is the MIPS equivalent to \MOV.

\myindex{MIPS!\Instructions!J}
The other instruction is the jump instruction (J or JR) which returns the execution flow to the \gls{caller}.

\myindex{MIPS!Branch delay slot}
You might be wondering why the positions of the load instruction (LI) and the jump instruction (J or JR) are swapped. This is due to a \ac{RISC} feature called ``branch delay slot''.

The reason this happens is a quirk in the architecture of some RISC \ac{ISA}s and isn't important for our
purposes---we must simply keep in mind that in MIPS, the instruction following a jump or branch instruction
is executed \IT{before} the jump/branch instruction itself.

As a consequence, branch instructions always swap places with the instruction executed immediately beforehand.


In practice, functions which merely return 1 (\IT{true}) or 0 (\IT{false}) are very frequent.

The smallest ever of the standard UNIX utilities, \IT{/bin/true} and \IT{/bin/false} return 0 and 1 respectively, as an exit code.
(Zero as an exit code usually means success, non-zero means error.)
}
\RU{\subsubsection{std::string}
\myindex{\Cpp!STL!std::string}
\label{std_string}

\myparagraph{Как устроена структура}

Многие строковые библиотеки \InSqBrackets{\CNotes 2.2} обеспечивают структуру содержащую ссылку 
на буфер собственно со строкой, переменная всегда содержащую длину строки 
(что очень удобно для массы функций \InSqBrackets{\CNotes 2.2.1}) и переменную содержащую текущий размер буфера.

Строка в буфере обыкновенно оканчивается нулем: это для того чтобы указатель на буфер можно было
передавать в функции требующие на вход обычную сишную \ac{ASCIIZ}-строку.

Стандарт \Cpp не описывает, как именно нужно реализовывать std::string,
но, как правило, они реализованы как описано выше, с небольшими дополнениями.

Строки в \Cpp это не класс (как, например, QString в Qt), а темплейт (basic\_string), 
это сделано для того чтобы поддерживать 
строки содержащие разного типа символы: как минимум \Tchar и \IT{wchar\_t}.

Так что, std::string это класс с базовым типом \Tchar.

А std::wstring это класс с базовым типом \IT{wchar\_t}.

\mysubparagraph{MSVC}

В реализации MSVC, вместо ссылки на буфер может содержаться сам буфер (если строка короче 16-и символов).

Это означает, что каждая короткая строка будет занимать в памяти по крайней мере $16 + 4 + 4 = 24$ 
байт для 32-битной среды либо $16 + 8 + 8 = 32$ 
байта в 64-битной, а если строка длиннее 16-и символов, то прибавьте еще длину самой строки.

\lstinputlisting[caption=пример для MSVC,style=customc]{\CURPATH/STL/string/MSVC_RU.cpp}

Собственно, из этого исходника почти всё ясно.

Несколько замечаний:

Если строка короче 16-и символов, 
то отдельный буфер для строки в \glslink{heap}{куче} выделяться не будет.

Это удобно потому что на практике, основная часть строк действительно короткие.
Вероятно, разработчики в Microsoft выбрали размер в 16 символов как разумный баланс.

Теперь очень важный момент в конце функции main(): мы не пользуемся методом c\_str(), тем не менее,
если это скомпилировать и запустить, то обе строки появятся в консоли!

Работает это вот почему.

В первом случае строка короче 16-и символов и в начале объекта std::string (его можно рассматривать
просто как структуру) расположен буфер с этой строкой.
\printf трактует указатель как указатель на массив символов оканчивающийся нулем и поэтому всё работает.

Вывод второй строки (длиннее 16-и символов) даже еще опаснее: это вообще типичная программистская ошибка 
(или опечатка), забыть дописать c\_str().
Это работает потому что в это время в начале структуры расположен указатель на буфер.
Это может надолго остаться незамеченным: до тех пока там не появится строка 
короче 16-и символов, тогда процесс упадет.

\mysubparagraph{GCC}

В реализации GCC в структуре есть еще одна переменная --- reference count.

Интересно, что указатель на экземпляр класса std::string в GCC указывает не на начало самой структуры, 
а на указатель на буфера.
В libstdc++-v3\textbackslash{}include\textbackslash{}bits\textbackslash{}basic\_string.h 
мы можем прочитать что это сделано для удобства отладки:

\begin{lstlisting}
   *  The reason you want _M_data pointing to the character %array and
   *  not the _Rep is so that the debugger can see the string
   *  contents. (Probably we should add a non-inline member to get
   *  the _Rep for the debugger to use, so users can check the actual
   *  string length.)
\end{lstlisting}

\href{http://go.yurichev.com/17085}{исходный код basic\_string.h}

В нашем примере мы учитываем это:

\lstinputlisting[caption=пример для GCC,style=customc]{\CURPATH/STL/string/GCC_RU.cpp}

Нужны еще небольшие хаки чтобы сымитировать типичную ошибку, которую мы уже видели выше, из-за
более ужесточенной проверки типов в GCC, тем не менее, printf() работает и здесь без c\_str().

\myparagraph{Чуть более сложный пример}

\lstinputlisting[style=customc]{\CURPATH/STL/string/3.cpp}

\lstinputlisting[caption=MSVC 2012,style=customasmx86]{\CURPATH/STL/string/3_MSVC_RU.asm}

Собственно, компилятор не конструирует строки статически: да в общем-то и как
это возможно, если буфер с ней нужно хранить в \glslink{heap}{куче}?

Вместо этого в сегменте данных хранятся обычные \ac{ASCIIZ}-строки, а позже, во время выполнения, 
при помощи метода \q{assign}, конструируются строки s1 и s2
.
При помощи \TT{operator+}, создается строка s3.

Обратите внимание на то что вызов метода c\_str() отсутствует,
потому что его код достаточно короткий и компилятор вставил его прямо здесь:
если строка короче 16-и байт, то в регистре EAX остается указатель на буфер,
а если длиннее, то из этого же места достается адрес на буфер расположенный в \glslink{heap}{куче}.

Далее следуют вызовы трех деструкторов, причем, они вызываются только если строка длиннее 16-и байт:
тогда нужно освободить буфера в \glslink{heap}{куче}.
В противном случае, так как все три объекта std::string хранятся в стеке,
они освобождаются автоматически после выхода из функции.

Следовательно, работа с короткими строками более быстрая из-за м\'{е}ньшего обращения к \glslink{heap}{куче}.

Код на GCC даже проще (из-за того, что в GCC, как мы уже видели, не реализована возможность хранить короткую
строку прямо в структуре):

% TODO1 comment each function meaning
\lstinputlisting[caption=GCC 4.8.1,style=customasmx86]{\CURPATH/STL/string/3_GCC_RU.s}

Можно заметить, что в деструкторы передается не указатель на объект,
а указатель на место за 12 байт (или 3 слова) перед ним, то есть, на настоящее начало структуры.

\myparagraph{std::string как глобальная переменная}
\label{sec:std_string_as_global_variable}

Опытные программисты на \Cpp знают, что глобальные переменные \ac{STL}-типов вполне можно объявлять.

Да, действительно:

\lstinputlisting[style=customc]{\CURPATH/STL/string/5.cpp}

Но как и где будет вызываться конструктор \TT{std::string}?

На самом деле, эта переменная будет инициализирована даже перед началом \main.

\lstinputlisting[caption=MSVC 2012: здесь конструируется глобальная переменная{,} а также регистрируется её деструктор,style=customasmx86]{\CURPATH/STL/string/5_MSVC_p2.asm}

\lstinputlisting[caption=MSVC 2012: здесь глобальная переменная используется в \main,style=customasmx86]{\CURPATH/STL/string/5_MSVC_p1.asm}

\lstinputlisting[caption=MSVC 2012: эта функция-деструктор вызывается перед выходом,style=customasmx86]{\CURPATH/STL/string/5_MSVC_p3.asm}

\myindex{\CStandardLibrary!atexit()}
В реальности, из \ac{CRT}, еще до вызова main(), вызывается специальная функция,
в которой перечислены все конструкторы подобных переменных.
Более того: при помощи atexit() регистрируется функция, которая будет вызвана в конце работы программы:
в этой функции компилятор собирает вызовы деструкторов всех подобных глобальных переменных.

GCC работает похожим образом:

\lstinputlisting[caption=GCC 4.8.1,style=customasmx86]{\CURPATH/STL/string/5_GCC.s}

Но он не выделяет отдельной функции в которой будут собраны деструкторы: 
каждый деструктор передается в atexit() по одному.

% TODO а если глобальная STL-переменная в другом модуле? надо проверить.

}

\EN{\section{Returning Values}
\label{ret_val_func}

Another simple function is the one that simply returns a constant value:

\lstinputlisting[caption=\EN{\CCpp Code},style=customc]{patterns/011_ret/1.c}

Let's compile it.

\subsection{x86}

Here's what both the GCC and MSVC compilers produce (with optimization) on the x86 platform:

\lstinputlisting[caption=\Optimizing GCC/MSVC (\assemblyOutput),style=customasmx86]{patterns/011_ret/1.s}

\myindex{x86!\Instructions!RET}
There are just two instructions: the first places the value 123 into the \EAX register,
which is used by convention for storing the return
value, and the second one is \RET, which returns execution to the \gls{caller}.

The caller will take the result from the \EAX register.

\subsection{ARM}

There are a few differences on the ARM platform:

\lstinputlisting[caption=\OptimizingKeilVI (\ARMMode) ASM Output,style=customasmARM]{patterns/011_ret/1_Keil_ARM_O3.s}

ARM uses the register \Reg{0} for returning the results of functions, so 123 is copied into \Reg{0}.

\myindex{ARM!\Instructions!MOV}
\myindex{x86!\Instructions!MOV}
It is worth noting that \MOV is a misleading name for the instruction in both the x86 and ARM \ac{ISA}s.

The data is not in fact \IT{moved}, but \IT{copied}.

\subsection{MIPS}

\label{MIPS_leaf_function_ex1}

The GCC assembly output below lists registers by number:

\lstinputlisting[caption=\Optimizing GCC 4.4.5 (\assemblyOutput),style=customasmMIPS]{patterns/011_ret/MIPS.s}

\dots while \IDA does it by their pseudo names:

\lstinputlisting[caption=\Optimizing GCC 4.4.5 (IDA),style=customasmMIPS]{patterns/011_ret/MIPS_IDA.lst}

The \$2 (or \$V0) register is used to store the function's return value.
\myindex{MIPS!\Pseudoinstructions!LI}
\INS{LI} stands for ``Load Immediate'' and is the MIPS equivalent to \MOV.

\myindex{MIPS!\Instructions!J}
The other instruction is the jump instruction (J or JR) which returns the execution flow to the \gls{caller}.

\myindex{MIPS!Branch delay slot}
You might be wondering why the positions of the load instruction (LI) and the jump instruction (J or JR) are swapped. This is due to a \ac{RISC} feature called ``branch delay slot''.

The reason this happens is a quirk in the architecture of some RISC \ac{ISA}s and isn't important for our
purposes---we must simply keep in mind that in MIPS, the instruction following a jump or branch instruction
is executed \IT{before} the jump/branch instruction itself.

As a consequence, branch instructions always swap places with the instruction executed immediately beforehand.


In practice, functions which merely return 1 (\IT{true}) or 0 (\IT{false}) are very frequent.

The smallest ever of the standard UNIX utilities, \IT{/bin/true} and \IT{/bin/false} return 0 and 1 respectively, as an exit code.
(Zero as an exit code usually means success, non-zero means error.)
}
\RU{\subsubsection{std::string}
\myindex{\Cpp!STL!std::string}
\label{std_string}

\myparagraph{Как устроена структура}

Многие строковые библиотеки \InSqBrackets{\CNotes 2.2} обеспечивают структуру содержащую ссылку 
на буфер собственно со строкой, переменная всегда содержащую длину строки 
(что очень удобно для массы функций \InSqBrackets{\CNotes 2.2.1}) и переменную содержащую текущий размер буфера.

Строка в буфере обыкновенно оканчивается нулем: это для того чтобы указатель на буфер можно было
передавать в функции требующие на вход обычную сишную \ac{ASCIIZ}-строку.

Стандарт \Cpp не описывает, как именно нужно реализовывать std::string,
но, как правило, они реализованы как описано выше, с небольшими дополнениями.

Строки в \Cpp это не класс (как, например, QString в Qt), а темплейт (basic\_string), 
это сделано для того чтобы поддерживать 
строки содержащие разного типа символы: как минимум \Tchar и \IT{wchar\_t}.

Так что, std::string это класс с базовым типом \Tchar.

А std::wstring это класс с базовым типом \IT{wchar\_t}.

\mysubparagraph{MSVC}

В реализации MSVC, вместо ссылки на буфер может содержаться сам буфер (если строка короче 16-и символов).

Это означает, что каждая короткая строка будет занимать в памяти по крайней мере $16 + 4 + 4 = 24$ 
байт для 32-битной среды либо $16 + 8 + 8 = 32$ 
байта в 64-битной, а если строка длиннее 16-и символов, то прибавьте еще длину самой строки.

\lstinputlisting[caption=пример для MSVC,style=customc]{\CURPATH/STL/string/MSVC_RU.cpp}

Собственно, из этого исходника почти всё ясно.

Несколько замечаний:

Если строка короче 16-и символов, 
то отдельный буфер для строки в \glslink{heap}{куче} выделяться не будет.

Это удобно потому что на практике, основная часть строк действительно короткие.
Вероятно, разработчики в Microsoft выбрали размер в 16 символов как разумный баланс.

Теперь очень важный момент в конце функции main(): мы не пользуемся методом c\_str(), тем не менее,
если это скомпилировать и запустить, то обе строки появятся в консоли!

Работает это вот почему.

В первом случае строка короче 16-и символов и в начале объекта std::string (его можно рассматривать
просто как структуру) расположен буфер с этой строкой.
\printf трактует указатель как указатель на массив символов оканчивающийся нулем и поэтому всё работает.

Вывод второй строки (длиннее 16-и символов) даже еще опаснее: это вообще типичная программистская ошибка 
(или опечатка), забыть дописать c\_str().
Это работает потому что в это время в начале структуры расположен указатель на буфер.
Это может надолго остаться незамеченным: до тех пока там не появится строка 
короче 16-и символов, тогда процесс упадет.

\mysubparagraph{GCC}

В реализации GCC в структуре есть еще одна переменная --- reference count.

Интересно, что указатель на экземпляр класса std::string в GCC указывает не на начало самой структуры, 
а на указатель на буфера.
В libstdc++-v3\textbackslash{}include\textbackslash{}bits\textbackslash{}basic\_string.h 
мы можем прочитать что это сделано для удобства отладки:

\begin{lstlisting}
   *  The reason you want _M_data pointing to the character %array and
   *  not the _Rep is so that the debugger can see the string
   *  contents. (Probably we should add a non-inline member to get
   *  the _Rep for the debugger to use, so users can check the actual
   *  string length.)
\end{lstlisting}

\href{http://go.yurichev.com/17085}{исходный код basic\_string.h}

В нашем примере мы учитываем это:

\lstinputlisting[caption=пример для GCC,style=customc]{\CURPATH/STL/string/GCC_RU.cpp}

Нужны еще небольшие хаки чтобы сымитировать типичную ошибку, которую мы уже видели выше, из-за
более ужесточенной проверки типов в GCC, тем не менее, printf() работает и здесь без c\_str().

\myparagraph{Чуть более сложный пример}

\lstinputlisting[style=customc]{\CURPATH/STL/string/3.cpp}

\lstinputlisting[caption=MSVC 2012,style=customasmx86]{\CURPATH/STL/string/3_MSVC_RU.asm}

Собственно, компилятор не конструирует строки статически: да в общем-то и как
это возможно, если буфер с ней нужно хранить в \glslink{heap}{куче}?

Вместо этого в сегменте данных хранятся обычные \ac{ASCIIZ}-строки, а позже, во время выполнения, 
при помощи метода \q{assign}, конструируются строки s1 и s2
.
При помощи \TT{operator+}, создается строка s3.

Обратите внимание на то что вызов метода c\_str() отсутствует,
потому что его код достаточно короткий и компилятор вставил его прямо здесь:
если строка короче 16-и байт, то в регистре EAX остается указатель на буфер,
а если длиннее, то из этого же места достается адрес на буфер расположенный в \glslink{heap}{куче}.

Далее следуют вызовы трех деструкторов, причем, они вызываются только если строка длиннее 16-и байт:
тогда нужно освободить буфера в \glslink{heap}{куче}.
В противном случае, так как все три объекта std::string хранятся в стеке,
они освобождаются автоматически после выхода из функции.

Следовательно, работа с короткими строками более быстрая из-за м\'{е}ньшего обращения к \glslink{heap}{куче}.

Код на GCC даже проще (из-за того, что в GCC, как мы уже видели, не реализована возможность хранить короткую
строку прямо в структуре):

% TODO1 comment each function meaning
\lstinputlisting[caption=GCC 4.8.1,style=customasmx86]{\CURPATH/STL/string/3_GCC_RU.s}

Можно заметить, что в деструкторы передается не указатель на объект,
а указатель на место за 12 байт (или 3 слова) перед ним, то есть, на настоящее начало структуры.

\myparagraph{std::string как глобальная переменная}
\label{sec:std_string_as_global_variable}

Опытные программисты на \Cpp знают, что глобальные переменные \ac{STL}-типов вполне можно объявлять.

Да, действительно:

\lstinputlisting[style=customc]{\CURPATH/STL/string/5.cpp}

Но как и где будет вызываться конструктор \TT{std::string}?

На самом деле, эта переменная будет инициализирована даже перед началом \main.

\lstinputlisting[caption=MSVC 2012: здесь конструируется глобальная переменная{,} а также регистрируется её деструктор,style=customasmx86]{\CURPATH/STL/string/5_MSVC_p2.asm}

\lstinputlisting[caption=MSVC 2012: здесь глобальная переменная используется в \main,style=customasmx86]{\CURPATH/STL/string/5_MSVC_p1.asm}

\lstinputlisting[caption=MSVC 2012: эта функция-деструктор вызывается перед выходом,style=customasmx86]{\CURPATH/STL/string/5_MSVC_p3.asm}

\myindex{\CStandardLibrary!atexit()}
В реальности, из \ac{CRT}, еще до вызова main(), вызывается специальная функция,
в которой перечислены все конструкторы подобных переменных.
Более того: при помощи atexit() регистрируется функция, которая будет вызвана в конце работы программы:
в этой функции компилятор собирает вызовы деструкторов всех подобных глобальных переменных.

GCC работает похожим образом:

\lstinputlisting[caption=GCC 4.8.1,style=customasmx86]{\CURPATH/STL/string/5_GCC.s}

Но он не выделяет отдельной функции в которой будут собраны деструкторы: 
каждый деструктор передается в atexit() по одному.

% TODO а если глобальная STL-переменная в другом модуле? надо проверить.

}
\DE{\subsection{Einfachste XOR-Verschlüsselung überhaupt}

Ich habe einmal eine Software gesehen, bei der alle Debugging-Ausgaben mit XOR mit dem Wert 3
verschlüsselt wurden. Mit anderen Worten, die beiden niedrigsten Bits aller Buchstaben wurden invertiert.

``Hello, world'' wurde zu ``Kfool/\#tlqog'':

\begin{lstlisting}
#!/usr/bin/python

msg="Hello, world!"

print "".join(map(lambda x: chr(ord(x)^3), msg))
\end{lstlisting}

Das ist eine ziemlich interessante Verschlüsselung (oder besser eine Verschleierung),
weil sie zwei wichtige Eigenschaften hat:
1) es ist eine einzige Funktion zum Verschlüsseln und entschlüsseln, sie muss nur wiederholt angewendet werden
2) die entstehenden Buchstaben befinden sich im druckbaren Bereich, also die ganze Zeichenkette kann ohne
Escape-Symbole im Code verwendet werden.

Die zweite Eigenschaft nutzt die Tatsache, dass alle druckbaren Zeichen in Reihen organisiert sind: 0x2x-0x7x,
und wenn die beiden niederwertigsten Bits invertiert werden, wird der Buchstabe um eine oder drei Stellen nach
links oder rechts \IT{verschoben}, aber niemals in eine andere Reihe:

\begin{figure}[H]
\centering
\includegraphics[width=0.7\textwidth]{ascii_clean.png}
\caption{7-Bit \ac{ASCII} Tabelle in Emacs}
\end{figure}

\dots mit dem Zeichen 0x7F als einziger Ausnahme.

Im Folgenden werden also beispielsweise die Zeichen A-Z \IT{verschlüsselt}:

\begin{lstlisting}
#!/usr/bin/python

msg="@ABCDEFGHIJKLMNO"

print "".join(map(lambda x: chr(ord(x)^3), msg))
\end{lstlisting}

Ergebnis:
% FIXME \verb  --  relevant comment for German?
\begin{lstlisting}
CBA@GFEDKJIHONML
\end{lstlisting}

Es sieht so aus als würden die Zeichen ``@'' und ``C'' sowie ``B'' und ``A'' vertauscht werden.

Hier ist noch ein interessantes Beispiel, in dem gezeigt wird, wie die Eigenschaften von XOR
ausgenutzt werden können: Exakt den gleichen Effekt, dass druckbare Zeichen auch druckbar bleiben,
kann man dadurch erzielen, dass irgendeine Kombination der niedrigsten vier Bits invertiert wird.
}

\EN{\section{Returning Values}
\label{ret_val_func}

Another simple function is the one that simply returns a constant value:

\lstinputlisting[caption=\EN{\CCpp Code},style=customc]{patterns/011_ret/1.c}

Let's compile it.

\subsection{x86}

Here's what both the GCC and MSVC compilers produce (with optimization) on the x86 platform:

\lstinputlisting[caption=\Optimizing GCC/MSVC (\assemblyOutput),style=customasmx86]{patterns/011_ret/1.s}

\myindex{x86!\Instructions!RET}
There are just two instructions: the first places the value 123 into the \EAX register,
which is used by convention for storing the return
value, and the second one is \RET, which returns execution to the \gls{caller}.

The caller will take the result from the \EAX register.

\subsection{ARM}

There are a few differences on the ARM platform:

\lstinputlisting[caption=\OptimizingKeilVI (\ARMMode) ASM Output,style=customasmARM]{patterns/011_ret/1_Keil_ARM_O3.s}

ARM uses the register \Reg{0} for returning the results of functions, so 123 is copied into \Reg{0}.

\myindex{ARM!\Instructions!MOV}
\myindex{x86!\Instructions!MOV}
It is worth noting that \MOV is a misleading name for the instruction in both the x86 and ARM \ac{ISA}s.

The data is not in fact \IT{moved}, but \IT{copied}.

\subsection{MIPS}

\label{MIPS_leaf_function_ex1}

The GCC assembly output below lists registers by number:

\lstinputlisting[caption=\Optimizing GCC 4.4.5 (\assemblyOutput),style=customasmMIPS]{patterns/011_ret/MIPS.s}

\dots while \IDA does it by their pseudo names:

\lstinputlisting[caption=\Optimizing GCC 4.4.5 (IDA),style=customasmMIPS]{patterns/011_ret/MIPS_IDA.lst}

The \$2 (or \$V0) register is used to store the function's return value.
\myindex{MIPS!\Pseudoinstructions!LI}
\INS{LI} stands for ``Load Immediate'' and is the MIPS equivalent to \MOV.

\myindex{MIPS!\Instructions!J}
The other instruction is the jump instruction (J or JR) which returns the execution flow to the \gls{caller}.

\myindex{MIPS!Branch delay slot}
You might be wondering why the positions of the load instruction (LI) and the jump instruction (J or JR) are swapped. This is due to a \ac{RISC} feature called ``branch delay slot''.

The reason this happens is a quirk in the architecture of some RISC \ac{ISA}s and isn't important for our
purposes---we must simply keep in mind that in MIPS, the instruction following a jump or branch instruction
is executed \IT{before} the jump/branch instruction itself.

As a consequence, branch instructions always swap places with the instruction executed immediately beforehand.


In practice, functions which merely return 1 (\IT{true}) or 0 (\IT{false}) are very frequent.

The smallest ever of the standard UNIX utilities, \IT{/bin/true} and \IT{/bin/false} return 0 and 1 respectively, as an exit code.
(Zero as an exit code usually means success, non-zero means error.)
}
\RU{\subsubsection{std::string}
\myindex{\Cpp!STL!std::string}
\label{std_string}

\myparagraph{Как устроена структура}

Многие строковые библиотеки \InSqBrackets{\CNotes 2.2} обеспечивают структуру содержащую ссылку 
на буфер собственно со строкой, переменная всегда содержащую длину строки 
(что очень удобно для массы функций \InSqBrackets{\CNotes 2.2.1}) и переменную содержащую текущий размер буфера.

Строка в буфере обыкновенно оканчивается нулем: это для того чтобы указатель на буфер можно было
передавать в функции требующие на вход обычную сишную \ac{ASCIIZ}-строку.

Стандарт \Cpp не описывает, как именно нужно реализовывать std::string,
но, как правило, они реализованы как описано выше, с небольшими дополнениями.

Строки в \Cpp это не класс (как, например, QString в Qt), а темплейт (basic\_string), 
это сделано для того чтобы поддерживать 
строки содержащие разного типа символы: как минимум \Tchar и \IT{wchar\_t}.

Так что, std::string это класс с базовым типом \Tchar.

А std::wstring это класс с базовым типом \IT{wchar\_t}.

\mysubparagraph{MSVC}

В реализации MSVC, вместо ссылки на буфер может содержаться сам буфер (если строка короче 16-и символов).

Это означает, что каждая короткая строка будет занимать в памяти по крайней мере $16 + 4 + 4 = 24$ 
байт для 32-битной среды либо $16 + 8 + 8 = 32$ 
байта в 64-битной, а если строка длиннее 16-и символов, то прибавьте еще длину самой строки.

\lstinputlisting[caption=пример для MSVC,style=customc]{\CURPATH/STL/string/MSVC_RU.cpp}

Собственно, из этого исходника почти всё ясно.

Несколько замечаний:

Если строка короче 16-и символов, 
то отдельный буфер для строки в \glslink{heap}{куче} выделяться не будет.

Это удобно потому что на практике, основная часть строк действительно короткие.
Вероятно, разработчики в Microsoft выбрали размер в 16 символов как разумный баланс.

Теперь очень важный момент в конце функции main(): мы не пользуемся методом c\_str(), тем не менее,
если это скомпилировать и запустить, то обе строки появятся в консоли!

Работает это вот почему.

В первом случае строка короче 16-и символов и в начале объекта std::string (его можно рассматривать
просто как структуру) расположен буфер с этой строкой.
\printf трактует указатель как указатель на массив символов оканчивающийся нулем и поэтому всё работает.

Вывод второй строки (длиннее 16-и символов) даже еще опаснее: это вообще типичная программистская ошибка 
(или опечатка), забыть дописать c\_str().
Это работает потому что в это время в начале структуры расположен указатель на буфер.
Это может надолго остаться незамеченным: до тех пока там не появится строка 
короче 16-и символов, тогда процесс упадет.

\mysubparagraph{GCC}

В реализации GCC в структуре есть еще одна переменная --- reference count.

Интересно, что указатель на экземпляр класса std::string в GCC указывает не на начало самой структуры, 
а на указатель на буфера.
В libstdc++-v3\textbackslash{}include\textbackslash{}bits\textbackslash{}basic\_string.h 
мы можем прочитать что это сделано для удобства отладки:

\begin{lstlisting}
   *  The reason you want _M_data pointing to the character %array and
   *  not the _Rep is so that the debugger can see the string
   *  contents. (Probably we should add a non-inline member to get
   *  the _Rep for the debugger to use, so users can check the actual
   *  string length.)
\end{lstlisting}

\href{http://go.yurichev.com/17085}{исходный код basic\_string.h}

В нашем примере мы учитываем это:

\lstinputlisting[caption=пример для GCC,style=customc]{\CURPATH/STL/string/GCC_RU.cpp}

Нужны еще небольшие хаки чтобы сымитировать типичную ошибку, которую мы уже видели выше, из-за
более ужесточенной проверки типов в GCC, тем не менее, printf() работает и здесь без c\_str().

\myparagraph{Чуть более сложный пример}

\lstinputlisting[style=customc]{\CURPATH/STL/string/3.cpp}

\lstinputlisting[caption=MSVC 2012,style=customasmx86]{\CURPATH/STL/string/3_MSVC_RU.asm}

Собственно, компилятор не конструирует строки статически: да в общем-то и как
это возможно, если буфер с ней нужно хранить в \glslink{heap}{куче}?

Вместо этого в сегменте данных хранятся обычные \ac{ASCIIZ}-строки, а позже, во время выполнения, 
при помощи метода \q{assign}, конструируются строки s1 и s2
.
При помощи \TT{operator+}, создается строка s3.

Обратите внимание на то что вызов метода c\_str() отсутствует,
потому что его код достаточно короткий и компилятор вставил его прямо здесь:
если строка короче 16-и байт, то в регистре EAX остается указатель на буфер,
а если длиннее, то из этого же места достается адрес на буфер расположенный в \glslink{heap}{куче}.

Далее следуют вызовы трех деструкторов, причем, они вызываются только если строка длиннее 16-и байт:
тогда нужно освободить буфера в \glslink{heap}{куче}.
В противном случае, так как все три объекта std::string хранятся в стеке,
они освобождаются автоматически после выхода из функции.

Следовательно, работа с короткими строками более быстрая из-за м\'{е}ньшего обращения к \glslink{heap}{куче}.

Код на GCC даже проще (из-за того, что в GCC, как мы уже видели, не реализована возможность хранить короткую
строку прямо в структуре):

% TODO1 comment each function meaning
\lstinputlisting[caption=GCC 4.8.1,style=customasmx86]{\CURPATH/STL/string/3_GCC_RU.s}

Можно заметить, что в деструкторы передается не указатель на объект,
а указатель на место за 12 байт (или 3 слова) перед ним, то есть, на настоящее начало структуры.

\myparagraph{std::string как глобальная переменная}
\label{sec:std_string_as_global_variable}

Опытные программисты на \Cpp знают, что глобальные переменные \ac{STL}-типов вполне можно объявлять.

Да, действительно:

\lstinputlisting[style=customc]{\CURPATH/STL/string/5.cpp}

Но как и где будет вызываться конструктор \TT{std::string}?

На самом деле, эта переменная будет инициализирована даже перед началом \main.

\lstinputlisting[caption=MSVC 2012: здесь конструируется глобальная переменная{,} а также регистрируется её деструктор,style=customasmx86]{\CURPATH/STL/string/5_MSVC_p2.asm}

\lstinputlisting[caption=MSVC 2012: здесь глобальная переменная используется в \main,style=customasmx86]{\CURPATH/STL/string/5_MSVC_p1.asm}

\lstinputlisting[caption=MSVC 2012: эта функция-деструктор вызывается перед выходом,style=customasmx86]{\CURPATH/STL/string/5_MSVC_p3.asm}

\myindex{\CStandardLibrary!atexit()}
В реальности, из \ac{CRT}, еще до вызова main(), вызывается специальная функция,
в которой перечислены все конструкторы подобных переменных.
Более того: при помощи atexit() регистрируется функция, которая будет вызвана в конце работы программы:
в этой функции компилятор собирает вызовы деструкторов всех подобных глобальных переменных.

GCC работает похожим образом:

\lstinputlisting[caption=GCC 4.8.1,style=customasmx86]{\CURPATH/STL/string/5_GCC.s}

Но он не выделяет отдельной функции в которой будут собраны деструкторы: 
каждый деструктор передается в atexit() по одному.

% TODO а если глобальная STL-переменная в другом модуле? надо проверить.

}
\DE{\subsection{Einfachste XOR-Verschlüsselung überhaupt}

Ich habe einmal eine Software gesehen, bei der alle Debugging-Ausgaben mit XOR mit dem Wert 3
verschlüsselt wurden. Mit anderen Worten, die beiden niedrigsten Bits aller Buchstaben wurden invertiert.

``Hello, world'' wurde zu ``Kfool/\#tlqog'':

\begin{lstlisting}
#!/usr/bin/python

msg="Hello, world!"

print "".join(map(lambda x: chr(ord(x)^3), msg))
\end{lstlisting}

Das ist eine ziemlich interessante Verschlüsselung (oder besser eine Verschleierung),
weil sie zwei wichtige Eigenschaften hat:
1) es ist eine einzige Funktion zum Verschlüsseln und entschlüsseln, sie muss nur wiederholt angewendet werden
2) die entstehenden Buchstaben befinden sich im druckbaren Bereich, also die ganze Zeichenkette kann ohne
Escape-Symbole im Code verwendet werden.

Die zweite Eigenschaft nutzt die Tatsache, dass alle druckbaren Zeichen in Reihen organisiert sind: 0x2x-0x7x,
und wenn die beiden niederwertigsten Bits invertiert werden, wird der Buchstabe um eine oder drei Stellen nach
links oder rechts \IT{verschoben}, aber niemals in eine andere Reihe:

\begin{figure}[H]
\centering
\includegraphics[width=0.7\textwidth]{ascii_clean.png}
\caption{7-Bit \ac{ASCII} Tabelle in Emacs}
\end{figure}

\dots mit dem Zeichen 0x7F als einziger Ausnahme.

Im Folgenden werden also beispielsweise die Zeichen A-Z \IT{verschlüsselt}:

\begin{lstlisting}
#!/usr/bin/python

msg="@ABCDEFGHIJKLMNO"

print "".join(map(lambda x: chr(ord(x)^3), msg))
\end{lstlisting}

Ergebnis:
% FIXME \verb  --  relevant comment for German?
\begin{lstlisting}
CBA@GFEDKJIHONML
\end{lstlisting}

Es sieht so aus als würden die Zeichen ``@'' und ``C'' sowie ``B'' und ``A'' vertauscht werden.

Hier ist noch ein interessantes Beispiel, in dem gezeigt wird, wie die Eigenschaften von XOR
ausgenutzt werden können: Exakt den gleichen Effekt, dass druckbare Zeichen auch druckbar bleiben,
kann man dadurch erzielen, dass irgendeine Kombination der niedrigsten vier Bits invertiert wird.
}

\ifdefined\SPANISH
\chapter{Patrones de código}
\fi % SPANISH

\ifdefined\GERMAN
\chapter{Code-Muster}
\fi % GERMAN

\ifdefined\ENGLISH
\chapter{Code Patterns}
\fi % ENGLISH

\ifdefined\ITALIAN
\chapter{Forme di codice}
\fi % ITALIAN

\ifdefined\RUSSIAN
\chapter{Образцы кода}
\fi % RUSSIAN

\ifdefined\BRAZILIAN
\chapter{Padrões de códigos}
\fi % BRAZILIAN

\ifdefined\THAI
\chapter{รูปแบบของโค้ด}
\fi % THAI

\ifdefined\FRENCH
\chapter{Modèle de code}
\fi % FRENCH

\ifdefined\POLISH
\chapter{\PLph{}}
\fi % POLISH

% sections
\EN{\input{patterns/patterns_opt_dbg_EN}}
\ES{\input{patterns/patterns_opt_dbg_ES}}
\ITA{\input{patterns/patterns_opt_dbg_ITA}}
\PTBR{\input{patterns/patterns_opt_dbg_PTBR}}
\RU{\input{patterns/patterns_opt_dbg_RU}}
\THA{\input{patterns/patterns_opt_dbg_THA}}
\DE{\input{patterns/patterns_opt_dbg_DE}}
\FR{\input{patterns/patterns_opt_dbg_FR}}
\PL{\input{patterns/patterns_opt_dbg_PL}}

\RU{\section{Некоторые базовые понятия}}
\EN{\section{Some basics}}
\DE{\section{Einige Grundlagen}}
\FR{\section{Quelques bases}}
\ES{\section{\ESph{}}}
\ITA{\section{Alcune basi teoriche}}
\PTBR{\section{\PTBRph{}}}
\THA{\section{\THAph{}}}
\PL{\section{\PLph{}}}

% sections:
\EN{\input{patterns/intro_CPU_ISA_EN}}
\ES{\input{patterns/intro_CPU_ISA_ES}}
\ITA{\input{patterns/intro_CPU_ISA_ITA}}
\PTBR{\input{patterns/intro_CPU_ISA_PTBR}}
\RU{\input{patterns/intro_CPU_ISA_RU}}
\DE{\input{patterns/intro_CPU_ISA_DE}}
\FR{\input{patterns/intro_CPU_ISA_FR}}
\PL{\input{patterns/intro_CPU_ISA_PL}}

\EN{\input{patterns/numeral_EN}}
\RU{\input{patterns/numeral_RU}}
\ITA{\input{patterns/numeral_ITA}}
\DE{\input{patterns/numeral_DE}}
\FR{\input{patterns/numeral_FR}}
\PL{\input{patterns/numeral_PL}}

% chapters
\input{patterns/00_empty/main}
\input{patterns/011_ret/main}
\input{patterns/01_helloworld/main}
\input{patterns/015_prolog_epilogue/main}
\input{patterns/02_stack/main}
\input{patterns/03_printf/main}
\input{patterns/04_scanf/main}
\input{patterns/05_passing_arguments/main}
\input{patterns/06_return_results/main}
\input{patterns/061_pointers/main}
\input{patterns/065_GOTO/main}
\input{patterns/07_jcc/main}
\input{patterns/08_switch/main}
\input{patterns/09_loops/main}
\input{patterns/10_strings/main}
\input{patterns/11_arith_optimizations/main}
\input{patterns/12_FPU/main}
\input{patterns/13_arrays/main}
\input{patterns/14_bitfields/main}
\EN{\input{patterns/145_LCG/main_EN}}
\RU{\input{patterns/145_LCG/main_RU}}
\input{patterns/15_structs/main}
\input{patterns/17_unions/main}
\input{patterns/18_pointers_to_functions/main}
\input{patterns/185_64bit_in_32_env/main}

\EN{\input{patterns/19_SIMD/main_EN}}
\RU{\input{patterns/19_SIMD/main_RU}}
\DE{\input{patterns/19_SIMD/main_DE}}

\EN{\input{patterns/20_x64/main_EN}}
\RU{\input{patterns/20_x64/main_RU}}

\EN{\input{patterns/205_floating_SIMD/main_EN}}
\RU{\input{patterns/205_floating_SIMD/main_RU}}
\DE{\input{patterns/205_floating_SIMD/main_DE}}

\EN{\input{patterns/ARM/main_EN}}
\RU{\input{patterns/ARM/main_RU}}
\DE{\input{patterns/ARM/main_DE}}

\input{patterns/MIPS/main}


\ifdefined\SPANISH
\chapter{Patrones de código}
\fi % SPANISH

\ifdefined\GERMAN
\chapter{Code-Muster}
\fi % GERMAN

\ifdefined\ENGLISH
\chapter{Code Patterns}
\fi % ENGLISH

\ifdefined\ITALIAN
\chapter{Forme di codice}
\fi % ITALIAN

\ifdefined\RUSSIAN
\chapter{Образцы кода}
\fi % RUSSIAN

\ifdefined\BRAZILIAN
\chapter{Padrões de códigos}
\fi % BRAZILIAN

\ifdefined\THAI
\chapter{รูปแบบของโค้ด}
\fi % THAI

\ifdefined\FRENCH
\chapter{Modèle de code}
\fi % FRENCH

\ifdefined\POLISH
\chapter{\PLph{}}
\fi % POLISH

% sections
\EN{\section{The method}

When the author of this book first started learning C and, later, \Cpp, he used to write small pieces of code, compile them,
and then look at the assembly language output. This made it very easy for him to understand what was going on in the code that he had written.
\footnote{In fact, he still does this when he can't understand what a particular bit of code does.}.
He did this so many times that the relationship between the \CCpp code and what the compiler produced was imprinted deeply in his mind.
It's now easy for him to imagine instantly a rough outline of a C code's appearance and function.
Perhaps this technique could be helpful for others.

%There are a lot of examples for both x86/x64 and ARM.
%Those who already familiar with one of architectures, may freely skim over pages.

By the way, there is a great website where you can do the same, with various compilers, instead of installing them on your box.
You can use it as well: \url{https://gcc.godbolt.org/}.

\section*{\Exercises}

When the author of this book studied assembly language, he also often compiled small C functions and then rewrote
them gradually to assembly, trying to make their code as short as possible.
This probably is not worth doing in real-world scenarios today,
because it's hard to compete with the latest compilers in terms of efficiency. It is, however, a very good way to gain a better understanding of assembly.
Feel free, therefore, to take any assembly code from this book and try to make it shorter.
However, don't forget to test what you have written.

% rewrote to show that debug\release and optimisations levels are orthogonal concepts.
\section*{Optimization levels and debug information}

Source code can be compiled by different compilers with various optimization levels.
A typical compiler has about three such levels, where level zero means that optimization is completely disabled.
Optimization can also be targeted towards code size or code speed.
A non-optimizing compiler is faster and produces more understandable (albeit verbose) code,
whereas an optimizing compiler is slower and tries to produce code that runs faster (but is not necessarily more compact).
In addition to optimization levels, a compiler can include some debug information in the resulting file,
producing code that is easy to debug.
One of the important features of the ´debug' code is that it might contain links
between each line of the source code and its respective machine code address.
Optimizing compilers, on the other hand, tend to produce output where entire lines of source code
can be optimized away and thus not even be present in the resulting machine code.
Reverse engineers can encounter either version, simply because some developers turn on the compiler's optimization flags and others do not.
Because of this, we'll try to work on examples of both debug and release versions of the code featured in this book, wherever possible.

Sometimes some pretty ancient compilers are used in this book, in order to get the shortest (or simplest) possible code snippet.
}
\ES{\input{patterns/patterns_opt_dbg_ES}}
\ITA{\input{patterns/patterns_opt_dbg_ITA}}
\PTBR{\input{patterns/patterns_opt_dbg_PTBR}}
\RU{\input{patterns/patterns_opt_dbg_RU}}
\THA{\input{patterns/patterns_opt_dbg_THA}}
\DE{\input{patterns/patterns_opt_dbg_DE}}
\FR{\input{patterns/patterns_opt_dbg_FR}}
\PL{\input{patterns/patterns_opt_dbg_PL}}

\RU{\section{Некоторые базовые понятия}}
\EN{\section{Some basics}}
\DE{\section{Einige Grundlagen}}
\FR{\section{Quelques bases}}
\ES{\section{\ESph{}}}
\ITA{\section{Alcune basi teoriche}}
\PTBR{\section{\PTBRph{}}}
\THA{\section{\THAph{}}}
\PL{\section{\PLph{}}}

% sections:
\EN{\input{patterns/intro_CPU_ISA_EN}}
\ES{\input{patterns/intro_CPU_ISA_ES}}
\ITA{\input{patterns/intro_CPU_ISA_ITA}}
\PTBR{\input{patterns/intro_CPU_ISA_PTBR}}
\RU{\input{patterns/intro_CPU_ISA_RU}}
\DE{\input{patterns/intro_CPU_ISA_DE}}
\FR{\input{patterns/intro_CPU_ISA_FR}}
\PL{\input{patterns/intro_CPU_ISA_PL}}

\EN{\subsection{Numeral Systems}

Humans have become accustomed to a decimal numeral system, probably because almost everyone has 10 fingers.
Nevertheless, the number \q{10} has no significant meaning in science and mathematics.
The natural numeral system in digital electronics is binary: 0 is for an absence of current in the wire, and 1 for presence.
10 in binary is 2 in decimal, 100 in binary is 4 in decimal, and so on.

% This sentence is a bit unweildy - maybe try 'Our ten-digit system would be described as having a radix...' - Renaissance
If the numeral system has 10 digits, it has a \IT{radix} (or \IT{base}) of 10.
The binary numeral system has a \IT{radix} of 2.

Important things to recall:

1) A \IT{number} is a number, while a \IT{digit} is a term from writing systems, and is usually one character

% The original is 'number' is not changed; I think the intent is value, and changed it - Renaissance
2) The value of a number does not change when converted to another radix; only the writing notation for that value has changed (and therefore the way of representing it in \ac{RAM}).

\subsection{Converting From One Radix To Another}

Positional notation is used almost every numerical system. This means that a digit has weight relative to where it is placed inside of the larger number.
If 2 is placed at the rightmost place, it's 2, but if it's placed one digit before rightmost, it's 20.

What does $1234$ stand for?

$10^3 \cdot 1 + 10^2 \cdot 2 + 10^1 \cdot 3 + 1 \cdot 4 = 1234$ or
$1000 \cdot 1 + 100 \cdot 2 + 10 \cdot 3 + 4 = 1234$

It's the same story for binary numbers, but the base is 2 instead of 10.
What does 0b101011 stand for?

$2^5 \cdot 1 + 2^4 \cdot 0 + 2^3 \cdot 1 + 2^2 \cdot 0 + 2^1 \cdot 1 + 2^0 \cdot 1 = 43$ or
$32 \cdot 1 + 16 \cdot 0 + 8 \cdot 1 + 4 \cdot 0 + 2 \cdot 1 + 1 = 43$

There is such a thing as non-positional notation, such as the Roman numeral system.
\footnote{About numeric system evolution, see \InSqBrackets{\TAOCPvolII{}, 195--213.}}.
% Maybe add a sentence to fill in that X is always 10, and is therefore non-positional, even though putting an I before subtracts and after adds, and is in that sense positional
Perhaps, humankind switched to positional notation because it's easier to do basic operations (addition, multiplication, etc.) on paper by hand.

Binary numbers can be added, subtracted and so on in the very same as taught in schools, but only 2 digits are available.

Binary numbers are bulky when represented in source code and dumps, so that is where the hexadecimal numeral system can be useful.
A hexadecimal radix uses the digits 0..9, and also 6 Latin characters: A..F.
Each hexadecimal digit takes 4 bits or 4 binary digits, so it's very easy to convert from binary number to hexadecimal and back, even manually, in one's mind.

\begin{center}
\begin{longtable}{ | l | l | l | }
\hline
\HeaderColor hexadecimal & \HeaderColor binary & \HeaderColor decimal \\
\hline
0	&0000	&0 \\
1	&0001	&1 \\
2	&0010	&2 \\
3	&0011	&3 \\
4	&0100	&4 \\
5	&0101	&5 \\
6	&0110	&6 \\
7	&0111	&7 \\
8	&1000	&8 \\
9	&1001	&9 \\
A	&1010	&10 \\
B	&1011	&11 \\
C	&1100	&12 \\
D	&1101	&13 \\
E	&1110	&14 \\
F	&1111	&15 \\
\hline
\end{longtable}
\end{center}

How can one tell which radix is being used in a specific instance?

Decimal numbers are usually written as is, i.e., 1234. Some assemblers allow an identifier on decimal radix numbers, in which the number would be written with a "d" suffix: 1234d.

Binary numbers are sometimes prepended with the "0b" prefix: 0b100110111 (\ac{GCC} has a non-standard language extension for this\footnote{\url{https://gcc.gnu.org/onlinedocs/gcc/Binary-constants.html}}).
There is also another way: using a "b" suffix, for example: 100110111b.
This book tries to use the "0b" prefix consistently throughout the book for binary numbers.

Hexadecimal numbers are prepended with "0x" prefix in \CCpp and other \ac{PL}s: 0x1234ABCD.
Alternatively, they are given a "h" suffix: 1234ABCDh. This is common way of representing them in assemblers and debuggers.
In this convention, if the number is started with a Latin (A..F) digit, a 0 is added at the beginning: 0ABCDEFh.
There was also convention that was popular in 8-bit home computers era, using \$ prefix, like \$ABCD.
The book will try to stick to "0x" prefix throughout the book for hexadecimal numbers.

Should one learn to convert numbers mentally? A table of 1-digit hexadecimal numbers can easily be memorized.
As for larger numbers, it's probably not worth tormenting yourself.

Perhaps the most visible hexadecimal numbers are in \ac{URL}s.
This is the way that non-Latin characters are encoded.
For example:
\url{https://en.wiktionary.org/wiki/na\%C3\%AFvet\%C3\%A9} is the \ac{URL} of Wiktionary article about \q{naïveté} word.

\subsubsection{Octal Radix}

Another numeral system heavily used in the past of computer programming is octal. In octal there are 8 digits (0..7), and each is mapped to 3 bits, so it's easy to convert numbers back and forth.
It has been superseded by the hexadecimal system almost everywhere, but, surprisingly, there is a *NIX utility, used often by many people, which takes octal numbers as argument: \TT{chmod}.

\myindex{UNIX!chmod}
As many *NIX users know, \TT{chmod} argument can be a number of 3 digits. The first digit represents the rights of the owner of the file (read, write and/or execute), the second is the rights for the group to which the file belongs, and the third is for everyone else.
Each digit that \TT{chmod} takes can be represented in binary form:

\begin{center}
\begin{longtable}{ | l | l | l | }
\hline
\HeaderColor decimal & \HeaderColor binary & \HeaderColor meaning \\
\hline
7	&111	&\textbf{rwx} \\
6	&110	&\textbf{rw-} \\
5	&101	&\textbf{r-x} \\
4	&100	&\textbf{r-{}-} \\
3	&011	&\textbf{-wx} \\
2	&010	&\textbf{-w-} \\
1	&001	&\textbf{-{}-x} \\
0	&000	&\textbf{-{}-{}-} \\
\hline
\end{longtable}
\end{center}

So each bit is mapped to a flag: read/write/execute.

The importance of \TT{chmod} here is that the whole number in argument can be represented as octal number.
Let's take, for example, 644.
When you run \TT{chmod 644 file}, you set read/write permissions for owner, read permissions for group and again, read permissions for everyone else.
If we convert the octal number 644 to binary, it would be \TT{110100100}, or, in groups of 3 bits, \TT{110 100 100}.

Now we see that each triplet describe permissions for owner/group/others: first is \TT{rw-}, second is \TT{r--} and third is \TT{r--}.

The octal numeral system was also popular on old computers like PDP-8, because word there could be 12, 24 or 36 bits, and these numbers are all divisible by 3, so the octal system was natural in that environment.
Nowadays, all popular computers employ word/address sizes of 16, 32 or 64 bits, and these numbers are all divisible by 4, so the hexadecimal system is more natural there.

The octal numeral system is supported by all standard \CCpp compilers.
This is a source of confusion sometimes, because octal numbers are encoded with a zero prepended, for example, 0377 is 255.
Sometimes, you might make a typo and write "09" instead of 9, and the compiler would report an error.
GCC might report something like this:\\
\TT{error: invalid digit "9" in octal constant}.

Also, the octal system is somewhat popular in Java. When the IDA shows Java strings with non-printable characters,
they are encoded in the octal system instead of hexadecimal.
\myindex{JAD}
The JAD Java decompiler behaves the same way.

\subsubsection{Divisibility}

When you see a decimal number like 120, you can quickly deduce that it's divisible by 10, because the last digit is zero.
In the same way, 123400 is divisible by 100, because the two last digits are zeros.

Likewise, the hexadecimal number 0x1230 is divisible by 0x10 (or 16), 0x123000 is divisible by 0x1000 (or 4096), etc.

The binary number 0b1000101000 is divisible by 0b1000 (8), etc.

This property can often be used to quickly realize if the size of some block in memory is padded to some boundary.
For example, sections in \ac{PE} files are almost always started at addresses ending with 3 hexadecimal zeros: 0x41000, 0x10001000, etc.
The reason behind this is the fact that almost all \ac{PE} sections are padded to a boundary of 0x1000 (4096) bytes.

\subsubsection{Multi-Precision Arithmetic and Radix}

\index{RSA}
Multi-precision arithmetic can use huge numbers, and each one may be stored in several bytes.
For example, RSA keys, both public and private, span up to 4096 bits, and maybe even more.

% I'm not sure how to change this, but the normal format for quoting would be just to mention the author or book, and footnote to the full reference
In \InSqBrackets{\TAOCPvolII, 265} we find the following idea: when you store a multi-precision number in several bytes,
the whole number can be represented as having a radix of $2^8=256$, and each digit goes to the corresponding byte.
Likewise, if you store a multi-precision number in several 32-bit integer values, each digit goes to each 32-bit slot,
and you may think about this number as stored in radix of $2^{32}$.

\subsubsection{How to Pronounce Non-Decimal Numbers}

Numbers in a non-decimal base are usually pronounced by digit by digit: ``one-zero-zero-one-one-...''.
Words like ``ten'' and ``thousand'' are usually not pronounced, to prevent confusion with the decimal base system.

\subsubsection{Floating point numbers}

To distinguish floating point numbers from integers, they are usually written with ``.0'' at the end,
like $0.0$, $123.0$, etc.
}
\RU{\subsection{Представление чисел}

Люди привыкли к десятичной системе счисления вероятно потому что почти у каждого есть по 10 пальцев.
Тем не менее, число 10 не имеет особого значения в науке и математике.
Двоичная система естествена для цифровой электроники: 0 означает отсутствие тока в проводе и 1 --- его присутствие.
10 в двоичной системе это 2 в десятичной; 100 в двоичной это 4 в десятичной, итд.

Если в системе счисления есть 10 цифр, её \IT{основание} или \IT{radix} это 10.
Двоичная система имеет \IT{основание} 2.

Важные вещи, которые полезно вспомнить:
1) \IT{число} это число, в то время как \IT{цифра} это термин из системы письменности, и это обычно один символ;
2) само число не меняется, когда конвертируется из одного основания в другое: меняется способ его записи (или представления
в памяти).

Как сконвертировать число из одного основания в другое?

Позиционная нотация используется почти везде, это означает, что всякая цифра имеет свой вес, в зависимости от её расположения
внутри числа.
Если 2 расположена в самом последнем месте справа, это 2.
Если она расположена в месте перед последним, это 20.

Что означает $1234$?

$10^3 \cdot 1 + 10^2 \cdot 2 + 10^1 \cdot 3 + 1 \cdot 4$ = 1234 или
$1000 \cdot 1 + 100 \cdot 2 + 10 \cdot 3 + 4 = 1234$

Та же история и для двоичных чисел, только основание там 2 вместо 10.
Что означает 0b101011?

$2^5 \cdot 1 + 2^4 \cdot 0 + 2^3 \cdot 1 + 2^2 \cdot 0 + 2^1 \cdot 1 + 2^0 \cdot 1 = 43$ или
$32 \cdot 1 + 16 \cdot 0 + 8 \cdot 1 + 4 \cdot 0 + 2 \cdot 1 + 1 = 43$

Позиционную нотацию можно противопоставить непозиционной нотации, такой как римская система записи чисел
\footnote{Об эволюции способов записи чисел, см.также: \InSqBrackets{\TAOCPvolII{}, 195--213.}}.
Вероятно, человечество перешло на позиционную нотацию, потому что так проще работать с числами (сложение, умножение, итд)
на бумаге, в ручную.

Действительно, двоичные числа можно складывать, вычитать, итд, точно также, как этому обычно обучают в школах,
только доступны лишь 2 цифры.

Двоичные числа громоздки, когда их используют в исходных кодах и дампах, так что в этих случаях применяется шестнадцатеричная
система.
Используются цифры 0..9 и еще 6 латинских букв: A..F.
Каждая шестнадцатеричная цифра занимает 4 бита или 4 двоичных цифры, так что конвертировать из двоичной системы в
шестнадцатеричную и назад, можно легко вручную, или даже в уме.

\begin{center}
\begin{longtable}{ | l | l | l | }
\hline
\HeaderColor шестнадцатеричная & \HeaderColor двоичная & \HeaderColor десятичная \\
\hline
0	&0000	&0 \\
1	&0001	&1 \\
2	&0010	&2 \\
3	&0011	&3 \\
4	&0100	&4 \\
5	&0101	&5 \\
6	&0110	&6 \\
7	&0111	&7 \\
8	&1000	&8 \\
9	&1001	&9 \\
A	&1010	&10 \\
B	&1011	&11 \\
C	&1100	&12 \\
D	&1101	&13 \\
E	&1110	&14 \\
F	&1111	&15 \\
\hline
\end{longtable}
\end{center}

Как понять, какое основание используется в конкретном месте?

Десятичные числа обычно записываются как есть, т.е., 1234. Но некоторые ассемблеры позволяют подчеркивать
этот факт для ясности, и это число может быть дополнено суффиксом "d": 1234d.

К двоичным числам иногда спереди добавляют префикс "0b": 0b100110111
(В \ac{GCC} для этого есть нестандартное расширение языка
\footnote{\url{https://gcc.gnu.org/onlinedocs/gcc/Binary-constants.html}}).
Есть также еще один способ: суффикс "b", например: 100110111b.
В этой книге я буду пытаться придерживаться префикса "0b" для двоичных чисел.

Шестнадцатеричные числа имеют префикс "0x" в \CCpp и некоторых других \ac{PL}: 0x1234ABCD.
Либо они имеют суффикс "h": 1234ABCDh --- обычно так они представляются в ассемблерах и отладчиках.
Если число начинается с цифры A..F, перед ним добавляется 0: 0ABCDEFh.
Во времена 8-битных домашних компьютеров, был также способ записи чисел используя префикс \$, например, \$ABCD.
В книге я попытаюсь придерживаться префикса "0x" для шестнадцатеричных чисел.

Нужно ли учиться конвертировать числа в уме? Таблицу шестнадцатеричных чисел из одной цифры легко запомнить.
А запоминать б\'{о}льшие числа, наверное, не стоит.

Наверное, чаще всего шестнадцатеричные числа можно увидеть в \ac{URL}-ах.
Так кодируются буквы не из числа латинских.
Например:
\url{https://en.wiktionary.org/wiki/na\%C3\%AFvet\%C3\%A9} это \ac{URL} страницы в Wiktionary о слове \q{naïveté}.

\subsubsection{Восьмеричная система}

Еще одна система, которая в прошлом много использовалась в программировании это восьмеричная: есть 8 цифр (0..7) и каждая
описывает 3 бита, так что легко конвертировать числа туда и назад.
Она почти везде была заменена шестнадцатеричной, но удивительно, в *NIX имеется утилита использующаяся многими людьми,
которая принимает на вход восьмеричное число: \TT{chmod}.

\myindex{UNIX!chmod}
Как знают многие пользователи *NIX, аргумент \TT{chmod} это число из трех цифр. Первая цифра это права владельца файла,
вторая это права группы (которой файл принадлежит), третья для всех остальных.
И каждая цифра может быть представлена в двоичном виде:

\begin{center}
\begin{longtable}{ | l | l | l | }
\hline
\HeaderColor десятичная & \HeaderColor двоичная & \HeaderColor значение \\
\hline
7	&111	&\textbf{rwx} \\
6	&110	&\textbf{rw-} \\
5	&101	&\textbf{r-x} \\
4	&100	&\textbf{r-{}-} \\
3	&011	&\textbf{-wx} \\
2	&010	&\textbf{-w-} \\
1	&001	&\textbf{-{}-x} \\
0	&000	&\textbf{-{}-{}-} \\
\hline
\end{longtable}
\end{center}

Так что каждый бит привязан к флагу: read/write/execute (чтение/запись/исполнение).

И вот почему я вспомнил здесь о \TT{chmod}, это потому что всё число может быть представлено как число в восьмеричной системе.
Для примера возьмем 644.
Когда вы запускаете \TT{chmod 644 file}, вы выставляете права read/write для владельца, права read для группы, и снова,
read для всех остальных.
Сконвертируем число 644 из восьмеричной системы в двоичную, это будет \TT{110100100}, или (в группах по 3 бита) \TT{110 100 100}.

Теперь мы видим, что каждая тройка описывает права для владельца/группы/остальных:
первая это \TT{rw-}, вторая это \TT{r--} и третья это \TT{r--}.

Восьмеричная система была также популярная на старых компьютерах вроде PDP-8, потому что слово там могло содержать 12, 24 или
36 бит, и эти числа делятся на 3, так что выбор восьмеричной системы в той среде был логичен.
Сейчас, все популярные компьютеры имеют размер слова/адреса 16, 32 или 64 бита, и эти числа делятся на 4,
так что шестнадцатеричная система здесь удобнее.

Восьмеричная система поддерживается всеми стандартными компиляторами \CCpp{}.
Это иногда источник недоумения, потому что восьмеричные числа кодируются с нулем вперед, например, 0377 это 255.
И иногда, вы можете сделать опечатку, и написать "09" вместо 9, и компилятор выдаст ошибку.
GCC может выдать что-то вроде:\\
\TT{error: invalid digit "9" in octal constant}.

Также, восьмеричная система популярна в Java: когда IDA показывает строку с непечатаемыми символами,
они кодируются в восьмеричной системе вместо шестнадцатеричной.
\myindex{JAD}
Точно также себя ведет декомпилятор с Java JAD.

\subsubsection{Делимость}

Когда вы видите десятичное число вроде 120, вы можете быстро понять что оно делится на 10, потому что последняя цифра это 0.
Точно также, 123400 делится на 100, потому что две последних цифры это нули.

Точно также, шестнадцатеричное число 0x1230 делится на 0x10 (или 16), 0x123000 делится на 0x1000 (или 4096), итд.

Двоичное число 0b1000101000 делится на 0b1000 (8), итд.

Это свойство можно часто использовать, чтобы быстро понять,
что длина какого-либо блока в памяти выровнена по некоторой границе.
Например, секции в \ac{PE}-файлах почти всегда начинаются с адресов заканчивающихся 3 шестнадцатеричными нулями:
0x41000, 0x10001000, итд.
Причина в том, что почти все секции в \ac{PE} выровнены по границе 0x1000 (4096) байт.

\subsubsection{Арифметика произвольной точности и основание}

\index{RSA}
Арифметика произвольной точности (multi-precision arithmetic) может использовать огромные числа,
которые могут храниться в нескольких байтах.
Например, ключи RSA, и открытые и закрытые, могут занимать до 4096 бит и даже больше.

В \InSqBrackets{\TAOCPvolII, 265} можно найти такую идею: когда вы сохраняете число произвольной точности в нескольких байтах,
всё число может быть представлено как имеющую систему счисления по основанию $2^8=256$, и каждая цифра находится
в соответствующем байте.
Точно также, если вы сохраняете число произвольной точности в нескольких 32-битных целочисленных значениях,
каждая цифра отправляется в каждый 32-битный слот, и вы можете считать что это число записано в системе с основанием $2^{32}$.

\subsubsection{Произношение}

Числа в недесятичных системах счислениях обычно произносятся по одной цифре: ``один-ноль-ноль-один-один-...''.
Слова вроде ``десять'', ``тысяча'', итд, обычно не произносятся, потому что тогда можно спутать с десятичной системой.

\subsubsection{Числа с плавающей запятой}

Чтобы отличать числа с плавающей запятой от целочисленных, часто, в конце добавляют ``.0'',
например $0.0$, $123.0$, итд.

}
\ITA{\input{patterns/numeral_ITA}}
\DE{\input{patterns/numeral_DE}}
\FR{\input{patterns/numeral_FR}}
\PL{\input{patterns/numeral_PL}}

% chapters
\ifdefined\SPANISH
\chapter{Patrones de código}
\fi % SPANISH

\ifdefined\GERMAN
\chapter{Code-Muster}
\fi % GERMAN

\ifdefined\ENGLISH
\chapter{Code Patterns}
\fi % ENGLISH

\ifdefined\ITALIAN
\chapter{Forme di codice}
\fi % ITALIAN

\ifdefined\RUSSIAN
\chapter{Образцы кода}
\fi % RUSSIAN

\ifdefined\BRAZILIAN
\chapter{Padrões de códigos}
\fi % BRAZILIAN

\ifdefined\THAI
\chapter{รูปแบบของโค้ด}
\fi % THAI

\ifdefined\FRENCH
\chapter{Modèle de code}
\fi % FRENCH

\ifdefined\POLISH
\chapter{\PLph{}}
\fi % POLISH

% sections
\EN{\input{patterns/patterns_opt_dbg_EN}}
\ES{\input{patterns/patterns_opt_dbg_ES}}
\ITA{\input{patterns/patterns_opt_dbg_ITA}}
\PTBR{\input{patterns/patterns_opt_dbg_PTBR}}
\RU{\input{patterns/patterns_opt_dbg_RU}}
\THA{\input{patterns/patterns_opt_dbg_THA}}
\DE{\input{patterns/patterns_opt_dbg_DE}}
\FR{\input{patterns/patterns_opt_dbg_FR}}
\PL{\input{patterns/patterns_opt_dbg_PL}}

\RU{\section{Некоторые базовые понятия}}
\EN{\section{Some basics}}
\DE{\section{Einige Grundlagen}}
\FR{\section{Quelques bases}}
\ES{\section{\ESph{}}}
\ITA{\section{Alcune basi teoriche}}
\PTBR{\section{\PTBRph{}}}
\THA{\section{\THAph{}}}
\PL{\section{\PLph{}}}

% sections:
\EN{\input{patterns/intro_CPU_ISA_EN}}
\ES{\input{patterns/intro_CPU_ISA_ES}}
\ITA{\input{patterns/intro_CPU_ISA_ITA}}
\PTBR{\input{patterns/intro_CPU_ISA_PTBR}}
\RU{\input{patterns/intro_CPU_ISA_RU}}
\DE{\input{patterns/intro_CPU_ISA_DE}}
\FR{\input{patterns/intro_CPU_ISA_FR}}
\PL{\input{patterns/intro_CPU_ISA_PL}}

\EN{\input{patterns/numeral_EN}}
\RU{\input{patterns/numeral_RU}}
\ITA{\input{patterns/numeral_ITA}}
\DE{\input{patterns/numeral_DE}}
\FR{\input{patterns/numeral_FR}}
\PL{\input{patterns/numeral_PL}}

% chapters
\input{patterns/00_empty/main}
\input{patterns/011_ret/main}
\input{patterns/01_helloworld/main}
\input{patterns/015_prolog_epilogue/main}
\input{patterns/02_stack/main}
\input{patterns/03_printf/main}
\input{patterns/04_scanf/main}
\input{patterns/05_passing_arguments/main}
\input{patterns/06_return_results/main}
\input{patterns/061_pointers/main}
\input{patterns/065_GOTO/main}
\input{patterns/07_jcc/main}
\input{patterns/08_switch/main}
\input{patterns/09_loops/main}
\input{patterns/10_strings/main}
\input{patterns/11_arith_optimizations/main}
\input{patterns/12_FPU/main}
\input{patterns/13_arrays/main}
\input{patterns/14_bitfields/main}
\EN{\input{patterns/145_LCG/main_EN}}
\RU{\input{patterns/145_LCG/main_RU}}
\input{patterns/15_structs/main}
\input{patterns/17_unions/main}
\input{patterns/18_pointers_to_functions/main}
\input{patterns/185_64bit_in_32_env/main}

\EN{\input{patterns/19_SIMD/main_EN}}
\RU{\input{patterns/19_SIMD/main_RU}}
\DE{\input{patterns/19_SIMD/main_DE}}

\EN{\input{patterns/20_x64/main_EN}}
\RU{\input{patterns/20_x64/main_RU}}

\EN{\input{patterns/205_floating_SIMD/main_EN}}
\RU{\input{patterns/205_floating_SIMD/main_RU}}
\DE{\input{patterns/205_floating_SIMD/main_DE}}

\EN{\input{patterns/ARM/main_EN}}
\RU{\input{patterns/ARM/main_RU}}
\DE{\input{patterns/ARM/main_DE}}

\input{patterns/MIPS/main}

\ifdefined\SPANISH
\chapter{Patrones de código}
\fi % SPANISH

\ifdefined\GERMAN
\chapter{Code-Muster}
\fi % GERMAN

\ifdefined\ENGLISH
\chapter{Code Patterns}
\fi % ENGLISH

\ifdefined\ITALIAN
\chapter{Forme di codice}
\fi % ITALIAN

\ifdefined\RUSSIAN
\chapter{Образцы кода}
\fi % RUSSIAN

\ifdefined\BRAZILIAN
\chapter{Padrões de códigos}
\fi % BRAZILIAN

\ifdefined\THAI
\chapter{รูปแบบของโค้ด}
\fi % THAI

\ifdefined\FRENCH
\chapter{Modèle de code}
\fi % FRENCH

\ifdefined\POLISH
\chapter{\PLph{}}
\fi % POLISH

% sections
\EN{\input{patterns/patterns_opt_dbg_EN}}
\ES{\input{patterns/patterns_opt_dbg_ES}}
\ITA{\input{patterns/patterns_opt_dbg_ITA}}
\PTBR{\input{patterns/patterns_opt_dbg_PTBR}}
\RU{\input{patterns/patterns_opt_dbg_RU}}
\THA{\input{patterns/patterns_opt_dbg_THA}}
\DE{\input{patterns/patterns_opt_dbg_DE}}
\FR{\input{patterns/patterns_opt_dbg_FR}}
\PL{\input{patterns/patterns_opt_dbg_PL}}

\RU{\section{Некоторые базовые понятия}}
\EN{\section{Some basics}}
\DE{\section{Einige Grundlagen}}
\FR{\section{Quelques bases}}
\ES{\section{\ESph{}}}
\ITA{\section{Alcune basi teoriche}}
\PTBR{\section{\PTBRph{}}}
\THA{\section{\THAph{}}}
\PL{\section{\PLph{}}}

% sections:
\EN{\input{patterns/intro_CPU_ISA_EN}}
\ES{\input{patterns/intro_CPU_ISA_ES}}
\ITA{\input{patterns/intro_CPU_ISA_ITA}}
\PTBR{\input{patterns/intro_CPU_ISA_PTBR}}
\RU{\input{patterns/intro_CPU_ISA_RU}}
\DE{\input{patterns/intro_CPU_ISA_DE}}
\FR{\input{patterns/intro_CPU_ISA_FR}}
\PL{\input{patterns/intro_CPU_ISA_PL}}

\EN{\input{patterns/numeral_EN}}
\RU{\input{patterns/numeral_RU}}
\ITA{\input{patterns/numeral_ITA}}
\DE{\input{patterns/numeral_DE}}
\FR{\input{patterns/numeral_FR}}
\PL{\input{patterns/numeral_PL}}

% chapters
\input{patterns/00_empty/main}
\input{patterns/011_ret/main}
\input{patterns/01_helloworld/main}
\input{patterns/015_prolog_epilogue/main}
\input{patterns/02_stack/main}
\input{patterns/03_printf/main}
\input{patterns/04_scanf/main}
\input{patterns/05_passing_arguments/main}
\input{patterns/06_return_results/main}
\input{patterns/061_pointers/main}
\input{patterns/065_GOTO/main}
\input{patterns/07_jcc/main}
\input{patterns/08_switch/main}
\input{patterns/09_loops/main}
\input{patterns/10_strings/main}
\input{patterns/11_arith_optimizations/main}
\input{patterns/12_FPU/main}
\input{patterns/13_arrays/main}
\input{patterns/14_bitfields/main}
\EN{\input{patterns/145_LCG/main_EN}}
\RU{\input{patterns/145_LCG/main_RU}}
\input{patterns/15_structs/main}
\input{patterns/17_unions/main}
\input{patterns/18_pointers_to_functions/main}
\input{patterns/185_64bit_in_32_env/main}

\EN{\input{patterns/19_SIMD/main_EN}}
\RU{\input{patterns/19_SIMD/main_RU}}
\DE{\input{patterns/19_SIMD/main_DE}}

\EN{\input{patterns/20_x64/main_EN}}
\RU{\input{patterns/20_x64/main_RU}}

\EN{\input{patterns/205_floating_SIMD/main_EN}}
\RU{\input{patterns/205_floating_SIMD/main_RU}}
\DE{\input{patterns/205_floating_SIMD/main_DE}}

\EN{\input{patterns/ARM/main_EN}}
\RU{\input{patterns/ARM/main_RU}}
\DE{\input{patterns/ARM/main_DE}}

\input{patterns/MIPS/main}

\ifdefined\SPANISH
\chapter{Patrones de código}
\fi % SPANISH

\ifdefined\GERMAN
\chapter{Code-Muster}
\fi % GERMAN

\ifdefined\ENGLISH
\chapter{Code Patterns}
\fi % ENGLISH

\ifdefined\ITALIAN
\chapter{Forme di codice}
\fi % ITALIAN

\ifdefined\RUSSIAN
\chapter{Образцы кода}
\fi % RUSSIAN

\ifdefined\BRAZILIAN
\chapter{Padrões de códigos}
\fi % BRAZILIAN

\ifdefined\THAI
\chapter{รูปแบบของโค้ด}
\fi % THAI

\ifdefined\FRENCH
\chapter{Modèle de code}
\fi % FRENCH

\ifdefined\POLISH
\chapter{\PLph{}}
\fi % POLISH

% sections
\EN{\input{patterns/patterns_opt_dbg_EN}}
\ES{\input{patterns/patterns_opt_dbg_ES}}
\ITA{\input{patterns/patterns_opt_dbg_ITA}}
\PTBR{\input{patterns/patterns_opt_dbg_PTBR}}
\RU{\input{patterns/patterns_opt_dbg_RU}}
\THA{\input{patterns/patterns_opt_dbg_THA}}
\DE{\input{patterns/patterns_opt_dbg_DE}}
\FR{\input{patterns/patterns_opt_dbg_FR}}
\PL{\input{patterns/patterns_opt_dbg_PL}}

\RU{\section{Некоторые базовые понятия}}
\EN{\section{Some basics}}
\DE{\section{Einige Grundlagen}}
\FR{\section{Quelques bases}}
\ES{\section{\ESph{}}}
\ITA{\section{Alcune basi teoriche}}
\PTBR{\section{\PTBRph{}}}
\THA{\section{\THAph{}}}
\PL{\section{\PLph{}}}

% sections:
\EN{\input{patterns/intro_CPU_ISA_EN}}
\ES{\input{patterns/intro_CPU_ISA_ES}}
\ITA{\input{patterns/intro_CPU_ISA_ITA}}
\PTBR{\input{patterns/intro_CPU_ISA_PTBR}}
\RU{\input{patterns/intro_CPU_ISA_RU}}
\DE{\input{patterns/intro_CPU_ISA_DE}}
\FR{\input{patterns/intro_CPU_ISA_FR}}
\PL{\input{patterns/intro_CPU_ISA_PL}}

\EN{\input{patterns/numeral_EN}}
\RU{\input{patterns/numeral_RU}}
\ITA{\input{patterns/numeral_ITA}}
\DE{\input{patterns/numeral_DE}}
\FR{\input{patterns/numeral_FR}}
\PL{\input{patterns/numeral_PL}}

% chapters
\input{patterns/00_empty/main}
\input{patterns/011_ret/main}
\input{patterns/01_helloworld/main}
\input{patterns/015_prolog_epilogue/main}
\input{patterns/02_stack/main}
\input{patterns/03_printf/main}
\input{patterns/04_scanf/main}
\input{patterns/05_passing_arguments/main}
\input{patterns/06_return_results/main}
\input{patterns/061_pointers/main}
\input{patterns/065_GOTO/main}
\input{patterns/07_jcc/main}
\input{patterns/08_switch/main}
\input{patterns/09_loops/main}
\input{patterns/10_strings/main}
\input{patterns/11_arith_optimizations/main}
\input{patterns/12_FPU/main}
\input{patterns/13_arrays/main}
\input{patterns/14_bitfields/main}
\EN{\input{patterns/145_LCG/main_EN}}
\RU{\input{patterns/145_LCG/main_RU}}
\input{patterns/15_structs/main}
\input{patterns/17_unions/main}
\input{patterns/18_pointers_to_functions/main}
\input{patterns/185_64bit_in_32_env/main}

\EN{\input{patterns/19_SIMD/main_EN}}
\RU{\input{patterns/19_SIMD/main_RU}}
\DE{\input{patterns/19_SIMD/main_DE}}

\EN{\input{patterns/20_x64/main_EN}}
\RU{\input{patterns/20_x64/main_RU}}

\EN{\input{patterns/205_floating_SIMD/main_EN}}
\RU{\input{patterns/205_floating_SIMD/main_RU}}
\DE{\input{patterns/205_floating_SIMD/main_DE}}

\EN{\input{patterns/ARM/main_EN}}
\RU{\input{patterns/ARM/main_RU}}
\DE{\input{patterns/ARM/main_DE}}

\input{patterns/MIPS/main}

\ifdefined\SPANISH
\chapter{Patrones de código}
\fi % SPANISH

\ifdefined\GERMAN
\chapter{Code-Muster}
\fi % GERMAN

\ifdefined\ENGLISH
\chapter{Code Patterns}
\fi % ENGLISH

\ifdefined\ITALIAN
\chapter{Forme di codice}
\fi % ITALIAN

\ifdefined\RUSSIAN
\chapter{Образцы кода}
\fi % RUSSIAN

\ifdefined\BRAZILIAN
\chapter{Padrões de códigos}
\fi % BRAZILIAN

\ifdefined\THAI
\chapter{รูปแบบของโค้ด}
\fi % THAI

\ifdefined\FRENCH
\chapter{Modèle de code}
\fi % FRENCH

\ifdefined\POLISH
\chapter{\PLph{}}
\fi % POLISH

% sections
\EN{\input{patterns/patterns_opt_dbg_EN}}
\ES{\input{patterns/patterns_opt_dbg_ES}}
\ITA{\input{patterns/patterns_opt_dbg_ITA}}
\PTBR{\input{patterns/patterns_opt_dbg_PTBR}}
\RU{\input{patterns/patterns_opt_dbg_RU}}
\THA{\input{patterns/patterns_opt_dbg_THA}}
\DE{\input{patterns/patterns_opt_dbg_DE}}
\FR{\input{patterns/patterns_opt_dbg_FR}}
\PL{\input{patterns/patterns_opt_dbg_PL}}

\RU{\section{Некоторые базовые понятия}}
\EN{\section{Some basics}}
\DE{\section{Einige Grundlagen}}
\FR{\section{Quelques bases}}
\ES{\section{\ESph{}}}
\ITA{\section{Alcune basi teoriche}}
\PTBR{\section{\PTBRph{}}}
\THA{\section{\THAph{}}}
\PL{\section{\PLph{}}}

% sections:
\EN{\input{patterns/intro_CPU_ISA_EN}}
\ES{\input{patterns/intro_CPU_ISA_ES}}
\ITA{\input{patterns/intro_CPU_ISA_ITA}}
\PTBR{\input{patterns/intro_CPU_ISA_PTBR}}
\RU{\input{patterns/intro_CPU_ISA_RU}}
\DE{\input{patterns/intro_CPU_ISA_DE}}
\FR{\input{patterns/intro_CPU_ISA_FR}}
\PL{\input{patterns/intro_CPU_ISA_PL}}

\EN{\input{patterns/numeral_EN}}
\RU{\input{patterns/numeral_RU}}
\ITA{\input{patterns/numeral_ITA}}
\DE{\input{patterns/numeral_DE}}
\FR{\input{patterns/numeral_FR}}
\PL{\input{patterns/numeral_PL}}

% chapters
\input{patterns/00_empty/main}
\input{patterns/011_ret/main}
\input{patterns/01_helloworld/main}
\input{patterns/015_prolog_epilogue/main}
\input{patterns/02_stack/main}
\input{patterns/03_printf/main}
\input{patterns/04_scanf/main}
\input{patterns/05_passing_arguments/main}
\input{patterns/06_return_results/main}
\input{patterns/061_pointers/main}
\input{patterns/065_GOTO/main}
\input{patterns/07_jcc/main}
\input{patterns/08_switch/main}
\input{patterns/09_loops/main}
\input{patterns/10_strings/main}
\input{patterns/11_arith_optimizations/main}
\input{patterns/12_FPU/main}
\input{patterns/13_arrays/main}
\input{patterns/14_bitfields/main}
\EN{\input{patterns/145_LCG/main_EN}}
\RU{\input{patterns/145_LCG/main_RU}}
\input{patterns/15_structs/main}
\input{patterns/17_unions/main}
\input{patterns/18_pointers_to_functions/main}
\input{patterns/185_64bit_in_32_env/main}

\EN{\input{patterns/19_SIMD/main_EN}}
\RU{\input{patterns/19_SIMD/main_RU}}
\DE{\input{patterns/19_SIMD/main_DE}}

\EN{\input{patterns/20_x64/main_EN}}
\RU{\input{patterns/20_x64/main_RU}}

\EN{\input{patterns/205_floating_SIMD/main_EN}}
\RU{\input{patterns/205_floating_SIMD/main_RU}}
\DE{\input{patterns/205_floating_SIMD/main_DE}}

\EN{\input{patterns/ARM/main_EN}}
\RU{\input{patterns/ARM/main_RU}}
\DE{\input{patterns/ARM/main_DE}}

\input{patterns/MIPS/main}

\ifdefined\SPANISH
\chapter{Patrones de código}
\fi % SPANISH

\ifdefined\GERMAN
\chapter{Code-Muster}
\fi % GERMAN

\ifdefined\ENGLISH
\chapter{Code Patterns}
\fi % ENGLISH

\ifdefined\ITALIAN
\chapter{Forme di codice}
\fi % ITALIAN

\ifdefined\RUSSIAN
\chapter{Образцы кода}
\fi % RUSSIAN

\ifdefined\BRAZILIAN
\chapter{Padrões de códigos}
\fi % BRAZILIAN

\ifdefined\THAI
\chapter{รูปแบบของโค้ด}
\fi % THAI

\ifdefined\FRENCH
\chapter{Modèle de code}
\fi % FRENCH

\ifdefined\POLISH
\chapter{\PLph{}}
\fi % POLISH

% sections
\EN{\input{patterns/patterns_opt_dbg_EN}}
\ES{\input{patterns/patterns_opt_dbg_ES}}
\ITA{\input{patterns/patterns_opt_dbg_ITA}}
\PTBR{\input{patterns/patterns_opt_dbg_PTBR}}
\RU{\input{patterns/patterns_opt_dbg_RU}}
\THA{\input{patterns/patterns_opt_dbg_THA}}
\DE{\input{patterns/patterns_opt_dbg_DE}}
\FR{\input{patterns/patterns_opt_dbg_FR}}
\PL{\input{patterns/patterns_opt_dbg_PL}}

\RU{\section{Некоторые базовые понятия}}
\EN{\section{Some basics}}
\DE{\section{Einige Grundlagen}}
\FR{\section{Quelques bases}}
\ES{\section{\ESph{}}}
\ITA{\section{Alcune basi teoriche}}
\PTBR{\section{\PTBRph{}}}
\THA{\section{\THAph{}}}
\PL{\section{\PLph{}}}

% sections:
\EN{\input{patterns/intro_CPU_ISA_EN}}
\ES{\input{patterns/intro_CPU_ISA_ES}}
\ITA{\input{patterns/intro_CPU_ISA_ITA}}
\PTBR{\input{patterns/intro_CPU_ISA_PTBR}}
\RU{\input{patterns/intro_CPU_ISA_RU}}
\DE{\input{patterns/intro_CPU_ISA_DE}}
\FR{\input{patterns/intro_CPU_ISA_FR}}
\PL{\input{patterns/intro_CPU_ISA_PL}}

\EN{\input{patterns/numeral_EN}}
\RU{\input{patterns/numeral_RU}}
\ITA{\input{patterns/numeral_ITA}}
\DE{\input{patterns/numeral_DE}}
\FR{\input{patterns/numeral_FR}}
\PL{\input{patterns/numeral_PL}}

% chapters
\input{patterns/00_empty/main}
\input{patterns/011_ret/main}
\input{patterns/01_helloworld/main}
\input{patterns/015_prolog_epilogue/main}
\input{patterns/02_stack/main}
\input{patterns/03_printf/main}
\input{patterns/04_scanf/main}
\input{patterns/05_passing_arguments/main}
\input{patterns/06_return_results/main}
\input{patterns/061_pointers/main}
\input{patterns/065_GOTO/main}
\input{patterns/07_jcc/main}
\input{patterns/08_switch/main}
\input{patterns/09_loops/main}
\input{patterns/10_strings/main}
\input{patterns/11_arith_optimizations/main}
\input{patterns/12_FPU/main}
\input{patterns/13_arrays/main}
\input{patterns/14_bitfields/main}
\EN{\input{patterns/145_LCG/main_EN}}
\RU{\input{patterns/145_LCG/main_RU}}
\input{patterns/15_structs/main}
\input{patterns/17_unions/main}
\input{patterns/18_pointers_to_functions/main}
\input{patterns/185_64bit_in_32_env/main}

\EN{\input{patterns/19_SIMD/main_EN}}
\RU{\input{patterns/19_SIMD/main_RU}}
\DE{\input{patterns/19_SIMD/main_DE}}

\EN{\input{patterns/20_x64/main_EN}}
\RU{\input{patterns/20_x64/main_RU}}

\EN{\input{patterns/205_floating_SIMD/main_EN}}
\RU{\input{patterns/205_floating_SIMD/main_RU}}
\DE{\input{patterns/205_floating_SIMD/main_DE}}

\EN{\input{patterns/ARM/main_EN}}
\RU{\input{patterns/ARM/main_RU}}
\DE{\input{patterns/ARM/main_DE}}

\input{patterns/MIPS/main}

\ifdefined\SPANISH
\chapter{Patrones de código}
\fi % SPANISH

\ifdefined\GERMAN
\chapter{Code-Muster}
\fi % GERMAN

\ifdefined\ENGLISH
\chapter{Code Patterns}
\fi % ENGLISH

\ifdefined\ITALIAN
\chapter{Forme di codice}
\fi % ITALIAN

\ifdefined\RUSSIAN
\chapter{Образцы кода}
\fi % RUSSIAN

\ifdefined\BRAZILIAN
\chapter{Padrões de códigos}
\fi % BRAZILIAN

\ifdefined\THAI
\chapter{รูปแบบของโค้ด}
\fi % THAI

\ifdefined\FRENCH
\chapter{Modèle de code}
\fi % FRENCH

\ifdefined\POLISH
\chapter{\PLph{}}
\fi % POLISH

% sections
\EN{\input{patterns/patterns_opt_dbg_EN}}
\ES{\input{patterns/patterns_opt_dbg_ES}}
\ITA{\input{patterns/patterns_opt_dbg_ITA}}
\PTBR{\input{patterns/patterns_opt_dbg_PTBR}}
\RU{\input{patterns/patterns_opt_dbg_RU}}
\THA{\input{patterns/patterns_opt_dbg_THA}}
\DE{\input{patterns/patterns_opt_dbg_DE}}
\FR{\input{patterns/patterns_opt_dbg_FR}}
\PL{\input{patterns/patterns_opt_dbg_PL}}

\RU{\section{Некоторые базовые понятия}}
\EN{\section{Some basics}}
\DE{\section{Einige Grundlagen}}
\FR{\section{Quelques bases}}
\ES{\section{\ESph{}}}
\ITA{\section{Alcune basi teoriche}}
\PTBR{\section{\PTBRph{}}}
\THA{\section{\THAph{}}}
\PL{\section{\PLph{}}}

% sections:
\EN{\input{patterns/intro_CPU_ISA_EN}}
\ES{\input{patterns/intro_CPU_ISA_ES}}
\ITA{\input{patterns/intro_CPU_ISA_ITA}}
\PTBR{\input{patterns/intro_CPU_ISA_PTBR}}
\RU{\input{patterns/intro_CPU_ISA_RU}}
\DE{\input{patterns/intro_CPU_ISA_DE}}
\FR{\input{patterns/intro_CPU_ISA_FR}}
\PL{\input{patterns/intro_CPU_ISA_PL}}

\EN{\input{patterns/numeral_EN}}
\RU{\input{patterns/numeral_RU}}
\ITA{\input{patterns/numeral_ITA}}
\DE{\input{patterns/numeral_DE}}
\FR{\input{patterns/numeral_FR}}
\PL{\input{patterns/numeral_PL}}

% chapters
\input{patterns/00_empty/main}
\input{patterns/011_ret/main}
\input{patterns/01_helloworld/main}
\input{patterns/015_prolog_epilogue/main}
\input{patterns/02_stack/main}
\input{patterns/03_printf/main}
\input{patterns/04_scanf/main}
\input{patterns/05_passing_arguments/main}
\input{patterns/06_return_results/main}
\input{patterns/061_pointers/main}
\input{patterns/065_GOTO/main}
\input{patterns/07_jcc/main}
\input{patterns/08_switch/main}
\input{patterns/09_loops/main}
\input{patterns/10_strings/main}
\input{patterns/11_arith_optimizations/main}
\input{patterns/12_FPU/main}
\input{patterns/13_arrays/main}
\input{patterns/14_bitfields/main}
\EN{\input{patterns/145_LCG/main_EN}}
\RU{\input{patterns/145_LCG/main_RU}}
\input{patterns/15_structs/main}
\input{patterns/17_unions/main}
\input{patterns/18_pointers_to_functions/main}
\input{patterns/185_64bit_in_32_env/main}

\EN{\input{patterns/19_SIMD/main_EN}}
\RU{\input{patterns/19_SIMD/main_RU}}
\DE{\input{patterns/19_SIMD/main_DE}}

\EN{\input{patterns/20_x64/main_EN}}
\RU{\input{patterns/20_x64/main_RU}}

\EN{\input{patterns/205_floating_SIMD/main_EN}}
\RU{\input{patterns/205_floating_SIMD/main_RU}}
\DE{\input{patterns/205_floating_SIMD/main_DE}}

\EN{\input{patterns/ARM/main_EN}}
\RU{\input{patterns/ARM/main_RU}}
\DE{\input{patterns/ARM/main_DE}}

\input{patterns/MIPS/main}

\ifdefined\SPANISH
\chapter{Patrones de código}
\fi % SPANISH

\ifdefined\GERMAN
\chapter{Code-Muster}
\fi % GERMAN

\ifdefined\ENGLISH
\chapter{Code Patterns}
\fi % ENGLISH

\ifdefined\ITALIAN
\chapter{Forme di codice}
\fi % ITALIAN

\ifdefined\RUSSIAN
\chapter{Образцы кода}
\fi % RUSSIAN

\ifdefined\BRAZILIAN
\chapter{Padrões de códigos}
\fi % BRAZILIAN

\ifdefined\THAI
\chapter{รูปแบบของโค้ด}
\fi % THAI

\ifdefined\FRENCH
\chapter{Modèle de code}
\fi % FRENCH

\ifdefined\POLISH
\chapter{\PLph{}}
\fi % POLISH

% sections
\EN{\input{patterns/patterns_opt_dbg_EN}}
\ES{\input{patterns/patterns_opt_dbg_ES}}
\ITA{\input{patterns/patterns_opt_dbg_ITA}}
\PTBR{\input{patterns/patterns_opt_dbg_PTBR}}
\RU{\input{patterns/patterns_opt_dbg_RU}}
\THA{\input{patterns/patterns_opt_dbg_THA}}
\DE{\input{patterns/patterns_opt_dbg_DE}}
\FR{\input{patterns/patterns_opt_dbg_FR}}
\PL{\input{patterns/patterns_opt_dbg_PL}}

\RU{\section{Некоторые базовые понятия}}
\EN{\section{Some basics}}
\DE{\section{Einige Grundlagen}}
\FR{\section{Quelques bases}}
\ES{\section{\ESph{}}}
\ITA{\section{Alcune basi teoriche}}
\PTBR{\section{\PTBRph{}}}
\THA{\section{\THAph{}}}
\PL{\section{\PLph{}}}

% sections:
\EN{\input{patterns/intro_CPU_ISA_EN}}
\ES{\input{patterns/intro_CPU_ISA_ES}}
\ITA{\input{patterns/intro_CPU_ISA_ITA}}
\PTBR{\input{patterns/intro_CPU_ISA_PTBR}}
\RU{\input{patterns/intro_CPU_ISA_RU}}
\DE{\input{patterns/intro_CPU_ISA_DE}}
\FR{\input{patterns/intro_CPU_ISA_FR}}
\PL{\input{patterns/intro_CPU_ISA_PL}}

\EN{\input{patterns/numeral_EN}}
\RU{\input{patterns/numeral_RU}}
\ITA{\input{patterns/numeral_ITA}}
\DE{\input{patterns/numeral_DE}}
\FR{\input{patterns/numeral_FR}}
\PL{\input{patterns/numeral_PL}}

% chapters
\input{patterns/00_empty/main}
\input{patterns/011_ret/main}
\input{patterns/01_helloworld/main}
\input{patterns/015_prolog_epilogue/main}
\input{patterns/02_stack/main}
\input{patterns/03_printf/main}
\input{patterns/04_scanf/main}
\input{patterns/05_passing_arguments/main}
\input{patterns/06_return_results/main}
\input{patterns/061_pointers/main}
\input{patterns/065_GOTO/main}
\input{patterns/07_jcc/main}
\input{patterns/08_switch/main}
\input{patterns/09_loops/main}
\input{patterns/10_strings/main}
\input{patterns/11_arith_optimizations/main}
\input{patterns/12_FPU/main}
\input{patterns/13_arrays/main}
\input{patterns/14_bitfields/main}
\EN{\input{patterns/145_LCG/main_EN}}
\RU{\input{patterns/145_LCG/main_RU}}
\input{patterns/15_structs/main}
\input{patterns/17_unions/main}
\input{patterns/18_pointers_to_functions/main}
\input{patterns/185_64bit_in_32_env/main}

\EN{\input{patterns/19_SIMD/main_EN}}
\RU{\input{patterns/19_SIMD/main_RU}}
\DE{\input{patterns/19_SIMD/main_DE}}

\EN{\input{patterns/20_x64/main_EN}}
\RU{\input{patterns/20_x64/main_RU}}

\EN{\input{patterns/205_floating_SIMD/main_EN}}
\RU{\input{patterns/205_floating_SIMD/main_RU}}
\DE{\input{patterns/205_floating_SIMD/main_DE}}

\EN{\input{patterns/ARM/main_EN}}
\RU{\input{patterns/ARM/main_RU}}
\DE{\input{patterns/ARM/main_DE}}

\input{patterns/MIPS/main}

\ifdefined\SPANISH
\chapter{Patrones de código}
\fi % SPANISH

\ifdefined\GERMAN
\chapter{Code-Muster}
\fi % GERMAN

\ifdefined\ENGLISH
\chapter{Code Patterns}
\fi % ENGLISH

\ifdefined\ITALIAN
\chapter{Forme di codice}
\fi % ITALIAN

\ifdefined\RUSSIAN
\chapter{Образцы кода}
\fi % RUSSIAN

\ifdefined\BRAZILIAN
\chapter{Padrões de códigos}
\fi % BRAZILIAN

\ifdefined\THAI
\chapter{รูปแบบของโค้ด}
\fi % THAI

\ifdefined\FRENCH
\chapter{Modèle de code}
\fi % FRENCH

\ifdefined\POLISH
\chapter{\PLph{}}
\fi % POLISH

% sections
\EN{\input{patterns/patterns_opt_dbg_EN}}
\ES{\input{patterns/patterns_opt_dbg_ES}}
\ITA{\input{patterns/patterns_opt_dbg_ITA}}
\PTBR{\input{patterns/patterns_opt_dbg_PTBR}}
\RU{\input{patterns/patterns_opt_dbg_RU}}
\THA{\input{patterns/patterns_opt_dbg_THA}}
\DE{\input{patterns/patterns_opt_dbg_DE}}
\FR{\input{patterns/patterns_opt_dbg_FR}}
\PL{\input{patterns/patterns_opt_dbg_PL}}

\RU{\section{Некоторые базовые понятия}}
\EN{\section{Some basics}}
\DE{\section{Einige Grundlagen}}
\FR{\section{Quelques bases}}
\ES{\section{\ESph{}}}
\ITA{\section{Alcune basi teoriche}}
\PTBR{\section{\PTBRph{}}}
\THA{\section{\THAph{}}}
\PL{\section{\PLph{}}}

% sections:
\EN{\input{patterns/intro_CPU_ISA_EN}}
\ES{\input{patterns/intro_CPU_ISA_ES}}
\ITA{\input{patterns/intro_CPU_ISA_ITA}}
\PTBR{\input{patterns/intro_CPU_ISA_PTBR}}
\RU{\input{patterns/intro_CPU_ISA_RU}}
\DE{\input{patterns/intro_CPU_ISA_DE}}
\FR{\input{patterns/intro_CPU_ISA_FR}}
\PL{\input{patterns/intro_CPU_ISA_PL}}

\EN{\input{patterns/numeral_EN}}
\RU{\input{patterns/numeral_RU}}
\ITA{\input{patterns/numeral_ITA}}
\DE{\input{patterns/numeral_DE}}
\FR{\input{patterns/numeral_FR}}
\PL{\input{patterns/numeral_PL}}

% chapters
\input{patterns/00_empty/main}
\input{patterns/011_ret/main}
\input{patterns/01_helloworld/main}
\input{patterns/015_prolog_epilogue/main}
\input{patterns/02_stack/main}
\input{patterns/03_printf/main}
\input{patterns/04_scanf/main}
\input{patterns/05_passing_arguments/main}
\input{patterns/06_return_results/main}
\input{patterns/061_pointers/main}
\input{patterns/065_GOTO/main}
\input{patterns/07_jcc/main}
\input{patterns/08_switch/main}
\input{patterns/09_loops/main}
\input{patterns/10_strings/main}
\input{patterns/11_arith_optimizations/main}
\input{patterns/12_FPU/main}
\input{patterns/13_arrays/main}
\input{patterns/14_bitfields/main}
\EN{\input{patterns/145_LCG/main_EN}}
\RU{\input{patterns/145_LCG/main_RU}}
\input{patterns/15_structs/main}
\input{patterns/17_unions/main}
\input{patterns/18_pointers_to_functions/main}
\input{patterns/185_64bit_in_32_env/main}

\EN{\input{patterns/19_SIMD/main_EN}}
\RU{\input{patterns/19_SIMD/main_RU}}
\DE{\input{patterns/19_SIMD/main_DE}}

\EN{\input{patterns/20_x64/main_EN}}
\RU{\input{patterns/20_x64/main_RU}}

\EN{\input{patterns/205_floating_SIMD/main_EN}}
\RU{\input{patterns/205_floating_SIMD/main_RU}}
\DE{\input{patterns/205_floating_SIMD/main_DE}}

\EN{\input{patterns/ARM/main_EN}}
\RU{\input{patterns/ARM/main_RU}}
\DE{\input{patterns/ARM/main_DE}}

\input{patterns/MIPS/main}

\ifdefined\SPANISH
\chapter{Patrones de código}
\fi % SPANISH

\ifdefined\GERMAN
\chapter{Code-Muster}
\fi % GERMAN

\ifdefined\ENGLISH
\chapter{Code Patterns}
\fi % ENGLISH

\ifdefined\ITALIAN
\chapter{Forme di codice}
\fi % ITALIAN

\ifdefined\RUSSIAN
\chapter{Образцы кода}
\fi % RUSSIAN

\ifdefined\BRAZILIAN
\chapter{Padrões de códigos}
\fi % BRAZILIAN

\ifdefined\THAI
\chapter{รูปแบบของโค้ด}
\fi % THAI

\ifdefined\FRENCH
\chapter{Modèle de code}
\fi % FRENCH

\ifdefined\POLISH
\chapter{\PLph{}}
\fi % POLISH

% sections
\EN{\input{patterns/patterns_opt_dbg_EN}}
\ES{\input{patterns/patterns_opt_dbg_ES}}
\ITA{\input{patterns/patterns_opt_dbg_ITA}}
\PTBR{\input{patterns/patterns_opt_dbg_PTBR}}
\RU{\input{patterns/patterns_opt_dbg_RU}}
\THA{\input{patterns/patterns_opt_dbg_THA}}
\DE{\input{patterns/patterns_opt_dbg_DE}}
\FR{\input{patterns/patterns_opt_dbg_FR}}
\PL{\input{patterns/patterns_opt_dbg_PL}}

\RU{\section{Некоторые базовые понятия}}
\EN{\section{Some basics}}
\DE{\section{Einige Grundlagen}}
\FR{\section{Quelques bases}}
\ES{\section{\ESph{}}}
\ITA{\section{Alcune basi teoriche}}
\PTBR{\section{\PTBRph{}}}
\THA{\section{\THAph{}}}
\PL{\section{\PLph{}}}

% sections:
\EN{\input{patterns/intro_CPU_ISA_EN}}
\ES{\input{patterns/intro_CPU_ISA_ES}}
\ITA{\input{patterns/intro_CPU_ISA_ITA}}
\PTBR{\input{patterns/intro_CPU_ISA_PTBR}}
\RU{\input{patterns/intro_CPU_ISA_RU}}
\DE{\input{patterns/intro_CPU_ISA_DE}}
\FR{\input{patterns/intro_CPU_ISA_FR}}
\PL{\input{patterns/intro_CPU_ISA_PL}}

\EN{\input{patterns/numeral_EN}}
\RU{\input{patterns/numeral_RU}}
\ITA{\input{patterns/numeral_ITA}}
\DE{\input{patterns/numeral_DE}}
\FR{\input{patterns/numeral_FR}}
\PL{\input{patterns/numeral_PL}}

% chapters
\input{patterns/00_empty/main}
\input{patterns/011_ret/main}
\input{patterns/01_helloworld/main}
\input{patterns/015_prolog_epilogue/main}
\input{patterns/02_stack/main}
\input{patterns/03_printf/main}
\input{patterns/04_scanf/main}
\input{patterns/05_passing_arguments/main}
\input{patterns/06_return_results/main}
\input{patterns/061_pointers/main}
\input{patterns/065_GOTO/main}
\input{patterns/07_jcc/main}
\input{patterns/08_switch/main}
\input{patterns/09_loops/main}
\input{patterns/10_strings/main}
\input{patterns/11_arith_optimizations/main}
\input{patterns/12_FPU/main}
\input{patterns/13_arrays/main}
\input{patterns/14_bitfields/main}
\EN{\input{patterns/145_LCG/main_EN}}
\RU{\input{patterns/145_LCG/main_RU}}
\input{patterns/15_structs/main}
\input{patterns/17_unions/main}
\input{patterns/18_pointers_to_functions/main}
\input{patterns/185_64bit_in_32_env/main}

\EN{\input{patterns/19_SIMD/main_EN}}
\RU{\input{patterns/19_SIMD/main_RU}}
\DE{\input{patterns/19_SIMD/main_DE}}

\EN{\input{patterns/20_x64/main_EN}}
\RU{\input{patterns/20_x64/main_RU}}

\EN{\input{patterns/205_floating_SIMD/main_EN}}
\RU{\input{patterns/205_floating_SIMD/main_RU}}
\DE{\input{patterns/205_floating_SIMD/main_DE}}

\EN{\input{patterns/ARM/main_EN}}
\RU{\input{patterns/ARM/main_RU}}
\DE{\input{patterns/ARM/main_DE}}

\input{patterns/MIPS/main}

\ifdefined\SPANISH
\chapter{Patrones de código}
\fi % SPANISH

\ifdefined\GERMAN
\chapter{Code-Muster}
\fi % GERMAN

\ifdefined\ENGLISH
\chapter{Code Patterns}
\fi % ENGLISH

\ifdefined\ITALIAN
\chapter{Forme di codice}
\fi % ITALIAN

\ifdefined\RUSSIAN
\chapter{Образцы кода}
\fi % RUSSIAN

\ifdefined\BRAZILIAN
\chapter{Padrões de códigos}
\fi % BRAZILIAN

\ifdefined\THAI
\chapter{รูปแบบของโค้ด}
\fi % THAI

\ifdefined\FRENCH
\chapter{Modèle de code}
\fi % FRENCH

\ifdefined\POLISH
\chapter{\PLph{}}
\fi % POLISH

% sections
\EN{\input{patterns/patterns_opt_dbg_EN}}
\ES{\input{patterns/patterns_opt_dbg_ES}}
\ITA{\input{patterns/patterns_opt_dbg_ITA}}
\PTBR{\input{patterns/patterns_opt_dbg_PTBR}}
\RU{\input{patterns/patterns_opt_dbg_RU}}
\THA{\input{patterns/patterns_opt_dbg_THA}}
\DE{\input{patterns/patterns_opt_dbg_DE}}
\FR{\input{patterns/patterns_opt_dbg_FR}}
\PL{\input{patterns/patterns_opt_dbg_PL}}

\RU{\section{Некоторые базовые понятия}}
\EN{\section{Some basics}}
\DE{\section{Einige Grundlagen}}
\FR{\section{Quelques bases}}
\ES{\section{\ESph{}}}
\ITA{\section{Alcune basi teoriche}}
\PTBR{\section{\PTBRph{}}}
\THA{\section{\THAph{}}}
\PL{\section{\PLph{}}}

% sections:
\EN{\input{patterns/intro_CPU_ISA_EN}}
\ES{\input{patterns/intro_CPU_ISA_ES}}
\ITA{\input{patterns/intro_CPU_ISA_ITA}}
\PTBR{\input{patterns/intro_CPU_ISA_PTBR}}
\RU{\input{patterns/intro_CPU_ISA_RU}}
\DE{\input{patterns/intro_CPU_ISA_DE}}
\FR{\input{patterns/intro_CPU_ISA_FR}}
\PL{\input{patterns/intro_CPU_ISA_PL}}

\EN{\input{patterns/numeral_EN}}
\RU{\input{patterns/numeral_RU}}
\ITA{\input{patterns/numeral_ITA}}
\DE{\input{patterns/numeral_DE}}
\FR{\input{patterns/numeral_FR}}
\PL{\input{patterns/numeral_PL}}

% chapters
\input{patterns/00_empty/main}
\input{patterns/011_ret/main}
\input{patterns/01_helloworld/main}
\input{patterns/015_prolog_epilogue/main}
\input{patterns/02_stack/main}
\input{patterns/03_printf/main}
\input{patterns/04_scanf/main}
\input{patterns/05_passing_arguments/main}
\input{patterns/06_return_results/main}
\input{patterns/061_pointers/main}
\input{patterns/065_GOTO/main}
\input{patterns/07_jcc/main}
\input{patterns/08_switch/main}
\input{patterns/09_loops/main}
\input{patterns/10_strings/main}
\input{patterns/11_arith_optimizations/main}
\input{patterns/12_FPU/main}
\input{patterns/13_arrays/main}
\input{patterns/14_bitfields/main}
\EN{\input{patterns/145_LCG/main_EN}}
\RU{\input{patterns/145_LCG/main_RU}}
\input{patterns/15_structs/main}
\input{patterns/17_unions/main}
\input{patterns/18_pointers_to_functions/main}
\input{patterns/185_64bit_in_32_env/main}

\EN{\input{patterns/19_SIMD/main_EN}}
\RU{\input{patterns/19_SIMD/main_RU}}
\DE{\input{patterns/19_SIMD/main_DE}}

\EN{\input{patterns/20_x64/main_EN}}
\RU{\input{patterns/20_x64/main_RU}}

\EN{\input{patterns/205_floating_SIMD/main_EN}}
\RU{\input{patterns/205_floating_SIMD/main_RU}}
\DE{\input{patterns/205_floating_SIMD/main_DE}}

\EN{\input{patterns/ARM/main_EN}}
\RU{\input{patterns/ARM/main_RU}}
\DE{\input{patterns/ARM/main_DE}}

\input{patterns/MIPS/main}

\ifdefined\SPANISH
\chapter{Patrones de código}
\fi % SPANISH

\ifdefined\GERMAN
\chapter{Code-Muster}
\fi % GERMAN

\ifdefined\ENGLISH
\chapter{Code Patterns}
\fi % ENGLISH

\ifdefined\ITALIAN
\chapter{Forme di codice}
\fi % ITALIAN

\ifdefined\RUSSIAN
\chapter{Образцы кода}
\fi % RUSSIAN

\ifdefined\BRAZILIAN
\chapter{Padrões de códigos}
\fi % BRAZILIAN

\ifdefined\THAI
\chapter{รูปแบบของโค้ด}
\fi % THAI

\ifdefined\FRENCH
\chapter{Modèle de code}
\fi % FRENCH

\ifdefined\POLISH
\chapter{\PLph{}}
\fi % POLISH

% sections
\EN{\input{patterns/patterns_opt_dbg_EN}}
\ES{\input{patterns/patterns_opt_dbg_ES}}
\ITA{\input{patterns/patterns_opt_dbg_ITA}}
\PTBR{\input{patterns/patterns_opt_dbg_PTBR}}
\RU{\input{patterns/patterns_opt_dbg_RU}}
\THA{\input{patterns/patterns_opt_dbg_THA}}
\DE{\input{patterns/patterns_opt_dbg_DE}}
\FR{\input{patterns/patterns_opt_dbg_FR}}
\PL{\input{patterns/patterns_opt_dbg_PL}}

\RU{\section{Некоторые базовые понятия}}
\EN{\section{Some basics}}
\DE{\section{Einige Grundlagen}}
\FR{\section{Quelques bases}}
\ES{\section{\ESph{}}}
\ITA{\section{Alcune basi teoriche}}
\PTBR{\section{\PTBRph{}}}
\THA{\section{\THAph{}}}
\PL{\section{\PLph{}}}

% sections:
\EN{\input{patterns/intro_CPU_ISA_EN}}
\ES{\input{patterns/intro_CPU_ISA_ES}}
\ITA{\input{patterns/intro_CPU_ISA_ITA}}
\PTBR{\input{patterns/intro_CPU_ISA_PTBR}}
\RU{\input{patterns/intro_CPU_ISA_RU}}
\DE{\input{patterns/intro_CPU_ISA_DE}}
\FR{\input{patterns/intro_CPU_ISA_FR}}
\PL{\input{patterns/intro_CPU_ISA_PL}}

\EN{\input{patterns/numeral_EN}}
\RU{\input{patterns/numeral_RU}}
\ITA{\input{patterns/numeral_ITA}}
\DE{\input{patterns/numeral_DE}}
\FR{\input{patterns/numeral_FR}}
\PL{\input{patterns/numeral_PL}}

% chapters
\input{patterns/00_empty/main}
\input{patterns/011_ret/main}
\input{patterns/01_helloworld/main}
\input{patterns/015_prolog_epilogue/main}
\input{patterns/02_stack/main}
\input{patterns/03_printf/main}
\input{patterns/04_scanf/main}
\input{patterns/05_passing_arguments/main}
\input{patterns/06_return_results/main}
\input{patterns/061_pointers/main}
\input{patterns/065_GOTO/main}
\input{patterns/07_jcc/main}
\input{patterns/08_switch/main}
\input{patterns/09_loops/main}
\input{patterns/10_strings/main}
\input{patterns/11_arith_optimizations/main}
\input{patterns/12_FPU/main}
\input{patterns/13_arrays/main}
\input{patterns/14_bitfields/main}
\EN{\input{patterns/145_LCG/main_EN}}
\RU{\input{patterns/145_LCG/main_RU}}
\input{patterns/15_structs/main}
\input{patterns/17_unions/main}
\input{patterns/18_pointers_to_functions/main}
\input{patterns/185_64bit_in_32_env/main}

\EN{\input{patterns/19_SIMD/main_EN}}
\RU{\input{patterns/19_SIMD/main_RU}}
\DE{\input{patterns/19_SIMD/main_DE}}

\EN{\input{patterns/20_x64/main_EN}}
\RU{\input{patterns/20_x64/main_RU}}

\EN{\input{patterns/205_floating_SIMD/main_EN}}
\RU{\input{patterns/205_floating_SIMD/main_RU}}
\DE{\input{patterns/205_floating_SIMD/main_DE}}

\EN{\input{patterns/ARM/main_EN}}
\RU{\input{patterns/ARM/main_RU}}
\DE{\input{patterns/ARM/main_DE}}

\input{patterns/MIPS/main}

\ifdefined\SPANISH
\chapter{Patrones de código}
\fi % SPANISH

\ifdefined\GERMAN
\chapter{Code-Muster}
\fi % GERMAN

\ifdefined\ENGLISH
\chapter{Code Patterns}
\fi % ENGLISH

\ifdefined\ITALIAN
\chapter{Forme di codice}
\fi % ITALIAN

\ifdefined\RUSSIAN
\chapter{Образцы кода}
\fi % RUSSIAN

\ifdefined\BRAZILIAN
\chapter{Padrões de códigos}
\fi % BRAZILIAN

\ifdefined\THAI
\chapter{รูปแบบของโค้ด}
\fi % THAI

\ifdefined\FRENCH
\chapter{Modèle de code}
\fi % FRENCH

\ifdefined\POLISH
\chapter{\PLph{}}
\fi % POLISH

% sections
\EN{\input{patterns/patterns_opt_dbg_EN}}
\ES{\input{patterns/patterns_opt_dbg_ES}}
\ITA{\input{patterns/patterns_opt_dbg_ITA}}
\PTBR{\input{patterns/patterns_opt_dbg_PTBR}}
\RU{\input{patterns/patterns_opt_dbg_RU}}
\THA{\input{patterns/patterns_opt_dbg_THA}}
\DE{\input{patterns/patterns_opt_dbg_DE}}
\FR{\input{patterns/patterns_opt_dbg_FR}}
\PL{\input{patterns/patterns_opt_dbg_PL}}

\RU{\section{Некоторые базовые понятия}}
\EN{\section{Some basics}}
\DE{\section{Einige Grundlagen}}
\FR{\section{Quelques bases}}
\ES{\section{\ESph{}}}
\ITA{\section{Alcune basi teoriche}}
\PTBR{\section{\PTBRph{}}}
\THA{\section{\THAph{}}}
\PL{\section{\PLph{}}}

% sections:
\EN{\input{patterns/intro_CPU_ISA_EN}}
\ES{\input{patterns/intro_CPU_ISA_ES}}
\ITA{\input{patterns/intro_CPU_ISA_ITA}}
\PTBR{\input{patterns/intro_CPU_ISA_PTBR}}
\RU{\input{patterns/intro_CPU_ISA_RU}}
\DE{\input{patterns/intro_CPU_ISA_DE}}
\FR{\input{patterns/intro_CPU_ISA_FR}}
\PL{\input{patterns/intro_CPU_ISA_PL}}

\EN{\input{patterns/numeral_EN}}
\RU{\input{patterns/numeral_RU}}
\ITA{\input{patterns/numeral_ITA}}
\DE{\input{patterns/numeral_DE}}
\FR{\input{patterns/numeral_FR}}
\PL{\input{patterns/numeral_PL}}

% chapters
\input{patterns/00_empty/main}
\input{patterns/011_ret/main}
\input{patterns/01_helloworld/main}
\input{patterns/015_prolog_epilogue/main}
\input{patterns/02_stack/main}
\input{patterns/03_printf/main}
\input{patterns/04_scanf/main}
\input{patterns/05_passing_arguments/main}
\input{patterns/06_return_results/main}
\input{patterns/061_pointers/main}
\input{patterns/065_GOTO/main}
\input{patterns/07_jcc/main}
\input{patterns/08_switch/main}
\input{patterns/09_loops/main}
\input{patterns/10_strings/main}
\input{patterns/11_arith_optimizations/main}
\input{patterns/12_FPU/main}
\input{patterns/13_arrays/main}
\input{patterns/14_bitfields/main}
\EN{\input{patterns/145_LCG/main_EN}}
\RU{\input{patterns/145_LCG/main_RU}}
\input{patterns/15_structs/main}
\input{patterns/17_unions/main}
\input{patterns/18_pointers_to_functions/main}
\input{patterns/185_64bit_in_32_env/main}

\EN{\input{patterns/19_SIMD/main_EN}}
\RU{\input{patterns/19_SIMD/main_RU}}
\DE{\input{patterns/19_SIMD/main_DE}}

\EN{\input{patterns/20_x64/main_EN}}
\RU{\input{patterns/20_x64/main_RU}}

\EN{\input{patterns/205_floating_SIMD/main_EN}}
\RU{\input{patterns/205_floating_SIMD/main_RU}}
\DE{\input{patterns/205_floating_SIMD/main_DE}}

\EN{\input{patterns/ARM/main_EN}}
\RU{\input{patterns/ARM/main_RU}}
\DE{\input{patterns/ARM/main_DE}}

\input{patterns/MIPS/main}

\ifdefined\SPANISH
\chapter{Patrones de código}
\fi % SPANISH

\ifdefined\GERMAN
\chapter{Code-Muster}
\fi % GERMAN

\ifdefined\ENGLISH
\chapter{Code Patterns}
\fi % ENGLISH

\ifdefined\ITALIAN
\chapter{Forme di codice}
\fi % ITALIAN

\ifdefined\RUSSIAN
\chapter{Образцы кода}
\fi % RUSSIAN

\ifdefined\BRAZILIAN
\chapter{Padrões de códigos}
\fi % BRAZILIAN

\ifdefined\THAI
\chapter{รูปแบบของโค้ด}
\fi % THAI

\ifdefined\FRENCH
\chapter{Modèle de code}
\fi % FRENCH

\ifdefined\POLISH
\chapter{\PLph{}}
\fi % POLISH

% sections
\EN{\input{patterns/patterns_opt_dbg_EN}}
\ES{\input{patterns/patterns_opt_dbg_ES}}
\ITA{\input{patterns/patterns_opt_dbg_ITA}}
\PTBR{\input{patterns/patterns_opt_dbg_PTBR}}
\RU{\input{patterns/patterns_opt_dbg_RU}}
\THA{\input{patterns/patterns_opt_dbg_THA}}
\DE{\input{patterns/patterns_opt_dbg_DE}}
\FR{\input{patterns/patterns_opt_dbg_FR}}
\PL{\input{patterns/patterns_opt_dbg_PL}}

\RU{\section{Некоторые базовые понятия}}
\EN{\section{Some basics}}
\DE{\section{Einige Grundlagen}}
\FR{\section{Quelques bases}}
\ES{\section{\ESph{}}}
\ITA{\section{Alcune basi teoriche}}
\PTBR{\section{\PTBRph{}}}
\THA{\section{\THAph{}}}
\PL{\section{\PLph{}}}

% sections:
\EN{\input{patterns/intro_CPU_ISA_EN}}
\ES{\input{patterns/intro_CPU_ISA_ES}}
\ITA{\input{patterns/intro_CPU_ISA_ITA}}
\PTBR{\input{patterns/intro_CPU_ISA_PTBR}}
\RU{\input{patterns/intro_CPU_ISA_RU}}
\DE{\input{patterns/intro_CPU_ISA_DE}}
\FR{\input{patterns/intro_CPU_ISA_FR}}
\PL{\input{patterns/intro_CPU_ISA_PL}}

\EN{\input{patterns/numeral_EN}}
\RU{\input{patterns/numeral_RU}}
\ITA{\input{patterns/numeral_ITA}}
\DE{\input{patterns/numeral_DE}}
\FR{\input{patterns/numeral_FR}}
\PL{\input{patterns/numeral_PL}}

% chapters
\input{patterns/00_empty/main}
\input{patterns/011_ret/main}
\input{patterns/01_helloworld/main}
\input{patterns/015_prolog_epilogue/main}
\input{patterns/02_stack/main}
\input{patterns/03_printf/main}
\input{patterns/04_scanf/main}
\input{patterns/05_passing_arguments/main}
\input{patterns/06_return_results/main}
\input{patterns/061_pointers/main}
\input{patterns/065_GOTO/main}
\input{patterns/07_jcc/main}
\input{patterns/08_switch/main}
\input{patterns/09_loops/main}
\input{patterns/10_strings/main}
\input{patterns/11_arith_optimizations/main}
\input{patterns/12_FPU/main}
\input{patterns/13_arrays/main}
\input{patterns/14_bitfields/main}
\EN{\input{patterns/145_LCG/main_EN}}
\RU{\input{patterns/145_LCG/main_RU}}
\input{patterns/15_structs/main}
\input{patterns/17_unions/main}
\input{patterns/18_pointers_to_functions/main}
\input{patterns/185_64bit_in_32_env/main}

\EN{\input{patterns/19_SIMD/main_EN}}
\RU{\input{patterns/19_SIMD/main_RU}}
\DE{\input{patterns/19_SIMD/main_DE}}

\EN{\input{patterns/20_x64/main_EN}}
\RU{\input{patterns/20_x64/main_RU}}

\EN{\input{patterns/205_floating_SIMD/main_EN}}
\RU{\input{patterns/205_floating_SIMD/main_RU}}
\DE{\input{patterns/205_floating_SIMD/main_DE}}

\EN{\input{patterns/ARM/main_EN}}
\RU{\input{patterns/ARM/main_RU}}
\DE{\input{patterns/ARM/main_DE}}

\input{patterns/MIPS/main}

\ifdefined\SPANISH
\chapter{Patrones de código}
\fi % SPANISH

\ifdefined\GERMAN
\chapter{Code-Muster}
\fi % GERMAN

\ifdefined\ENGLISH
\chapter{Code Patterns}
\fi % ENGLISH

\ifdefined\ITALIAN
\chapter{Forme di codice}
\fi % ITALIAN

\ifdefined\RUSSIAN
\chapter{Образцы кода}
\fi % RUSSIAN

\ifdefined\BRAZILIAN
\chapter{Padrões de códigos}
\fi % BRAZILIAN

\ifdefined\THAI
\chapter{รูปแบบของโค้ด}
\fi % THAI

\ifdefined\FRENCH
\chapter{Modèle de code}
\fi % FRENCH

\ifdefined\POLISH
\chapter{\PLph{}}
\fi % POLISH

% sections
\EN{\input{patterns/patterns_opt_dbg_EN}}
\ES{\input{patterns/patterns_opt_dbg_ES}}
\ITA{\input{patterns/patterns_opt_dbg_ITA}}
\PTBR{\input{patterns/patterns_opt_dbg_PTBR}}
\RU{\input{patterns/patterns_opt_dbg_RU}}
\THA{\input{patterns/patterns_opt_dbg_THA}}
\DE{\input{patterns/patterns_opt_dbg_DE}}
\FR{\input{patterns/patterns_opt_dbg_FR}}
\PL{\input{patterns/patterns_opt_dbg_PL}}

\RU{\section{Некоторые базовые понятия}}
\EN{\section{Some basics}}
\DE{\section{Einige Grundlagen}}
\FR{\section{Quelques bases}}
\ES{\section{\ESph{}}}
\ITA{\section{Alcune basi teoriche}}
\PTBR{\section{\PTBRph{}}}
\THA{\section{\THAph{}}}
\PL{\section{\PLph{}}}

% sections:
\EN{\input{patterns/intro_CPU_ISA_EN}}
\ES{\input{patterns/intro_CPU_ISA_ES}}
\ITA{\input{patterns/intro_CPU_ISA_ITA}}
\PTBR{\input{patterns/intro_CPU_ISA_PTBR}}
\RU{\input{patterns/intro_CPU_ISA_RU}}
\DE{\input{patterns/intro_CPU_ISA_DE}}
\FR{\input{patterns/intro_CPU_ISA_FR}}
\PL{\input{patterns/intro_CPU_ISA_PL}}

\EN{\input{patterns/numeral_EN}}
\RU{\input{patterns/numeral_RU}}
\ITA{\input{patterns/numeral_ITA}}
\DE{\input{patterns/numeral_DE}}
\FR{\input{patterns/numeral_FR}}
\PL{\input{patterns/numeral_PL}}

% chapters
\input{patterns/00_empty/main}
\input{patterns/011_ret/main}
\input{patterns/01_helloworld/main}
\input{patterns/015_prolog_epilogue/main}
\input{patterns/02_stack/main}
\input{patterns/03_printf/main}
\input{patterns/04_scanf/main}
\input{patterns/05_passing_arguments/main}
\input{patterns/06_return_results/main}
\input{patterns/061_pointers/main}
\input{patterns/065_GOTO/main}
\input{patterns/07_jcc/main}
\input{patterns/08_switch/main}
\input{patterns/09_loops/main}
\input{patterns/10_strings/main}
\input{patterns/11_arith_optimizations/main}
\input{patterns/12_FPU/main}
\input{patterns/13_arrays/main}
\input{patterns/14_bitfields/main}
\EN{\input{patterns/145_LCG/main_EN}}
\RU{\input{patterns/145_LCG/main_RU}}
\input{patterns/15_structs/main}
\input{patterns/17_unions/main}
\input{patterns/18_pointers_to_functions/main}
\input{patterns/185_64bit_in_32_env/main}

\EN{\input{patterns/19_SIMD/main_EN}}
\RU{\input{patterns/19_SIMD/main_RU}}
\DE{\input{patterns/19_SIMD/main_DE}}

\EN{\input{patterns/20_x64/main_EN}}
\RU{\input{patterns/20_x64/main_RU}}

\EN{\input{patterns/205_floating_SIMD/main_EN}}
\RU{\input{patterns/205_floating_SIMD/main_RU}}
\DE{\input{patterns/205_floating_SIMD/main_DE}}

\EN{\input{patterns/ARM/main_EN}}
\RU{\input{patterns/ARM/main_RU}}
\DE{\input{patterns/ARM/main_DE}}

\input{patterns/MIPS/main}

\ifdefined\SPANISH
\chapter{Patrones de código}
\fi % SPANISH

\ifdefined\GERMAN
\chapter{Code-Muster}
\fi % GERMAN

\ifdefined\ENGLISH
\chapter{Code Patterns}
\fi % ENGLISH

\ifdefined\ITALIAN
\chapter{Forme di codice}
\fi % ITALIAN

\ifdefined\RUSSIAN
\chapter{Образцы кода}
\fi % RUSSIAN

\ifdefined\BRAZILIAN
\chapter{Padrões de códigos}
\fi % BRAZILIAN

\ifdefined\THAI
\chapter{รูปแบบของโค้ด}
\fi % THAI

\ifdefined\FRENCH
\chapter{Modèle de code}
\fi % FRENCH

\ifdefined\POLISH
\chapter{\PLph{}}
\fi % POLISH

% sections
\EN{\input{patterns/patterns_opt_dbg_EN}}
\ES{\input{patterns/patterns_opt_dbg_ES}}
\ITA{\input{patterns/patterns_opt_dbg_ITA}}
\PTBR{\input{patterns/patterns_opt_dbg_PTBR}}
\RU{\input{patterns/patterns_opt_dbg_RU}}
\THA{\input{patterns/patterns_opt_dbg_THA}}
\DE{\input{patterns/patterns_opt_dbg_DE}}
\FR{\input{patterns/patterns_opt_dbg_FR}}
\PL{\input{patterns/patterns_opt_dbg_PL}}

\RU{\section{Некоторые базовые понятия}}
\EN{\section{Some basics}}
\DE{\section{Einige Grundlagen}}
\FR{\section{Quelques bases}}
\ES{\section{\ESph{}}}
\ITA{\section{Alcune basi teoriche}}
\PTBR{\section{\PTBRph{}}}
\THA{\section{\THAph{}}}
\PL{\section{\PLph{}}}

% sections:
\EN{\input{patterns/intro_CPU_ISA_EN}}
\ES{\input{patterns/intro_CPU_ISA_ES}}
\ITA{\input{patterns/intro_CPU_ISA_ITA}}
\PTBR{\input{patterns/intro_CPU_ISA_PTBR}}
\RU{\input{patterns/intro_CPU_ISA_RU}}
\DE{\input{patterns/intro_CPU_ISA_DE}}
\FR{\input{patterns/intro_CPU_ISA_FR}}
\PL{\input{patterns/intro_CPU_ISA_PL}}

\EN{\input{patterns/numeral_EN}}
\RU{\input{patterns/numeral_RU}}
\ITA{\input{patterns/numeral_ITA}}
\DE{\input{patterns/numeral_DE}}
\FR{\input{patterns/numeral_FR}}
\PL{\input{patterns/numeral_PL}}

% chapters
\input{patterns/00_empty/main}
\input{patterns/011_ret/main}
\input{patterns/01_helloworld/main}
\input{patterns/015_prolog_epilogue/main}
\input{patterns/02_stack/main}
\input{patterns/03_printf/main}
\input{patterns/04_scanf/main}
\input{patterns/05_passing_arguments/main}
\input{patterns/06_return_results/main}
\input{patterns/061_pointers/main}
\input{patterns/065_GOTO/main}
\input{patterns/07_jcc/main}
\input{patterns/08_switch/main}
\input{patterns/09_loops/main}
\input{patterns/10_strings/main}
\input{patterns/11_arith_optimizations/main}
\input{patterns/12_FPU/main}
\input{patterns/13_arrays/main}
\input{patterns/14_bitfields/main}
\EN{\input{patterns/145_LCG/main_EN}}
\RU{\input{patterns/145_LCG/main_RU}}
\input{patterns/15_structs/main}
\input{patterns/17_unions/main}
\input{patterns/18_pointers_to_functions/main}
\input{patterns/185_64bit_in_32_env/main}

\EN{\input{patterns/19_SIMD/main_EN}}
\RU{\input{patterns/19_SIMD/main_RU}}
\DE{\input{patterns/19_SIMD/main_DE}}

\EN{\input{patterns/20_x64/main_EN}}
\RU{\input{patterns/20_x64/main_RU}}

\EN{\input{patterns/205_floating_SIMD/main_EN}}
\RU{\input{patterns/205_floating_SIMD/main_RU}}
\DE{\input{patterns/205_floating_SIMD/main_DE}}

\EN{\input{patterns/ARM/main_EN}}
\RU{\input{patterns/ARM/main_RU}}
\DE{\input{patterns/ARM/main_DE}}

\input{patterns/MIPS/main}

\ifdefined\SPANISH
\chapter{Patrones de código}
\fi % SPANISH

\ifdefined\GERMAN
\chapter{Code-Muster}
\fi % GERMAN

\ifdefined\ENGLISH
\chapter{Code Patterns}
\fi % ENGLISH

\ifdefined\ITALIAN
\chapter{Forme di codice}
\fi % ITALIAN

\ifdefined\RUSSIAN
\chapter{Образцы кода}
\fi % RUSSIAN

\ifdefined\BRAZILIAN
\chapter{Padrões de códigos}
\fi % BRAZILIAN

\ifdefined\THAI
\chapter{รูปแบบของโค้ด}
\fi % THAI

\ifdefined\FRENCH
\chapter{Modèle de code}
\fi % FRENCH

\ifdefined\POLISH
\chapter{\PLph{}}
\fi % POLISH

% sections
\EN{\input{patterns/patterns_opt_dbg_EN}}
\ES{\input{patterns/patterns_opt_dbg_ES}}
\ITA{\input{patterns/patterns_opt_dbg_ITA}}
\PTBR{\input{patterns/patterns_opt_dbg_PTBR}}
\RU{\input{patterns/patterns_opt_dbg_RU}}
\THA{\input{patterns/patterns_opt_dbg_THA}}
\DE{\input{patterns/patterns_opt_dbg_DE}}
\FR{\input{patterns/patterns_opt_dbg_FR}}
\PL{\input{patterns/patterns_opt_dbg_PL}}

\RU{\section{Некоторые базовые понятия}}
\EN{\section{Some basics}}
\DE{\section{Einige Grundlagen}}
\FR{\section{Quelques bases}}
\ES{\section{\ESph{}}}
\ITA{\section{Alcune basi teoriche}}
\PTBR{\section{\PTBRph{}}}
\THA{\section{\THAph{}}}
\PL{\section{\PLph{}}}

% sections:
\EN{\input{patterns/intro_CPU_ISA_EN}}
\ES{\input{patterns/intro_CPU_ISA_ES}}
\ITA{\input{patterns/intro_CPU_ISA_ITA}}
\PTBR{\input{patterns/intro_CPU_ISA_PTBR}}
\RU{\input{patterns/intro_CPU_ISA_RU}}
\DE{\input{patterns/intro_CPU_ISA_DE}}
\FR{\input{patterns/intro_CPU_ISA_FR}}
\PL{\input{patterns/intro_CPU_ISA_PL}}

\EN{\input{patterns/numeral_EN}}
\RU{\input{patterns/numeral_RU}}
\ITA{\input{patterns/numeral_ITA}}
\DE{\input{patterns/numeral_DE}}
\FR{\input{patterns/numeral_FR}}
\PL{\input{patterns/numeral_PL}}

% chapters
\input{patterns/00_empty/main}
\input{patterns/011_ret/main}
\input{patterns/01_helloworld/main}
\input{patterns/015_prolog_epilogue/main}
\input{patterns/02_stack/main}
\input{patterns/03_printf/main}
\input{patterns/04_scanf/main}
\input{patterns/05_passing_arguments/main}
\input{patterns/06_return_results/main}
\input{patterns/061_pointers/main}
\input{patterns/065_GOTO/main}
\input{patterns/07_jcc/main}
\input{patterns/08_switch/main}
\input{patterns/09_loops/main}
\input{patterns/10_strings/main}
\input{patterns/11_arith_optimizations/main}
\input{patterns/12_FPU/main}
\input{patterns/13_arrays/main}
\input{patterns/14_bitfields/main}
\EN{\input{patterns/145_LCG/main_EN}}
\RU{\input{patterns/145_LCG/main_RU}}
\input{patterns/15_structs/main}
\input{patterns/17_unions/main}
\input{patterns/18_pointers_to_functions/main}
\input{patterns/185_64bit_in_32_env/main}

\EN{\input{patterns/19_SIMD/main_EN}}
\RU{\input{patterns/19_SIMD/main_RU}}
\DE{\input{patterns/19_SIMD/main_DE}}

\EN{\input{patterns/20_x64/main_EN}}
\RU{\input{patterns/20_x64/main_RU}}

\EN{\input{patterns/205_floating_SIMD/main_EN}}
\RU{\input{patterns/205_floating_SIMD/main_RU}}
\DE{\input{patterns/205_floating_SIMD/main_DE}}

\EN{\input{patterns/ARM/main_EN}}
\RU{\input{patterns/ARM/main_RU}}
\DE{\input{patterns/ARM/main_DE}}

\input{patterns/MIPS/main}

\EN{\input{patterns/12_FPU/main_EN}}
\RU{\input{patterns/12_FPU/main_RU}}
\DE{\input{patterns/12_FPU/main_DE}}
\FR{\input{patterns/12_FPU/main_FR}}


\ifdefined\SPANISH
\chapter{Patrones de código}
\fi % SPANISH

\ifdefined\GERMAN
\chapter{Code-Muster}
\fi % GERMAN

\ifdefined\ENGLISH
\chapter{Code Patterns}
\fi % ENGLISH

\ifdefined\ITALIAN
\chapter{Forme di codice}
\fi % ITALIAN

\ifdefined\RUSSIAN
\chapter{Образцы кода}
\fi % RUSSIAN

\ifdefined\BRAZILIAN
\chapter{Padrões de códigos}
\fi % BRAZILIAN

\ifdefined\THAI
\chapter{รูปแบบของโค้ด}
\fi % THAI

\ifdefined\FRENCH
\chapter{Modèle de code}
\fi % FRENCH

\ifdefined\POLISH
\chapter{\PLph{}}
\fi % POLISH

% sections
\EN{\input{patterns/patterns_opt_dbg_EN}}
\ES{\input{patterns/patterns_opt_dbg_ES}}
\ITA{\input{patterns/patterns_opt_dbg_ITA}}
\PTBR{\input{patterns/patterns_opt_dbg_PTBR}}
\RU{\input{patterns/patterns_opt_dbg_RU}}
\THA{\input{patterns/patterns_opt_dbg_THA}}
\DE{\input{patterns/patterns_opt_dbg_DE}}
\FR{\input{patterns/patterns_opt_dbg_FR}}
\PL{\input{patterns/patterns_opt_dbg_PL}}

\RU{\section{Некоторые базовые понятия}}
\EN{\section{Some basics}}
\DE{\section{Einige Grundlagen}}
\FR{\section{Quelques bases}}
\ES{\section{\ESph{}}}
\ITA{\section{Alcune basi teoriche}}
\PTBR{\section{\PTBRph{}}}
\THA{\section{\THAph{}}}
\PL{\section{\PLph{}}}

% sections:
\EN{\input{patterns/intro_CPU_ISA_EN}}
\ES{\input{patterns/intro_CPU_ISA_ES}}
\ITA{\input{patterns/intro_CPU_ISA_ITA}}
\PTBR{\input{patterns/intro_CPU_ISA_PTBR}}
\RU{\input{patterns/intro_CPU_ISA_RU}}
\DE{\input{patterns/intro_CPU_ISA_DE}}
\FR{\input{patterns/intro_CPU_ISA_FR}}
\PL{\input{patterns/intro_CPU_ISA_PL}}

\EN{\input{patterns/numeral_EN}}
\RU{\input{patterns/numeral_RU}}
\ITA{\input{patterns/numeral_ITA}}
\DE{\input{patterns/numeral_DE}}
\FR{\input{patterns/numeral_FR}}
\PL{\input{patterns/numeral_PL}}

% chapters
\input{patterns/00_empty/main}
\input{patterns/011_ret/main}
\input{patterns/01_helloworld/main}
\input{patterns/015_prolog_epilogue/main}
\input{patterns/02_stack/main}
\input{patterns/03_printf/main}
\input{patterns/04_scanf/main}
\input{patterns/05_passing_arguments/main}
\input{patterns/06_return_results/main}
\input{patterns/061_pointers/main}
\input{patterns/065_GOTO/main}
\input{patterns/07_jcc/main}
\input{patterns/08_switch/main}
\input{patterns/09_loops/main}
\input{patterns/10_strings/main}
\input{patterns/11_arith_optimizations/main}
\input{patterns/12_FPU/main}
\input{patterns/13_arrays/main}
\input{patterns/14_bitfields/main}
\EN{\input{patterns/145_LCG/main_EN}}
\RU{\input{patterns/145_LCG/main_RU}}
\input{patterns/15_structs/main}
\input{patterns/17_unions/main}
\input{patterns/18_pointers_to_functions/main}
\input{patterns/185_64bit_in_32_env/main}

\EN{\input{patterns/19_SIMD/main_EN}}
\RU{\input{patterns/19_SIMD/main_RU}}
\DE{\input{patterns/19_SIMD/main_DE}}

\EN{\input{patterns/20_x64/main_EN}}
\RU{\input{patterns/20_x64/main_RU}}

\EN{\input{patterns/205_floating_SIMD/main_EN}}
\RU{\input{patterns/205_floating_SIMD/main_RU}}
\DE{\input{patterns/205_floating_SIMD/main_DE}}

\EN{\input{patterns/ARM/main_EN}}
\RU{\input{patterns/ARM/main_RU}}
\DE{\input{patterns/ARM/main_DE}}

\input{patterns/MIPS/main}

\ifdefined\SPANISH
\chapter{Patrones de código}
\fi % SPANISH

\ifdefined\GERMAN
\chapter{Code-Muster}
\fi % GERMAN

\ifdefined\ENGLISH
\chapter{Code Patterns}
\fi % ENGLISH

\ifdefined\ITALIAN
\chapter{Forme di codice}
\fi % ITALIAN

\ifdefined\RUSSIAN
\chapter{Образцы кода}
\fi % RUSSIAN

\ifdefined\BRAZILIAN
\chapter{Padrões de códigos}
\fi % BRAZILIAN

\ifdefined\THAI
\chapter{รูปแบบของโค้ด}
\fi % THAI

\ifdefined\FRENCH
\chapter{Modèle de code}
\fi % FRENCH

\ifdefined\POLISH
\chapter{\PLph{}}
\fi % POLISH

% sections
\EN{\input{patterns/patterns_opt_dbg_EN}}
\ES{\input{patterns/patterns_opt_dbg_ES}}
\ITA{\input{patterns/patterns_opt_dbg_ITA}}
\PTBR{\input{patterns/patterns_opt_dbg_PTBR}}
\RU{\input{patterns/patterns_opt_dbg_RU}}
\THA{\input{patterns/patterns_opt_dbg_THA}}
\DE{\input{patterns/patterns_opt_dbg_DE}}
\FR{\input{patterns/patterns_opt_dbg_FR}}
\PL{\input{patterns/patterns_opt_dbg_PL}}

\RU{\section{Некоторые базовые понятия}}
\EN{\section{Some basics}}
\DE{\section{Einige Grundlagen}}
\FR{\section{Quelques bases}}
\ES{\section{\ESph{}}}
\ITA{\section{Alcune basi teoriche}}
\PTBR{\section{\PTBRph{}}}
\THA{\section{\THAph{}}}
\PL{\section{\PLph{}}}

% sections:
\EN{\input{patterns/intro_CPU_ISA_EN}}
\ES{\input{patterns/intro_CPU_ISA_ES}}
\ITA{\input{patterns/intro_CPU_ISA_ITA}}
\PTBR{\input{patterns/intro_CPU_ISA_PTBR}}
\RU{\input{patterns/intro_CPU_ISA_RU}}
\DE{\input{patterns/intro_CPU_ISA_DE}}
\FR{\input{patterns/intro_CPU_ISA_FR}}
\PL{\input{patterns/intro_CPU_ISA_PL}}

\EN{\input{patterns/numeral_EN}}
\RU{\input{patterns/numeral_RU}}
\ITA{\input{patterns/numeral_ITA}}
\DE{\input{patterns/numeral_DE}}
\FR{\input{patterns/numeral_FR}}
\PL{\input{patterns/numeral_PL}}

% chapters
\input{patterns/00_empty/main}
\input{patterns/011_ret/main}
\input{patterns/01_helloworld/main}
\input{patterns/015_prolog_epilogue/main}
\input{patterns/02_stack/main}
\input{patterns/03_printf/main}
\input{patterns/04_scanf/main}
\input{patterns/05_passing_arguments/main}
\input{patterns/06_return_results/main}
\input{patterns/061_pointers/main}
\input{patterns/065_GOTO/main}
\input{patterns/07_jcc/main}
\input{patterns/08_switch/main}
\input{patterns/09_loops/main}
\input{patterns/10_strings/main}
\input{patterns/11_arith_optimizations/main}
\input{patterns/12_FPU/main}
\input{patterns/13_arrays/main}
\input{patterns/14_bitfields/main}
\EN{\input{patterns/145_LCG/main_EN}}
\RU{\input{patterns/145_LCG/main_RU}}
\input{patterns/15_structs/main}
\input{patterns/17_unions/main}
\input{patterns/18_pointers_to_functions/main}
\input{patterns/185_64bit_in_32_env/main}

\EN{\input{patterns/19_SIMD/main_EN}}
\RU{\input{patterns/19_SIMD/main_RU}}
\DE{\input{patterns/19_SIMD/main_DE}}

\EN{\input{patterns/20_x64/main_EN}}
\RU{\input{patterns/20_x64/main_RU}}

\EN{\input{patterns/205_floating_SIMD/main_EN}}
\RU{\input{patterns/205_floating_SIMD/main_RU}}
\DE{\input{patterns/205_floating_SIMD/main_DE}}

\EN{\input{patterns/ARM/main_EN}}
\RU{\input{patterns/ARM/main_RU}}
\DE{\input{patterns/ARM/main_DE}}

\input{patterns/MIPS/main}

\EN{\section{Returning Values}
\label{ret_val_func}

Another simple function is the one that simply returns a constant value:

\lstinputlisting[caption=\EN{\CCpp Code},style=customc]{patterns/011_ret/1.c}

Let's compile it.

\subsection{x86}

Here's what both the GCC and MSVC compilers produce (with optimization) on the x86 platform:

\lstinputlisting[caption=\Optimizing GCC/MSVC (\assemblyOutput),style=customasmx86]{patterns/011_ret/1.s}

\myindex{x86!\Instructions!RET}
There are just two instructions: the first places the value 123 into the \EAX register,
which is used by convention for storing the return
value, and the second one is \RET, which returns execution to the \gls{caller}.

The caller will take the result from the \EAX register.

\subsection{ARM}

There are a few differences on the ARM platform:

\lstinputlisting[caption=\OptimizingKeilVI (\ARMMode) ASM Output,style=customasmARM]{patterns/011_ret/1_Keil_ARM_O3.s}

ARM uses the register \Reg{0} for returning the results of functions, so 123 is copied into \Reg{0}.

\myindex{ARM!\Instructions!MOV}
\myindex{x86!\Instructions!MOV}
It is worth noting that \MOV is a misleading name for the instruction in both the x86 and ARM \ac{ISA}s.

The data is not in fact \IT{moved}, but \IT{copied}.

\subsection{MIPS}

\label{MIPS_leaf_function_ex1}

The GCC assembly output below lists registers by number:

\lstinputlisting[caption=\Optimizing GCC 4.4.5 (\assemblyOutput),style=customasmMIPS]{patterns/011_ret/MIPS.s}

\dots while \IDA does it by their pseudo names:

\lstinputlisting[caption=\Optimizing GCC 4.4.5 (IDA),style=customasmMIPS]{patterns/011_ret/MIPS_IDA.lst}

The \$2 (or \$V0) register is used to store the function's return value.
\myindex{MIPS!\Pseudoinstructions!LI}
\INS{LI} stands for ``Load Immediate'' and is the MIPS equivalent to \MOV.

\myindex{MIPS!\Instructions!J}
The other instruction is the jump instruction (J or JR) which returns the execution flow to the \gls{caller}.

\myindex{MIPS!Branch delay slot}
You might be wondering why the positions of the load instruction (LI) and the jump instruction (J or JR) are swapped. This is due to a \ac{RISC} feature called ``branch delay slot''.

The reason this happens is a quirk in the architecture of some RISC \ac{ISA}s and isn't important for our
purposes---we must simply keep in mind that in MIPS, the instruction following a jump or branch instruction
is executed \IT{before} the jump/branch instruction itself.

As a consequence, branch instructions always swap places with the instruction executed immediately beforehand.


In practice, functions which merely return 1 (\IT{true}) or 0 (\IT{false}) are very frequent.

The smallest ever of the standard UNIX utilities, \IT{/bin/true} and \IT{/bin/false} return 0 and 1 respectively, as an exit code.
(Zero as an exit code usually means success, non-zero means error.)
}
\RU{\subsubsection{std::string}
\myindex{\Cpp!STL!std::string}
\label{std_string}

\myparagraph{Как устроена структура}

Многие строковые библиотеки \InSqBrackets{\CNotes 2.2} обеспечивают структуру содержащую ссылку 
на буфер собственно со строкой, переменная всегда содержащую длину строки 
(что очень удобно для массы функций \InSqBrackets{\CNotes 2.2.1}) и переменную содержащую текущий размер буфера.

Строка в буфере обыкновенно оканчивается нулем: это для того чтобы указатель на буфер можно было
передавать в функции требующие на вход обычную сишную \ac{ASCIIZ}-строку.

Стандарт \Cpp не описывает, как именно нужно реализовывать std::string,
но, как правило, они реализованы как описано выше, с небольшими дополнениями.

Строки в \Cpp это не класс (как, например, QString в Qt), а темплейт (basic\_string), 
это сделано для того чтобы поддерживать 
строки содержащие разного типа символы: как минимум \Tchar и \IT{wchar\_t}.

Так что, std::string это класс с базовым типом \Tchar.

А std::wstring это класс с базовым типом \IT{wchar\_t}.

\mysubparagraph{MSVC}

В реализации MSVC, вместо ссылки на буфер может содержаться сам буфер (если строка короче 16-и символов).

Это означает, что каждая короткая строка будет занимать в памяти по крайней мере $16 + 4 + 4 = 24$ 
байт для 32-битной среды либо $16 + 8 + 8 = 32$ 
байта в 64-битной, а если строка длиннее 16-и символов, то прибавьте еще длину самой строки.

\lstinputlisting[caption=пример для MSVC,style=customc]{\CURPATH/STL/string/MSVC_RU.cpp}

Собственно, из этого исходника почти всё ясно.

Несколько замечаний:

Если строка короче 16-и символов, 
то отдельный буфер для строки в \glslink{heap}{куче} выделяться не будет.

Это удобно потому что на практике, основная часть строк действительно короткие.
Вероятно, разработчики в Microsoft выбрали размер в 16 символов как разумный баланс.

Теперь очень важный момент в конце функции main(): мы не пользуемся методом c\_str(), тем не менее,
если это скомпилировать и запустить, то обе строки появятся в консоли!

Работает это вот почему.

В первом случае строка короче 16-и символов и в начале объекта std::string (его можно рассматривать
просто как структуру) расположен буфер с этой строкой.
\printf трактует указатель как указатель на массив символов оканчивающийся нулем и поэтому всё работает.

Вывод второй строки (длиннее 16-и символов) даже еще опаснее: это вообще типичная программистская ошибка 
(или опечатка), забыть дописать c\_str().
Это работает потому что в это время в начале структуры расположен указатель на буфер.
Это может надолго остаться незамеченным: до тех пока там не появится строка 
короче 16-и символов, тогда процесс упадет.

\mysubparagraph{GCC}

В реализации GCC в структуре есть еще одна переменная --- reference count.

Интересно, что указатель на экземпляр класса std::string в GCC указывает не на начало самой структуры, 
а на указатель на буфера.
В libstdc++-v3\textbackslash{}include\textbackslash{}bits\textbackslash{}basic\_string.h 
мы можем прочитать что это сделано для удобства отладки:

\begin{lstlisting}
   *  The reason you want _M_data pointing to the character %array and
   *  not the _Rep is so that the debugger can see the string
   *  contents. (Probably we should add a non-inline member to get
   *  the _Rep for the debugger to use, so users can check the actual
   *  string length.)
\end{lstlisting}

\href{http://go.yurichev.com/17085}{исходный код basic\_string.h}

В нашем примере мы учитываем это:

\lstinputlisting[caption=пример для GCC,style=customc]{\CURPATH/STL/string/GCC_RU.cpp}

Нужны еще небольшие хаки чтобы сымитировать типичную ошибку, которую мы уже видели выше, из-за
более ужесточенной проверки типов в GCC, тем не менее, printf() работает и здесь без c\_str().

\myparagraph{Чуть более сложный пример}

\lstinputlisting[style=customc]{\CURPATH/STL/string/3.cpp}

\lstinputlisting[caption=MSVC 2012,style=customasmx86]{\CURPATH/STL/string/3_MSVC_RU.asm}

Собственно, компилятор не конструирует строки статически: да в общем-то и как
это возможно, если буфер с ней нужно хранить в \glslink{heap}{куче}?

Вместо этого в сегменте данных хранятся обычные \ac{ASCIIZ}-строки, а позже, во время выполнения, 
при помощи метода \q{assign}, конструируются строки s1 и s2
.
При помощи \TT{operator+}, создается строка s3.

Обратите внимание на то что вызов метода c\_str() отсутствует,
потому что его код достаточно короткий и компилятор вставил его прямо здесь:
если строка короче 16-и байт, то в регистре EAX остается указатель на буфер,
а если длиннее, то из этого же места достается адрес на буфер расположенный в \glslink{heap}{куче}.

Далее следуют вызовы трех деструкторов, причем, они вызываются только если строка длиннее 16-и байт:
тогда нужно освободить буфера в \glslink{heap}{куче}.
В противном случае, так как все три объекта std::string хранятся в стеке,
они освобождаются автоматически после выхода из функции.

Следовательно, работа с короткими строками более быстрая из-за м\'{е}ньшего обращения к \glslink{heap}{куче}.

Код на GCC даже проще (из-за того, что в GCC, как мы уже видели, не реализована возможность хранить короткую
строку прямо в структуре):

% TODO1 comment each function meaning
\lstinputlisting[caption=GCC 4.8.1,style=customasmx86]{\CURPATH/STL/string/3_GCC_RU.s}

Можно заметить, что в деструкторы передается не указатель на объект,
а указатель на место за 12 байт (или 3 слова) перед ним, то есть, на настоящее начало структуры.

\myparagraph{std::string как глобальная переменная}
\label{sec:std_string_as_global_variable}

Опытные программисты на \Cpp знают, что глобальные переменные \ac{STL}-типов вполне можно объявлять.

Да, действительно:

\lstinputlisting[style=customc]{\CURPATH/STL/string/5.cpp}

Но как и где будет вызываться конструктор \TT{std::string}?

На самом деле, эта переменная будет инициализирована даже перед началом \main.

\lstinputlisting[caption=MSVC 2012: здесь конструируется глобальная переменная{,} а также регистрируется её деструктор,style=customasmx86]{\CURPATH/STL/string/5_MSVC_p2.asm}

\lstinputlisting[caption=MSVC 2012: здесь глобальная переменная используется в \main,style=customasmx86]{\CURPATH/STL/string/5_MSVC_p1.asm}

\lstinputlisting[caption=MSVC 2012: эта функция-деструктор вызывается перед выходом,style=customasmx86]{\CURPATH/STL/string/5_MSVC_p3.asm}

\myindex{\CStandardLibrary!atexit()}
В реальности, из \ac{CRT}, еще до вызова main(), вызывается специальная функция,
в которой перечислены все конструкторы подобных переменных.
Более того: при помощи atexit() регистрируется функция, которая будет вызвана в конце работы программы:
в этой функции компилятор собирает вызовы деструкторов всех подобных глобальных переменных.

GCC работает похожим образом:

\lstinputlisting[caption=GCC 4.8.1,style=customasmx86]{\CURPATH/STL/string/5_GCC.s}

Но он не выделяет отдельной функции в которой будут собраны деструкторы: 
каждый деструктор передается в atexit() по одному.

% TODO а если глобальная STL-переменная в другом модуле? надо проверить.

}
\ifdefined\SPANISH
\chapter{Patrones de código}
\fi % SPANISH

\ifdefined\GERMAN
\chapter{Code-Muster}
\fi % GERMAN

\ifdefined\ENGLISH
\chapter{Code Patterns}
\fi % ENGLISH

\ifdefined\ITALIAN
\chapter{Forme di codice}
\fi % ITALIAN

\ifdefined\RUSSIAN
\chapter{Образцы кода}
\fi % RUSSIAN

\ifdefined\BRAZILIAN
\chapter{Padrões de códigos}
\fi % BRAZILIAN

\ifdefined\THAI
\chapter{รูปแบบของโค้ด}
\fi % THAI

\ifdefined\FRENCH
\chapter{Modèle de code}
\fi % FRENCH

\ifdefined\POLISH
\chapter{\PLph{}}
\fi % POLISH

% sections
\EN{\input{patterns/patterns_opt_dbg_EN}}
\ES{\input{patterns/patterns_opt_dbg_ES}}
\ITA{\input{patterns/patterns_opt_dbg_ITA}}
\PTBR{\input{patterns/patterns_opt_dbg_PTBR}}
\RU{\input{patterns/patterns_opt_dbg_RU}}
\THA{\input{patterns/patterns_opt_dbg_THA}}
\DE{\input{patterns/patterns_opt_dbg_DE}}
\FR{\input{patterns/patterns_opt_dbg_FR}}
\PL{\input{patterns/patterns_opt_dbg_PL}}

\RU{\section{Некоторые базовые понятия}}
\EN{\section{Some basics}}
\DE{\section{Einige Grundlagen}}
\FR{\section{Quelques bases}}
\ES{\section{\ESph{}}}
\ITA{\section{Alcune basi teoriche}}
\PTBR{\section{\PTBRph{}}}
\THA{\section{\THAph{}}}
\PL{\section{\PLph{}}}

% sections:
\EN{\input{patterns/intro_CPU_ISA_EN}}
\ES{\input{patterns/intro_CPU_ISA_ES}}
\ITA{\input{patterns/intro_CPU_ISA_ITA}}
\PTBR{\input{patterns/intro_CPU_ISA_PTBR}}
\RU{\input{patterns/intro_CPU_ISA_RU}}
\DE{\input{patterns/intro_CPU_ISA_DE}}
\FR{\input{patterns/intro_CPU_ISA_FR}}
\PL{\input{patterns/intro_CPU_ISA_PL}}

\EN{\input{patterns/numeral_EN}}
\RU{\input{patterns/numeral_RU}}
\ITA{\input{patterns/numeral_ITA}}
\DE{\input{patterns/numeral_DE}}
\FR{\input{patterns/numeral_FR}}
\PL{\input{patterns/numeral_PL}}

% chapters
\input{patterns/00_empty/main}
\input{patterns/011_ret/main}
\input{patterns/01_helloworld/main}
\input{patterns/015_prolog_epilogue/main}
\input{patterns/02_stack/main}
\input{patterns/03_printf/main}
\input{patterns/04_scanf/main}
\input{patterns/05_passing_arguments/main}
\input{patterns/06_return_results/main}
\input{patterns/061_pointers/main}
\input{patterns/065_GOTO/main}
\input{patterns/07_jcc/main}
\input{patterns/08_switch/main}
\input{patterns/09_loops/main}
\input{patterns/10_strings/main}
\input{patterns/11_arith_optimizations/main}
\input{patterns/12_FPU/main}
\input{patterns/13_arrays/main}
\input{patterns/14_bitfields/main}
\EN{\input{patterns/145_LCG/main_EN}}
\RU{\input{patterns/145_LCG/main_RU}}
\input{patterns/15_structs/main}
\input{patterns/17_unions/main}
\input{patterns/18_pointers_to_functions/main}
\input{patterns/185_64bit_in_32_env/main}

\EN{\input{patterns/19_SIMD/main_EN}}
\RU{\input{patterns/19_SIMD/main_RU}}
\DE{\input{patterns/19_SIMD/main_DE}}

\EN{\input{patterns/20_x64/main_EN}}
\RU{\input{patterns/20_x64/main_RU}}

\EN{\input{patterns/205_floating_SIMD/main_EN}}
\RU{\input{patterns/205_floating_SIMD/main_RU}}
\DE{\input{patterns/205_floating_SIMD/main_DE}}

\EN{\input{patterns/ARM/main_EN}}
\RU{\input{patterns/ARM/main_RU}}
\DE{\input{patterns/ARM/main_DE}}

\input{patterns/MIPS/main}

\ifdefined\SPANISH
\chapter{Patrones de código}
\fi % SPANISH

\ifdefined\GERMAN
\chapter{Code-Muster}
\fi % GERMAN

\ifdefined\ENGLISH
\chapter{Code Patterns}
\fi % ENGLISH

\ifdefined\ITALIAN
\chapter{Forme di codice}
\fi % ITALIAN

\ifdefined\RUSSIAN
\chapter{Образцы кода}
\fi % RUSSIAN

\ifdefined\BRAZILIAN
\chapter{Padrões de códigos}
\fi % BRAZILIAN

\ifdefined\THAI
\chapter{รูปแบบของโค้ด}
\fi % THAI

\ifdefined\FRENCH
\chapter{Modèle de code}
\fi % FRENCH

\ifdefined\POLISH
\chapter{\PLph{}}
\fi % POLISH

% sections
\EN{\input{patterns/patterns_opt_dbg_EN}}
\ES{\input{patterns/patterns_opt_dbg_ES}}
\ITA{\input{patterns/patterns_opt_dbg_ITA}}
\PTBR{\input{patterns/patterns_opt_dbg_PTBR}}
\RU{\input{patterns/patterns_opt_dbg_RU}}
\THA{\input{patterns/patterns_opt_dbg_THA}}
\DE{\input{patterns/patterns_opt_dbg_DE}}
\FR{\input{patterns/patterns_opt_dbg_FR}}
\PL{\input{patterns/patterns_opt_dbg_PL}}

\RU{\section{Некоторые базовые понятия}}
\EN{\section{Some basics}}
\DE{\section{Einige Grundlagen}}
\FR{\section{Quelques bases}}
\ES{\section{\ESph{}}}
\ITA{\section{Alcune basi teoriche}}
\PTBR{\section{\PTBRph{}}}
\THA{\section{\THAph{}}}
\PL{\section{\PLph{}}}

% sections:
\EN{\input{patterns/intro_CPU_ISA_EN}}
\ES{\input{patterns/intro_CPU_ISA_ES}}
\ITA{\input{patterns/intro_CPU_ISA_ITA}}
\PTBR{\input{patterns/intro_CPU_ISA_PTBR}}
\RU{\input{patterns/intro_CPU_ISA_RU}}
\DE{\input{patterns/intro_CPU_ISA_DE}}
\FR{\input{patterns/intro_CPU_ISA_FR}}
\PL{\input{patterns/intro_CPU_ISA_PL}}

\EN{\input{patterns/numeral_EN}}
\RU{\input{patterns/numeral_RU}}
\ITA{\input{patterns/numeral_ITA}}
\DE{\input{patterns/numeral_DE}}
\FR{\input{patterns/numeral_FR}}
\PL{\input{patterns/numeral_PL}}

% chapters
\input{patterns/00_empty/main}
\input{patterns/011_ret/main}
\input{patterns/01_helloworld/main}
\input{patterns/015_prolog_epilogue/main}
\input{patterns/02_stack/main}
\input{patterns/03_printf/main}
\input{patterns/04_scanf/main}
\input{patterns/05_passing_arguments/main}
\input{patterns/06_return_results/main}
\input{patterns/061_pointers/main}
\input{patterns/065_GOTO/main}
\input{patterns/07_jcc/main}
\input{patterns/08_switch/main}
\input{patterns/09_loops/main}
\input{patterns/10_strings/main}
\input{patterns/11_arith_optimizations/main}
\input{patterns/12_FPU/main}
\input{patterns/13_arrays/main}
\input{patterns/14_bitfields/main}
\EN{\input{patterns/145_LCG/main_EN}}
\RU{\input{patterns/145_LCG/main_RU}}
\input{patterns/15_structs/main}
\input{patterns/17_unions/main}
\input{patterns/18_pointers_to_functions/main}
\input{patterns/185_64bit_in_32_env/main}

\EN{\input{patterns/19_SIMD/main_EN}}
\RU{\input{patterns/19_SIMD/main_RU}}
\DE{\input{patterns/19_SIMD/main_DE}}

\EN{\input{patterns/20_x64/main_EN}}
\RU{\input{patterns/20_x64/main_RU}}

\EN{\input{patterns/205_floating_SIMD/main_EN}}
\RU{\input{patterns/205_floating_SIMD/main_RU}}
\DE{\input{patterns/205_floating_SIMD/main_DE}}

\EN{\input{patterns/ARM/main_EN}}
\RU{\input{patterns/ARM/main_RU}}
\DE{\input{patterns/ARM/main_DE}}

\input{patterns/MIPS/main}

\ifdefined\SPANISH
\chapter{Patrones de código}
\fi % SPANISH

\ifdefined\GERMAN
\chapter{Code-Muster}
\fi % GERMAN

\ifdefined\ENGLISH
\chapter{Code Patterns}
\fi % ENGLISH

\ifdefined\ITALIAN
\chapter{Forme di codice}
\fi % ITALIAN

\ifdefined\RUSSIAN
\chapter{Образцы кода}
\fi % RUSSIAN

\ifdefined\BRAZILIAN
\chapter{Padrões de códigos}
\fi % BRAZILIAN

\ifdefined\THAI
\chapter{รูปแบบของโค้ด}
\fi % THAI

\ifdefined\FRENCH
\chapter{Modèle de code}
\fi % FRENCH

\ifdefined\POLISH
\chapter{\PLph{}}
\fi % POLISH

% sections
\EN{\input{patterns/patterns_opt_dbg_EN}}
\ES{\input{patterns/patterns_opt_dbg_ES}}
\ITA{\input{patterns/patterns_opt_dbg_ITA}}
\PTBR{\input{patterns/patterns_opt_dbg_PTBR}}
\RU{\input{patterns/patterns_opt_dbg_RU}}
\THA{\input{patterns/patterns_opt_dbg_THA}}
\DE{\input{patterns/patterns_opt_dbg_DE}}
\FR{\input{patterns/patterns_opt_dbg_FR}}
\PL{\input{patterns/patterns_opt_dbg_PL}}

\RU{\section{Некоторые базовые понятия}}
\EN{\section{Some basics}}
\DE{\section{Einige Grundlagen}}
\FR{\section{Quelques bases}}
\ES{\section{\ESph{}}}
\ITA{\section{Alcune basi teoriche}}
\PTBR{\section{\PTBRph{}}}
\THA{\section{\THAph{}}}
\PL{\section{\PLph{}}}

% sections:
\EN{\input{patterns/intro_CPU_ISA_EN}}
\ES{\input{patterns/intro_CPU_ISA_ES}}
\ITA{\input{patterns/intro_CPU_ISA_ITA}}
\PTBR{\input{patterns/intro_CPU_ISA_PTBR}}
\RU{\input{patterns/intro_CPU_ISA_RU}}
\DE{\input{patterns/intro_CPU_ISA_DE}}
\FR{\input{patterns/intro_CPU_ISA_FR}}
\PL{\input{patterns/intro_CPU_ISA_PL}}

\EN{\input{patterns/numeral_EN}}
\RU{\input{patterns/numeral_RU}}
\ITA{\input{patterns/numeral_ITA}}
\DE{\input{patterns/numeral_DE}}
\FR{\input{patterns/numeral_FR}}
\PL{\input{patterns/numeral_PL}}

% chapters
\input{patterns/00_empty/main}
\input{patterns/011_ret/main}
\input{patterns/01_helloworld/main}
\input{patterns/015_prolog_epilogue/main}
\input{patterns/02_stack/main}
\input{patterns/03_printf/main}
\input{patterns/04_scanf/main}
\input{patterns/05_passing_arguments/main}
\input{patterns/06_return_results/main}
\input{patterns/061_pointers/main}
\input{patterns/065_GOTO/main}
\input{patterns/07_jcc/main}
\input{patterns/08_switch/main}
\input{patterns/09_loops/main}
\input{patterns/10_strings/main}
\input{patterns/11_arith_optimizations/main}
\input{patterns/12_FPU/main}
\input{patterns/13_arrays/main}
\input{patterns/14_bitfields/main}
\EN{\input{patterns/145_LCG/main_EN}}
\RU{\input{patterns/145_LCG/main_RU}}
\input{patterns/15_structs/main}
\input{patterns/17_unions/main}
\input{patterns/18_pointers_to_functions/main}
\input{patterns/185_64bit_in_32_env/main}

\EN{\input{patterns/19_SIMD/main_EN}}
\RU{\input{patterns/19_SIMD/main_RU}}
\DE{\input{patterns/19_SIMD/main_DE}}

\EN{\input{patterns/20_x64/main_EN}}
\RU{\input{patterns/20_x64/main_RU}}

\EN{\input{patterns/205_floating_SIMD/main_EN}}
\RU{\input{patterns/205_floating_SIMD/main_RU}}
\DE{\input{patterns/205_floating_SIMD/main_DE}}

\EN{\input{patterns/ARM/main_EN}}
\RU{\input{patterns/ARM/main_RU}}
\DE{\input{patterns/ARM/main_DE}}

\input{patterns/MIPS/main}

\ifdefined\SPANISH
\chapter{Patrones de código}
\fi % SPANISH

\ifdefined\GERMAN
\chapter{Code-Muster}
\fi % GERMAN

\ifdefined\ENGLISH
\chapter{Code Patterns}
\fi % ENGLISH

\ifdefined\ITALIAN
\chapter{Forme di codice}
\fi % ITALIAN

\ifdefined\RUSSIAN
\chapter{Образцы кода}
\fi % RUSSIAN

\ifdefined\BRAZILIAN
\chapter{Padrões de códigos}
\fi % BRAZILIAN

\ifdefined\THAI
\chapter{รูปแบบของโค้ด}
\fi % THAI

\ifdefined\FRENCH
\chapter{Modèle de code}
\fi % FRENCH

\ifdefined\POLISH
\chapter{\PLph{}}
\fi % POLISH

% sections
\EN{\input{patterns/patterns_opt_dbg_EN}}
\ES{\input{patterns/patterns_opt_dbg_ES}}
\ITA{\input{patterns/patterns_opt_dbg_ITA}}
\PTBR{\input{patterns/patterns_opt_dbg_PTBR}}
\RU{\input{patterns/patterns_opt_dbg_RU}}
\THA{\input{patterns/patterns_opt_dbg_THA}}
\DE{\input{patterns/patterns_opt_dbg_DE}}
\FR{\input{patterns/patterns_opt_dbg_FR}}
\PL{\input{patterns/patterns_opt_dbg_PL}}

\RU{\section{Некоторые базовые понятия}}
\EN{\section{Some basics}}
\DE{\section{Einige Grundlagen}}
\FR{\section{Quelques bases}}
\ES{\section{\ESph{}}}
\ITA{\section{Alcune basi teoriche}}
\PTBR{\section{\PTBRph{}}}
\THA{\section{\THAph{}}}
\PL{\section{\PLph{}}}

% sections:
\EN{\input{patterns/intro_CPU_ISA_EN}}
\ES{\input{patterns/intro_CPU_ISA_ES}}
\ITA{\input{patterns/intro_CPU_ISA_ITA}}
\PTBR{\input{patterns/intro_CPU_ISA_PTBR}}
\RU{\input{patterns/intro_CPU_ISA_RU}}
\DE{\input{patterns/intro_CPU_ISA_DE}}
\FR{\input{patterns/intro_CPU_ISA_FR}}
\PL{\input{patterns/intro_CPU_ISA_PL}}

\EN{\input{patterns/numeral_EN}}
\RU{\input{patterns/numeral_RU}}
\ITA{\input{patterns/numeral_ITA}}
\DE{\input{patterns/numeral_DE}}
\FR{\input{patterns/numeral_FR}}
\PL{\input{patterns/numeral_PL}}

% chapters
\input{patterns/00_empty/main}
\input{patterns/011_ret/main}
\input{patterns/01_helloworld/main}
\input{patterns/015_prolog_epilogue/main}
\input{patterns/02_stack/main}
\input{patterns/03_printf/main}
\input{patterns/04_scanf/main}
\input{patterns/05_passing_arguments/main}
\input{patterns/06_return_results/main}
\input{patterns/061_pointers/main}
\input{patterns/065_GOTO/main}
\input{patterns/07_jcc/main}
\input{patterns/08_switch/main}
\input{patterns/09_loops/main}
\input{patterns/10_strings/main}
\input{patterns/11_arith_optimizations/main}
\input{patterns/12_FPU/main}
\input{patterns/13_arrays/main}
\input{patterns/14_bitfields/main}
\EN{\input{patterns/145_LCG/main_EN}}
\RU{\input{patterns/145_LCG/main_RU}}
\input{patterns/15_structs/main}
\input{patterns/17_unions/main}
\input{patterns/18_pointers_to_functions/main}
\input{patterns/185_64bit_in_32_env/main}

\EN{\input{patterns/19_SIMD/main_EN}}
\RU{\input{patterns/19_SIMD/main_RU}}
\DE{\input{patterns/19_SIMD/main_DE}}

\EN{\input{patterns/20_x64/main_EN}}
\RU{\input{patterns/20_x64/main_RU}}

\EN{\input{patterns/205_floating_SIMD/main_EN}}
\RU{\input{patterns/205_floating_SIMD/main_RU}}
\DE{\input{patterns/205_floating_SIMD/main_DE}}

\EN{\input{patterns/ARM/main_EN}}
\RU{\input{patterns/ARM/main_RU}}
\DE{\input{patterns/ARM/main_DE}}

\input{patterns/MIPS/main}


\EN{\section{Returning Values}
\label{ret_val_func}

Another simple function is the one that simply returns a constant value:

\lstinputlisting[caption=\EN{\CCpp Code},style=customc]{patterns/011_ret/1.c}

Let's compile it.

\subsection{x86}

Here's what both the GCC and MSVC compilers produce (with optimization) on the x86 platform:

\lstinputlisting[caption=\Optimizing GCC/MSVC (\assemblyOutput),style=customasmx86]{patterns/011_ret/1.s}

\myindex{x86!\Instructions!RET}
There are just two instructions: the first places the value 123 into the \EAX register,
which is used by convention for storing the return
value, and the second one is \RET, which returns execution to the \gls{caller}.

The caller will take the result from the \EAX register.

\subsection{ARM}

There are a few differences on the ARM platform:

\lstinputlisting[caption=\OptimizingKeilVI (\ARMMode) ASM Output,style=customasmARM]{patterns/011_ret/1_Keil_ARM_O3.s}

ARM uses the register \Reg{0} for returning the results of functions, so 123 is copied into \Reg{0}.

\myindex{ARM!\Instructions!MOV}
\myindex{x86!\Instructions!MOV}
It is worth noting that \MOV is a misleading name for the instruction in both the x86 and ARM \ac{ISA}s.

The data is not in fact \IT{moved}, but \IT{copied}.

\subsection{MIPS}

\label{MIPS_leaf_function_ex1}

The GCC assembly output below lists registers by number:

\lstinputlisting[caption=\Optimizing GCC 4.4.5 (\assemblyOutput),style=customasmMIPS]{patterns/011_ret/MIPS.s}

\dots while \IDA does it by their pseudo names:

\lstinputlisting[caption=\Optimizing GCC 4.4.5 (IDA),style=customasmMIPS]{patterns/011_ret/MIPS_IDA.lst}

The \$2 (or \$V0) register is used to store the function's return value.
\myindex{MIPS!\Pseudoinstructions!LI}
\INS{LI} stands for ``Load Immediate'' and is the MIPS equivalent to \MOV.

\myindex{MIPS!\Instructions!J}
The other instruction is the jump instruction (J or JR) which returns the execution flow to the \gls{caller}.

\myindex{MIPS!Branch delay slot}
You might be wondering why the positions of the load instruction (LI) and the jump instruction (J or JR) are swapped. This is due to a \ac{RISC} feature called ``branch delay slot''.

The reason this happens is a quirk in the architecture of some RISC \ac{ISA}s and isn't important for our
purposes---we must simply keep in mind that in MIPS, the instruction following a jump or branch instruction
is executed \IT{before} the jump/branch instruction itself.

As a consequence, branch instructions always swap places with the instruction executed immediately beforehand.


In practice, functions which merely return 1 (\IT{true}) or 0 (\IT{false}) are very frequent.

The smallest ever of the standard UNIX utilities, \IT{/bin/true} and \IT{/bin/false} return 0 and 1 respectively, as an exit code.
(Zero as an exit code usually means success, non-zero means error.)
}
\RU{\subsubsection{std::string}
\myindex{\Cpp!STL!std::string}
\label{std_string}

\myparagraph{Как устроена структура}

Многие строковые библиотеки \InSqBrackets{\CNotes 2.2} обеспечивают структуру содержащую ссылку 
на буфер собственно со строкой, переменная всегда содержащую длину строки 
(что очень удобно для массы функций \InSqBrackets{\CNotes 2.2.1}) и переменную содержащую текущий размер буфера.

Строка в буфере обыкновенно оканчивается нулем: это для того чтобы указатель на буфер можно было
передавать в функции требующие на вход обычную сишную \ac{ASCIIZ}-строку.

Стандарт \Cpp не описывает, как именно нужно реализовывать std::string,
но, как правило, они реализованы как описано выше, с небольшими дополнениями.

Строки в \Cpp это не класс (как, например, QString в Qt), а темплейт (basic\_string), 
это сделано для того чтобы поддерживать 
строки содержащие разного типа символы: как минимум \Tchar и \IT{wchar\_t}.

Так что, std::string это класс с базовым типом \Tchar.

А std::wstring это класс с базовым типом \IT{wchar\_t}.

\mysubparagraph{MSVC}

В реализации MSVC, вместо ссылки на буфер может содержаться сам буфер (если строка короче 16-и символов).

Это означает, что каждая короткая строка будет занимать в памяти по крайней мере $16 + 4 + 4 = 24$ 
байт для 32-битной среды либо $16 + 8 + 8 = 32$ 
байта в 64-битной, а если строка длиннее 16-и символов, то прибавьте еще длину самой строки.

\lstinputlisting[caption=пример для MSVC,style=customc]{\CURPATH/STL/string/MSVC_RU.cpp}

Собственно, из этого исходника почти всё ясно.

Несколько замечаний:

Если строка короче 16-и символов, 
то отдельный буфер для строки в \glslink{heap}{куче} выделяться не будет.

Это удобно потому что на практике, основная часть строк действительно короткие.
Вероятно, разработчики в Microsoft выбрали размер в 16 символов как разумный баланс.

Теперь очень важный момент в конце функции main(): мы не пользуемся методом c\_str(), тем не менее,
если это скомпилировать и запустить, то обе строки появятся в консоли!

Работает это вот почему.

В первом случае строка короче 16-и символов и в начале объекта std::string (его можно рассматривать
просто как структуру) расположен буфер с этой строкой.
\printf трактует указатель как указатель на массив символов оканчивающийся нулем и поэтому всё работает.

Вывод второй строки (длиннее 16-и символов) даже еще опаснее: это вообще типичная программистская ошибка 
(или опечатка), забыть дописать c\_str().
Это работает потому что в это время в начале структуры расположен указатель на буфер.
Это может надолго остаться незамеченным: до тех пока там не появится строка 
короче 16-и символов, тогда процесс упадет.

\mysubparagraph{GCC}

В реализации GCC в структуре есть еще одна переменная --- reference count.

Интересно, что указатель на экземпляр класса std::string в GCC указывает не на начало самой структуры, 
а на указатель на буфера.
В libstdc++-v3\textbackslash{}include\textbackslash{}bits\textbackslash{}basic\_string.h 
мы можем прочитать что это сделано для удобства отладки:

\begin{lstlisting}
   *  The reason you want _M_data pointing to the character %array and
   *  not the _Rep is so that the debugger can see the string
   *  contents. (Probably we should add a non-inline member to get
   *  the _Rep for the debugger to use, so users can check the actual
   *  string length.)
\end{lstlisting}

\href{http://go.yurichev.com/17085}{исходный код basic\_string.h}

В нашем примере мы учитываем это:

\lstinputlisting[caption=пример для GCC,style=customc]{\CURPATH/STL/string/GCC_RU.cpp}

Нужны еще небольшие хаки чтобы сымитировать типичную ошибку, которую мы уже видели выше, из-за
более ужесточенной проверки типов в GCC, тем не менее, printf() работает и здесь без c\_str().

\myparagraph{Чуть более сложный пример}

\lstinputlisting[style=customc]{\CURPATH/STL/string/3.cpp}

\lstinputlisting[caption=MSVC 2012,style=customasmx86]{\CURPATH/STL/string/3_MSVC_RU.asm}

Собственно, компилятор не конструирует строки статически: да в общем-то и как
это возможно, если буфер с ней нужно хранить в \glslink{heap}{куче}?

Вместо этого в сегменте данных хранятся обычные \ac{ASCIIZ}-строки, а позже, во время выполнения, 
при помощи метода \q{assign}, конструируются строки s1 и s2
.
При помощи \TT{operator+}, создается строка s3.

Обратите внимание на то что вызов метода c\_str() отсутствует,
потому что его код достаточно короткий и компилятор вставил его прямо здесь:
если строка короче 16-и байт, то в регистре EAX остается указатель на буфер,
а если длиннее, то из этого же места достается адрес на буфер расположенный в \glslink{heap}{куче}.

Далее следуют вызовы трех деструкторов, причем, они вызываются только если строка длиннее 16-и байт:
тогда нужно освободить буфера в \glslink{heap}{куче}.
В противном случае, так как все три объекта std::string хранятся в стеке,
они освобождаются автоматически после выхода из функции.

Следовательно, работа с короткими строками более быстрая из-за м\'{е}ньшего обращения к \glslink{heap}{куче}.

Код на GCC даже проще (из-за того, что в GCC, как мы уже видели, не реализована возможность хранить короткую
строку прямо в структуре):

% TODO1 comment each function meaning
\lstinputlisting[caption=GCC 4.8.1,style=customasmx86]{\CURPATH/STL/string/3_GCC_RU.s}

Можно заметить, что в деструкторы передается не указатель на объект,
а указатель на место за 12 байт (или 3 слова) перед ним, то есть, на настоящее начало структуры.

\myparagraph{std::string как глобальная переменная}
\label{sec:std_string_as_global_variable}

Опытные программисты на \Cpp знают, что глобальные переменные \ac{STL}-типов вполне можно объявлять.

Да, действительно:

\lstinputlisting[style=customc]{\CURPATH/STL/string/5.cpp}

Но как и где будет вызываться конструктор \TT{std::string}?

На самом деле, эта переменная будет инициализирована даже перед началом \main.

\lstinputlisting[caption=MSVC 2012: здесь конструируется глобальная переменная{,} а также регистрируется её деструктор,style=customasmx86]{\CURPATH/STL/string/5_MSVC_p2.asm}

\lstinputlisting[caption=MSVC 2012: здесь глобальная переменная используется в \main,style=customasmx86]{\CURPATH/STL/string/5_MSVC_p1.asm}

\lstinputlisting[caption=MSVC 2012: эта функция-деструктор вызывается перед выходом,style=customasmx86]{\CURPATH/STL/string/5_MSVC_p3.asm}

\myindex{\CStandardLibrary!atexit()}
В реальности, из \ac{CRT}, еще до вызова main(), вызывается специальная функция,
в которой перечислены все конструкторы подобных переменных.
Более того: при помощи atexit() регистрируется функция, которая будет вызвана в конце работы программы:
в этой функции компилятор собирает вызовы деструкторов всех подобных глобальных переменных.

GCC работает похожим образом:

\lstinputlisting[caption=GCC 4.8.1,style=customasmx86]{\CURPATH/STL/string/5_GCC.s}

Но он не выделяет отдельной функции в которой будут собраны деструкторы: 
каждый деструктор передается в atexit() по одному.

% TODO а если глобальная STL-переменная в другом модуле? надо проверить.

}
\DE{\subsection{Einfachste XOR-Verschlüsselung überhaupt}

Ich habe einmal eine Software gesehen, bei der alle Debugging-Ausgaben mit XOR mit dem Wert 3
verschlüsselt wurden. Mit anderen Worten, die beiden niedrigsten Bits aller Buchstaben wurden invertiert.

``Hello, world'' wurde zu ``Kfool/\#tlqog'':

\begin{lstlisting}
#!/usr/bin/python

msg="Hello, world!"

print "".join(map(lambda x: chr(ord(x)^3), msg))
\end{lstlisting}

Das ist eine ziemlich interessante Verschlüsselung (oder besser eine Verschleierung),
weil sie zwei wichtige Eigenschaften hat:
1) es ist eine einzige Funktion zum Verschlüsseln und entschlüsseln, sie muss nur wiederholt angewendet werden
2) die entstehenden Buchstaben befinden sich im druckbaren Bereich, also die ganze Zeichenkette kann ohne
Escape-Symbole im Code verwendet werden.

Die zweite Eigenschaft nutzt die Tatsache, dass alle druckbaren Zeichen in Reihen organisiert sind: 0x2x-0x7x,
und wenn die beiden niederwertigsten Bits invertiert werden, wird der Buchstabe um eine oder drei Stellen nach
links oder rechts \IT{verschoben}, aber niemals in eine andere Reihe:

\begin{figure}[H]
\centering
\includegraphics[width=0.7\textwidth]{ascii_clean.png}
\caption{7-Bit \ac{ASCII} Tabelle in Emacs}
\end{figure}

\dots mit dem Zeichen 0x7F als einziger Ausnahme.

Im Folgenden werden also beispielsweise die Zeichen A-Z \IT{verschlüsselt}:

\begin{lstlisting}
#!/usr/bin/python

msg="@ABCDEFGHIJKLMNO"

print "".join(map(lambda x: chr(ord(x)^3), msg))
\end{lstlisting}

Ergebnis:
% FIXME \verb  --  relevant comment for German?
\begin{lstlisting}
CBA@GFEDKJIHONML
\end{lstlisting}

Es sieht so aus als würden die Zeichen ``@'' und ``C'' sowie ``B'' und ``A'' vertauscht werden.

Hier ist noch ein interessantes Beispiel, in dem gezeigt wird, wie die Eigenschaften von XOR
ausgenutzt werden können: Exakt den gleichen Effekt, dass druckbare Zeichen auch druckbar bleiben,
kann man dadurch erzielen, dass irgendeine Kombination der niedrigsten vier Bits invertiert wird.
}

\EN{\section{Returning Values}
\label{ret_val_func}

Another simple function is the one that simply returns a constant value:

\lstinputlisting[caption=\EN{\CCpp Code},style=customc]{patterns/011_ret/1.c}

Let's compile it.

\subsection{x86}

Here's what both the GCC and MSVC compilers produce (with optimization) on the x86 platform:

\lstinputlisting[caption=\Optimizing GCC/MSVC (\assemblyOutput),style=customasmx86]{patterns/011_ret/1.s}

\myindex{x86!\Instructions!RET}
There are just two instructions: the first places the value 123 into the \EAX register,
which is used by convention for storing the return
value, and the second one is \RET, which returns execution to the \gls{caller}.

The caller will take the result from the \EAX register.

\subsection{ARM}

There are a few differences on the ARM platform:

\lstinputlisting[caption=\OptimizingKeilVI (\ARMMode) ASM Output,style=customasmARM]{patterns/011_ret/1_Keil_ARM_O3.s}

ARM uses the register \Reg{0} for returning the results of functions, so 123 is copied into \Reg{0}.

\myindex{ARM!\Instructions!MOV}
\myindex{x86!\Instructions!MOV}
It is worth noting that \MOV is a misleading name for the instruction in both the x86 and ARM \ac{ISA}s.

The data is not in fact \IT{moved}, but \IT{copied}.

\subsection{MIPS}

\label{MIPS_leaf_function_ex1}

The GCC assembly output below lists registers by number:

\lstinputlisting[caption=\Optimizing GCC 4.4.5 (\assemblyOutput),style=customasmMIPS]{patterns/011_ret/MIPS.s}

\dots while \IDA does it by their pseudo names:

\lstinputlisting[caption=\Optimizing GCC 4.4.5 (IDA),style=customasmMIPS]{patterns/011_ret/MIPS_IDA.lst}

The \$2 (or \$V0) register is used to store the function's return value.
\myindex{MIPS!\Pseudoinstructions!LI}
\INS{LI} stands for ``Load Immediate'' and is the MIPS equivalent to \MOV.

\myindex{MIPS!\Instructions!J}
The other instruction is the jump instruction (J or JR) which returns the execution flow to the \gls{caller}.

\myindex{MIPS!Branch delay slot}
You might be wondering why the positions of the load instruction (LI) and the jump instruction (J or JR) are swapped. This is due to a \ac{RISC} feature called ``branch delay slot''.

The reason this happens is a quirk in the architecture of some RISC \ac{ISA}s and isn't important for our
purposes---we must simply keep in mind that in MIPS, the instruction following a jump or branch instruction
is executed \IT{before} the jump/branch instruction itself.

As a consequence, branch instructions always swap places with the instruction executed immediately beforehand.


In practice, functions which merely return 1 (\IT{true}) or 0 (\IT{false}) are very frequent.

The smallest ever of the standard UNIX utilities, \IT{/bin/true} and \IT{/bin/false} return 0 and 1 respectively, as an exit code.
(Zero as an exit code usually means success, non-zero means error.)
}
\RU{\subsubsection{std::string}
\myindex{\Cpp!STL!std::string}
\label{std_string}

\myparagraph{Как устроена структура}

Многие строковые библиотеки \InSqBrackets{\CNotes 2.2} обеспечивают структуру содержащую ссылку 
на буфер собственно со строкой, переменная всегда содержащую длину строки 
(что очень удобно для массы функций \InSqBrackets{\CNotes 2.2.1}) и переменную содержащую текущий размер буфера.

Строка в буфере обыкновенно оканчивается нулем: это для того чтобы указатель на буфер можно было
передавать в функции требующие на вход обычную сишную \ac{ASCIIZ}-строку.

Стандарт \Cpp не описывает, как именно нужно реализовывать std::string,
но, как правило, они реализованы как описано выше, с небольшими дополнениями.

Строки в \Cpp это не класс (как, например, QString в Qt), а темплейт (basic\_string), 
это сделано для того чтобы поддерживать 
строки содержащие разного типа символы: как минимум \Tchar и \IT{wchar\_t}.

Так что, std::string это класс с базовым типом \Tchar.

А std::wstring это класс с базовым типом \IT{wchar\_t}.

\mysubparagraph{MSVC}

В реализации MSVC, вместо ссылки на буфер может содержаться сам буфер (если строка короче 16-и символов).

Это означает, что каждая короткая строка будет занимать в памяти по крайней мере $16 + 4 + 4 = 24$ 
байт для 32-битной среды либо $16 + 8 + 8 = 32$ 
байта в 64-битной, а если строка длиннее 16-и символов, то прибавьте еще длину самой строки.

\lstinputlisting[caption=пример для MSVC,style=customc]{\CURPATH/STL/string/MSVC_RU.cpp}

Собственно, из этого исходника почти всё ясно.

Несколько замечаний:

Если строка короче 16-и символов, 
то отдельный буфер для строки в \glslink{heap}{куче} выделяться не будет.

Это удобно потому что на практике, основная часть строк действительно короткие.
Вероятно, разработчики в Microsoft выбрали размер в 16 символов как разумный баланс.

Теперь очень важный момент в конце функции main(): мы не пользуемся методом c\_str(), тем не менее,
если это скомпилировать и запустить, то обе строки появятся в консоли!

Работает это вот почему.

В первом случае строка короче 16-и символов и в начале объекта std::string (его можно рассматривать
просто как структуру) расположен буфер с этой строкой.
\printf трактует указатель как указатель на массив символов оканчивающийся нулем и поэтому всё работает.

Вывод второй строки (длиннее 16-и символов) даже еще опаснее: это вообще типичная программистская ошибка 
(или опечатка), забыть дописать c\_str().
Это работает потому что в это время в начале структуры расположен указатель на буфер.
Это может надолго остаться незамеченным: до тех пока там не появится строка 
короче 16-и символов, тогда процесс упадет.

\mysubparagraph{GCC}

В реализации GCC в структуре есть еще одна переменная --- reference count.

Интересно, что указатель на экземпляр класса std::string в GCC указывает не на начало самой структуры, 
а на указатель на буфера.
В libstdc++-v3\textbackslash{}include\textbackslash{}bits\textbackslash{}basic\_string.h 
мы можем прочитать что это сделано для удобства отладки:

\begin{lstlisting}
   *  The reason you want _M_data pointing to the character %array and
   *  not the _Rep is so that the debugger can see the string
   *  contents. (Probably we should add a non-inline member to get
   *  the _Rep for the debugger to use, so users can check the actual
   *  string length.)
\end{lstlisting}

\href{http://go.yurichev.com/17085}{исходный код basic\_string.h}

В нашем примере мы учитываем это:

\lstinputlisting[caption=пример для GCC,style=customc]{\CURPATH/STL/string/GCC_RU.cpp}

Нужны еще небольшие хаки чтобы сымитировать типичную ошибку, которую мы уже видели выше, из-за
более ужесточенной проверки типов в GCC, тем не менее, printf() работает и здесь без c\_str().

\myparagraph{Чуть более сложный пример}

\lstinputlisting[style=customc]{\CURPATH/STL/string/3.cpp}

\lstinputlisting[caption=MSVC 2012,style=customasmx86]{\CURPATH/STL/string/3_MSVC_RU.asm}

Собственно, компилятор не конструирует строки статически: да в общем-то и как
это возможно, если буфер с ней нужно хранить в \glslink{heap}{куче}?

Вместо этого в сегменте данных хранятся обычные \ac{ASCIIZ}-строки, а позже, во время выполнения, 
при помощи метода \q{assign}, конструируются строки s1 и s2
.
При помощи \TT{operator+}, создается строка s3.

Обратите внимание на то что вызов метода c\_str() отсутствует,
потому что его код достаточно короткий и компилятор вставил его прямо здесь:
если строка короче 16-и байт, то в регистре EAX остается указатель на буфер,
а если длиннее, то из этого же места достается адрес на буфер расположенный в \glslink{heap}{куче}.

Далее следуют вызовы трех деструкторов, причем, они вызываются только если строка длиннее 16-и байт:
тогда нужно освободить буфера в \glslink{heap}{куче}.
В противном случае, так как все три объекта std::string хранятся в стеке,
они освобождаются автоматически после выхода из функции.

Следовательно, работа с короткими строками более быстрая из-за м\'{е}ньшего обращения к \glslink{heap}{куче}.

Код на GCC даже проще (из-за того, что в GCC, как мы уже видели, не реализована возможность хранить короткую
строку прямо в структуре):

% TODO1 comment each function meaning
\lstinputlisting[caption=GCC 4.8.1,style=customasmx86]{\CURPATH/STL/string/3_GCC_RU.s}

Можно заметить, что в деструкторы передается не указатель на объект,
а указатель на место за 12 байт (или 3 слова) перед ним, то есть, на настоящее начало структуры.

\myparagraph{std::string как глобальная переменная}
\label{sec:std_string_as_global_variable}

Опытные программисты на \Cpp знают, что глобальные переменные \ac{STL}-типов вполне можно объявлять.

Да, действительно:

\lstinputlisting[style=customc]{\CURPATH/STL/string/5.cpp}

Но как и где будет вызываться конструктор \TT{std::string}?

На самом деле, эта переменная будет инициализирована даже перед началом \main.

\lstinputlisting[caption=MSVC 2012: здесь конструируется глобальная переменная{,} а также регистрируется её деструктор,style=customasmx86]{\CURPATH/STL/string/5_MSVC_p2.asm}

\lstinputlisting[caption=MSVC 2012: здесь глобальная переменная используется в \main,style=customasmx86]{\CURPATH/STL/string/5_MSVC_p1.asm}

\lstinputlisting[caption=MSVC 2012: эта функция-деструктор вызывается перед выходом,style=customasmx86]{\CURPATH/STL/string/5_MSVC_p3.asm}

\myindex{\CStandardLibrary!atexit()}
В реальности, из \ac{CRT}, еще до вызова main(), вызывается специальная функция,
в которой перечислены все конструкторы подобных переменных.
Более того: при помощи atexit() регистрируется функция, которая будет вызвана в конце работы программы:
в этой функции компилятор собирает вызовы деструкторов всех подобных глобальных переменных.

GCC работает похожим образом:

\lstinputlisting[caption=GCC 4.8.1,style=customasmx86]{\CURPATH/STL/string/5_GCC.s}

Но он не выделяет отдельной функции в которой будут собраны деструкторы: 
каждый деструктор передается в atexit() по одному.

% TODO а если глобальная STL-переменная в другом модуле? надо проверить.

}

\EN{\section{Returning Values}
\label{ret_val_func}

Another simple function is the one that simply returns a constant value:

\lstinputlisting[caption=\EN{\CCpp Code},style=customc]{patterns/011_ret/1.c}

Let's compile it.

\subsection{x86}

Here's what both the GCC and MSVC compilers produce (with optimization) on the x86 platform:

\lstinputlisting[caption=\Optimizing GCC/MSVC (\assemblyOutput),style=customasmx86]{patterns/011_ret/1.s}

\myindex{x86!\Instructions!RET}
There are just two instructions: the first places the value 123 into the \EAX register,
which is used by convention for storing the return
value, and the second one is \RET, which returns execution to the \gls{caller}.

The caller will take the result from the \EAX register.

\subsection{ARM}

There are a few differences on the ARM platform:

\lstinputlisting[caption=\OptimizingKeilVI (\ARMMode) ASM Output,style=customasmARM]{patterns/011_ret/1_Keil_ARM_O3.s}

ARM uses the register \Reg{0} for returning the results of functions, so 123 is copied into \Reg{0}.

\myindex{ARM!\Instructions!MOV}
\myindex{x86!\Instructions!MOV}
It is worth noting that \MOV is a misleading name for the instruction in both the x86 and ARM \ac{ISA}s.

The data is not in fact \IT{moved}, but \IT{copied}.

\subsection{MIPS}

\label{MIPS_leaf_function_ex1}

The GCC assembly output below lists registers by number:

\lstinputlisting[caption=\Optimizing GCC 4.4.5 (\assemblyOutput),style=customasmMIPS]{patterns/011_ret/MIPS.s}

\dots while \IDA does it by their pseudo names:

\lstinputlisting[caption=\Optimizing GCC 4.4.5 (IDA),style=customasmMIPS]{patterns/011_ret/MIPS_IDA.lst}

The \$2 (or \$V0) register is used to store the function's return value.
\myindex{MIPS!\Pseudoinstructions!LI}
\INS{LI} stands for ``Load Immediate'' and is the MIPS equivalent to \MOV.

\myindex{MIPS!\Instructions!J}
The other instruction is the jump instruction (J or JR) which returns the execution flow to the \gls{caller}.

\myindex{MIPS!Branch delay slot}
You might be wondering why the positions of the load instruction (LI) and the jump instruction (J or JR) are swapped. This is due to a \ac{RISC} feature called ``branch delay slot''.

The reason this happens is a quirk in the architecture of some RISC \ac{ISA}s and isn't important for our
purposes---we must simply keep in mind that in MIPS, the instruction following a jump or branch instruction
is executed \IT{before} the jump/branch instruction itself.

As a consequence, branch instructions always swap places with the instruction executed immediately beforehand.


In practice, functions which merely return 1 (\IT{true}) or 0 (\IT{false}) are very frequent.

The smallest ever of the standard UNIX utilities, \IT{/bin/true} and \IT{/bin/false} return 0 and 1 respectively, as an exit code.
(Zero as an exit code usually means success, non-zero means error.)
}
\RU{\subsubsection{std::string}
\myindex{\Cpp!STL!std::string}
\label{std_string}

\myparagraph{Как устроена структура}

Многие строковые библиотеки \InSqBrackets{\CNotes 2.2} обеспечивают структуру содержащую ссылку 
на буфер собственно со строкой, переменная всегда содержащую длину строки 
(что очень удобно для массы функций \InSqBrackets{\CNotes 2.2.1}) и переменную содержащую текущий размер буфера.

Строка в буфере обыкновенно оканчивается нулем: это для того чтобы указатель на буфер можно было
передавать в функции требующие на вход обычную сишную \ac{ASCIIZ}-строку.

Стандарт \Cpp не описывает, как именно нужно реализовывать std::string,
но, как правило, они реализованы как описано выше, с небольшими дополнениями.

Строки в \Cpp это не класс (как, например, QString в Qt), а темплейт (basic\_string), 
это сделано для того чтобы поддерживать 
строки содержащие разного типа символы: как минимум \Tchar и \IT{wchar\_t}.

Так что, std::string это класс с базовым типом \Tchar.

А std::wstring это класс с базовым типом \IT{wchar\_t}.

\mysubparagraph{MSVC}

В реализации MSVC, вместо ссылки на буфер может содержаться сам буфер (если строка короче 16-и символов).

Это означает, что каждая короткая строка будет занимать в памяти по крайней мере $16 + 4 + 4 = 24$ 
байт для 32-битной среды либо $16 + 8 + 8 = 32$ 
байта в 64-битной, а если строка длиннее 16-и символов, то прибавьте еще длину самой строки.

\lstinputlisting[caption=пример для MSVC,style=customc]{\CURPATH/STL/string/MSVC_RU.cpp}

Собственно, из этого исходника почти всё ясно.

Несколько замечаний:

Если строка короче 16-и символов, 
то отдельный буфер для строки в \glslink{heap}{куче} выделяться не будет.

Это удобно потому что на практике, основная часть строк действительно короткие.
Вероятно, разработчики в Microsoft выбрали размер в 16 символов как разумный баланс.

Теперь очень важный момент в конце функции main(): мы не пользуемся методом c\_str(), тем не менее,
если это скомпилировать и запустить, то обе строки появятся в консоли!

Работает это вот почему.

В первом случае строка короче 16-и символов и в начале объекта std::string (его можно рассматривать
просто как структуру) расположен буфер с этой строкой.
\printf трактует указатель как указатель на массив символов оканчивающийся нулем и поэтому всё работает.

Вывод второй строки (длиннее 16-и символов) даже еще опаснее: это вообще типичная программистская ошибка 
(или опечатка), забыть дописать c\_str().
Это работает потому что в это время в начале структуры расположен указатель на буфер.
Это может надолго остаться незамеченным: до тех пока там не появится строка 
короче 16-и символов, тогда процесс упадет.

\mysubparagraph{GCC}

В реализации GCC в структуре есть еще одна переменная --- reference count.

Интересно, что указатель на экземпляр класса std::string в GCC указывает не на начало самой структуры, 
а на указатель на буфера.
В libstdc++-v3\textbackslash{}include\textbackslash{}bits\textbackslash{}basic\_string.h 
мы можем прочитать что это сделано для удобства отладки:

\begin{lstlisting}
   *  The reason you want _M_data pointing to the character %array and
   *  not the _Rep is so that the debugger can see the string
   *  contents. (Probably we should add a non-inline member to get
   *  the _Rep for the debugger to use, so users can check the actual
   *  string length.)
\end{lstlisting}

\href{http://go.yurichev.com/17085}{исходный код basic\_string.h}

В нашем примере мы учитываем это:

\lstinputlisting[caption=пример для GCC,style=customc]{\CURPATH/STL/string/GCC_RU.cpp}

Нужны еще небольшие хаки чтобы сымитировать типичную ошибку, которую мы уже видели выше, из-за
более ужесточенной проверки типов в GCC, тем не менее, printf() работает и здесь без c\_str().

\myparagraph{Чуть более сложный пример}

\lstinputlisting[style=customc]{\CURPATH/STL/string/3.cpp}

\lstinputlisting[caption=MSVC 2012,style=customasmx86]{\CURPATH/STL/string/3_MSVC_RU.asm}

Собственно, компилятор не конструирует строки статически: да в общем-то и как
это возможно, если буфер с ней нужно хранить в \glslink{heap}{куче}?

Вместо этого в сегменте данных хранятся обычные \ac{ASCIIZ}-строки, а позже, во время выполнения, 
при помощи метода \q{assign}, конструируются строки s1 и s2
.
При помощи \TT{operator+}, создается строка s3.

Обратите внимание на то что вызов метода c\_str() отсутствует,
потому что его код достаточно короткий и компилятор вставил его прямо здесь:
если строка короче 16-и байт, то в регистре EAX остается указатель на буфер,
а если длиннее, то из этого же места достается адрес на буфер расположенный в \glslink{heap}{куче}.

Далее следуют вызовы трех деструкторов, причем, они вызываются только если строка длиннее 16-и байт:
тогда нужно освободить буфера в \glslink{heap}{куче}.
В противном случае, так как все три объекта std::string хранятся в стеке,
они освобождаются автоматически после выхода из функции.

Следовательно, работа с короткими строками более быстрая из-за м\'{е}ньшего обращения к \glslink{heap}{куче}.

Код на GCC даже проще (из-за того, что в GCC, как мы уже видели, не реализована возможность хранить короткую
строку прямо в структуре):

% TODO1 comment each function meaning
\lstinputlisting[caption=GCC 4.8.1,style=customasmx86]{\CURPATH/STL/string/3_GCC_RU.s}

Можно заметить, что в деструкторы передается не указатель на объект,
а указатель на место за 12 байт (или 3 слова) перед ним, то есть, на настоящее начало структуры.

\myparagraph{std::string как глобальная переменная}
\label{sec:std_string_as_global_variable}

Опытные программисты на \Cpp знают, что глобальные переменные \ac{STL}-типов вполне можно объявлять.

Да, действительно:

\lstinputlisting[style=customc]{\CURPATH/STL/string/5.cpp}

Но как и где будет вызываться конструктор \TT{std::string}?

На самом деле, эта переменная будет инициализирована даже перед началом \main.

\lstinputlisting[caption=MSVC 2012: здесь конструируется глобальная переменная{,} а также регистрируется её деструктор,style=customasmx86]{\CURPATH/STL/string/5_MSVC_p2.asm}

\lstinputlisting[caption=MSVC 2012: здесь глобальная переменная используется в \main,style=customasmx86]{\CURPATH/STL/string/5_MSVC_p1.asm}

\lstinputlisting[caption=MSVC 2012: эта функция-деструктор вызывается перед выходом,style=customasmx86]{\CURPATH/STL/string/5_MSVC_p3.asm}

\myindex{\CStandardLibrary!atexit()}
В реальности, из \ac{CRT}, еще до вызова main(), вызывается специальная функция,
в которой перечислены все конструкторы подобных переменных.
Более того: при помощи atexit() регистрируется функция, которая будет вызвана в конце работы программы:
в этой функции компилятор собирает вызовы деструкторов всех подобных глобальных переменных.

GCC работает похожим образом:

\lstinputlisting[caption=GCC 4.8.1,style=customasmx86]{\CURPATH/STL/string/5_GCC.s}

Но он не выделяет отдельной функции в которой будут собраны деструкторы: 
каждый деструктор передается в atexit() по одному.

% TODO а если глобальная STL-переменная в другом модуле? надо проверить.

}
\DE{\subsection{Einfachste XOR-Verschlüsselung überhaupt}

Ich habe einmal eine Software gesehen, bei der alle Debugging-Ausgaben mit XOR mit dem Wert 3
verschlüsselt wurden. Mit anderen Worten, die beiden niedrigsten Bits aller Buchstaben wurden invertiert.

``Hello, world'' wurde zu ``Kfool/\#tlqog'':

\begin{lstlisting}
#!/usr/bin/python

msg="Hello, world!"

print "".join(map(lambda x: chr(ord(x)^3), msg))
\end{lstlisting}

Das ist eine ziemlich interessante Verschlüsselung (oder besser eine Verschleierung),
weil sie zwei wichtige Eigenschaften hat:
1) es ist eine einzige Funktion zum Verschlüsseln und entschlüsseln, sie muss nur wiederholt angewendet werden
2) die entstehenden Buchstaben befinden sich im druckbaren Bereich, also die ganze Zeichenkette kann ohne
Escape-Symbole im Code verwendet werden.

Die zweite Eigenschaft nutzt die Tatsache, dass alle druckbaren Zeichen in Reihen organisiert sind: 0x2x-0x7x,
und wenn die beiden niederwertigsten Bits invertiert werden, wird der Buchstabe um eine oder drei Stellen nach
links oder rechts \IT{verschoben}, aber niemals in eine andere Reihe:

\begin{figure}[H]
\centering
\includegraphics[width=0.7\textwidth]{ascii_clean.png}
\caption{7-Bit \ac{ASCII} Tabelle in Emacs}
\end{figure}

\dots mit dem Zeichen 0x7F als einziger Ausnahme.

Im Folgenden werden also beispielsweise die Zeichen A-Z \IT{verschlüsselt}:

\begin{lstlisting}
#!/usr/bin/python

msg="@ABCDEFGHIJKLMNO"

print "".join(map(lambda x: chr(ord(x)^3), msg))
\end{lstlisting}

Ergebnis:
% FIXME \verb  --  relevant comment for German?
\begin{lstlisting}
CBA@GFEDKJIHONML
\end{lstlisting}

Es sieht so aus als würden die Zeichen ``@'' und ``C'' sowie ``B'' und ``A'' vertauscht werden.

Hier ist noch ein interessantes Beispiel, in dem gezeigt wird, wie die Eigenschaften von XOR
ausgenutzt werden können: Exakt den gleichen Effekt, dass druckbare Zeichen auch druckbar bleiben,
kann man dadurch erzielen, dass irgendeine Kombination der niedrigsten vier Bits invertiert wird.
}

\EN{\section{Returning Values}
\label{ret_val_func}

Another simple function is the one that simply returns a constant value:

\lstinputlisting[caption=\EN{\CCpp Code},style=customc]{patterns/011_ret/1.c}

Let's compile it.

\subsection{x86}

Here's what both the GCC and MSVC compilers produce (with optimization) on the x86 platform:

\lstinputlisting[caption=\Optimizing GCC/MSVC (\assemblyOutput),style=customasmx86]{patterns/011_ret/1.s}

\myindex{x86!\Instructions!RET}
There are just two instructions: the first places the value 123 into the \EAX register,
which is used by convention for storing the return
value, and the second one is \RET, which returns execution to the \gls{caller}.

The caller will take the result from the \EAX register.

\subsection{ARM}

There are a few differences on the ARM platform:

\lstinputlisting[caption=\OptimizingKeilVI (\ARMMode) ASM Output,style=customasmARM]{patterns/011_ret/1_Keil_ARM_O3.s}

ARM uses the register \Reg{0} for returning the results of functions, so 123 is copied into \Reg{0}.

\myindex{ARM!\Instructions!MOV}
\myindex{x86!\Instructions!MOV}
It is worth noting that \MOV is a misleading name for the instruction in both the x86 and ARM \ac{ISA}s.

The data is not in fact \IT{moved}, but \IT{copied}.

\subsection{MIPS}

\label{MIPS_leaf_function_ex1}

The GCC assembly output below lists registers by number:

\lstinputlisting[caption=\Optimizing GCC 4.4.5 (\assemblyOutput),style=customasmMIPS]{patterns/011_ret/MIPS.s}

\dots while \IDA does it by their pseudo names:

\lstinputlisting[caption=\Optimizing GCC 4.4.5 (IDA),style=customasmMIPS]{patterns/011_ret/MIPS_IDA.lst}

The \$2 (or \$V0) register is used to store the function's return value.
\myindex{MIPS!\Pseudoinstructions!LI}
\INS{LI} stands for ``Load Immediate'' and is the MIPS equivalent to \MOV.

\myindex{MIPS!\Instructions!J}
The other instruction is the jump instruction (J or JR) which returns the execution flow to the \gls{caller}.

\myindex{MIPS!Branch delay slot}
You might be wondering why the positions of the load instruction (LI) and the jump instruction (J or JR) are swapped. This is due to a \ac{RISC} feature called ``branch delay slot''.

The reason this happens is a quirk in the architecture of some RISC \ac{ISA}s and isn't important for our
purposes---we must simply keep in mind that in MIPS, the instruction following a jump or branch instruction
is executed \IT{before} the jump/branch instruction itself.

As a consequence, branch instructions always swap places with the instruction executed immediately beforehand.


In practice, functions which merely return 1 (\IT{true}) or 0 (\IT{false}) are very frequent.

The smallest ever of the standard UNIX utilities, \IT{/bin/true} and \IT{/bin/false} return 0 and 1 respectively, as an exit code.
(Zero as an exit code usually means success, non-zero means error.)
}
\RU{\subsubsection{std::string}
\myindex{\Cpp!STL!std::string}
\label{std_string}

\myparagraph{Как устроена структура}

Многие строковые библиотеки \InSqBrackets{\CNotes 2.2} обеспечивают структуру содержащую ссылку 
на буфер собственно со строкой, переменная всегда содержащую длину строки 
(что очень удобно для массы функций \InSqBrackets{\CNotes 2.2.1}) и переменную содержащую текущий размер буфера.

Строка в буфере обыкновенно оканчивается нулем: это для того чтобы указатель на буфер можно было
передавать в функции требующие на вход обычную сишную \ac{ASCIIZ}-строку.

Стандарт \Cpp не описывает, как именно нужно реализовывать std::string,
но, как правило, они реализованы как описано выше, с небольшими дополнениями.

Строки в \Cpp это не класс (как, например, QString в Qt), а темплейт (basic\_string), 
это сделано для того чтобы поддерживать 
строки содержащие разного типа символы: как минимум \Tchar и \IT{wchar\_t}.

Так что, std::string это класс с базовым типом \Tchar.

А std::wstring это класс с базовым типом \IT{wchar\_t}.

\mysubparagraph{MSVC}

В реализации MSVC, вместо ссылки на буфер может содержаться сам буфер (если строка короче 16-и символов).

Это означает, что каждая короткая строка будет занимать в памяти по крайней мере $16 + 4 + 4 = 24$ 
байт для 32-битной среды либо $16 + 8 + 8 = 32$ 
байта в 64-битной, а если строка длиннее 16-и символов, то прибавьте еще длину самой строки.

\lstinputlisting[caption=пример для MSVC,style=customc]{\CURPATH/STL/string/MSVC_RU.cpp}

Собственно, из этого исходника почти всё ясно.

Несколько замечаний:

Если строка короче 16-и символов, 
то отдельный буфер для строки в \glslink{heap}{куче} выделяться не будет.

Это удобно потому что на практике, основная часть строк действительно короткие.
Вероятно, разработчики в Microsoft выбрали размер в 16 символов как разумный баланс.

Теперь очень важный момент в конце функции main(): мы не пользуемся методом c\_str(), тем не менее,
если это скомпилировать и запустить, то обе строки появятся в консоли!

Работает это вот почему.

В первом случае строка короче 16-и символов и в начале объекта std::string (его можно рассматривать
просто как структуру) расположен буфер с этой строкой.
\printf трактует указатель как указатель на массив символов оканчивающийся нулем и поэтому всё работает.

Вывод второй строки (длиннее 16-и символов) даже еще опаснее: это вообще типичная программистская ошибка 
(или опечатка), забыть дописать c\_str().
Это работает потому что в это время в начале структуры расположен указатель на буфер.
Это может надолго остаться незамеченным: до тех пока там не появится строка 
короче 16-и символов, тогда процесс упадет.

\mysubparagraph{GCC}

В реализации GCC в структуре есть еще одна переменная --- reference count.

Интересно, что указатель на экземпляр класса std::string в GCC указывает не на начало самой структуры, 
а на указатель на буфера.
В libstdc++-v3\textbackslash{}include\textbackslash{}bits\textbackslash{}basic\_string.h 
мы можем прочитать что это сделано для удобства отладки:

\begin{lstlisting}
   *  The reason you want _M_data pointing to the character %array and
   *  not the _Rep is so that the debugger can see the string
   *  contents. (Probably we should add a non-inline member to get
   *  the _Rep for the debugger to use, so users can check the actual
   *  string length.)
\end{lstlisting}

\href{http://go.yurichev.com/17085}{исходный код basic\_string.h}

В нашем примере мы учитываем это:

\lstinputlisting[caption=пример для GCC,style=customc]{\CURPATH/STL/string/GCC_RU.cpp}

Нужны еще небольшие хаки чтобы сымитировать типичную ошибку, которую мы уже видели выше, из-за
более ужесточенной проверки типов в GCC, тем не менее, printf() работает и здесь без c\_str().

\myparagraph{Чуть более сложный пример}

\lstinputlisting[style=customc]{\CURPATH/STL/string/3.cpp}

\lstinputlisting[caption=MSVC 2012,style=customasmx86]{\CURPATH/STL/string/3_MSVC_RU.asm}

Собственно, компилятор не конструирует строки статически: да в общем-то и как
это возможно, если буфер с ней нужно хранить в \glslink{heap}{куче}?

Вместо этого в сегменте данных хранятся обычные \ac{ASCIIZ}-строки, а позже, во время выполнения, 
при помощи метода \q{assign}, конструируются строки s1 и s2
.
При помощи \TT{operator+}, создается строка s3.

Обратите внимание на то что вызов метода c\_str() отсутствует,
потому что его код достаточно короткий и компилятор вставил его прямо здесь:
если строка короче 16-и байт, то в регистре EAX остается указатель на буфер,
а если длиннее, то из этого же места достается адрес на буфер расположенный в \glslink{heap}{куче}.

Далее следуют вызовы трех деструкторов, причем, они вызываются только если строка длиннее 16-и байт:
тогда нужно освободить буфера в \glslink{heap}{куче}.
В противном случае, так как все три объекта std::string хранятся в стеке,
они освобождаются автоматически после выхода из функции.

Следовательно, работа с короткими строками более быстрая из-за м\'{е}ньшего обращения к \glslink{heap}{куче}.

Код на GCC даже проще (из-за того, что в GCC, как мы уже видели, не реализована возможность хранить короткую
строку прямо в структуре):

% TODO1 comment each function meaning
\lstinputlisting[caption=GCC 4.8.1,style=customasmx86]{\CURPATH/STL/string/3_GCC_RU.s}

Можно заметить, что в деструкторы передается не указатель на объект,
а указатель на место за 12 байт (или 3 слова) перед ним, то есть, на настоящее начало структуры.

\myparagraph{std::string как глобальная переменная}
\label{sec:std_string_as_global_variable}

Опытные программисты на \Cpp знают, что глобальные переменные \ac{STL}-типов вполне можно объявлять.

Да, действительно:

\lstinputlisting[style=customc]{\CURPATH/STL/string/5.cpp}

Но как и где будет вызываться конструктор \TT{std::string}?

На самом деле, эта переменная будет инициализирована даже перед началом \main.

\lstinputlisting[caption=MSVC 2012: здесь конструируется глобальная переменная{,} а также регистрируется её деструктор,style=customasmx86]{\CURPATH/STL/string/5_MSVC_p2.asm}

\lstinputlisting[caption=MSVC 2012: здесь глобальная переменная используется в \main,style=customasmx86]{\CURPATH/STL/string/5_MSVC_p1.asm}

\lstinputlisting[caption=MSVC 2012: эта функция-деструктор вызывается перед выходом,style=customasmx86]{\CURPATH/STL/string/5_MSVC_p3.asm}

\myindex{\CStandardLibrary!atexit()}
В реальности, из \ac{CRT}, еще до вызова main(), вызывается специальная функция,
в которой перечислены все конструкторы подобных переменных.
Более того: при помощи atexit() регистрируется функция, которая будет вызвана в конце работы программы:
в этой функции компилятор собирает вызовы деструкторов всех подобных глобальных переменных.

GCC работает похожим образом:

\lstinputlisting[caption=GCC 4.8.1,style=customasmx86]{\CURPATH/STL/string/5_GCC.s}

Но он не выделяет отдельной функции в которой будут собраны деструкторы: 
каждый деструктор передается в atexit() по одному.

% TODO а если глобальная STL-переменная в другом модуле? надо проверить.

}
\DE{\subsection{Einfachste XOR-Verschlüsselung überhaupt}

Ich habe einmal eine Software gesehen, bei der alle Debugging-Ausgaben mit XOR mit dem Wert 3
verschlüsselt wurden. Mit anderen Worten, die beiden niedrigsten Bits aller Buchstaben wurden invertiert.

``Hello, world'' wurde zu ``Kfool/\#tlqog'':

\begin{lstlisting}
#!/usr/bin/python

msg="Hello, world!"

print "".join(map(lambda x: chr(ord(x)^3), msg))
\end{lstlisting}

Das ist eine ziemlich interessante Verschlüsselung (oder besser eine Verschleierung),
weil sie zwei wichtige Eigenschaften hat:
1) es ist eine einzige Funktion zum Verschlüsseln und entschlüsseln, sie muss nur wiederholt angewendet werden
2) die entstehenden Buchstaben befinden sich im druckbaren Bereich, also die ganze Zeichenkette kann ohne
Escape-Symbole im Code verwendet werden.

Die zweite Eigenschaft nutzt die Tatsache, dass alle druckbaren Zeichen in Reihen organisiert sind: 0x2x-0x7x,
und wenn die beiden niederwertigsten Bits invertiert werden, wird der Buchstabe um eine oder drei Stellen nach
links oder rechts \IT{verschoben}, aber niemals in eine andere Reihe:

\begin{figure}[H]
\centering
\includegraphics[width=0.7\textwidth]{ascii_clean.png}
\caption{7-Bit \ac{ASCII} Tabelle in Emacs}
\end{figure}

\dots mit dem Zeichen 0x7F als einziger Ausnahme.

Im Folgenden werden also beispielsweise die Zeichen A-Z \IT{verschlüsselt}:

\begin{lstlisting}
#!/usr/bin/python

msg="@ABCDEFGHIJKLMNO"

print "".join(map(lambda x: chr(ord(x)^3), msg))
\end{lstlisting}

Ergebnis:
% FIXME \verb  --  relevant comment for German?
\begin{lstlisting}
CBA@GFEDKJIHONML
\end{lstlisting}

Es sieht so aus als würden die Zeichen ``@'' und ``C'' sowie ``B'' und ``A'' vertauscht werden.

Hier ist noch ein interessantes Beispiel, in dem gezeigt wird, wie die Eigenschaften von XOR
ausgenutzt werden können: Exakt den gleichen Effekt, dass druckbare Zeichen auch druckbar bleiben,
kann man dadurch erzielen, dass irgendeine Kombination der niedrigsten vier Bits invertiert wird.
}

\ifdefined\SPANISH
\chapter{Patrones de código}
\fi % SPANISH

\ifdefined\GERMAN
\chapter{Code-Muster}
\fi % GERMAN

\ifdefined\ENGLISH
\chapter{Code Patterns}
\fi % ENGLISH

\ifdefined\ITALIAN
\chapter{Forme di codice}
\fi % ITALIAN

\ifdefined\RUSSIAN
\chapter{Образцы кода}
\fi % RUSSIAN

\ifdefined\BRAZILIAN
\chapter{Padrões de códigos}
\fi % BRAZILIAN

\ifdefined\THAI
\chapter{รูปแบบของโค้ด}
\fi % THAI

\ifdefined\FRENCH
\chapter{Modèle de code}
\fi % FRENCH

\ifdefined\POLISH
\chapter{\PLph{}}
\fi % POLISH

% sections
\EN{\input{patterns/patterns_opt_dbg_EN}}
\ES{\input{patterns/patterns_opt_dbg_ES}}
\ITA{\input{patterns/patterns_opt_dbg_ITA}}
\PTBR{\input{patterns/patterns_opt_dbg_PTBR}}
\RU{\input{patterns/patterns_opt_dbg_RU}}
\THA{\input{patterns/patterns_opt_dbg_THA}}
\DE{\input{patterns/patterns_opt_dbg_DE}}
\FR{\input{patterns/patterns_opt_dbg_FR}}
\PL{\input{patterns/patterns_opt_dbg_PL}}

\RU{\section{Некоторые базовые понятия}}
\EN{\section{Some basics}}
\DE{\section{Einige Grundlagen}}
\FR{\section{Quelques bases}}
\ES{\section{\ESph{}}}
\ITA{\section{Alcune basi teoriche}}
\PTBR{\section{\PTBRph{}}}
\THA{\section{\THAph{}}}
\PL{\section{\PLph{}}}

% sections:
\EN{\input{patterns/intro_CPU_ISA_EN}}
\ES{\input{patterns/intro_CPU_ISA_ES}}
\ITA{\input{patterns/intro_CPU_ISA_ITA}}
\PTBR{\input{patterns/intro_CPU_ISA_PTBR}}
\RU{\input{patterns/intro_CPU_ISA_RU}}
\DE{\input{patterns/intro_CPU_ISA_DE}}
\FR{\input{patterns/intro_CPU_ISA_FR}}
\PL{\input{patterns/intro_CPU_ISA_PL}}

\EN{\input{patterns/numeral_EN}}
\RU{\input{patterns/numeral_RU}}
\ITA{\input{patterns/numeral_ITA}}
\DE{\input{patterns/numeral_DE}}
\FR{\input{patterns/numeral_FR}}
\PL{\input{patterns/numeral_PL}}

% chapters
\input{patterns/00_empty/main}
\input{patterns/011_ret/main}
\input{patterns/01_helloworld/main}
\input{patterns/015_prolog_epilogue/main}
\input{patterns/02_stack/main}
\input{patterns/03_printf/main}
\input{patterns/04_scanf/main}
\input{patterns/05_passing_arguments/main}
\input{patterns/06_return_results/main}
\input{patterns/061_pointers/main}
\input{patterns/065_GOTO/main}
\input{patterns/07_jcc/main}
\input{patterns/08_switch/main}
\input{patterns/09_loops/main}
\input{patterns/10_strings/main}
\input{patterns/11_arith_optimizations/main}
\input{patterns/12_FPU/main}
\input{patterns/13_arrays/main}
\input{patterns/14_bitfields/main}
\EN{\input{patterns/145_LCG/main_EN}}
\RU{\input{patterns/145_LCG/main_RU}}
\input{patterns/15_structs/main}
\input{patterns/17_unions/main}
\input{patterns/18_pointers_to_functions/main}
\input{patterns/185_64bit_in_32_env/main}

\EN{\input{patterns/19_SIMD/main_EN}}
\RU{\input{patterns/19_SIMD/main_RU}}
\DE{\input{patterns/19_SIMD/main_DE}}

\EN{\input{patterns/20_x64/main_EN}}
\RU{\input{patterns/20_x64/main_RU}}

\EN{\input{patterns/205_floating_SIMD/main_EN}}
\RU{\input{patterns/205_floating_SIMD/main_RU}}
\DE{\input{patterns/205_floating_SIMD/main_DE}}

\EN{\input{patterns/ARM/main_EN}}
\RU{\input{patterns/ARM/main_RU}}
\DE{\input{patterns/ARM/main_DE}}

\input{patterns/MIPS/main}


\ifdefined\SPANISH
\chapter{Patrones de código}
\fi % SPANISH

\ifdefined\GERMAN
\chapter{Code-Muster}
\fi % GERMAN

\ifdefined\ENGLISH
\chapter{Code Patterns}
\fi % ENGLISH

\ifdefined\ITALIAN
\chapter{Forme di codice}
\fi % ITALIAN

\ifdefined\RUSSIAN
\chapter{Образцы кода}
\fi % RUSSIAN

\ifdefined\BRAZILIAN
\chapter{Padrões de códigos}
\fi % BRAZILIAN

\ifdefined\THAI
\chapter{รูปแบบของโค้ด}
\fi % THAI

\ifdefined\FRENCH
\chapter{Modèle de code}
\fi % FRENCH

\ifdefined\POLISH
\chapter{\PLph{}}
\fi % POLISH

% sections
\EN{\section{The method}

When the author of this book first started learning C and, later, \Cpp, he used to write small pieces of code, compile them,
and then look at the assembly language output. This made it very easy for him to understand what was going on in the code that he had written.
\footnote{In fact, he still does this when he can't understand what a particular bit of code does.}.
He did this so many times that the relationship between the \CCpp code and what the compiler produced was imprinted deeply in his mind.
It's now easy for him to imagine instantly a rough outline of a C code's appearance and function.
Perhaps this technique could be helpful for others.

%There are a lot of examples for both x86/x64 and ARM.
%Those who already familiar with one of architectures, may freely skim over pages.

By the way, there is a great website where you can do the same, with various compilers, instead of installing them on your box.
You can use it as well: \url{https://gcc.godbolt.org/}.

\section*{\Exercises}

When the author of this book studied assembly language, he also often compiled small C functions and then rewrote
them gradually to assembly, trying to make their code as short as possible.
This probably is not worth doing in real-world scenarios today,
because it's hard to compete with the latest compilers in terms of efficiency. It is, however, a very good way to gain a better understanding of assembly.
Feel free, therefore, to take any assembly code from this book and try to make it shorter.
However, don't forget to test what you have written.

% rewrote to show that debug\release and optimisations levels are orthogonal concepts.
\section*{Optimization levels and debug information}

Source code can be compiled by different compilers with various optimization levels.
A typical compiler has about three such levels, where level zero means that optimization is completely disabled.
Optimization can also be targeted towards code size or code speed.
A non-optimizing compiler is faster and produces more understandable (albeit verbose) code,
whereas an optimizing compiler is slower and tries to produce code that runs faster (but is not necessarily more compact).
In addition to optimization levels, a compiler can include some debug information in the resulting file,
producing code that is easy to debug.
One of the important features of the ´debug' code is that it might contain links
between each line of the source code and its respective machine code address.
Optimizing compilers, on the other hand, tend to produce output where entire lines of source code
can be optimized away and thus not even be present in the resulting machine code.
Reverse engineers can encounter either version, simply because some developers turn on the compiler's optimization flags and others do not.
Because of this, we'll try to work on examples of both debug and release versions of the code featured in this book, wherever possible.

Sometimes some pretty ancient compilers are used in this book, in order to get the shortest (or simplest) possible code snippet.
}
\ES{\input{patterns/patterns_opt_dbg_ES}}
\ITA{\input{patterns/patterns_opt_dbg_ITA}}
\PTBR{\input{patterns/patterns_opt_dbg_PTBR}}
\RU{\input{patterns/patterns_opt_dbg_RU}}
\THA{\input{patterns/patterns_opt_dbg_THA}}
\DE{\input{patterns/patterns_opt_dbg_DE}}
\FR{\input{patterns/patterns_opt_dbg_FR}}
\PL{\input{patterns/patterns_opt_dbg_PL}}

\RU{\section{Некоторые базовые понятия}}
\EN{\section{Some basics}}
\DE{\section{Einige Grundlagen}}
\FR{\section{Quelques bases}}
\ES{\section{\ESph{}}}
\ITA{\section{Alcune basi teoriche}}
\PTBR{\section{\PTBRph{}}}
\THA{\section{\THAph{}}}
\PL{\section{\PLph{}}}

% sections:
\EN{\input{patterns/intro_CPU_ISA_EN}}
\ES{\input{patterns/intro_CPU_ISA_ES}}
\ITA{\input{patterns/intro_CPU_ISA_ITA}}
\PTBR{\input{patterns/intro_CPU_ISA_PTBR}}
\RU{\input{patterns/intro_CPU_ISA_RU}}
\DE{\input{patterns/intro_CPU_ISA_DE}}
\FR{\input{patterns/intro_CPU_ISA_FR}}
\PL{\input{patterns/intro_CPU_ISA_PL}}

\EN{\subsection{Numeral Systems}

Humans have become accustomed to a decimal numeral system, probably because almost everyone has 10 fingers.
Nevertheless, the number \q{10} has no significant meaning in science and mathematics.
The natural numeral system in digital electronics is binary: 0 is for an absence of current in the wire, and 1 for presence.
10 in binary is 2 in decimal, 100 in binary is 4 in decimal, and so on.

% This sentence is a bit unweildy - maybe try 'Our ten-digit system would be described as having a radix...' - Renaissance
If the numeral system has 10 digits, it has a \IT{radix} (or \IT{base}) of 10.
The binary numeral system has a \IT{radix} of 2.

Important things to recall:

1) A \IT{number} is a number, while a \IT{digit} is a term from writing systems, and is usually one character

% The original is 'number' is not changed; I think the intent is value, and changed it - Renaissance
2) The value of a number does not change when converted to another radix; only the writing notation for that value has changed (and therefore the way of representing it in \ac{RAM}).

\subsection{Converting From One Radix To Another}

Positional notation is used almost every numerical system. This means that a digit has weight relative to where it is placed inside of the larger number.
If 2 is placed at the rightmost place, it's 2, but if it's placed one digit before rightmost, it's 20.

What does $1234$ stand for?

$10^3 \cdot 1 + 10^2 \cdot 2 + 10^1 \cdot 3 + 1 \cdot 4 = 1234$ or
$1000 \cdot 1 + 100 \cdot 2 + 10 \cdot 3 + 4 = 1234$

It's the same story for binary numbers, but the base is 2 instead of 10.
What does 0b101011 stand for?

$2^5 \cdot 1 + 2^4 \cdot 0 + 2^3 \cdot 1 + 2^2 \cdot 0 + 2^1 \cdot 1 + 2^0 \cdot 1 = 43$ or
$32 \cdot 1 + 16 \cdot 0 + 8 \cdot 1 + 4 \cdot 0 + 2 \cdot 1 + 1 = 43$

There is such a thing as non-positional notation, such as the Roman numeral system.
\footnote{About numeric system evolution, see \InSqBrackets{\TAOCPvolII{}, 195--213.}}.
% Maybe add a sentence to fill in that X is always 10, and is therefore non-positional, even though putting an I before subtracts and after adds, and is in that sense positional
Perhaps, humankind switched to positional notation because it's easier to do basic operations (addition, multiplication, etc.) on paper by hand.

Binary numbers can be added, subtracted and so on in the very same as taught in schools, but only 2 digits are available.

Binary numbers are bulky when represented in source code and dumps, so that is where the hexadecimal numeral system can be useful.
A hexadecimal radix uses the digits 0..9, and also 6 Latin characters: A..F.
Each hexadecimal digit takes 4 bits or 4 binary digits, so it's very easy to convert from binary number to hexadecimal and back, even manually, in one's mind.

\begin{center}
\begin{longtable}{ | l | l | l | }
\hline
\HeaderColor hexadecimal & \HeaderColor binary & \HeaderColor decimal \\
\hline
0	&0000	&0 \\
1	&0001	&1 \\
2	&0010	&2 \\
3	&0011	&3 \\
4	&0100	&4 \\
5	&0101	&5 \\
6	&0110	&6 \\
7	&0111	&7 \\
8	&1000	&8 \\
9	&1001	&9 \\
A	&1010	&10 \\
B	&1011	&11 \\
C	&1100	&12 \\
D	&1101	&13 \\
E	&1110	&14 \\
F	&1111	&15 \\
\hline
\end{longtable}
\end{center}

How can one tell which radix is being used in a specific instance?

Decimal numbers are usually written as is, i.e., 1234. Some assemblers allow an identifier on decimal radix numbers, in which the number would be written with a "d" suffix: 1234d.

Binary numbers are sometimes prepended with the "0b" prefix: 0b100110111 (\ac{GCC} has a non-standard language extension for this\footnote{\url{https://gcc.gnu.org/onlinedocs/gcc/Binary-constants.html}}).
There is also another way: using a "b" suffix, for example: 100110111b.
This book tries to use the "0b" prefix consistently throughout the book for binary numbers.

Hexadecimal numbers are prepended with "0x" prefix in \CCpp and other \ac{PL}s: 0x1234ABCD.
Alternatively, they are given a "h" suffix: 1234ABCDh. This is common way of representing them in assemblers and debuggers.
In this convention, if the number is started with a Latin (A..F) digit, a 0 is added at the beginning: 0ABCDEFh.
There was also convention that was popular in 8-bit home computers era, using \$ prefix, like \$ABCD.
The book will try to stick to "0x" prefix throughout the book for hexadecimal numbers.

Should one learn to convert numbers mentally? A table of 1-digit hexadecimal numbers can easily be memorized.
As for larger numbers, it's probably not worth tormenting yourself.

Perhaps the most visible hexadecimal numbers are in \ac{URL}s.
This is the way that non-Latin characters are encoded.
For example:
\url{https://en.wiktionary.org/wiki/na\%C3\%AFvet\%C3\%A9} is the \ac{URL} of Wiktionary article about \q{naïveté} word.

\subsubsection{Octal Radix}

Another numeral system heavily used in the past of computer programming is octal. In octal there are 8 digits (0..7), and each is mapped to 3 bits, so it's easy to convert numbers back and forth.
It has been superseded by the hexadecimal system almost everywhere, but, surprisingly, there is a *NIX utility, used often by many people, which takes octal numbers as argument: \TT{chmod}.

\myindex{UNIX!chmod}
As many *NIX users know, \TT{chmod} argument can be a number of 3 digits. The first digit represents the rights of the owner of the file (read, write and/or execute), the second is the rights for the group to which the file belongs, and the third is for everyone else.
Each digit that \TT{chmod} takes can be represented in binary form:

\begin{center}
\begin{longtable}{ | l | l | l | }
\hline
\HeaderColor decimal & \HeaderColor binary & \HeaderColor meaning \\
\hline
7	&111	&\textbf{rwx} \\
6	&110	&\textbf{rw-} \\
5	&101	&\textbf{r-x} \\
4	&100	&\textbf{r-{}-} \\
3	&011	&\textbf{-wx} \\
2	&010	&\textbf{-w-} \\
1	&001	&\textbf{-{}-x} \\
0	&000	&\textbf{-{}-{}-} \\
\hline
\end{longtable}
\end{center}

So each bit is mapped to a flag: read/write/execute.

The importance of \TT{chmod} here is that the whole number in argument can be represented as octal number.
Let's take, for example, 644.
When you run \TT{chmod 644 file}, you set read/write permissions for owner, read permissions for group and again, read permissions for everyone else.
If we convert the octal number 644 to binary, it would be \TT{110100100}, or, in groups of 3 bits, \TT{110 100 100}.

Now we see that each triplet describe permissions for owner/group/others: first is \TT{rw-}, second is \TT{r--} and third is \TT{r--}.

The octal numeral system was also popular on old computers like PDP-8, because word there could be 12, 24 or 36 bits, and these numbers are all divisible by 3, so the octal system was natural in that environment.
Nowadays, all popular computers employ word/address sizes of 16, 32 or 64 bits, and these numbers are all divisible by 4, so the hexadecimal system is more natural there.

The octal numeral system is supported by all standard \CCpp compilers.
This is a source of confusion sometimes, because octal numbers are encoded with a zero prepended, for example, 0377 is 255.
Sometimes, you might make a typo and write "09" instead of 9, and the compiler would report an error.
GCC might report something like this:\\
\TT{error: invalid digit "9" in octal constant}.

Also, the octal system is somewhat popular in Java. When the IDA shows Java strings with non-printable characters,
they are encoded in the octal system instead of hexadecimal.
\myindex{JAD}
The JAD Java decompiler behaves the same way.

\subsubsection{Divisibility}

When you see a decimal number like 120, you can quickly deduce that it's divisible by 10, because the last digit is zero.
In the same way, 123400 is divisible by 100, because the two last digits are zeros.

Likewise, the hexadecimal number 0x1230 is divisible by 0x10 (or 16), 0x123000 is divisible by 0x1000 (or 4096), etc.

The binary number 0b1000101000 is divisible by 0b1000 (8), etc.

This property can often be used to quickly realize if the size of some block in memory is padded to some boundary.
For example, sections in \ac{PE} files are almost always started at addresses ending with 3 hexadecimal zeros: 0x41000, 0x10001000, etc.
The reason behind this is the fact that almost all \ac{PE} sections are padded to a boundary of 0x1000 (4096) bytes.

\subsubsection{Multi-Precision Arithmetic and Radix}

\index{RSA}
Multi-precision arithmetic can use huge numbers, and each one may be stored in several bytes.
For example, RSA keys, both public and private, span up to 4096 bits, and maybe even more.

% I'm not sure how to change this, but the normal format for quoting would be just to mention the author or book, and footnote to the full reference
In \InSqBrackets{\TAOCPvolII, 265} we find the following idea: when you store a multi-precision number in several bytes,
the whole number can be represented as having a radix of $2^8=256$, and each digit goes to the corresponding byte.
Likewise, if you store a multi-precision number in several 32-bit integer values, each digit goes to each 32-bit slot,
and you may think about this number as stored in radix of $2^{32}$.

\subsubsection{How to Pronounce Non-Decimal Numbers}

Numbers in a non-decimal base are usually pronounced by digit by digit: ``one-zero-zero-one-one-...''.
Words like ``ten'' and ``thousand'' are usually not pronounced, to prevent confusion with the decimal base system.

\subsubsection{Floating point numbers}

To distinguish floating point numbers from integers, they are usually written with ``.0'' at the end,
like $0.0$, $123.0$, etc.
}
\RU{\subsection{Представление чисел}

Люди привыкли к десятичной системе счисления вероятно потому что почти у каждого есть по 10 пальцев.
Тем не менее, число 10 не имеет особого значения в науке и математике.
Двоичная система естествена для цифровой электроники: 0 означает отсутствие тока в проводе и 1 --- его присутствие.
10 в двоичной системе это 2 в десятичной; 100 в двоичной это 4 в десятичной, итд.

Если в системе счисления есть 10 цифр, её \IT{основание} или \IT{radix} это 10.
Двоичная система имеет \IT{основание} 2.

Важные вещи, которые полезно вспомнить:
1) \IT{число} это число, в то время как \IT{цифра} это термин из системы письменности, и это обычно один символ;
2) само число не меняется, когда конвертируется из одного основания в другое: меняется способ его записи (или представления
в памяти).

Как сконвертировать число из одного основания в другое?

Позиционная нотация используется почти везде, это означает, что всякая цифра имеет свой вес, в зависимости от её расположения
внутри числа.
Если 2 расположена в самом последнем месте справа, это 2.
Если она расположена в месте перед последним, это 20.

Что означает $1234$?

$10^3 \cdot 1 + 10^2 \cdot 2 + 10^1 \cdot 3 + 1 \cdot 4$ = 1234 или
$1000 \cdot 1 + 100 \cdot 2 + 10 \cdot 3 + 4 = 1234$

Та же история и для двоичных чисел, только основание там 2 вместо 10.
Что означает 0b101011?

$2^5 \cdot 1 + 2^4 \cdot 0 + 2^3 \cdot 1 + 2^2 \cdot 0 + 2^1 \cdot 1 + 2^0 \cdot 1 = 43$ или
$32 \cdot 1 + 16 \cdot 0 + 8 \cdot 1 + 4 \cdot 0 + 2 \cdot 1 + 1 = 43$

Позиционную нотацию можно противопоставить непозиционной нотации, такой как римская система записи чисел
\footnote{Об эволюции способов записи чисел, см.также: \InSqBrackets{\TAOCPvolII{}, 195--213.}}.
Вероятно, человечество перешло на позиционную нотацию, потому что так проще работать с числами (сложение, умножение, итд)
на бумаге, в ручную.

Действительно, двоичные числа можно складывать, вычитать, итд, точно также, как этому обычно обучают в школах,
только доступны лишь 2 цифры.

Двоичные числа громоздки, когда их используют в исходных кодах и дампах, так что в этих случаях применяется шестнадцатеричная
система.
Используются цифры 0..9 и еще 6 латинских букв: A..F.
Каждая шестнадцатеричная цифра занимает 4 бита или 4 двоичных цифры, так что конвертировать из двоичной системы в
шестнадцатеричную и назад, можно легко вручную, или даже в уме.

\begin{center}
\begin{longtable}{ | l | l | l | }
\hline
\HeaderColor шестнадцатеричная & \HeaderColor двоичная & \HeaderColor десятичная \\
\hline
0	&0000	&0 \\
1	&0001	&1 \\
2	&0010	&2 \\
3	&0011	&3 \\
4	&0100	&4 \\
5	&0101	&5 \\
6	&0110	&6 \\
7	&0111	&7 \\
8	&1000	&8 \\
9	&1001	&9 \\
A	&1010	&10 \\
B	&1011	&11 \\
C	&1100	&12 \\
D	&1101	&13 \\
E	&1110	&14 \\
F	&1111	&15 \\
\hline
\end{longtable}
\end{center}

Как понять, какое основание используется в конкретном месте?

Десятичные числа обычно записываются как есть, т.е., 1234. Но некоторые ассемблеры позволяют подчеркивать
этот факт для ясности, и это число может быть дополнено суффиксом "d": 1234d.

К двоичным числам иногда спереди добавляют префикс "0b": 0b100110111
(В \ac{GCC} для этого есть нестандартное расширение языка
\footnote{\url{https://gcc.gnu.org/onlinedocs/gcc/Binary-constants.html}}).
Есть также еще один способ: суффикс "b", например: 100110111b.
В этой книге я буду пытаться придерживаться префикса "0b" для двоичных чисел.

Шестнадцатеричные числа имеют префикс "0x" в \CCpp и некоторых других \ac{PL}: 0x1234ABCD.
Либо они имеют суффикс "h": 1234ABCDh --- обычно так они представляются в ассемблерах и отладчиках.
Если число начинается с цифры A..F, перед ним добавляется 0: 0ABCDEFh.
Во времена 8-битных домашних компьютеров, был также способ записи чисел используя префикс \$, например, \$ABCD.
В книге я попытаюсь придерживаться префикса "0x" для шестнадцатеричных чисел.

Нужно ли учиться конвертировать числа в уме? Таблицу шестнадцатеричных чисел из одной цифры легко запомнить.
А запоминать б\'{о}льшие числа, наверное, не стоит.

Наверное, чаще всего шестнадцатеричные числа можно увидеть в \ac{URL}-ах.
Так кодируются буквы не из числа латинских.
Например:
\url{https://en.wiktionary.org/wiki/na\%C3\%AFvet\%C3\%A9} это \ac{URL} страницы в Wiktionary о слове \q{naïveté}.

\subsubsection{Восьмеричная система}

Еще одна система, которая в прошлом много использовалась в программировании это восьмеричная: есть 8 цифр (0..7) и каждая
описывает 3 бита, так что легко конвертировать числа туда и назад.
Она почти везде была заменена шестнадцатеричной, но удивительно, в *NIX имеется утилита использующаяся многими людьми,
которая принимает на вход восьмеричное число: \TT{chmod}.

\myindex{UNIX!chmod}
Как знают многие пользователи *NIX, аргумент \TT{chmod} это число из трех цифр. Первая цифра это права владельца файла,
вторая это права группы (которой файл принадлежит), третья для всех остальных.
И каждая цифра может быть представлена в двоичном виде:

\begin{center}
\begin{longtable}{ | l | l | l | }
\hline
\HeaderColor десятичная & \HeaderColor двоичная & \HeaderColor значение \\
\hline
7	&111	&\textbf{rwx} \\
6	&110	&\textbf{rw-} \\
5	&101	&\textbf{r-x} \\
4	&100	&\textbf{r-{}-} \\
3	&011	&\textbf{-wx} \\
2	&010	&\textbf{-w-} \\
1	&001	&\textbf{-{}-x} \\
0	&000	&\textbf{-{}-{}-} \\
\hline
\end{longtable}
\end{center}

Так что каждый бит привязан к флагу: read/write/execute (чтение/запись/исполнение).

И вот почему я вспомнил здесь о \TT{chmod}, это потому что всё число может быть представлено как число в восьмеричной системе.
Для примера возьмем 644.
Когда вы запускаете \TT{chmod 644 file}, вы выставляете права read/write для владельца, права read для группы, и снова,
read для всех остальных.
Сконвертируем число 644 из восьмеричной системы в двоичную, это будет \TT{110100100}, или (в группах по 3 бита) \TT{110 100 100}.

Теперь мы видим, что каждая тройка описывает права для владельца/группы/остальных:
первая это \TT{rw-}, вторая это \TT{r--} и третья это \TT{r--}.

Восьмеричная система была также популярная на старых компьютерах вроде PDP-8, потому что слово там могло содержать 12, 24 или
36 бит, и эти числа делятся на 3, так что выбор восьмеричной системы в той среде был логичен.
Сейчас, все популярные компьютеры имеют размер слова/адреса 16, 32 или 64 бита, и эти числа делятся на 4,
так что шестнадцатеричная система здесь удобнее.

Восьмеричная система поддерживается всеми стандартными компиляторами \CCpp{}.
Это иногда источник недоумения, потому что восьмеричные числа кодируются с нулем вперед, например, 0377 это 255.
И иногда, вы можете сделать опечатку, и написать "09" вместо 9, и компилятор выдаст ошибку.
GCC может выдать что-то вроде:\\
\TT{error: invalid digit "9" in octal constant}.

Также, восьмеричная система популярна в Java: когда IDA показывает строку с непечатаемыми символами,
они кодируются в восьмеричной системе вместо шестнадцатеричной.
\myindex{JAD}
Точно также себя ведет декомпилятор с Java JAD.

\subsubsection{Делимость}

Когда вы видите десятичное число вроде 120, вы можете быстро понять что оно делится на 10, потому что последняя цифра это 0.
Точно также, 123400 делится на 100, потому что две последних цифры это нули.

Точно также, шестнадцатеричное число 0x1230 делится на 0x10 (или 16), 0x123000 делится на 0x1000 (или 4096), итд.

Двоичное число 0b1000101000 делится на 0b1000 (8), итд.

Это свойство можно часто использовать, чтобы быстро понять,
что длина какого-либо блока в памяти выровнена по некоторой границе.
Например, секции в \ac{PE}-файлах почти всегда начинаются с адресов заканчивающихся 3 шестнадцатеричными нулями:
0x41000, 0x10001000, итд.
Причина в том, что почти все секции в \ac{PE} выровнены по границе 0x1000 (4096) байт.

\subsubsection{Арифметика произвольной точности и основание}

\index{RSA}
Арифметика произвольной точности (multi-precision arithmetic) может использовать огромные числа,
которые могут храниться в нескольких байтах.
Например, ключи RSA, и открытые и закрытые, могут занимать до 4096 бит и даже больше.

В \InSqBrackets{\TAOCPvolII, 265} можно найти такую идею: когда вы сохраняете число произвольной точности в нескольких байтах,
всё число может быть представлено как имеющую систему счисления по основанию $2^8=256$, и каждая цифра находится
в соответствующем байте.
Точно также, если вы сохраняете число произвольной точности в нескольких 32-битных целочисленных значениях,
каждая цифра отправляется в каждый 32-битный слот, и вы можете считать что это число записано в системе с основанием $2^{32}$.

\subsubsection{Произношение}

Числа в недесятичных системах счислениях обычно произносятся по одной цифре: ``один-ноль-ноль-один-один-...''.
Слова вроде ``десять'', ``тысяча'', итд, обычно не произносятся, потому что тогда можно спутать с десятичной системой.

\subsubsection{Числа с плавающей запятой}

Чтобы отличать числа с плавающей запятой от целочисленных, часто, в конце добавляют ``.0'',
например $0.0$, $123.0$, итд.

}
\ITA{\input{patterns/numeral_ITA}}
\DE{\input{patterns/numeral_DE}}
\FR{\input{patterns/numeral_FR}}
\PL{\input{patterns/numeral_PL}}

% chapters
\ifdefined\SPANISH
\chapter{Patrones de código}
\fi % SPANISH

\ifdefined\GERMAN
\chapter{Code-Muster}
\fi % GERMAN

\ifdefined\ENGLISH
\chapter{Code Patterns}
\fi % ENGLISH

\ifdefined\ITALIAN
\chapter{Forme di codice}
\fi % ITALIAN

\ifdefined\RUSSIAN
\chapter{Образцы кода}
\fi % RUSSIAN

\ifdefined\BRAZILIAN
\chapter{Padrões de códigos}
\fi % BRAZILIAN

\ifdefined\THAI
\chapter{รูปแบบของโค้ด}
\fi % THAI

\ifdefined\FRENCH
\chapter{Modèle de code}
\fi % FRENCH

\ifdefined\POLISH
\chapter{\PLph{}}
\fi % POLISH

% sections
\EN{\input{patterns/patterns_opt_dbg_EN}}
\ES{\input{patterns/patterns_opt_dbg_ES}}
\ITA{\input{patterns/patterns_opt_dbg_ITA}}
\PTBR{\input{patterns/patterns_opt_dbg_PTBR}}
\RU{\input{patterns/patterns_opt_dbg_RU}}
\THA{\input{patterns/patterns_opt_dbg_THA}}
\DE{\input{patterns/patterns_opt_dbg_DE}}
\FR{\input{patterns/patterns_opt_dbg_FR}}
\PL{\input{patterns/patterns_opt_dbg_PL}}

\RU{\section{Некоторые базовые понятия}}
\EN{\section{Some basics}}
\DE{\section{Einige Grundlagen}}
\FR{\section{Quelques bases}}
\ES{\section{\ESph{}}}
\ITA{\section{Alcune basi teoriche}}
\PTBR{\section{\PTBRph{}}}
\THA{\section{\THAph{}}}
\PL{\section{\PLph{}}}

% sections:
\EN{\input{patterns/intro_CPU_ISA_EN}}
\ES{\input{patterns/intro_CPU_ISA_ES}}
\ITA{\input{patterns/intro_CPU_ISA_ITA}}
\PTBR{\input{patterns/intro_CPU_ISA_PTBR}}
\RU{\input{patterns/intro_CPU_ISA_RU}}
\DE{\input{patterns/intro_CPU_ISA_DE}}
\FR{\input{patterns/intro_CPU_ISA_FR}}
\PL{\input{patterns/intro_CPU_ISA_PL}}

\EN{\input{patterns/numeral_EN}}
\RU{\input{patterns/numeral_RU}}
\ITA{\input{patterns/numeral_ITA}}
\DE{\input{patterns/numeral_DE}}
\FR{\input{patterns/numeral_FR}}
\PL{\input{patterns/numeral_PL}}

% chapters
\input{patterns/00_empty/main}
\input{patterns/011_ret/main}
\input{patterns/01_helloworld/main}
\input{patterns/015_prolog_epilogue/main}
\input{patterns/02_stack/main}
\input{patterns/03_printf/main}
\input{patterns/04_scanf/main}
\input{patterns/05_passing_arguments/main}
\input{patterns/06_return_results/main}
\input{patterns/061_pointers/main}
\input{patterns/065_GOTO/main}
\input{patterns/07_jcc/main}
\input{patterns/08_switch/main}
\input{patterns/09_loops/main}
\input{patterns/10_strings/main}
\input{patterns/11_arith_optimizations/main}
\input{patterns/12_FPU/main}
\input{patterns/13_arrays/main}
\input{patterns/14_bitfields/main}
\EN{\input{patterns/145_LCG/main_EN}}
\RU{\input{patterns/145_LCG/main_RU}}
\input{patterns/15_structs/main}
\input{patterns/17_unions/main}
\input{patterns/18_pointers_to_functions/main}
\input{patterns/185_64bit_in_32_env/main}

\EN{\input{patterns/19_SIMD/main_EN}}
\RU{\input{patterns/19_SIMD/main_RU}}
\DE{\input{patterns/19_SIMD/main_DE}}

\EN{\input{patterns/20_x64/main_EN}}
\RU{\input{patterns/20_x64/main_RU}}

\EN{\input{patterns/205_floating_SIMD/main_EN}}
\RU{\input{patterns/205_floating_SIMD/main_RU}}
\DE{\input{patterns/205_floating_SIMD/main_DE}}

\EN{\input{patterns/ARM/main_EN}}
\RU{\input{patterns/ARM/main_RU}}
\DE{\input{patterns/ARM/main_DE}}

\input{patterns/MIPS/main}

\ifdefined\SPANISH
\chapter{Patrones de código}
\fi % SPANISH

\ifdefined\GERMAN
\chapter{Code-Muster}
\fi % GERMAN

\ifdefined\ENGLISH
\chapter{Code Patterns}
\fi % ENGLISH

\ifdefined\ITALIAN
\chapter{Forme di codice}
\fi % ITALIAN

\ifdefined\RUSSIAN
\chapter{Образцы кода}
\fi % RUSSIAN

\ifdefined\BRAZILIAN
\chapter{Padrões de códigos}
\fi % BRAZILIAN

\ifdefined\THAI
\chapter{รูปแบบของโค้ด}
\fi % THAI

\ifdefined\FRENCH
\chapter{Modèle de code}
\fi % FRENCH

\ifdefined\POLISH
\chapter{\PLph{}}
\fi % POLISH

% sections
\EN{\input{patterns/patterns_opt_dbg_EN}}
\ES{\input{patterns/patterns_opt_dbg_ES}}
\ITA{\input{patterns/patterns_opt_dbg_ITA}}
\PTBR{\input{patterns/patterns_opt_dbg_PTBR}}
\RU{\input{patterns/patterns_opt_dbg_RU}}
\THA{\input{patterns/patterns_opt_dbg_THA}}
\DE{\input{patterns/patterns_opt_dbg_DE}}
\FR{\input{patterns/patterns_opt_dbg_FR}}
\PL{\input{patterns/patterns_opt_dbg_PL}}

\RU{\section{Некоторые базовые понятия}}
\EN{\section{Some basics}}
\DE{\section{Einige Grundlagen}}
\FR{\section{Quelques bases}}
\ES{\section{\ESph{}}}
\ITA{\section{Alcune basi teoriche}}
\PTBR{\section{\PTBRph{}}}
\THA{\section{\THAph{}}}
\PL{\section{\PLph{}}}

% sections:
\EN{\input{patterns/intro_CPU_ISA_EN}}
\ES{\input{patterns/intro_CPU_ISA_ES}}
\ITA{\input{patterns/intro_CPU_ISA_ITA}}
\PTBR{\input{patterns/intro_CPU_ISA_PTBR}}
\RU{\input{patterns/intro_CPU_ISA_RU}}
\DE{\input{patterns/intro_CPU_ISA_DE}}
\FR{\input{patterns/intro_CPU_ISA_FR}}
\PL{\input{patterns/intro_CPU_ISA_PL}}

\EN{\input{patterns/numeral_EN}}
\RU{\input{patterns/numeral_RU}}
\ITA{\input{patterns/numeral_ITA}}
\DE{\input{patterns/numeral_DE}}
\FR{\input{patterns/numeral_FR}}
\PL{\input{patterns/numeral_PL}}

% chapters
\input{patterns/00_empty/main}
\input{patterns/011_ret/main}
\input{patterns/01_helloworld/main}
\input{patterns/015_prolog_epilogue/main}
\input{patterns/02_stack/main}
\input{patterns/03_printf/main}
\input{patterns/04_scanf/main}
\input{patterns/05_passing_arguments/main}
\input{patterns/06_return_results/main}
\input{patterns/061_pointers/main}
\input{patterns/065_GOTO/main}
\input{patterns/07_jcc/main}
\input{patterns/08_switch/main}
\input{patterns/09_loops/main}
\input{patterns/10_strings/main}
\input{patterns/11_arith_optimizations/main}
\input{patterns/12_FPU/main}
\input{patterns/13_arrays/main}
\input{patterns/14_bitfields/main}
\EN{\input{patterns/145_LCG/main_EN}}
\RU{\input{patterns/145_LCG/main_RU}}
\input{patterns/15_structs/main}
\input{patterns/17_unions/main}
\input{patterns/18_pointers_to_functions/main}
\input{patterns/185_64bit_in_32_env/main}

\EN{\input{patterns/19_SIMD/main_EN}}
\RU{\input{patterns/19_SIMD/main_RU}}
\DE{\input{patterns/19_SIMD/main_DE}}

\EN{\input{patterns/20_x64/main_EN}}
\RU{\input{patterns/20_x64/main_RU}}

\EN{\input{patterns/205_floating_SIMD/main_EN}}
\RU{\input{patterns/205_floating_SIMD/main_RU}}
\DE{\input{patterns/205_floating_SIMD/main_DE}}

\EN{\input{patterns/ARM/main_EN}}
\RU{\input{patterns/ARM/main_RU}}
\DE{\input{patterns/ARM/main_DE}}

\input{patterns/MIPS/main}

\ifdefined\SPANISH
\chapter{Patrones de código}
\fi % SPANISH

\ifdefined\GERMAN
\chapter{Code-Muster}
\fi % GERMAN

\ifdefined\ENGLISH
\chapter{Code Patterns}
\fi % ENGLISH

\ifdefined\ITALIAN
\chapter{Forme di codice}
\fi % ITALIAN

\ifdefined\RUSSIAN
\chapter{Образцы кода}
\fi % RUSSIAN

\ifdefined\BRAZILIAN
\chapter{Padrões de códigos}
\fi % BRAZILIAN

\ifdefined\THAI
\chapter{รูปแบบของโค้ด}
\fi % THAI

\ifdefined\FRENCH
\chapter{Modèle de code}
\fi % FRENCH

\ifdefined\POLISH
\chapter{\PLph{}}
\fi % POLISH

% sections
\EN{\input{patterns/patterns_opt_dbg_EN}}
\ES{\input{patterns/patterns_opt_dbg_ES}}
\ITA{\input{patterns/patterns_opt_dbg_ITA}}
\PTBR{\input{patterns/patterns_opt_dbg_PTBR}}
\RU{\input{patterns/patterns_opt_dbg_RU}}
\THA{\input{patterns/patterns_opt_dbg_THA}}
\DE{\input{patterns/patterns_opt_dbg_DE}}
\FR{\input{patterns/patterns_opt_dbg_FR}}
\PL{\input{patterns/patterns_opt_dbg_PL}}

\RU{\section{Некоторые базовые понятия}}
\EN{\section{Some basics}}
\DE{\section{Einige Grundlagen}}
\FR{\section{Quelques bases}}
\ES{\section{\ESph{}}}
\ITA{\section{Alcune basi teoriche}}
\PTBR{\section{\PTBRph{}}}
\THA{\section{\THAph{}}}
\PL{\section{\PLph{}}}

% sections:
\EN{\input{patterns/intro_CPU_ISA_EN}}
\ES{\input{patterns/intro_CPU_ISA_ES}}
\ITA{\input{patterns/intro_CPU_ISA_ITA}}
\PTBR{\input{patterns/intro_CPU_ISA_PTBR}}
\RU{\input{patterns/intro_CPU_ISA_RU}}
\DE{\input{patterns/intro_CPU_ISA_DE}}
\FR{\input{patterns/intro_CPU_ISA_FR}}
\PL{\input{patterns/intro_CPU_ISA_PL}}

\EN{\input{patterns/numeral_EN}}
\RU{\input{patterns/numeral_RU}}
\ITA{\input{patterns/numeral_ITA}}
\DE{\input{patterns/numeral_DE}}
\FR{\input{patterns/numeral_FR}}
\PL{\input{patterns/numeral_PL}}

% chapters
\input{patterns/00_empty/main}
\input{patterns/011_ret/main}
\input{patterns/01_helloworld/main}
\input{patterns/015_prolog_epilogue/main}
\input{patterns/02_stack/main}
\input{patterns/03_printf/main}
\input{patterns/04_scanf/main}
\input{patterns/05_passing_arguments/main}
\input{patterns/06_return_results/main}
\input{patterns/061_pointers/main}
\input{patterns/065_GOTO/main}
\input{patterns/07_jcc/main}
\input{patterns/08_switch/main}
\input{patterns/09_loops/main}
\input{patterns/10_strings/main}
\input{patterns/11_arith_optimizations/main}
\input{patterns/12_FPU/main}
\input{patterns/13_arrays/main}
\input{patterns/14_bitfields/main}
\EN{\input{patterns/145_LCG/main_EN}}
\RU{\input{patterns/145_LCG/main_RU}}
\input{patterns/15_structs/main}
\input{patterns/17_unions/main}
\input{patterns/18_pointers_to_functions/main}
\input{patterns/185_64bit_in_32_env/main}

\EN{\input{patterns/19_SIMD/main_EN}}
\RU{\input{patterns/19_SIMD/main_RU}}
\DE{\input{patterns/19_SIMD/main_DE}}

\EN{\input{patterns/20_x64/main_EN}}
\RU{\input{patterns/20_x64/main_RU}}

\EN{\input{patterns/205_floating_SIMD/main_EN}}
\RU{\input{patterns/205_floating_SIMD/main_RU}}
\DE{\input{patterns/205_floating_SIMD/main_DE}}

\EN{\input{patterns/ARM/main_EN}}
\RU{\input{patterns/ARM/main_RU}}
\DE{\input{patterns/ARM/main_DE}}

\input{patterns/MIPS/main}

\ifdefined\SPANISH
\chapter{Patrones de código}
\fi % SPANISH

\ifdefined\GERMAN
\chapter{Code-Muster}
\fi % GERMAN

\ifdefined\ENGLISH
\chapter{Code Patterns}
\fi % ENGLISH

\ifdefined\ITALIAN
\chapter{Forme di codice}
\fi % ITALIAN

\ifdefined\RUSSIAN
\chapter{Образцы кода}
\fi % RUSSIAN

\ifdefined\BRAZILIAN
\chapter{Padrões de códigos}
\fi % BRAZILIAN

\ifdefined\THAI
\chapter{รูปแบบของโค้ด}
\fi % THAI

\ifdefined\FRENCH
\chapter{Modèle de code}
\fi % FRENCH

\ifdefined\POLISH
\chapter{\PLph{}}
\fi % POLISH

% sections
\EN{\input{patterns/patterns_opt_dbg_EN}}
\ES{\input{patterns/patterns_opt_dbg_ES}}
\ITA{\input{patterns/patterns_opt_dbg_ITA}}
\PTBR{\input{patterns/patterns_opt_dbg_PTBR}}
\RU{\input{patterns/patterns_opt_dbg_RU}}
\THA{\input{patterns/patterns_opt_dbg_THA}}
\DE{\input{patterns/patterns_opt_dbg_DE}}
\FR{\input{patterns/patterns_opt_dbg_FR}}
\PL{\input{patterns/patterns_opt_dbg_PL}}

\RU{\section{Некоторые базовые понятия}}
\EN{\section{Some basics}}
\DE{\section{Einige Grundlagen}}
\FR{\section{Quelques bases}}
\ES{\section{\ESph{}}}
\ITA{\section{Alcune basi teoriche}}
\PTBR{\section{\PTBRph{}}}
\THA{\section{\THAph{}}}
\PL{\section{\PLph{}}}

% sections:
\EN{\input{patterns/intro_CPU_ISA_EN}}
\ES{\input{patterns/intro_CPU_ISA_ES}}
\ITA{\input{patterns/intro_CPU_ISA_ITA}}
\PTBR{\input{patterns/intro_CPU_ISA_PTBR}}
\RU{\input{patterns/intro_CPU_ISA_RU}}
\DE{\input{patterns/intro_CPU_ISA_DE}}
\FR{\input{patterns/intro_CPU_ISA_FR}}
\PL{\input{patterns/intro_CPU_ISA_PL}}

\EN{\input{patterns/numeral_EN}}
\RU{\input{patterns/numeral_RU}}
\ITA{\input{patterns/numeral_ITA}}
\DE{\input{patterns/numeral_DE}}
\FR{\input{patterns/numeral_FR}}
\PL{\input{patterns/numeral_PL}}

% chapters
\input{patterns/00_empty/main}
\input{patterns/011_ret/main}
\input{patterns/01_helloworld/main}
\input{patterns/015_prolog_epilogue/main}
\input{patterns/02_stack/main}
\input{patterns/03_printf/main}
\input{patterns/04_scanf/main}
\input{patterns/05_passing_arguments/main}
\input{patterns/06_return_results/main}
\input{patterns/061_pointers/main}
\input{patterns/065_GOTO/main}
\input{patterns/07_jcc/main}
\input{patterns/08_switch/main}
\input{patterns/09_loops/main}
\input{patterns/10_strings/main}
\input{patterns/11_arith_optimizations/main}
\input{patterns/12_FPU/main}
\input{patterns/13_arrays/main}
\input{patterns/14_bitfields/main}
\EN{\input{patterns/145_LCG/main_EN}}
\RU{\input{patterns/145_LCG/main_RU}}
\input{patterns/15_structs/main}
\input{patterns/17_unions/main}
\input{patterns/18_pointers_to_functions/main}
\input{patterns/185_64bit_in_32_env/main}

\EN{\input{patterns/19_SIMD/main_EN}}
\RU{\input{patterns/19_SIMD/main_RU}}
\DE{\input{patterns/19_SIMD/main_DE}}

\EN{\input{patterns/20_x64/main_EN}}
\RU{\input{patterns/20_x64/main_RU}}

\EN{\input{patterns/205_floating_SIMD/main_EN}}
\RU{\input{patterns/205_floating_SIMD/main_RU}}
\DE{\input{patterns/205_floating_SIMD/main_DE}}

\EN{\input{patterns/ARM/main_EN}}
\RU{\input{patterns/ARM/main_RU}}
\DE{\input{patterns/ARM/main_DE}}

\input{patterns/MIPS/main}

\ifdefined\SPANISH
\chapter{Patrones de código}
\fi % SPANISH

\ifdefined\GERMAN
\chapter{Code-Muster}
\fi % GERMAN

\ifdefined\ENGLISH
\chapter{Code Patterns}
\fi % ENGLISH

\ifdefined\ITALIAN
\chapter{Forme di codice}
\fi % ITALIAN

\ifdefined\RUSSIAN
\chapter{Образцы кода}
\fi % RUSSIAN

\ifdefined\BRAZILIAN
\chapter{Padrões de códigos}
\fi % BRAZILIAN

\ifdefined\THAI
\chapter{รูปแบบของโค้ด}
\fi % THAI

\ifdefined\FRENCH
\chapter{Modèle de code}
\fi % FRENCH

\ifdefined\POLISH
\chapter{\PLph{}}
\fi % POLISH

% sections
\EN{\input{patterns/patterns_opt_dbg_EN}}
\ES{\input{patterns/patterns_opt_dbg_ES}}
\ITA{\input{patterns/patterns_opt_dbg_ITA}}
\PTBR{\input{patterns/patterns_opt_dbg_PTBR}}
\RU{\input{patterns/patterns_opt_dbg_RU}}
\THA{\input{patterns/patterns_opt_dbg_THA}}
\DE{\input{patterns/patterns_opt_dbg_DE}}
\FR{\input{patterns/patterns_opt_dbg_FR}}
\PL{\input{patterns/patterns_opt_dbg_PL}}

\RU{\section{Некоторые базовые понятия}}
\EN{\section{Some basics}}
\DE{\section{Einige Grundlagen}}
\FR{\section{Quelques bases}}
\ES{\section{\ESph{}}}
\ITA{\section{Alcune basi teoriche}}
\PTBR{\section{\PTBRph{}}}
\THA{\section{\THAph{}}}
\PL{\section{\PLph{}}}

% sections:
\EN{\input{patterns/intro_CPU_ISA_EN}}
\ES{\input{patterns/intro_CPU_ISA_ES}}
\ITA{\input{patterns/intro_CPU_ISA_ITA}}
\PTBR{\input{patterns/intro_CPU_ISA_PTBR}}
\RU{\input{patterns/intro_CPU_ISA_RU}}
\DE{\input{patterns/intro_CPU_ISA_DE}}
\FR{\input{patterns/intro_CPU_ISA_FR}}
\PL{\input{patterns/intro_CPU_ISA_PL}}

\EN{\input{patterns/numeral_EN}}
\RU{\input{patterns/numeral_RU}}
\ITA{\input{patterns/numeral_ITA}}
\DE{\input{patterns/numeral_DE}}
\FR{\input{patterns/numeral_FR}}
\PL{\input{patterns/numeral_PL}}

% chapters
\input{patterns/00_empty/main}
\input{patterns/011_ret/main}
\input{patterns/01_helloworld/main}
\input{patterns/015_prolog_epilogue/main}
\input{patterns/02_stack/main}
\input{patterns/03_printf/main}
\input{patterns/04_scanf/main}
\input{patterns/05_passing_arguments/main}
\input{patterns/06_return_results/main}
\input{patterns/061_pointers/main}
\input{patterns/065_GOTO/main}
\input{patterns/07_jcc/main}
\input{patterns/08_switch/main}
\input{patterns/09_loops/main}
\input{patterns/10_strings/main}
\input{patterns/11_arith_optimizations/main}
\input{patterns/12_FPU/main}
\input{patterns/13_arrays/main}
\input{patterns/14_bitfields/main}
\EN{\input{patterns/145_LCG/main_EN}}
\RU{\input{patterns/145_LCG/main_RU}}
\input{patterns/15_structs/main}
\input{patterns/17_unions/main}
\input{patterns/18_pointers_to_functions/main}
\input{patterns/185_64bit_in_32_env/main}

\EN{\input{patterns/19_SIMD/main_EN}}
\RU{\input{patterns/19_SIMD/main_RU}}
\DE{\input{patterns/19_SIMD/main_DE}}

\EN{\input{patterns/20_x64/main_EN}}
\RU{\input{patterns/20_x64/main_RU}}

\EN{\input{patterns/205_floating_SIMD/main_EN}}
\RU{\input{patterns/205_floating_SIMD/main_RU}}
\DE{\input{patterns/205_floating_SIMD/main_DE}}

\EN{\input{patterns/ARM/main_EN}}
\RU{\input{patterns/ARM/main_RU}}
\DE{\input{patterns/ARM/main_DE}}

\input{patterns/MIPS/main}

\ifdefined\SPANISH
\chapter{Patrones de código}
\fi % SPANISH

\ifdefined\GERMAN
\chapter{Code-Muster}
\fi % GERMAN

\ifdefined\ENGLISH
\chapter{Code Patterns}
\fi % ENGLISH

\ifdefined\ITALIAN
\chapter{Forme di codice}
\fi % ITALIAN

\ifdefined\RUSSIAN
\chapter{Образцы кода}
\fi % RUSSIAN

\ifdefined\BRAZILIAN
\chapter{Padrões de códigos}
\fi % BRAZILIAN

\ifdefined\THAI
\chapter{รูปแบบของโค้ด}
\fi % THAI

\ifdefined\FRENCH
\chapter{Modèle de code}
\fi % FRENCH

\ifdefined\POLISH
\chapter{\PLph{}}
\fi % POLISH

% sections
\EN{\input{patterns/patterns_opt_dbg_EN}}
\ES{\input{patterns/patterns_opt_dbg_ES}}
\ITA{\input{patterns/patterns_opt_dbg_ITA}}
\PTBR{\input{patterns/patterns_opt_dbg_PTBR}}
\RU{\input{patterns/patterns_opt_dbg_RU}}
\THA{\input{patterns/patterns_opt_dbg_THA}}
\DE{\input{patterns/patterns_opt_dbg_DE}}
\FR{\input{patterns/patterns_opt_dbg_FR}}
\PL{\input{patterns/patterns_opt_dbg_PL}}

\RU{\section{Некоторые базовые понятия}}
\EN{\section{Some basics}}
\DE{\section{Einige Grundlagen}}
\FR{\section{Quelques bases}}
\ES{\section{\ESph{}}}
\ITA{\section{Alcune basi teoriche}}
\PTBR{\section{\PTBRph{}}}
\THA{\section{\THAph{}}}
\PL{\section{\PLph{}}}

% sections:
\EN{\input{patterns/intro_CPU_ISA_EN}}
\ES{\input{patterns/intro_CPU_ISA_ES}}
\ITA{\input{patterns/intro_CPU_ISA_ITA}}
\PTBR{\input{patterns/intro_CPU_ISA_PTBR}}
\RU{\input{patterns/intro_CPU_ISA_RU}}
\DE{\input{patterns/intro_CPU_ISA_DE}}
\FR{\input{patterns/intro_CPU_ISA_FR}}
\PL{\input{patterns/intro_CPU_ISA_PL}}

\EN{\input{patterns/numeral_EN}}
\RU{\input{patterns/numeral_RU}}
\ITA{\input{patterns/numeral_ITA}}
\DE{\input{patterns/numeral_DE}}
\FR{\input{patterns/numeral_FR}}
\PL{\input{patterns/numeral_PL}}

% chapters
\input{patterns/00_empty/main}
\input{patterns/011_ret/main}
\input{patterns/01_helloworld/main}
\input{patterns/015_prolog_epilogue/main}
\input{patterns/02_stack/main}
\input{patterns/03_printf/main}
\input{patterns/04_scanf/main}
\input{patterns/05_passing_arguments/main}
\input{patterns/06_return_results/main}
\input{patterns/061_pointers/main}
\input{patterns/065_GOTO/main}
\input{patterns/07_jcc/main}
\input{patterns/08_switch/main}
\input{patterns/09_loops/main}
\input{patterns/10_strings/main}
\input{patterns/11_arith_optimizations/main}
\input{patterns/12_FPU/main}
\input{patterns/13_arrays/main}
\input{patterns/14_bitfields/main}
\EN{\input{patterns/145_LCG/main_EN}}
\RU{\input{patterns/145_LCG/main_RU}}
\input{patterns/15_structs/main}
\input{patterns/17_unions/main}
\input{patterns/18_pointers_to_functions/main}
\input{patterns/185_64bit_in_32_env/main}

\EN{\input{patterns/19_SIMD/main_EN}}
\RU{\input{patterns/19_SIMD/main_RU}}
\DE{\input{patterns/19_SIMD/main_DE}}

\EN{\input{patterns/20_x64/main_EN}}
\RU{\input{patterns/20_x64/main_RU}}

\EN{\input{patterns/205_floating_SIMD/main_EN}}
\RU{\input{patterns/205_floating_SIMD/main_RU}}
\DE{\input{patterns/205_floating_SIMD/main_DE}}

\EN{\input{patterns/ARM/main_EN}}
\RU{\input{patterns/ARM/main_RU}}
\DE{\input{patterns/ARM/main_DE}}

\input{patterns/MIPS/main}

\ifdefined\SPANISH
\chapter{Patrones de código}
\fi % SPANISH

\ifdefined\GERMAN
\chapter{Code-Muster}
\fi % GERMAN

\ifdefined\ENGLISH
\chapter{Code Patterns}
\fi % ENGLISH

\ifdefined\ITALIAN
\chapter{Forme di codice}
\fi % ITALIAN

\ifdefined\RUSSIAN
\chapter{Образцы кода}
\fi % RUSSIAN

\ifdefined\BRAZILIAN
\chapter{Padrões de códigos}
\fi % BRAZILIAN

\ifdefined\THAI
\chapter{รูปแบบของโค้ด}
\fi % THAI

\ifdefined\FRENCH
\chapter{Modèle de code}
\fi % FRENCH

\ifdefined\POLISH
\chapter{\PLph{}}
\fi % POLISH

% sections
\EN{\input{patterns/patterns_opt_dbg_EN}}
\ES{\input{patterns/patterns_opt_dbg_ES}}
\ITA{\input{patterns/patterns_opt_dbg_ITA}}
\PTBR{\input{patterns/patterns_opt_dbg_PTBR}}
\RU{\input{patterns/patterns_opt_dbg_RU}}
\THA{\input{patterns/patterns_opt_dbg_THA}}
\DE{\input{patterns/patterns_opt_dbg_DE}}
\FR{\input{patterns/patterns_opt_dbg_FR}}
\PL{\input{patterns/patterns_opt_dbg_PL}}

\RU{\section{Некоторые базовые понятия}}
\EN{\section{Some basics}}
\DE{\section{Einige Grundlagen}}
\FR{\section{Quelques bases}}
\ES{\section{\ESph{}}}
\ITA{\section{Alcune basi teoriche}}
\PTBR{\section{\PTBRph{}}}
\THA{\section{\THAph{}}}
\PL{\section{\PLph{}}}

% sections:
\EN{\input{patterns/intro_CPU_ISA_EN}}
\ES{\input{patterns/intro_CPU_ISA_ES}}
\ITA{\input{patterns/intro_CPU_ISA_ITA}}
\PTBR{\input{patterns/intro_CPU_ISA_PTBR}}
\RU{\input{patterns/intro_CPU_ISA_RU}}
\DE{\input{patterns/intro_CPU_ISA_DE}}
\FR{\input{patterns/intro_CPU_ISA_FR}}
\PL{\input{patterns/intro_CPU_ISA_PL}}

\EN{\input{patterns/numeral_EN}}
\RU{\input{patterns/numeral_RU}}
\ITA{\input{patterns/numeral_ITA}}
\DE{\input{patterns/numeral_DE}}
\FR{\input{patterns/numeral_FR}}
\PL{\input{patterns/numeral_PL}}

% chapters
\input{patterns/00_empty/main}
\input{patterns/011_ret/main}
\input{patterns/01_helloworld/main}
\input{patterns/015_prolog_epilogue/main}
\input{patterns/02_stack/main}
\input{patterns/03_printf/main}
\input{patterns/04_scanf/main}
\input{patterns/05_passing_arguments/main}
\input{patterns/06_return_results/main}
\input{patterns/061_pointers/main}
\input{patterns/065_GOTO/main}
\input{patterns/07_jcc/main}
\input{patterns/08_switch/main}
\input{patterns/09_loops/main}
\input{patterns/10_strings/main}
\input{patterns/11_arith_optimizations/main}
\input{patterns/12_FPU/main}
\input{patterns/13_arrays/main}
\input{patterns/14_bitfields/main}
\EN{\input{patterns/145_LCG/main_EN}}
\RU{\input{patterns/145_LCG/main_RU}}
\input{patterns/15_structs/main}
\input{patterns/17_unions/main}
\input{patterns/18_pointers_to_functions/main}
\input{patterns/185_64bit_in_32_env/main}

\EN{\input{patterns/19_SIMD/main_EN}}
\RU{\input{patterns/19_SIMD/main_RU}}
\DE{\input{patterns/19_SIMD/main_DE}}

\EN{\input{patterns/20_x64/main_EN}}
\RU{\input{patterns/20_x64/main_RU}}

\EN{\input{patterns/205_floating_SIMD/main_EN}}
\RU{\input{patterns/205_floating_SIMD/main_RU}}
\DE{\input{patterns/205_floating_SIMD/main_DE}}

\EN{\input{patterns/ARM/main_EN}}
\RU{\input{patterns/ARM/main_RU}}
\DE{\input{patterns/ARM/main_DE}}

\input{patterns/MIPS/main}

\ifdefined\SPANISH
\chapter{Patrones de código}
\fi % SPANISH

\ifdefined\GERMAN
\chapter{Code-Muster}
\fi % GERMAN

\ifdefined\ENGLISH
\chapter{Code Patterns}
\fi % ENGLISH

\ifdefined\ITALIAN
\chapter{Forme di codice}
\fi % ITALIAN

\ifdefined\RUSSIAN
\chapter{Образцы кода}
\fi % RUSSIAN

\ifdefined\BRAZILIAN
\chapter{Padrões de códigos}
\fi % BRAZILIAN

\ifdefined\THAI
\chapter{รูปแบบของโค้ด}
\fi % THAI

\ifdefined\FRENCH
\chapter{Modèle de code}
\fi % FRENCH

\ifdefined\POLISH
\chapter{\PLph{}}
\fi % POLISH

% sections
\EN{\input{patterns/patterns_opt_dbg_EN}}
\ES{\input{patterns/patterns_opt_dbg_ES}}
\ITA{\input{patterns/patterns_opt_dbg_ITA}}
\PTBR{\input{patterns/patterns_opt_dbg_PTBR}}
\RU{\input{patterns/patterns_opt_dbg_RU}}
\THA{\input{patterns/patterns_opt_dbg_THA}}
\DE{\input{patterns/patterns_opt_dbg_DE}}
\FR{\input{patterns/patterns_opt_dbg_FR}}
\PL{\input{patterns/patterns_opt_dbg_PL}}

\RU{\section{Некоторые базовые понятия}}
\EN{\section{Some basics}}
\DE{\section{Einige Grundlagen}}
\FR{\section{Quelques bases}}
\ES{\section{\ESph{}}}
\ITA{\section{Alcune basi teoriche}}
\PTBR{\section{\PTBRph{}}}
\THA{\section{\THAph{}}}
\PL{\section{\PLph{}}}

% sections:
\EN{\input{patterns/intro_CPU_ISA_EN}}
\ES{\input{patterns/intro_CPU_ISA_ES}}
\ITA{\input{patterns/intro_CPU_ISA_ITA}}
\PTBR{\input{patterns/intro_CPU_ISA_PTBR}}
\RU{\input{patterns/intro_CPU_ISA_RU}}
\DE{\input{patterns/intro_CPU_ISA_DE}}
\FR{\input{patterns/intro_CPU_ISA_FR}}
\PL{\input{patterns/intro_CPU_ISA_PL}}

\EN{\input{patterns/numeral_EN}}
\RU{\input{patterns/numeral_RU}}
\ITA{\input{patterns/numeral_ITA}}
\DE{\input{patterns/numeral_DE}}
\FR{\input{patterns/numeral_FR}}
\PL{\input{patterns/numeral_PL}}

% chapters
\input{patterns/00_empty/main}
\input{patterns/011_ret/main}
\input{patterns/01_helloworld/main}
\input{patterns/015_prolog_epilogue/main}
\input{patterns/02_stack/main}
\input{patterns/03_printf/main}
\input{patterns/04_scanf/main}
\input{patterns/05_passing_arguments/main}
\input{patterns/06_return_results/main}
\input{patterns/061_pointers/main}
\input{patterns/065_GOTO/main}
\input{patterns/07_jcc/main}
\input{patterns/08_switch/main}
\input{patterns/09_loops/main}
\input{patterns/10_strings/main}
\input{patterns/11_arith_optimizations/main}
\input{patterns/12_FPU/main}
\input{patterns/13_arrays/main}
\input{patterns/14_bitfields/main}
\EN{\input{patterns/145_LCG/main_EN}}
\RU{\input{patterns/145_LCG/main_RU}}
\input{patterns/15_structs/main}
\input{patterns/17_unions/main}
\input{patterns/18_pointers_to_functions/main}
\input{patterns/185_64bit_in_32_env/main}

\EN{\input{patterns/19_SIMD/main_EN}}
\RU{\input{patterns/19_SIMD/main_RU}}
\DE{\input{patterns/19_SIMD/main_DE}}

\EN{\input{patterns/20_x64/main_EN}}
\RU{\input{patterns/20_x64/main_RU}}

\EN{\input{patterns/205_floating_SIMD/main_EN}}
\RU{\input{patterns/205_floating_SIMD/main_RU}}
\DE{\input{patterns/205_floating_SIMD/main_DE}}

\EN{\input{patterns/ARM/main_EN}}
\RU{\input{patterns/ARM/main_RU}}
\DE{\input{patterns/ARM/main_DE}}

\input{patterns/MIPS/main}

\ifdefined\SPANISH
\chapter{Patrones de código}
\fi % SPANISH

\ifdefined\GERMAN
\chapter{Code-Muster}
\fi % GERMAN

\ifdefined\ENGLISH
\chapter{Code Patterns}
\fi % ENGLISH

\ifdefined\ITALIAN
\chapter{Forme di codice}
\fi % ITALIAN

\ifdefined\RUSSIAN
\chapter{Образцы кода}
\fi % RUSSIAN

\ifdefined\BRAZILIAN
\chapter{Padrões de códigos}
\fi % BRAZILIAN

\ifdefined\THAI
\chapter{รูปแบบของโค้ด}
\fi % THAI

\ifdefined\FRENCH
\chapter{Modèle de code}
\fi % FRENCH

\ifdefined\POLISH
\chapter{\PLph{}}
\fi % POLISH

% sections
\EN{\input{patterns/patterns_opt_dbg_EN}}
\ES{\input{patterns/patterns_opt_dbg_ES}}
\ITA{\input{patterns/patterns_opt_dbg_ITA}}
\PTBR{\input{patterns/patterns_opt_dbg_PTBR}}
\RU{\input{patterns/patterns_opt_dbg_RU}}
\THA{\input{patterns/patterns_opt_dbg_THA}}
\DE{\input{patterns/patterns_opt_dbg_DE}}
\FR{\input{patterns/patterns_opt_dbg_FR}}
\PL{\input{patterns/patterns_opt_dbg_PL}}

\RU{\section{Некоторые базовые понятия}}
\EN{\section{Some basics}}
\DE{\section{Einige Grundlagen}}
\FR{\section{Quelques bases}}
\ES{\section{\ESph{}}}
\ITA{\section{Alcune basi teoriche}}
\PTBR{\section{\PTBRph{}}}
\THA{\section{\THAph{}}}
\PL{\section{\PLph{}}}

% sections:
\EN{\input{patterns/intro_CPU_ISA_EN}}
\ES{\input{patterns/intro_CPU_ISA_ES}}
\ITA{\input{patterns/intro_CPU_ISA_ITA}}
\PTBR{\input{patterns/intro_CPU_ISA_PTBR}}
\RU{\input{patterns/intro_CPU_ISA_RU}}
\DE{\input{patterns/intro_CPU_ISA_DE}}
\FR{\input{patterns/intro_CPU_ISA_FR}}
\PL{\input{patterns/intro_CPU_ISA_PL}}

\EN{\input{patterns/numeral_EN}}
\RU{\input{patterns/numeral_RU}}
\ITA{\input{patterns/numeral_ITA}}
\DE{\input{patterns/numeral_DE}}
\FR{\input{patterns/numeral_FR}}
\PL{\input{patterns/numeral_PL}}

% chapters
\input{patterns/00_empty/main}
\input{patterns/011_ret/main}
\input{patterns/01_helloworld/main}
\input{patterns/015_prolog_epilogue/main}
\input{patterns/02_stack/main}
\input{patterns/03_printf/main}
\input{patterns/04_scanf/main}
\input{patterns/05_passing_arguments/main}
\input{patterns/06_return_results/main}
\input{patterns/061_pointers/main}
\input{patterns/065_GOTO/main}
\input{patterns/07_jcc/main}
\input{patterns/08_switch/main}
\input{patterns/09_loops/main}
\input{patterns/10_strings/main}
\input{patterns/11_arith_optimizations/main}
\input{patterns/12_FPU/main}
\input{patterns/13_arrays/main}
\input{patterns/14_bitfields/main}
\EN{\input{patterns/145_LCG/main_EN}}
\RU{\input{patterns/145_LCG/main_RU}}
\input{patterns/15_structs/main}
\input{patterns/17_unions/main}
\input{patterns/18_pointers_to_functions/main}
\input{patterns/185_64bit_in_32_env/main}

\EN{\input{patterns/19_SIMD/main_EN}}
\RU{\input{patterns/19_SIMD/main_RU}}
\DE{\input{patterns/19_SIMD/main_DE}}

\EN{\input{patterns/20_x64/main_EN}}
\RU{\input{patterns/20_x64/main_RU}}

\EN{\input{patterns/205_floating_SIMD/main_EN}}
\RU{\input{patterns/205_floating_SIMD/main_RU}}
\DE{\input{patterns/205_floating_SIMD/main_DE}}

\EN{\input{patterns/ARM/main_EN}}
\RU{\input{patterns/ARM/main_RU}}
\DE{\input{patterns/ARM/main_DE}}

\input{patterns/MIPS/main}

\ifdefined\SPANISH
\chapter{Patrones de código}
\fi % SPANISH

\ifdefined\GERMAN
\chapter{Code-Muster}
\fi % GERMAN

\ifdefined\ENGLISH
\chapter{Code Patterns}
\fi % ENGLISH

\ifdefined\ITALIAN
\chapter{Forme di codice}
\fi % ITALIAN

\ifdefined\RUSSIAN
\chapter{Образцы кода}
\fi % RUSSIAN

\ifdefined\BRAZILIAN
\chapter{Padrões de códigos}
\fi % BRAZILIAN

\ifdefined\THAI
\chapter{รูปแบบของโค้ด}
\fi % THAI

\ifdefined\FRENCH
\chapter{Modèle de code}
\fi % FRENCH

\ifdefined\POLISH
\chapter{\PLph{}}
\fi % POLISH

% sections
\EN{\input{patterns/patterns_opt_dbg_EN}}
\ES{\input{patterns/patterns_opt_dbg_ES}}
\ITA{\input{patterns/patterns_opt_dbg_ITA}}
\PTBR{\input{patterns/patterns_opt_dbg_PTBR}}
\RU{\input{patterns/patterns_opt_dbg_RU}}
\THA{\input{patterns/patterns_opt_dbg_THA}}
\DE{\input{patterns/patterns_opt_dbg_DE}}
\FR{\input{patterns/patterns_opt_dbg_FR}}
\PL{\input{patterns/patterns_opt_dbg_PL}}

\RU{\section{Некоторые базовые понятия}}
\EN{\section{Some basics}}
\DE{\section{Einige Grundlagen}}
\FR{\section{Quelques bases}}
\ES{\section{\ESph{}}}
\ITA{\section{Alcune basi teoriche}}
\PTBR{\section{\PTBRph{}}}
\THA{\section{\THAph{}}}
\PL{\section{\PLph{}}}

% sections:
\EN{\input{patterns/intro_CPU_ISA_EN}}
\ES{\input{patterns/intro_CPU_ISA_ES}}
\ITA{\input{patterns/intro_CPU_ISA_ITA}}
\PTBR{\input{patterns/intro_CPU_ISA_PTBR}}
\RU{\input{patterns/intro_CPU_ISA_RU}}
\DE{\input{patterns/intro_CPU_ISA_DE}}
\FR{\input{patterns/intro_CPU_ISA_FR}}
\PL{\input{patterns/intro_CPU_ISA_PL}}

\EN{\input{patterns/numeral_EN}}
\RU{\input{patterns/numeral_RU}}
\ITA{\input{patterns/numeral_ITA}}
\DE{\input{patterns/numeral_DE}}
\FR{\input{patterns/numeral_FR}}
\PL{\input{patterns/numeral_PL}}

% chapters
\input{patterns/00_empty/main}
\input{patterns/011_ret/main}
\input{patterns/01_helloworld/main}
\input{patterns/015_prolog_epilogue/main}
\input{patterns/02_stack/main}
\input{patterns/03_printf/main}
\input{patterns/04_scanf/main}
\input{patterns/05_passing_arguments/main}
\input{patterns/06_return_results/main}
\input{patterns/061_pointers/main}
\input{patterns/065_GOTO/main}
\input{patterns/07_jcc/main}
\input{patterns/08_switch/main}
\input{patterns/09_loops/main}
\input{patterns/10_strings/main}
\input{patterns/11_arith_optimizations/main}
\input{patterns/12_FPU/main}
\input{patterns/13_arrays/main}
\input{patterns/14_bitfields/main}
\EN{\input{patterns/145_LCG/main_EN}}
\RU{\input{patterns/145_LCG/main_RU}}
\input{patterns/15_structs/main}
\input{patterns/17_unions/main}
\input{patterns/18_pointers_to_functions/main}
\input{patterns/185_64bit_in_32_env/main}

\EN{\input{patterns/19_SIMD/main_EN}}
\RU{\input{patterns/19_SIMD/main_RU}}
\DE{\input{patterns/19_SIMD/main_DE}}

\EN{\input{patterns/20_x64/main_EN}}
\RU{\input{patterns/20_x64/main_RU}}

\EN{\input{patterns/205_floating_SIMD/main_EN}}
\RU{\input{patterns/205_floating_SIMD/main_RU}}
\DE{\input{patterns/205_floating_SIMD/main_DE}}

\EN{\input{patterns/ARM/main_EN}}
\RU{\input{patterns/ARM/main_RU}}
\DE{\input{patterns/ARM/main_DE}}

\input{patterns/MIPS/main}

\ifdefined\SPANISH
\chapter{Patrones de código}
\fi % SPANISH

\ifdefined\GERMAN
\chapter{Code-Muster}
\fi % GERMAN

\ifdefined\ENGLISH
\chapter{Code Patterns}
\fi % ENGLISH

\ifdefined\ITALIAN
\chapter{Forme di codice}
\fi % ITALIAN

\ifdefined\RUSSIAN
\chapter{Образцы кода}
\fi % RUSSIAN

\ifdefined\BRAZILIAN
\chapter{Padrões de códigos}
\fi % BRAZILIAN

\ifdefined\THAI
\chapter{รูปแบบของโค้ด}
\fi % THAI

\ifdefined\FRENCH
\chapter{Modèle de code}
\fi % FRENCH

\ifdefined\POLISH
\chapter{\PLph{}}
\fi % POLISH

% sections
\EN{\input{patterns/patterns_opt_dbg_EN}}
\ES{\input{patterns/patterns_opt_dbg_ES}}
\ITA{\input{patterns/patterns_opt_dbg_ITA}}
\PTBR{\input{patterns/patterns_opt_dbg_PTBR}}
\RU{\input{patterns/patterns_opt_dbg_RU}}
\THA{\input{patterns/patterns_opt_dbg_THA}}
\DE{\input{patterns/patterns_opt_dbg_DE}}
\FR{\input{patterns/patterns_opt_dbg_FR}}
\PL{\input{patterns/patterns_opt_dbg_PL}}

\RU{\section{Некоторые базовые понятия}}
\EN{\section{Some basics}}
\DE{\section{Einige Grundlagen}}
\FR{\section{Quelques bases}}
\ES{\section{\ESph{}}}
\ITA{\section{Alcune basi teoriche}}
\PTBR{\section{\PTBRph{}}}
\THA{\section{\THAph{}}}
\PL{\section{\PLph{}}}

% sections:
\EN{\input{patterns/intro_CPU_ISA_EN}}
\ES{\input{patterns/intro_CPU_ISA_ES}}
\ITA{\input{patterns/intro_CPU_ISA_ITA}}
\PTBR{\input{patterns/intro_CPU_ISA_PTBR}}
\RU{\input{patterns/intro_CPU_ISA_RU}}
\DE{\input{patterns/intro_CPU_ISA_DE}}
\FR{\input{patterns/intro_CPU_ISA_FR}}
\PL{\input{patterns/intro_CPU_ISA_PL}}

\EN{\input{patterns/numeral_EN}}
\RU{\input{patterns/numeral_RU}}
\ITA{\input{patterns/numeral_ITA}}
\DE{\input{patterns/numeral_DE}}
\FR{\input{patterns/numeral_FR}}
\PL{\input{patterns/numeral_PL}}

% chapters
\input{patterns/00_empty/main}
\input{patterns/011_ret/main}
\input{patterns/01_helloworld/main}
\input{patterns/015_prolog_epilogue/main}
\input{patterns/02_stack/main}
\input{patterns/03_printf/main}
\input{patterns/04_scanf/main}
\input{patterns/05_passing_arguments/main}
\input{patterns/06_return_results/main}
\input{patterns/061_pointers/main}
\input{patterns/065_GOTO/main}
\input{patterns/07_jcc/main}
\input{patterns/08_switch/main}
\input{patterns/09_loops/main}
\input{patterns/10_strings/main}
\input{patterns/11_arith_optimizations/main}
\input{patterns/12_FPU/main}
\input{patterns/13_arrays/main}
\input{patterns/14_bitfields/main}
\EN{\input{patterns/145_LCG/main_EN}}
\RU{\input{patterns/145_LCG/main_RU}}
\input{patterns/15_structs/main}
\input{patterns/17_unions/main}
\input{patterns/18_pointers_to_functions/main}
\input{patterns/185_64bit_in_32_env/main}

\EN{\input{patterns/19_SIMD/main_EN}}
\RU{\input{patterns/19_SIMD/main_RU}}
\DE{\input{patterns/19_SIMD/main_DE}}

\EN{\input{patterns/20_x64/main_EN}}
\RU{\input{patterns/20_x64/main_RU}}

\EN{\input{patterns/205_floating_SIMD/main_EN}}
\RU{\input{patterns/205_floating_SIMD/main_RU}}
\DE{\input{patterns/205_floating_SIMD/main_DE}}

\EN{\input{patterns/ARM/main_EN}}
\RU{\input{patterns/ARM/main_RU}}
\DE{\input{patterns/ARM/main_DE}}

\input{patterns/MIPS/main}

\ifdefined\SPANISH
\chapter{Patrones de código}
\fi % SPANISH

\ifdefined\GERMAN
\chapter{Code-Muster}
\fi % GERMAN

\ifdefined\ENGLISH
\chapter{Code Patterns}
\fi % ENGLISH

\ifdefined\ITALIAN
\chapter{Forme di codice}
\fi % ITALIAN

\ifdefined\RUSSIAN
\chapter{Образцы кода}
\fi % RUSSIAN

\ifdefined\BRAZILIAN
\chapter{Padrões de códigos}
\fi % BRAZILIAN

\ifdefined\THAI
\chapter{รูปแบบของโค้ด}
\fi % THAI

\ifdefined\FRENCH
\chapter{Modèle de code}
\fi % FRENCH

\ifdefined\POLISH
\chapter{\PLph{}}
\fi % POLISH

% sections
\EN{\input{patterns/patterns_opt_dbg_EN}}
\ES{\input{patterns/patterns_opt_dbg_ES}}
\ITA{\input{patterns/patterns_opt_dbg_ITA}}
\PTBR{\input{patterns/patterns_opt_dbg_PTBR}}
\RU{\input{patterns/patterns_opt_dbg_RU}}
\THA{\input{patterns/patterns_opt_dbg_THA}}
\DE{\input{patterns/patterns_opt_dbg_DE}}
\FR{\input{patterns/patterns_opt_dbg_FR}}
\PL{\input{patterns/patterns_opt_dbg_PL}}

\RU{\section{Некоторые базовые понятия}}
\EN{\section{Some basics}}
\DE{\section{Einige Grundlagen}}
\FR{\section{Quelques bases}}
\ES{\section{\ESph{}}}
\ITA{\section{Alcune basi teoriche}}
\PTBR{\section{\PTBRph{}}}
\THA{\section{\THAph{}}}
\PL{\section{\PLph{}}}

% sections:
\EN{\input{patterns/intro_CPU_ISA_EN}}
\ES{\input{patterns/intro_CPU_ISA_ES}}
\ITA{\input{patterns/intro_CPU_ISA_ITA}}
\PTBR{\input{patterns/intro_CPU_ISA_PTBR}}
\RU{\input{patterns/intro_CPU_ISA_RU}}
\DE{\input{patterns/intro_CPU_ISA_DE}}
\FR{\input{patterns/intro_CPU_ISA_FR}}
\PL{\input{patterns/intro_CPU_ISA_PL}}

\EN{\input{patterns/numeral_EN}}
\RU{\input{patterns/numeral_RU}}
\ITA{\input{patterns/numeral_ITA}}
\DE{\input{patterns/numeral_DE}}
\FR{\input{patterns/numeral_FR}}
\PL{\input{patterns/numeral_PL}}

% chapters
\input{patterns/00_empty/main}
\input{patterns/011_ret/main}
\input{patterns/01_helloworld/main}
\input{patterns/015_prolog_epilogue/main}
\input{patterns/02_stack/main}
\input{patterns/03_printf/main}
\input{patterns/04_scanf/main}
\input{patterns/05_passing_arguments/main}
\input{patterns/06_return_results/main}
\input{patterns/061_pointers/main}
\input{patterns/065_GOTO/main}
\input{patterns/07_jcc/main}
\input{patterns/08_switch/main}
\input{patterns/09_loops/main}
\input{patterns/10_strings/main}
\input{patterns/11_arith_optimizations/main}
\input{patterns/12_FPU/main}
\input{patterns/13_arrays/main}
\input{patterns/14_bitfields/main}
\EN{\input{patterns/145_LCG/main_EN}}
\RU{\input{patterns/145_LCG/main_RU}}
\input{patterns/15_structs/main}
\input{patterns/17_unions/main}
\input{patterns/18_pointers_to_functions/main}
\input{patterns/185_64bit_in_32_env/main}

\EN{\input{patterns/19_SIMD/main_EN}}
\RU{\input{patterns/19_SIMD/main_RU}}
\DE{\input{patterns/19_SIMD/main_DE}}

\EN{\input{patterns/20_x64/main_EN}}
\RU{\input{patterns/20_x64/main_RU}}

\EN{\input{patterns/205_floating_SIMD/main_EN}}
\RU{\input{patterns/205_floating_SIMD/main_RU}}
\DE{\input{patterns/205_floating_SIMD/main_DE}}

\EN{\input{patterns/ARM/main_EN}}
\RU{\input{patterns/ARM/main_RU}}
\DE{\input{patterns/ARM/main_DE}}

\input{patterns/MIPS/main}

\ifdefined\SPANISH
\chapter{Patrones de código}
\fi % SPANISH

\ifdefined\GERMAN
\chapter{Code-Muster}
\fi % GERMAN

\ifdefined\ENGLISH
\chapter{Code Patterns}
\fi % ENGLISH

\ifdefined\ITALIAN
\chapter{Forme di codice}
\fi % ITALIAN

\ifdefined\RUSSIAN
\chapter{Образцы кода}
\fi % RUSSIAN

\ifdefined\BRAZILIAN
\chapter{Padrões de códigos}
\fi % BRAZILIAN

\ifdefined\THAI
\chapter{รูปแบบของโค้ด}
\fi % THAI

\ifdefined\FRENCH
\chapter{Modèle de code}
\fi % FRENCH

\ifdefined\POLISH
\chapter{\PLph{}}
\fi % POLISH

% sections
\EN{\input{patterns/patterns_opt_dbg_EN}}
\ES{\input{patterns/patterns_opt_dbg_ES}}
\ITA{\input{patterns/patterns_opt_dbg_ITA}}
\PTBR{\input{patterns/patterns_opt_dbg_PTBR}}
\RU{\input{patterns/patterns_opt_dbg_RU}}
\THA{\input{patterns/patterns_opt_dbg_THA}}
\DE{\input{patterns/patterns_opt_dbg_DE}}
\FR{\input{patterns/patterns_opt_dbg_FR}}
\PL{\input{patterns/patterns_opt_dbg_PL}}

\RU{\section{Некоторые базовые понятия}}
\EN{\section{Some basics}}
\DE{\section{Einige Grundlagen}}
\FR{\section{Quelques bases}}
\ES{\section{\ESph{}}}
\ITA{\section{Alcune basi teoriche}}
\PTBR{\section{\PTBRph{}}}
\THA{\section{\THAph{}}}
\PL{\section{\PLph{}}}

% sections:
\EN{\input{patterns/intro_CPU_ISA_EN}}
\ES{\input{patterns/intro_CPU_ISA_ES}}
\ITA{\input{patterns/intro_CPU_ISA_ITA}}
\PTBR{\input{patterns/intro_CPU_ISA_PTBR}}
\RU{\input{patterns/intro_CPU_ISA_RU}}
\DE{\input{patterns/intro_CPU_ISA_DE}}
\FR{\input{patterns/intro_CPU_ISA_FR}}
\PL{\input{patterns/intro_CPU_ISA_PL}}

\EN{\input{patterns/numeral_EN}}
\RU{\input{patterns/numeral_RU}}
\ITA{\input{patterns/numeral_ITA}}
\DE{\input{patterns/numeral_DE}}
\FR{\input{patterns/numeral_FR}}
\PL{\input{patterns/numeral_PL}}

% chapters
\input{patterns/00_empty/main}
\input{patterns/011_ret/main}
\input{patterns/01_helloworld/main}
\input{patterns/015_prolog_epilogue/main}
\input{patterns/02_stack/main}
\input{patterns/03_printf/main}
\input{patterns/04_scanf/main}
\input{patterns/05_passing_arguments/main}
\input{patterns/06_return_results/main}
\input{patterns/061_pointers/main}
\input{patterns/065_GOTO/main}
\input{patterns/07_jcc/main}
\input{patterns/08_switch/main}
\input{patterns/09_loops/main}
\input{patterns/10_strings/main}
\input{patterns/11_arith_optimizations/main}
\input{patterns/12_FPU/main}
\input{patterns/13_arrays/main}
\input{patterns/14_bitfields/main}
\EN{\input{patterns/145_LCG/main_EN}}
\RU{\input{patterns/145_LCG/main_RU}}
\input{patterns/15_structs/main}
\input{patterns/17_unions/main}
\input{patterns/18_pointers_to_functions/main}
\input{patterns/185_64bit_in_32_env/main}

\EN{\input{patterns/19_SIMD/main_EN}}
\RU{\input{patterns/19_SIMD/main_RU}}
\DE{\input{patterns/19_SIMD/main_DE}}

\EN{\input{patterns/20_x64/main_EN}}
\RU{\input{patterns/20_x64/main_RU}}

\EN{\input{patterns/205_floating_SIMD/main_EN}}
\RU{\input{patterns/205_floating_SIMD/main_RU}}
\DE{\input{patterns/205_floating_SIMD/main_DE}}

\EN{\input{patterns/ARM/main_EN}}
\RU{\input{patterns/ARM/main_RU}}
\DE{\input{patterns/ARM/main_DE}}

\input{patterns/MIPS/main}

\ifdefined\SPANISH
\chapter{Patrones de código}
\fi % SPANISH

\ifdefined\GERMAN
\chapter{Code-Muster}
\fi % GERMAN

\ifdefined\ENGLISH
\chapter{Code Patterns}
\fi % ENGLISH

\ifdefined\ITALIAN
\chapter{Forme di codice}
\fi % ITALIAN

\ifdefined\RUSSIAN
\chapter{Образцы кода}
\fi % RUSSIAN

\ifdefined\BRAZILIAN
\chapter{Padrões de códigos}
\fi % BRAZILIAN

\ifdefined\THAI
\chapter{รูปแบบของโค้ด}
\fi % THAI

\ifdefined\FRENCH
\chapter{Modèle de code}
\fi % FRENCH

\ifdefined\POLISH
\chapter{\PLph{}}
\fi % POLISH

% sections
\EN{\input{patterns/patterns_opt_dbg_EN}}
\ES{\input{patterns/patterns_opt_dbg_ES}}
\ITA{\input{patterns/patterns_opt_dbg_ITA}}
\PTBR{\input{patterns/patterns_opt_dbg_PTBR}}
\RU{\input{patterns/patterns_opt_dbg_RU}}
\THA{\input{patterns/patterns_opt_dbg_THA}}
\DE{\input{patterns/patterns_opt_dbg_DE}}
\FR{\input{patterns/patterns_opt_dbg_FR}}
\PL{\input{patterns/patterns_opt_dbg_PL}}

\RU{\section{Некоторые базовые понятия}}
\EN{\section{Some basics}}
\DE{\section{Einige Grundlagen}}
\FR{\section{Quelques bases}}
\ES{\section{\ESph{}}}
\ITA{\section{Alcune basi teoriche}}
\PTBR{\section{\PTBRph{}}}
\THA{\section{\THAph{}}}
\PL{\section{\PLph{}}}

% sections:
\EN{\input{patterns/intro_CPU_ISA_EN}}
\ES{\input{patterns/intro_CPU_ISA_ES}}
\ITA{\input{patterns/intro_CPU_ISA_ITA}}
\PTBR{\input{patterns/intro_CPU_ISA_PTBR}}
\RU{\input{patterns/intro_CPU_ISA_RU}}
\DE{\input{patterns/intro_CPU_ISA_DE}}
\FR{\input{patterns/intro_CPU_ISA_FR}}
\PL{\input{patterns/intro_CPU_ISA_PL}}

\EN{\input{patterns/numeral_EN}}
\RU{\input{patterns/numeral_RU}}
\ITA{\input{patterns/numeral_ITA}}
\DE{\input{patterns/numeral_DE}}
\FR{\input{patterns/numeral_FR}}
\PL{\input{patterns/numeral_PL}}

% chapters
\input{patterns/00_empty/main}
\input{patterns/011_ret/main}
\input{patterns/01_helloworld/main}
\input{patterns/015_prolog_epilogue/main}
\input{patterns/02_stack/main}
\input{patterns/03_printf/main}
\input{patterns/04_scanf/main}
\input{patterns/05_passing_arguments/main}
\input{patterns/06_return_results/main}
\input{patterns/061_pointers/main}
\input{patterns/065_GOTO/main}
\input{patterns/07_jcc/main}
\input{patterns/08_switch/main}
\input{patterns/09_loops/main}
\input{patterns/10_strings/main}
\input{patterns/11_arith_optimizations/main}
\input{patterns/12_FPU/main}
\input{patterns/13_arrays/main}
\input{patterns/14_bitfields/main}
\EN{\input{patterns/145_LCG/main_EN}}
\RU{\input{patterns/145_LCG/main_RU}}
\input{patterns/15_structs/main}
\input{patterns/17_unions/main}
\input{patterns/18_pointers_to_functions/main}
\input{patterns/185_64bit_in_32_env/main}

\EN{\input{patterns/19_SIMD/main_EN}}
\RU{\input{patterns/19_SIMD/main_RU}}
\DE{\input{patterns/19_SIMD/main_DE}}

\EN{\input{patterns/20_x64/main_EN}}
\RU{\input{patterns/20_x64/main_RU}}

\EN{\input{patterns/205_floating_SIMD/main_EN}}
\RU{\input{patterns/205_floating_SIMD/main_RU}}
\DE{\input{patterns/205_floating_SIMD/main_DE}}

\EN{\input{patterns/ARM/main_EN}}
\RU{\input{patterns/ARM/main_RU}}
\DE{\input{patterns/ARM/main_DE}}

\input{patterns/MIPS/main}

\ifdefined\SPANISH
\chapter{Patrones de código}
\fi % SPANISH

\ifdefined\GERMAN
\chapter{Code-Muster}
\fi % GERMAN

\ifdefined\ENGLISH
\chapter{Code Patterns}
\fi % ENGLISH

\ifdefined\ITALIAN
\chapter{Forme di codice}
\fi % ITALIAN

\ifdefined\RUSSIAN
\chapter{Образцы кода}
\fi % RUSSIAN

\ifdefined\BRAZILIAN
\chapter{Padrões de códigos}
\fi % BRAZILIAN

\ifdefined\THAI
\chapter{รูปแบบของโค้ด}
\fi % THAI

\ifdefined\FRENCH
\chapter{Modèle de code}
\fi % FRENCH

\ifdefined\POLISH
\chapter{\PLph{}}
\fi % POLISH

% sections
\EN{\input{patterns/patterns_opt_dbg_EN}}
\ES{\input{patterns/patterns_opt_dbg_ES}}
\ITA{\input{patterns/patterns_opt_dbg_ITA}}
\PTBR{\input{patterns/patterns_opt_dbg_PTBR}}
\RU{\input{patterns/patterns_opt_dbg_RU}}
\THA{\input{patterns/patterns_opt_dbg_THA}}
\DE{\input{patterns/patterns_opt_dbg_DE}}
\FR{\input{patterns/patterns_opt_dbg_FR}}
\PL{\input{patterns/patterns_opt_dbg_PL}}

\RU{\section{Некоторые базовые понятия}}
\EN{\section{Some basics}}
\DE{\section{Einige Grundlagen}}
\FR{\section{Quelques bases}}
\ES{\section{\ESph{}}}
\ITA{\section{Alcune basi teoriche}}
\PTBR{\section{\PTBRph{}}}
\THA{\section{\THAph{}}}
\PL{\section{\PLph{}}}

% sections:
\EN{\input{patterns/intro_CPU_ISA_EN}}
\ES{\input{patterns/intro_CPU_ISA_ES}}
\ITA{\input{patterns/intro_CPU_ISA_ITA}}
\PTBR{\input{patterns/intro_CPU_ISA_PTBR}}
\RU{\input{patterns/intro_CPU_ISA_RU}}
\DE{\input{patterns/intro_CPU_ISA_DE}}
\FR{\input{patterns/intro_CPU_ISA_FR}}
\PL{\input{patterns/intro_CPU_ISA_PL}}

\EN{\input{patterns/numeral_EN}}
\RU{\input{patterns/numeral_RU}}
\ITA{\input{patterns/numeral_ITA}}
\DE{\input{patterns/numeral_DE}}
\FR{\input{patterns/numeral_FR}}
\PL{\input{patterns/numeral_PL}}

% chapters
\input{patterns/00_empty/main}
\input{patterns/011_ret/main}
\input{patterns/01_helloworld/main}
\input{patterns/015_prolog_epilogue/main}
\input{patterns/02_stack/main}
\input{patterns/03_printf/main}
\input{patterns/04_scanf/main}
\input{patterns/05_passing_arguments/main}
\input{patterns/06_return_results/main}
\input{patterns/061_pointers/main}
\input{patterns/065_GOTO/main}
\input{patterns/07_jcc/main}
\input{patterns/08_switch/main}
\input{patterns/09_loops/main}
\input{patterns/10_strings/main}
\input{patterns/11_arith_optimizations/main}
\input{patterns/12_FPU/main}
\input{patterns/13_arrays/main}
\input{patterns/14_bitfields/main}
\EN{\input{patterns/145_LCG/main_EN}}
\RU{\input{patterns/145_LCG/main_RU}}
\input{patterns/15_structs/main}
\input{patterns/17_unions/main}
\input{patterns/18_pointers_to_functions/main}
\input{patterns/185_64bit_in_32_env/main}

\EN{\input{patterns/19_SIMD/main_EN}}
\RU{\input{patterns/19_SIMD/main_RU}}
\DE{\input{patterns/19_SIMD/main_DE}}

\EN{\input{patterns/20_x64/main_EN}}
\RU{\input{patterns/20_x64/main_RU}}

\EN{\input{patterns/205_floating_SIMD/main_EN}}
\RU{\input{patterns/205_floating_SIMD/main_RU}}
\DE{\input{patterns/205_floating_SIMD/main_DE}}

\EN{\input{patterns/ARM/main_EN}}
\RU{\input{patterns/ARM/main_RU}}
\DE{\input{patterns/ARM/main_DE}}

\input{patterns/MIPS/main}

\ifdefined\SPANISH
\chapter{Patrones de código}
\fi % SPANISH

\ifdefined\GERMAN
\chapter{Code-Muster}
\fi % GERMAN

\ifdefined\ENGLISH
\chapter{Code Patterns}
\fi % ENGLISH

\ifdefined\ITALIAN
\chapter{Forme di codice}
\fi % ITALIAN

\ifdefined\RUSSIAN
\chapter{Образцы кода}
\fi % RUSSIAN

\ifdefined\BRAZILIAN
\chapter{Padrões de códigos}
\fi % BRAZILIAN

\ifdefined\THAI
\chapter{รูปแบบของโค้ด}
\fi % THAI

\ifdefined\FRENCH
\chapter{Modèle de code}
\fi % FRENCH

\ifdefined\POLISH
\chapter{\PLph{}}
\fi % POLISH

% sections
\EN{\input{patterns/patterns_opt_dbg_EN}}
\ES{\input{patterns/patterns_opt_dbg_ES}}
\ITA{\input{patterns/patterns_opt_dbg_ITA}}
\PTBR{\input{patterns/patterns_opt_dbg_PTBR}}
\RU{\input{patterns/patterns_opt_dbg_RU}}
\THA{\input{patterns/patterns_opt_dbg_THA}}
\DE{\input{patterns/patterns_opt_dbg_DE}}
\FR{\input{patterns/patterns_opt_dbg_FR}}
\PL{\input{patterns/patterns_opt_dbg_PL}}

\RU{\section{Некоторые базовые понятия}}
\EN{\section{Some basics}}
\DE{\section{Einige Grundlagen}}
\FR{\section{Quelques bases}}
\ES{\section{\ESph{}}}
\ITA{\section{Alcune basi teoriche}}
\PTBR{\section{\PTBRph{}}}
\THA{\section{\THAph{}}}
\PL{\section{\PLph{}}}

% sections:
\EN{\input{patterns/intro_CPU_ISA_EN}}
\ES{\input{patterns/intro_CPU_ISA_ES}}
\ITA{\input{patterns/intro_CPU_ISA_ITA}}
\PTBR{\input{patterns/intro_CPU_ISA_PTBR}}
\RU{\input{patterns/intro_CPU_ISA_RU}}
\DE{\input{patterns/intro_CPU_ISA_DE}}
\FR{\input{patterns/intro_CPU_ISA_FR}}
\PL{\input{patterns/intro_CPU_ISA_PL}}

\EN{\input{patterns/numeral_EN}}
\RU{\input{patterns/numeral_RU}}
\ITA{\input{patterns/numeral_ITA}}
\DE{\input{patterns/numeral_DE}}
\FR{\input{patterns/numeral_FR}}
\PL{\input{patterns/numeral_PL}}

% chapters
\input{patterns/00_empty/main}
\input{patterns/011_ret/main}
\input{patterns/01_helloworld/main}
\input{patterns/015_prolog_epilogue/main}
\input{patterns/02_stack/main}
\input{patterns/03_printf/main}
\input{patterns/04_scanf/main}
\input{patterns/05_passing_arguments/main}
\input{patterns/06_return_results/main}
\input{patterns/061_pointers/main}
\input{patterns/065_GOTO/main}
\input{patterns/07_jcc/main}
\input{patterns/08_switch/main}
\input{patterns/09_loops/main}
\input{patterns/10_strings/main}
\input{patterns/11_arith_optimizations/main}
\input{patterns/12_FPU/main}
\input{patterns/13_arrays/main}
\input{patterns/14_bitfields/main}
\EN{\input{patterns/145_LCG/main_EN}}
\RU{\input{patterns/145_LCG/main_RU}}
\input{patterns/15_structs/main}
\input{patterns/17_unions/main}
\input{patterns/18_pointers_to_functions/main}
\input{patterns/185_64bit_in_32_env/main}

\EN{\input{patterns/19_SIMD/main_EN}}
\RU{\input{patterns/19_SIMD/main_RU}}
\DE{\input{patterns/19_SIMD/main_DE}}

\EN{\input{patterns/20_x64/main_EN}}
\RU{\input{patterns/20_x64/main_RU}}

\EN{\input{patterns/205_floating_SIMD/main_EN}}
\RU{\input{patterns/205_floating_SIMD/main_RU}}
\DE{\input{patterns/205_floating_SIMD/main_DE}}

\EN{\input{patterns/ARM/main_EN}}
\RU{\input{patterns/ARM/main_RU}}
\DE{\input{patterns/ARM/main_DE}}

\input{patterns/MIPS/main}

\EN{\input{patterns/12_FPU/main_EN}}
\RU{\input{patterns/12_FPU/main_RU}}
\DE{\input{patterns/12_FPU/main_DE}}
\FR{\input{patterns/12_FPU/main_FR}}


\ifdefined\SPANISH
\chapter{Patrones de código}
\fi % SPANISH

\ifdefined\GERMAN
\chapter{Code-Muster}
\fi % GERMAN

\ifdefined\ENGLISH
\chapter{Code Patterns}
\fi % ENGLISH

\ifdefined\ITALIAN
\chapter{Forme di codice}
\fi % ITALIAN

\ifdefined\RUSSIAN
\chapter{Образцы кода}
\fi % RUSSIAN

\ifdefined\BRAZILIAN
\chapter{Padrões de códigos}
\fi % BRAZILIAN

\ifdefined\THAI
\chapter{รูปแบบของโค้ด}
\fi % THAI

\ifdefined\FRENCH
\chapter{Modèle de code}
\fi % FRENCH

\ifdefined\POLISH
\chapter{\PLph{}}
\fi % POLISH

% sections
\EN{\input{patterns/patterns_opt_dbg_EN}}
\ES{\input{patterns/patterns_opt_dbg_ES}}
\ITA{\input{patterns/patterns_opt_dbg_ITA}}
\PTBR{\input{patterns/patterns_opt_dbg_PTBR}}
\RU{\input{patterns/patterns_opt_dbg_RU}}
\THA{\input{patterns/patterns_opt_dbg_THA}}
\DE{\input{patterns/patterns_opt_dbg_DE}}
\FR{\input{patterns/patterns_opt_dbg_FR}}
\PL{\input{patterns/patterns_opt_dbg_PL}}

\RU{\section{Некоторые базовые понятия}}
\EN{\section{Some basics}}
\DE{\section{Einige Grundlagen}}
\FR{\section{Quelques bases}}
\ES{\section{\ESph{}}}
\ITA{\section{Alcune basi teoriche}}
\PTBR{\section{\PTBRph{}}}
\THA{\section{\THAph{}}}
\PL{\section{\PLph{}}}

% sections:
\EN{\input{patterns/intro_CPU_ISA_EN}}
\ES{\input{patterns/intro_CPU_ISA_ES}}
\ITA{\input{patterns/intro_CPU_ISA_ITA}}
\PTBR{\input{patterns/intro_CPU_ISA_PTBR}}
\RU{\input{patterns/intro_CPU_ISA_RU}}
\DE{\input{patterns/intro_CPU_ISA_DE}}
\FR{\input{patterns/intro_CPU_ISA_FR}}
\PL{\input{patterns/intro_CPU_ISA_PL}}

\EN{\input{patterns/numeral_EN}}
\RU{\input{patterns/numeral_RU}}
\ITA{\input{patterns/numeral_ITA}}
\DE{\input{patterns/numeral_DE}}
\FR{\input{patterns/numeral_FR}}
\PL{\input{patterns/numeral_PL}}

% chapters
\input{patterns/00_empty/main}
\input{patterns/011_ret/main}
\input{patterns/01_helloworld/main}
\input{patterns/015_prolog_epilogue/main}
\input{patterns/02_stack/main}
\input{patterns/03_printf/main}
\input{patterns/04_scanf/main}
\input{patterns/05_passing_arguments/main}
\input{patterns/06_return_results/main}
\input{patterns/061_pointers/main}
\input{patterns/065_GOTO/main}
\input{patterns/07_jcc/main}
\input{patterns/08_switch/main}
\input{patterns/09_loops/main}
\input{patterns/10_strings/main}
\input{patterns/11_arith_optimizations/main}
\input{patterns/12_FPU/main}
\input{patterns/13_arrays/main}
\input{patterns/14_bitfields/main}
\EN{\input{patterns/145_LCG/main_EN}}
\RU{\input{patterns/145_LCG/main_RU}}
\input{patterns/15_structs/main}
\input{patterns/17_unions/main}
\input{patterns/18_pointers_to_functions/main}
\input{patterns/185_64bit_in_32_env/main}

\EN{\input{patterns/19_SIMD/main_EN}}
\RU{\input{patterns/19_SIMD/main_RU}}
\DE{\input{patterns/19_SIMD/main_DE}}

\EN{\input{patterns/20_x64/main_EN}}
\RU{\input{patterns/20_x64/main_RU}}

\EN{\input{patterns/205_floating_SIMD/main_EN}}
\RU{\input{patterns/205_floating_SIMD/main_RU}}
\DE{\input{patterns/205_floating_SIMD/main_DE}}

\EN{\input{patterns/ARM/main_EN}}
\RU{\input{patterns/ARM/main_RU}}
\DE{\input{patterns/ARM/main_DE}}

\input{patterns/MIPS/main}

\ifdefined\SPANISH
\chapter{Patrones de código}
\fi % SPANISH

\ifdefined\GERMAN
\chapter{Code-Muster}
\fi % GERMAN

\ifdefined\ENGLISH
\chapter{Code Patterns}
\fi % ENGLISH

\ifdefined\ITALIAN
\chapter{Forme di codice}
\fi % ITALIAN

\ifdefined\RUSSIAN
\chapter{Образцы кода}
\fi % RUSSIAN

\ifdefined\BRAZILIAN
\chapter{Padrões de códigos}
\fi % BRAZILIAN

\ifdefined\THAI
\chapter{รูปแบบของโค้ด}
\fi % THAI

\ifdefined\FRENCH
\chapter{Modèle de code}
\fi % FRENCH

\ifdefined\POLISH
\chapter{\PLph{}}
\fi % POLISH

% sections
\EN{\input{patterns/patterns_opt_dbg_EN}}
\ES{\input{patterns/patterns_opt_dbg_ES}}
\ITA{\input{patterns/patterns_opt_dbg_ITA}}
\PTBR{\input{patterns/patterns_opt_dbg_PTBR}}
\RU{\input{patterns/patterns_opt_dbg_RU}}
\THA{\input{patterns/patterns_opt_dbg_THA}}
\DE{\input{patterns/patterns_opt_dbg_DE}}
\FR{\input{patterns/patterns_opt_dbg_FR}}
\PL{\input{patterns/patterns_opt_dbg_PL}}

\RU{\section{Некоторые базовые понятия}}
\EN{\section{Some basics}}
\DE{\section{Einige Grundlagen}}
\FR{\section{Quelques bases}}
\ES{\section{\ESph{}}}
\ITA{\section{Alcune basi teoriche}}
\PTBR{\section{\PTBRph{}}}
\THA{\section{\THAph{}}}
\PL{\section{\PLph{}}}

% sections:
\EN{\input{patterns/intro_CPU_ISA_EN}}
\ES{\input{patterns/intro_CPU_ISA_ES}}
\ITA{\input{patterns/intro_CPU_ISA_ITA}}
\PTBR{\input{patterns/intro_CPU_ISA_PTBR}}
\RU{\input{patterns/intro_CPU_ISA_RU}}
\DE{\input{patterns/intro_CPU_ISA_DE}}
\FR{\input{patterns/intro_CPU_ISA_FR}}
\PL{\input{patterns/intro_CPU_ISA_PL}}

\EN{\input{patterns/numeral_EN}}
\RU{\input{patterns/numeral_RU}}
\ITA{\input{patterns/numeral_ITA}}
\DE{\input{patterns/numeral_DE}}
\FR{\input{patterns/numeral_FR}}
\PL{\input{patterns/numeral_PL}}

% chapters
\input{patterns/00_empty/main}
\input{patterns/011_ret/main}
\input{patterns/01_helloworld/main}
\input{patterns/015_prolog_epilogue/main}
\input{patterns/02_stack/main}
\input{patterns/03_printf/main}
\input{patterns/04_scanf/main}
\input{patterns/05_passing_arguments/main}
\input{patterns/06_return_results/main}
\input{patterns/061_pointers/main}
\input{patterns/065_GOTO/main}
\input{patterns/07_jcc/main}
\input{patterns/08_switch/main}
\input{patterns/09_loops/main}
\input{patterns/10_strings/main}
\input{patterns/11_arith_optimizations/main}
\input{patterns/12_FPU/main}
\input{patterns/13_arrays/main}
\input{patterns/14_bitfields/main}
\EN{\input{patterns/145_LCG/main_EN}}
\RU{\input{patterns/145_LCG/main_RU}}
\input{patterns/15_structs/main}
\input{patterns/17_unions/main}
\input{patterns/18_pointers_to_functions/main}
\input{patterns/185_64bit_in_32_env/main}

\EN{\input{patterns/19_SIMD/main_EN}}
\RU{\input{patterns/19_SIMD/main_RU}}
\DE{\input{patterns/19_SIMD/main_DE}}

\EN{\input{patterns/20_x64/main_EN}}
\RU{\input{patterns/20_x64/main_RU}}

\EN{\input{patterns/205_floating_SIMD/main_EN}}
\RU{\input{patterns/205_floating_SIMD/main_RU}}
\DE{\input{patterns/205_floating_SIMD/main_DE}}

\EN{\input{patterns/ARM/main_EN}}
\RU{\input{patterns/ARM/main_RU}}
\DE{\input{patterns/ARM/main_DE}}

\input{patterns/MIPS/main}

\EN{\section{Returning Values}
\label{ret_val_func}

Another simple function is the one that simply returns a constant value:

\lstinputlisting[caption=\EN{\CCpp Code},style=customc]{patterns/011_ret/1.c}

Let's compile it.

\subsection{x86}

Here's what both the GCC and MSVC compilers produce (with optimization) on the x86 platform:

\lstinputlisting[caption=\Optimizing GCC/MSVC (\assemblyOutput),style=customasmx86]{patterns/011_ret/1.s}

\myindex{x86!\Instructions!RET}
There are just two instructions: the first places the value 123 into the \EAX register,
which is used by convention for storing the return
value, and the second one is \RET, which returns execution to the \gls{caller}.

The caller will take the result from the \EAX register.

\subsection{ARM}

There are a few differences on the ARM platform:

\lstinputlisting[caption=\OptimizingKeilVI (\ARMMode) ASM Output,style=customasmARM]{patterns/011_ret/1_Keil_ARM_O3.s}

ARM uses the register \Reg{0} for returning the results of functions, so 123 is copied into \Reg{0}.

\myindex{ARM!\Instructions!MOV}
\myindex{x86!\Instructions!MOV}
It is worth noting that \MOV is a misleading name for the instruction in both the x86 and ARM \ac{ISA}s.

The data is not in fact \IT{moved}, but \IT{copied}.

\subsection{MIPS}

\label{MIPS_leaf_function_ex1}

The GCC assembly output below lists registers by number:

\lstinputlisting[caption=\Optimizing GCC 4.4.5 (\assemblyOutput),style=customasmMIPS]{patterns/011_ret/MIPS.s}

\dots while \IDA does it by their pseudo names:

\lstinputlisting[caption=\Optimizing GCC 4.4.5 (IDA),style=customasmMIPS]{patterns/011_ret/MIPS_IDA.lst}

The \$2 (or \$V0) register is used to store the function's return value.
\myindex{MIPS!\Pseudoinstructions!LI}
\INS{LI} stands for ``Load Immediate'' and is the MIPS equivalent to \MOV.

\myindex{MIPS!\Instructions!J}
The other instruction is the jump instruction (J or JR) which returns the execution flow to the \gls{caller}.

\myindex{MIPS!Branch delay slot}
You might be wondering why the positions of the load instruction (LI) and the jump instruction (J or JR) are swapped. This is due to a \ac{RISC} feature called ``branch delay slot''.

The reason this happens is a quirk in the architecture of some RISC \ac{ISA}s and isn't important for our
purposes---we must simply keep in mind that in MIPS, the instruction following a jump or branch instruction
is executed \IT{before} the jump/branch instruction itself.

As a consequence, branch instructions always swap places with the instruction executed immediately beforehand.


In practice, functions which merely return 1 (\IT{true}) or 0 (\IT{false}) are very frequent.

The smallest ever of the standard UNIX utilities, \IT{/bin/true} and \IT{/bin/false} return 0 and 1 respectively, as an exit code.
(Zero as an exit code usually means success, non-zero means error.)
}
\RU{\subsubsection{std::string}
\myindex{\Cpp!STL!std::string}
\label{std_string}

\myparagraph{Как устроена структура}

Многие строковые библиотеки \InSqBrackets{\CNotes 2.2} обеспечивают структуру содержащую ссылку 
на буфер собственно со строкой, переменная всегда содержащую длину строки 
(что очень удобно для массы функций \InSqBrackets{\CNotes 2.2.1}) и переменную содержащую текущий размер буфера.

Строка в буфере обыкновенно оканчивается нулем: это для того чтобы указатель на буфер можно было
передавать в функции требующие на вход обычную сишную \ac{ASCIIZ}-строку.

Стандарт \Cpp не описывает, как именно нужно реализовывать std::string,
но, как правило, они реализованы как описано выше, с небольшими дополнениями.

Строки в \Cpp это не класс (как, например, QString в Qt), а темплейт (basic\_string), 
это сделано для того чтобы поддерживать 
строки содержащие разного типа символы: как минимум \Tchar и \IT{wchar\_t}.

Так что, std::string это класс с базовым типом \Tchar.

А std::wstring это класс с базовым типом \IT{wchar\_t}.

\mysubparagraph{MSVC}

В реализации MSVC, вместо ссылки на буфер может содержаться сам буфер (если строка короче 16-и символов).

Это означает, что каждая короткая строка будет занимать в памяти по крайней мере $16 + 4 + 4 = 24$ 
байт для 32-битной среды либо $16 + 8 + 8 = 32$ 
байта в 64-битной, а если строка длиннее 16-и символов, то прибавьте еще длину самой строки.

\lstinputlisting[caption=пример для MSVC,style=customc]{\CURPATH/STL/string/MSVC_RU.cpp}

Собственно, из этого исходника почти всё ясно.

Несколько замечаний:

Если строка короче 16-и символов, 
то отдельный буфер для строки в \glslink{heap}{куче} выделяться не будет.

Это удобно потому что на практике, основная часть строк действительно короткие.
Вероятно, разработчики в Microsoft выбрали размер в 16 символов как разумный баланс.

Теперь очень важный момент в конце функции main(): мы не пользуемся методом c\_str(), тем не менее,
если это скомпилировать и запустить, то обе строки появятся в консоли!

Работает это вот почему.

В первом случае строка короче 16-и символов и в начале объекта std::string (его можно рассматривать
просто как структуру) расположен буфер с этой строкой.
\printf трактует указатель как указатель на массив символов оканчивающийся нулем и поэтому всё работает.

Вывод второй строки (длиннее 16-и символов) даже еще опаснее: это вообще типичная программистская ошибка 
(или опечатка), забыть дописать c\_str().
Это работает потому что в это время в начале структуры расположен указатель на буфер.
Это может надолго остаться незамеченным: до тех пока там не появится строка 
короче 16-и символов, тогда процесс упадет.

\mysubparagraph{GCC}

В реализации GCC в структуре есть еще одна переменная --- reference count.

Интересно, что указатель на экземпляр класса std::string в GCC указывает не на начало самой структуры, 
а на указатель на буфера.
В libstdc++-v3\textbackslash{}include\textbackslash{}bits\textbackslash{}basic\_string.h 
мы можем прочитать что это сделано для удобства отладки:

\begin{lstlisting}
   *  The reason you want _M_data pointing to the character %array and
   *  not the _Rep is so that the debugger can see the string
   *  contents. (Probably we should add a non-inline member to get
   *  the _Rep for the debugger to use, so users can check the actual
   *  string length.)
\end{lstlisting}

\href{http://go.yurichev.com/17085}{исходный код basic\_string.h}

В нашем примере мы учитываем это:

\lstinputlisting[caption=пример для GCC,style=customc]{\CURPATH/STL/string/GCC_RU.cpp}

Нужны еще небольшие хаки чтобы сымитировать типичную ошибку, которую мы уже видели выше, из-за
более ужесточенной проверки типов в GCC, тем не менее, printf() работает и здесь без c\_str().

\myparagraph{Чуть более сложный пример}

\lstinputlisting[style=customc]{\CURPATH/STL/string/3.cpp}

\lstinputlisting[caption=MSVC 2012,style=customasmx86]{\CURPATH/STL/string/3_MSVC_RU.asm}

Собственно, компилятор не конструирует строки статически: да в общем-то и как
это возможно, если буфер с ней нужно хранить в \glslink{heap}{куче}?

Вместо этого в сегменте данных хранятся обычные \ac{ASCIIZ}-строки, а позже, во время выполнения, 
при помощи метода \q{assign}, конструируются строки s1 и s2
.
При помощи \TT{operator+}, создается строка s3.

Обратите внимание на то что вызов метода c\_str() отсутствует,
потому что его код достаточно короткий и компилятор вставил его прямо здесь:
если строка короче 16-и байт, то в регистре EAX остается указатель на буфер,
а если длиннее, то из этого же места достается адрес на буфер расположенный в \glslink{heap}{куче}.

Далее следуют вызовы трех деструкторов, причем, они вызываются только если строка длиннее 16-и байт:
тогда нужно освободить буфера в \glslink{heap}{куче}.
В противном случае, так как все три объекта std::string хранятся в стеке,
они освобождаются автоматически после выхода из функции.

Следовательно, работа с короткими строками более быстрая из-за м\'{е}ньшего обращения к \glslink{heap}{куче}.

Код на GCC даже проще (из-за того, что в GCC, как мы уже видели, не реализована возможность хранить короткую
строку прямо в структуре):

% TODO1 comment each function meaning
\lstinputlisting[caption=GCC 4.8.1,style=customasmx86]{\CURPATH/STL/string/3_GCC_RU.s}

Можно заметить, что в деструкторы передается не указатель на объект,
а указатель на место за 12 байт (или 3 слова) перед ним, то есть, на настоящее начало структуры.

\myparagraph{std::string как глобальная переменная}
\label{sec:std_string_as_global_variable}

Опытные программисты на \Cpp знают, что глобальные переменные \ac{STL}-типов вполне можно объявлять.

Да, действительно:

\lstinputlisting[style=customc]{\CURPATH/STL/string/5.cpp}

Но как и где будет вызываться конструктор \TT{std::string}?

На самом деле, эта переменная будет инициализирована даже перед началом \main.

\lstinputlisting[caption=MSVC 2012: здесь конструируется глобальная переменная{,} а также регистрируется её деструктор,style=customasmx86]{\CURPATH/STL/string/5_MSVC_p2.asm}

\lstinputlisting[caption=MSVC 2012: здесь глобальная переменная используется в \main,style=customasmx86]{\CURPATH/STL/string/5_MSVC_p1.asm}

\lstinputlisting[caption=MSVC 2012: эта функция-деструктор вызывается перед выходом,style=customasmx86]{\CURPATH/STL/string/5_MSVC_p3.asm}

\myindex{\CStandardLibrary!atexit()}
В реальности, из \ac{CRT}, еще до вызова main(), вызывается специальная функция,
в которой перечислены все конструкторы подобных переменных.
Более того: при помощи atexit() регистрируется функция, которая будет вызвана в конце работы программы:
в этой функции компилятор собирает вызовы деструкторов всех подобных глобальных переменных.

GCC работает похожим образом:

\lstinputlisting[caption=GCC 4.8.1,style=customasmx86]{\CURPATH/STL/string/5_GCC.s}

Но он не выделяет отдельной функции в которой будут собраны деструкторы: 
каждый деструктор передается в atexit() по одному.

% TODO а если глобальная STL-переменная в другом модуле? надо проверить.

}
\ifdefined\SPANISH
\chapter{Patrones de código}
\fi % SPANISH

\ifdefined\GERMAN
\chapter{Code-Muster}
\fi % GERMAN

\ifdefined\ENGLISH
\chapter{Code Patterns}
\fi % ENGLISH

\ifdefined\ITALIAN
\chapter{Forme di codice}
\fi % ITALIAN

\ifdefined\RUSSIAN
\chapter{Образцы кода}
\fi % RUSSIAN

\ifdefined\BRAZILIAN
\chapter{Padrões de códigos}
\fi % BRAZILIAN

\ifdefined\THAI
\chapter{รูปแบบของโค้ด}
\fi % THAI

\ifdefined\FRENCH
\chapter{Modèle de code}
\fi % FRENCH

\ifdefined\POLISH
\chapter{\PLph{}}
\fi % POLISH

% sections
\EN{\input{patterns/patterns_opt_dbg_EN}}
\ES{\input{patterns/patterns_opt_dbg_ES}}
\ITA{\input{patterns/patterns_opt_dbg_ITA}}
\PTBR{\input{patterns/patterns_opt_dbg_PTBR}}
\RU{\input{patterns/patterns_opt_dbg_RU}}
\THA{\input{patterns/patterns_opt_dbg_THA}}
\DE{\input{patterns/patterns_opt_dbg_DE}}
\FR{\input{patterns/patterns_opt_dbg_FR}}
\PL{\input{patterns/patterns_opt_dbg_PL}}

\RU{\section{Некоторые базовые понятия}}
\EN{\section{Some basics}}
\DE{\section{Einige Grundlagen}}
\FR{\section{Quelques bases}}
\ES{\section{\ESph{}}}
\ITA{\section{Alcune basi teoriche}}
\PTBR{\section{\PTBRph{}}}
\THA{\section{\THAph{}}}
\PL{\section{\PLph{}}}

% sections:
\EN{\input{patterns/intro_CPU_ISA_EN}}
\ES{\input{patterns/intro_CPU_ISA_ES}}
\ITA{\input{patterns/intro_CPU_ISA_ITA}}
\PTBR{\input{patterns/intro_CPU_ISA_PTBR}}
\RU{\input{patterns/intro_CPU_ISA_RU}}
\DE{\input{patterns/intro_CPU_ISA_DE}}
\FR{\input{patterns/intro_CPU_ISA_FR}}
\PL{\input{patterns/intro_CPU_ISA_PL}}

\EN{\input{patterns/numeral_EN}}
\RU{\input{patterns/numeral_RU}}
\ITA{\input{patterns/numeral_ITA}}
\DE{\input{patterns/numeral_DE}}
\FR{\input{patterns/numeral_FR}}
\PL{\input{patterns/numeral_PL}}

% chapters
\input{patterns/00_empty/main}
\input{patterns/011_ret/main}
\input{patterns/01_helloworld/main}
\input{patterns/015_prolog_epilogue/main}
\input{patterns/02_stack/main}
\input{patterns/03_printf/main}
\input{patterns/04_scanf/main}
\input{patterns/05_passing_arguments/main}
\input{patterns/06_return_results/main}
\input{patterns/061_pointers/main}
\input{patterns/065_GOTO/main}
\input{patterns/07_jcc/main}
\input{patterns/08_switch/main}
\input{patterns/09_loops/main}
\input{patterns/10_strings/main}
\input{patterns/11_arith_optimizations/main}
\input{patterns/12_FPU/main}
\input{patterns/13_arrays/main}
\input{patterns/14_bitfields/main}
\EN{\input{patterns/145_LCG/main_EN}}
\RU{\input{patterns/145_LCG/main_RU}}
\input{patterns/15_structs/main}
\input{patterns/17_unions/main}
\input{patterns/18_pointers_to_functions/main}
\input{patterns/185_64bit_in_32_env/main}

\EN{\input{patterns/19_SIMD/main_EN}}
\RU{\input{patterns/19_SIMD/main_RU}}
\DE{\input{patterns/19_SIMD/main_DE}}

\EN{\input{patterns/20_x64/main_EN}}
\RU{\input{patterns/20_x64/main_RU}}

\EN{\input{patterns/205_floating_SIMD/main_EN}}
\RU{\input{patterns/205_floating_SIMD/main_RU}}
\DE{\input{patterns/205_floating_SIMD/main_DE}}

\EN{\input{patterns/ARM/main_EN}}
\RU{\input{patterns/ARM/main_RU}}
\DE{\input{patterns/ARM/main_DE}}

\input{patterns/MIPS/main}

\ifdefined\SPANISH
\chapter{Patrones de código}
\fi % SPANISH

\ifdefined\GERMAN
\chapter{Code-Muster}
\fi % GERMAN

\ifdefined\ENGLISH
\chapter{Code Patterns}
\fi % ENGLISH

\ifdefined\ITALIAN
\chapter{Forme di codice}
\fi % ITALIAN

\ifdefined\RUSSIAN
\chapter{Образцы кода}
\fi % RUSSIAN

\ifdefined\BRAZILIAN
\chapter{Padrões de códigos}
\fi % BRAZILIAN

\ifdefined\THAI
\chapter{รูปแบบของโค้ด}
\fi % THAI

\ifdefined\FRENCH
\chapter{Modèle de code}
\fi % FRENCH

\ifdefined\POLISH
\chapter{\PLph{}}
\fi % POLISH

% sections
\EN{\input{patterns/patterns_opt_dbg_EN}}
\ES{\input{patterns/patterns_opt_dbg_ES}}
\ITA{\input{patterns/patterns_opt_dbg_ITA}}
\PTBR{\input{patterns/patterns_opt_dbg_PTBR}}
\RU{\input{patterns/patterns_opt_dbg_RU}}
\THA{\input{patterns/patterns_opt_dbg_THA}}
\DE{\input{patterns/patterns_opt_dbg_DE}}
\FR{\input{patterns/patterns_opt_dbg_FR}}
\PL{\input{patterns/patterns_opt_dbg_PL}}

\RU{\section{Некоторые базовые понятия}}
\EN{\section{Some basics}}
\DE{\section{Einige Grundlagen}}
\FR{\section{Quelques bases}}
\ES{\section{\ESph{}}}
\ITA{\section{Alcune basi teoriche}}
\PTBR{\section{\PTBRph{}}}
\THA{\section{\THAph{}}}
\PL{\section{\PLph{}}}

% sections:
\EN{\input{patterns/intro_CPU_ISA_EN}}
\ES{\input{patterns/intro_CPU_ISA_ES}}
\ITA{\input{patterns/intro_CPU_ISA_ITA}}
\PTBR{\input{patterns/intro_CPU_ISA_PTBR}}
\RU{\input{patterns/intro_CPU_ISA_RU}}
\DE{\input{patterns/intro_CPU_ISA_DE}}
\FR{\input{patterns/intro_CPU_ISA_FR}}
\PL{\input{patterns/intro_CPU_ISA_PL}}

\EN{\input{patterns/numeral_EN}}
\RU{\input{patterns/numeral_RU}}
\ITA{\input{patterns/numeral_ITA}}
\DE{\input{patterns/numeral_DE}}
\FR{\input{patterns/numeral_FR}}
\PL{\input{patterns/numeral_PL}}

% chapters
\input{patterns/00_empty/main}
\input{patterns/011_ret/main}
\input{patterns/01_helloworld/main}
\input{patterns/015_prolog_epilogue/main}
\input{patterns/02_stack/main}
\input{patterns/03_printf/main}
\input{patterns/04_scanf/main}
\input{patterns/05_passing_arguments/main}
\input{patterns/06_return_results/main}
\input{patterns/061_pointers/main}
\input{patterns/065_GOTO/main}
\input{patterns/07_jcc/main}
\input{patterns/08_switch/main}
\input{patterns/09_loops/main}
\input{patterns/10_strings/main}
\input{patterns/11_arith_optimizations/main}
\input{patterns/12_FPU/main}
\input{patterns/13_arrays/main}
\input{patterns/14_bitfields/main}
\EN{\input{patterns/145_LCG/main_EN}}
\RU{\input{patterns/145_LCG/main_RU}}
\input{patterns/15_structs/main}
\input{patterns/17_unions/main}
\input{patterns/18_pointers_to_functions/main}
\input{patterns/185_64bit_in_32_env/main}

\EN{\input{patterns/19_SIMD/main_EN}}
\RU{\input{patterns/19_SIMD/main_RU}}
\DE{\input{patterns/19_SIMD/main_DE}}

\EN{\input{patterns/20_x64/main_EN}}
\RU{\input{patterns/20_x64/main_RU}}

\EN{\input{patterns/205_floating_SIMD/main_EN}}
\RU{\input{patterns/205_floating_SIMD/main_RU}}
\DE{\input{patterns/205_floating_SIMD/main_DE}}

\EN{\input{patterns/ARM/main_EN}}
\RU{\input{patterns/ARM/main_RU}}
\DE{\input{patterns/ARM/main_DE}}

\input{patterns/MIPS/main}

\ifdefined\SPANISH
\chapter{Patrones de código}
\fi % SPANISH

\ifdefined\GERMAN
\chapter{Code-Muster}
\fi % GERMAN

\ifdefined\ENGLISH
\chapter{Code Patterns}
\fi % ENGLISH

\ifdefined\ITALIAN
\chapter{Forme di codice}
\fi % ITALIAN

\ifdefined\RUSSIAN
\chapter{Образцы кода}
\fi % RUSSIAN

\ifdefined\BRAZILIAN
\chapter{Padrões de códigos}
\fi % BRAZILIAN

\ifdefined\THAI
\chapter{รูปแบบของโค้ด}
\fi % THAI

\ifdefined\FRENCH
\chapter{Modèle de code}
\fi % FRENCH

\ifdefined\POLISH
\chapter{\PLph{}}
\fi % POLISH

% sections
\EN{\input{patterns/patterns_opt_dbg_EN}}
\ES{\input{patterns/patterns_opt_dbg_ES}}
\ITA{\input{patterns/patterns_opt_dbg_ITA}}
\PTBR{\input{patterns/patterns_opt_dbg_PTBR}}
\RU{\input{patterns/patterns_opt_dbg_RU}}
\THA{\input{patterns/patterns_opt_dbg_THA}}
\DE{\input{patterns/patterns_opt_dbg_DE}}
\FR{\input{patterns/patterns_opt_dbg_FR}}
\PL{\input{patterns/patterns_opt_dbg_PL}}

\RU{\section{Некоторые базовые понятия}}
\EN{\section{Some basics}}
\DE{\section{Einige Grundlagen}}
\FR{\section{Quelques bases}}
\ES{\section{\ESph{}}}
\ITA{\section{Alcune basi teoriche}}
\PTBR{\section{\PTBRph{}}}
\THA{\section{\THAph{}}}
\PL{\section{\PLph{}}}

% sections:
\EN{\input{patterns/intro_CPU_ISA_EN}}
\ES{\input{patterns/intro_CPU_ISA_ES}}
\ITA{\input{patterns/intro_CPU_ISA_ITA}}
\PTBR{\input{patterns/intro_CPU_ISA_PTBR}}
\RU{\input{patterns/intro_CPU_ISA_RU}}
\DE{\input{patterns/intro_CPU_ISA_DE}}
\FR{\input{patterns/intro_CPU_ISA_FR}}
\PL{\input{patterns/intro_CPU_ISA_PL}}

\EN{\input{patterns/numeral_EN}}
\RU{\input{patterns/numeral_RU}}
\ITA{\input{patterns/numeral_ITA}}
\DE{\input{patterns/numeral_DE}}
\FR{\input{patterns/numeral_FR}}
\PL{\input{patterns/numeral_PL}}

% chapters
\input{patterns/00_empty/main}
\input{patterns/011_ret/main}
\input{patterns/01_helloworld/main}
\input{patterns/015_prolog_epilogue/main}
\input{patterns/02_stack/main}
\input{patterns/03_printf/main}
\input{patterns/04_scanf/main}
\input{patterns/05_passing_arguments/main}
\input{patterns/06_return_results/main}
\input{patterns/061_pointers/main}
\input{patterns/065_GOTO/main}
\input{patterns/07_jcc/main}
\input{patterns/08_switch/main}
\input{patterns/09_loops/main}
\input{patterns/10_strings/main}
\input{patterns/11_arith_optimizations/main}
\input{patterns/12_FPU/main}
\input{patterns/13_arrays/main}
\input{patterns/14_bitfields/main}
\EN{\input{patterns/145_LCG/main_EN}}
\RU{\input{patterns/145_LCG/main_RU}}
\input{patterns/15_structs/main}
\input{patterns/17_unions/main}
\input{patterns/18_pointers_to_functions/main}
\input{patterns/185_64bit_in_32_env/main}

\EN{\input{patterns/19_SIMD/main_EN}}
\RU{\input{patterns/19_SIMD/main_RU}}
\DE{\input{patterns/19_SIMD/main_DE}}

\EN{\input{patterns/20_x64/main_EN}}
\RU{\input{patterns/20_x64/main_RU}}

\EN{\input{patterns/205_floating_SIMD/main_EN}}
\RU{\input{patterns/205_floating_SIMD/main_RU}}
\DE{\input{patterns/205_floating_SIMD/main_DE}}

\EN{\input{patterns/ARM/main_EN}}
\RU{\input{patterns/ARM/main_RU}}
\DE{\input{patterns/ARM/main_DE}}

\input{patterns/MIPS/main}

\ifdefined\SPANISH
\chapter{Patrones de código}
\fi % SPANISH

\ifdefined\GERMAN
\chapter{Code-Muster}
\fi % GERMAN

\ifdefined\ENGLISH
\chapter{Code Patterns}
\fi % ENGLISH

\ifdefined\ITALIAN
\chapter{Forme di codice}
\fi % ITALIAN

\ifdefined\RUSSIAN
\chapter{Образцы кода}
\fi % RUSSIAN

\ifdefined\BRAZILIAN
\chapter{Padrões de códigos}
\fi % BRAZILIAN

\ifdefined\THAI
\chapter{รูปแบบของโค้ด}
\fi % THAI

\ifdefined\FRENCH
\chapter{Modèle de code}
\fi % FRENCH

\ifdefined\POLISH
\chapter{\PLph{}}
\fi % POLISH

% sections
\EN{\input{patterns/patterns_opt_dbg_EN}}
\ES{\input{patterns/patterns_opt_dbg_ES}}
\ITA{\input{patterns/patterns_opt_dbg_ITA}}
\PTBR{\input{patterns/patterns_opt_dbg_PTBR}}
\RU{\input{patterns/patterns_opt_dbg_RU}}
\THA{\input{patterns/patterns_opt_dbg_THA}}
\DE{\input{patterns/patterns_opt_dbg_DE}}
\FR{\input{patterns/patterns_opt_dbg_FR}}
\PL{\input{patterns/patterns_opt_dbg_PL}}

\RU{\section{Некоторые базовые понятия}}
\EN{\section{Some basics}}
\DE{\section{Einige Grundlagen}}
\FR{\section{Quelques bases}}
\ES{\section{\ESph{}}}
\ITA{\section{Alcune basi teoriche}}
\PTBR{\section{\PTBRph{}}}
\THA{\section{\THAph{}}}
\PL{\section{\PLph{}}}

% sections:
\EN{\input{patterns/intro_CPU_ISA_EN}}
\ES{\input{patterns/intro_CPU_ISA_ES}}
\ITA{\input{patterns/intro_CPU_ISA_ITA}}
\PTBR{\input{patterns/intro_CPU_ISA_PTBR}}
\RU{\input{patterns/intro_CPU_ISA_RU}}
\DE{\input{patterns/intro_CPU_ISA_DE}}
\FR{\input{patterns/intro_CPU_ISA_FR}}
\PL{\input{patterns/intro_CPU_ISA_PL}}

\EN{\input{patterns/numeral_EN}}
\RU{\input{patterns/numeral_RU}}
\ITA{\input{patterns/numeral_ITA}}
\DE{\input{patterns/numeral_DE}}
\FR{\input{patterns/numeral_FR}}
\PL{\input{patterns/numeral_PL}}

% chapters
\input{patterns/00_empty/main}
\input{patterns/011_ret/main}
\input{patterns/01_helloworld/main}
\input{patterns/015_prolog_epilogue/main}
\input{patterns/02_stack/main}
\input{patterns/03_printf/main}
\input{patterns/04_scanf/main}
\input{patterns/05_passing_arguments/main}
\input{patterns/06_return_results/main}
\input{patterns/061_pointers/main}
\input{patterns/065_GOTO/main}
\input{patterns/07_jcc/main}
\input{patterns/08_switch/main}
\input{patterns/09_loops/main}
\input{patterns/10_strings/main}
\input{patterns/11_arith_optimizations/main}
\input{patterns/12_FPU/main}
\input{patterns/13_arrays/main}
\input{patterns/14_bitfields/main}
\EN{\input{patterns/145_LCG/main_EN}}
\RU{\input{patterns/145_LCG/main_RU}}
\input{patterns/15_structs/main}
\input{patterns/17_unions/main}
\input{patterns/18_pointers_to_functions/main}
\input{patterns/185_64bit_in_32_env/main}

\EN{\input{patterns/19_SIMD/main_EN}}
\RU{\input{patterns/19_SIMD/main_RU}}
\DE{\input{patterns/19_SIMD/main_DE}}

\EN{\input{patterns/20_x64/main_EN}}
\RU{\input{patterns/20_x64/main_RU}}

\EN{\input{patterns/205_floating_SIMD/main_EN}}
\RU{\input{patterns/205_floating_SIMD/main_RU}}
\DE{\input{patterns/205_floating_SIMD/main_DE}}

\EN{\input{patterns/ARM/main_EN}}
\RU{\input{patterns/ARM/main_RU}}
\DE{\input{patterns/ARM/main_DE}}

\input{patterns/MIPS/main}


\EN{\section{Returning Values}
\label{ret_val_func}

Another simple function is the one that simply returns a constant value:

\lstinputlisting[caption=\EN{\CCpp Code},style=customc]{patterns/011_ret/1.c}

Let's compile it.

\subsection{x86}

Here's what both the GCC and MSVC compilers produce (with optimization) on the x86 platform:

\lstinputlisting[caption=\Optimizing GCC/MSVC (\assemblyOutput),style=customasmx86]{patterns/011_ret/1.s}

\myindex{x86!\Instructions!RET}
There are just two instructions: the first places the value 123 into the \EAX register,
which is used by convention for storing the return
value, and the second one is \RET, which returns execution to the \gls{caller}.

The caller will take the result from the \EAX register.

\subsection{ARM}

There are a few differences on the ARM platform:

\lstinputlisting[caption=\OptimizingKeilVI (\ARMMode) ASM Output,style=customasmARM]{patterns/011_ret/1_Keil_ARM_O3.s}

ARM uses the register \Reg{0} for returning the results of functions, so 123 is copied into \Reg{0}.

\myindex{ARM!\Instructions!MOV}
\myindex{x86!\Instructions!MOV}
It is worth noting that \MOV is a misleading name for the instruction in both the x86 and ARM \ac{ISA}s.

The data is not in fact \IT{moved}, but \IT{copied}.

\subsection{MIPS}

\label{MIPS_leaf_function_ex1}

The GCC assembly output below lists registers by number:

\lstinputlisting[caption=\Optimizing GCC 4.4.5 (\assemblyOutput),style=customasmMIPS]{patterns/011_ret/MIPS.s}

\dots while \IDA does it by their pseudo names:

\lstinputlisting[caption=\Optimizing GCC 4.4.5 (IDA),style=customasmMIPS]{patterns/011_ret/MIPS_IDA.lst}

The \$2 (or \$V0) register is used to store the function's return value.
\myindex{MIPS!\Pseudoinstructions!LI}
\INS{LI} stands for ``Load Immediate'' and is the MIPS equivalent to \MOV.

\myindex{MIPS!\Instructions!J}
The other instruction is the jump instruction (J or JR) which returns the execution flow to the \gls{caller}.

\myindex{MIPS!Branch delay slot}
You might be wondering why the positions of the load instruction (LI) and the jump instruction (J or JR) are swapped. This is due to a \ac{RISC} feature called ``branch delay slot''.

The reason this happens is a quirk in the architecture of some RISC \ac{ISA}s and isn't important for our
purposes---we must simply keep in mind that in MIPS, the instruction following a jump or branch instruction
is executed \IT{before} the jump/branch instruction itself.

As a consequence, branch instructions always swap places with the instruction executed immediately beforehand.


In practice, functions which merely return 1 (\IT{true}) or 0 (\IT{false}) are very frequent.

The smallest ever of the standard UNIX utilities, \IT{/bin/true} and \IT{/bin/false} return 0 and 1 respectively, as an exit code.
(Zero as an exit code usually means success, non-zero means error.)
}
\RU{\subsubsection{std::string}
\myindex{\Cpp!STL!std::string}
\label{std_string}

\myparagraph{Как устроена структура}

Многие строковые библиотеки \InSqBrackets{\CNotes 2.2} обеспечивают структуру содержащую ссылку 
на буфер собственно со строкой, переменная всегда содержащую длину строки 
(что очень удобно для массы функций \InSqBrackets{\CNotes 2.2.1}) и переменную содержащую текущий размер буфера.

Строка в буфере обыкновенно оканчивается нулем: это для того чтобы указатель на буфер можно было
передавать в функции требующие на вход обычную сишную \ac{ASCIIZ}-строку.

Стандарт \Cpp не описывает, как именно нужно реализовывать std::string,
но, как правило, они реализованы как описано выше, с небольшими дополнениями.

Строки в \Cpp это не класс (как, например, QString в Qt), а темплейт (basic\_string), 
это сделано для того чтобы поддерживать 
строки содержащие разного типа символы: как минимум \Tchar и \IT{wchar\_t}.

Так что, std::string это класс с базовым типом \Tchar.

А std::wstring это класс с базовым типом \IT{wchar\_t}.

\mysubparagraph{MSVC}

В реализации MSVC, вместо ссылки на буфер может содержаться сам буфер (если строка короче 16-и символов).

Это означает, что каждая короткая строка будет занимать в памяти по крайней мере $16 + 4 + 4 = 24$ 
байт для 32-битной среды либо $16 + 8 + 8 = 32$ 
байта в 64-битной, а если строка длиннее 16-и символов, то прибавьте еще длину самой строки.

\lstinputlisting[caption=пример для MSVC,style=customc]{\CURPATH/STL/string/MSVC_RU.cpp}

Собственно, из этого исходника почти всё ясно.

Несколько замечаний:

Если строка короче 16-и символов, 
то отдельный буфер для строки в \glslink{heap}{куче} выделяться не будет.

Это удобно потому что на практике, основная часть строк действительно короткие.
Вероятно, разработчики в Microsoft выбрали размер в 16 символов как разумный баланс.

Теперь очень важный момент в конце функции main(): мы не пользуемся методом c\_str(), тем не менее,
если это скомпилировать и запустить, то обе строки появятся в консоли!

Работает это вот почему.

В первом случае строка короче 16-и символов и в начале объекта std::string (его можно рассматривать
просто как структуру) расположен буфер с этой строкой.
\printf трактует указатель как указатель на массив символов оканчивающийся нулем и поэтому всё работает.

Вывод второй строки (длиннее 16-и символов) даже еще опаснее: это вообще типичная программистская ошибка 
(или опечатка), забыть дописать c\_str().
Это работает потому что в это время в начале структуры расположен указатель на буфер.
Это может надолго остаться незамеченным: до тех пока там не появится строка 
короче 16-и символов, тогда процесс упадет.

\mysubparagraph{GCC}

В реализации GCC в структуре есть еще одна переменная --- reference count.

Интересно, что указатель на экземпляр класса std::string в GCC указывает не на начало самой структуры, 
а на указатель на буфера.
В libstdc++-v3\textbackslash{}include\textbackslash{}bits\textbackslash{}basic\_string.h 
мы можем прочитать что это сделано для удобства отладки:

\begin{lstlisting}
   *  The reason you want _M_data pointing to the character %array and
   *  not the _Rep is so that the debugger can see the string
   *  contents. (Probably we should add a non-inline member to get
   *  the _Rep for the debugger to use, so users can check the actual
   *  string length.)
\end{lstlisting}

\href{http://go.yurichev.com/17085}{исходный код basic\_string.h}

В нашем примере мы учитываем это:

\lstinputlisting[caption=пример для GCC,style=customc]{\CURPATH/STL/string/GCC_RU.cpp}

Нужны еще небольшие хаки чтобы сымитировать типичную ошибку, которую мы уже видели выше, из-за
более ужесточенной проверки типов в GCC, тем не менее, printf() работает и здесь без c\_str().

\myparagraph{Чуть более сложный пример}

\lstinputlisting[style=customc]{\CURPATH/STL/string/3.cpp}

\lstinputlisting[caption=MSVC 2012,style=customasmx86]{\CURPATH/STL/string/3_MSVC_RU.asm}

Собственно, компилятор не конструирует строки статически: да в общем-то и как
это возможно, если буфер с ней нужно хранить в \glslink{heap}{куче}?

Вместо этого в сегменте данных хранятся обычные \ac{ASCIIZ}-строки, а позже, во время выполнения, 
при помощи метода \q{assign}, конструируются строки s1 и s2
.
При помощи \TT{operator+}, создается строка s3.

Обратите внимание на то что вызов метода c\_str() отсутствует,
потому что его код достаточно короткий и компилятор вставил его прямо здесь:
если строка короче 16-и байт, то в регистре EAX остается указатель на буфер,
а если длиннее, то из этого же места достается адрес на буфер расположенный в \glslink{heap}{куче}.

Далее следуют вызовы трех деструкторов, причем, они вызываются только если строка длиннее 16-и байт:
тогда нужно освободить буфера в \glslink{heap}{куче}.
В противном случае, так как все три объекта std::string хранятся в стеке,
они освобождаются автоматически после выхода из функции.

Следовательно, работа с короткими строками более быстрая из-за м\'{е}ньшего обращения к \glslink{heap}{куче}.

Код на GCC даже проще (из-за того, что в GCC, как мы уже видели, не реализована возможность хранить короткую
строку прямо в структуре):

% TODO1 comment each function meaning
\lstinputlisting[caption=GCC 4.8.1,style=customasmx86]{\CURPATH/STL/string/3_GCC_RU.s}

Можно заметить, что в деструкторы передается не указатель на объект,
а указатель на место за 12 байт (или 3 слова) перед ним, то есть, на настоящее начало структуры.

\myparagraph{std::string как глобальная переменная}
\label{sec:std_string_as_global_variable}

Опытные программисты на \Cpp знают, что глобальные переменные \ac{STL}-типов вполне можно объявлять.

Да, действительно:

\lstinputlisting[style=customc]{\CURPATH/STL/string/5.cpp}

Но как и где будет вызываться конструктор \TT{std::string}?

На самом деле, эта переменная будет инициализирована даже перед началом \main.

\lstinputlisting[caption=MSVC 2012: здесь конструируется глобальная переменная{,} а также регистрируется её деструктор,style=customasmx86]{\CURPATH/STL/string/5_MSVC_p2.asm}

\lstinputlisting[caption=MSVC 2012: здесь глобальная переменная используется в \main,style=customasmx86]{\CURPATH/STL/string/5_MSVC_p1.asm}

\lstinputlisting[caption=MSVC 2012: эта функция-деструктор вызывается перед выходом,style=customasmx86]{\CURPATH/STL/string/5_MSVC_p3.asm}

\myindex{\CStandardLibrary!atexit()}
В реальности, из \ac{CRT}, еще до вызова main(), вызывается специальная функция,
в которой перечислены все конструкторы подобных переменных.
Более того: при помощи atexit() регистрируется функция, которая будет вызвана в конце работы программы:
в этой функции компилятор собирает вызовы деструкторов всех подобных глобальных переменных.

GCC работает похожим образом:

\lstinputlisting[caption=GCC 4.8.1,style=customasmx86]{\CURPATH/STL/string/5_GCC.s}

Но он не выделяет отдельной функции в которой будут собраны деструкторы: 
каждый деструктор передается в atexit() по одному.

% TODO а если глобальная STL-переменная в другом модуле? надо проверить.

}
\DE{\subsection{Einfachste XOR-Verschlüsselung überhaupt}

Ich habe einmal eine Software gesehen, bei der alle Debugging-Ausgaben mit XOR mit dem Wert 3
verschlüsselt wurden. Mit anderen Worten, die beiden niedrigsten Bits aller Buchstaben wurden invertiert.

``Hello, world'' wurde zu ``Kfool/\#tlqog'':

\begin{lstlisting}
#!/usr/bin/python

msg="Hello, world!"

print "".join(map(lambda x: chr(ord(x)^3), msg))
\end{lstlisting}

Das ist eine ziemlich interessante Verschlüsselung (oder besser eine Verschleierung),
weil sie zwei wichtige Eigenschaften hat:
1) es ist eine einzige Funktion zum Verschlüsseln und entschlüsseln, sie muss nur wiederholt angewendet werden
2) die entstehenden Buchstaben befinden sich im druckbaren Bereich, also die ganze Zeichenkette kann ohne
Escape-Symbole im Code verwendet werden.

Die zweite Eigenschaft nutzt die Tatsache, dass alle druckbaren Zeichen in Reihen organisiert sind: 0x2x-0x7x,
und wenn die beiden niederwertigsten Bits invertiert werden, wird der Buchstabe um eine oder drei Stellen nach
links oder rechts \IT{verschoben}, aber niemals in eine andere Reihe:

\begin{figure}[H]
\centering
\includegraphics[width=0.7\textwidth]{ascii_clean.png}
\caption{7-Bit \ac{ASCII} Tabelle in Emacs}
\end{figure}

\dots mit dem Zeichen 0x7F als einziger Ausnahme.

Im Folgenden werden also beispielsweise die Zeichen A-Z \IT{verschlüsselt}:

\begin{lstlisting}
#!/usr/bin/python

msg="@ABCDEFGHIJKLMNO"

print "".join(map(lambda x: chr(ord(x)^3), msg))
\end{lstlisting}

Ergebnis:
% FIXME \verb  --  relevant comment for German?
\begin{lstlisting}
CBA@GFEDKJIHONML
\end{lstlisting}

Es sieht so aus als würden die Zeichen ``@'' und ``C'' sowie ``B'' und ``A'' vertauscht werden.

Hier ist noch ein interessantes Beispiel, in dem gezeigt wird, wie die Eigenschaften von XOR
ausgenutzt werden können: Exakt den gleichen Effekt, dass druckbare Zeichen auch druckbar bleiben,
kann man dadurch erzielen, dass irgendeine Kombination der niedrigsten vier Bits invertiert wird.
}

\EN{\section{Returning Values}
\label{ret_val_func}

Another simple function is the one that simply returns a constant value:

\lstinputlisting[caption=\EN{\CCpp Code},style=customc]{patterns/011_ret/1.c}

Let's compile it.

\subsection{x86}

Here's what both the GCC and MSVC compilers produce (with optimization) on the x86 platform:

\lstinputlisting[caption=\Optimizing GCC/MSVC (\assemblyOutput),style=customasmx86]{patterns/011_ret/1.s}

\myindex{x86!\Instructions!RET}
There are just two instructions: the first places the value 123 into the \EAX register,
which is used by convention for storing the return
value, and the second one is \RET, which returns execution to the \gls{caller}.

The caller will take the result from the \EAX register.

\subsection{ARM}

There are a few differences on the ARM platform:

\lstinputlisting[caption=\OptimizingKeilVI (\ARMMode) ASM Output,style=customasmARM]{patterns/011_ret/1_Keil_ARM_O3.s}

ARM uses the register \Reg{0} for returning the results of functions, so 123 is copied into \Reg{0}.

\myindex{ARM!\Instructions!MOV}
\myindex{x86!\Instructions!MOV}
It is worth noting that \MOV is a misleading name for the instruction in both the x86 and ARM \ac{ISA}s.

The data is not in fact \IT{moved}, but \IT{copied}.

\subsection{MIPS}

\label{MIPS_leaf_function_ex1}

The GCC assembly output below lists registers by number:

\lstinputlisting[caption=\Optimizing GCC 4.4.5 (\assemblyOutput),style=customasmMIPS]{patterns/011_ret/MIPS.s}

\dots while \IDA does it by their pseudo names:

\lstinputlisting[caption=\Optimizing GCC 4.4.5 (IDA),style=customasmMIPS]{patterns/011_ret/MIPS_IDA.lst}

The \$2 (or \$V0) register is used to store the function's return value.
\myindex{MIPS!\Pseudoinstructions!LI}
\INS{LI} stands for ``Load Immediate'' and is the MIPS equivalent to \MOV.

\myindex{MIPS!\Instructions!J}
The other instruction is the jump instruction (J or JR) which returns the execution flow to the \gls{caller}.

\myindex{MIPS!Branch delay slot}
You might be wondering why the positions of the load instruction (LI) and the jump instruction (J or JR) are swapped. This is due to a \ac{RISC} feature called ``branch delay slot''.

The reason this happens is a quirk in the architecture of some RISC \ac{ISA}s and isn't important for our
purposes---we must simply keep in mind that in MIPS, the instruction following a jump or branch instruction
is executed \IT{before} the jump/branch instruction itself.

As a consequence, branch instructions always swap places with the instruction executed immediately beforehand.


In practice, functions which merely return 1 (\IT{true}) or 0 (\IT{false}) are very frequent.

The smallest ever of the standard UNIX utilities, \IT{/bin/true} and \IT{/bin/false} return 0 and 1 respectively, as an exit code.
(Zero as an exit code usually means success, non-zero means error.)
}
\RU{\subsubsection{std::string}
\myindex{\Cpp!STL!std::string}
\label{std_string}

\myparagraph{Как устроена структура}

Многие строковые библиотеки \InSqBrackets{\CNotes 2.2} обеспечивают структуру содержащую ссылку 
на буфер собственно со строкой, переменная всегда содержащую длину строки 
(что очень удобно для массы функций \InSqBrackets{\CNotes 2.2.1}) и переменную содержащую текущий размер буфера.

Строка в буфере обыкновенно оканчивается нулем: это для того чтобы указатель на буфер можно было
передавать в функции требующие на вход обычную сишную \ac{ASCIIZ}-строку.

Стандарт \Cpp не описывает, как именно нужно реализовывать std::string,
но, как правило, они реализованы как описано выше, с небольшими дополнениями.

Строки в \Cpp это не класс (как, например, QString в Qt), а темплейт (basic\_string), 
это сделано для того чтобы поддерживать 
строки содержащие разного типа символы: как минимум \Tchar и \IT{wchar\_t}.

Так что, std::string это класс с базовым типом \Tchar.

А std::wstring это класс с базовым типом \IT{wchar\_t}.

\mysubparagraph{MSVC}

В реализации MSVC, вместо ссылки на буфер может содержаться сам буфер (если строка короче 16-и символов).

Это означает, что каждая короткая строка будет занимать в памяти по крайней мере $16 + 4 + 4 = 24$ 
байт для 32-битной среды либо $16 + 8 + 8 = 32$ 
байта в 64-битной, а если строка длиннее 16-и символов, то прибавьте еще длину самой строки.

\lstinputlisting[caption=пример для MSVC,style=customc]{\CURPATH/STL/string/MSVC_RU.cpp}

Собственно, из этого исходника почти всё ясно.

Несколько замечаний:

Если строка короче 16-и символов, 
то отдельный буфер для строки в \glslink{heap}{куче} выделяться не будет.

Это удобно потому что на практике, основная часть строк действительно короткие.
Вероятно, разработчики в Microsoft выбрали размер в 16 символов как разумный баланс.

Теперь очень важный момент в конце функции main(): мы не пользуемся методом c\_str(), тем не менее,
если это скомпилировать и запустить, то обе строки появятся в консоли!

Работает это вот почему.

В первом случае строка короче 16-и символов и в начале объекта std::string (его можно рассматривать
просто как структуру) расположен буфер с этой строкой.
\printf трактует указатель как указатель на массив символов оканчивающийся нулем и поэтому всё работает.

Вывод второй строки (длиннее 16-и символов) даже еще опаснее: это вообще типичная программистская ошибка 
(или опечатка), забыть дописать c\_str().
Это работает потому что в это время в начале структуры расположен указатель на буфер.
Это может надолго остаться незамеченным: до тех пока там не появится строка 
короче 16-и символов, тогда процесс упадет.

\mysubparagraph{GCC}

В реализации GCC в структуре есть еще одна переменная --- reference count.

Интересно, что указатель на экземпляр класса std::string в GCC указывает не на начало самой структуры, 
а на указатель на буфера.
В libstdc++-v3\textbackslash{}include\textbackslash{}bits\textbackslash{}basic\_string.h 
мы можем прочитать что это сделано для удобства отладки:

\begin{lstlisting}
   *  The reason you want _M_data pointing to the character %array and
   *  not the _Rep is so that the debugger can see the string
   *  contents. (Probably we should add a non-inline member to get
   *  the _Rep for the debugger to use, so users can check the actual
   *  string length.)
\end{lstlisting}

\href{http://go.yurichev.com/17085}{исходный код basic\_string.h}

В нашем примере мы учитываем это:

\lstinputlisting[caption=пример для GCC,style=customc]{\CURPATH/STL/string/GCC_RU.cpp}

Нужны еще небольшие хаки чтобы сымитировать типичную ошибку, которую мы уже видели выше, из-за
более ужесточенной проверки типов в GCC, тем не менее, printf() работает и здесь без c\_str().

\myparagraph{Чуть более сложный пример}

\lstinputlisting[style=customc]{\CURPATH/STL/string/3.cpp}

\lstinputlisting[caption=MSVC 2012,style=customasmx86]{\CURPATH/STL/string/3_MSVC_RU.asm}

Собственно, компилятор не конструирует строки статически: да в общем-то и как
это возможно, если буфер с ней нужно хранить в \glslink{heap}{куче}?

Вместо этого в сегменте данных хранятся обычные \ac{ASCIIZ}-строки, а позже, во время выполнения, 
при помощи метода \q{assign}, конструируются строки s1 и s2
.
При помощи \TT{operator+}, создается строка s3.

Обратите внимание на то что вызов метода c\_str() отсутствует,
потому что его код достаточно короткий и компилятор вставил его прямо здесь:
если строка короче 16-и байт, то в регистре EAX остается указатель на буфер,
а если длиннее, то из этого же места достается адрес на буфер расположенный в \glslink{heap}{куче}.

Далее следуют вызовы трех деструкторов, причем, они вызываются только если строка длиннее 16-и байт:
тогда нужно освободить буфера в \glslink{heap}{куче}.
В противном случае, так как все три объекта std::string хранятся в стеке,
они освобождаются автоматически после выхода из функции.

Следовательно, работа с короткими строками более быстрая из-за м\'{е}ньшего обращения к \glslink{heap}{куче}.

Код на GCC даже проще (из-за того, что в GCC, как мы уже видели, не реализована возможность хранить короткую
строку прямо в структуре):

% TODO1 comment each function meaning
\lstinputlisting[caption=GCC 4.8.1,style=customasmx86]{\CURPATH/STL/string/3_GCC_RU.s}

Можно заметить, что в деструкторы передается не указатель на объект,
а указатель на место за 12 байт (или 3 слова) перед ним, то есть, на настоящее начало структуры.

\myparagraph{std::string как глобальная переменная}
\label{sec:std_string_as_global_variable}

Опытные программисты на \Cpp знают, что глобальные переменные \ac{STL}-типов вполне можно объявлять.

Да, действительно:

\lstinputlisting[style=customc]{\CURPATH/STL/string/5.cpp}

Но как и где будет вызываться конструктор \TT{std::string}?

На самом деле, эта переменная будет инициализирована даже перед началом \main.

\lstinputlisting[caption=MSVC 2012: здесь конструируется глобальная переменная{,} а также регистрируется её деструктор,style=customasmx86]{\CURPATH/STL/string/5_MSVC_p2.asm}

\lstinputlisting[caption=MSVC 2012: здесь глобальная переменная используется в \main,style=customasmx86]{\CURPATH/STL/string/5_MSVC_p1.asm}

\lstinputlisting[caption=MSVC 2012: эта функция-деструктор вызывается перед выходом,style=customasmx86]{\CURPATH/STL/string/5_MSVC_p3.asm}

\myindex{\CStandardLibrary!atexit()}
В реальности, из \ac{CRT}, еще до вызова main(), вызывается специальная функция,
в которой перечислены все конструкторы подобных переменных.
Более того: при помощи atexit() регистрируется функция, которая будет вызвана в конце работы программы:
в этой функции компилятор собирает вызовы деструкторов всех подобных глобальных переменных.

GCC работает похожим образом:

\lstinputlisting[caption=GCC 4.8.1,style=customasmx86]{\CURPATH/STL/string/5_GCC.s}

Но он не выделяет отдельной функции в которой будут собраны деструкторы: 
каждый деструктор передается в atexit() по одному.

% TODO а если глобальная STL-переменная в другом модуле? надо проверить.

}

\EN{\section{Returning Values}
\label{ret_val_func}

Another simple function is the one that simply returns a constant value:

\lstinputlisting[caption=\EN{\CCpp Code},style=customc]{patterns/011_ret/1.c}

Let's compile it.

\subsection{x86}

Here's what both the GCC and MSVC compilers produce (with optimization) on the x86 platform:

\lstinputlisting[caption=\Optimizing GCC/MSVC (\assemblyOutput),style=customasmx86]{patterns/011_ret/1.s}

\myindex{x86!\Instructions!RET}
There are just two instructions: the first places the value 123 into the \EAX register,
which is used by convention for storing the return
value, and the second one is \RET, which returns execution to the \gls{caller}.

The caller will take the result from the \EAX register.

\subsection{ARM}

There are a few differences on the ARM platform:

\lstinputlisting[caption=\OptimizingKeilVI (\ARMMode) ASM Output,style=customasmARM]{patterns/011_ret/1_Keil_ARM_O3.s}

ARM uses the register \Reg{0} for returning the results of functions, so 123 is copied into \Reg{0}.

\myindex{ARM!\Instructions!MOV}
\myindex{x86!\Instructions!MOV}
It is worth noting that \MOV is a misleading name for the instruction in both the x86 and ARM \ac{ISA}s.

The data is not in fact \IT{moved}, but \IT{copied}.

\subsection{MIPS}

\label{MIPS_leaf_function_ex1}

The GCC assembly output below lists registers by number:

\lstinputlisting[caption=\Optimizing GCC 4.4.5 (\assemblyOutput),style=customasmMIPS]{patterns/011_ret/MIPS.s}

\dots while \IDA does it by their pseudo names:

\lstinputlisting[caption=\Optimizing GCC 4.4.5 (IDA),style=customasmMIPS]{patterns/011_ret/MIPS_IDA.lst}

The \$2 (or \$V0) register is used to store the function's return value.
\myindex{MIPS!\Pseudoinstructions!LI}
\INS{LI} stands for ``Load Immediate'' and is the MIPS equivalent to \MOV.

\myindex{MIPS!\Instructions!J}
The other instruction is the jump instruction (J or JR) which returns the execution flow to the \gls{caller}.

\myindex{MIPS!Branch delay slot}
You might be wondering why the positions of the load instruction (LI) and the jump instruction (J or JR) are swapped. This is due to a \ac{RISC} feature called ``branch delay slot''.

The reason this happens is a quirk in the architecture of some RISC \ac{ISA}s and isn't important for our
purposes---we must simply keep in mind that in MIPS, the instruction following a jump or branch instruction
is executed \IT{before} the jump/branch instruction itself.

As a consequence, branch instructions always swap places with the instruction executed immediately beforehand.


In practice, functions which merely return 1 (\IT{true}) or 0 (\IT{false}) are very frequent.

The smallest ever of the standard UNIX utilities, \IT{/bin/true} and \IT{/bin/false} return 0 and 1 respectively, as an exit code.
(Zero as an exit code usually means success, non-zero means error.)
}
\RU{\subsubsection{std::string}
\myindex{\Cpp!STL!std::string}
\label{std_string}

\myparagraph{Как устроена структура}

Многие строковые библиотеки \InSqBrackets{\CNotes 2.2} обеспечивают структуру содержащую ссылку 
на буфер собственно со строкой, переменная всегда содержащую длину строки 
(что очень удобно для массы функций \InSqBrackets{\CNotes 2.2.1}) и переменную содержащую текущий размер буфера.

Строка в буфере обыкновенно оканчивается нулем: это для того чтобы указатель на буфер можно было
передавать в функции требующие на вход обычную сишную \ac{ASCIIZ}-строку.

Стандарт \Cpp не описывает, как именно нужно реализовывать std::string,
но, как правило, они реализованы как описано выше, с небольшими дополнениями.

Строки в \Cpp это не класс (как, например, QString в Qt), а темплейт (basic\_string), 
это сделано для того чтобы поддерживать 
строки содержащие разного типа символы: как минимум \Tchar и \IT{wchar\_t}.

Так что, std::string это класс с базовым типом \Tchar.

А std::wstring это класс с базовым типом \IT{wchar\_t}.

\mysubparagraph{MSVC}

В реализации MSVC, вместо ссылки на буфер может содержаться сам буфер (если строка короче 16-и символов).

Это означает, что каждая короткая строка будет занимать в памяти по крайней мере $16 + 4 + 4 = 24$ 
байт для 32-битной среды либо $16 + 8 + 8 = 32$ 
байта в 64-битной, а если строка длиннее 16-и символов, то прибавьте еще длину самой строки.

\lstinputlisting[caption=пример для MSVC,style=customc]{\CURPATH/STL/string/MSVC_RU.cpp}

Собственно, из этого исходника почти всё ясно.

Несколько замечаний:

Если строка короче 16-и символов, 
то отдельный буфер для строки в \glslink{heap}{куче} выделяться не будет.

Это удобно потому что на практике, основная часть строк действительно короткие.
Вероятно, разработчики в Microsoft выбрали размер в 16 символов как разумный баланс.

Теперь очень важный момент в конце функции main(): мы не пользуемся методом c\_str(), тем не менее,
если это скомпилировать и запустить, то обе строки появятся в консоли!

Работает это вот почему.

В первом случае строка короче 16-и символов и в начале объекта std::string (его можно рассматривать
просто как структуру) расположен буфер с этой строкой.
\printf трактует указатель как указатель на массив символов оканчивающийся нулем и поэтому всё работает.

Вывод второй строки (длиннее 16-и символов) даже еще опаснее: это вообще типичная программистская ошибка 
(или опечатка), забыть дописать c\_str().
Это работает потому что в это время в начале структуры расположен указатель на буфер.
Это может надолго остаться незамеченным: до тех пока там не появится строка 
короче 16-и символов, тогда процесс упадет.

\mysubparagraph{GCC}

В реализации GCC в структуре есть еще одна переменная --- reference count.

Интересно, что указатель на экземпляр класса std::string в GCC указывает не на начало самой структуры, 
а на указатель на буфера.
В libstdc++-v3\textbackslash{}include\textbackslash{}bits\textbackslash{}basic\_string.h 
мы можем прочитать что это сделано для удобства отладки:

\begin{lstlisting}
   *  The reason you want _M_data pointing to the character %array and
   *  not the _Rep is so that the debugger can see the string
   *  contents. (Probably we should add a non-inline member to get
   *  the _Rep for the debugger to use, so users can check the actual
   *  string length.)
\end{lstlisting}

\href{http://go.yurichev.com/17085}{исходный код basic\_string.h}

В нашем примере мы учитываем это:

\lstinputlisting[caption=пример для GCC,style=customc]{\CURPATH/STL/string/GCC_RU.cpp}

Нужны еще небольшие хаки чтобы сымитировать типичную ошибку, которую мы уже видели выше, из-за
более ужесточенной проверки типов в GCC, тем не менее, printf() работает и здесь без c\_str().

\myparagraph{Чуть более сложный пример}

\lstinputlisting[style=customc]{\CURPATH/STL/string/3.cpp}

\lstinputlisting[caption=MSVC 2012,style=customasmx86]{\CURPATH/STL/string/3_MSVC_RU.asm}

Собственно, компилятор не конструирует строки статически: да в общем-то и как
это возможно, если буфер с ней нужно хранить в \glslink{heap}{куче}?

Вместо этого в сегменте данных хранятся обычные \ac{ASCIIZ}-строки, а позже, во время выполнения, 
при помощи метода \q{assign}, конструируются строки s1 и s2
.
При помощи \TT{operator+}, создается строка s3.

Обратите внимание на то что вызов метода c\_str() отсутствует,
потому что его код достаточно короткий и компилятор вставил его прямо здесь:
если строка короче 16-и байт, то в регистре EAX остается указатель на буфер,
а если длиннее, то из этого же места достается адрес на буфер расположенный в \glslink{heap}{куче}.

Далее следуют вызовы трех деструкторов, причем, они вызываются только если строка длиннее 16-и байт:
тогда нужно освободить буфера в \glslink{heap}{куче}.
В противном случае, так как все три объекта std::string хранятся в стеке,
они освобождаются автоматически после выхода из функции.

Следовательно, работа с короткими строками более быстрая из-за м\'{е}ньшего обращения к \glslink{heap}{куче}.

Код на GCC даже проще (из-за того, что в GCC, как мы уже видели, не реализована возможность хранить короткую
строку прямо в структуре):

% TODO1 comment each function meaning
\lstinputlisting[caption=GCC 4.8.1,style=customasmx86]{\CURPATH/STL/string/3_GCC_RU.s}

Можно заметить, что в деструкторы передается не указатель на объект,
а указатель на место за 12 байт (или 3 слова) перед ним, то есть, на настоящее начало структуры.

\myparagraph{std::string как глобальная переменная}
\label{sec:std_string_as_global_variable}

Опытные программисты на \Cpp знают, что глобальные переменные \ac{STL}-типов вполне можно объявлять.

Да, действительно:

\lstinputlisting[style=customc]{\CURPATH/STL/string/5.cpp}

Но как и где будет вызываться конструктор \TT{std::string}?

На самом деле, эта переменная будет инициализирована даже перед началом \main.

\lstinputlisting[caption=MSVC 2012: здесь конструируется глобальная переменная{,} а также регистрируется её деструктор,style=customasmx86]{\CURPATH/STL/string/5_MSVC_p2.asm}

\lstinputlisting[caption=MSVC 2012: здесь глобальная переменная используется в \main,style=customasmx86]{\CURPATH/STL/string/5_MSVC_p1.asm}

\lstinputlisting[caption=MSVC 2012: эта функция-деструктор вызывается перед выходом,style=customasmx86]{\CURPATH/STL/string/5_MSVC_p3.asm}

\myindex{\CStandardLibrary!atexit()}
В реальности, из \ac{CRT}, еще до вызова main(), вызывается специальная функция,
в которой перечислены все конструкторы подобных переменных.
Более того: при помощи atexit() регистрируется функция, которая будет вызвана в конце работы программы:
в этой функции компилятор собирает вызовы деструкторов всех подобных глобальных переменных.

GCC работает похожим образом:

\lstinputlisting[caption=GCC 4.8.1,style=customasmx86]{\CURPATH/STL/string/5_GCC.s}

Но он не выделяет отдельной функции в которой будут собраны деструкторы: 
каждый деструктор передается в atexit() по одному.

% TODO а если глобальная STL-переменная в другом модуле? надо проверить.

}
\DE{\subsection{Einfachste XOR-Verschlüsselung überhaupt}

Ich habe einmal eine Software gesehen, bei der alle Debugging-Ausgaben mit XOR mit dem Wert 3
verschlüsselt wurden. Mit anderen Worten, die beiden niedrigsten Bits aller Buchstaben wurden invertiert.

``Hello, world'' wurde zu ``Kfool/\#tlqog'':

\begin{lstlisting}
#!/usr/bin/python

msg="Hello, world!"

print "".join(map(lambda x: chr(ord(x)^3), msg))
\end{lstlisting}

Das ist eine ziemlich interessante Verschlüsselung (oder besser eine Verschleierung),
weil sie zwei wichtige Eigenschaften hat:
1) es ist eine einzige Funktion zum Verschlüsseln und entschlüsseln, sie muss nur wiederholt angewendet werden
2) die entstehenden Buchstaben befinden sich im druckbaren Bereich, also die ganze Zeichenkette kann ohne
Escape-Symbole im Code verwendet werden.

Die zweite Eigenschaft nutzt die Tatsache, dass alle druckbaren Zeichen in Reihen organisiert sind: 0x2x-0x7x,
und wenn die beiden niederwertigsten Bits invertiert werden, wird der Buchstabe um eine oder drei Stellen nach
links oder rechts \IT{verschoben}, aber niemals in eine andere Reihe:

\begin{figure}[H]
\centering
\includegraphics[width=0.7\textwidth]{ascii_clean.png}
\caption{7-Bit \ac{ASCII} Tabelle in Emacs}
\end{figure}

\dots mit dem Zeichen 0x7F als einziger Ausnahme.

Im Folgenden werden also beispielsweise die Zeichen A-Z \IT{verschlüsselt}:

\begin{lstlisting}
#!/usr/bin/python

msg="@ABCDEFGHIJKLMNO"

print "".join(map(lambda x: chr(ord(x)^3), msg))
\end{lstlisting}

Ergebnis:
% FIXME \verb  --  relevant comment for German?
\begin{lstlisting}
CBA@GFEDKJIHONML
\end{lstlisting}

Es sieht so aus als würden die Zeichen ``@'' und ``C'' sowie ``B'' und ``A'' vertauscht werden.

Hier ist noch ein interessantes Beispiel, in dem gezeigt wird, wie die Eigenschaften von XOR
ausgenutzt werden können: Exakt den gleichen Effekt, dass druckbare Zeichen auch druckbar bleiben,
kann man dadurch erzielen, dass irgendeine Kombination der niedrigsten vier Bits invertiert wird.
}

\EN{\section{Returning Values}
\label{ret_val_func}

Another simple function is the one that simply returns a constant value:

\lstinputlisting[caption=\EN{\CCpp Code},style=customc]{patterns/011_ret/1.c}

Let's compile it.

\subsection{x86}

Here's what both the GCC and MSVC compilers produce (with optimization) on the x86 platform:

\lstinputlisting[caption=\Optimizing GCC/MSVC (\assemblyOutput),style=customasmx86]{patterns/011_ret/1.s}

\myindex{x86!\Instructions!RET}
There are just two instructions: the first places the value 123 into the \EAX register,
which is used by convention for storing the return
value, and the second one is \RET, which returns execution to the \gls{caller}.

The caller will take the result from the \EAX register.

\subsection{ARM}

There are a few differences on the ARM platform:

\lstinputlisting[caption=\OptimizingKeilVI (\ARMMode) ASM Output,style=customasmARM]{patterns/011_ret/1_Keil_ARM_O3.s}

ARM uses the register \Reg{0} for returning the results of functions, so 123 is copied into \Reg{0}.

\myindex{ARM!\Instructions!MOV}
\myindex{x86!\Instructions!MOV}
It is worth noting that \MOV is a misleading name for the instruction in both the x86 and ARM \ac{ISA}s.

The data is not in fact \IT{moved}, but \IT{copied}.

\subsection{MIPS}

\label{MIPS_leaf_function_ex1}

The GCC assembly output below lists registers by number:

\lstinputlisting[caption=\Optimizing GCC 4.4.5 (\assemblyOutput),style=customasmMIPS]{patterns/011_ret/MIPS.s}

\dots while \IDA does it by their pseudo names:

\lstinputlisting[caption=\Optimizing GCC 4.4.5 (IDA),style=customasmMIPS]{patterns/011_ret/MIPS_IDA.lst}

The \$2 (or \$V0) register is used to store the function's return value.
\myindex{MIPS!\Pseudoinstructions!LI}
\INS{LI} stands for ``Load Immediate'' and is the MIPS equivalent to \MOV.

\myindex{MIPS!\Instructions!J}
The other instruction is the jump instruction (J or JR) which returns the execution flow to the \gls{caller}.

\myindex{MIPS!Branch delay slot}
You might be wondering why the positions of the load instruction (LI) and the jump instruction (J or JR) are swapped. This is due to a \ac{RISC} feature called ``branch delay slot''.

The reason this happens is a quirk in the architecture of some RISC \ac{ISA}s and isn't important for our
purposes---we must simply keep in mind that in MIPS, the instruction following a jump or branch instruction
is executed \IT{before} the jump/branch instruction itself.

As a consequence, branch instructions always swap places with the instruction executed immediately beforehand.


In practice, functions which merely return 1 (\IT{true}) or 0 (\IT{false}) are very frequent.

The smallest ever of the standard UNIX utilities, \IT{/bin/true} and \IT{/bin/false} return 0 and 1 respectively, as an exit code.
(Zero as an exit code usually means success, non-zero means error.)
}
\RU{\subsubsection{std::string}
\myindex{\Cpp!STL!std::string}
\label{std_string}

\myparagraph{Как устроена структура}

Многие строковые библиотеки \InSqBrackets{\CNotes 2.2} обеспечивают структуру содержащую ссылку 
на буфер собственно со строкой, переменная всегда содержащую длину строки 
(что очень удобно для массы функций \InSqBrackets{\CNotes 2.2.1}) и переменную содержащую текущий размер буфера.

Строка в буфере обыкновенно оканчивается нулем: это для того чтобы указатель на буфер можно было
передавать в функции требующие на вход обычную сишную \ac{ASCIIZ}-строку.

Стандарт \Cpp не описывает, как именно нужно реализовывать std::string,
но, как правило, они реализованы как описано выше, с небольшими дополнениями.

Строки в \Cpp это не класс (как, например, QString в Qt), а темплейт (basic\_string), 
это сделано для того чтобы поддерживать 
строки содержащие разного типа символы: как минимум \Tchar и \IT{wchar\_t}.

Так что, std::string это класс с базовым типом \Tchar.

А std::wstring это класс с базовым типом \IT{wchar\_t}.

\mysubparagraph{MSVC}

В реализации MSVC, вместо ссылки на буфер может содержаться сам буфер (если строка короче 16-и символов).

Это означает, что каждая короткая строка будет занимать в памяти по крайней мере $16 + 4 + 4 = 24$ 
байт для 32-битной среды либо $16 + 8 + 8 = 32$ 
байта в 64-битной, а если строка длиннее 16-и символов, то прибавьте еще длину самой строки.

\lstinputlisting[caption=пример для MSVC,style=customc]{\CURPATH/STL/string/MSVC_RU.cpp}

Собственно, из этого исходника почти всё ясно.

Несколько замечаний:

Если строка короче 16-и символов, 
то отдельный буфер для строки в \glslink{heap}{куче} выделяться не будет.

Это удобно потому что на практике, основная часть строк действительно короткие.
Вероятно, разработчики в Microsoft выбрали размер в 16 символов как разумный баланс.

Теперь очень важный момент в конце функции main(): мы не пользуемся методом c\_str(), тем не менее,
если это скомпилировать и запустить, то обе строки появятся в консоли!

Работает это вот почему.

В первом случае строка короче 16-и символов и в начале объекта std::string (его можно рассматривать
просто как структуру) расположен буфер с этой строкой.
\printf трактует указатель как указатель на массив символов оканчивающийся нулем и поэтому всё работает.

Вывод второй строки (длиннее 16-и символов) даже еще опаснее: это вообще типичная программистская ошибка 
(или опечатка), забыть дописать c\_str().
Это работает потому что в это время в начале структуры расположен указатель на буфер.
Это может надолго остаться незамеченным: до тех пока там не появится строка 
короче 16-и символов, тогда процесс упадет.

\mysubparagraph{GCC}

В реализации GCC в структуре есть еще одна переменная --- reference count.

Интересно, что указатель на экземпляр класса std::string в GCC указывает не на начало самой структуры, 
а на указатель на буфера.
В libstdc++-v3\textbackslash{}include\textbackslash{}bits\textbackslash{}basic\_string.h 
мы можем прочитать что это сделано для удобства отладки:

\begin{lstlisting}
   *  The reason you want _M_data pointing to the character %array and
   *  not the _Rep is so that the debugger can see the string
   *  contents. (Probably we should add a non-inline member to get
   *  the _Rep for the debugger to use, so users can check the actual
   *  string length.)
\end{lstlisting}

\href{http://go.yurichev.com/17085}{исходный код basic\_string.h}

В нашем примере мы учитываем это:

\lstinputlisting[caption=пример для GCC,style=customc]{\CURPATH/STL/string/GCC_RU.cpp}

Нужны еще небольшие хаки чтобы сымитировать типичную ошибку, которую мы уже видели выше, из-за
более ужесточенной проверки типов в GCC, тем не менее, printf() работает и здесь без c\_str().

\myparagraph{Чуть более сложный пример}

\lstinputlisting[style=customc]{\CURPATH/STL/string/3.cpp}

\lstinputlisting[caption=MSVC 2012,style=customasmx86]{\CURPATH/STL/string/3_MSVC_RU.asm}

Собственно, компилятор не конструирует строки статически: да в общем-то и как
это возможно, если буфер с ней нужно хранить в \glslink{heap}{куче}?

Вместо этого в сегменте данных хранятся обычные \ac{ASCIIZ}-строки, а позже, во время выполнения, 
при помощи метода \q{assign}, конструируются строки s1 и s2
.
При помощи \TT{operator+}, создается строка s3.

Обратите внимание на то что вызов метода c\_str() отсутствует,
потому что его код достаточно короткий и компилятор вставил его прямо здесь:
если строка короче 16-и байт, то в регистре EAX остается указатель на буфер,
а если длиннее, то из этого же места достается адрес на буфер расположенный в \glslink{heap}{куче}.

Далее следуют вызовы трех деструкторов, причем, они вызываются только если строка длиннее 16-и байт:
тогда нужно освободить буфера в \glslink{heap}{куче}.
В противном случае, так как все три объекта std::string хранятся в стеке,
они освобождаются автоматически после выхода из функции.

Следовательно, работа с короткими строками более быстрая из-за м\'{е}ньшего обращения к \glslink{heap}{куче}.

Код на GCC даже проще (из-за того, что в GCC, как мы уже видели, не реализована возможность хранить короткую
строку прямо в структуре):

% TODO1 comment each function meaning
\lstinputlisting[caption=GCC 4.8.1,style=customasmx86]{\CURPATH/STL/string/3_GCC_RU.s}

Можно заметить, что в деструкторы передается не указатель на объект,
а указатель на место за 12 байт (или 3 слова) перед ним, то есть, на настоящее начало структуры.

\myparagraph{std::string как глобальная переменная}
\label{sec:std_string_as_global_variable}

Опытные программисты на \Cpp знают, что глобальные переменные \ac{STL}-типов вполне можно объявлять.

Да, действительно:

\lstinputlisting[style=customc]{\CURPATH/STL/string/5.cpp}

Но как и где будет вызываться конструктор \TT{std::string}?

На самом деле, эта переменная будет инициализирована даже перед началом \main.

\lstinputlisting[caption=MSVC 2012: здесь конструируется глобальная переменная{,} а также регистрируется её деструктор,style=customasmx86]{\CURPATH/STL/string/5_MSVC_p2.asm}

\lstinputlisting[caption=MSVC 2012: здесь глобальная переменная используется в \main,style=customasmx86]{\CURPATH/STL/string/5_MSVC_p1.asm}

\lstinputlisting[caption=MSVC 2012: эта функция-деструктор вызывается перед выходом,style=customasmx86]{\CURPATH/STL/string/5_MSVC_p3.asm}

\myindex{\CStandardLibrary!atexit()}
В реальности, из \ac{CRT}, еще до вызова main(), вызывается специальная функция,
в которой перечислены все конструкторы подобных переменных.
Более того: при помощи atexit() регистрируется функция, которая будет вызвана в конце работы программы:
в этой функции компилятор собирает вызовы деструкторов всех подобных глобальных переменных.

GCC работает похожим образом:

\lstinputlisting[caption=GCC 4.8.1,style=customasmx86]{\CURPATH/STL/string/5_GCC.s}

Но он не выделяет отдельной функции в которой будут собраны деструкторы: 
каждый деструктор передается в atexit() по одному.

% TODO а если глобальная STL-переменная в другом модуле? надо проверить.

}
\DE{\subsection{Einfachste XOR-Verschlüsselung überhaupt}

Ich habe einmal eine Software gesehen, bei der alle Debugging-Ausgaben mit XOR mit dem Wert 3
verschlüsselt wurden. Mit anderen Worten, die beiden niedrigsten Bits aller Buchstaben wurden invertiert.

``Hello, world'' wurde zu ``Kfool/\#tlqog'':

\begin{lstlisting}
#!/usr/bin/python

msg="Hello, world!"

print "".join(map(lambda x: chr(ord(x)^3), msg))
\end{lstlisting}

Das ist eine ziemlich interessante Verschlüsselung (oder besser eine Verschleierung),
weil sie zwei wichtige Eigenschaften hat:
1) es ist eine einzige Funktion zum Verschlüsseln und entschlüsseln, sie muss nur wiederholt angewendet werden
2) die entstehenden Buchstaben befinden sich im druckbaren Bereich, also die ganze Zeichenkette kann ohne
Escape-Symbole im Code verwendet werden.

Die zweite Eigenschaft nutzt die Tatsache, dass alle druckbaren Zeichen in Reihen organisiert sind: 0x2x-0x7x,
und wenn die beiden niederwertigsten Bits invertiert werden, wird der Buchstabe um eine oder drei Stellen nach
links oder rechts \IT{verschoben}, aber niemals in eine andere Reihe:

\begin{figure}[H]
\centering
\includegraphics[width=0.7\textwidth]{ascii_clean.png}
\caption{7-Bit \ac{ASCII} Tabelle in Emacs}
\end{figure}

\dots mit dem Zeichen 0x7F als einziger Ausnahme.

Im Folgenden werden also beispielsweise die Zeichen A-Z \IT{verschlüsselt}:

\begin{lstlisting}
#!/usr/bin/python

msg="@ABCDEFGHIJKLMNO"

print "".join(map(lambda x: chr(ord(x)^3), msg))
\end{lstlisting}

Ergebnis:
% FIXME \verb  --  relevant comment for German?
\begin{lstlisting}
CBA@GFEDKJIHONML
\end{lstlisting}

Es sieht so aus als würden die Zeichen ``@'' und ``C'' sowie ``B'' und ``A'' vertauscht werden.

Hier ist noch ein interessantes Beispiel, in dem gezeigt wird, wie die Eigenschaften von XOR
ausgenutzt werden können: Exakt den gleichen Effekt, dass druckbare Zeichen auch druckbar bleiben,
kann man dadurch erzielen, dass irgendeine Kombination der niedrigsten vier Bits invertiert wird.
}

\ifdefined\SPANISH
\chapter{Patrones de código}
\fi % SPANISH

\ifdefined\GERMAN
\chapter{Code-Muster}
\fi % GERMAN

\ifdefined\ENGLISH
\chapter{Code Patterns}
\fi % ENGLISH

\ifdefined\ITALIAN
\chapter{Forme di codice}
\fi % ITALIAN

\ifdefined\RUSSIAN
\chapter{Образцы кода}
\fi % RUSSIAN

\ifdefined\BRAZILIAN
\chapter{Padrões de códigos}
\fi % BRAZILIAN

\ifdefined\THAI
\chapter{รูปแบบของโค้ด}
\fi % THAI

\ifdefined\FRENCH
\chapter{Modèle de code}
\fi % FRENCH

\ifdefined\POLISH
\chapter{\PLph{}}
\fi % POLISH

% sections
\EN{\input{patterns/patterns_opt_dbg_EN}}
\ES{\input{patterns/patterns_opt_dbg_ES}}
\ITA{\input{patterns/patterns_opt_dbg_ITA}}
\PTBR{\input{patterns/patterns_opt_dbg_PTBR}}
\RU{\input{patterns/patterns_opt_dbg_RU}}
\THA{\input{patterns/patterns_opt_dbg_THA}}
\DE{\input{patterns/patterns_opt_dbg_DE}}
\FR{\input{patterns/patterns_opt_dbg_FR}}
\PL{\input{patterns/patterns_opt_dbg_PL}}

\RU{\section{Некоторые базовые понятия}}
\EN{\section{Some basics}}
\DE{\section{Einige Grundlagen}}
\FR{\section{Quelques bases}}
\ES{\section{\ESph{}}}
\ITA{\section{Alcune basi teoriche}}
\PTBR{\section{\PTBRph{}}}
\THA{\section{\THAph{}}}
\PL{\section{\PLph{}}}

% sections:
\EN{\input{patterns/intro_CPU_ISA_EN}}
\ES{\input{patterns/intro_CPU_ISA_ES}}
\ITA{\input{patterns/intro_CPU_ISA_ITA}}
\PTBR{\input{patterns/intro_CPU_ISA_PTBR}}
\RU{\input{patterns/intro_CPU_ISA_RU}}
\DE{\input{patterns/intro_CPU_ISA_DE}}
\FR{\input{patterns/intro_CPU_ISA_FR}}
\PL{\input{patterns/intro_CPU_ISA_PL}}

\EN{\input{patterns/numeral_EN}}
\RU{\input{patterns/numeral_RU}}
\ITA{\input{patterns/numeral_ITA}}
\DE{\input{patterns/numeral_DE}}
\FR{\input{patterns/numeral_FR}}
\PL{\input{patterns/numeral_PL}}

% chapters
\input{patterns/00_empty/main}
\input{patterns/011_ret/main}
\input{patterns/01_helloworld/main}
\input{patterns/015_prolog_epilogue/main}
\input{patterns/02_stack/main}
\input{patterns/03_printf/main}
\input{patterns/04_scanf/main}
\input{patterns/05_passing_arguments/main}
\input{patterns/06_return_results/main}
\input{patterns/061_pointers/main}
\input{patterns/065_GOTO/main}
\input{patterns/07_jcc/main}
\input{patterns/08_switch/main}
\input{patterns/09_loops/main}
\input{patterns/10_strings/main}
\input{patterns/11_arith_optimizations/main}
\input{patterns/12_FPU/main}
\input{patterns/13_arrays/main}
\input{patterns/14_bitfields/main}
\EN{\input{patterns/145_LCG/main_EN}}
\RU{\input{patterns/145_LCG/main_RU}}
\input{patterns/15_structs/main}
\input{patterns/17_unions/main}
\input{patterns/18_pointers_to_functions/main}
\input{patterns/185_64bit_in_32_env/main}

\EN{\input{patterns/19_SIMD/main_EN}}
\RU{\input{patterns/19_SIMD/main_RU}}
\DE{\input{patterns/19_SIMD/main_DE}}

\EN{\input{patterns/20_x64/main_EN}}
\RU{\input{patterns/20_x64/main_RU}}

\EN{\input{patterns/205_floating_SIMD/main_EN}}
\RU{\input{patterns/205_floating_SIMD/main_RU}}
\DE{\input{patterns/205_floating_SIMD/main_DE}}

\EN{\input{patterns/ARM/main_EN}}
\RU{\input{patterns/ARM/main_RU}}
\DE{\input{patterns/ARM/main_DE}}

\input{patterns/MIPS/main}


\ifdefined\SPANISH
\chapter{Patrones de código}
\fi % SPANISH

\ifdefined\GERMAN
\chapter{Code-Muster}
\fi % GERMAN

\ifdefined\ENGLISH
\chapter{Code Patterns}
\fi % ENGLISH

\ifdefined\ITALIAN
\chapter{Forme di codice}
\fi % ITALIAN

\ifdefined\RUSSIAN
\chapter{Образцы кода}
\fi % RUSSIAN

\ifdefined\BRAZILIAN
\chapter{Padrões de códigos}
\fi % BRAZILIAN

\ifdefined\THAI
\chapter{รูปแบบของโค้ด}
\fi % THAI

\ifdefined\FRENCH
\chapter{Modèle de code}
\fi % FRENCH

\ifdefined\POLISH
\chapter{\PLph{}}
\fi % POLISH

% sections
\EN{\section{The method}

When the author of this book first started learning C and, later, \Cpp, he used to write small pieces of code, compile them,
and then look at the assembly language output. This made it very easy for him to understand what was going on in the code that he had written.
\footnote{In fact, he still does this when he can't understand what a particular bit of code does.}.
He did this so many times that the relationship between the \CCpp code and what the compiler produced was imprinted deeply in his mind.
It's now easy for him to imagine instantly a rough outline of a C code's appearance and function.
Perhaps this technique could be helpful for others.

%There are a lot of examples for both x86/x64 and ARM.
%Those who already familiar with one of architectures, may freely skim over pages.

By the way, there is a great website where you can do the same, with various compilers, instead of installing them on your box.
You can use it as well: \url{https://gcc.godbolt.org/}.

\section*{\Exercises}

When the author of this book studied assembly language, he also often compiled small C functions and then rewrote
them gradually to assembly, trying to make their code as short as possible.
This probably is not worth doing in real-world scenarios today,
because it's hard to compete with the latest compilers in terms of efficiency. It is, however, a very good way to gain a better understanding of assembly.
Feel free, therefore, to take any assembly code from this book and try to make it shorter.
However, don't forget to test what you have written.

% rewrote to show that debug\release and optimisations levels are orthogonal concepts.
\section*{Optimization levels and debug information}

Source code can be compiled by different compilers with various optimization levels.
A typical compiler has about three such levels, where level zero means that optimization is completely disabled.
Optimization can also be targeted towards code size or code speed.
A non-optimizing compiler is faster and produces more understandable (albeit verbose) code,
whereas an optimizing compiler is slower and tries to produce code that runs faster (but is not necessarily more compact).
In addition to optimization levels, a compiler can include some debug information in the resulting file,
producing code that is easy to debug.
One of the important features of the ´debug' code is that it might contain links
between each line of the source code and its respective machine code address.
Optimizing compilers, on the other hand, tend to produce output where entire lines of source code
can be optimized away and thus not even be present in the resulting machine code.
Reverse engineers can encounter either version, simply because some developers turn on the compiler's optimization flags and others do not.
Because of this, we'll try to work on examples of both debug and release versions of the code featured in this book, wherever possible.

Sometimes some pretty ancient compilers are used in this book, in order to get the shortest (or simplest) possible code snippet.
}
\ES{\input{patterns/patterns_opt_dbg_ES}}
\ITA{\input{patterns/patterns_opt_dbg_ITA}}
\PTBR{\input{patterns/patterns_opt_dbg_PTBR}}
\RU{\input{patterns/patterns_opt_dbg_RU}}
\THA{\input{patterns/patterns_opt_dbg_THA}}
\DE{\input{patterns/patterns_opt_dbg_DE}}
\FR{\input{patterns/patterns_opt_dbg_FR}}
\PL{\input{patterns/patterns_opt_dbg_PL}}

\RU{\section{Некоторые базовые понятия}}
\EN{\section{Some basics}}
\DE{\section{Einige Grundlagen}}
\FR{\section{Quelques bases}}
\ES{\section{\ESph{}}}
\ITA{\section{Alcune basi teoriche}}
\PTBR{\section{\PTBRph{}}}
\THA{\section{\THAph{}}}
\PL{\section{\PLph{}}}

% sections:
\EN{\input{patterns/intro_CPU_ISA_EN}}
\ES{\input{patterns/intro_CPU_ISA_ES}}
\ITA{\input{patterns/intro_CPU_ISA_ITA}}
\PTBR{\input{patterns/intro_CPU_ISA_PTBR}}
\RU{\input{patterns/intro_CPU_ISA_RU}}
\DE{\input{patterns/intro_CPU_ISA_DE}}
\FR{\input{patterns/intro_CPU_ISA_FR}}
\PL{\input{patterns/intro_CPU_ISA_PL}}

\EN{\subsection{Numeral Systems}

Humans have become accustomed to a decimal numeral system, probably because almost everyone has 10 fingers.
Nevertheless, the number \q{10} has no significant meaning in science and mathematics.
The natural numeral system in digital electronics is binary: 0 is for an absence of current in the wire, and 1 for presence.
10 in binary is 2 in decimal, 100 in binary is 4 in decimal, and so on.

% This sentence is a bit unweildy - maybe try 'Our ten-digit system would be described as having a radix...' - Renaissance
If the numeral system has 10 digits, it has a \IT{radix} (or \IT{base}) of 10.
The binary numeral system has a \IT{radix} of 2.

Important things to recall:

1) A \IT{number} is a number, while a \IT{digit} is a term from writing systems, and is usually one character

% The original is 'number' is not changed; I think the intent is value, and changed it - Renaissance
2) The value of a number does not change when converted to another radix; only the writing notation for that value has changed (and therefore the way of representing it in \ac{RAM}).

\subsection{Converting From One Radix To Another}

Positional notation is used almost every numerical system. This means that a digit has weight relative to where it is placed inside of the larger number.
If 2 is placed at the rightmost place, it's 2, but if it's placed one digit before rightmost, it's 20.

What does $1234$ stand for?

$10^3 \cdot 1 + 10^2 \cdot 2 + 10^1 \cdot 3 + 1 \cdot 4 = 1234$ or
$1000 \cdot 1 + 100 \cdot 2 + 10 \cdot 3 + 4 = 1234$

It's the same story for binary numbers, but the base is 2 instead of 10.
What does 0b101011 stand for?

$2^5 \cdot 1 + 2^4 \cdot 0 + 2^3 \cdot 1 + 2^2 \cdot 0 + 2^1 \cdot 1 + 2^0 \cdot 1 = 43$ or
$32 \cdot 1 + 16 \cdot 0 + 8 \cdot 1 + 4 \cdot 0 + 2 \cdot 1 + 1 = 43$

There is such a thing as non-positional notation, such as the Roman numeral system.
\footnote{About numeric system evolution, see \InSqBrackets{\TAOCPvolII{}, 195--213.}}.
% Maybe add a sentence to fill in that X is always 10, and is therefore non-positional, even though putting an I before subtracts and after adds, and is in that sense positional
Perhaps, humankind switched to positional notation because it's easier to do basic operations (addition, multiplication, etc.) on paper by hand.

Binary numbers can be added, subtracted and so on in the very same as taught in schools, but only 2 digits are available.

Binary numbers are bulky when represented in source code and dumps, so that is where the hexadecimal numeral system can be useful.
A hexadecimal radix uses the digits 0..9, and also 6 Latin characters: A..F.
Each hexadecimal digit takes 4 bits or 4 binary digits, so it's very easy to convert from binary number to hexadecimal and back, even manually, in one's mind.

\begin{center}
\begin{longtable}{ | l | l | l | }
\hline
\HeaderColor hexadecimal & \HeaderColor binary & \HeaderColor decimal \\
\hline
0	&0000	&0 \\
1	&0001	&1 \\
2	&0010	&2 \\
3	&0011	&3 \\
4	&0100	&4 \\
5	&0101	&5 \\
6	&0110	&6 \\
7	&0111	&7 \\
8	&1000	&8 \\
9	&1001	&9 \\
A	&1010	&10 \\
B	&1011	&11 \\
C	&1100	&12 \\
D	&1101	&13 \\
E	&1110	&14 \\
F	&1111	&15 \\
\hline
\end{longtable}
\end{center}

How can one tell which radix is being used in a specific instance?

Decimal numbers are usually written as is, i.e., 1234. Some assemblers allow an identifier on decimal radix numbers, in which the number would be written with a "d" suffix: 1234d.

Binary numbers are sometimes prepended with the "0b" prefix: 0b100110111 (\ac{GCC} has a non-standard language extension for this\footnote{\url{https://gcc.gnu.org/onlinedocs/gcc/Binary-constants.html}}).
There is also another way: using a "b" suffix, for example: 100110111b.
This book tries to use the "0b" prefix consistently throughout the book for binary numbers.

Hexadecimal numbers are prepended with "0x" prefix in \CCpp and other \ac{PL}s: 0x1234ABCD.
Alternatively, they are given a "h" suffix: 1234ABCDh. This is common way of representing them in assemblers and debuggers.
In this convention, if the number is started with a Latin (A..F) digit, a 0 is added at the beginning: 0ABCDEFh.
There was also convention that was popular in 8-bit home computers era, using \$ prefix, like \$ABCD.
The book will try to stick to "0x" prefix throughout the book for hexadecimal numbers.

Should one learn to convert numbers mentally? A table of 1-digit hexadecimal numbers can easily be memorized.
As for larger numbers, it's probably not worth tormenting yourself.

Perhaps the most visible hexadecimal numbers are in \ac{URL}s.
This is the way that non-Latin characters are encoded.
For example:
\url{https://en.wiktionary.org/wiki/na\%C3\%AFvet\%C3\%A9} is the \ac{URL} of Wiktionary article about \q{naïveté} word.

\subsubsection{Octal Radix}

Another numeral system heavily used in the past of computer programming is octal. In octal there are 8 digits (0..7), and each is mapped to 3 bits, so it's easy to convert numbers back and forth.
It has been superseded by the hexadecimal system almost everywhere, but, surprisingly, there is a *NIX utility, used often by many people, which takes octal numbers as argument: \TT{chmod}.

\myindex{UNIX!chmod}
As many *NIX users know, \TT{chmod} argument can be a number of 3 digits. The first digit represents the rights of the owner of the file (read, write and/or execute), the second is the rights for the group to which the file belongs, and the third is for everyone else.
Each digit that \TT{chmod} takes can be represented in binary form:

\begin{center}
\begin{longtable}{ | l | l | l | }
\hline
\HeaderColor decimal & \HeaderColor binary & \HeaderColor meaning \\
\hline
7	&111	&\textbf{rwx} \\
6	&110	&\textbf{rw-} \\
5	&101	&\textbf{r-x} \\
4	&100	&\textbf{r-{}-} \\
3	&011	&\textbf{-wx} \\
2	&010	&\textbf{-w-} \\
1	&001	&\textbf{-{}-x} \\
0	&000	&\textbf{-{}-{}-} \\
\hline
\end{longtable}
\end{center}

So each bit is mapped to a flag: read/write/execute.

The importance of \TT{chmod} here is that the whole number in argument can be represented as octal number.
Let's take, for example, 644.
When you run \TT{chmod 644 file}, you set read/write permissions for owner, read permissions for group and again, read permissions for everyone else.
If we convert the octal number 644 to binary, it would be \TT{110100100}, or, in groups of 3 bits, \TT{110 100 100}.

Now we see that each triplet describe permissions for owner/group/others: first is \TT{rw-}, second is \TT{r--} and third is \TT{r--}.

The octal numeral system was also popular on old computers like PDP-8, because word there could be 12, 24 or 36 bits, and these numbers are all divisible by 3, so the octal system was natural in that environment.
Nowadays, all popular computers employ word/address sizes of 16, 32 or 64 bits, and these numbers are all divisible by 4, so the hexadecimal system is more natural there.

The octal numeral system is supported by all standard \CCpp compilers.
This is a source of confusion sometimes, because octal numbers are encoded with a zero prepended, for example, 0377 is 255.
Sometimes, you might make a typo and write "09" instead of 9, and the compiler would report an error.
GCC might report something like this:\\
\TT{error: invalid digit "9" in octal constant}.

Also, the octal system is somewhat popular in Java. When the IDA shows Java strings with non-printable characters,
they are encoded in the octal system instead of hexadecimal.
\myindex{JAD}
The JAD Java decompiler behaves the same way.

\subsubsection{Divisibility}

When you see a decimal number like 120, you can quickly deduce that it's divisible by 10, because the last digit is zero.
In the same way, 123400 is divisible by 100, because the two last digits are zeros.

Likewise, the hexadecimal number 0x1230 is divisible by 0x10 (or 16), 0x123000 is divisible by 0x1000 (or 4096), etc.

The binary number 0b1000101000 is divisible by 0b1000 (8), etc.

This property can often be used to quickly realize if the size of some block in memory is padded to some boundary.
For example, sections in \ac{PE} files are almost always started at addresses ending with 3 hexadecimal zeros: 0x41000, 0x10001000, etc.
The reason behind this is the fact that almost all \ac{PE} sections are padded to a boundary of 0x1000 (4096) bytes.

\subsubsection{Multi-Precision Arithmetic and Radix}

\index{RSA}
Multi-precision arithmetic can use huge numbers, and each one may be stored in several bytes.
For example, RSA keys, both public and private, span up to 4096 bits, and maybe even more.

% I'm not sure how to change this, but the normal format for quoting would be just to mention the author or book, and footnote to the full reference
In \InSqBrackets{\TAOCPvolII, 265} we find the following idea: when you store a multi-precision number in several bytes,
the whole number can be represented as having a radix of $2^8=256$, and each digit goes to the corresponding byte.
Likewise, if you store a multi-precision number in several 32-bit integer values, each digit goes to each 32-bit slot,
and you may think about this number as stored in radix of $2^{32}$.

\subsubsection{How to Pronounce Non-Decimal Numbers}

Numbers in a non-decimal base are usually pronounced by digit by digit: ``one-zero-zero-one-one-...''.
Words like ``ten'' and ``thousand'' are usually not pronounced, to prevent confusion with the decimal base system.

\subsubsection{Floating point numbers}

To distinguish floating point numbers from integers, they are usually written with ``.0'' at the end,
like $0.0$, $123.0$, etc.
}
\RU{\subsection{Представление чисел}

Люди привыкли к десятичной системе счисления вероятно потому что почти у каждого есть по 10 пальцев.
Тем не менее, число 10 не имеет особого значения в науке и математике.
Двоичная система естествена для цифровой электроники: 0 означает отсутствие тока в проводе и 1 --- его присутствие.
10 в двоичной системе это 2 в десятичной; 100 в двоичной это 4 в десятичной, итд.

Если в системе счисления есть 10 цифр, её \IT{основание} или \IT{radix} это 10.
Двоичная система имеет \IT{основание} 2.

Важные вещи, которые полезно вспомнить:
1) \IT{число} это число, в то время как \IT{цифра} это термин из системы письменности, и это обычно один символ;
2) само число не меняется, когда конвертируется из одного основания в другое: меняется способ его записи (или представления
в памяти).

Как сконвертировать число из одного основания в другое?

Позиционная нотация используется почти везде, это означает, что всякая цифра имеет свой вес, в зависимости от её расположения
внутри числа.
Если 2 расположена в самом последнем месте справа, это 2.
Если она расположена в месте перед последним, это 20.

Что означает $1234$?

$10^3 \cdot 1 + 10^2 \cdot 2 + 10^1 \cdot 3 + 1 \cdot 4$ = 1234 или
$1000 \cdot 1 + 100 \cdot 2 + 10 \cdot 3 + 4 = 1234$

Та же история и для двоичных чисел, только основание там 2 вместо 10.
Что означает 0b101011?

$2^5 \cdot 1 + 2^4 \cdot 0 + 2^3 \cdot 1 + 2^2 \cdot 0 + 2^1 \cdot 1 + 2^0 \cdot 1 = 43$ или
$32 \cdot 1 + 16 \cdot 0 + 8 \cdot 1 + 4 \cdot 0 + 2 \cdot 1 + 1 = 43$

Позиционную нотацию можно противопоставить непозиционной нотации, такой как римская система записи чисел
\footnote{Об эволюции способов записи чисел, см.также: \InSqBrackets{\TAOCPvolII{}, 195--213.}}.
Вероятно, человечество перешло на позиционную нотацию, потому что так проще работать с числами (сложение, умножение, итд)
на бумаге, в ручную.

Действительно, двоичные числа можно складывать, вычитать, итд, точно также, как этому обычно обучают в школах,
только доступны лишь 2 цифры.

Двоичные числа громоздки, когда их используют в исходных кодах и дампах, так что в этих случаях применяется шестнадцатеричная
система.
Используются цифры 0..9 и еще 6 латинских букв: A..F.
Каждая шестнадцатеричная цифра занимает 4 бита или 4 двоичных цифры, так что конвертировать из двоичной системы в
шестнадцатеричную и назад, можно легко вручную, или даже в уме.

\begin{center}
\begin{longtable}{ | l | l | l | }
\hline
\HeaderColor шестнадцатеричная & \HeaderColor двоичная & \HeaderColor десятичная \\
\hline
0	&0000	&0 \\
1	&0001	&1 \\
2	&0010	&2 \\
3	&0011	&3 \\
4	&0100	&4 \\
5	&0101	&5 \\
6	&0110	&6 \\
7	&0111	&7 \\
8	&1000	&8 \\
9	&1001	&9 \\
A	&1010	&10 \\
B	&1011	&11 \\
C	&1100	&12 \\
D	&1101	&13 \\
E	&1110	&14 \\
F	&1111	&15 \\
\hline
\end{longtable}
\end{center}

Как понять, какое основание используется в конкретном месте?

Десятичные числа обычно записываются как есть, т.е., 1234. Но некоторые ассемблеры позволяют подчеркивать
этот факт для ясности, и это число может быть дополнено суффиксом "d": 1234d.

К двоичным числам иногда спереди добавляют префикс "0b": 0b100110111
(В \ac{GCC} для этого есть нестандартное расширение языка
\footnote{\url{https://gcc.gnu.org/onlinedocs/gcc/Binary-constants.html}}).
Есть также еще один способ: суффикс "b", например: 100110111b.
В этой книге я буду пытаться придерживаться префикса "0b" для двоичных чисел.

Шестнадцатеричные числа имеют префикс "0x" в \CCpp и некоторых других \ac{PL}: 0x1234ABCD.
Либо они имеют суффикс "h": 1234ABCDh --- обычно так они представляются в ассемблерах и отладчиках.
Если число начинается с цифры A..F, перед ним добавляется 0: 0ABCDEFh.
Во времена 8-битных домашних компьютеров, был также способ записи чисел используя префикс \$, например, \$ABCD.
В книге я попытаюсь придерживаться префикса "0x" для шестнадцатеричных чисел.

Нужно ли учиться конвертировать числа в уме? Таблицу шестнадцатеричных чисел из одной цифры легко запомнить.
А запоминать б\'{о}льшие числа, наверное, не стоит.

Наверное, чаще всего шестнадцатеричные числа можно увидеть в \ac{URL}-ах.
Так кодируются буквы не из числа латинских.
Например:
\url{https://en.wiktionary.org/wiki/na\%C3\%AFvet\%C3\%A9} это \ac{URL} страницы в Wiktionary о слове \q{naïveté}.

\subsubsection{Восьмеричная система}

Еще одна система, которая в прошлом много использовалась в программировании это восьмеричная: есть 8 цифр (0..7) и каждая
описывает 3 бита, так что легко конвертировать числа туда и назад.
Она почти везде была заменена шестнадцатеричной, но удивительно, в *NIX имеется утилита использующаяся многими людьми,
которая принимает на вход восьмеричное число: \TT{chmod}.

\myindex{UNIX!chmod}
Как знают многие пользователи *NIX, аргумент \TT{chmod} это число из трех цифр. Первая цифра это права владельца файла,
вторая это права группы (которой файл принадлежит), третья для всех остальных.
И каждая цифра может быть представлена в двоичном виде:

\begin{center}
\begin{longtable}{ | l | l | l | }
\hline
\HeaderColor десятичная & \HeaderColor двоичная & \HeaderColor значение \\
\hline
7	&111	&\textbf{rwx} \\
6	&110	&\textbf{rw-} \\
5	&101	&\textbf{r-x} \\
4	&100	&\textbf{r-{}-} \\
3	&011	&\textbf{-wx} \\
2	&010	&\textbf{-w-} \\
1	&001	&\textbf{-{}-x} \\
0	&000	&\textbf{-{}-{}-} \\
\hline
\end{longtable}
\end{center}

Так что каждый бит привязан к флагу: read/write/execute (чтение/запись/исполнение).

И вот почему я вспомнил здесь о \TT{chmod}, это потому что всё число может быть представлено как число в восьмеричной системе.
Для примера возьмем 644.
Когда вы запускаете \TT{chmod 644 file}, вы выставляете права read/write для владельца, права read для группы, и снова,
read для всех остальных.
Сконвертируем число 644 из восьмеричной системы в двоичную, это будет \TT{110100100}, или (в группах по 3 бита) \TT{110 100 100}.

Теперь мы видим, что каждая тройка описывает права для владельца/группы/остальных:
первая это \TT{rw-}, вторая это \TT{r--} и третья это \TT{r--}.

Восьмеричная система была также популярная на старых компьютерах вроде PDP-8, потому что слово там могло содержать 12, 24 или
36 бит, и эти числа делятся на 3, так что выбор восьмеричной системы в той среде был логичен.
Сейчас, все популярные компьютеры имеют размер слова/адреса 16, 32 или 64 бита, и эти числа делятся на 4,
так что шестнадцатеричная система здесь удобнее.

Восьмеричная система поддерживается всеми стандартными компиляторами \CCpp{}.
Это иногда источник недоумения, потому что восьмеричные числа кодируются с нулем вперед, например, 0377 это 255.
И иногда, вы можете сделать опечатку, и написать "09" вместо 9, и компилятор выдаст ошибку.
GCC может выдать что-то вроде:\\
\TT{error: invalid digit "9" in octal constant}.

Также, восьмеричная система популярна в Java: когда IDA показывает строку с непечатаемыми символами,
они кодируются в восьмеричной системе вместо шестнадцатеричной.
\myindex{JAD}
Точно также себя ведет декомпилятор с Java JAD.

\subsubsection{Делимость}

Когда вы видите десятичное число вроде 120, вы можете быстро понять что оно делится на 10, потому что последняя цифра это 0.
Точно также, 123400 делится на 100, потому что две последних цифры это нули.

Точно также, шестнадцатеричное число 0x1230 делится на 0x10 (или 16), 0x123000 делится на 0x1000 (или 4096), итд.

Двоичное число 0b1000101000 делится на 0b1000 (8), итд.

Это свойство можно часто использовать, чтобы быстро понять,
что длина какого-либо блока в памяти выровнена по некоторой границе.
Например, секции в \ac{PE}-файлах почти всегда начинаются с адресов заканчивающихся 3 шестнадцатеричными нулями:
0x41000, 0x10001000, итд.
Причина в том, что почти все секции в \ac{PE} выровнены по границе 0x1000 (4096) байт.

\subsubsection{Арифметика произвольной точности и основание}

\index{RSA}
Арифметика произвольной точности (multi-precision arithmetic) может использовать огромные числа,
которые могут храниться в нескольких байтах.
Например, ключи RSA, и открытые и закрытые, могут занимать до 4096 бит и даже больше.

В \InSqBrackets{\TAOCPvolII, 265} можно найти такую идею: когда вы сохраняете число произвольной точности в нескольких байтах,
всё число может быть представлено как имеющую систему счисления по основанию $2^8=256$, и каждая цифра находится
в соответствующем байте.
Точно также, если вы сохраняете число произвольной точности в нескольких 32-битных целочисленных значениях,
каждая цифра отправляется в каждый 32-битный слот, и вы можете считать что это число записано в системе с основанием $2^{32}$.

\subsubsection{Произношение}

Числа в недесятичных системах счислениях обычно произносятся по одной цифре: ``один-ноль-ноль-один-один-...''.
Слова вроде ``десять'', ``тысяча'', итд, обычно не произносятся, потому что тогда можно спутать с десятичной системой.

\subsubsection{Числа с плавающей запятой}

Чтобы отличать числа с плавающей запятой от целочисленных, часто, в конце добавляют ``.0'',
например $0.0$, $123.0$, итд.

}
\ITA{\input{patterns/numeral_ITA}}
\DE{\input{patterns/numeral_DE}}
\FR{\input{patterns/numeral_FR}}
\PL{\input{patterns/numeral_PL}}

% chapters
\ifdefined\SPANISH
\chapter{Patrones de código}
\fi % SPANISH

\ifdefined\GERMAN
\chapter{Code-Muster}
\fi % GERMAN

\ifdefined\ENGLISH
\chapter{Code Patterns}
\fi % ENGLISH

\ifdefined\ITALIAN
\chapter{Forme di codice}
\fi % ITALIAN

\ifdefined\RUSSIAN
\chapter{Образцы кода}
\fi % RUSSIAN

\ifdefined\BRAZILIAN
\chapter{Padrões de códigos}
\fi % BRAZILIAN

\ifdefined\THAI
\chapter{รูปแบบของโค้ด}
\fi % THAI

\ifdefined\FRENCH
\chapter{Modèle de code}
\fi % FRENCH

\ifdefined\POLISH
\chapter{\PLph{}}
\fi % POLISH

% sections
\EN{\input{patterns/patterns_opt_dbg_EN}}
\ES{\input{patterns/patterns_opt_dbg_ES}}
\ITA{\input{patterns/patterns_opt_dbg_ITA}}
\PTBR{\input{patterns/patterns_opt_dbg_PTBR}}
\RU{\input{patterns/patterns_opt_dbg_RU}}
\THA{\input{patterns/patterns_opt_dbg_THA}}
\DE{\input{patterns/patterns_opt_dbg_DE}}
\FR{\input{patterns/patterns_opt_dbg_FR}}
\PL{\input{patterns/patterns_opt_dbg_PL}}

\RU{\section{Некоторые базовые понятия}}
\EN{\section{Some basics}}
\DE{\section{Einige Grundlagen}}
\FR{\section{Quelques bases}}
\ES{\section{\ESph{}}}
\ITA{\section{Alcune basi teoriche}}
\PTBR{\section{\PTBRph{}}}
\THA{\section{\THAph{}}}
\PL{\section{\PLph{}}}

% sections:
\EN{\input{patterns/intro_CPU_ISA_EN}}
\ES{\input{patterns/intro_CPU_ISA_ES}}
\ITA{\input{patterns/intro_CPU_ISA_ITA}}
\PTBR{\input{patterns/intro_CPU_ISA_PTBR}}
\RU{\input{patterns/intro_CPU_ISA_RU}}
\DE{\input{patterns/intro_CPU_ISA_DE}}
\FR{\input{patterns/intro_CPU_ISA_FR}}
\PL{\input{patterns/intro_CPU_ISA_PL}}

\EN{\input{patterns/numeral_EN}}
\RU{\input{patterns/numeral_RU}}
\ITA{\input{patterns/numeral_ITA}}
\DE{\input{patterns/numeral_DE}}
\FR{\input{patterns/numeral_FR}}
\PL{\input{patterns/numeral_PL}}

% chapters
\input{patterns/00_empty/main}
\input{patterns/011_ret/main}
\input{patterns/01_helloworld/main}
\input{patterns/015_prolog_epilogue/main}
\input{patterns/02_stack/main}
\input{patterns/03_printf/main}
\input{patterns/04_scanf/main}
\input{patterns/05_passing_arguments/main}
\input{patterns/06_return_results/main}
\input{patterns/061_pointers/main}
\input{patterns/065_GOTO/main}
\input{patterns/07_jcc/main}
\input{patterns/08_switch/main}
\input{patterns/09_loops/main}
\input{patterns/10_strings/main}
\input{patterns/11_arith_optimizations/main}
\input{patterns/12_FPU/main}
\input{patterns/13_arrays/main}
\input{patterns/14_bitfields/main}
\EN{\input{patterns/145_LCG/main_EN}}
\RU{\input{patterns/145_LCG/main_RU}}
\input{patterns/15_structs/main}
\input{patterns/17_unions/main}
\input{patterns/18_pointers_to_functions/main}
\input{patterns/185_64bit_in_32_env/main}

\EN{\input{patterns/19_SIMD/main_EN}}
\RU{\input{patterns/19_SIMD/main_RU}}
\DE{\input{patterns/19_SIMD/main_DE}}

\EN{\input{patterns/20_x64/main_EN}}
\RU{\input{patterns/20_x64/main_RU}}

\EN{\input{patterns/205_floating_SIMD/main_EN}}
\RU{\input{patterns/205_floating_SIMD/main_RU}}
\DE{\input{patterns/205_floating_SIMD/main_DE}}

\EN{\input{patterns/ARM/main_EN}}
\RU{\input{patterns/ARM/main_RU}}
\DE{\input{patterns/ARM/main_DE}}

\input{patterns/MIPS/main}

\ifdefined\SPANISH
\chapter{Patrones de código}
\fi % SPANISH

\ifdefined\GERMAN
\chapter{Code-Muster}
\fi % GERMAN

\ifdefined\ENGLISH
\chapter{Code Patterns}
\fi % ENGLISH

\ifdefined\ITALIAN
\chapter{Forme di codice}
\fi % ITALIAN

\ifdefined\RUSSIAN
\chapter{Образцы кода}
\fi % RUSSIAN

\ifdefined\BRAZILIAN
\chapter{Padrões de códigos}
\fi % BRAZILIAN

\ifdefined\THAI
\chapter{รูปแบบของโค้ด}
\fi % THAI

\ifdefined\FRENCH
\chapter{Modèle de code}
\fi % FRENCH

\ifdefined\POLISH
\chapter{\PLph{}}
\fi % POLISH

% sections
\EN{\input{patterns/patterns_opt_dbg_EN}}
\ES{\input{patterns/patterns_opt_dbg_ES}}
\ITA{\input{patterns/patterns_opt_dbg_ITA}}
\PTBR{\input{patterns/patterns_opt_dbg_PTBR}}
\RU{\input{patterns/patterns_opt_dbg_RU}}
\THA{\input{patterns/patterns_opt_dbg_THA}}
\DE{\input{patterns/patterns_opt_dbg_DE}}
\FR{\input{patterns/patterns_opt_dbg_FR}}
\PL{\input{patterns/patterns_opt_dbg_PL}}

\RU{\section{Некоторые базовые понятия}}
\EN{\section{Some basics}}
\DE{\section{Einige Grundlagen}}
\FR{\section{Quelques bases}}
\ES{\section{\ESph{}}}
\ITA{\section{Alcune basi teoriche}}
\PTBR{\section{\PTBRph{}}}
\THA{\section{\THAph{}}}
\PL{\section{\PLph{}}}

% sections:
\EN{\input{patterns/intro_CPU_ISA_EN}}
\ES{\input{patterns/intro_CPU_ISA_ES}}
\ITA{\input{patterns/intro_CPU_ISA_ITA}}
\PTBR{\input{patterns/intro_CPU_ISA_PTBR}}
\RU{\input{patterns/intro_CPU_ISA_RU}}
\DE{\input{patterns/intro_CPU_ISA_DE}}
\FR{\input{patterns/intro_CPU_ISA_FR}}
\PL{\input{patterns/intro_CPU_ISA_PL}}

\EN{\input{patterns/numeral_EN}}
\RU{\input{patterns/numeral_RU}}
\ITA{\input{patterns/numeral_ITA}}
\DE{\input{patterns/numeral_DE}}
\FR{\input{patterns/numeral_FR}}
\PL{\input{patterns/numeral_PL}}

% chapters
\input{patterns/00_empty/main}
\input{patterns/011_ret/main}
\input{patterns/01_helloworld/main}
\input{patterns/015_prolog_epilogue/main}
\input{patterns/02_stack/main}
\input{patterns/03_printf/main}
\input{patterns/04_scanf/main}
\input{patterns/05_passing_arguments/main}
\input{patterns/06_return_results/main}
\input{patterns/061_pointers/main}
\input{patterns/065_GOTO/main}
\input{patterns/07_jcc/main}
\input{patterns/08_switch/main}
\input{patterns/09_loops/main}
\input{patterns/10_strings/main}
\input{patterns/11_arith_optimizations/main}
\input{patterns/12_FPU/main}
\input{patterns/13_arrays/main}
\input{patterns/14_bitfields/main}
\EN{\input{patterns/145_LCG/main_EN}}
\RU{\input{patterns/145_LCG/main_RU}}
\input{patterns/15_structs/main}
\input{patterns/17_unions/main}
\input{patterns/18_pointers_to_functions/main}
\input{patterns/185_64bit_in_32_env/main}

\EN{\input{patterns/19_SIMD/main_EN}}
\RU{\input{patterns/19_SIMD/main_RU}}
\DE{\input{patterns/19_SIMD/main_DE}}

\EN{\input{patterns/20_x64/main_EN}}
\RU{\input{patterns/20_x64/main_RU}}

\EN{\input{patterns/205_floating_SIMD/main_EN}}
\RU{\input{patterns/205_floating_SIMD/main_RU}}
\DE{\input{patterns/205_floating_SIMD/main_DE}}

\EN{\input{patterns/ARM/main_EN}}
\RU{\input{patterns/ARM/main_RU}}
\DE{\input{patterns/ARM/main_DE}}

\input{patterns/MIPS/main}

\ifdefined\SPANISH
\chapter{Patrones de código}
\fi % SPANISH

\ifdefined\GERMAN
\chapter{Code-Muster}
\fi % GERMAN

\ifdefined\ENGLISH
\chapter{Code Patterns}
\fi % ENGLISH

\ifdefined\ITALIAN
\chapter{Forme di codice}
\fi % ITALIAN

\ifdefined\RUSSIAN
\chapter{Образцы кода}
\fi % RUSSIAN

\ifdefined\BRAZILIAN
\chapter{Padrões de códigos}
\fi % BRAZILIAN

\ifdefined\THAI
\chapter{รูปแบบของโค้ด}
\fi % THAI

\ifdefined\FRENCH
\chapter{Modèle de code}
\fi % FRENCH

\ifdefined\POLISH
\chapter{\PLph{}}
\fi % POLISH

% sections
\EN{\input{patterns/patterns_opt_dbg_EN}}
\ES{\input{patterns/patterns_opt_dbg_ES}}
\ITA{\input{patterns/patterns_opt_dbg_ITA}}
\PTBR{\input{patterns/patterns_opt_dbg_PTBR}}
\RU{\input{patterns/patterns_opt_dbg_RU}}
\THA{\input{patterns/patterns_opt_dbg_THA}}
\DE{\input{patterns/patterns_opt_dbg_DE}}
\FR{\input{patterns/patterns_opt_dbg_FR}}
\PL{\input{patterns/patterns_opt_dbg_PL}}

\RU{\section{Некоторые базовые понятия}}
\EN{\section{Some basics}}
\DE{\section{Einige Grundlagen}}
\FR{\section{Quelques bases}}
\ES{\section{\ESph{}}}
\ITA{\section{Alcune basi teoriche}}
\PTBR{\section{\PTBRph{}}}
\THA{\section{\THAph{}}}
\PL{\section{\PLph{}}}

% sections:
\EN{\input{patterns/intro_CPU_ISA_EN}}
\ES{\input{patterns/intro_CPU_ISA_ES}}
\ITA{\input{patterns/intro_CPU_ISA_ITA}}
\PTBR{\input{patterns/intro_CPU_ISA_PTBR}}
\RU{\input{patterns/intro_CPU_ISA_RU}}
\DE{\input{patterns/intro_CPU_ISA_DE}}
\FR{\input{patterns/intro_CPU_ISA_FR}}
\PL{\input{patterns/intro_CPU_ISA_PL}}

\EN{\input{patterns/numeral_EN}}
\RU{\input{patterns/numeral_RU}}
\ITA{\input{patterns/numeral_ITA}}
\DE{\input{patterns/numeral_DE}}
\FR{\input{patterns/numeral_FR}}
\PL{\input{patterns/numeral_PL}}

% chapters
\input{patterns/00_empty/main}
\input{patterns/011_ret/main}
\input{patterns/01_helloworld/main}
\input{patterns/015_prolog_epilogue/main}
\input{patterns/02_stack/main}
\input{patterns/03_printf/main}
\input{patterns/04_scanf/main}
\input{patterns/05_passing_arguments/main}
\input{patterns/06_return_results/main}
\input{patterns/061_pointers/main}
\input{patterns/065_GOTO/main}
\input{patterns/07_jcc/main}
\input{patterns/08_switch/main}
\input{patterns/09_loops/main}
\input{patterns/10_strings/main}
\input{patterns/11_arith_optimizations/main}
\input{patterns/12_FPU/main}
\input{patterns/13_arrays/main}
\input{patterns/14_bitfields/main}
\EN{\input{patterns/145_LCG/main_EN}}
\RU{\input{patterns/145_LCG/main_RU}}
\input{patterns/15_structs/main}
\input{patterns/17_unions/main}
\input{patterns/18_pointers_to_functions/main}
\input{patterns/185_64bit_in_32_env/main}

\EN{\input{patterns/19_SIMD/main_EN}}
\RU{\input{patterns/19_SIMD/main_RU}}
\DE{\input{patterns/19_SIMD/main_DE}}

\EN{\input{patterns/20_x64/main_EN}}
\RU{\input{patterns/20_x64/main_RU}}

\EN{\input{patterns/205_floating_SIMD/main_EN}}
\RU{\input{patterns/205_floating_SIMD/main_RU}}
\DE{\input{patterns/205_floating_SIMD/main_DE}}

\EN{\input{patterns/ARM/main_EN}}
\RU{\input{patterns/ARM/main_RU}}
\DE{\input{patterns/ARM/main_DE}}

\input{patterns/MIPS/main}

\ifdefined\SPANISH
\chapter{Patrones de código}
\fi % SPANISH

\ifdefined\GERMAN
\chapter{Code-Muster}
\fi % GERMAN

\ifdefined\ENGLISH
\chapter{Code Patterns}
\fi % ENGLISH

\ifdefined\ITALIAN
\chapter{Forme di codice}
\fi % ITALIAN

\ifdefined\RUSSIAN
\chapter{Образцы кода}
\fi % RUSSIAN

\ifdefined\BRAZILIAN
\chapter{Padrões de códigos}
\fi % BRAZILIAN

\ifdefined\THAI
\chapter{รูปแบบของโค้ด}
\fi % THAI

\ifdefined\FRENCH
\chapter{Modèle de code}
\fi % FRENCH

\ifdefined\POLISH
\chapter{\PLph{}}
\fi % POLISH

% sections
\EN{\input{patterns/patterns_opt_dbg_EN}}
\ES{\input{patterns/patterns_opt_dbg_ES}}
\ITA{\input{patterns/patterns_opt_dbg_ITA}}
\PTBR{\input{patterns/patterns_opt_dbg_PTBR}}
\RU{\input{patterns/patterns_opt_dbg_RU}}
\THA{\input{patterns/patterns_opt_dbg_THA}}
\DE{\input{patterns/patterns_opt_dbg_DE}}
\FR{\input{patterns/patterns_opt_dbg_FR}}
\PL{\input{patterns/patterns_opt_dbg_PL}}

\RU{\section{Некоторые базовые понятия}}
\EN{\section{Some basics}}
\DE{\section{Einige Grundlagen}}
\FR{\section{Quelques bases}}
\ES{\section{\ESph{}}}
\ITA{\section{Alcune basi teoriche}}
\PTBR{\section{\PTBRph{}}}
\THA{\section{\THAph{}}}
\PL{\section{\PLph{}}}

% sections:
\EN{\input{patterns/intro_CPU_ISA_EN}}
\ES{\input{patterns/intro_CPU_ISA_ES}}
\ITA{\input{patterns/intro_CPU_ISA_ITA}}
\PTBR{\input{patterns/intro_CPU_ISA_PTBR}}
\RU{\input{patterns/intro_CPU_ISA_RU}}
\DE{\input{patterns/intro_CPU_ISA_DE}}
\FR{\input{patterns/intro_CPU_ISA_FR}}
\PL{\input{patterns/intro_CPU_ISA_PL}}

\EN{\input{patterns/numeral_EN}}
\RU{\input{patterns/numeral_RU}}
\ITA{\input{patterns/numeral_ITA}}
\DE{\input{patterns/numeral_DE}}
\FR{\input{patterns/numeral_FR}}
\PL{\input{patterns/numeral_PL}}

% chapters
\input{patterns/00_empty/main}
\input{patterns/011_ret/main}
\input{patterns/01_helloworld/main}
\input{patterns/015_prolog_epilogue/main}
\input{patterns/02_stack/main}
\input{patterns/03_printf/main}
\input{patterns/04_scanf/main}
\input{patterns/05_passing_arguments/main}
\input{patterns/06_return_results/main}
\input{patterns/061_pointers/main}
\input{patterns/065_GOTO/main}
\input{patterns/07_jcc/main}
\input{patterns/08_switch/main}
\input{patterns/09_loops/main}
\input{patterns/10_strings/main}
\input{patterns/11_arith_optimizations/main}
\input{patterns/12_FPU/main}
\input{patterns/13_arrays/main}
\input{patterns/14_bitfields/main}
\EN{\input{patterns/145_LCG/main_EN}}
\RU{\input{patterns/145_LCG/main_RU}}
\input{patterns/15_structs/main}
\input{patterns/17_unions/main}
\input{patterns/18_pointers_to_functions/main}
\input{patterns/185_64bit_in_32_env/main}

\EN{\input{patterns/19_SIMD/main_EN}}
\RU{\input{patterns/19_SIMD/main_RU}}
\DE{\input{patterns/19_SIMD/main_DE}}

\EN{\input{patterns/20_x64/main_EN}}
\RU{\input{patterns/20_x64/main_RU}}

\EN{\input{patterns/205_floating_SIMD/main_EN}}
\RU{\input{patterns/205_floating_SIMD/main_RU}}
\DE{\input{patterns/205_floating_SIMD/main_DE}}

\EN{\input{patterns/ARM/main_EN}}
\RU{\input{patterns/ARM/main_RU}}
\DE{\input{patterns/ARM/main_DE}}

\input{patterns/MIPS/main}

\ifdefined\SPANISH
\chapter{Patrones de código}
\fi % SPANISH

\ifdefined\GERMAN
\chapter{Code-Muster}
\fi % GERMAN

\ifdefined\ENGLISH
\chapter{Code Patterns}
\fi % ENGLISH

\ifdefined\ITALIAN
\chapter{Forme di codice}
\fi % ITALIAN

\ifdefined\RUSSIAN
\chapter{Образцы кода}
\fi % RUSSIAN

\ifdefined\BRAZILIAN
\chapter{Padrões de códigos}
\fi % BRAZILIAN

\ifdefined\THAI
\chapter{รูปแบบของโค้ด}
\fi % THAI

\ifdefined\FRENCH
\chapter{Modèle de code}
\fi % FRENCH

\ifdefined\POLISH
\chapter{\PLph{}}
\fi % POLISH

% sections
\EN{\input{patterns/patterns_opt_dbg_EN}}
\ES{\input{patterns/patterns_opt_dbg_ES}}
\ITA{\input{patterns/patterns_opt_dbg_ITA}}
\PTBR{\input{patterns/patterns_opt_dbg_PTBR}}
\RU{\input{patterns/patterns_opt_dbg_RU}}
\THA{\input{patterns/patterns_opt_dbg_THA}}
\DE{\input{patterns/patterns_opt_dbg_DE}}
\FR{\input{patterns/patterns_opt_dbg_FR}}
\PL{\input{patterns/patterns_opt_dbg_PL}}

\RU{\section{Некоторые базовые понятия}}
\EN{\section{Some basics}}
\DE{\section{Einige Grundlagen}}
\FR{\section{Quelques bases}}
\ES{\section{\ESph{}}}
\ITA{\section{Alcune basi teoriche}}
\PTBR{\section{\PTBRph{}}}
\THA{\section{\THAph{}}}
\PL{\section{\PLph{}}}

% sections:
\EN{\input{patterns/intro_CPU_ISA_EN}}
\ES{\input{patterns/intro_CPU_ISA_ES}}
\ITA{\input{patterns/intro_CPU_ISA_ITA}}
\PTBR{\input{patterns/intro_CPU_ISA_PTBR}}
\RU{\input{patterns/intro_CPU_ISA_RU}}
\DE{\input{patterns/intro_CPU_ISA_DE}}
\FR{\input{patterns/intro_CPU_ISA_FR}}
\PL{\input{patterns/intro_CPU_ISA_PL}}

\EN{\input{patterns/numeral_EN}}
\RU{\input{patterns/numeral_RU}}
\ITA{\input{patterns/numeral_ITA}}
\DE{\input{patterns/numeral_DE}}
\FR{\input{patterns/numeral_FR}}
\PL{\input{patterns/numeral_PL}}

% chapters
\input{patterns/00_empty/main}
\input{patterns/011_ret/main}
\input{patterns/01_helloworld/main}
\input{patterns/015_prolog_epilogue/main}
\input{patterns/02_stack/main}
\input{patterns/03_printf/main}
\input{patterns/04_scanf/main}
\input{patterns/05_passing_arguments/main}
\input{patterns/06_return_results/main}
\input{patterns/061_pointers/main}
\input{patterns/065_GOTO/main}
\input{patterns/07_jcc/main}
\input{patterns/08_switch/main}
\input{patterns/09_loops/main}
\input{patterns/10_strings/main}
\input{patterns/11_arith_optimizations/main}
\input{patterns/12_FPU/main}
\input{patterns/13_arrays/main}
\input{patterns/14_bitfields/main}
\EN{\input{patterns/145_LCG/main_EN}}
\RU{\input{patterns/145_LCG/main_RU}}
\input{patterns/15_structs/main}
\input{patterns/17_unions/main}
\input{patterns/18_pointers_to_functions/main}
\input{patterns/185_64bit_in_32_env/main}

\EN{\input{patterns/19_SIMD/main_EN}}
\RU{\input{patterns/19_SIMD/main_RU}}
\DE{\input{patterns/19_SIMD/main_DE}}

\EN{\input{patterns/20_x64/main_EN}}
\RU{\input{patterns/20_x64/main_RU}}

\EN{\input{patterns/205_floating_SIMD/main_EN}}
\RU{\input{patterns/205_floating_SIMD/main_RU}}
\DE{\input{patterns/205_floating_SIMD/main_DE}}

\EN{\input{patterns/ARM/main_EN}}
\RU{\input{patterns/ARM/main_RU}}
\DE{\input{patterns/ARM/main_DE}}

\input{patterns/MIPS/main}

\ifdefined\SPANISH
\chapter{Patrones de código}
\fi % SPANISH

\ifdefined\GERMAN
\chapter{Code-Muster}
\fi % GERMAN

\ifdefined\ENGLISH
\chapter{Code Patterns}
\fi % ENGLISH

\ifdefined\ITALIAN
\chapter{Forme di codice}
\fi % ITALIAN

\ifdefined\RUSSIAN
\chapter{Образцы кода}
\fi % RUSSIAN

\ifdefined\BRAZILIAN
\chapter{Padrões de códigos}
\fi % BRAZILIAN

\ifdefined\THAI
\chapter{รูปแบบของโค้ด}
\fi % THAI

\ifdefined\FRENCH
\chapter{Modèle de code}
\fi % FRENCH

\ifdefined\POLISH
\chapter{\PLph{}}
\fi % POLISH

% sections
\EN{\input{patterns/patterns_opt_dbg_EN}}
\ES{\input{patterns/patterns_opt_dbg_ES}}
\ITA{\input{patterns/patterns_opt_dbg_ITA}}
\PTBR{\input{patterns/patterns_opt_dbg_PTBR}}
\RU{\input{patterns/patterns_opt_dbg_RU}}
\THA{\input{patterns/patterns_opt_dbg_THA}}
\DE{\input{patterns/patterns_opt_dbg_DE}}
\FR{\input{patterns/patterns_opt_dbg_FR}}
\PL{\input{patterns/patterns_opt_dbg_PL}}

\RU{\section{Некоторые базовые понятия}}
\EN{\section{Some basics}}
\DE{\section{Einige Grundlagen}}
\FR{\section{Quelques bases}}
\ES{\section{\ESph{}}}
\ITA{\section{Alcune basi teoriche}}
\PTBR{\section{\PTBRph{}}}
\THA{\section{\THAph{}}}
\PL{\section{\PLph{}}}

% sections:
\EN{\input{patterns/intro_CPU_ISA_EN}}
\ES{\input{patterns/intro_CPU_ISA_ES}}
\ITA{\input{patterns/intro_CPU_ISA_ITA}}
\PTBR{\input{patterns/intro_CPU_ISA_PTBR}}
\RU{\input{patterns/intro_CPU_ISA_RU}}
\DE{\input{patterns/intro_CPU_ISA_DE}}
\FR{\input{patterns/intro_CPU_ISA_FR}}
\PL{\input{patterns/intro_CPU_ISA_PL}}

\EN{\input{patterns/numeral_EN}}
\RU{\input{patterns/numeral_RU}}
\ITA{\input{patterns/numeral_ITA}}
\DE{\input{patterns/numeral_DE}}
\FR{\input{patterns/numeral_FR}}
\PL{\input{patterns/numeral_PL}}

% chapters
\input{patterns/00_empty/main}
\input{patterns/011_ret/main}
\input{patterns/01_helloworld/main}
\input{patterns/015_prolog_epilogue/main}
\input{patterns/02_stack/main}
\input{patterns/03_printf/main}
\input{patterns/04_scanf/main}
\input{patterns/05_passing_arguments/main}
\input{patterns/06_return_results/main}
\input{patterns/061_pointers/main}
\input{patterns/065_GOTO/main}
\input{patterns/07_jcc/main}
\input{patterns/08_switch/main}
\input{patterns/09_loops/main}
\input{patterns/10_strings/main}
\input{patterns/11_arith_optimizations/main}
\input{patterns/12_FPU/main}
\input{patterns/13_arrays/main}
\input{patterns/14_bitfields/main}
\EN{\input{patterns/145_LCG/main_EN}}
\RU{\input{patterns/145_LCG/main_RU}}
\input{patterns/15_structs/main}
\input{patterns/17_unions/main}
\input{patterns/18_pointers_to_functions/main}
\input{patterns/185_64bit_in_32_env/main}

\EN{\input{patterns/19_SIMD/main_EN}}
\RU{\input{patterns/19_SIMD/main_RU}}
\DE{\input{patterns/19_SIMD/main_DE}}

\EN{\input{patterns/20_x64/main_EN}}
\RU{\input{patterns/20_x64/main_RU}}

\EN{\input{patterns/205_floating_SIMD/main_EN}}
\RU{\input{patterns/205_floating_SIMD/main_RU}}
\DE{\input{patterns/205_floating_SIMD/main_DE}}

\EN{\input{patterns/ARM/main_EN}}
\RU{\input{patterns/ARM/main_RU}}
\DE{\input{patterns/ARM/main_DE}}

\input{patterns/MIPS/main}

\ifdefined\SPANISH
\chapter{Patrones de código}
\fi % SPANISH

\ifdefined\GERMAN
\chapter{Code-Muster}
\fi % GERMAN

\ifdefined\ENGLISH
\chapter{Code Patterns}
\fi % ENGLISH

\ifdefined\ITALIAN
\chapter{Forme di codice}
\fi % ITALIAN

\ifdefined\RUSSIAN
\chapter{Образцы кода}
\fi % RUSSIAN

\ifdefined\BRAZILIAN
\chapter{Padrões de códigos}
\fi % BRAZILIAN

\ifdefined\THAI
\chapter{รูปแบบของโค้ด}
\fi % THAI

\ifdefined\FRENCH
\chapter{Modèle de code}
\fi % FRENCH

\ifdefined\POLISH
\chapter{\PLph{}}
\fi % POLISH

% sections
\EN{\input{patterns/patterns_opt_dbg_EN}}
\ES{\input{patterns/patterns_opt_dbg_ES}}
\ITA{\input{patterns/patterns_opt_dbg_ITA}}
\PTBR{\input{patterns/patterns_opt_dbg_PTBR}}
\RU{\input{patterns/patterns_opt_dbg_RU}}
\THA{\input{patterns/patterns_opt_dbg_THA}}
\DE{\input{patterns/patterns_opt_dbg_DE}}
\FR{\input{patterns/patterns_opt_dbg_FR}}
\PL{\input{patterns/patterns_opt_dbg_PL}}

\RU{\section{Некоторые базовые понятия}}
\EN{\section{Some basics}}
\DE{\section{Einige Grundlagen}}
\FR{\section{Quelques bases}}
\ES{\section{\ESph{}}}
\ITA{\section{Alcune basi teoriche}}
\PTBR{\section{\PTBRph{}}}
\THA{\section{\THAph{}}}
\PL{\section{\PLph{}}}

% sections:
\EN{\input{patterns/intro_CPU_ISA_EN}}
\ES{\input{patterns/intro_CPU_ISA_ES}}
\ITA{\input{patterns/intro_CPU_ISA_ITA}}
\PTBR{\input{patterns/intro_CPU_ISA_PTBR}}
\RU{\input{patterns/intro_CPU_ISA_RU}}
\DE{\input{patterns/intro_CPU_ISA_DE}}
\FR{\input{patterns/intro_CPU_ISA_FR}}
\PL{\input{patterns/intro_CPU_ISA_PL}}

\EN{\input{patterns/numeral_EN}}
\RU{\input{patterns/numeral_RU}}
\ITA{\input{patterns/numeral_ITA}}
\DE{\input{patterns/numeral_DE}}
\FR{\input{patterns/numeral_FR}}
\PL{\input{patterns/numeral_PL}}

% chapters
\input{patterns/00_empty/main}
\input{patterns/011_ret/main}
\input{patterns/01_helloworld/main}
\input{patterns/015_prolog_epilogue/main}
\input{patterns/02_stack/main}
\input{patterns/03_printf/main}
\input{patterns/04_scanf/main}
\input{patterns/05_passing_arguments/main}
\input{patterns/06_return_results/main}
\input{patterns/061_pointers/main}
\input{patterns/065_GOTO/main}
\input{patterns/07_jcc/main}
\input{patterns/08_switch/main}
\input{patterns/09_loops/main}
\input{patterns/10_strings/main}
\input{patterns/11_arith_optimizations/main}
\input{patterns/12_FPU/main}
\input{patterns/13_arrays/main}
\input{patterns/14_bitfields/main}
\EN{\input{patterns/145_LCG/main_EN}}
\RU{\input{patterns/145_LCG/main_RU}}
\input{patterns/15_structs/main}
\input{patterns/17_unions/main}
\input{patterns/18_pointers_to_functions/main}
\input{patterns/185_64bit_in_32_env/main}

\EN{\input{patterns/19_SIMD/main_EN}}
\RU{\input{patterns/19_SIMD/main_RU}}
\DE{\input{patterns/19_SIMD/main_DE}}

\EN{\input{patterns/20_x64/main_EN}}
\RU{\input{patterns/20_x64/main_RU}}

\EN{\input{patterns/205_floating_SIMD/main_EN}}
\RU{\input{patterns/205_floating_SIMD/main_RU}}
\DE{\input{patterns/205_floating_SIMD/main_DE}}

\EN{\input{patterns/ARM/main_EN}}
\RU{\input{patterns/ARM/main_RU}}
\DE{\input{patterns/ARM/main_DE}}

\input{patterns/MIPS/main}

\ifdefined\SPANISH
\chapter{Patrones de código}
\fi % SPANISH

\ifdefined\GERMAN
\chapter{Code-Muster}
\fi % GERMAN

\ifdefined\ENGLISH
\chapter{Code Patterns}
\fi % ENGLISH

\ifdefined\ITALIAN
\chapter{Forme di codice}
\fi % ITALIAN

\ifdefined\RUSSIAN
\chapter{Образцы кода}
\fi % RUSSIAN

\ifdefined\BRAZILIAN
\chapter{Padrões de códigos}
\fi % BRAZILIAN

\ifdefined\THAI
\chapter{รูปแบบของโค้ด}
\fi % THAI

\ifdefined\FRENCH
\chapter{Modèle de code}
\fi % FRENCH

\ifdefined\POLISH
\chapter{\PLph{}}
\fi % POLISH

% sections
\EN{\input{patterns/patterns_opt_dbg_EN}}
\ES{\input{patterns/patterns_opt_dbg_ES}}
\ITA{\input{patterns/patterns_opt_dbg_ITA}}
\PTBR{\input{patterns/patterns_opt_dbg_PTBR}}
\RU{\input{patterns/patterns_opt_dbg_RU}}
\THA{\input{patterns/patterns_opt_dbg_THA}}
\DE{\input{patterns/patterns_opt_dbg_DE}}
\FR{\input{patterns/patterns_opt_dbg_FR}}
\PL{\input{patterns/patterns_opt_dbg_PL}}

\RU{\section{Некоторые базовые понятия}}
\EN{\section{Some basics}}
\DE{\section{Einige Grundlagen}}
\FR{\section{Quelques bases}}
\ES{\section{\ESph{}}}
\ITA{\section{Alcune basi teoriche}}
\PTBR{\section{\PTBRph{}}}
\THA{\section{\THAph{}}}
\PL{\section{\PLph{}}}

% sections:
\EN{\input{patterns/intro_CPU_ISA_EN}}
\ES{\input{patterns/intro_CPU_ISA_ES}}
\ITA{\input{patterns/intro_CPU_ISA_ITA}}
\PTBR{\input{patterns/intro_CPU_ISA_PTBR}}
\RU{\input{patterns/intro_CPU_ISA_RU}}
\DE{\input{patterns/intro_CPU_ISA_DE}}
\FR{\input{patterns/intro_CPU_ISA_FR}}
\PL{\input{patterns/intro_CPU_ISA_PL}}

\EN{\input{patterns/numeral_EN}}
\RU{\input{patterns/numeral_RU}}
\ITA{\input{patterns/numeral_ITA}}
\DE{\input{patterns/numeral_DE}}
\FR{\input{patterns/numeral_FR}}
\PL{\input{patterns/numeral_PL}}

% chapters
\input{patterns/00_empty/main}
\input{patterns/011_ret/main}
\input{patterns/01_helloworld/main}
\input{patterns/015_prolog_epilogue/main}
\input{patterns/02_stack/main}
\input{patterns/03_printf/main}
\input{patterns/04_scanf/main}
\input{patterns/05_passing_arguments/main}
\input{patterns/06_return_results/main}
\input{patterns/061_pointers/main}
\input{patterns/065_GOTO/main}
\input{patterns/07_jcc/main}
\input{patterns/08_switch/main}
\input{patterns/09_loops/main}
\input{patterns/10_strings/main}
\input{patterns/11_arith_optimizations/main}
\input{patterns/12_FPU/main}
\input{patterns/13_arrays/main}
\input{patterns/14_bitfields/main}
\EN{\input{patterns/145_LCG/main_EN}}
\RU{\input{patterns/145_LCG/main_RU}}
\input{patterns/15_structs/main}
\input{patterns/17_unions/main}
\input{patterns/18_pointers_to_functions/main}
\input{patterns/185_64bit_in_32_env/main}

\EN{\input{patterns/19_SIMD/main_EN}}
\RU{\input{patterns/19_SIMD/main_RU}}
\DE{\input{patterns/19_SIMD/main_DE}}

\EN{\input{patterns/20_x64/main_EN}}
\RU{\input{patterns/20_x64/main_RU}}

\EN{\input{patterns/205_floating_SIMD/main_EN}}
\RU{\input{patterns/205_floating_SIMD/main_RU}}
\DE{\input{patterns/205_floating_SIMD/main_DE}}

\EN{\input{patterns/ARM/main_EN}}
\RU{\input{patterns/ARM/main_RU}}
\DE{\input{patterns/ARM/main_DE}}

\input{patterns/MIPS/main}

\ifdefined\SPANISH
\chapter{Patrones de código}
\fi % SPANISH

\ifdefined\GERMAN
\chapter{Code-Muster}
\fi % GERMAN

\ifdefined\ENGLISH
\chapter{Code Patterns}
\fi % ENGLISH

\ifdefined\ITALIAN
\chapter{Forme di codice}
\fi % ITALIAN

\ifdefined\RUSSIAN
\chapter{Образцы кода}
\fi % RUSSIAN

\ifdefined\BRAZILIAN
\chapter{Padrões de códigos}
\fi % BRAZILIAN

\ifdefined\THAI
\chapter{รูปแบบของโค้ด}
\fi % THAI

\ifdefined\FRENCH
\chapter{Modèle de code}
\fi % FRENCH

\ifdefined\POLISH
\chapter{\PLph{}}
\fi % POLISH

% sections
\EN{\input{patterns/patterns_opt_dbg_EN}}
\ES{\input{patterns/patterns_opt_dbg_ES}}
\ITA{\input{patterns/patterns_opt_dbg_ITA}}
\PTBR{\input{patterns/patterns_opt_dbg_PTBR}}
\RU{\input{patterns/patterns_opt_dbg_RU}}
\THA{\input{patterns/patterns_opt_dbg_THA}}
\DE{\input{patterns/patterns_opt_dbg_DE}}
\FR{\input{patterns/patterns_opt_dbg_FR}}
\PL{\input{patterns/patterns_opt_dbg_PL}}

\RU{\section{Некоторые базовые понятия}}
\EN{\section{Some basics}}
\DE{\section{Einige Grundlagen}}
\FR{\section{Quelques bases}}
\ES{\section{\ESph{}}}
\ITA{\section{Alcune basi teoriche}}
\PTBR{\section{\PTBRph{}}}
\THA{\section{\THAph{}}}
\PL{\section{\PLph{}}}

% sections:
\EN{\input{patterns/intro_CPU_ISA_EN}}
\ES{\input{patterns/intro_CPU_ISA_ES}}
\ITA{\input{patterns/intro_CPU_ISA_ITA}}
\PTBR{\input{patterns/intro_CPU_ISA_PTBR}}
\RU{\input{patterns/intro_CPU_ISA_RU}}
\DE{\input{patterns/intro_CPU_ISA_DE}}
\FR{\input{patterns/intro_CPU_ISA_FR}}
\PL{\input{patterns/intro_CPU_ISA_PL}}

\EN{\input{patterns/numeral_EN}}
\RU{\input{patterns/numeral_RU}}
\ITA{\input{patterns/numeral_ITA}}
\DE{\input{patterns/numeral_DE}}
\FR{\input{patterns/numeral_FR}}
\PL{\input{patterns/numeral_PL}}

% chapters
\input{patterns/00_empty/main}
\input{patterns/011_ret/main}
\input{patterns/01_helloworld/main}
\input{patterns/015_prolog_epilogue/main}
\input{patterns/02_stack/main}
\input{patterns/03_printf/main}
\input{patterns/04_scanf/main}
\input{patterns/05_passing_arguments/main}
\input{patterns/06_return_results/main}
\input{patterns/061_pointers/main}
\input{patterns/065_GOTO/main}
\input{patterns/07_jcc/main}
\input{patterns/08_switch/main}
\input{patterns/09_loops/main}
\input{patterns/10_strings/main}
\input{patterns/11_arith_optimizations/main}
\input{patterns/12_FPU/main}
\input{patterns/13_arrays/main}
\input{patterns/14_bitfields/main}
\EN{\input{patterns/145_LCG/main_EN}}
\RU{\input{patterns/145_LCG/main_RU}}
\input{patterns/15_structs/main}
\input{patterns/17_unions/main}
\input{patterns/18_pointers_to_functions/main}
\input{patterns/185_64bit_in_32_env/main}

\EN{\input{patterns/19_SIMD/main_EN}}
\RU{\input{patterns/19_SIMD/main_RU}}
\DE{\input{patterns/19_SIMD/main_DE}}

\EN{\input{patterns/20_x64/main_EN}}
\RU{\input{patterns/20_x64/main_RU}}

\EN{\input{patterns/205_floating_SIMD/main_EN}}
\RU{\input{patterns/205_floating_SIMD/main_RU}}
\DE{\input{patterns/205_floating_SIMD/main_DE}}

\EN{\input{patterns/ARM/main_EN}}
\RU{\input{patterns/ARM/main_RU}}
\DE{\input{patterns/ARM/main_DE}}

\input{patterns/MIPS/main}

\ifdefined\SPANISH
\chapter{Patrones de código}
\fi % SPANISH

\ifdefined\GERMAN
\chapter{Code-Muster}
\fi % GERMAN

\ifdefined\ENGLISH
\chapter{Code Patterns}
\fi % ENGLISH

\ifdefined\ITALIAN
\chapter{Forme di codice}
\fi % ITALIAN

\ifdefined\RUSSIAN
\chapter{Образцы кода}
\fi % RUSSIAN

\ifdefined\BRAZILIAN
\chapter{Padrões de códigos}
\fi % BRAZILIAN

\ifdefined\THAI
\chapter{รูปแบบของโค้ด}
\fi % THAI

\ifdefined\FRENCH
\chapter{Modèle de code}
\fi % FRENCH

\ifdefined\POLISH
\chapter{\PLph{}}
\fi % POLISH

% sections
\EN{\input{patterns/patterns_opt_dbg_EN}}
\ES{\input{patterns/patterns_opt_dbg_ES}}
\ITA{\input{patterns/patterns_opt_dbg_ITA}}
\PTBR{\input{patterns/patterns_opt_dbg_PTBR}}
\RU{\input{patterns/patterns_opt_dbg_RU}}
\THA{\input{patterns/patterns_opt_dbg_THA}}
\DE{\input{patterns/patterns_opt_dbg_DE}}
\FR{\input{patterns/patterns_opt_dbg_FR}}
\PL{\input{patterns/patterns_opt_dbg_PL}}

\RU{\section{Некоторые базовые понятия}}
\EN{\section{Some basics}}
\DE{\section{Einige Grundlagen}}
\FR{\section{Quelques bases}}
\ES{\section{\ESph{}}}
\ITA{\section{Alcune basi teoriche}}
\PTBR{\section{\PTBRph{}}}
\THA{\section{\THAph{}}}
\PL{\section{\PLph{}}}

% sections:
\EN{\input{patterns/intro_CPU_ISA_EN}}
\ES{\input{patterns/intro_CPU_ISA_ES}}
\ITA{\input{patterns/intro_CPU_ISA_ITA}}
\PTBR{\input{patterns/intro_CPU_ISA_PTBR}}
\RU{\input{patterns/intro_CPU_ISA_RU}}
\DE{\input{patterns/intro_CPU_ISA_DE}}
\FR{\input{patterns/intro_CPU_ISA_FR}}
\PL{\input{patterns/intro_CPU_ISA_PL}}

\EN{\input{patterns/numeral_EN}}
\RU{\input{patterns/numeral_RU}}
\ITA{\input{patterns/numeral_ITA}}
\DE{\input{patterns/numeral_DE}}
\FR{\input{patterns/numeral_FR}}
\PL{\input{patterns/numeral_PL}}

% chapters
\input{patterns/00_empty/main}
\input{patterns/011_ret/main}
\input{patterns/01_helloworld/main}
\input{patterns/015_prolog_epilogue/main}
\input{patterns/02_stack/main}
\input{patterns/03_printf/main}
\input{patterns/04_scanf/main}
\input{patterns/05_passing_arguments/main}
\input{patterns/06_return_results/main}
\input{patterns/061_pointers/main}
\input{patterns/065_GOTO/main}
\input{patterns/07_jcc/main}
\input{patterns/08_switch/main}
\input{patterns/09_loops/main}
\input{patterns/10_strings/main}
\input{patterns/11_arith_optimizations/main}
\input{patterns/12_FPU/main}
\input{patterns/13_arrays/main}
\input{patterns/14_bitfields/main}
\EN{\input{patterns/145_LCG/main_EN}}
\RU{\input{patterns/145_LCG/main_RU}}
\input{patterns/15_structs/main}
\input{patterns/17_unions/main}
\input{patterns/18_pointers_to_functions/main}
\input{patterns/185_64bit_in_32_env/main}

\EN{\input{patterns/19_SIMD/main_EN}}
\RU{\input{patterns/19_SIMD/main_RU}}
\DE{\input{patterns/19_SIMD/main_DE}}

\EN{\input{patterns/20_x64/main_EN}}
\RU{\input{patterns/20_x64/main_RU}}

\EN{\input{patterns/205_floating_SIMD/main_EN}}
\RU{\input{patterns/205_floating_SIMD/main_RU}}
\DE{\input{patterns/205_floating_SIMD/main_DE}}

\EN{\input{patterns/ARM/main_EN}}
\RU{\input{patterns/ARM/main_RU}}
\DE{\input{patterns/ARM/main_DE}}

\input{patterns/MIPS/main}

\ifdefined\SPANISH
\chapter{Patrones de código}
\fi % SPANISH

\ifdefined\GERMAN
\chapter{Code-Muster}
\fi % GERMAN

\ifdefined\ENGLISH
\chapter{Code Patterns}
\fi % ENGLISH

\ifdefined\ITALIAN
\chapter{Forme di codice}
\fi % ITALIAN

\ifdefined\RUSSIAN
\chapter{Образцы кода}
\fi % RUSSIAN

\ifdefined\BRAZILIAN
\chapter{Padrões de códigos}
\fi % BRAZILIAN

\ifdefined\THAI
\chapter{รูปแบบของโค้ด}
\fi % THAI

\ifdefined\FRENCH
\chapter{Modèle de code}
\fi % FRENCH

\ifdefined\POLISH
\chapter{\PLph{}}
\fi % POLISH

% sections
\EN{\input{patterns/patterns_opt_dbg_EN}}
\ES{\input{patterns/patterns_opt_dbg_ES}}
\ITA{\input{patterns/patterns_opt_dbg_ITA}}
\PTBR{\input{patterns/patterns_opt_dbg_PTBR}}
\RU{\input{patterns/patterns_opt_dbg_RU}}
\THA{\input{patterns/patterns_opt_dbg_THA}}
\DE{\input{patterns/patterns_opt_dbg_DE}}
\FR{\input{patterns/patterns_opt_dbg_FR}}
\PL{\input{patterns/patterns_opt_dbg_PL}}

\RU{\section{Некоторые базовые понятия}}
\EN{\section{Some basics}}
\DE{\section{Einige Grundlagen}}
\FR{\section{Quelques bases}}
\ES{\section{\ESph{}}}
\ITA{\section{Alcune basi teoriche}}
\PTBR{\section{\PTBRph{}}}
\THA{\section{\THAph{}}}
\PL{\section{\PLph{}}}

% sections:
\EN{\input{patterns/intro_CPU_ISA_EN}}
\ES{\input{patterns/intro_CPU_ISA_ES}}
\ITA{\input{patterns/intro_CPU_ISA_ITA}}
\PTBR{\input{patterns/intro_CPU_ISA_PTBR}}
\RU{\input{patterns/intro_CPU_ISA_RU}}
\DE{\input{patterns/intro_CPU_ISA_DE}}
\FR{\input{patterns/intro_CPU_ISA_FR}}
\PL{\input{patterns/intro_CPU_ISA_PL}}

\EN{\input{patterns/numeral_EN}}
\RU{\input{patterns/numeral_RU}}
\ITA{\input{patterns/numeral_ITA}}
\DE{\input{patterns/numeral_DE}}
\FR{\input{patterns/numeral_FR}}
\PL{\input{patterns/numeral_PL}}

% chapters
\input{patterns/00_empty/main}
\input{patterns/011_ret/main}
\input{patterns/01_helloworld/main}
\input{patterns/015_prolog_epilogue/main}
\input{patterns/02_stack/main}
\input{patterns/03_printf/main}
\input{patterns/04_scanf/main}
\input{patterns/05_passing_arguments/main}
\input{patterns/06_return_results/main}
\input{patterns/061_pointers/main}
\input{patterns/065_GOTO/main}
\input{patterns/07_jcc/main}
\input{patterns/08_switch/main}
\input{patterns/09_loops/main}
\input{patterns/10_strings/main}
\input{patterns/11_arith_optimizations/main}
\input{patterns/12_FPU/main}
\input{patterns/13_arrays/main}
\input{patterns/14_bitfields/main}
\EN{\input{patterns/145_LCG/main_EN}}
\RU{\input{patterns/145_LCG/main_RU}}
\input{patterns/15_structs/main}
\input{patterns/17_unions/main}
\input{patterns/18_pointers_to_functions/main}
\input{patterns/185_64bit_in_32_env/main}

\EN{\input{patterns/19_SIMD/main_EN}}
\RU{\input{patterns/19_SIMD/main_RU}}
\DE{\input{patterns/19_SIMD/main_DE}}

\EN{\input{patterns/20_x64/main_EN}}
\RU{\input{patterns/20_x64/main_RU}}

\EN{\input{patterns/205_floating_SIMD/main_EN}}
\RU{\input{patterns/205_floating_SIMD/main_RU}}
\DE{\input{patterns/205_floating_SIMD/main_DE}}

\EN{\input{patterns/ARM/main_EN}}
\RU{\input{patterns/ARM/main_RU}}
\DE{\input{patterns/ARM/main_DE}}

\input{patterns/MIPS/main}

\ifdefined\SPANISH
\chapter{Patrones de código}
\fi % SPANISH

\ifdefined\GERMAN
\chapter{Code-Muster}
\fi % GERMAN

\ifdefined\ENGLISH
\chapter{Code Patterns}
\fi % ENGLISH

\ifdefined\ITALIAN
\chapter{Forme di codice}
\fi % ITALIAN

\ifdefined\RUSSIAN
\chapter{Образцы кода}
\fi % RUSSIAN

\ifdefined\BRAZILIAN
\chapter{Padrões de códigos}
\fi % BRAZILIAN

\ifdefined\THAI
\chapter{รูปแบบของโค้ด}
\fi % THAI

\ifdefined\FRENCH
\chapter{Modèle de code}
\fi % FRENCH

\ifdefined\POLISH
\chapter{\PLph{}}
\fi % POLISH

% sections
\EN{\input{patterns/patterns_opt_dbg_EN}}
\ES{\input{patterns/patterns_opt_dbg_ES}}
\ITA{\input{patterns/patterns_opt_dbg_ITA}}
\PTBR{\input{patterns/patterns_opt_dbg_PTBR}}
\RU{\input{patterns/patterns_opt_dbg_RU}}
\THA{\input{patterns/patterns_opt_dbg_THA}}
\DE{\input{patterns/patterns_opt_dbg_DE}}
\FR{\input{patterns/patterns_opt_dbg_FR}}
\PL{\input{patterns/patterns_opt_dbg_PL}}

\RU{\section{Некоторые базовые понятия}}
\EN{\section{Some basics}}
\DE{\section{Einige Grundlagen}}
\FR{\section{Quelques bases}}
\ES{\section{\ESph{}}}
\ITA{\section{Alcune basi teoriche}}
\PTBR{\section{\PTBRph{}}}
\THA{\section{\THAph{}}}
\PL{\section{\PLph{}}}

% sections:
\EN{\input{patterns/intro_CPU_ISA_EN}}
\ES{\input{patterns/intro_CPU_ISA_ES}}
\ITA{\input{patterns/intro_CPU_ISA_ITA}}
\PTBR{\input{patterns/intro_CPU_ISA_PTBR}}
\RU{\input{patterns/intro_CPU_ISA_RU}}
\DE{\input{patterns/intro_CPU_ISA_DE}}
\FR{\input{patterns/intro_CPU_ISA_FR}}
\PL{\input{patterns/intro_CPU_ISA_PL}}

\EN{\input{patterns/numeral_EN}}
\RU{\input{patterns/numeral_RU}}
\ITA{\input{patterns/numeral_ITA}}
\DE{\input{patterns/numeral_DE}}
\FR{\input{patterns/numeral_FR}}
\PL{\input{patterns/numeral_PL}}

% chapters
\input{patterns/00_empty/main}
\input{patterns/011_ret/main}
\input{patterns/01_helloworld/main}
\input{patterns/015_prolog_epilogue/main}
\input{patterns/02_stack/main}
\input{patterns/03_printf/main}
\input{patterns/04_scanf/main}
\input{patterns/05_passing_arguments/main}
\input{patterns/06_return_results/main}
\input{patterns/061_pointers/main}
\input{patterns/065_GOTO/main}
\input{patterns/07_jcc/main}
\input{patterns/08_switch/main}
\input{patterns/09_loops/main}
\input{patterns/10_strings/main}
\input{patterns/11_arith_optimizations/main}
\input{patterns/12_FPU/main}
\input{patterns/13_arrays/main}
\input{patterns/14_bitfields/main}
\EN{\input{patterns/145_LCG/main_EN}}
\RU{\input{patterns/145_LCG/main_RU}}
\input{patterns/15_structs/main}
\input{patterns/17_unions/main}
\input{patterns/18_pointers_to_functions/main}
\input{patterns/185_64bit_in_32_env/main}

\EN{\input{patterns/19_SIMD/main_EN}}
\RU{\input{patterns/19_SIMD/main_RU}}
\DE{\input{patterns/19_SIMD/main_DE}}

\EN{\input{patterns/20_x64/main_EN}}
\RU{\input{patterns/20_x64/main_RU}}

\EN{\input{patterns/205_floating_SIMD/main_EN}}
\RU{\input{patterns/205_floating_SIMD/main_RU}}
\DE{\input{patterns/205_floating_SIMD/main_DE}}

\EN{\input{patterns/ARM/main_EN}}
\RU{\input{patterns/ARM/main_RU}}
\DE{\input{patterns/ARM/main_DE}}

\input{patterns/MIPS/main}

\ifdefined\SPANISH
\chapter{Patrones de código}
\fi % SPANISH

\ifdefined\GERMAN
\chapter{Code-Muster}
\fi % GERMAN

\ifdefined\ENGLISH
\chapter{Code Patterns}
\fi % ENGLISH

\ifdefined\ITALIAN
\chapter{Forme di codice}
\fi % ITALIAN

\ifdefined\RUSSIAN
\chapter{Образцы кода}
\fi % RUSSIAN

\ifdefined\BRAZILIAN
\chapter{Padrões de códigos}
\fi % BRAZILIAN

\ifdefined\THAI
\chapter{รูปแบบของโค้ด}
\fi % THAI

\ifdefined\FRENCH
\chapter{Modèle de code}
\fi % FRENCH

\ifdefined\POLISH
\chapter{\PLph{}}
\fi % POLISH

% sections
\EN{\input{patterns/patterns_opt_dbg_EN}}
\ES{\input{patterns/patterns_opt_dbg_ES}}
\ITA{\input{patterns/patterns_opt_dbg_ITA}}
\PTBR{\input{patterns/patterns_opt_dbg_PTBR}}
\RU{\input{patterns/patterns_opt_dbg_RU}}
\THA{\input{patterns/patterns_opt_dbg_THA}}
\DE{\input{patterns/patterns_opt_dbg_DE}}
\FR{\input{patterns/patterns_opt_dbg_FR}}
\PL{\input{patterns/patterns_opt_dbg_PL}}

\RU{\section{Некоторые базовые понятия}}
\EN{\section{Some basics}}
\DE{\section{Einige Grundlagen}}
\FR{\section{Quelques bases}}
\ES{\section{\ESph{}}}
\ITA{\section{Alcune basi teoriche}}
\PTBR{\section{\PTBRph{}}}
\THA{\section{\THAph{}}}
\PL{\section{\PLph{}}}

% sections:
\EN{\input{patterns/intro_CPU_ISA_EN}}
\ES{\input{patterns/intro_CPU_ISA_ES}}
\ITA{\input{patterns/intro_CPU_ISA_ITA}}
\PTBR{\input{patterns/intro_CPU_ISA_PTBR}}
\RU{\input{patterns/intro_CPU_ISA_RU}}
\DE{\input{patterns/intro_CPU_ISA_DE}}
\FR{\input{patterns/intro_CPU_ISA_FR}}
\PL{\input{patterns/intro_CPU_ISA_PL}}

\EN{\input{patterns/numeral_EN}}
\RU{\input{patterns/numeral_RU}}
\ITA{\input{patterns/numeral_ITA}}
\DE{\input{patterns/numeral_DE}}
\FR{\input{patterns/numeral_FR}}
\PL{\input{patterns/numeral_PL}}

% chapters
\input{patterns/00_empty/main}
\input{patterns/011_ret/main}
\input{patterns/01_helloworld/main}
\input{patterns/015_prolog_epilogue/main}
\input{patterns/02_stack/main}
\input{patterns/03_printf/main}
\input{patterns/04_scanf/main}
\input{patterns/05_passing_arguments/main}
\input{patterns/06_return_results/main}
\input{patterns/061_pointers/main}
\input{patterns/065_GOTO/main}
\input{patterns/07_jcc/main}
\input{patterns/08_switch/main}
\input{patterns/09_loops/main}
\input{patterns/10_strings/main}
\input{patterns/11_arith_optimizations/main}
\input{patterns/12_FPU/main}
\input{patterns/13_arrays/main}
\input{patterns/14_bitfields/main}
\EN{\input{patterns/145_LCG/main_EN}}
\RU{\input{patterns/145_LCG/main_RU}}
\input{patterns/15_structs/main}
\input{patterns/17_unions/main}
\input{patterns/18_pointers_to_functions/main}
\input{patterns/185_64bit_in_32_env/main}

\EN{\input{patterns/19_SIMD/main_EN}}
\RU{\input{patterns/19_SIMD/main_RU}}
\DE{\input{patterns/19_SIMD/main_DE}}

\EN{\input{patterns/20_x64/main_EN}}
\RU{\input{patterns/20_x64/main_RU}}

\EN{\input{patterns/205_floating_SIMD/main_EN}}
\RU{\input{patterns/205_floating_SIMD/main_RU}}
\DE{\input{patterns/205_floating_SIMD/main_DE}}

\EN{\input{patterns/ARM/main_EN}}
\RU{\input{patterns/ARM/main_RU}}
\DE{\input{patterns/ARM/main_DE}}

\input{patterns/MIPS/main}

\ifdefined\SPANISH
\chapter{Patrones de código}
\fi % SPANISH

\ifdefined\GERMAN
\chapter{Code-Muster}
\fi % GERMAN

\ifdefined\ENGLISH
\chapter{Code Patterns}
\fi % ENGLISH

\ifdefined\ITALIAN
\chapter{Forme di codice}
\fi % ITALIAN

\ifdefined\RUSSIAN
\chapter{Образцы кода}
\fi % RUSSIAN

\ifdefined\BRAZILIAN
\chapter{Padrões de códigos}
\fi % BRAZILIAN

\ifdefined\THAI
\chapter{รูปแบบของโค้ด}
\fi % THAI

\ifdefined\FRENCH
\chapter{Modèle de code}
\fi % FRENCH

\ifdefined\POLISH
\chapter{\PLph{}}
\fi % POLISH

% sections
\EN{\input{patterns/patterns_opt_dbg_EN}}
\ES{\input{patterns/patterns_opt_dbg_ES}}
\ITA{\input{patterns/patterns_opt_dbg_ITA}}
\PTBR{\input{patterns/patterns_opt_dbg_PTBR}}
\RU{\input{patterns/patterns_opt_dbg_RU}}
\THA{\input{patterns/patterns_opt_dbg_THA}}
\DE{\input{patterns/patterns_opt_dbg_DE}}
\FR{\input{patterns/patterns_opt_dbg_FR}}
\PL{\input{patterns/patterns_opt_dbg_PL}}

\RU{\section{Некоторые базовые понятия}}
\EN{\section{Some basics}}
\DE{\section{Einige Grundlagen}}
\FR{\section{Quelques bases}}
\ES{\section{\ESph{}}}
\ITA{\section{Alcune basi teoriche}}
\PTBR{\section{\PTBRph{}}}
\THA{\section{\THAph{}}}
\PL{\section{\PLph{}}}

% sections:
\EN{\input{patterns/intro_CPU_ISA_EN}}
\ES{\input{patterns/intro_CPU_ISA_ES}}
\ITA{\input{patterns/intro_CPU_ISA_ITA}}
\PTBR{\input{patterns/intro_CPU_ISA_PTBR}}
\RU{\input{patterns/intro_CPU_ISA_RU}}
\DE{\input{patterns/intro_CPU_ISA_DE}}
\FR{\input{patterns/intro_CPU_ISA_FR}}
\PL{\input{patterns/intro_CPU_ISA_PL}}

\EN{\input{patterns/numeral_EN}}
\RU{\input{patterns/numeral_RU}}
\ITA{\input{patterns/numeral_ITA}}
\DE{\input{patterns/numeral_DE}}
\FR{\input{patterns/numeral_FR}}
\PL{\input{patterns/numeral_PL}}

% chapters
\input{patterns/00_empty/main}
\input{patterns/011_ret/main}
\input{patterns/01_helloworld/main}
\input{patterns/015_prolog_epilogue/main}
\input{patterns/02_stack/main}
\input{patterns/03_printf/main}
\input{patterns/04_scanf/main}
\input{patterns/05_passing_arguments/main}
\input{patterns/06_return_results/main}
\input{patterns/061_pointers/main}
\input{patterns/065_GOTO/main}
\input{patterns/07_jcc/main}
\input{patterns/08_switch/main}
\input{patterns/09_loops/main}
\input{patterns/10_strings/main}
\input{patterns/11_arith_optimizations/main}
\input{patterns/12_FPU/main}
\input{patterns/13_arrays/main}
\input{patterns/14_bitfields/main}
\EN{\input{patterns/145_LCG/main_EN}}
\RU{\input{patterns/145_LCG/main_RU}}
\input{patterns/15_structs/main}
\input{patterns/17_unions/main}
\input{patterns/18_pointers_to_functions/main}
\input{patterns/185_64bit_in_32_env/main}

\EN{\input{patterns/19_SIMD/main_EN}}
\RU{\input{patterns/19_SIMD/main_RU}}
\DE{\input{patterns/19_SIMD/main_DE}}

\EN{\input{patterns/20_x64/main_EN}}
\RU{\input{patterns/20_x64/main_RU}}

\EN{\input{patterns/205_floating_SIMD/main_EN}}
\RU{\input{patterns/205_floating_SIMD/main_RU}}
\DE{\input{patterns/205_floating_SIMD/main_DE}}

\EN{\input{patterns/ARM/main_EN}}
\RU{\input{patterns/ARM/main_RU}}
\DE{\input{patterns/ARM/main_DE}}

\input{patterns/MIPS/main}

\ifdefined\SPANISH
\chapter{Patrones de código}
\fi % SPANISH

\ifdefined\GERMAN
\chapter{Code-Muster}
\fi % GERMAN

\ifdefined\ENGLISH
\chapter{Code Patterns}
\fi % ENGLISH

\ifdefined\ITALIAN
\chapter{Forme di codice}
\fi % ITALIAN

\ifdefined\RUSSIAN
\chapter{Образцы кода}
\fi % RUSSIAN

\ifdefined\BRAZILIAN
\chapter{Padrões de códigos}
\fi % BRAZILIAN

\ifdefined\THAI
\chapter{รูปแบบของโค้ด}
\fi % THAI

\ifdefined\FRENCH
\chapter{Modèle de code}
\fi % FRENCH

\ifdefined\POLISH
\chapter{\PLph{}}
\fi % POLISH

% sections
\EN{\input{patterns/patterns_opt_dbg_EN}}
\ES{\input{patterns/patterns_opt_dbg_ES}}
\ITA{\input{patterns/patterns_opt_dbg_ITA}}
\PTBR{\input{patterns/patterns_opt_dbg_PTBR}}
\RU{\input{patterns/patterns_opt_dbg_RU}}
\THA{\input{patterns/patterns_opt_dbg_THA}}
\DE{\input{patterns/patterns_opt_dbg_DE}}
\FR{\input{patterns/patterns_opt_dbg_FR}}
\PL{\input{patterns/patterns_opt_dbg_PL}}

\RU{\section{Некоторые базовые понятия}}
\EN{\section{Some basics}}
\DE{\section{Einige Grundlagen}}
\FR{\section{Quelques bases}}
\ES{\section{\ESph{}}}
\ITA{\section{Alcune basi teoriche}}
\PTBR{\section{\PTBRph{}}}
\THA{\section{\THAph{}}}
\PL{\section{\PLph{}}}

% sections:
\EN{\input{patterns/intro_CPU_ISA_EN}}
\ES{\input{patterns/intro_CPU_ISA_ES}}
\ITA{\input{patterns/intro_CPU_ISA_ITA}}
\PTBR{\input{patterns/intro_CPU_ISA_PTBR}}
\RU{\input{patterns/intro_CPU_ISA_RU}}
\DE{\input{patterns/intro_CPU_ISA_DE}}
\FR{\input{patterns/intro_CPU_ISA_FR}}
\PL{\input{patterns/intro_CPU_ISA_PL}}

\EN{\input{patterns/numeral_EN}}
\RU{\input{patterns/numeral_RU}}
\ITA{\input{patterns/numeral_ITA}}
\DE{\input{patterns/numeral_DE}}
\FR{\input{patterns/numeral_FR}}
\PL{\input{patterns/numeral_PL}}

% chapters
\input{patterns/00_empty/main}
\input{patterns/011_ret/main}
\input{patterns/01_helloworld/main}
\input{patterns/015_prolog_epilogue/main}
\input{patterns/02_stack/main}
\input{patterns/03_printf/main}
\input{patterns/04_scanf/main}
\input{patterns/05_passing_arguments/main}
\input{patterns/06_return_results/main}
\input{patterns/061_pointers/main}
\input{patterns/065_GOTO/main}
\input{patterns/07_jcc/main}
\input{patterns/08_switch/main}
\input{patterns/09_loops/main}
\input{patterns/10_strings/main}
\input{patterns/11_arith_optimizations/main}
\input{patterns/12_FPU/main}
\input{patterns/13_arrays/main}
\input{patterns/14_bitfields/main}
\EN{\input{patterns/145_LCG/main_EN}}
\RU{\input{patterns/145_LCG/main_RU}}
\input{patterns/15_structs/main}
\input{patterns/17_unions/main}
\input{patterns/18_pointers_to_functions/main}
\input{patterns/185_64bit_in_32_env/main}

\EN{\input{patterns/19_SIMD/main_EN}}
\RU{\input{patterns/19_SIMD/main_RU}}
\DE{\input{patterns/19_SIMD/main_DE}}

\EN{\input{patterns/20_x64/main_EN}}
\RU{\input{patterns/20_x64/main_RU}}

\EN{\input{patterns/205_floating_SIMD/main_EN}}
\RU{\input{patterns/205_floating_SIMD/main_RU}}
\DE{\input{patterns/205_floating_SIMD/main_DE}}

\EN{\input{patterns/ARM/main_EN}}
\RU{\input{patterns/ARM/main_RU}}
\DE{\input{patterns/ARM/main_DE}}

\input{patterns/MIPS/main}

\ifdefined\SPANISH
\chapter{Patrones de código}
\fi % SPANISH

\ifdefined\GERMAN
\chapter{Code-Muster}
\fi % GERMAN

\ifdefined\ENGLISH
\chapter{Code Patterns}
\fi % ENGLISH

\ifdefined\ITALIAN
\chapter{Forme di codice}
\fi % ITALIAN

\ifdefined\RUSSIAN
\chapter{Образцы кода}
\fi % RUSSIAN

\ifdefined\BRAZILIAN
\chapter{Padrões de códigos}
\fi % BRAZILIAN

\ifdefined\THAI
\chapter{รูปแบบของโค้ด}
\fi % THAI

\ifdefined\FRENCH
\chapter{Modèle de code}
\fi % FRENCH

\ifdefined\POLISH
\chapter{\PLph{}}
\fi % POLISH

% sections
\EN{\input{patterns/patterns_opt_dbg_EN}}
\ES{\input{patterns/patterns_opt_dbg_ES}}
\ITA{\input{patterns/patterns_opt_dbg_ITA}}
\PTBR{\input{patterns/patterns_opt_dbg_PTBR}}
\RU{\input{patterns/patterns_opt_dbg_RU}}
\THA{\input{patterns/patterns_opt_dbg_THA}}
\DE{\input{patterns/patterns_opt_dbg_DE}}
\FR{\input{patterns/patterns_opt_dbg_FR}}
\PL{\input{patterns/patterns_opt_dbg_PL}}

\RU{\section{Некоторые базовые понятия}}
\EN{\section{Some basics}}
\DE{\section{Einige Grundlagen}}
\FR{\section{Quelques bases}}
\ES{\section{\ESph{}}}
\ITA{\section{Alcune basi teoriche}}
\PTBR{\section{\PTBRph{}}}
\THA{\section{\THAph{}}}
\PL{\section{\PLph{}}}

% sections:
\EN{\input{patterns/intro_CPU_ISA_EN}}
\ES{\input{patterns/intro_CPU_ISA_ES}}
\ITA{\input{patterns/intro_CPU_ISA_ITA}}
\PTBR{\input{patterns/intro_CPU_ISA_PTBR}}
\RU{\input{patterns/intro_CPU_ISA_RU}}
\DE{\input{patterns/intro_CPU_ISA_DE}}
\FR{\input{patterns/intro_CPU_ISA_FR}}
\PL{\input{patterns/intro_CPU_ISA_PL}}

\EN{\input{patterns/numeral_EN}}
\RU{\input{patterns/numeral_RU}}
\ITA{\input{patterns/numeral_ITA}}
\DE{\input{patterns/numeral_DE}}
\FR{\input{patterns/numeral_FR}}
\PL{\input{patterns/numeral_PL}}

% chapters
\input{patterns/00_empty/main}
\input{patterns/011_ret/main}
\input{patterns/01_helloworld/main}
\input{patterns/015_prolog_epilogue/main}
\input{patterns/02_stack/main}
\input{patterns/03_printf/main}
\input{patterns/04_scanf/main}
\input{patterns/05_passing_arguments/main}
\input{patterns/06_return_results/main}
\input{patterns/061_pointers/main}
\input{patterns/065_GOTO/main}
\input{patterns/07_jcc/main}
\input{patterns/08_switch/main}
\input{patterns/09_loops/main}
\input{patterns/10_strings/main}
\input{patterns/11_arith_optimizations/main}
\input{patterns/12_FPU/main}
\input{patterns/13_arrays/main}
\input{patterns/14_bitfields/main}
\EN{\input{patterns/145_LCG/main_EN}}
\RU{\input{patterns/145_LCG/main_RU}}
\input{patterns/15_structs/main}
\input{patterns/17_unions/main}
\input{patterns/18_pointers_to_functions/main}
\input{patterns/185_64bit_in_32_env/main}

\EN{\input{patterns/19_SIMD/main_EN}}
\RU{\input{patterns/19_SIMD/main_RU}}
\DE{\input{patterns/19_SIMD/main_DE}}

\EN{\input{patterns/20_x64/main_EN}}
\RU{\input{patterns/20_x64/main_RU}}

\EN{\input{patterns/205_floating_SIMD/main_EN}}
\RU{\input{patterns/205_floating_SIMD/main_RU}}
\DE{\input{patterns/205_floating_SIMD/main_DE}}

\EN{\input{patterns/ARM/main_EN}}
\RU{\input{patterns/ARM/main_RU}}
\DE{\input{patterns/ARM/main_DE}}

\input{patterns/MIPS/main}

\EN{\input{patterns/12_FPU/main_EN}}
\RU{\input{patterns/12_FPU/main_RU}}
\DE{\input{patterns/12_FPU/main_DE}}
\FR{\input{patterns/12_FPU/main_FR}}


\ifdefined\SPANISH
\chapter{Patrones de código}
\fi % SPANISH

\ifdefined\GERMAN
\chapter{Code-Muster}
\fi % GERMAN

\ifdefined\ENGLISH
\chapter{Code Patterns}
\fi % ENGLISH

\ifdefined\ITALIAN
\chapter{Forme di codice}
\fi % ITALIAN

\ifdefined\RUSSIAN
\chapter{Образцы кода}
\fi % RUSSIAN

\ifdefined\BRAZILIAN
\chapter{Padrões de códigos}
\fi % BRAZILIAN

\ifdefined\THAI
\chapter{รูปแบบของโค้ด}
\fi % THAI

\ifdefined\FRENCH
\chapter{Modèle de code}
\fi % FRENCH

\ifdefined\POLISH
\chapter{\PLph{}}
\fi % POLISH

% sections
\EN{\input{patterns/patterns_opt_dbg_EN}}
\ES{\input{patterns/patterns_opt_dbg_ES}}
\ITA{\input{patterns/patterns_opt_dbg_ITA}}
\PTBR{\input{patterns/patterns_opt_dbg_PTBR}}
\RU{\input{patterns/patterns_opt_dbg_RU}}
\THA{\input{patterns/patterns_opt_dbg_THA}}
\DE{\input{patterns/patterns_opt_dbg_DE}}
\FR{\input{patterns/patterns_opt_dbg_FR}}
\PL{\input{patterns/patterns_opt_dbg_PL}}

\RU{\section{Некоторые базовые понятия}}
\EN{\section{Some basics}}
\DE{\section{Einige Grundlagen}}
\FR{\section{Quelques bases}}
\ES{\section{\ESph{}}}
\ITA{\section{Alcune basi teoriche}}
\PTBR{\section{\PTBRph{}}}
\THA{\section{\THAph{}}}
\PL{\section{\PLph{}}}

% sections:
\EN{\input{patterns/intro_CPU_ISA_EN}}
\ES{\input{patterns/intro_CPU_ISA_ES}}
\ITA{\input{patterns/intro_CPU_ISA_ITA}}
\PTBR{\input{patterns/intro_CPU_ISA_PTBR}}
\RU{\input{patterns/intro_CPU_ISA_RU}}
\DE{\input{patterns/intro_CPU_ISA_DE}}
\FR{\input{patterns/intro_CPU_ISA_FR}}
\PL{\input{patterns/intro_CPU_ISA_PL}}

\EN{\input{patterns/numeral_EN}}
\RU{\input{patterns/numeral_RU}}
\ITA{\input{patterns/numeral_ITA}}
\DE{\input{patterns/numeral_DE}}
\FR{\input{patterns/numeral_FR}}
\PL{\input{patterns/numeral_PL}}

% chapters
\input{patterns/00_empty/main}
\input{patterns/011_ret/main}
\input{patterns/01_helloworld/main}
\input{patterns/015_prolog_epilogue/main}
\input{patterns/02_stack/main}
\input{patterns/03_printf/main}
\input{patterns/04_scanf/main}
\input{patterns/05_passing_arguments/main}
\input{patterns/06_return_results/main}
\input{patterns/061_pointers/main}
\input{patterns/065_GOTO/main}
\input{patterns/07_jcc/main}
\input{patterns/08_switch/main}
\input{patterns/09_loops/main}
\input{patterns/10_strings/main}
\input{patterns/11_arith_optimizations/main}
\input{patterns/12_FPU/main}
\input{patterns/13_arrays/main}
\input{patterns/14_bitfields/main}
\EN{\input{patterns/145_LCG/main_EN}}
\RU{\input{patterns/145_LCG/main_RU}}
\input{patterns/15_structs/main}
\input{patterns/17_unions/main}
\input{patterns/18_pointers_to_functions/main}
\input{patterns/185_64bit_in_32_env/main}

\EN{\input{patterns/19_SIMD/main_EN}}
\RU{\input{patterns/19_SIMD/main_RU}}
\DE{\input{patterns/19_SIMD/main_DE}}

\EN{\input{patterns/20_x64/main_EN}}
\RU{\input{patterns/20_x64/main_RU}}

\EN{\input{patterns/205_floating_SIMD/main_EN}}
\RU{\input{patterns/205_floating_SIMD/main_RU}}
\DE{\input{patterns/205_floating_SIMD/main_DE}}

\EN{\input{patterns/ARM/main_EN}}
\RU{\input{patterns/ARM/main_RU}}
\DE{\input{patterns/ARM/main_DE}}

\input{patterns/MIPS/main}

\ifdefined\SPANISH
\chapter{Patrones de código}
\fi % SPANISH

\ifdefined\GERMAN
\chapter{Code-Muster}
\fi % GERMAN

\ifdefined\ENGLISH
\chapter{Code Patterns}
\fi % ENGLISH

\ifdefined\ITALIAN
\chapter{Forme di codice}
\fi % ITALIAN

\ifdefined\RUSSIAN
\chapter{Образцы кода}
\fi % RUSSIAN

\ifdefined\BRAZILIAN
\chapter{Padrões de códigos}
\fi % BRAZILIAN

\ifdefined\THAI
\chapter{รูปแบบของโค้ด}
\fi % THAI

\ifdefined\FRENCH
\chapter{Modèle de code}
\fi % FRENCH

\ifdefined\POLISH
\chapter{\PLph{}}
\fi % POLISH

% sections
\EN{\input{patterns/patterns_opt_dbg_EN}}
\ES{\input{patterns/patterns_opt_dbg_ES}}
\ITA{\input{patterns/patterns_opt_dbg_ITA}}
\PTBR{\input{patterns/patterns_opt_dbg_PTBR}}
\RU{\input{patterns/patterns_opt_dbg_RU}}
\THA{\input{patterns/patterns_opt_dbg_THA}}
\DE{\input{patterns/patterns_opt_dbg_DE}}
\FR{\input{patterns/patterns_opt_dbg_FR}}
\PL{\input{patterns/patterns_opt_dbg_PL}}

\RU{\section{Некоторые базовые понятия}}
\EN{\section{Some basics}}
\DE{\section{Einige Grundlagen}}
\FR{\section{Quelques bases}}
\ES{\section{\ESph{}}}
\ITA{\section{Alcune basi teoriche}}
\PTBR{\section{\PTBRph{}}}
\THA{\section{\THAph{}}}
\PL{\section{\PLph{}}}

% sections:
\EN{\input{patterns/intro_CPU_ISA_EN}}
\ES{\input{patterns/intro_CPU_ISA_ES}}
\ITA{\input{patterns/intro_CPU_ISA_ITA}}
\PTBR{\input{patterns/intro_CPU_ISA_PTBR}}
\RU{\input{patterns/intro_CPU_ISA_RU}}
\DE{\input{patterns/intro_CPU_ISA_DE}}
\FR{\input{patterns/intro_CPU_ISA_FR}}
\PL{\input{patterns/intro_CPU_ISA_PL}}

\EN{\input{patterns/numeral_EN}}
\RU{\input{patterns/numeral_RU}}
\ITA{\input{patterns/numeral_ITA}}
\DE{\input{patterns/numeral_DE}}
\FR{\input{patterns/numeral_FR}}
\PL{\input{patterns/numeral_PL}}

% chapters
\input{patterns/00_empty/main}
\input{patterns/011_ret/main}
\input{patterns/01_helloworld/main}
\input{patterns/015_prolog_epilogue/main}
\input{patterns/02_stack/main}
\input{patterns/03_printf/main}
\input{patterns/04_scanf/main}
\input{patterns/05_passing_arguments/main}
\input{patterns/06_return_results/main}
\input{patterns/061_pointers/main}
\input{patterns/065_GOTO/main}
\input{patterns/07_jcc/main}
\input{patterns/08_switch/main}
\input{patterns/09_loops/main}
\input{patterns/10_strings/main}
\input{patterns/11_arith_optimizations/main}
\input{patterns/12_FPU/main}
\input{patterns/13_arrays/main}
\input{patterns/14_bitfields/main}
\EN{\input{patterns/145_LCG/main_EN}}
\RU{\input{patterns/145_LCG/main_RU}}
\input{patterns/15_structs/main}
\input{patterns/17_unions/main}
\input{patterns/18_pointers_to_functions/main}
\input{patterns/185_64bit_in_32_env/main}

\EN{\input{patterns/19_SIMD/main_EN}}
\RU{\input{patterns/19_SIMD/main_RU}}
\DE{\input{patterns/19_SIMD/main_DE}}

\EN{\input{patterns/20_x64/main_EN}}
\RU{\input{patterns/20_x64/main_RU}}

\EN{\input{patterns/205_floating_SIMD/main_EN}}
\RU{\input{patterns/205_floating_SIMD/main_RU}}
\DE{\input{patterns/205_floating_SIMD/main_DE}}

\EN{\input{patterns/ARM/main_EN}}
\RU{\input{patterns/ARM/main_RU}}
\DE{\input{patterns/ARM/main_DE}}

\input{patterns/MIPS/main}

\EN{\section{Returning Values}
\label{ret_val_func}

Another simple function is the one that simply returns a constant value:

\lstinputlisting[caption=\EN{\CCpp Code},style=customc]{patterns/011_ret/1.c}

Let's compile it.

\subsection{x86}

Here's what both the GCC and MSVC compilers produce (with optimization) on the x86 platform:

\lstinputlisting[caption=\Optimizing GCC/MSVC (\assemblyOutput),style=customasmx86]{patterns/011_ret/1.s}

\myindex{x86!\Instructions!RET}
There are just two instructions: the first places the value 123 into the \EAX register,
which is used by convention for storing the return
value, and the second one is \RET, which returns execution to the \gls{caller}.

The caller will take the result from the \EAX register.

\subsection{ARM}

There are a few differences on the ARM platform:

\lstinputlisting[caption=\OptimizingKeilVI (\ARMMode) ASM Output,style=customasmARM]{patterns/011_ret/1_Keil_ARM_O3.s}

ARM uses the register \Reg{0} for returning the results of functions, so 123 is copied into \Reg{0}.

\myindex{ARM!\Instructions!MOV}
\myindex{x86!\Instructions!MOV}
It is worth noting that \MOV is a misleading name for the instruction in both the x86 and ARM \ac{ISA}s.

The data is not in fact \IT{moved}, but \IT{copied}.

\subsection{MIPS}

\label{MIPS_leaf_function_ex1}

The GCC assembly output below lists registers by number:

\lstinputlisting[caption=\Optimizing GCC 4.4.5 (\assemblyOutput),style=customasmMIPS]{patterns/011_ret/MIPS.s}

\dots while \IDA does it by their pseudo names:

\lstinputlisting[caption=\Optimizing GCC 4.4.5 (IDA),style=customasmMIPS]{patterns/011_ret/MIPS_IDA.lst}

The \$2 (or \$V0) register is used to store the function's return value.
\myindex{MIPS!\Pseudoinstructions!LI}
\INS{LI} stands for ``Load Immediate'' and is the MIPS equivalent to \MOV.

\myindex{MIPS!\Instructions!J}
The other instruction is the jump instruction (J or JR) which returns the execution flow to the \gls{caller}.

\myindex{MIPS!Branch delay slot}
You might be wondering why the positions of the load instruction (LI) and the jump instruction (J or JR) are swapped. This is due to a \ac{RISC} feature called ``branch delay slot''.

The reason this happens is a quirk in the architecture of some RISC \ac{ISA}s and isn't important for our
purposes---we must simply keep in mind that in MIPS, the instruction following a jump or branch instruction
is executed \IT{before} the jump/branch instruction itself.

As a consequence, branch instructions always swap places with the instruction executed immediately beforehand.


In practice, functions which merely return 1 (\IT{true}) or 0 (\IT{false}) are very frequent.

The smallest ever of the standard UNIX utilities, \IT{/bin/true} and \IT{/bin/false} return 0 and 1 respectively, as an exit code.
(Zero as an exit code usually means success, non-zero means error.)
}
\RU{\subsubsection{std::string}
\myindex{\Cpp!STL!std::string}
\label{std_string}

\myparagraph{Как устроена структура}

Многие строковые библиотеки \InSqBrackets{\CNotes 2.2} обеспечивают структуру содержащую ссылку 
на буфер собственно со строкой, переменная всегда содержащую длину строки 
(что очень удобно для массы функций \InSqBrackets{\CNotes 2.2.1}) и переменную содержащую текущий размер буфера.

Строка в буфере обыкновенно оканчивается нулем: это для того чтобы указатель на буфер можно было
передавать в функции требующие на вход обычную сишную \ac{ASCIIZ}-строку.

Стандарт \Cpp не описывает, как именно нужно реализовывать std::string,
но, как правило, они реализованы как описано выше, с небольшими дополнениями.

Строки в \Cpp это не класс (как, например, QString в Qt), а темплейт (basic\_string), 
это сделано для того чтобы поддерживать 
строки содержащие разного типа символы: как минимум \Tchar и \IT{wchar\_t}.

Так что, std::string это класс с базовым типом \Tchar.

А std::wstring это класс с базовым типом \IT{wchar\_t}.

\mysubparagraph{MSVC}

В реализации MSVC, вместо ссылки на буфер может содержаться сам буфер (если строка короче 16-и символов).

Это означает, что каждая короткая строка будет занимать в памяти по крайней мере $16 + 4 + 4 = 24$ 
байт для 32-битной среды либо $16 + 8 + 8 = 32$ 
байта в 64-битной, а если строка длиннее 16-и символов, то прибавьте еще длину самой строки.

\lstinputlisting[caption=пример для MSVC,style=customc]{\CURPATH/STL/string/MSVC_RU.cpp}

Собственно, из этого исходника почти всё ясно.

Несколько замечаний:

Если строка короче 16-и символов, 
то отдельный буфер для строки в \glslink{heap}{куче} выделяться не будет.

Это удобно потому что на практике, основная часть строк действительно короткие.
Вероятно, разработчики в Microsoft выбрали размер в 16 символов как разумный баланс.

Теперь очень важный момент в конце функции main(): мы не пользуемся методом c\_str(), тем не менее,
если это скомпилировать и запустить, то обе строки появятся в консоли!

Работает это вот почему.

В первом случае строка короче 16-и символов и в начале объекта std::string (его можно рассматривать
просто как структуру) расположен буфер с этой строкой.
\printf трактует указатель как указатель на массив символов оканчивающийся нулем и поэтому всё работает.

Вывод второй строки (длиннее 16-и символов) даже еще опаснее: это вообще типичная программистская ошибка 
(или опечатка), забыть дописать c\_str().
Это работает потому что в это время в начале структуры расположен указатель на буфер.
Это может надолго остаться незамеченным: до тех пока там не появится строка 
короче 16-и символов, тогда процесс упадет.

\mysubparagraph{GCC}

В реализации GCC в структуре есть еще одна переменная --- reference count.

Интересно, что указатель на экземпляр класса std::string в GCC указывает не на начало самой структуры, 
а на указатель на буфера.
В libstdc++-v3\textbackslash{}include\textbackslash{}bits\textbackslash{}basic\_string.h 
мы можем прочитать что это сделано для удобства отладки:

\begin{lstlisting}
   *  The reason you want _M_data pointing to the character %array and
   *  not the _Rep is so that the debugger can see the string
   *  contents. (Probably we should add a non-inline member to get
   *  the _Rep for the debugger to use, so users can check the actual
   *  string length.)
\end{lstlisting}

\href{http://go.yurichev.com/17085}{исходный код basic\_string.h}

В нашем примере мы учитываем это:

\lstinputlisting[caption=пример для GCC,style=customc]{\CURPATH/STL/string/GCC_RU.cpp}

Нужны еще небольшие хаки чтобы сымитировать типичную ошибку, которую мы уже видели выше, из-за
более ужесточенной проверки типов в GCC, тем не менее, printf() работает и здесь без c\_str().

\myparagraph{Чуть более сложный пример}

\lstinputlisting[style=customc]{\CURPATH/STL/string/3.cpp}

\lstinputlisting[caption=MSVC 2012,style=customasmx86]{\CURPATH/STL/string/3_MSVC_RU.asm}

Собственно, компилятор не конструирует строки статически: да в общем-то и как
это возможно, если буфер с ней нужно хранить в \glslink{heap}{куче}?

Вместо этого в сегменте данных хранятся обычные \ac{ASCIIZ}-строки, а позже, во время выполнения, 
при помощи метода \q{assign}, конструируются строки s1 и s2
.
При помощи \TT{operator+}, создается строка s3.

Обратите внимание на то что вызов метода c\_str() отсутствует,
потому что его код достаточно короткий и компилятор вставил его прямо здесь:
если строка короче 16-и байт, то в регистре EAX остается указатель на буфер,
а если длиннее, то из этого же места достается адрес на буфер расположенный в \glslink{heap}{куче}.

Далее следуют вызовы трех деструкторов, причем, они вызываются только если строка длиннее 16-и байт:
тогда нужно освободить буфера в \glslink{heap}{куче}.
В противном случае, так как все три объекта std::string хранятся в стеке,
они освобождаются автоматически после выхода из функции.

Следовательно, работа с короткими строками более быстрая из-за м\'{е}ньшего обращения к \glslink{heap}{куче}.

Код на GCC даже проще (из-за того, что в GCC, как мы уже видели, не реализована возможность хранить короткую
строку прямо в структуре):

% TODO1 comment each function meaning
\lstinputlisting[caption=GCC 4.8.1,style=customasmx86]{\CURPATH/STL/string/3_GCC_RU.s}

Можно заметить, что в деструкторы передается не указатель на объект,
а указатель на место за 12 байт (или 3 слова) перед ним, то есть, на настоящее начало структуры.

\myparagraph{std::string как глобальная переменная}
\label{sec:std_string_as_global_variable}

Опытные программисты на \Cpp знают, что глобальные переменные \ac{STL}-типов вполне можно объявлять.

Да, действительно:

\lstinputlisting[style=customc]{\CURPATH/STL/string/5.cpp}

Но как и где будет вызываться конструктор \TT{std::string}?

На самом деле, эта переменная будет инициализирована даже перед началом \main.

\lstinputlisting[caption=MSVC 2012: здесь конструируется глобальная переменная{,} а также регистрируется её деструктор,style=customasmx86]{\CURPATH/STL/string/5_MSVC_p2.asm}

\lstinputlisting[caption=MSVC 2012: здесь глобальная переменная используется в \main,style=customasmx86]{\CURPATH/STL/string/5_MSVC_p1.asm}

\lstinputlisting[caption=MSVC 2012: эта функция-деструктор вызывается перед выходом,style=customasmx86]{\CURPATH/STL/string/5_MSVC_p3.asm}

\myindex{\CStandardLibrary!atexit()}
В реальности, из \ac{CRT}, еще до вызова main(), вызывается специальная функция,
в которой перечислены все конструкторы подобных переменных.
Более того: при помощи atexit() регистрируется функция, которая будет вызвана в конце работы программы:
в этой функции компилятор собирает вызовы деструкторов всех подобных глобальных переменных.

GCC работает похожим образом:

\lstinputlisting[caption=GCC 4.8.1,style=customasmx86]{\CURPATH/STL/string/5_GCC.s}

Но он не выделяет отдельной функции в которой будут собраны деструкторы: 
каждый деструктор передается в atexit() по одному.

% TODO а если глобальная STL-переменная в другом модуле? надо проверить.

}
\ifdefined\SPANISH
\chapter{Patrones de código}
\fi % SPANISH

\ifdefined\GERMAN
\chapter{Code-Muster}
\fi % GERMAN

\ifdefined\ENGLISH
\chapter{Code Patterns}
\fi % ENGLISH

\ifdefined\ITALIAN
\chapter{Forme di codice}
\fi % ITALIAN

\ifdefined\RUSSIAN
\chapter{Образцы кода}
\fi % RUSSIAN

\ifdefined\BRAZILIAN
\chapter{Padrões de códigos}
\fi % BRAZILIAN

\ifdefined\THAI
\chapter{รูปแบบของโค้ด}
\fi % THAI

\ifdefined\FRENCH
\chapter{Modèle de code}
\fi % FRENCH

\ifdefined\POLISH
\chapter{\PLph{}}
\fi % POLISH

% sections
\EN{\input{patterns/patterns_opt_dbg_EN}}
\ES{\input{patterns/patterns_opt_dbg_ES}}
\ITA{\input{patterns/patterns_opt_dbg_ITA}}
\PTBR{\input{patterns/patterns_opt_dbg_PTBR}}
\RU{\input{patterns/patterns_opt_dbg_RU}}
\THA{\input{patterns/patterns_opt_dbg_THA}}
\DE{\input{patterns/patterns_opt_dbg_DE}}
\FR{\input{patterns/patterns_opt_dbg_FR}}
\PL{\input{patterns/patterns_opt_dbg_PL}}

\RU{\section{Некоторые базовые понятия}}
\EN{\section{Some basics}}
\DE{\section{Einige Grundlagen}}
\FR{\section{Quelques bases}}
\ES{\section{\ESph{}}}
\ITA{\section{Alcune basi teoriche}}
\PTBR{\section{\PTBRph{}}}
\THA{\section{\THAph{}}}
\PL{\section{\PLph{}}}

% sections:
\EN{\input{patterns/intro_CPU_ISA_EN}}
\ES{\input{patterns/intro_CPU_ISA_ES}}
\ITA{\input{patterns/intro_CPU_ISA_ITA}}
\PTBR{\input{patterns/intro_CPU_ISA_PTBR}}
\RU{\input{patterns/intro_CPU_ISA_RU}}
\DE{\input{patterns/intro_CPU_ISA_DE}}
\FR{\input{patterns/intro_CPU_ISA_FR}}
\PL{\input{patterns/intro_CPU_ISA_PL}}

\EN{\input{patterns/numeral_EN}}
\RU{\input{patterns/numeral_RU}}
\ITA{\input{patterns/numeral_ITA}}
\DE{\input{patterns/numeral_DE}}
\FR{\input{patterns/numeral_FR}}
\PL{\input{patterns/numeral_PL}}

% chapters
\input{patterns/00_empty/main}
\input{patterns/011_ret/main}
\input{patterns/01_helloworld/main}
\input{patterns/015_prolog_epilogue/main}
\input{patterns/02_stack/main}
\input{patterns/03_printf/main}
\input{patterns/04_scanf/main}
\input{patterns/05_passing_arguments/main}
\input{patterns/06_return_results/main}
\input{patterns/061_pointers/main}
\input{patterns/065_GOTO/main}
\input{patterns/07_jcc/main}
\input{patterns/08_switch/main}
\input{patterns/09_loops/main}
\input{patterns/10_strings/main}
\input{patterns/11_arith_optimizations/main}
\input{patterns/12_FPU/main}
\input{patterns/13_arrays/main}
\input{patterns/14_bitfields/main}
\EN{\input{patterns/145_LCG/main_EN}}
\RU{\input{patterns/145_LCG/main_RU}}
\input{patterns/15_structs/main}
\input{patterns/17_unions/main}
\input{patterns/18_pointers_to_functions/main}
\input{patterns/185_64bit_in_32_env/main}

\EN{\input{patterns/19_SIMD/main_EN}}
\RU{\input{patterns/19_SIMD/main_RU}}
\DE{\input{patterns/19_SIMD/main_DE}}

\EN{\input{patterns/20_x64/main_EN}}
\RU{\input{patterns/20_x64/main_RU}}

\EN{\input{patterns/205_floating_SIMD/main_EN}}
\RU{\input{patterns/205_floating_SIMD/main_RU}}
\DE{\input{patterns/205_floating_SIMD/main_DE}}

\EN{\input{patterns/ARM/main_EN}}
\RU{\input{patterns/ARM/main_RU}}
\DE{\input{patterns/ARM/main_DE}}

\input{patterns/MIPS/main}

\ifdefined\SPANISH
\chapter{Patrones de código}
\fi % SPANISH

\ifdefined\GERMAN
\chapter{Code-Muster}
\fi % GERMAN

\ifdefined\ENGLISH
\chapter{Code Patterns}
\fi % ENGLISH

\ifdefined\ITALIAN
\chapter{Forme di codice}
\fi % ITALIAN

\ifdefined\RUSSIAN
\chapter{Образцы кода}
\fi % RUSSIAN

\ifdefined\BRAZILIAN
\chapter{Padrões de códigos}
\fi % BRAZILIAN

\ifdefined\THAI
\chapter{รูปแบบของโค้ด}
\fi % THAI

\ifdefined\FRENCH
\chapter{Modèle de code}
\fi % FRENCH

\ifdefined\POLISH
\chapter{\PLph{}}
\fi % POLISH

% sections
\EN{\input{patterns/patterns_opt_dbg_EN}}
\ES{\input{patterns/patterns_opt_dbg_ES}}
\ITA{\input{patterns/patterns_opt_dbg_ITA}}
\PTBR{\input{patterns/patterns_opt_dbg_PTBR}}
\RU{\input{patterns/patterns_opt_dbg_RU}}
\THA{\input{patterns/patterns_opt_dbg_THA}}
\DE{\input{patterns/patterns_opt_dbg_DE}}
\FR{\input{patterns/patterns_opt_dbg_FR}}
\PL{\input{patterns/patterns_opt_dbg_PL}}

\RU{\section{Некоторые базовые понятия}}
\EN{\section{Some basics}}
\DE{\section{Einige Grundlagen}}
\FR{\section{Quelques bases}}
\ES{\section{\ESph{}}}
\ITA{\section{Alcune basi teoriche}}
\PTBR{\section{\PTBRph{}}}
\THA{\section{\THAph{}}}
\PL{\section{\PLph{}}}

% sections:
\EN{\input{patterns/intro_CPU_ISA_EN}}
\ES{\input{patterns/intro_CPU_ISA_ES}}
\ITA{\input{patterns/intro_CPU_ISA_ITA}}
\PTBR{\input{patterns/intro_CPU_ISA_PTBR}}
\RU{\input{patterns/intro_CPU_ISA_RU}}
\DE{\input{patterns/intro_CPU_ISA_DE}}
\FR{\input{patterns/intro_CPU_ISA_FR}}
\PL{\input{patterns/intro_CPU_ISA_PL}}

\EN{\input{patterns/numeral_EN}}
\RU{\input{patterns/numeral_RU}}
\ITA{\input{patterns/numeral_ITA}}
\DE{\input{patterns/numeral_DE}}
\FR{\input{patterns/numeral_FR}}
\PL{\input{patterns/numeral_PL}}

% chapters
\input{patterns/00_empty/main}
\input{patterns/011_ret/main}
\input{patterns/01_helloworld/main}
\input{patterns/015_prolog_epilogue/main}
\input{patterns/02_stack/main}
\input{patterns/03_printf/main}
\input{patterns/04_scanf/main}
\input{patterns/05_passing_arguments/main}
\input{patterns/06_return_results/main}
\input{patterns/061_pointers/main}
\input{patterns/065_GOTO/main}
\input{patterns/07_jcc/main}
\input{patterns/08_switch/main}
\input{patterns/09_loops/main}
\input{patterns/10_strings/main}
\input{patterns/11_arith_optimizations/main}
\input{patterns/12_FPU/main}
\input{patterns/13_arrays/main}
\input{patterns/14_bitfields/main}
\EN{\input{patterns/145_LCG/main_EN}}
\RU{\input{patterns/145_LCG/main_RU}}
\input{patterns/15_structs/main}
\input{patterns/17_unions/main}
\input{patterns/18_pointers_to_functions/main}
\input{patterns/185_64bit_in_32_env/main}

\EN{\input{patterns/19_SIMD/main_EN}}
\RU{\input{patterns/19_SIMD/main_RU}}
\DE{\input{patterns/19_SIMD/main_DE}}

\EN{\input{patterns/20_x64/main_EN}}
\RU{\input{patterns/20_x64/main_RU}}

\EN{\input{patterns/205_floating_SIMD/main_EN}}
\RU{\input{patterns/205_floating_SIMD/main_RU}}
\DE{\input{patterns/205_floating_SIMD/main_DE}}

\EN{\input{patterns/ARM/main_EN}}
\RU{\input{patterns/ARM/main_RU}}
\DE{\input{patterns/ARM/main_DE}}

\input{patterns/MIPS/main}

\ifdefined\SPANISH
\chapter{Patrones de código}
\fi % SPANISH

\ifdefined\GERMAN
\chapter{Code-Muster}
\fi % GERMAN

\ifdefined\ENGLISH
\chapter{Code Patterns}
\fi % ENGLISH

\ifdefined\ITALIAN
\chapter{Forme di codice}
\fi % ITALIAN

\ifdefined\RUSSIAN
\chapter{Образцы кода}
\fi % RUSSIAN

\ifdefined\BRAZILIAN
\chapter{Padrões de códigos}
\fi % BRAZILIAN

\ifdefined\THAI
\chapter{รูปแบบของโค้ด}
\fi % THAI

\ifdefined\FRENCH
\chapter{Modèle de code}
\fi % FRENCH

\ifdefined\POLISH
\chapter{\PLph{}}
\fi % POLISH

% sections
\EN{\input{patterns/patterns_opt_dbg_EN}}
\ES{\input{patterns/patterns_opt_dbg_ES}}
\ITA{\input{patterns/patterns_opt_dbg_ITA}}
\PTBR{\input{patterns/patterns_opt_dbg_PTBR}}
\RU{\input{patterns/patterns_opt_dbg_RU}}
\THA{\input{patterns/patterns_opt_dbg_THA}}
\DE{\input{patterns/patterns_opt_dbg_DE}}
\FR{\input{patterns/patterns_opt_dbg_FR}}
\PL{\input{patterns/patterns_opt_dbg_PL}}

\RU{\section{Некоторые базовые понятия}}
\EN{\section{Some basics}}
\DE{\section{Einige Grundlagen}}
\FR{\section{Quelques bases}}
\ES{\section{\ESph{}}}
\ITA{\section{Alcune basi teoriche}}
\PTBR{\section{\PTBRph{}}}
\THA{\section{\THAph{}}}
\PL{\section{\PLph{}}}

% sections:
\EN{\input{patterns/intro_CPU_ISA_EN}}
\ES{\input{patterns/intro_CPU_ISA_ES}}
\ITA{\input{patterns/intro_CPU_ISA_ITA}}
\PTBR{\input{patterns/intro_CPU_ISA_PTBR}}
\RU{\input{patterns/intro_CPU_ISA_RU}}
\DE{\input{patterns/intro_CPU_ISA_DE}}
\FR{\input{patterns/intro_CPU_ISA_FR}}
\PL{\input{patterns/intro_CPU_ISA_PL}}

\EN{\input{patterns/numeral_EN}}
\RU{\input{patterns/numeral_RU}}
\ITA{\input{patterns/numeral_ITA}}
\DE{\input{patterns/numeral_DE}}
\FR{\input{patterns/numeral_FR}}
\PL{\input{patterns/numeral_PL}}

% chapters
\input{patterns/00_empty/main}
\input{patterns/011_ret/main}
\input{patterns/01_helloworld/main}
\input{patterns/015_prolog_epilogue/main}
\input{patterns/02_stack/main}
\input{patterns/03_printf/main}
\input{patterns/04_scanf/main}
\input{patterns/05_passing_arguments/main}
\input{patterns/06_return_results/main}
\input{patterns/061_pointers/main}
\input{patterns/065_GOTO/main}
\input{patterns/07_jcc/main}
\input{patterns/08_switch/main}
\input{patterns/09_loops/main}
\input{patterns/10_strings/main}
\input{patterns/11_arith_optimizations/main}
\input{patterns/12_FPU/main}
\input{patterns/13_arrays/main}
\input{patterns/14_bitfields/main}
\EN{\input{patterns/145_LCG/main_EN}}
\RU{\input{patterns/145_LCG/main_RU}}
\input{patterns/15_structs/main}
\input{patterns/17_unions/main}
\input{patterns/18_pointers_to_functions/main}
\input{patterns/185_64bit_in_32_env/main}

\EN{\input{patterns/19_SIMD/main_EN}}
\RU{\input{patterns/19_SIMD/main_RU}}
\DE{\input{patterns/19_SIMD/main_DE}}

\EN{\input{patterns/20_x64/main_EN}}
\RU{\input{patterns/20_x64/main_RU}}

\EN{\input{patterns/205_floating_SIMD/main_EN}}
\RU{\input{patterns/205_floating_SIMD/main_RU}}
\DE{\input{patterns/205_floating_SIMD/main_DE}}

\EN{\input{patterns/ARM/main_EN}}
\RU{\input{patterns/ARM/main_RU}}
\DE{\input{patterns/ARM/main_DE}}

\input{patterns/MIPS/main}

\ifdefined\SPANISH
\chapter{Patrones de código}
\fi % SPANISH

\ifdefined\GERMAN
\chapter{Code-Muster}
\fi % GERMAN

\ifdefined\ENGLISH
\chapter{Code Patterns}
\fi % ENGLISH

\ifdefined\ITALIAN
\chapter{Forme di codice}
\fi % ITALIAN

\ifdefined\RUSSIAN
\chapter{Образцы кода}
\fi % RUSSIAN

\ifdefined\BRAZILIAN
\chapter{Padrões de códigos}
\fi % BRAZILIAN

\ifdefined\THAI
\chapter{รูปแบบของโค้ด}
\fi % THAI

\ifdefined\FRENCH
\chapter{Modèle de code}
\fi % FRENCH

\ifdefined\POLISH
\chapter{\PLph{}}
\fi % POLISH

% sections
\EN{\input{patterns/patterns_opt_dbg_EN}}
\ES{\input{patterns/patterns_opt_dbg_ES}}
\ITA{\input{patterns/patterns_opt_dbg_ITA}}
\PTBR{\input{patterns/patterns_opt_dbg_PTBR}}
\RU{\input{patterns/patterns_opt_dbg_RU}}
\THA{\input{patterns/patterns_opt_dbg_THA}}
\DE{\input{patterns/patterns_opt_dbg_DE}}
\FR{\input{patterns/patterns_opt_dbg_FR}}
\PL{\input{patterns/patterns_opt_dbg_PL}}

\RU{\section{Некоторые базовые понятия}}
\EN{\section{Some basics}}
\DE{\section{Einige Grundlagen}}
\FR{\section{Quelques bases}}
\ES{\section{\ESph{}}}
\ITA{\section{Alcune basi teoriche}}
\PTBR{\section{\PTBRph{}}}
\THA{\section{\THAph{}}}
\PL{\section{\PLph{}}}

% sections:
\EN{\input{patterns/intro_CPU_ISA_EN}}
\ES{\input{patterns/intro_CPU_ISA_ES}}
\ITA{\input{patterns/intro_CPU_ISA_ITA}}
\PTBR{\input{patterns/intro_CPU_ISA_PTBR}}
\RU{\input{patterns/intro_CPU_ISA_RU}}
\DE{\input{patterns/intro_CPU_ISA_DE}}
\FR{\input{patterns/intro_CPU_ISA_FR}}
\PL{\input{patterns/intro_CPU_ISA_PL}}

\EN{\input{patterns/numeral_EN}}
\RU{\input{patterns/numeral_RU}}
\ITA{\input{patterns/numeral_ITA}}
\DE{\input{patterns/numeral_DE}}
\FR{\input{patterns/numeral_FR}}
\PL{\input{patterns/numeral_PL}}

% chapters
\input{patterns/00_empty/main}
\input{patterns/011_ret/main}
\input{patterns/01_helloworld/main}
\input{patterns/015_prolog_epilogue/main}
\input{patterns/02_stack/main}
\input{patterns/03_printf/main}
\input{patterns/04_scanf/main}
\input{patterns/05_passing_arguments/main}
\input{patterns/06_return_results/main}
\input{patterns/061_pointers/main}
\input{patterns/065_GOTO/main}
\input{patterns/07_jcc/main}
\input{patterns/08_switch/main}
\input{patterns/09_loops/main}
\input{patterns/10_strings/main}
\input{patterns/11_arith_optimizations/main}
\input{patterns/12_FPU/main}
\input{patterns/13_arrays/main}
\input{patterns/14_bitfields/main}
\EN{\input{patterns/145_LCG/main_EN}}
\RU{\input{patterns/145_LCG/main_RU}}
\input{patterns/15_structs/main}
\input{patterns/17_unions/main}
\input{patterns/18_pointers_to_functions/main}
\input{patterns/185_64bit_in_32_env/main}

\EN{\input{patterns/19_SIMD/main_EN}}
\RU{\input{patterns/19_SIMD/main_RU}}
\DE{\input{patterns/19_SIMD/main_DE}}

\EN{\input{patterns/20_x64/main_EN}}
\RU{\input{patterns/20_x64/main_RU}}

\EN{\input{patterns/205_floating_SIMD/main_EN}}
\RU{\input{patterns/205_floating_SIMD/main_RU}}
\DE{\input{patterns/205_floating_SIMD/main_DE}}

\EN{\input{patterns/ARM/main_EN}}
\RU{\input{patterns/ARM/main_RU}}
\DE{\input{patterns/ARM/main_DE}}

\input{patterns/MIPS/main}


\EN{\section{Returning Values}
\label{ret_val_func}

Another simple function is the one that simply returns a constant value:

\lstinputlisting[caption=\EN{\CCpp Code},style=customc]{patterns/011_ret/1.c}

Let's compile it.

\subsection{x86}

Here's what both the GCC and MSVC compilers produce (with optimization) on the x86 platform:

\lstinputlisting[caption=\Optimizing GCC/MSVC (\assemblyOutput),style=customasmx86]{patterns/011_ret/1.s}

\myindex{x86!\Instructions!RET}
There are just two instructions: the first places the value 123 into the \EAX register,
which is used by convention for storing the return
value, and the second one is \RET, which returns execution to the \gls{caller}.

The caller will take the result from the \EAX register.

\subsection{ARM}

There are a few differences on the ARM platform:

\lstinputlisting[caption=\OptimizingKeilVI (\ARMMode) ASM Output,style=customasmARM]{patterns/011_ret/1_Keil_ARM_O3.s}

ARM uses the register \Reg{0} for returning the results of functions, so 123 is copied into \Reg{0}.

\myindex{ARM!\Instructions!MOV}
\myindex{x86!\Instructions!MOV}
It is worth noting that \MOV is a misleading name for the instruction in both the x86 and ARM \ac{ISA}s.

The data is not in fact \IT{moved}, but \IT{copied}.

\subsection{MIPS}

\label{MIPS_leaf_function_ex1}

The GCC assembly output below lists registers by number:

\lstinputlisting[caption=\Optimizing GCC 4.4.5 (\assemblyOutput),style=customasmMIPS]{patterns/011_ret/MIPS.s}

\dots while \IDA does it by their pseudo names:

\lstinputlisting[caption=\Optimizing GCC 4.4.5 (IDA),style=customasmMIPS]{patterns/011_ret/MIPS_IDA.lst}

The \$2 (or \$V0) register is used to store the function's return value.
\myindex{MIPS!\Pseudoinstructions!LI}
\INS{LI} stands for ``Load Immediate'' and is the MIPS equivalent to \MOV.

\myindex{MIPS!\Instructions!J}
The other instruction is the jump instruction (J or JR) which returns the execution flow to the \gls{caller}.

\myindex{MIPS!Branch delay slot}
You might be wondering why the positions of the load instruction (LI) and the jump instruction (J or JR) are swapped. This is due to a \ac{RISC} feature called ``branch delay slot''.

The reason this happens is a quirk in the architecture of some RISC \ac{ISA}s and isn't important for our
purposes---we must simply keep in mind that in MIPS, the instruction following a jump or branch instruction
is executed \IT{before} the jump/branch instruction itself.

As a consequence, branch instructions always swap places with the instruction executed immediately beforehand.


In practice, functions which merely return 1 (\IT{true}) or 0 (\IT{false}) are very frequent.

The smallest ever of the standard UNIX utilities, \IT{/bin/true} and \IT{/bin/false} return 0 and 1 respectively, as an exit code.
(Zero as an exit code usually means success, non-zero means error.)
}
\RU{\subsubsection{std::string}
\myindex{\Cpp!STL!std::string}
\label{std_string}

\myparagraph{Как устроена структура}

Многие строковые библиотеки \InSqBrackets{\CNotes 2.2} обеспечивают структуру содержащую ссылку 
на буфер собственно со строкой, переменная всегда содержащую длину строки 
(что очень удобно для массы функций \InSqBrackets{\CNotes 2.2.1}) и переменную содержащую текущий размер буфера.

Строка в буфере обыкновенно оканчивается нулем: это для того чтобы указатель на буфер можно было
передавать в функции требующие на вход обычную сишную \ac{ASCIIZ}-строку.

Стандарт \Cpp не описывает, как именно нужно реализовывать std::string,
но, как правило, они реализованы как описано выше, с небольшими дополнениями.

Строки в \Cpp это не класс (как, например, QString в Qt), а темплейт (basic\_string), 
это сделано для того чтобы поддерживать 
строки содержащие разного типа символы: как минимум \Tchar и \IT{wchar\_t}.

Так что, std::string это класс с базовым типом \Tchar.

А std::wstring это класс с базовым типом \IT{wchar\_t}.

\mysubparagraph{MSVC}

В реализации MSVC, вместо ссылки на буфер может содержаться сам буфер (если строка короче 16-и символов).

Это означает, что каждая короткая строка будет занимать в памяти по крайней мере $16 + 4 + 4 = 24$ 
байт для 32-битной среды либо $16 + 8 + 8 = 32$ 
байта в 64-битной, а если строка длиннее 16-и символов, то прибавьте еще длину самой строки.

\lstinputlisting[caption=пример для MSVC,style=customc]{\CURPATH/STL/string/MSVC_RU.cpp}

Собственно, из этого исходника почти всё ясно.

Несколько замечаний:

Если строка короче 16-и символов, 
то отдельный буфер для строки в \glslink{heap}{куче} выделяться не будет.

Это удобно потому что на практике, основная часть строк действительно короткие.
Вероятно, разработчики в Microsoft выбрали размер в 16 символов как разумный баланс.

Теперь очень важный момент в конце функции main(): мы не пользуемся методом c\_str(), тем не менее,
если это скомпилировать и запустить, то обе строки появятся в консоли!

Работает это вот почему.

В первом случае строка короче 16-и символов и в начале объекта std::string (его можно рассматривать
просто как структуру) расположен буфер с этой строкой.
\printf трактует указатель как указатель на массив символов оканчивающийся нулем и поэтому всё работает.

Вывод второй строки (длиннее 16-и символов) даже еще опаснее: это вообще типичная программистская ошибка 
(или опечатка), забыть дописать c\_str().
Это работает потому что в это время в начале структуры расположен указатель на буфер.
Это может надолго остаться незамеченным: до тех пока там не появится строка 
короче 16-и символов, тогда процесс упадет.

\mysubparagraph{GCC}

В реализации GCC в структуре есть еще одна переменная --- reference count.

Интересно, что указатель на экземпляр класса std::string в GCC указывает не на начало самой структуры, 
а на указатель на буфера.
В libstdc++-v3\textbackslash{}include\textbackslash{}bits\textbackslash{}basic\_string.h 
мы можем прочитать что это сделано для удобства отладки:

\begin{lstlisting}
   *  The reason you want _M_data pointing to the character %array and
   *  not the _Rep is so that the debugger can see the string
   *  contents. (Probably we should add a non-inline member to get
   *  the _Rep for the debugger to use, so users can check the actual
   *  string length.)
\end{lstlisting}

\href{http://go.yurichev.com/17085}{исходный код basic\_string.h}

В нашем примере мы учитываем это:

\lstinputlisting[caption=пример для GCC,style=customc]{\CURPATH/STL/string/GCC_RU.cpp}

Нужны еще небольшие хаки чтобы сымитировать типичную ошибку, которую мы уже видели выше, из-за
более ужесточенной проверки типов в GCC, тем не менее, printf() работает и здесь без c\_str().

\myparagraph{Чуть более сложный пример}

\lstinputlisting[style=customc]{\CURPATH/STL/string/3.cpp}

\lstinputlisting[caption=MSVC 2012,style=customasmx86]{\CURPATH/STL/string/3_MSVC_RU.asm}

Собственно, компилятор не конструирует строки статически: да в общем-то и как
это возможно, если буфер с ней нужно хранить в \glslink{heap}{куче}?

Вместо этого в сегменте данных хранятся обычные \ac{ASCIIZ}-строки, а позже, во время выполнения, 
при помощи метода \q{assign}, конструируются строки s1 и s2
.
При помощи \TT{operator+}, создается строка s3.

Обратите внимание на то что вызов метода c\_str() отсутствует,
потому что его код достаточно короткий и компилятор вставил его прямо здесь:
если строка короче 16-и байт, то в регистре EAX остается указатель на буфер,
а если длиннее, то из этого же места достается адрес на буфер расположенный в \glslink{heap}{куче}.

Далее следуют вызовы трех деструкторов, причем, они вызываются только если строка длиннее 16-и байт:
тогда нужно освободить буфера в \glslink{heap}{куче}.
В противном случае, так как все три объекта std::string хранятся в стеке,
они освобождаются автоматически после выхода из функции.

Следовательно, работа с короткими строками более быстрая из-за м\'{е}ньшего обращения к \glslink{heap}{куче}.

Код на GCC даже проще (из-за того, что в GCC, как мы уже видели, не реализована возможность хранить короткую
строку прямо в структуре):

% TODO1 comment each function meaning
\lstinputlisting[caption=GCC 4.8.1,style=customasmx86]{\CURPATH/STL/string/3_GCC_RU.s}

Можно заметить, что в деструкторы передается не указатель на объект,
а указатель на место за 12 байт (или 3 слова) перед ним, то есть, на настоящее начало структуры.

\myparagraph{std::string как глобальная переменная}
\label{sec:std_string_as_global_variable}

Опытные программисты на \Cpp знают, что глобальные переменные \ac{STL}-типов вполне можно объявлять.

Да, действительно:

\lstinputlisting[style=customc]{\CURPATH/STL/string/5.cpp}

Но как и где будет вызываться конструктор \TT{std::string}?

На самом деле, эта переменная будет инициализирована даже перед началом \main.

\lstinputlisting[caption=MSVC 2012: здесь конструируется глобальная переменная{,} а также регистрируется её деструктор,style=customasmx86]{\CURPATH/STL/string/5_MSVC_p2.asm}

\lstinputlisting[caption=MSVC 2012: здесь глобальная переменная используется в \main,style=customasmx86]{\CURPATH/STL/string/5_MSVC_p1.asm}

\lstinputlisting[caption=MSVC 2012: эта функция-деструктор вызывается перед выходом,style=customasmx86]{\CURPATH/STL/string/5_MSVC_p3.asm}

\myindex{\CStandardLibrary!atexit()}
В реальности, из \ac{CRT}, еще до вызова main(), вызывается специальная функция,
в которой перечислены все конструкторы подобных переменных.
Более того: при помощи atexit() регистрируется функция, которая будет вызвана в конце работы программы:
в этой функции компилятор собирает вызовы деструкторов всех подобных глобальных переменных.

GCC работает похожим образом:

\lstinputlisting[caption=GCC 4.8.1,style=customasmx86]{\CURPATH/STL/string/5_GCC.s}

Но он не выделяет отдельной функции в которой будут собраны деструкторы: 
каждый деструктор передается в atexit() по одному.

% TODO а если глобальная STL-переменная в другом модуле? надо проверить.

}
\DE{\subsection{Einfachste XOR-Verschlüsselung überhaupt}

Ich habe einmal eine Software gesehen, bei der alle Debugging-Ausgaben mit XOR mit dem Wert 3
verschlüsselt wurden. Mit anderen Worten, die beiden niedrigsten Bits aller Buchstaben wurden invertiert.

``Hello, world'' wurde zu ``Kfool/\#tlqog'':

\begin{lstlisting}
#!/usr/bin/python

msg="Hello, world!"

print "".join(map(lambda x: chr(ord(x)^3), msg))
\end{lstlisting}

Das ist eine ziemlich interessante Verschlüsselung (oder besser eine Verschleierung),
weil sie zwei wichtige Eigenschaften hat:
1) es ist eine einzige Funktion zum Verschlüsseln und entschlüsseln, sie muss nur wiederholt angewendet werden
2) die entstehenden Buchstaben befinden sich im druckbaren Bereich, also die ganze Zeichenkette kann ohne
Escape-Symbole im Code verwendet werden.

Die zweite Eigenschaft nutzt die Tatsache, dass alle druckbaren Zeichen in Reihen organisiert sind: 0x2x-0x7x,
und wenn die beiden niederwertigsten Bits invertiert werden, wird der Buchstabe um eine oder drei Stellen nach
links oder rechts \IT{verschoben}, aber niemals in eine andere Reihe:

\begin{figure}[H]
\centering
\includegraphics[width=0.7\textwidth]{ascii_clean.png}
\caption{7-Bit \ac{ASCII} Tabelle in Emacs}
\end{figure}

\dots mit dem Zeichen 0x7F als einziger Ausnahme.

Im Folgenden werden also beispielsweise die Zeichen A-Z \IT{verschlüsselt}:

\begin{lstlisting}
#!/usr/bin/python

msg="@ABCDEFGHIJKLMNO"

print "".join(map(lambda x: chr(ord(x)^3), msg))
\end{lstlisting}

Ergebnis:
% FIXME \verb  --  relevant comment for German?
\begin{lstlisting}
CBA@GFEDKJIHONML
\end{lstlisting}

Es sieht so aus als würden die Zeichen ``@'' und ``C'' sowie ``B'' und ``A'' vertauscht werden.

Hier ist noch ein interessantes Beispiel, in dem gezeigt wird, wie die Eigenschaften von XOR
ausgenutzt werden können: Exakt den gleichen Effekt, dass druckbare Zeichen auch druckbar bleiben,
kann man dadurch erzielen, dass irgendeine Kombination der niedrigsten vier Bits invertiert wird.
}

\EN{\section{Returning Values}
\label{ret_val_func}

Another simple function is the one that simply returns a constant value:

\lstinputlisting[caption=\EN{\CCpp Code},style=customc]{patterns/011_ret/1.c}

Let's compile it.

\subsection{x86}

Here's what both the GCC and MSVC compilers produce (with optimization) on the x86 platform:

\lstinputlisting[caption=\Optimizing GCC/MSVC (\assemblyOutput),style=customasmx86]{patterns/011_ret/1.s}

\myindex{x86!\Instructions!RET}
There are just two instructions: the first places the value 123 into the \EAX register,
which is used by convention for storing the return
value, and the second one is \RET, which returns execution to the \gls{caller}.

The caller will take the result from the \EAX register.

\subsection{ARM}

There are a few differences on the ARM platform:

\lstinputlisting[caption=\OptimizingKeilVI (\ARMMode) ASM Output,style=customasmARM]{patterns/011_ret/1_Keil_ARM_O3.s}

ARM uses the register \Reg{0} for returning the results of functions, so 123 is copied into \Reg{0}.

\myindex{ARM!\Instructions!MOV}
\myindex{x86!\Instructions!MOV}
It is worth noting that \MOV is a misleading name for the instruction in both the x86 and ARM \ac{ISA}s.

The data is not in fact \IT{moved}, but \IT{copied}.

\subsection{MIPS}

\label{MIPS_leaf_function_ex1}

The GCC assembly output below lists registers by number:

\lstinputlisting[caption=\Optimizing GCC 4.4.5 (\assemblyOutput),style=customasmMIPS]{patterns/011_ret/MIPS.s}

\dots while \IDA does it by their pseudo names:

\lstinputlisting[caption=\Optimizing GCC 4.4.5 (IDA),style=customasmMIPS]{patterns/011_ret/MIPS_IDA.lst}

The \$2 (or \$V0) register is used to store the function's return value.
\myindex{MIPS!\Pseudoinstructions!LI}
\INS{LI} stands for ``Load Immediate'' and is the MIPS equivalent to \MOV.

\myindex{MIPS!\Instructions!J}
The other instruction is the jump instruction (J or JR) which returns the execution flow to the \gls{caller}.

\myindex{MIPS!Branch delay slot}
You might be wondering why the positions of the load instruction (LI) and the jump instruction (J or JR) are swapped. This is due to a \ac{RISC} feature called ``branch delay slot''.

The reason this happens is a quirk in the architecture of some RISC \ac{ISA}s and isn't important for our
purposes---we must simply keep in mind that in MIPS, the instruction following a jump or branch instruction
is executed \IT{before} the jump/branch instruction itself.

As a consequence, branch instructions always swap places with the instruction executed immediately beforehand.


In practice, functions which merely return 1 (\IT{true}) or 0 (\IT{false}) are very frequent.

The smallest ever of the standard UNIX utilities, \IT{/bin/true} and \IT{/bin/false} return 0 and 1 respectively, as an exit code.
(Zero as an exit code usually means success, non-zero means error.)
}
\RU{\subsubsection{std::string}
\myindex{\Cpp!STL!std::string}
\label{std_string}

\myparagraph{Как устроена структура}

Многие строковые библиотеки \InSqBrackets{\CNotes 2.2} обеспечивают структуру содержащую ссылку 
на буфер собственно со строкой, переменная всегда содержащую длину строки 
(что очень удобно для массы функций \InSqBrackets{\CNotes 2.2.1}) и переменную содержащую текущий размер буфера.

Строка в буфере обыкновенно оканчивается нулем: это для того чтобы указатель на буфер можно было
передавать в функции требующие на вход обычную сишную \ac{ASCIIZ}-строку.

Стандарт \Cpp не описывает, как именно нужно реализовывать std::string,
но, как правило, они реализованы как описано выше, с небольшими дополнениями.

Строки в \Cpp это не класс (как, например, QString в Qt), а темплейт (basic\_string), 
это сделано для того чтобы поддерживать 
строки содержащие разного типа символы: как минимум \Tchar и \IT{wchar\_t}.

Так что, std::string это класс с базовым типом \Tchar.

А std::wstring это класс с базовым типом \IT{wchar\_t}.

\mysubparagraph{MSVC}

В реализации MSVC, вместо ссылки на буфер может содержаться сам буфер (если строка короче 16-и символов).

Это означает, что каждая короткая строка будет занимать в памяти по крайней мере $16 + 4 + 4 = 24$ 
байт для 32-битной среды либо $16 + 8 + 8 = 32$ 
байта в 64-битной, а если строка длиннее 16-и символов, то прибавьте еще длину самой строки.

\lstinputlisting[caption=пример для MSVC,style=customc]{\CURPATH/STL/string/MSVC_RU.cpp}

Собственно, из этого исходника почти всё ясно.

Несколько замечаний:

Если строка короче 16-и символов, 
то отдельный буфер для строки в \glslink{heap}{куче} выделяться не будет.

Это удобно потому что на практике, основная часть строк действительно короткие.
Вероятно, разработчики в Microsoft выбрали размер в 16 символов как разумный баланс.

Теперь очень важный момент в конце функции main(): мы не пользуемся методом c\_str(), тем не менее,
если это скомпилировать и запустить, то обе строки появятся в консоли!

Работает это вот почему.

В первом случае строка короче 16-и символов и в начале объекта std::string (его можно рассматривать
просто как структуру) расположен буфер с этой строкой.
\printf трактует указатель как указатель на массив символов оканчивающийся нулем и поэтому всё работает.

Вывод второй строки (длиннее 16-и символов) даже еще опаснее: это вообще типичная программистская ошибка 
(или опечатка), забыть дописать c\_str().
Это работает потому что в это время в начале структуры расположен указатель на буфер.
Это может надолго остаться незамеченным: до тех пока там не появится строка 
короче 16-и символов, тогда процесс упадет.

\mysubparagraph{GCC}

В реализации GCC в структуре есть еще одна переменная --- reference count.

Интересно, что указатель на экземпляр класса std::string в GCC указывает не на начало самой структуры, 
а на указатель на буфера.
В libstdc++-v3\textbackslash{}include\textbackslash{}bits\textbackslash{}basic\_string.h 
мы можем прочитать что это сделано для удобства отладки:

\begin{lstlisting}
   *  The reason you want _M_data pointing to the character %array and
   *  not the _Rep is so that the debugger can see the string
   *  contents. (Probably we should add a non-inline member to get
   *  the _Rep for the debugger to use, so users can check the actual
   *  string length.)
\end{lstlisting}

\href{http://go.yurichev.com/17085}{исходный код basic\_string.h}

В нашем примере мы учитываем это:

\lstinputlisting[caption=пример для GCC,style=customc]{\CURPATH/STL/string/GCC_RU.cpp}

Нужны еще небольшие хаки чтобы сымитировать типичную ошибку, которую мы уже видели выше, из-за
более ужесточенной проверки типов в GCC, тем не менее, printf() работает и здесь без c\_str().

\myparagraph{Чуть более сложный пример}

\lstinputlisting[style=customc]{\CURPATH/STL/string/3.cpp}

\lstinputlisting[caption=MSVC 2012,style=customasmx86]{\CURPATH/STL/string/3_MSVC_RU.asm}

Собственно, компилятор не конструирует строки статически: да в общем-то и как
это возможно, если буфер с ней нужно хранить в \glslink{heap}{куче}?

Вместо этого в сегменте данных хранятся обычные \ac{ASCIIZ}-строки, а позже, во время выполнения, 
при помощи метода \q{assign}, конструируются строки s1 и s2
.
При помощи \TT{operator+}, создается строка s3.

Обратите внимание на то что вызов метода c\_str() отсутствует,
потому что его код достаточно короткий и компилятор вставил его прямо здесь:
если строка короче 16-и байт, то в регистре EAX остается указатель на буфер,
а если длиннее, то из этого же места достается адрес на буфер расположенный в \glslink{heap}{куче}.

Далее следуют вызовы трех деструкторов, причем, они вызываются только если строка длиннее 16-и байт:
тогда нужно освободить буфера в \glslink{heap}{куче}.
В противном случае, так как все три объекта std::string хранятся в стеке,
они освобождаются автоматически после выхода из функции.

Следовательно, работа с короткими строками более быстрая из-за м\'{е}ньшего обращения к \glslink{heap}{куче}.

Код на GCC даже проще (из-за того, что в GCC, как мы уже видели, не реализована возможность хранить короткую
строку прямо в структуре):

% TODO1 comment each function meaning
\lstinputlisting[caption=GCC 4.8.1,style=customasmx86]{\CURPATH/STL/string/3_GCC_RU.s}

Можно заметить, что в деструкторы передается не указатель на объект,
а указатель на место за 12 байт (или 3 слова) перед ним, то есть, на настоящее начало структуры.

\myparagraph{std::string как глобальная переменная}
\label{sec:std_string_as_global_variable}

Опытные программисты на \Cpp знают, что глобальные переменные \ac{STL}-типов вполне можно объявлять.

Да, действительно:

\lstinputlisting[style=customc]{\CURPATH/STL/string/5.cpp}

Но как и где будет вызываться конструктор \TT{std::string}?

На самом деле, эта переменная будет инициализирована даже перед началом \main.

\lstinputlisting[caption=MSVC 2012: здесь конструируется глобальная переменная{,} а также регистрируется её деструктор,style=customasmx86]{\CURPATH/STL/string/5_MSVC_p2.asm}

\lstinputlisting[caption=MSVC 2012: здесь глобальная переменная используется в \main,style=customasmx86]{\CURPATH/STL/string/5_MSVC_p1.asm}

\lstinputlisting[caption=MSVC 2012: эта функция-деструктор вызывается перед выходом,style=customasmx86]{\CURPATH/STL/string/5_MSVC_p3.asm}

\myindex{\CStandardLibrary!atexit()}
В реальности, из \ac{CRT}, еще до вызова main(), вызывается специальная функция,
в которой перечислены все конструкторы подобных переменных.
Более того: при помощи atexit() регистрируется функция, которая будет вызвана в конце работы программы:
в этой функции компилятор собирает вызовы деструкторов всех подобных глобальных переменных.

GCC работает похожим образом:

\lstinputlisting[caption=GCC 4.8.1,style=customasmx86]{\CURPATH/STL/string/5_GCC.s}

Но он не выделяет отдельной функции в которой будут собраны деструкторы: 
каждый деструктор передается в atexit() по одному.

% TODO а если глобальная STL-переменная в другом модуле? надо проверить.

}

\EN{\section{Returning Values}
\label{ret_val_func}

Another simple function is the one that simply returns a constant value:

\lstinputlisting[caption=\EN{\CCpp Code},style=customc]{patterns/011_ret/1.c}

Let's compile it.

\subsection{x86}

Here's what both the GCC and MSVC compilers produce (with optimization) on the x86 platform:

\lstinputlisting[caption=\Optimizing GCC/MSVC (\assemblyOutput),style=customasmx86]{patterns/011_ret/1.s}

\myindex{x86!\Instructions!RET}
There are just two instructions: the first places the value 123 into the \EAX register,
which is used by convention for storing the return
value, and the second one is \RET, which returns execution to the \gls{caller}.

The caller will take the result from the \EAX register.

\subsection{ARM}

There are a few differences on the ARM platform:

\lstinputlisting[caption=\OptimizingKeilVI (\ARMMode) ASM Output,style=customasmARM]{patterns/011_ret/1_Keil_ARM_O3.s}

ARM uses the register \Reg{0} for returning the results of functions, so 123 is copied into \Reg{0}.

\myindex{ARM!\Instructions!MOV}
\myindex{x86!\Instructions!MOV}
It is worth noting that \MOV is a misleading name for the instruction in both the x86 and ARM \ac{ISA}s.

The data is not in fact \IT{moved}, but \IT{copied}.

\subsection{MIPS}

\label{MIPS_leaf_function_ex1}

The GCC assembly output below lists registers by number:

\lstinputlisting[caption=\Optimizing GCC 4.4.5 (\assemblyOutput),style=customasmMIPS]{patterns/011_ret/MIPS.s}

\dots while \IDA does it by their pseudo names:

\lstinputlisting[caption=\Optimizing GCC 4.4.5 (IDA),style=customasmMIPS]{patterns/011_ret/MIPS_IDA.lst}

The \$2 (or \$V0) register is used to store the function's return value.
\myindex{MIPS!\Pseudoinstructions!LI}
\INS{LI} stands for ``Load Immediate'' and is the MIPS equivalent to \MOV.

\myindex{MIPS!\Instructions!J}
The other instruction is the jump instruction (J or JR) which returns the execution flow to the \gls{caller}.

\myindex{MIPS!Branch delay slot}
You might be wondering why the positions of the load instruction (LI) and the jump instruction (J or JR) are swapped. This is due to a \ac{RISC} feature called ``branch delay slot''.

The reason this happens is a quirk in the architecture of some RISC \ac{ISA}s and isn't important for our
purposes---we must simply keep in mind that in MIPS, the instruction following a jump or branch instruction
is executed \IT{before} the jump/branch instruction itself.

As a consequence, branch instructions always swap places with the instruction executed immediately beforehand.


In practice, functions which merely return 1 (\IT{true}) or 0 (\IT{false}) are very frequent.

The smallest ever of the standard UNIX utilities, \IT{/bin/true} and \IT{/bin/false} return 0 and 1 respectively, as an exit code.
(Zero as an exit code usually means success, non-zero means error.)
}
\RU{\subsubsection{std::string}
\myindex{\Cpp!STL!std::string}
\label{std_string}

\myparagraph{Как устроена структура}

Многие строковые библиотеки \InSqBrackets{\CNotes 2.2} обеспечивают структуру содержащую ссылку 
на буфер собственно со строкой, переменная всегда содержащую длину строки 
(что очень удобно для массы функций \InSqBrackets{\CNotes 2.2.1}) и переменную содержащую текущий размер буфера.

Строка в буфере обыкновенно оканчивается нулем: это для того чтобы указатель на буфер можно было
передавать в функции требующие на вход обычную сишную \ac{ASCIIZ}-строку.

Стандарт \Cpp не описывает, как именно нужно реализовывать std::string,
но, как правило, они реализованы как описано выше, с небольшими дополнениями.

Строки в \Cpp это не класс (как, например, QString в Qt), а темплейт (basic\_string), 
это сделано для того чтобы поддерживать 
строки содержащие разного типа символы: как минимум \Tchar и \IT{wchar\_t}.

Так что, std::string это класс с базовым типом \Tchar.

А std::wstring это класс с базовым типом \IT{wchar\_t}.

\mysubparagraph{MSVC}

В реализации MSVC, вместо ссылки на буфер может содержаться сам буфер (если строка короче 16-и символов).

Это означает, что каждая короткая строка будет занимать в памяти по крайней мере $16 + 4 + 4 = 24$ 
байт для 32-битной среды либо $16 + 8 + 8 = 32$ 
байта в 64-битной, а если строка длиннее 16-и символов, то прибавьте еще длину самой строки.

\lstinputlisting[caption=пример для MSVC,style=customc]{\CURPATH/STL/string/MSVC_RU.cpp}

Собственно, из этого исходника почти всё ясно.

Несколько замечаний:

Если строка короче 16-и символов, 
то отдельный буфер для строки в \glslink{heap}{куче} выделяться не будет.

Это удобно потому что на практике, основная часть строк действительно короткие.
Вероятно, разработчики в Microsoft выбрали размер в 16 символов как разумный баланс.

Теперь очень важный момент в конце функции main(): мы не пользуемся методом c\_str(), тем не менее,
если это скомпилировать и запустить, то обе строки появятся в консоли!

Работает это вот почему.

В первом случае строка короче 16-и символов и в начале объекта std::string (его можно рассматривать
просто как структуру) расположен буфер с этой строкой.
\printf трактует указатель как указатель на массив символов оканчивающийся нулем и поэтому всё работает.

Вывод второй строки (длиннее 16-и символов) даже еще опаснее: это вообще типичная программистская ошибка 
(или опечатка), забыть дописать c\_str().
Это работает потому что в это время в начале структуры расположен указатель на буфер.
Это может надолго остаться незамеченным: до тех пока там не появится строка 
короче 16-и символов, тогда процесс упадет.

\mysubparagraph{GCC}

В реализации GCC в структуре есть еще одна переменная --- reference count.

Интересно, что указатель на экземпляр класса std::string в GCC указывает не на начало самой структуры, 
а на указатель на буфера.
В libstdc++-v3\textbackslash{}include\textbackslash{}bits\textbackslash{}basic\_string.h 
мы можем прочитать что это сделано для удобства отладки:

\begin{lstlisting}
   *  The reason you want _M_data pointing to the character %array and
   *  not the _Rep is so that the debugger can see the string
   *  contents. (Probably we should add a non-inline member to get
   *  the _Rep for the debugger to use, so users can check the actual
   *  string length.)
\end{lstlisting}

\href{http://go.yurichev.com/17085}{исходный код basic\_string.h}

В нашем примере мы учитываем это:

\lstinputlisting[caption=пример для GCC,style=customc]{\CURPATH/STL/string/GCC_RU.cpp}

Нужны еще небольшие хаки чтобы сымитировать типичную ошибку, которую мы уже видели выше, из-за
более ужесточенной проверки типов в GCC, тем не менее, printf() работает и здесь без c\_str().

\myparagraph{Чуть более сложный пример}

\lstinputlisting[style=customc]{\CURPATH/STL/string/3.cpp}

\lstinputlisting[caption=MSVC 2012,style=customasmx86]{\CURPATH/STL/string/3_MSVC_RU.asm}

Собственно, компилятор не конструирует строки статически: да в общем-то и как
это возможно, если буфер с ней нужно хранить в \glslink{heap}{куче}?

Вместо этого в сегменте данных хранятся обычные \ac{ASCIIZ}-строки, а позже, во время выполнения, 
при помощи метода \q{assign}, конструируются строки s1 и s2
.
При помощи \TT{operator+}, создается строка s3.

Обратите внимание на то что вызов метода c\_str() отсутствует,
потому что его код достаточно короткий и компилятор вставил его прямо здесь:
если строка короче 16-и байт, то в регистре EAX остается указатель на буфер,
а если длиннее, то из этого же места достается адрес на буфер расположенный в \glslink{heap}{куче}.

Далее следуют вызовы трех деструкторов, причем, они вызываются только если строка длиннее 16-и байт:
тогда нужно освободить буфера в \glslink{heap}{куче}.
В противном случае, так как все три объекта std::string хранятся в стеке,
они освобождаются автоматически после выхода из функции.

Следовательно, работа с короткими строками более быстрая из-за м\'{е}ньшего обращения к \glslink{heap}{куче}.

Код на GCC даже проще (из-за того, что в GCC, как мы уже видели, не реализована возможность хранить короткую
строку прямо в структуре):

% TODO1 comment each function meaning
\lstinputlisting[caption=GCC 4.8.1,style=customasmx86]{\CURPATH/STL/string/3_GCC_RU.s}

Можно заметить, что в деструкторы передается не указатель на объект,
а указатель на место за 12 байт (или 3 слова) перед ним, то есть, на настоящее начало структуры.

\myparagraph{std::string как глобальная переменная}
\label{sec:std_string_as_global_variable}

Опытные программисты на \Cpp знают, что глобальные переменные \ac{STL}-типов вполне можно объявлять.

Да, действительно:

\lstinputlisting[style=customc]{\CURPATH/STL/string/5.cpp}

Но как и где будет вызываться конструктор \TT{std::string}?

На самом деле, эта переменная будет инициализирована даже перед началом \main.

\lstinputlisting[caption=MSVC 2012: здесь конструируется глобальная переменная{,} а также регистрируется её деструктор,style=customasmx86]{\CURPATH/STL/string/5_MSVC_p2.asm}

\lstinputlisting[caption=MSVC 2012: здесь глобальная переменная используется в \main,style=customasmx86]{\CURPATH/STL/string/5_MSVC_p1.asm}

\lstinputlisting[caption=MSVC 2012: эта функция-деструктор вызывается перед выходом,style=customasmx86]{\CURPATH/STL/string/5_MSVC_p3.asm}

\myindex{\CStandardLibrary!atexit()}
В реальности, из \ac{CRT}, еще до вызова main(), вызывается специальная функция,
в которой перечислены все конструкторы подобных переменных.
Более того: при помощи atexit() регистрируется функция, которая будет вызвана в конце работы программы:
в этой функции компилятор собирает вызовы деструкторов всех подобных глобальных переменных.

GCC работает похожим образом:

\lstinputlisting[caption=GCC 4.8.1,style=customasmx86]{\CURPATH/STL/string/5_GCC.s}

Но он не выделяет отдельной функции в которой будут собраны деструкторы: 
каждый деструктор передается в atexit() по одному.

% TODO а если глобальная STL-переменная в другом модуле? надо проверить.

}
\DE{\subsection{Einfachste XOR-Verschlüsselung überhaupt}

Ich habe einmal eine Software gesehen, bei der alle Debugging-Ausgaben mit XOR mit dem Wert 3
verschlüsselt wurden. Mit anderen Worten, die beiden niedrigsten Bits aller Buchstaben wurden invertiert.

``Hello, world'' wurde zu ``Kfool/\#tlqog'':

\begin{lstlisting}
#!/usr/bin/python

msg="Hello, world!"

print "".join(map(lambda x: chr(ord(x)^3), msg))
\end{lstlisting}

Das ist eine ziemlich interessante Verschlüsselung (oder besser eine Verschleierung),
weil sie zwei wichtige Eigenschaften hat:
1) es ist eine einzige Funktion zum Verschlüsseln und entschlüsseln, sie muss nur wiederholt angewendet werden
2) die entstehenden Buchstaben befinden sich im druckbaren Bereich, also die ganze Zeichenkette kann ohne
Escape-Symbole im Code verwendet werden.

Die zweite Eigenschaft nutzt die Tatsache, dass alle druckbaren Zeichen in Reihen organisiert sind: 0x2x-0x7x,
und wenn die beiden niederwertigsten Bits invertiert werden, wird der Buchstabe um eine oder drei Stellen nach
links oder rechts \IT{verschoben}, aber niemals in eine andere Reihe:

\begin{figure}[H]
\centering
\includegraphics[width=0.7\textwidth]{ascii_clean.png}
\caption{7-Bit \ac{ASCII} Tabelle in Emacs}
\end{figure}

\dots mit dem Zeichen 0x7F als einziger Ausnahme.

Im Folgenden werden also beispielsweise die Zeichen A-Z \IT{verschlüsselt}:

\begin{lstlisting}
#!/usr/bin/python

msg="@ABCDEFGHIJKLMNO"

print "".join(map(lambda x: chr(ord(x)^3), msg))
\end{lstlisting}

Ergebnis:
% FIXME \verb  --  relevant comment for German?
\begin{lstlisting}
CBA@GFEDKJIHONML
\end{lstlisting}

Es sieht so aus als würden die Zeichen ``@'' und ``C'' sowie ``B'' und ``A'' vertauscht werden.

Hier ist noch ein interessantes Beispiel, in dem gezeigt wird, wie die Eigenschaften von XOR
ausgenutzt werden können: Exakt den gleichen Effekt, dass druckbare Zeichen auch druckbar bleiben,
kann man dadurch erzielen, dass irgendeine Kombination der niedrigsten vier Bits invertiert wird.
}

\EN{\section{Returning Values}
\label{ret_val_func}

Another simple function is the one that simply returns a constant value:

\lstinputlisting[caption=\EN{\CCpp Code},style=customc]{patterns/011_ret/1.c}

Let's compile it.

\subsection{x86}

Here's what both the GCC and MSVC compilers produce (with optimization) on the x86 platform:

\lstinputlisting[caption=\Optimizing GCC/MSVC (\assemblyOutput),style=customasmx86]{patterns/011_ret/1.s}

\myindex{x86!\Instructions!RET}
There are just two instructions: the first places the value 123 into the \EAX register,
which is used by convention for storing the return
value, and the second one is \RET, which returns execution to the \gls{caller}.

The caller will take the result from the \EAX register.

\subsection{ARM}

There are a few differences on the ARM platform:

\lstinputlisting[caption=\OptimizingKeilVI (\ARMMode) ASM Output,style=customasmARM]{patterns/011_ret/1_Keil_ARM_O3.s}

ARM uses the register \Reg{0} for returning the results of functions, so 123 is copied into \Reg{0}.

\myindex{ARM!\Instructions!MOV}
\myindex{x86!\Instructions!MOV}
It is worth noting that \MOV is a misleading name for the instruction in both the x86 and ARM \ac{ISA}s.

The data is not in fact \IT{moved}, but \IT{copied}.

\subsection{MIPS}

\label{MIPS_leaf_function_ex1}

The GCC assembly output below lists registers by number:

\lstinputlisting[caption=\Optimizing GCC 4.4.5 (\assemblyOutput),style=customasmMIPS]{patterns/011_ret/MIPS.s}

\dots while \IDA does it by their pseudo names:

\lstinputlisting[caption=\Optimizing GCC 4.4.5 (IDA),style=customasmMIPS]{patterns/011_ret/MIPS_IDA.lst}

The \$2 (or \$V0) register is used to store the function's return value.
\myindex{MIPS!\Pseudoinstructions!LI}
\INS{LI} stands for ``Load Immediate'' and is the MIPS equivalent to \MOV.

\myindex{MIPS!\Instructions!J}
The other instruction is the jump instruction (J or JR) which returns the execution flow to the \gls{caller}.

\myindex{MIPS!Branch delay slot}
You might be wondering why the positions of the load instruction (LI) and the jump instruction (J or JR) are swapped. This is due to a \ac{RISC} feature called ``branch delay slot''.

The reason this happens is a quirk in the architecture of some RISC \ac{ISA}s and isn't important for our
purposes---we must simply keep in mind that in MIPS, the instruction following a jump or branch instruction
is executed \IT{before} the jump/branch instruction itself.

As a consequence, branch instructions always swap places with the instruction executed immediately beforehand.


In practice, functions which merely return 1 (\IT{true}) or 0 (\IT{false}) are very frequent.

The smallest ever of the standard UNIX utilities, \IT{/bin/true} and \IT{/bin/false} return 0 and 1 respectively, as an exit code.
(Zero as an exit code usually means success, non-zero means error.)
}
\RU{\subsubsection{std::string}
\myindex{\Cpp!STL!std::string}
\label{std_string}

\myparagraph{Как устроена структура}

Многие строковые библиотеки \InSqBrackets{\CNotes 2.2} обеспечивают структуру содержащую ссылку 
на буфер собственно со строкой, переменная всегда содержащую длину строки 
(что очень удобно для массы функций \InSqBrackets{\CNotes 2.2.1}) и переменную содержащую текущий размер буфера.

Строка в буфере обыкновенно оканчивается нулем: это для того чтобы указатель на буфер можно было
передавать в функции требующие на вход обычную сишную \ac{ASCIIZ}-строку.

Стандарт \Cpp не описывает, как именно нужно реализовывать std::string,
но, как правило, они реализованы как описано выше, с небольшими дополнениями.

Строки в \Cpp это не класс (как, например, QString в Qt), а темплейт (basic\_string), 
это сделано для того чтобы поддерживать 
строки содержащие разного типа символы: как минимум \Tchar и \IT{wchar\_t}.

Так что, std::string это класс с базовым типом \Tchar.

А std::wstring это класс с базовым типом \IT{wchar\_t}.

\mysubparagraph{MSVC}

В реализации MSVC, вместо ссылки на буфер может содержаться сам буфер (если строка короче 16-и символов).

Это означает, что каждая короткая строка будет занимать в памяти по крайней мере $16 + 4 + 4 = 24$ 
байт для 32-битной среды либо $16 + 8 + 8 = 32$ 
байта в 64-битной, а если строка длиннее 16-и символов, то прибавьте еще длину самой строки.

\lstinputlisting[caption=пример для MSVC,style=customc]{\CURPATH/STL/string/MSVC_RU.cpp}

Собственно, из этого исходника почти всё ясно.

Несколько замечаний:

Если строка короче 16-и символов, 
то отдельный буфер для строки в \glslink{heap}{куче} выделяться не будет.

Это удобно потому что на практике, основная часть строк действительно короткие.
Вероятно, разработчики в Microsoft выбрали размер в 16 символов как разумный баланс.

Теперь очень важный момент в конце функции main(): мы не пользуемся методом c\_str(), тем не менее,
если это скомпилировать и запустить, то обе строки появятся в консоли!

Работает это вот почему.

В первом случае строка короче 16-и символов и в начале объекта std::string (его можно рассматривать
просто как структуру) расположен буфер с этой строкой.
\printf трактует указатель как указатель на массив символов оканчивающийся нулем и поэтому всё работает.

Вывод второй строки (длиннее 16-и символов) даже еще опаснее: это вообще типичная программистская ошибка 
(или опечатка), забыть дописать c\_str().
Это работает потому что в это время в начале структуры расположен указатель на буфер.
Это может надолго остаться незамеченным: до тех пока там не появится строка 
короче 16-и символов, тогда процесс упадет.

\mysubparagraph{GCC}

В реализации GCC в структуре есть еще одна переменная --- reference count.

Интересно, что указатель на экземпляр класса std::string в GCC указывает не на начало самой структуры, 
а на указатель на буфера.
В libstdc++-v3\textbackslash{}include\textbackslash{}bits\textbackslash{}basic\_string.h 
мы можем прочитать что это сделано для удобства отладки:

\begin{lstlisting}
   *  The reason you want _M_data pointing to the character %array and
   *  not the _Rep is so that the debugger can see the string
   *  contents. (Probably we should add a non-inline member to get
   *  the _Rep for the debugger to use, so users can check the actual
   *  string length.)
\end{lstlisting}

\href{http://go.yurichev.com/17085}{исходный код basic\_string.h}

В нашем примере мы учитываем это:

\lstinputlisting[caption=пример для GCC,style=customc]{\CURPATH/STL/string/GCC_RU.cpp}

Нужны еще небольшие хаки чтобы сымитировать типичную ошибку, которую мы уже видели выше, из-за
более ужесточенной проверки типов в GCC, тем не менее, printf() работает и здесь без c\_str().

\myparagraph{Чуть более сложный пример}

\lstinputlisting[style=customc]{\CURPATH/STL/string/3.cpp}

\lstinputlisting[caption=MSVC 2012,style=customasmx86]{\CURPATH/STL/string/3_MSVC_RU.asm}

Собственно, компилятор не конструирует строки статически: да в общем-то и как
это возможно, если буфер с ней нужно хранить в \glslink{heap}{куче}?

Вместо этого в сегменте данных хранятся обычные \ac{ASCIIZ}-строки, а позже, во время выполнения, 
при помощи метода \q{assign}, конструируются строки s1 и s2
.
При помощи \TT{operator+}, создается строка s3.

Обратите внимание на то что вызов метода c\_str() отсутствует,
потому что его код достаточно короткий и компилятор вставил его прямо здесь:
если строка короче 16-и байт, то в регистре EAX остается указатель на буфер,
а если длиннее, то из этого же места достается адрес на буфер расположенный в \glslink{heap}{куче}.

Далее следуют вызовы трех деструкторов, причем, они вызываются только если строка длиннее 16-и байт:
тогда нужно освободить буфера в \glslink{heap}{куче}.
В противном случае, так как все три объекта std::string хранятся в стеке,
они освобождаются автоматически после выхода из функции.

Следовательно, работа с короткими строками более быстрая из-за м\'{е}ньшего обращения к \glslink{heap}{куче}.

Код на GCC даже проще (из-за того, что в GCC, как мы уже видели, не реализована возможность хранить короткую
строку прямо в структуре):

% TODO1 comment each function meaning
\lstinputlisting[caption=GCC 4.8.1,style=customasmx86]{\CURPATH/STL/string/3_GCC_RU.s}

Можно заметить, что в деструкторы передается не указатель на объект,
а указатель на место за 12 байт (или 3 слова) перед ним, то есть, на настоящее начало структуры.

\myparagraph{std::string как глобальная переменная}
\label{sec:std_string_as_global_variable}

Опытные программисты на \Cpp знают, что глобальные переменные \ac{STL}-типов вполне можно объявлять.

Да, действительно:

\lstinputlisting[style=customc]{\CURPATH/STL/string/5.cpp}

Но как и где будет вызываться конструктор \TT{std::string}?

На самом деле, эта переменная будет инициализирована даже перед началом \main.

\lstinputlisting[caption=MSVC 2012: здесь конструируется глобальная переменная{,} а также регистрируется её деструктор,style=customasmx86]{\CURPATH/STL/string/5_MSVC_p2.asm}

\lstinputlisting[caption=MSVC 2012: здесь глобальная переменная используется в \main,style=customasmx86]{\CURPATH/STL/string/5_MSVC_p1.asm}

\lstinputlisting[caption=MSVC 2012: эта функция-деструктор вызывается перед выходом,style=customasmx86]{\CURPATH/STL/string/5_MSVC_p3.asm}

\myindex{\CStandardLibrary!atexit()}
В реальности, из \ac{CRT}, еще до вызова main(), вызывается специальная функция,
в которой перечислены все конструкторы подобных переменных.
Более того: при помощи atexit() регистрируется функция, которая будет вызвана в конце работы программы:
в этой функции компилятор собирает вызовы деструкторов всех подобных глобальных переменных.

GCC работает похожим образом:

\lstinputlisting[caption=GCC 4.8.1,style=customasmx86]{\CURPATH/STL/string/5_GCC.s}

Но он не выделяет отдельной функции в которой будут собраны деструкторы: 
каждый деструктор передается в atexit() по одному.

% TODO а если глобальная STL-переменная в другом модуле? надо проверить.

}
\DE{\subsection{Einfachste XOR-Verschlüsselung überhaupt}

Ich habe einmal eine Software gesehen, bei der alle Debugging-Ausgaben mit XOR mit dem Wert 3
verschlüsselt wurden. Mit anderen Worten, die beiden niedrigsten Bits aller Buchstaben wurden invertiert.

``Hello, world'' wurde zu ``Kfool/\#tlqog'':

\begin{lstlisting}
#!/usr/bin/python

msg="Hello, world!"

print "".join(map(lambda x: chr(ord(x)^3), msg))
\end{lstlisting}

Das ist eine ziemlich interessante Verschlüsselung (oder besser eine Verschleierung),
weil sie zwei wichtige Eigenschaften hat:
1) es ist eine einzige Funktion zum Verschlüsseln und entschlüsseln, sie muss nur wiederholt angewendet werden
2) die entstehenden Buchstaben befinden sich im druckbaren Bereich, also die ganze Zeichenkette kann ohne
Escape-Symbole im Code verwendet werden.

Die zweite Eigenschaft nutzt die Tatsache, dass alle druckbaren Zeichen in Reihen organisiert sind: 0x2x-0x7x,
und wenn die beiden niederwertigsten Bits invertiert werden, wird der Buchstabe um eine oder drei Stellen nach
links oder rechts \IT{verschoben}, aber niemals in eine andere Reihe:

\begin{figure}[H]
\centering
\includegraphics[width=0.7\textwidth]{ascii_clean.png}
\caption{7-Bit \ac{ASCII} Tabelle in Emacs}
\end{figure}

\dots mit dem Zeichen 0x7F als einziger Ausnahme.

Im Folgenden werden also beispielsweise die Zeichen A-Z \IT{verschlüsselt}:

\begin{lstlisting}
#!/usr/bin/python

msg="@ABCDEFGHIJKLMNO"

print "".join(map(lambda x: chr(ord(x)^3), msg))
\end{lstlisting}

Ergebnis:
% FIXME \verb  --  relevant comment for German?
\begin{lstlisting}
CBA@GFEDKJIHONML
\end{lstlisting}

Es sieht so aus als würden die Zeichen ``@'' und ``C'' sowie ``B'' und ``A'' vertauscht werden.

Hier ist noch ein interessantes Beispiel, in dem gezeigt wird, wie die Eigenschaften von XOR
ausgenutzt werden können: Exakt den gleichen Effekt, dass druckbare Zeichen auch druckbar bleiben,
kann man dadurch erzielen, dass irgendeine Kombination der niedrigsten vier Bits invertiert wird.
}

\ifdefined\SPANISH
\chapter{Patrones de código}
\fi % SPANISH

\ifdefined\GERMAN
\chapter{Code-Muster}
\fi % GERMAN

\ifdefined\ENGLISH
\chapter{Code Patterns}
\fi % ENGLISH

\ifdefined\ITALIAN
\chapter{Forme di codice}
\fi % ITALIAN

\ifdefined\RUSSIAN
\chapter{Образцы кода}
\fi % RUSSIAN

\ifdefined\BRAZILIAN
\chapter{Padrões de códigos}
\fi % BRAZILIAN

\ifdefined\THAI
\chapter{รูปแบบของโค้ด}
\fi % THAI

\ifdefined\FRENCH
\chapter{Modèle de code}
\fi % FRENCH

\ifdefined\POLISH
\chapter{\PLph{}}
\fi % POLISH

% sections
\EN{\input{patterns/patterns_opt_dbg_EN}}
\ES{\input{patterns/patterns_opt_dbg_ES}}
\ITA{\input{patterns/patterns_opt_dbg_ITA}}
\PTBR{\input{patterns/patterns_opt_dbg_PTBR}}
\RU{\input{patterns/patterns_opt_dbg_RU}}
\THA{\input{patterns/patterns_opt_dbg_THA}}
\DE{\input{patterns/patterns_opt_dbg_DE}}
\FR{\input{patterns/patterns_opt_dbg_FR}}
\PL{\input{patterns/patterns_opt_dbg_PL}}

\RU{\section{Некоторые базовые понятия}}
\EN{\section{Some basics}}
\DE{\section{Einige Grundlagen}}
\FR{\section{Quelques bases}}
\ES{\section{\ESph{}}}
\ITA{\section{Alcune basi teoriche}}
\PTBR{\section{\PTBRph{}}}
\THA{\section{\THAph{}}}
\PL{\section{\PLph{}}}

% sections:
\EN{\input{patterns/intro_CPU_ISA_EN}}
\ES{\input{patterns/intro_CPU_ISA_ES}}
\ITA{\input{patterns/intro_CPU_ISA_ITA}}
\PTBR{\input{patterns/intro_CPU_ISA_PTBR}}
\RU{\input{patterns/intro_CPU_ISA_RU}}
\DE{\input{patterns/intro_CPU_ISA_DE}}
\FR{\input{patterns/intro_CPU_ISA_FR}}
\PL{\input{patterns/intro_CPU_ISA_PL}}

\EN{\input{patterns/numeral_EN}}
\RU{\input{patterns/numeral_RU}}
\ITA{\input{patterns/numeral_ITA}}
\DE{\input{patterns/numeral_DE}}
\FR{\input{patterns/numeral_FR}}
\PL{\input{patterns/numeral_PL}}

% chapters
\input{patterns/00_empty/main}
\input{patterns/011_ret/main}
\input{patterns/01_helloworld/main}
\input{patterns/015_prolog_epilogue/main}
\input{patterns/02_stack/main}
\input{patterns/03_printf/main}
\input{patterns/04_scanf/main}
\input{patterns/05_passing_arguments/main}
\input{patterns/06_return_results/main}
\input{patterns/061_pointers/main}
\input{patterns/065_GOTO/main}
\input{patterns/07_jcc/main}
\input{patterns/08_switch/main}
\input{patterns/09_loops/main}
\input{patterns/10_strings/main}
\input{patterns/11_arith_optimizations/main}
\input{patterns/12_FPU/main}
\input{patterns/13_arrays/main}
\input{patterns/14_bitfields/main}
\EN{\input{patterns/145_LCG/main_EN}}
\RU{\input{patterns/145_LCG/main_RU}}
\input{patterns/15_structs/main}
\input{patterns/17_unions/main}
\input{patterns/18_pointers_to_functions/main}
\input{patterns/185_64bit_in_32_env/main}

\EN{\input{patterns/19_SIMD/main_EN}}
\RU{\input{patterns/19_SIMD/main_RU}}
\DE{\input{patterns/19_SIMD/main_DE}}

\EN{\input{patterns/20_x64/main_EN}}
\RU{\input{patterns/20_x64/main_RU}}

\EN{\input{patterns/205_floating_SIMD/main_EN}}
\RU{\input{patterns/205_floating_SIMD/main_RU}}
\DE{\input{patterns/205_floating_SIMD/main_DE}}

\EN{\input{patterns/ARM/main_EN}}
\RU{\input{patterns/ARM/main_RU}}
\DE{\input{patterns/ARM/main_DE}}

\input{patterns/MIPS/main}


\ifdefined\SPANISH
\chapter{Patrones de código}
\fi % SPANISH

\ifdefined\GERMAN
\chapter{Code-Muster}
\fi % GERMAN

\ifdefined\ENGLISH
\chapter{Code Patterns}
\fi % ENGLISH

\ifdefined\ITALIAN
\chapter{Forme di codice}
\fi % ITALIAN

\ifdefined\RUSSIAN
\chapter{Образцы кода}
\fi % RUSSIAN

\ifdefined\BRAZILIAN
\chapter{Padrões de códigos}
\fi % BRAZILIAN

\ifdefined\THAI
\chapter{รูปแบบของโค้ด}
\fi % THAI

\ifdefined\FRENCH
\chapter{Modèle de code}
\fi % FRENCH

\ifdefined\POLISH
\chapter{\PLph{}}
\fi % POLISH

% sections
\EN{\section{The method}

When the author of this book first started learning C and, later, \Cpp, he used to write small pieces of code, compile them,
and then look at the assembly language output. This made it very easy for him to understand what was going on in the code that he had written.
\footnote{In fact, he still does this when he can't understand what a particular bit of code does.}.
He did this so many times that the relationship between the \CCpp code and what the compiler produced was imprinted deeply in his mind.
It's now easy for him to imagine instantly a rough outline of a C code's appearance and function.
Perhaps this technique could be helpful for others.

%There are a lot of examples for both x86/x64 and ARM.
%Those who already familiar with one of architectures, may freely skim over pages.

By the way, there is a great website where you can do the same, with various compilers, instead of installing them on your box.
You can use it as well: \url{https://gcc.godbolt.org/}.

\section*{\Exercises}

When the author of this book studied assembly language, he also often compiled small C functions and then rewrote
them gradually to assembly, trying to make their code as short as possible.
This probably is not worth doing in real-world scenarios today,
because it's hard to compete with the latest compilers in terms of efficiency. It is, however, a very good way to gain a better understanding of assembly.
Feel free, therefore, to take any assembly code from this book and try to make it shorter.
However, don't forget to test what you have written.

% rewrote to show that debug\release and optimisations levels are orthogonal concepts.
\section*{Optimization levels and debug information}

Source code can be compiled by different compilers with various optimization levels.
A typical compiler has about three such levels, where level zero means that optimization is completely disabled.
Optimization can also be targeted towards code size or code speed.
A non-optimizing compiler is faster and produces more understandable (albeit verbose) code,
whereas an optimizing compiler is slower and tries to produce code that runs faster (but is not necessarily more compact).
In addition to optimization levels, a compiler can include some debug information in the resulting file,
producing code that is easy to debug.
One of the important features of the ´debug' code is that it might contain links
between each line of the source code and its respective machine code address.
Optimizing compilers, on the other hand, tend to produce output where entire lines of source code
can be optimized away and thus not even be present in the resulting machine code.
Reverse engineers can encounter either version, simply because some developers turn on the compiler's optimization flags and others do not.
Because of this, we'll try to work on examples of both debug and release versions of the code featured in this book, wherever possible.

Sometimes some pretty ancient compilers are used in this book, in order to get the shortest (or simplest) possible code snippet.
}
\ES{\input{patterns/patterns_opt_dbg_ES}}
\ITA{\input{patterns/patterns_opt_dbg_ITA}}
\PTBR{\input{patterns/patterns_opt_dbg_PTBR}}
\RU{\input{patterns/patterns_opt_dbg_RU}}
\THA{\input{patterns/patterns_opt_dbg_THA}}
\DE{\input{patterns/patterns_opt_dbg_DE}}
\FR{\input{patterns/patterns_opt_dbg_FR}}
\PL{\input{patterns/patterns_opt_dbg_PL}}

\RU{\section{Некоторые базовые понятия}}
\EN{\section{Some basics}}
\DE{\section{Einige Grundlagen}}
\FR{\section{Quelques bases}}
\ES{\section{\ESph{}}}
\ITA{\section{Alcune basi teoriche}}
\PTBR{\section{\PTBRph{}}}
\THA{\section{\THAph{}}}
\PL{\section{\PLph{}}}

% sections:
\EN{\input{patterns/intro_CPU_ISA_EN}}
\ES{\input{patterns/intro_CPU_ISA_ES}}
\ITA{\input{patterns/intro_CPU_ISA_ITA}}
\PTBR{\input{patterns/intro_CPU_ISA_PTBR}}
\RU{\input{patterns/intro_CPU_ISA_RU}}
\DE{\input{patterns/intro_CPU_ISA_DE}}
\FR{\input{patterns/intro_CPU_ISA_FR}}
\PL{\input{patterns/intro_CPU_ISA_PL}}

\EN{\subsection{Numeral Systems}

Humans have become accustomed to a decimal numeral system, probably because almost everyone has 10 fingers.
Nevertheless, the number \q{10} has no significant meaning in science and mathematics.
The natural numeral system in digital electronics is binary: 0 is for an absence of current in the wire, and 1 for presence.
10 in binary is 2 in decimal, 100 in binary is 4 in decimal, and so on.

% This sentence is a bit unweildy - maybe try 'Our ten-digit system would be described as having a radix...' - Renaissance
If the numeral system has 10 digits, it has a \IT{radix} (or \IT{base}) of 10.
The binary numeral system has a \IT{radix} of 2.

Important things to recall:

1) A \IT{number} is a number, while a \IT{digit} is a term from writing systems, and is usually one character

% The original is 'number' is not changed; I think the intent is value, and changed it - Renaissance
2) The value of a number does not change when converted to another radix; only the writing notation for that value has changed (and therefore the way of representing it in \ac{RAM}).

\subsection{Converting From One Radix To Another}

Positional notation is used almost every numerical system. This means that a digit has weight relative to where it is placed inside of the larger number.
If 2 is placed at the rightmost place, it's 2, but if it's placed one digit before rightmost, it's 20.

What does $1234$ stand for?

$10^3 \cdot 1 + 10^2 \cdot 2 + 10^1 \cdot 3 + 1 \cdot 4 = 1234$ or
$1000 \cdot 1 + 100 \cdot 2 + 10 \cdot 3 + 4 = 1234$

It's the same story for binary numbers, but the base is 2 instead of 10.
What does 0b101011 stand for?

$2^5 \cdot 1 + 2^4 \cdot 0 + 2^3 \cdot 1 + 2^2 \cdot 0 + 2^1 \cdot 1 + 2^0 \cdot 1 = 43$ or
$32 \cdot 1 + 16 \cdot 0 + 8 \cdot 1 + 4 \cdot 0 + 2 \cdot 1 + 1 = 43$

There is such a thing as non-positional notation, such as the Roman numeral system.
\footnote{About numeric system evolution, see \InSqBrackets{\TAOCPvolII{}, 195--213.}}.
% Maybe add a sentence to fill in that X is always 10, and is therefore non-positional, even though putting an I before subtracts and after adds, and is in that sense positional
Perhaps, humankind switched to positional notation because it's easier to do basic operations (addition, multiplication, etc.) on paper by hand.

Binary numbers can be added, subtracted and so on in the very same as taught in schools, but only 2 digits are available.

Binary numbers are bulky when represented in source code and dumps, so that is where the hexadecimal numeral system can be useful.
A hexadecimal radix uses the digits 0..9, and also 6 Latin characters: A..F.
Each hexadecimal digit takes 4 bits or 4 binary digits, so it's very easy to convert from binary number to hexadecimal and back, even manually, in one's mind.

\begin{center}
\begin{longtable}{ | l | l | l | }
\hline
\HeaderColor hexadecimal & \HeaderColor binary & \HeaderColor decimal \\
\hline
0	&0000	&0 \\
1	&0001	&1 \\
2	&0010	&2 \\
3	&0011	&3 \\
4	&0100	&4 \\
5	&0101	&5 \\
6	&0110	&6 \\
7	&0111	&7 \\
8	&1000	&8 \\
9	&1001	&9 \\
A	&1010	&10 \\
B	&1011	&11 \\
C	&1100	&12 \\
D	&1101	&13 \\
E	&1110	&14 \\
F	&1111	&15 \\
\hline
\end{longtable}
\end{center}

How can one tell which radix is being used in a specific instance?

Decimal numbers are usually written as is, i.e., 1234. Some assemblers allow an identifier on decimal radix numbers, in which the number would be written with a "d" suffix: 1234d.

Binary numbers are sometimes prepended with the "0b" prefix: 0b100110111 (\ac{GCC} has a non-standard language extension for this\footnote{\url{https://gcc.gnu.org/onlinedocs/gcc/Binary-constants.html}}).
There is also another way: using a "b" suffix, for example: 100110111b.
This book tries to use the "0b" prefix consistently throughout the book for binary numbers.

Hexadecimal numbers are prepended with "0x" prefix in \CCpp and other \ac{PL}s: 0x1234ABCD.
Alternatively, they are given a "h" suffix: 1234ABCDh. This is common way of representing them in assemblers and debuggers.
In this convention, if the number is started with a Latin (A..F) digit, a 0 is added at the beginning: 0ABCDEFh.
There was also convention that was popular in 8-bit home computers era, using \$ prefix, like \$ABCD.
The book will try to stick to "0x" prefix throughout the book for hexadecimal numbers.

Should one learn to convert numbers mentally? A table of 1-digit hexadecimal numbers can easily be memorized.
As for larger numbers, it's probably not worth tormenting yourself.

Perhaps the most visible hexadecimal numbers are in \ac{URL}s.
This is the way that non-Latin characters are encoded.
For example:
\url{https://en.wiktionary.org/wiki/na\%C3\%AFvet\%C3\%A9} is the \ac{URL} of Wiktionary article about \q{naïveté} word.

\subsubsection{Octal Radix}

Another numeral system heavily used in the past of computer programming is octal. In octal there are 8 digits (0..7), and each is mapped to 3 bits, so it's easy to convert numbers back and forth.
It has been superseded by the hexadecimal system almost everywhere, but, surprisingly, there is a *NIX utility, used often by many people, which takes octal numbers as argument: \TT{chmod}.

\myindex{UNIX!chmod}
As many *NIX users know, \TT{chmod} argument can be a number of 3 digits. The first digit represents the rights of the owner of the file (read, write and/or execute), the second is the rights for the group to which the file belongs, and the third is for everyone else.
Each digit that \TT{chmod} takes can be represented in binary form:

\begin{center}
\begin{longtable}{ | l | l | l | }
\hline
\HeaderColor decimal & \HeaderColor binary & \HeaderColor meaning \\
\hline
7	&111	&\textbf{rwx} \\
6	&110	&\textbf{rw-} \\
5	&101	&\textbf{r-x} \\
4	&100	&\textbf{r-{}-} \\
3	&011	&\textbf{-wx} \\
2	&010	&\textbf{-w-} \\
1	&001	&\textbf{-{}-x} \\
0	&000	&\textbf{-{}-{}-} \\
\hline
\end{longtable}
\end{center}

So each bit is mapped to a flag: read/write/execute.

The importance of \TT{chmod} here is that the whole number in argument can be represented as octal number.
Let's take, for example, 644.
When you run \TT{chmod 644 file}, you set read/write permissions for owner, read permissions for group and again, read permissions for everyone else.
If we convert the octal number 644 to binary, it would be \TT{110100100}, or, in groups of 3 bits, \TT{110 100 100}.

Now we see that each triplet describe permissions for owner/group/others: first is \TT{rw-}, second is \TT{r--} and third is \TT{r--}.

The octal numeral system was also popular on old computers like PDP-8, because word there could be 12, 24 or 36 bits, and these numbers are all divisible by 3, so the octal system was natural in that environment.
Nowadays, all popular computers employ word/address sizes of 16, 32 or 64 bits, and these numbers are all divisible by 4, so the hexadecimal system is more natural there.

The octal numeral system is supported by all standard \CCpp compilers.
This is a source of confusion sometimes, because octal numbers are encoded with a zero prepended, for example, 0377 is 255.
Sometimes, you might make a typo and write "09" instead of 9, and the compiler would report an error.
GCC might report something like this:\\
\TT{error: invalid digit "9" in octal constant}.

Also, the octal system is somewhat popular in Java. When the IDA shows Java strings with non-printable characters,
they are encoded in the octal system instead of hexadecimal.
\myindex{JAD}
The JAD Java decompiler behaves the same way.

\subsubsection{Divisibility}

When you see a decimal number like 120, you can quickly deduce that it's divisible by 10, because the last digit is zero.
In the same way, 123400 is divisible by 100, because the two last digits are zeros.

Likewise, the hexadecimal number 0x1230 is divisible by 0x10 (or 16), 0x123000 is divisible by 0x1000 (or 4096), etc.

The binary number 0b1000101000 is divisible by 0b1000 (8), etc.

This property can often be used to quickly realize if the size of some block in memory is padded to some boundary.
For example, sections in \ac{PE} files are almost always started at addresses ending with 3 hexadecimal zeros: 0x41000, 0x10001000, etc.
The reason behind this is the fact that almost all \ac{PE} sections are padded to a boundary of 0x1000 (4096) bytes.

\subsubsection{Multi-Precision Arithmetic and Radix}

\index{RSA}
Multi-precision arithmetic can use huge numbers, and each one may be stored in several bytes.
For example, RSA keys, both public and private, span up to 4096 bits, and maybe even more.

% I'm not sure how to change this, but the normal format for quoting would be just to mention the author or book, and footnote to the full reference
In \InSqBrackets{\TAOCPvolII, 265} we find the following idea: when you store a multi-precision number in several bytes,
the whole number can be represented as having a radix of $2^8=256$, and each digit goes to the corresponding byte.
Likewise, if you store a multi-precision number in several 32-bit integer values, each digit goes to each 32-bit slot,
and you may think about this number as stored in radix of $2^{32}$.

\subsubsection{How to Pronounce Non-Decimal Numbers}

Numbers in a non-decimal base are usually pronounced by digit by digit: ``one-zero-zero-one-one-...''.
Words like ``ten'' and ``thousand'' are usually not pronounced, to prevent confusion with the decimal base system.

\subsubsection{Floating point numbers}

To distinguish floating point numbers from integers, they are usually written with ``.0'' at the end,
like $0.0$, $123.0$, etc.
}
\RU{\subsection{Представление чисел}

Люди привыкли к десятичной системе счисления вероятно потому что почти у каждого есть по 10 пальцев.
Тем не менее, число 10 не имеет особого значения в науке и математике.
Двоичная система естествена для цифровой электроники: 0 означает отсутствие тока в проводе и 1 --- его присутствие.
10 в двоичной системе это 2 в десятичной; 100 в двоичной это 4 в десятичной, итд.

Если в системе счисления есть 10 цифр, её \IT{основание} или \IT{radix} это 10.
Двоичная система имеет \IT{основание} 2.

Важные вещи, которые полезно вспомнить:
1) \IT{число} это число, в то время как \IT{цифра} это термин из системы письменности, и это обычно один символ;
2) само число не меняется, когда конвертируется из одного основания в другое: меняется способ его записи (или представления
в памяти).

Как сконвертировать число из одного основания в другое?

Позиционная нотация используется почти везде, это означает, что всякая цифра имеет свой вес, в зависимости от её расположения
внутри числа.
Если 2 расположена в самом последнем месте справа, это 2.
Если она расположена в месте перед последним, это 20.

Что означает $1234$?

$10^3 \cdot 1 + 10^2 \cdot 2 + 10^1 \cdot 3 + 1 \cdot 4$ = 1234 или
$1000 \cdot 1 + 100 \cdot 2 + 10 \cdot 3 + 4 = 1234$

Та же история и для двоичных чисел, только основание там 2 вместо 10.
Что означает 0b101011?

$2^5 \cdot 1 + 2^4 \cdot 0 + 2^3 \cdot 1 + 2^2 \cdot 0 + 2^1 \cdot 1 + 2^0 \cdot 1 = 43$ или
$32 \cdot 1 + 16 \cdot 0 + 8 \cdot 1 + 4 \cdot 0 + 2 \cdot 1 + 1 = 43$

Позиционную нотацию можно противопоставить непозиционной нотации, такой как римская система записи чисел
\footnote{Об эволюции способов записи чисел, см.также: \InSqBrackets{\TAOCPvolII{}, 195--213.}}.
Вероятно, человечество перешло на позиционную нотацию, потому что так проще работать с числами (сложение, умножение, итд)
на бумаге, в ручную.

Действительно, двоичные числа можно складывать, вычитать, итд, точно также, как этому обычно обучают в школах,
только доступны лишь 2 цифры.

Двоичные числа громоздки, когда их используют в исходных кодах и дампах, так что в этих случаях применяется шестнадцатеричная
система.
Используются цифры 0..9 и еще 6 латинских букв: A..F.
Каждая шестнадцатеричная цифра занимает 4 бита или 4 двоичных цифры, так что конвертировать из двоичной системы в
шестнадцатеричную и назад, можно легко вручную, или даже в уме.

\begin{center}
\begin{longtable}{ | l | l | l | }
\hline
\HeaderColor шестнадцатеричная & \HeaderColor двоичная & \HeaderColor десятичная \\
\hline
0	&0000	&0 \\
1	&0001	&1 \\
2	&0010	&2 \\
3	&0011	&3 \\
4	&0100	&4 \\
5	&0101	&5 \\
6	&0110	&6 \\
7	&0111	&7 \\
8	&1000	&8 \\
9	&1001	&9 \\
A	&1010	&10 \\
B	&1011	&11 \\
C	&1100	&12 \\
D	&1101	&13 \\
E	&1110	&14 \\
F	&1111	&15 \\
\hline
\end{longtable}
\end{center}

Как понять, какое основание используется в конкретном месте?

Десятичные числа обычно записываются как есть, т.е., 1234. Но некоторые ассемблеры позволяют подчеркивать
этот факт для ясности, и это число может быть дополнено суффиксом "d": 1234d.

К двоичным числам иногда спереди добавляют префикс "0b": 0b100110111
(В \ac{GCC} для этого есть нестандартное расширение языка
\footnote{\url{https://gcc.gnu.org/onlinedocs/gcc/Binary-constants.html}}).
Есть также еще один способ: суффикс "b", например: 100110111b.
В этой книге я буду пытаться придерживаться префикса "0b" для двоичных чисел.

Шестнадцатеричные числа имеют префикс "0x" в \CCpp и некоторых других \ac{PL}: 0x1234ABCD.
Либо они имеют суффикс "h": 1234ABCDh --- обычно так они представляются в ассемблерах и отладчиках.
Если число начинается с цифры A..F, перед ним добавляется 0: 0ABCDEFh.
Во времена 8-битных домашних компьютеров, был также способ записи чисел используя префикс \$, например, \$ABCD.
В книге я попытаюсь придерживаться префикса "0x" для шестнадцатеричных чисел.

Нужно ли учиться конвертировать числа в уме? Таблицу шестнадцатеричных чисел из одной цифры легко запомнить.
А запоминать б\'{о}льшие числа, наверное, не стоит.

Наверное, чаще всего шестнадцатеричные числа можно увидеть в \ac{URL}-ах.
Так кодируются буквы не из числа латинских.
Например:
\url{https://en.wiktionary.org/wiki/na\%C3\%AFvet\%C3\%A9} это \ac{URL} страницы в Wiktionary о слове \q{naïveté}.

\subsubsection{Восьмеричная система}

Еще одна система, которая в прошлом много использовалась в программировании это восьмеричная: есть 8 цифр (0..7) и каждая
описывает 3 бита, так что легко конвертировать числа туда и назад.
Она почти везде была заменена шестнадцатеричной, но удивительно, в *NIX имеется утилита использующаяся многими людьми,
которая принимает на вход восьмеричное число: \TT{chmod}.

\myindex{UNIX!chmod}
Как знают многие пользователи *NIX, аргумент \TT{chmod} это число из трех цифр. Первая цифра это права владельца файла,
вторая это права группы (которой файл принадлежит), третья для всех остальных.
И каждая цифра может быть представлена в двоичном виде:

\begin{center}
\begin{longtable}{ | l | l | l | }
\hline
\HeaderColor десятичная & \HeaderColor двоичная & \HeaderColor значение \\
\hline
7	&111	&\textbf{rwx} \\
6	&110	&\textbf{rw-} \\
5	&101	&\textbf{r-x} \\
4	&100	&\textbf{r-{}-} \\
3	&011	&\textbf{-wx} \\
2	&010	&\textbf{-w-} \\
1	&001	&\textbf{-{}-x} \\
0	&000	&\textbf{-{}-{}-} \\
\hline
\end{longtable}
\end{center}

Так что каждый бит привязан к флагу: read/write/execute (чтение/запись/исполнение).

И вот почему я вспомнил здесь о \TT{chmod}, это потому что всё число может быть представлено как число в восьмеричной системе.
Для примера возьмем 644.
Когда вы запускаете \TT{chmod 644 file}, вы выставляете права read/write для владельца, права read для группы, и снова,
read для всех остальных.
Сконвертируем число 644 из восьмеричной системы в двоичную, это будет \TT{110100100}, или (в группах по 3 бита) \TT{110 100 100}.

Теперь мы видим, что каждая тройка описывает права для владельца/группы/остальных:
первая это \TT{rw-}, вторая это \TT{r--} и третья это \TT{r--}.

Восьмеричная система была также популярная на старых компьютерах вроде PDP-8, потому что слово там могло содержать 12, 24 или
36 бит, и эти числа делятся на 3, так что выбор восьмеричной системы в той среде был логичен.
Сейчас, все популярные компьютеры имеют размер слова/адреса 16, 32 или 64 бита, и эти числа делятся на 4,
так что шестнадцатеричная система здесь удобнее.

Восьмеричная система поддерживается всеми стандартными компиляторами \CCpp{}.
Это иногда источник недоумения, потому что восьмеричные числа кодируются с нулем вперед, например, 0377 это 255.
И иногда, вы можете сделать опечатку, и написать "09" вместо 9, и компилятор выдаст ошибку.
GCC может выдать что-то вроде:\\
\TT{error: invalid digit "9" in octal constant}.

Также, восьмеричная система популярна в Java: когда IDA показывает строку с непечатаемыми символами,
они кодируются в восьмеричной системе вместо шестнадцатеричной.
\myindex{JAD}
Точно также себя ведет декомпилятор с Java JAD.

\subsubsection{Делимость}

Когда вы видите десятичное число вроде 120, вы можете быстро понять что оно делится на 10, потому что последняя цифра это 0.
Точно также, 123400 делится на 100, потому что две последних цифры это нули.

Точно также, шестнадцатеричное число 0x1230 делится на 0x10 (или 16), 0x123000 делится на 0x1000 (или 4096), итд.

Двоичное число 0b1000101000 делится на 0b1000 (8), итд.

Это свойство можно часто использовать, чтобы быстро понять,
что длина какого-либо блока в памяти выровнена по некоторой границе.
Например, секции в \ac{PE}-файлах почти всегда начинаются с адресов заканчивающихся 3 шестнадцатеричными нулями:
0x41000, 0x10001000, итд.
Причина в том, что почти все секции в \ac{PE} выровнены по границе 0x1000 (4096) байт.

\subsubsection{Арифметика произвольной точности и основание}

\index{RSA}
Арифметика произвольной точности (multi-precision arithmetic) может использовать огромные числа,
которые могут храниться в нескольких байтах.
Например, ключи RSA, и открытые и закрытые, могут занимать до 4096 бит и даже больше.

В \InSqBrackets{\TAOCPvolII, 265} можно найти такую идею: когда вы сохраняете число произвольной точности в нескольких байтах,
всё число может быть представлено как имеющую систему счисления по основанию $2^8=256$, и каждая цифра находится
в соответствующем байте.
Точно также, если вы сохраняете число произвольной точности в нескольких 32-битных целочисленных значениях,
каждая цифра отправляется в каждый 32-битный слот, и вы можете считать что это число записано в системе с основанием $2^{32}$.

\subsubsection{Произношение}

Числа в недесятичных системах счислениях обычно произносятся по одной цифре: ``один-ноль-ноль-один-один-...''.
Слова вроде ``десять'', ``тысяча'', итд, обычно не произносятся, потому что тогда можно спутать с десятичной системой.

\subsubsection{Числа с плавающей запятой}

Чтобы отличать числа с плавающей запятой от целочисленных, часто, в конце добавляют ``.0'',
например $0.0$, $123.0$, итд.

}
\ITA{\input{patterns/numeral_ITA}}
\DE{\input{patterns/numeral_DE}}
\FR{\input{patterns/numeral_FR}}
\PL{\input{patterns/numeral_PL}}

% chapters
\ifdefined\SPANISH
\chapter{Patrones de código}
\fi % SPANISH

\ifdefined\GERMAN
\chapter{Code-Muster}
\fi % GERMAN

\ifdefined\ENGLISH
\chapter{Code Patterns}
\fi % ENGLISH

\ifdefined\ITALIAN
\chapter{Forme di codice}
\fi % ITALIAN

\ifdefined\RUSSIAN
\chapter{Образцы кода}
\fi % RUSSIAN

\ifdefined\BRAZILIAN
\chapter{Padrões de códigos}
\fi % BRAZILIAN

\ifdefined\THAI
\chapter{รูปแบบของโค้ด}
\fi % THAI

\ifdefined\FRENCH
\chapter{Modèle de code}
\fi % FRENCH

\ifdefined\POLISH
\chapter{\PLph{}}
\fi % POLISH

% sections
\EN{\input{patterns/patterns_opt_dbg_EN}}
\ES{\input{patterns/patterns_opt_dbg_ES}}
\ITA{\input{patterns/patterns_opt_dbg_ITA}}
\PTBR{\input{patterns/patterns_opt_dbg_PTBR}}
\RU{\input{patterns/patterns_opt_dbg_RU}}
\THA{\input{patterns/patterns_opt_dbg_THA}}
\DE{\input{patterns/patterns_opt_dbg_DE}}
\FR{\input{patterns/patterns_opt_dbg_FR}}
\PL{\input{patterns/patterns_opt_dbg_PL}}

\RU{\section{Некоторые базовые понятия}}
\EN{\section{Some basics}}
\DE{\section{Einige Grundlagen}}
\FR{\section{Quelques bases}}
\ES{\section{\ESph{}}}
\ITA{\section{Alcune basi teoriche}}
\PTBR{\section{\PTBRph{}}}
\THA{\section{\THAph{}}}
\PL{\section{\PLph{}}}

% sections:
\EN{\input{patterns/intro_CPU_ISA_EN}}
\ES{\input{patterns/intro_CPU_ISA_ES}}
\ITA{\input{patterns/intro_CPU_ISA_ITA}}
\PTBR{\input{patterns/intro_CPU_ISA_PTBR}}
\RU{\input{patterns/intro_CPU_ISA_RU}}
\DE{\input{patterns/intro_CPU_ISA_DE}}
\FR{\input{patterns/intro_CPU_ISA_FR}}
\PL{\input{patterns/intro_CPU_ISA_PL}}

\EN{\input{patterns/numeral_EN}}
\RU{\input{patterns/numeral_RU}}
\ITA{\input{patterns/numeral_ITA}}
\DE{\input{patterns/numeral_DE}}
\FR{\input{patterns/numeral_FR}}
\PL{\input{patterns/numeral_PL}}

% chapters
\input{patterns/00_empty/main}
\input{patterns/011_ret/main}
\input{patterns/01_helloworld/main}
\input{patterns/015_prolog_epilogue/main}
\input{patterns/02_stack/main}
\input{patterns/03_printf/main}
\input{patterns/04_scanf/main}
\input{patterns/05_passing_arguments/main}
\input{patterns/06_return_results/main}
\input{patterns/061_pointers/main}
\input{patterns/065_GOTO/main}
\input{patterns/07_jcc/main}
\input{patterns/08_switch/main}
\input{patterns/09_loops/main}
\input{patterns/10_strings/main}
\input{patterns/11_arith_optimizations/main}
\input{patterns/12_FPU/main}
\input{patterns/13_arrays/main}
\input{patterns/14_bitfields/main}
\EN{\input{patterns/145_LCG/main_EN}}
\RU{\input{patterns/145_LCG/main_RU}}
\input{patterns/15_structs/main}
\input{patterns/17_unions/main}
\input{patterns/18_pointers_to_functions/main}
\input{patterns/185_64bit_in_32_env/main}

\EN{\input{patterns/19_SIMD/main_EN}}
\RU{\input{patterns/19_SIMD/main_RU}}
\DE{\input{patterns/19_SIMD/main_DE}}

\EN{\input{patterns/20_x64/main_EN}}
\RU{\input{patterns/20_x64/main_RU}}

\EN{\input{patterns/205_floating_SIMD/main_EN}}
\RU{\input{patterns/205_floating_SIMD/main_RU}}
\DE{\input{patterns/205_floating_SIMD/main_DE}}

\EN{\input{patterns/ARM/main_EN}}
\RU{\input{patterns/ARM/main_RU}}
\DE{\input{patterns/ARM/main_DE}}

\input{patterns/MIPS/main}

\ifdefined\SPANISH
\chapter{Patrones de código}
\fi % SPANISH

\ifdefined\GERMAN
\chapter{Code-Muster}
\fi % GERMAN

\ifdefined\ENGLISH
\chapter{Code Patterns}
\fi % ENGLISH

\ifdefined\ITALIAN
\chapter{Forme di codice}
\fi % ITALIAN

\ifdefined\RUSSIAN
\chapter{Образцы кода}
\fi % RUSSIAN

\ifdefined\BRAZILIAN
\chapter{Padrões de códigos}
\fi % BRAZILIAN

\ifdefined\THAI
\chapter{รูปแบบของโค้ด}
\fi % THAI

\ifdefined\FRENCH
\chapter{Modèle de code}
\fi % FRENCH

\ifdefined\POLISH
\chapter{\PLph{}}
\fi % POLISH

% sections
\EN{\input{patterns/patterns_opt_dbg_EN}}
\ES{\input{patterns/patterns_opt_dbg_ES}}
\ITA{\input{patterns/patterns_opt_dbg_ITA}}
\PTBR{\input{patterns/patterns_opt_dbg_PTBR}}
\RU{\input{patterns/patterns_opt_dbg_RU}}
\THA{\input{patterns/patterns_opt_dbg_THA}}
\DE{\input{patterns/patterns_opt_dbg_DE}}
\FR{\input{patterns/patterns_opt_dbg_FR}}
\PL{\input{patterns/patterns_opt_dbg_PL}}

\RU{\section{Некоторые базовые понятия}}
\EN{\section{Some basics}}
\DE{\section{Einige Grundlagen}}
\FR{\section{Quelques bases}}
\ES{\section{\ESph{}}}
\ITA{\section{Alcune basi teoriche}}
\PTBR{\section{\PTBRph{}}}
\THA{\section{\THAph{}}}
\PL{\section{\PLph{}}}

% sections:
\EN{\input{patterns/intro_CPU_ISA_EN}}
\ES{\input{patterns/intro_CPU_ISA_ES}}
\ITA{\input{patterns/intro_CPU_ISA_ITA}}
\PTBR{\input{patterns/intro_CPU_ISA_PTBR}}
\RU{\input{patterns/intro_CPU_ISA_RU}}
\DE{\input{patterns/intro_CPU_ISA_DE}}
\FR{\input{patterns/intro_CPU_ISA_FR}}
\PL{\input{patterns/intro_CPU_ISA_PL}}

\EN{\input{patterns/numeral_EN}}
\RU{\input{patterns/numeral_RU}}
\ITA{\input{patterns/numeral_ITA}}
\DE{\input{patterns/numeral_DE}}
\FR{\input{patterns/numeral_FR}}
\PL{\input{patterns/numeral_PL}}

% chapters
\input{patterns/00_empty/main}
\input{patterns/011_ret/main}
\input{patterns/01_helloworld/main}
\input{patterns/015_prolog_epilogue/main}
\input{patterns/02_stack/main}
\input{patterns/03_printf/main}
\input{patterns/04_scanf/main}
\input{patterns/05_passing_arguments/main}
\input{patterns/06_return_results/main}
\input{patterns/061_pointers/main}
\input{patterns/065_GOTO/main}
\input{patterns/07_jcc/main}
\input{patterns/08_switch/main}
\input{patterns/09_loops/main}
\input{patterns/10_strings/main}
\input{patterns/11_arith_optimizations/main}
\input{patterns/12_FPU/main}
\input{patterns/13_arrays/main}
\input{patterns/14_bitfields/main}
\EN{\input{patterns/145_LCG/main_EN}}
\RU{\input{patterns/145_LCG/main_RU}}
\input{patterns/15_structs/main}
\input{patterns/17_unions/main}
\input{patterns/18_pointers_to_functions/main}
\input{patterns/185_64bit_in_32_env/main}

\EN{\input{patterns/19_SIMD/main_EN}}
\RU{\input{patterns/19_SIMD/main_RU}}
\DE{\input{patterns/19_SIMD/main_DE}}

\EN{\input{patterns/20_x64/main_EN}}
\RU{\input{patterns/20_x64/main_RU}}

\EN{\input{patterns/205_floating_SIMD/main_EN}}
\RU{\input{patterns/205_floating_SIMD/main_RU}}
\DE{\input{patterns/205_floating_SIMD/main_DE}}

\EN{\input{patterns/ARM/main_EN}}
\RU{\input{patterns/ARM/main_RU}}
\DE{\input{patterns/ARM/main_DE}}

\input{patterns/MIPS/main}

\ifdefined\SPANISH
\chapter{Patrones de código}
\fi % SPANISH

\ifdefined\GERMAN
\chapter{Code-Muster}
\fi % GERMAN

\ifdefined\ENGLISH
\chapter{Code Patterns}
\fi % ENGLISH

\ifdefined\ITALIAN
\chapter{Forme di codice}
\fi % ITALIAN

\ifdefined\RUSSIAN
\chapter{Образцы кода}
\fi % RUSSIAN

\ifdefined\BRAZILIAN
\chapter{Padrões de códigos}
\fi % BRAZILIAN

\ifdefined\THAI
\chapter{รูปแบบของโค้ด}
\fi % THAI

\ifdefined\FRENCH
\chapter{Modèle de code}
\fi % FRENCH

\ifdefined\POLISH
\chapter{\PLph{}}
\fi % POLISH

% sections
\EN{\input{patterns/patterns_opt_dbg_EN}}
\ES{\input{patterns/patterns_opt_dbg_ES}}
\ITA{\input{patterns/patterns_opt_dbg_ITA}}
\PTBR{\input{patterns/patterns_opt_dbg_PTBR}}
\RU{\input{patterns/patterns_opt_dbg_RU}}
\THA{\input{patterns/patterns_opt_dbg_THA}}
\DE{\input{patterns/patterns_opt_dbg_DE}}
\FR{\input{patterns/patterns_opt_dbg_FR}}
\PL{\input{patterns/patterns_opt_dbg_PL}}

\RU{\section{Некоторые базовые понятия}}
\EN{\section{Some basics}}
\DE{\section{Einige Grundlagen}}
\FR{\section{Quelques bases}}
\ES{\section{\ESph{}}}
\ITA{\section{Alcune basi teoriche}}
\PTBR{\section{\PTBRph{}}}
\THA{\section{\THAph{}}}
\PL{\section{\PLph{}}}

% sections:
\EN{\input{patterns/intro_CPU_ISA_EN}}
\ES{\input{patterns/intro_CPU_ISA_ES}}
\ITA{\input{patterns/intro_CPU_ISA_ITA}}
\PTBR{\input{patterns/intro_CPU_ISA_PTBR}}
\RU{\input{patterns/intro_CPU_ISA_RU}}
\DE{\input{patterns/intro_CPU_ISA_DE}}
\FR{\input{patterns/intro_CPU_ISA_FR}}
\PL{\input{patterns/intro_CPU_ISA_PL}}

\EN{\input{patterns/numeral_EN}}
\RU{\input{patterns/numeral_RU}}
\ITA{\input{patterns/numeral_ITA}}
\DE{\input{patterns/numeral_DE}}
\FR{\input{patterns/numeral_FR}}
\PL{\input{patterns/numeral_PL}}

% chapters
\input{patterns/00_empty/main}
\input{patterns/011_ret/main}
\input{patterns/01_helloworld/main}
\input{patterns/015_prolog_epilogue/main}
\input{patterns/02_stack/main}
\input{patterns/03_printf/main}
\input{patterns/04_scanf/main}
\input{patterns/05_passing_arguments/main}
\input{patterns/06_return_results/main}
\input{patterns/061_pointers/main}
\input{patterns/065_GOTO/main}
\input{patterns/07_jcc/main}
\input{patterns/08_switch/main}
\input{patterns/09_loops/main}
\input{patterns/10_strings/main}
\input{patterns/11_arith_optimizations/main}
\input{patterns/12_FPU/main}
\input{patterns/13_arrays/main}
\input{patterns/14_bitfields/main}
\EN{\input{patterns/145_LCG/main_EN}}
\RU{\input{patterns/145_LCG/main_RU}}
\input{patterns/15_structs/main}
\input{patterns/17_unions/main}
\input{patterns/18_pointers_to_functions/main}
\input{patterns/185_64bit_in_32_env/main}

\EN{\input{patterns/19_SIMD/main_EN}}
\RU{\input{patterns/19_SIMD/main_RU}}
\DE{\input{patterns/19_SIMD/main_DE}}

\EN{\input{patterns/20_x64/main_EN}}
\RU{\input{patterns/20_x64/main_RU}}

\EN{\input{patterns/205_floating_SIMD/main_EN}}
\RU{\input{patterns/205_floating_SIMD/main_RU}}
\DE{\input{patterns/205_floating_SIMD/main_DE}}

\EN{\input{patterns/ARM/main_EN}}
\RU{\input{patterns/ARM/main_RU}}
\DE{\input{patterns/ARM/main_DE}}

\input{patterns/MIPS/main}

\ifdefined\SPANISH
\chapter{Patrones de código}
\fi % SPANISH

\ifdefined\GERMAN
\chapter{Code-Muster}
\fi % GERMAN

\ifdefined\ENGLISH
\chapter{Code Patterns}
\fi % ENGLISH

\ifdefined\ITALIAN
\chapter{Forme di codice}
\fi % ITALIAN

\ifdefined\RUSSIAN
\chapter{Образцы кода}
\fi % RUSSIAN

\ifdefined\BRAZILIAN
\chapter{Padrões de códigos}
\fi % BRAZILIAN

\ifdefined\THAI
\chapter{รูปแบบของโค้ด}
\fi % THAI

\ifdefined\FRENCH
\chapter{Modèle de code}
\fi % FRENCH

\ifdefined\POLISH
\chapter{\PLph{}}
\fi % POLISH

% sections
\EN{\input{patterns/patterns_opt_dbg_EN}}
\ES{\input{patterns/patterns_opt_dbg_ES}}
\ITA{\input{patterns/patterns_opt_dbg_ITA}}
\PTBR{\input{patterns/patterns_opt_dbg_PTBR}}
\RU{\input{patterns/patterns_opt_dbg_RU}}
\THA{\input{patterns/patterns_opt_dbg_THA}}
\DE{\input{patterns/patterns_opt_dbg_DE}}
\FR{\input{patterns/patterns_opt_dbg_FR}}
\PL{\input{patterns/patterns_opt_dbg_PL}}

\RU{\section{Некоторые базовые понятия}}
\EN{\section{Some basics}}
\DE{\section{Einige Grundlagen}}
\FR{\section{Quelques bases}}
\ES{\section{\ESph{}}}
\ITA{\section{Alcune basi teoriche}}
\PTBR{\section{\PTBRph{}}}
\THA{\section{\THAph{}}}
\PL{\section{\PLph{}}}

% sections:
\EN{\input{patterns/intro_CPU_ISA_EN}}
\ES{\input{patterns/intro_CPU_ISA_ES}}
\ITA{\input{patterns/intro_CPU_ISA_ITA}}
\PTBR{\input{patterns/intro_CPU_ISA_PTBR}}
\RU{\input{patterns/intro_CPU_ISA_RU}}
\DE{\input{patterns/intro_CPU_ISA_DE}}
\FR{\input{patterns/intro_CPU_ISA_FR}}
\PL{\input{patterns/intro_CPU_ISA_PL}}

\EN{\input{patterns/numeral_EN}}
\RU{\input{patterns/numeral_RU}}
\ITA{\input{patterns/numeral_ITA}}
\DE{\input{patterns/numeral_DE}}
\FR{\input{patterns/numeral_FR}}
\PL{\input{patterns/numeral_PL}}

% chapters
\input{patterns/00_empty/main}
\input{patterns/011_ret/main}
\input{patterns/01_helloworld/main}
\input{patterns/015_prolog_epilogue/main}
\input{patterns/02_stack/main}
\input{patterns/03_printf/main}
\input{patterns/04_scanf/main}
\input{patterns/05_passing_arguments/main}
\input{patterns/06_return_results/main}
\input{patterns/061_pointers/main}
\input{patterns/065_GOTO/main}
\input{patterns/07_jcc/main}
\input{patterns/08_switch/main}
\input{patterns/09_loops/main}
\input{patterns/10_strings/main}
\input{patterns/11_arith_optimizations/main}
\input{patterns/12_FPU/main}
\input{patterns/13_arrays/main}
\input{patterns/14_bitfields/main}
\EN{\input{patterns/145_LCG/main_EN}}
\RU{\input{patterns/145_LCG/main_RU}}
\input{patterns/15_structs/main}
\input{patterns/17_unions/main}
\input{patterns/18_pointers_to_functions/main}
\input{patterns/185_64bit_in_32_env/main}

\EN{\input{patterns/19_SIMD/main_EN}}
\RU{\input{patterns/19_SIMD/main_RU}}
\DE{\input{patterns/19_SIMD/main_DE}}

\EN{\input{patterns/20_x64/main_EN}}
\RU{\input{patterns/20_x64/main_RU}}

\EN{\input{patterns/205_floating_SIMD/main_EN}}
\RU{\input{patterns/205_floating_SIMD/main_RU}}
\DE{\input{patterns/205_floating_SIMD/main_DE}}

\EN{\input{patterns/ARM/main_EN}}
\RU{\input{patterns/ARM/main_RU}}
\DE{\input{patterns/ARM/main_DE}}

\input{patterns/MIPS/main}

\ifdefined\SPANISH
\chapter{Patrones de código}
\fi % SPANISH

\ifdefined\GERMAN
\chapter{Code-Muster}
\fi % GERMAN

\ifdefined\ENGLISH
\chapter{Code Patterns}
\fi % ENGLISH

\ifdefined\ITALIAN
\chapter{Forme di codice}
\fi % ITALIAN

\ifdefined\RUSSIAN
\chapter{Образцы кода}
\fi % RUSSIAN

\ifdefined\BRAZILIAN
\chapter{Padrões de códigos}
\fi % BRAZILIAN

\ifdefined\THAI
\chapter{รูปแบบของโค้ด}
\fi % THAI

\ifdefined\FRENCH
\chapter{Modèle de code}
\fi % FRENCH

\ifdefined\POLISH
\chapter{\PLph{}}
\fi % POLISH

% sections
\EN{\input{patterns/patterns_opt_dbg_EN}}
\ES{\input{patterns/patterns_opt_dbg_ES}}
\ITA{\input{patterns/patterns_opt_dbg_ITA}}
\PTBR{\input{patterns/patterns_opt_dbg_PTBR}}
\RU{\input{patterns/patterns_opt_dbg_RU}}
\THA{\input{patterns/patterns_opt_dbg_THA}}
\DE{\input{patterns/patterns_opt_dbg_DE}}
\FR{\input{patterns/patterns_opt_dbg_FR}}
\PL{\input{patterns/patterns_opt_dbg_PL}}

\RU{\section{Некоторые базовые понятия}}
\EN{\section{Some basics}}
\DE{\section{Einige Grundlagen}}
\FR{\section{Quelques bases}}
\ES{\section{\ESph{}}}
\ITA{\section{Alcune basi teoriche}}
\PTBR{\section{\PTBRph{}}}
\THA{\section{\THAph{}}}
\PL{\section{\PLph{}}}

% sections:
\EN{\input{patterns/intro_CPU_ISA_EN}}
\ES{\input{patterns/intro_CPU_ISA_ES}}
\ITA{\input{patterns/intro_CPU_ISA_ITA}}
\PTBR{\input{patterns/intro_CPU_ISA_PTBR}}
\RU{\input{patterns/intro_CPU_ISA_RU}}
\DE{\input{patterns/intro_CPU_ISA_DE}}
\FR{\input{patterns/intro_CPU_ISA_FR}}
\PL{\input{patterns/intro_CPU_ISA_PL}}

\EN{\input{patterns/numeral_EN}}
\RU{\input{patterns/numeral_RU}}
\ITA{\input{patterns/numeral_ITA}}
\DE{\input{patterns/numeral_DE}}
\FR{\input{patterns/numeral_FR}}
\PL{\input{patterns/numeral_PL}}

% chapters
\input{patterns/00_empty/main}
\input{patterns/011_ret/main}
\input{patterns/01_helloworld/main}
\input{patterns/015_prolog_epilogue/main}
\input{patterns/02_stack/main}
\input{patterns/03_printf/main}
\input{patterns/04_scanf/main}
\input{patterns/05_passing_arguments/main}
\input{patterns/06_return_results/main}
\input{patterns/061_pointers/main}
\input{patterns/065_GOTO/main}
\input{patterns/07_jcc/main}
\input{patterns/08_switch/main}
\input{patterns/09_loops/main}
\input{patterns/10_strings/main}
\input{patterns/11_arith_optimizations/main}
\input{patterns/12_FPU/main}
\input{patterns/13_arrays/main}
\input{patterns/14_bitfields/main}
\EN{\input{patterns/145_LCG/main_EN}}
\RU{\input{patterns/145_LCG/main_RU}}
\input{patterns/15_structs/main}
\input{patterns/17_unions/main}
\input{patterns/18_pointers_to_functions/main}
\input{patterns/185_64bit_in_32_env/main}

\EN{\input{patterns/19_SIMD/main_EN}}
\RU{\input{patterns/19_SIMD/main_RU}}
\DE{\input{patterns/19_SIMD/main_DE}}

\EN{\input{patterns/20_x64/main_EN}}
\RU{\input{patterns/20_x64/main_RU}}

\EN{\input{patterns/205_floating_SIMD/main_EN}}
\RU{\input{patterns/205_floating_SIMD/main_RU}}
\DE{\input{patterns/205_floating_SIMD/main_DE}}

\EN{\input{patterns/ARM/main_EN}}
\RU{\input{patterns/ARM/main_RU}}
\DE{\input{patterns/ARM/main_DE}}

\input{patterns/MIPS/main}

\ifdefined\SPANISH
\chapter{Patrones de código}
\fi % SPANISH

\ifdefined\GERMAN
\chapter{Code-Muster}
\fi % GERMAN

\ifdefined\ENGLISH
\chapter{Code Patterns}
\fi % ENGLISH

\ifdefined\ITALIAN
\chapter{Forme di codice}
\fi % ITALIAN

\ifdefined\RUSSIAN
\chapter{Образцы кода}
\fi % RUSSIAN

\ifdefined\BRAZILIAN
\chapter{Padrões de códigos}
\fi % BRAZILIAN

\ifdefined\THAI
\chapter{รูปแบบของโค้ด}
\fi % THAI

\ifdefined\FRENCH
\chapter{Modèle de code}
\fi % FRENCH

\ifdefined\POLISH
\chapter{\PLph{}}
\fi % POLISH

% sections
\EN{\input{patterns/patterns_opt_dbg_EN}}
\ES{\input{patterns/patterns_opt_dbg_ES}}
\ITA{\input{patterns/patterns_opt_dbg_ITA}}
\PTBR{\input{patterns/patterns_opt_dbg_PTBR}}
\RU{\input{patterns/patterns_opt_dbg_RU}}
\THA{\input{patterns/patterns_opt_dbg_THA}}
\DE{\input{patterns/patterns_opt_dbg_DE}}
\FR{\input{patterns/patterns_opt_dbg_FR}}
\PL{\input{patterns/patterns_opt_dbg_PL}}

\RU{\section{Некоторые базовые понятия}}
\EN{\section{Some basics}}
\DE{\section{Einige Grundlagen}}
\FR{\section{Quelques bases}}
\ES{\section{\ESph{}}}
\ITA{\section{Alcune basi teoriche}}
\PTBR{\section{\PTBRph{}}}
\THA{\section{\THAph{}}}
\PL{\section{\PLph{}}}

% sections:
\EN{\input{patterns/intro_CPU_ISA_EN}}
\ES{\input{patterns/intro_CPU_ISA_ES}}
\ITA{\input{patterns/intro_CPU_ISA_ITA}}
\PTBR{\input{patterns/intro_CPU_ISA_PTBR}}
\RU{\input{patterns/intro_CPU_ISA_RU}}
\DE{\input{patterns/intro_CPU_ISA_DE}}
\FR{\input{patterns/intro_CPU_ISA_FR}}
\PL{\input{patterns/intro_CPU_ISA_PL}}

\EN{\input{patterns/numeral_EN}}
\RU{\input{patterns/numeral_RU}}
\ITA{\input{patterns/numeral_ITA}}
\DE{\input{patterns/numeral_DE}}
\FR{\input{patterns/numeral_FR}}
\PL{\input{patterns/numeral_PL}}

% chapters
\input{patterns/00_empty/main}
\input{patterns/011_ret/main}
\input{patterns/01_helloworld/main}
\input{patterns/015_prolog_epilogue/main}
\input{patterns/02_stack/main}
\input{patterns/03_printf/main}
\input{patterns/04_scanf/main}
\input{patterns/05_passing_arguments/main}
\input{patterns/06_return_results/main}
\input{patterns/061_pointers/main}
\input{patterns/065_GOTO/main}
\input{patterns/07_jcc/main}
\input{patterns/08_switch/main}
\input{patterns/09_loops/main}
\input{patterns/10_strings/main}
\input{patterns/11_arith_optimizations/main}
\input{patterns/12_FPU/main}
\input{patterns/13_arrays/main}
\input{patterns/14_bitfields/main}
\EN{\input{patterns/145_LCG/main_EN}}
\RU{\input{patterns/145_LCG/main_RU}}
\input{patterns/15_structs/main}
\input{patterns/17_unions/main}
\input{patterns/18_pointers_to_functions/main}
\input{patterns/185_64bit_in_32_env/main}

\EN{\input{patterns/19_SIMD/main_EN}}
\RU{\input{patterns/19_SIMD/main_RU}}
\DE{\input{patterns/19_SIMD/main_DE}}

\EN{\input{patterns/20_x64/main_EN}}
\RU{\input{patterns/20_x64/main_RU}}

\EN{\input{patterns/205_floating_SIMD/main_EN}}
\RU{\input{patterns/205_floating_SIMD/main_RU}}
\DE{\input{patterns/205_floating_SIMD/main_DE}}

\EN{\input{patterns/ARM/main_EN}}
\RU{\input{patterns/ARM/main_RU}}
\DE{\input{patterns/ARM/main_DE}}

\input{patterns/MIPS/main}

\ifdefined\SPANISH
\chapter{Patrones de código}
\fi % SPANISH

\ifdefined\GERMAN
\chapter{Code-Muster}
\fi % GERMAN

\ifdefined\ENGLISH
\chapter{Code Patterns}
\fi % ENGLISH

\ifdefined\ITALIAN
\chapter{Forme di codice}
\fi % ITALIAN

\ifdefined\RUSSIAN
\chapter{Образцы кода}
\fi % RUSSIAN

\ifdefined\BRAZILIAN
\chapter{Padrões de códigos}
\fi % BRAZILIAN

\ifdefined\THAI
\chapter{รูปแบบของโค้ด}
\fi % THAI

\ifdefined\FRENCH
\chapter{Modèle de code}
\fi % FRENCH

\ifdefined\POLISH
\chapter{\PLph{}}
\fi % POLISH

% sections
\EN{\input{patterns/patterns_opt_dbg_EN}}
\ES{\input{patterns/patterns_opt_dbg_ES}}
\ITA{\input{patterns/patterns_opt_dbg_ITA}}
\PTBR{\input{patterns/patterns_opt_dbg_PTBR}}
\RU{\input{patterns/patterns_opt_dbg_RU}}
\THA{\input{patterns/patterns_opt_dbg_THA}}
\DE{\input{patterns/patterns_opt_dbg_DE}}
\FR{\input{patterns/patterns_opt_dbg_FR}}
\PL{\input{patterns/patterns_opt_dbg_PL}}

\RU{\section{Некоторые базовые понятия}}
\EN{\section{Some basics}}
\DE{\section{Einige Grundlagen}}
\FR{\section{Quelques bases}}
\ES{\section{\ESph{}}}
\ITA{\section{Alcune basi teoriche}}
\PTBR{\section{\PTBRph{}}}
\THA{\section{\THAph{}}}
\PL{\section{\PLph{}}}

% sections:
\EN{\input{patterns/intro_CPU_ISA_EN}}
\ES{\input{patterns/intro_CPU_ISA_ES}}
\ITA{\input{patterns/intro_CPU_ISA_ITA}}
\PTBR{\input{patterns/intro_CPU_ISA_PTBR}}
\RU{\input{patterns/intro_CPU_ISA_RU}}
\DE{\input{patterns/intro_CPU_ISA_DE}}
\FR{\input{patterns/intro_CPU_ISA_FR}}
\PL{\input{patterns/intro_CPU_ISA_PL}}

\EN{\input{patterns/numeral_EN}}
\RU{\input{patterns/numeral_RU}}
\ITA{\input{patterns/numeral_ITA}}
\DE{\input{patterns/numeral_DE}}
\FR{\input{patterns/numeral_FR}}
\PL{\input{patterns/numeral_PL}}

% chapters
\input{patterns/00_empty/main}
\input{patterns/011_ret/main}
\input{patterns/01_helloworld/main}
\input{patterns/015_prolog_epilogue/main}
\input{patterns/02_stack/main}
\input{patterns/03_printf/main}
\input{patterns/04_scanf/main}
\input{patterns/05_passing_arguments/main}
\input{patterns/06_return_results/main}
\input{patterns/061_pointers/main}
\input{patterns/065_GOTO/main}
\input{patterns/07_jcc/main}
\input{patterns/08_switch/main}
\input{patterns/09_loops/main}
\input{patterns/10_strings/main}
\input{patterns/11_arith_optimizations/main}
\input{patterns/12_FPU/main}
\input{patterns/13_arrays/main}
\input{patterns/14_bitfields/main}
\EN{\input{patterns/145_LCG/main_EN}}
\RU{\input{patterns/145_LCG/main_RU}}
\input{patterns/15_structs/main}
\input{patterns/17_unions/main}
\input{patterns/18_pointers_to_functions/main}
\input{patterns/185_64bit_in_32_env/main}

\EN{\input{patterns/19_SIMD/main_EN}}
\RU{\input{patterns/19_SIMD/main_RU}}
\DE{\input{patterns/19_SIMD/main_DE}}

\EN{\input{patterns/20_x64/main_EN}}
\RU{\input{patterns/20_x64/main_RU}}

\EN{\input{patterns/205_floating_SIMD/main_EN}}
\RU{\input{patterns/205_floating_SIMD/main_RU}}
\DE{\input{patterns/205_floating_SIMD/main_DE}}

\EN{\input{patterns/ARM/main_EN}}
\RU{\input{patterns/ARM/main_RU}}
\DE{\input{patterns/ARM/main_DE}}

\input{patterns/MIPS/main}

\ifdefined\SPANISH
\chapter{Patrones de código}
\fi % SPANISH

\ifdefined\GERMAN
\chapter{Code-Muster}
\fi % GERMAN

\ifdefined\ENGLISH
\chapter{Code Patterns}
\fi % ENGLISH

\ifdefined\ITALIAN
\chapter{Forme di codice}
\fi % ITALIAN

\ifdefined\RUSSIAN
\chapter{Образцы кода}
\fi % RUSSIAN

\ifdefined\BRAZILIAN
\chapter{Padrões de códigos}
\fi % BRAZILIAN

\ifdefined\THAI
\chapter{รูปแบบของโค้ด}
\fi % THAI

\ifdefined\FRENCH
\chapter{Modèle de code}
\fi % FRENCH

\ifdefined\POLISH
\chapter{\PLph{}}
\fi % POLISH

% sections
\EN{\input{patterns/patterns_opt_dbg_EN}}
\ES{\input{patterns/patterns_opt_dbg_ES}}
\ITA{\input{patterns/patterns_opt_dbg_ITA}}
\PTBR{\input{patterns/patterns_opt_dbg_PTBR}}
\RU{\input{patterns/patterns_opt_dbg_RU}}
\THA{\input{patterns/patterns_opt_dbg_THA}}
\DE{\input{patterns/patterns_opt_dbg_DE}}
\FR{\input{patterns/patterns_opt_dbg_FR}}
\PL{\input{patterns/patterns_opt_dbg_PL}}

\RU{\section{Некоторые базовые понятия}}
\EN{\section{Some basics}}
\DE{\section{Einige Grundlagen}}
\FR{\section{Quelques bases}}
\ES{\section{\ESph{}}}
\ITA{\section{Alcune basi teoriche}}
\PTBR{\section{\PTBRph{}}}
\THA{\section{\THAph{}}}
\PL{\section{\PLph{}}}

% sections:
\EN{\input{patterns/intro_CPU_ISA_EN}}
\ES{\input{patterns/intro_CPU_ISA_ES}}
\ITA{\input{patterns/intro_CPU_ISA_ITA}}
\PTBR{\input{patterns/intro_CPU_ISA_PTBR}}
\RU{\input{patterns/intro_CPU_ISA_RU}}
\DE{\input{patterns/intro_CPU_ISA_DE}}
\FR{\input{patterns/intro_CPU_ISA_FR}}
\PL{\input{patterns/intro_CPU_ISA_PL}}

\EN{\input{patterns/numeral_EN}}
\RU{\input{patterns/numeral_RU}}
\ITA{\input{patterns/numeral_ITA}}
\DE{\input{patterns/numeral_DE}}
\FR{\input{patterns/numeral_FR}}
\PL{\input{patterns/numeral_PL}}

% chapters
\input{patterns/00_empty/main}
\input{patterns/011_ret/main}
\input{patterns/01_helloworld/main}
\input{patterns/015_prolog_epilogue/main}
\input{patterns/02_stack/main}
\input{patterns/03_printf/main}
\input{patterns/04_scanf/main}
\input{patterns/05_passing_arguments/main}
\input{patterns/06_return_results/main}
\input{patterns/061_pointers/main}
\input{patterns/065_GOTO/main}
\input{patterns/07_jcc/main}
\input{patterns/08_switch/main}
\input{patterns/09_loops/main}
\input{patterns/10_strings/main}
\input{patterns/11_arith_optimizations/main}
\input{patterns/12_FPU/main}
\input{patterns/13_arrays/main}
\input{patterns/14_bitfields/main}
\EN{\input{patterns/145_LCG/main_EN}}
\RU{\input{patterns/145_LCG/main_RU}}
\input{patterns/15_structs/main}
\input{patterns/17_unions/main}
\input{patterns/18_pointers_to_functions/main}
\input{patterns/185_64bit_in_32_env/main}

\EN{\input{patterns/19_SIMD/main_EN}}
\RU{\input{patterns/19_SIMD/main_RU}}
\DE{\input{patterns/19_SIMD/main_DE}}

\EN{\input{patterns/20_x64/main_EN}}
\RU{\input{patterns/20_x64/main_RU}}

\EN{\input{patterns/205_floating_SIMD/main_EN}}
\RU{\input{patterns/205_floating_SIMD/main_RU}}
\DE{\input{patterns/205_floating_SIMD/main_DE}}

\EN{\input{patterns/ARM/main_EN}}
\RU{\input{patterns/ARM/main_RU}}
\DE{\input{patterns/ARM/main_DE}}

\input{patterns/MIPS/main}

\ifdefined\SPANISH
\chapter{Patrones de código}
\fi % SPANISH

\ifdefined\GERMAN
\chapter{Code-Muster}
\fi % GERMAN

\ifdefined\ENGLISH
\chapter{Code Patterns}
\fi % ENGLISH

\ifdefined\ITALIAN
\chapter{Forme di codice}
\fi % ITALIAN

\ifdefined\RUSSIAN
\chapter{Образцы кода}
\fi % RUSSIAN

\ifdefined\BRAZILIAN
\chapter{Padrões de códigos}
\fi % BRAZILIAN

\ifdefined\THAI
\chapter{รูปแบบของโค้ด}
\fi % THAI

\ifdefined\FRENCH
\chapter{Modèle de code}
\fi % FRENCH

\ifdefined\POLISH
\chapter{\PLph{}}
\fi % POLISH

% sections
\EN{\input{patterns/patterns_opt_dbg_EN}}
\ES{\input{patterns/patterns_opt_dbg_ES}}
\ITA{\input{patterns/patterns_opt_dbg_ITA}}
\PTBR{\input{patterns/patterns_opt_dbg_PTBR}}
\RU{\input{patterns/patterns_opt_dbg_RU}}
\THA{\input{patterns/patterns_opt_dbg_THA}}
\DE{\input{patterns/patterns_opt_dbg_DE}}
\FR{\input{patterns/patterns_opt_dbg_FR}}
\PL{\input{patterns/patterns_opt_dbg_PL}}

\RU{\section{Некоторые базовые понятия}}
\EN{\section{Some basics}}
\DE{\section{Einige Grundlagen}}
\FR{\section{Quelques bases}}
\ES{\section{\ESph{}}}
\ITA{\section{Alcune basi teoriche}}
\PTBR{\section{\PTBRph{}}}
\THA{\section{\THAph{}}}
\PL{\section{\PLph{}}}

% sections:
\EN{\input{patterns/intro_CPU_ISA_EN}}
\ES{\input{patterns/intro_CPU_ISA_ES}}
\ITA{\input{patterns/intro_CPU_ISA_ITA}}
\PTBR{\input{patterns/intro_CPU_ISA_PTBR}}
\RU{\input{patterns/intro_CPU_ISA_RU}}
\DE{\input{patterns/intro_CPU_ISA_DE}}
\FR{\input{patterns/intro_CPU_ISA_FR}}
\PL{\input{patterns/intro_CPU_ISA_PL}}

\EN{\input{patterns/numeral_EN}}
\RU{\input{patterns/numeral_RU}}
\ITA{\input{patterns/numeral_ITA}}
\DE{\input{patterns/numeral_DE}}
\FR{\input{patterns/numeral_FR}}
\PL{\input{patterns/numeral_PL}}

% chapters
\input{patterns/00_empty/main}
\input{patterns/011_ret/main}
\input{patterns/01_helloworld/main}
\input{patterns/015_prolog_epilogue/main}
\input{patterns/02_stack/main}
\input{patterns/03_printf/main}
\input{patterns/04_scanf/main}
\input{patterns/05_passing_arguments/main}
\input{patterns/06_return_results/main}
\input{patterns/061_pointers/main}
\input{patterns/065_GOTO/main}
\input{patterns/07_jcc/main}
\input{patterns/08_switch/main}
\input{patterns/09_loops/main}
\input{patterns/10_strings/main}
\input{patterns/11_arith_optimizations/main}
\input{patterns/12_FPU/main}
\input{patterns/13_arrays/main}
\input{patterns/14_bitfields/main}
\EN{\input{patterns/145_LCG/main_EN}}
\RU{\input{patterns/145_LCG/main_RU}}
\input{patterns/15_structs/main}
\input{patterns/17_unions/main}
\input{patterns/18_pointers_to_functions/main}
\input{patterns/185_64bit_in_32_env/main}

\EN{\input{patterns/19_SIMD/main_EN}}
\RU{\input{patterns/19_SIMD/main_RU}}
\DE{\input{patterns/19_SIMD/main_DE}}

\EN{\input{patterns/20_x64/main_EN}}
\RU{\input{patterns/20_x64/main_RU}}

\EN{\input{patterns/205_floating_SIMD/main_EN}}
\RU{\input{patterns/205_floating_SIMD/main_RU}}
\DE{\input{patterns/205_floating_SIMD/main_DE}}

\EN{\input{patterns/ARM/main_EN}}
\RU{\input{patterns/ARM/main_RU}}
\DE{\input{patterns/ARM/main_DE}}

\input{patterns/MIPS/main}

\ifdefined\SPANISH
\chapter{Patrones de código}
\fi % SPANISH

\ifdefined\GERMAN
\chapter{Code-Muster}
\fi % GERMAN

\ifdefined\ENGLISH
\chapter{Code Patterns}
\fi % ENGLISH

\ifdefined\ITALIAN
\chapter{Forme di codice}
\fi % ITALIAN

\ifdefined\RUSSIAN
\chapter{Образцы кода}
\fi % RUSSIAN

\ifdefined\BRAZILIAN
\chapter{Padrões de códigos}
\fi % BRAZILIAN

\ifdefined\THAI
\chapter{รูปแบบของโค้ด}
\fi % THAI

\ifdefined\FRENCH
\chapter{Modèle de code}
\fi % FRENCH

\ifdefined\POLISH
\chapter{\PLph{}}
\fi % POLISH

% sections
\EN{\input{patterns/patterns_opt_dbg_EN}}
\ES{\input{patterns/patterns_opt_dbg_ES}}
\ITA{\input{patterns/patterns_opt_dbg_ITA}}
\PTBR{\input{patterns/patterns_opt_dbg_PTBR}}
\RU{\input{patterns/patterns_opt_dbg_RU}}
\THA{\input{patterns/patterns_opt_dbg_THA}}
\DE{\input{patterns/patterns_opt_dbg_DE}}
\FR{\input{patterns/patterns_opt_dbg_FR}}
\PL{\input{patterns/patterns_opt_dbg_PL}}

\RU{\section{Некоторые базовые понятия}}
\EN{\section{Some basics}}
\DE{\section{Einige Grundlagen}}
\FR{\section{Quelques bases}}
\ES{\section{\ESph{}}}
\ITA{\section{Alcune basi teoriche}}
\PTBR{\section{\PTBRph{}}}
\THA{\section{\THAph{}}}
\PL{\section{\PLph{}}}

% sections:
\EN{\input{patterns/intro_CPU_ISA_EN}}
\ES{\input{patterns/intro_CPU_ISA_ES}}
\ITA{\input{patterns/intro_CPU_ISA_ITA}}
\PTBR{\input{patterns/intro_CPU_ISA_PTBR}}
\RU{\input{patterns/intro_CPU_ISA_RU}}
\DE{\input{patterns/intro_CPU_ISA_DE}}
\FR{\input{patterns/intro_CPU_ISA_FR}}
\PL{\input{patterns/intro_CPU_ISA_PL}}

\EN{\input{patterns/numeral_EN}}
\RU{\input{patterns/numeral_RU}}
\ITA{\input{patterns/numeral_ITA}}
\DE{\input{patterns/numeral_DE}}
\FR{\input{patterns/numeral_FR}}
\PL{\input{patterns/numeral_PL}}

% chapters
\input{patterns/00_empty/main}
\input{patterns/011_ret/main}
\input{patterns/01_helloworld/main}
\input{patterns/015_prolog_epilogue/main}
\input{patterns/02_stack/main}
\input{patterns/03_printf/main}
\input{patterns/04_scanf/main}
\input{patterns/05_passing_arguments/main}
\input{patterns/06_return_results/main}
\input{patterns/061_pointers/main}
\input{patterns/065_GOTO/main}
\input{patterns/07_jcc/main}
\input{patterns/08_switch/main}
\input{patterns/09_loops/main}
\input{patterns/10_strings/main}
\input{patterns/11_arith_optimizations/main}
\input{patterns/12_FPU/main}
\input{patterns/13_arrays/main}
\input{patterns/14_bitfields/main}
\EN{\input{patterns/145_LCG/main_EN}}
\RU{\input{patterns/145_LCG/main_RU}}
\input{patterns/15_structs/main}
\input{patterns/17_unions/main}
\input{patterns/18_pointers_to_functions/main}
\input{patterns/185_64bit_in_32_env/main}

\EN{\input{patterns/19_SIMD/main_EN}}
\RU{\input{patterns/19_SIMD/main_RU}}
\DE{\input{patterns/19_SIMD/main_DE}}

\EN{\input{patterns/20_x64/main_EN}}
\RU{\input{patterns/20_x64/main_RU}}

\EN{\input{patterns/205_floating_SIMD/main_EN}}
\RU{\input{patterns/205_floating_SIMD/main_RU}}
\DE{\input{patterns/205_floating_SIMD/main_DE}}

\EN{\input{patterns/ARM/main_EN}}
\RU{\input{patterns/ARM/main_RU}}
\DE{\input{patterns/ARM/main_DE}}

\input{patterns/MIPS/main}

\ifdefined\SPANISH
\chapter{Patrones de código}
\fi % SPANISH

\ifdefined\GERMAN
\chapter{Code-Muster}
\fi % GERMAN

\ifdefined\ENGLISH
\chapter{Code Patterns}
\fi % ENGLISH

\ifdefined\ITALIAN
\chapter{Forme di codice}
\fi % ITALIAN

\ifdefined\RUSSIAN
\chapter{Образцы кода}
\fi % RUSSIAN

\ifdefined\BRAZILIAN
\chapter{Padrões de códigos}
\fi % BRAZILIAN

\ifdefined\THAI
\chapter{รูปแบบของโค้ด}
\fi % THAI

\ifdefined\FRENCH
\chapter{Modèle de code}
\fi % FRENCH

\ifdefined\POLISH
\chapter{\PLph{}}
\fi % POLISH

% sections
\EN{\input{patterns/patterns_opt_dbg_EN}}
\ES{\input{patterns/patterns_opt_dbg_ES}}
\ITA{\input{patterns/patterns_opt_dbg_ITA}}
\PTBR{\input{patterns/patterns_opt_dbg_PTBR}}
\RU{\input{patterns/patterns_opt_dbg_RU}}
\THA{\input{patterns/patterns_opt_dbg_THA}}
\DE{\input{patterns/patterns_opt_dbg_DE}}
\FR{\input{patterns/patterns_opt_dbg_FR}}
\PL{\input{patterns/patterns_opt_dbg_PL}}

\RU{\section{Некоторые базовые понятия}}
\EN{\section{Some basics}}
\DE{\section{Einige Grundlagen}}
\FR{\section{Quelques bases}}
\ES{\section{\ESph{}}}
\ITA{\section{Alcune basi teoriche}}
\PTBR{\section{\PTBRph{}}}
\THA{\section{\THAph{}}}
\PL{\section{\PLph{}}}

% sections:
\EN{\input{patterns/intro_CPU_ISA_EN}}
\ES{\input{patterns/intro_CPU_ISA_ES}}
\ITA{\input{patterns/intro_CPU_ISA_ITA}}
\PTBR{\input{patterns/intro_CPU_ISA_PTBR}}
\RU{\input{patterns/intro_CPU_ISA_RU}}
\DE{\input{patterns/intro_CPU_ISA_DE}}
\FR{\input{patterns/intro_CPU_ISA_FR}}
\PL{\input{patterns/intro_CPU_ISA_PL}}

\EN{\input{patterns/numeral_EN}}
\RU{\input{patterns/numeral_RU}}
\ITA{\input{patterns/numeral_ITA}}
\DE{\input{patterns/numeral_DE}}
\FR{\input{patterns/numeral_FR}}
\PL{\input{patterns/numeral_PL}}

% chapters
\input{patterns/00_empty/main}
\input{patterns/011_ret/main}
\input{patterns/01_helloworld/main}
\input{patterns/015_prolog_epilogue/main}
\input{patterns/02_stack/main}
\input{patterns/03_printf/main}
\input{patterns/04_scanf/main}
\input{patterns/05_passing_arguments/main}
\input{patterns/06_return_results/main}
\input{patterns/061_pointers/main}
\input{patterns/065_GOTO/main}
\input{patterns/07_jcc/main}
\input{patterns/08_switch/main}
\input{patterns/09_loops/main}
\input{patterns/10_strings/main}
\input{patterns/11_arith_optimizations/main}
\input{patterns/12_FPU/main}
\input{patterns/13_arrays/main}
\input{patterns/14_bitfields/main}
\EN{\input{patterns/145_LCG/main_EN}}
\RU{\input{patterns/145_LCG/main_RU}}
\input{patterns/15_structs/main}
\input{patterns/17_unions/main}
\input{patterns/18_pointers_to_functions/main}
\input{patterns/185_64bit_in_32_env/main}

\EN{\input{patterns/19_SIMD/main_EN}}
\RU{\input{patterns/19_SIMD/main_RU}}
\DE{\input{patterns/19_SIMD/main_DE}}

\EN{\input{patterns/20_x64/main_EN}}
\RU{\input{patterns/20_x64/main_RU}}

\EN{\input{patterns/205_floating_SIMD/main_EN}}
\RU{\input{patterns/205_floating_SIMD/main_RU}}
\DE{\input{patterns/205_floating_SIMD/main_DE}}

\EN{\input{patterns/ARM/main_EN}}
\RU{\input{patterns/ARM/main_RU}}
\DE{\input{patterns/ARM/main_DE}}

\input{patterns/MIPS/main}

\ifdefined\SPANISH
\chapter{Patrones de código}
\fi % SPANISH

\ifdefined\GERMAN
\chapter{Code-Muster}
\fi % GERMAN

\ifdefined\ENGLISH
\chapter{Code Patterns}
\fi % ENGLISH

\ifdefined\ITALIAN
\chapter{Forme di codice}
\fi % ITALIAN

\ifdefined\RUSSIAN
\chapter{Образцы кода}
\fi % RUSSIAN

\ifdefined\BRAZILIAN
\chapter{Padrões de códigos}
\fi % BRAZILIAN

\ifdefined\THAI
\chapter{รูปแบบของโค้ด}
\fi % THAI

\ifdefined\FRENCH
\chapter{Modèle de code}
\fi % FRENCH

\ifdefined\POLISH
\chapter{\PLph{}}
\fi % POLISH

% sections
\EN{\input{patterns/patterns_opt_dbg_EN}}
\ES{\input{patterns/patterns_opt_dbg_ES}}
\ITA{\input{patterns/patterns_opt_dbg_ITA}}
\PTBR{\input{patterns/patterns_opt_dbg_PTBR}}
\RU{\input{patterns/patterns_opt_dbg_RU}}
\THA{\input{patterns/patterns_opt_dbg_THA}}
\DE{\input{patterns/patterns_opt_dbg_DE}}
\FR{\input{patterns/patterns_opt_dbg_FR}}
\PL{\input{patterns/patterns_opt_dbg_PL}}

\RU{\section{Некоторые базовые понятия}}
\EN{\section{Some basics}}
\DE{\section{Einige Grundlagen}}
\FR{\section{Quelques bases}}
\ES{\section{\ESph{}}}
\ITA{\section{Alcune basi teoriche}}
\PTBR{\section{\PTBRph{}}}
\THA{\section{\THAph{}}}
\PL{\section{\PLph{}}}

% sections:
\EN{\input{patterns/intro_CPU_ISA_EN}}
\ES{\input{patterns/intro_CPU_ISA_ES}}
\ITA{\input{patterns/intro_CPU_ISA_ITA}}
\PTBR{\input{patterns/intro_CPU_ISA_PTBR}}
\RU{\input{patterns/intro_CPU_ISA_RU}}
\DE{\input{patterns/intro_CPU_ISA_DE}}
\FR{\input{patterns/intro_CPU_ISA_FR}}
\PL{\input{patterns/intro_CPU_ISA_PL}}

\EN{\input{patterns/numeral_EN}}
\RU{\input{patterns/numeral_RU}}
\ITA{\input{patterns/numeral_ITA}}
\DE{\input{patterns/numeral_DE}}
\FR{\input{patterns/numeral_FR}}
\PL{\input{patterns/numeral_PL}}

% chapters
\input{patterns/00_empty/main}
\input{patterns/011_ret/main}
\input{patterns/01_helloworld/main}
\input{patterns/015_prolog_epilogue/main}
\input{patterns/02_stack/main}
\input{patterns/03_printf/main}
\input{patterns/04_scanf/main}
\input{patterns/05_passing_arguments/main}
\input{patterns/06_return_results/main}
\input{patterns/061_pointers/main}
\input{patterns/065_GOTO/main}
\input{patterns/07_jcc/main}
\input{patterns/08_switch/main}
\input{patterns/09_loops/main}
\input{patterns/10_strings/main}
\input{patterns/11_arith_optimizations/main}
\input{patterns/12_FPU/main}
\input{patterns/13_arrays/main}
\input{patterns/14_bitfields/main}
\EN{\input{patterns/145_LCG/main_EN}}
\RU{\input{patterns/145_LCG/main_RU}}
\input{patterns/15_structs/main}
\input{patterns/17_unions/main}
\input{patterns/18_pointers_to_functions/main}
\input{patterns/185_64bit_in_32_env/main}

\EN{\input{patterns/19_SIMD/main_EN}}
\RU{\input{patterns/19_SIMD/main_RU}}
\DE{\input{patterns/19_SIMD/main_DE}}

\EN{\input{patterns/20_x64/main_EN}}
\RU{\input{patterns/20_x64/main_RU}}

\EN{\input{patterns/205_floating_SIMD/main_EN}}
\RU{\input{patterns/205_floating_SIMD/main_RU}}
\DE{\input{patterns/205_floating_SIMD/main_DE}}

\EN{\input{patterns/ARM/main_EN}}
\RU{\input{patterns/ARM/main_RU}}
\DE{\input{patterns/ARM/main_DE}}

\input{patterns/MIPS/main}

\ifdefined\SPANISH
\chapter{Patrones de código}
\fi % SPANISH

\ifdefined\GERMAN
\chapter{Code-Muster}
\fi % GERMAN

\ifdefined\ENGLISH
\chapter{Code Patterns}
\fi % ENGLISH

\ifdefined\ITALIAN
\chapter{Forme di codice}
\fi % ITALIAN

\ifdefined\RUSSIAN
\chapter{Образцы кода}
\fi % RUSSIAN

\ifdefined\BRAZILIAN
\chapter{Padrões de códigos}
\fi % BRAZILIAN

\ifdefined\THAI
\chapter{รูปแบบของโค้ด}
\fi % THAI

\ifdefined\FRENCH
\chapter{Modèle de code}
\fi % FRENCH

\ifdefined\POLISH
\chapter{\PLph{}}
\fi % POLISH

% sections
\EN{\input{patterns/patterns_opt_dbg_EN}}
\ES{\input{patterns/patterns_opt_dbg_ES}}
\ITA{\input{patterns/patterns_opt_dbg_ITA}}
\PTBR{\input{patterns/patterns_opt_dbg_PTBR}}
\RU{\input{patterns/patterns_opt_dbg_RU}}
\THA{\input{patterns/patterns_opt_dbg_THA}}
\DE{\input{patterns/patterns_opt_dbg_DE}}
\FR{\input{patterns/patterns_opt_dbg_FR}}
\PL{\input{patterns/patterns_opt_dbg_PL}}

\RU{\section{Некоторые базовые понятия}}
\EN{\section{Some basics}}
\DE{\section{Einige Grundlagen}}
\FR{\section{Quelques bases}}
\ES{\section{\ESph{}}}
\ITA{\section{Alcune basi teoriche}}
\PTBR{\section{\PTBRph{}}}
\THA{\section{\THAph{}}}
\PL{\section{\PLph{}}}

% sections:
\EN{\input{patterns/intro_CPU_ISA_EN}}
\ES{\input{patterns/intro_CPU_ISA_ES}}
\ITA{\input{patterns/intro_CPU_ISA_ITA}}
\PTBR{\input{patterns/intro_CPU_ISA_PTBR}}
\RU{\input{patterns/intro_CPU_ISA_RU}}
\DE{\input{patterns/intro_CPU_ISA_DE}}
\FR{\input{patterns/intro_CPU_ISA_FR}}
\PL{\input{patterns/intro_CPU_ISA_PL}}

\EN{\input{patterns/numeral_EN}}
\RU{\input{patterns/numeral_RU}}
\ITA{\input{patterns/numeral_ITA}}
\DE{\input{patterns/numeral_DE}}
\FR{\input{patterns/numeral_FR}}
\PL{\input{patterns/numeral_PL}}

% chapters
\input{patterns/00_empty/main}
\input{patterns/011_ret/main}
\input{patterns/01_helloworld/main}
\input{patterns/015_prolog_epilogue/main}
\input{patterns/02_stack/main}
\input{patterns/03_printf/main}
\input{patterns/04_scanf/main}
\input{patterns/05_passing_arguments/main}
\input{patterns/06_return_results/main}
\input{patterns/061_pointers/main}
\input{patterns/065_GOTO/main}
\input{patterns/07_jcc/main}
\input{patterns/08_switch/main}
\input{patterns/09_loops/main}
\input{patterns/10_strings/main}
\input{patterns/11_arith_optimizations/main}
\input{patterns/12_FPU/main}
\input{patterns/13_arrays/main}
\input{patterns/14_bitfields/main}
\EN{\input{patterns/145_LCG/main_EN}}
\RU{\input{patterns/145_LCG/main_RU}}
\input{patterns/15_structs/main}
\input{patterns/17_unions/main}
\input{patterns/18_pointers_to_functions/main}
\input{patterns/185_64bit_in_32_env/main}

\EN{\input{patterns/19_SIMD/main_EN}}
\RU{\input{patterns/19_SIMD/main_RU}}
\DE{\input{patterns/19_SIMD/main_DE}}

\EN{\input{patterns/20_x64/main_EN}}
\RU{\input{patterns/20_x64/main_RU}}

\EN{\input{patterns/205_floating_SIMD/main_EN}}
\RU{\input{patterns/205_floating_SIMD/main_RU}}
\DE{\input{patterns/205_floating_SIMD/main_DE}}

\EN{\input{patterns/ARM/main_EN}}
\RU{\input{patterns/ARM/main_RU}}
\DE{\input{patterns/ARM/main_DE}}

\input{patterns/MIPS/main}

\ifdefined\SPANISH
\chapter{Patrones de código}
\fi % SPANISH

\ifdefined\GERMAN
\chapter{Code-Muster}
\fi % GERMAN

\ifdefined\ENGLISH
\chapter{Code Patterns}
\fi % ENGLISH

\ifdefined\ITALIAN
\chapter{Forme di codice}
\fi % ITALIAN

\ifdefined\RUSSIAN
\chapter{Образцы кода}
\fi % RUSSIAN

\ifdefined\BRAZILIAN
\chapter{Padrões de códigos}
\fi % BRAZILIAN

\ifdefined\THAI
\chapter{รูปแบบของโค้ด}
\fi % THAI

\ifdefined\FRENCH
\chapter{Modèle de code}
\fi % FRENCH

\ifdefined\POLISH
\chapter{\PLph{}}
\fi % POLISH

% sections
\EN{\input{patterns/patterns_opt_dbg_EN}}
\ES{\input{patterns/patterns_opt_dbg_ES}}
\ITA{\input{patterns/patterns_opt_dbg_ITA}}
\PTBR{\input{patterns/patterns_opt_dbg_PTBR}}
\RU{\input{patterns/patterns_opt_dbg_RU}}
\THA{\input{patterns/patterns_opt_dbg_THA}}
\DE{\input{patterns/patterns_opt_dbg_DE}}
\FR{\input{patterns/patterns_opt_dbg_FR}}
\PL{\input{patterns/patterns_opt_dbg_PL}}

\RU{\section{Некоторые базовые понятия}}
\EN{\section{Some basics}}
\DE{\section{Einige Grundlagen}}
\FR{\section{Quelques bases}}
\ES{\section{\ESph{}}}
\ITA{\section{Alcune basi teoriche}}
\PTBR{\section{\PTBRph{}}}
\THA{\section{\THAph{}}}
\PL{\section{\PLph{}}}

% sections:
\EN{\input{patterns/intro_CPU_ISA_EN}}
\ES{\input{patterns/intro_CPU_ISA_ES}}
\ITA{\input{patterns/intro_CPU_ISA_ITA}}
\PTBR{\input{patterns/intro_CPU_ISA_PTBR}}
\RU{\input{patterns/intro_CPU_ISA_RU}}
\DE{\input{patterns/intro_CPU_ISA_DE}}
\FR{\input{patterns/intro_CPU_ISA_FR}}
\PL{\input{patterns/intro_CPU_ISA_PL}}

\EN{\input{patterns/numeral_EN}}
\RU{\input{patterns/numeral_RU}}
\ITA{\input{patterns/numeral_ITA}}
\DE{\input{patterns/numeral_DE}}
\FR{\input{patterns/numeral_FR}}
\PL{\input{patterns/numeral_PL}}

% chapters
\input{patterns/00_empty/main}
\input{patterns/011_ret/main}
\input{patterns/01_helloworld/main}
\input{patterns/015_prolog_epilogue/main}
\input{patterns/02_stack/main}
\input{patterns/03_printf/main}
\input{patterns/04_scanf/main}
\input{patterns/05_passing_arguments/main}
\input{patterns/06_return_results/main}
\input{patterns/061_pointers/main}
\input{patterns/065_GOTO/main}
\input{patterns/07_jcc/main}
\input{patterns/08_switch/main}
\input{patterns/09_loops/main}
\input{patterns/10_strings/main}
\input{patterns/11_arith_optimizations/main}
\input{patterns/12_FPU/main}
\input{patterns/13_arrays/main}
\input{patterns/14_bitfields/main}
\EN{\input{patterns/145_LCG/main_EN}}
\RU{\input{patterns/145_LCG/main_RU}}
\input{patterns/15_structs/main}
\input{patterns/17_unions/main}
\input{patterns/18_pointers_to_functions/main}
\input{patterns/185_64bit_in_32_env/main}

\EN{\input{patterns/19_SIMD/main_EN}}
\RU{\input{patterns/19_SIMD/main_RU}}
\DE{\input{patterns/19_SIMD/main_DE}}

\EN{\input{patterns/20_x64/main_EN}}
\RU{\input{patterns/20_x64/main_RU}}

\EN{\input{patterns/205_floating_SIMD/main_EN}}
\RU{\input{patterns/205_floating_SIMD/main_RU}}
\DE{\input{patterns/205_floating_SIMD/main_DE}}

\EN{\input{patterns/ARM/main_EN}}
\RU{\input{patterns/ARM/main_RU}}
\DE{\input{patterns/ARM/main_DE}}

\input{patterns/MIPS/main}

\ifdefined\SPANISH
\chapter{Patrones de código}
\fi % SPANISH

\ifdefined\GERMAN
\chapter{Code-Muster}
\fi % GERMAN

\ifdefined\ENGLISH
\chapter{Code Patterns}
\fi % ENGLISH

\ifdefined\ITALIAN
\chapter{Forme di codice}
\fi % ITALIAN

\ifdefined\RUSSIAN
\chapter{Образцы кода}
\fi % RUSSIAN

\ifdefined\BRAZILIAN
\chapter{Padrões de códigos}
\fi % BRAZILIAN

\ifdefined\THAI
\chapter{รูปแบบของโค้ด}
\fi % THAI

\ifdefined\FRENCH
\chapter{Modèle de code}
\fi % FRENCH

\ifdefined\POLISH
\chapter{\PLph{}}
\fi % POLISH

% sections
\EN{\input{patterns/patterns_opt_dbg_EN}}
\ES{\input{patterns/patterns_opt_dbg_ES}}
\ITA{\input{patterns/patterns_opt_dbg_ITA}}
\PTBR{\input{patterns/patterns_opt_dbg_PTBR}}
\RU{\input{patterns/patterns_opt_dbg_RU}}
\THA{\input{patterns/patterns_opt_dbg_THA}}
\DE{\input{patterns/patterns_opt_dbg_DE}}
\FR{\input{patterns/patterns_opt_dbg_FR}}
\PL{\input{patterns/patterns_opt_dbg_PL}}

\RU{\section{Некоторые базовые понятия}}
\EN{\section{Some basics}}
\DE{\section{Einige Grundlagen}}
\FR{\section{Quelques bases}}
\ES{\section{\ESph{}}}
\ITA{\section{Alcune basi teoriche}}
\PTBR{\section{\PTBRph{}}}
\THA{\section{\THAph{}}}
\PL{\section{\PLph{}}}

% sections:
\EN{\input{patterns/intro_CPU_ISA_EN}}
\ES{\input{patterns/intro_CPU_ISA_ES}}
\ITA{\input{patterns/intro_CPU_ISA_ITA}}
\PTBR{\input{patterns/intro_CPU_ISA_PTBR}}
\RU{\input{patterns/intro_CPU_ISA_RU}}
\DE{\input{patterns/intro_CPU_ISA_DE}}
\FR{\input{patterns/intro_CPU_ISA_FR}}
\PL{\input{patterns/intro_CPU_ISA_PL}}

\EN{\input{patterns/numeral_EN}}
\RU{\input{patterns/numeral_RU}}
\ITA{\input{patterns/numeral_ITA}}
\DE{\input{patterns/numeral_DE}}
\FR{\input{patterns/numeral_FR}}
\PL{\input{patterns/numeral_PL}}

% chapters
\input{patterns/00_empty/main}
\input{patterns/011_ret/main}
\input{patterns/01_helloworld/main}
\input{patterns/015_prolog_epilogue/main}
\input{patterns/02_stack/main}
\input{patterns/03_printf/main}
\input{patterns/04_scanf/main}
\input{patterns/05_passing_arguments/main}
\input{patterns/06_return_results/main}
\input{patterns/061_pointers/main}
\input{patterns/065_GOTO/main}
\input{patterns/07_jcc/main}
\input{patterns/08_switch/main}
\input{patterns/09_loops/main}
\input{patterns/10_strings/main}
\input{patterns/11_arith_optimizations/main}
\input{patterns/12_FPU/main}
\input{patterns/13_arrays/main}
\input{patterns/14_bitfields/main}
\EN{\input{patterns/145_LCG/main_EN}}
\RU{\input{patterns/145_LCG/main_RU}}
\input{patterns/15_structs/main}
\input{patterns/17_unions/main}
\input{patterns/18_pointers_to_functions/main}
\input{patterns/185_64bit_in_32_env/main}

\EN{\input{patterns/19_SIMD/main_EN}}
\RU{\input{patterns/19_SIMD/main_RU}}
\DE{\input{patterns/19_SIMD/main_DE}}

\EN{\input{patterns/20_x64/main_EN}}
\RU{\input{patterns/20_x64/main_RU}}

\EN{\input{patterns/205_floating_SIMD/main_EN}}
\RU{\input{patterns/205_floating_SIMD/main_RU}}
\DE{\input{patterns/205_floating_SIMD/main_DE}}

\EN{\input{patterns/ARM/main_EN}}
\RU{\input{patterns/ARM/main_RU}}
\DE{\input{patterns/ARM/main_DE}}

\input{patterns/MIPS/main}

\ifdefined\SPANISH
\chapter{Patrones de código}
\fi % SPANISH

\ifdefined\GERMAN
\chapter{Code-Muster}
\fi % GERMAN

\ifdefined\ENGLISH
\chapter{Code Patterns}
\fi % ENGLISH

\ifdefined\ITALIAN
\chapter{Forme di codice}
\fi % ITALIAN

\ifdefined\RUSSIAN
\chapter{Образцы кода}
\fi % RUSSIAN

\ifdefined\BRAZILIAN
\chapter{Padrões de códigos}
\fi % BRAZILIAN

\ifdefined\THAI
\chapter{รูปแบบของโค้ด}
\fi % THAI

\ifdefined\FRENCH
\chapter{Modèle de code}
\fi % FRENCH

\ifdefined\POLISH
\chapter{\PLph{}}
\fi % POLISH

% sections
\EN{\input{patterns/patterns_opt_dbg_EN}}
\ES{\input{patterns/patterns_opt_dbg_ES}}
\ITA{\input{patterns/patterns_opt_dbg_ITA}}
\PTBR{\input{patterns/patterns_opt_dbg_PTBR}}
\RU{\input{patterns/patterns_opt_dbg_RU}}
\THA{\input{patterns/patterns_opt_dbg_THA}}
\DE{\input{patterns/patterns_opt_dbg_DE}}
\FR{\input{patterns/patterns_opt_dbg_FR}}
\PL{\input{patterns/patterns_opt_dbg_PL}}

\RU{\section{Некоторые базовые понятия}}
\EN{\section{Some basics}}
\DE{\section{Einige Grundlagen}}
\FR{\section{Quelques bases}}
\ES{\section{\ESph{}}}
\ITA{\section{Alcune basi teoriche}}
\PTBR{\section{\PTBRph{}}}
\THA{\section{\THAph{}}}
\PL{\section{\PLph{}}}

% sections:
\EN{\input{patterns/intro_CPU_ISA_EN}}
\ES{\input{patterns/intro_CPU_ISA_ES}}
\ITA{\input{patterns/intro_CPU_ISA_ITA}}
\PTBR{\input{patterns/intro_CPU_ISA_PTBR}}
\RU{\input{patterns/intro_CPU_ISA_RU}}
\DE{\input{patterns/intro_CPU_ISA_DE}}
\FR{\input{patterns/intro_CPU_ISA_FR}}
\PL{\input{patterns/intro_CPU_ISA_PL}}

\EN{\input{patterns/numeral_EN}}
\RU{\input{patterns/numeral_RU}}
\ITA{\input{patterns/numeral_ITA}}
\DE{\input{patterns/numeral_DE}}
\FR{\input{patterns/numeral_FR}}
\PL{\input{patterns/numeral_PL}}

% chapters
\input{patterns/00_empty/main}
\input{patterns/011_ret/main}
\input{patterns/01_helloworld/main}
\input{patterns/015_prolog_epilogue/main}
\input{patterns/02_stack/main}
\input{patterns/03_printf/main}
\input{patterns/04_scanf/main}
\input{patterns/05_passing_arguments/main}
\input{patterns/06_return_results/main}
\input{patterns/061_pointers/main}
\input{patterns/065_GOTO/main}
\input{patterns/07_jcc/main}
\input{patterns/08_switch/main}
\input{patterns/09_loops/main}
\input{patterns/10_strings/main}
\input{patterns/11_arith_optimizations/main}
\input{patterns/12_FPU/main}
\input{patterns/13_arrays/main}
\input{patterns/14_bitfields/main}
\EN{\input{patterns/145_LCG/main_EN}}
\RU{\input{patterns/145_LCG/main_RU}}
\input{patterns/15_structs/main}
\input{patterns/17_unions/main}
\input{patterns/18_pointers_to_functions/main}
\input{patterns/185_64bit_in_32_env/main}

\EN{\input{patterns/19_SIMD/main_EN}}
\RU{\input{patterns/19_SIMD/main_RU}}
\DE{\input{patterns/19_SIMD/main_DE}}

\EN{\input{patterns/20_x64/main_EN}}
\RU{\input{patterns/20_x64/main_RU}}

\EN{\input{patterns/205_floating_SIMD/main_EN}}
\RU{\input{patterns/205_floating_SIMD/main_RU}}
\DE{\input{patterns/205_floating_SIMD/main_DE}}

\EN{\input{patterns/ARM/main_EN}}
\RU{\input{patterns/ARM/main_RU}}
\DE{\input{patterns/ARM/main_DE}}

\input{patterns/MIPS/main}

\EN{\input{patterns/12_FPU/main_EN}}
\RU{\input{patterns/12_FPU/main_RU}}
\DE{\input{patterns/12_FPU/main_DE}}
\FR{\input{patterns/12_FPU/main_FR}}


\ifdefined\SPANISH
\chapter{Patrones de código}
\fi % SPANISH

\ifdefined\GERMAN
\chapter{Code-Muster}
\fi % GERMAN

\ifdefined\ENGLISH
\chapter{Code Patterns}
\fi % ENGLISH

\ifdefined\ITALIAN
\chapter{Forme di codice}
\fi % ITALIAN

\ifdefined\RUSSIAN
\chapter{Образцы кода}
\fi % RUSSIAN

\ifdefined\BRAZILIAN
\chapter{Padrões de códigos}
\fi % BRAZILIAN

\ifdefined\THAI
\chapter{รูปแบบของโค้ด}
\fi % THAI

\ifdefined\FRENCH
\chapter{Modèle de code}
\fi % FRENCH

\ifdefined\POLISH
\chapter{\PLph{}}
\fi % POLISH

% sections
\EN{\input{patterns/patterns_opt_dbg_EN}}
\ES{\input{patterns/patterns_opt_dbg_ES}}
\ITA{\input{patterns/patterns_opt_dbg_ITA}}
\PTBR{\input{patterns/patterns_opt_dbg_PTBR}}
\RU{\input{patterns/patterns_opt_dbg_RU}}
\THA{\input{patterns/patterns_opt_dbg_THA}}
\DE{\input{patterns/patterns_opt_dbg_DE}}
\FR{\input{patterns/patterns_opt_dbg_FR}}
\PL{\input{patterns/patterns_opt_dbg_PL}}

\RU{\section{Некоторые базовые понятия}}
\EN{\section{Some basics}}
\DE{\section{Einige Grundlagen}}
\FR{\section{Quelques bases}}
\ES{\section{\ESph{}}}
\ITA{\section{Alcune basi teoriche}}
\PTBR{\section{\PTBRph{}}}
\THA{\section{\THAph{}}}
\PL{\section{\PLph{}}}

% sections:
\EN{\input{patterns/intro_CPU_ISA_EN}}
\ES{\input{patterns/intro_CPU_ISA_ES}}
\ITA{\input{patterns/intro_CPU_ISA_ITA}}
\PTBR{\input{patterns/intro_CPU_ISA_PTBR}}
\RU{\input{patterns/intro_CPU_ISA_RU}}
\DE{\input{patterns/intro_CPU_ISA_DE}}
\FR{\input{patterns/intro_CPU_ISA_FR}}
\PL{\input{patterns/intro_CPU_ISA_PL}}

\EN{\input{patterns/numeral_EN}}
\RU{\input{patterns/numeral_RU}}
\ITA{\input{patterns/numeral_ITA}}
\DE{\input{patterns/numeral_DE}}
\FR{\input{patterns/numeral_FR}}
\PL{\input{patterns/numeral_PL}}

% chapters
\input{patterns/00_empty/main}
\input{patterns/011_ret/main}
\input{patterns/01_helloworld/main}
\input{patterns/015_prolog_epilogue/main}
\input{patterns/02_stack/main}
\input{patterns/03_printf/main}
\input{patterns/04_scanf/main}
\input{patterns/05_passing_arguments/main}
\input{patterns/06_return_results/main}
\input{patterns/061_pointers/main}
\input{patterns/065_GOTO/main}
\input{patterns/07_jcc/main}
\input{patterns/08_switch/main}
\input{patterns/09_loops/main}
\input{patterns/10_strings/main}
\input{patterns/11_arith_optimizations/main}
\input{patterns/12_FPU/main}
\input{patterns/13_arrays/main}
\input{patterns/14_bitfields/main}
\EN{\input{patterns/145_LCG/main_EN}}
\RU{\input{patterns/145_LCG/main_RU}}
\input{patterns/15_structs/main}
\input{patterns/17_unions/main}
\input{patterns/18_pointers_to_functions/main}
\input{patterns/185_64bit_in_32_env/main}

\EN{\input{patterns/19_SIMD/main_EN}}
\RU{\input{patterns/19_SIMD/main_RU}}
\DE{\input{patterns/19_SIMD/main_DE}}

\EN{\input{patterns/20_x64/main_EN}}
\RU{\input{patterns/20_x64/main_RU}}

\EN{\input{patterns/205_floating_SIMD/main_EN}}
\RU{\input{patterns/205_floating_SIMD/main_RU}}
\DE{\input{patterns/205_floating_SIMD/main_DE}}

\EN{\input{patterns/ARM/main_EN}}
\RU{\input{patterns/ARM/main_RU}}
\DE{\input{patterns/ARM/main_DE}}

\input{patterns/MIPS/main}

\ifdefined\SPANISH
\chapter{Patrones de código}
\fi % SPANISH

\ifdefined\GERMAN
\chapter{Code-Muster}
\fi % GERMAN

\ifdefined\ENGLISH
\chapter{Code Patterns}
\fi % ENGLISH

\ifdefined\ITALIAN
\chapter{Forme di codice}
\fi % ITALIAN

\ifdefined\RUSSIAN
\chapter{Образцы кода}
\fi % RUSSIAN

\ifdefined\BRAZILIAN
\chapter{Padrões de códigos}
\fi % BRAZILIAN

\ifdefined\THAI
\chapter{รูปแบบของโค้ด}
\fi % THAI

\ifdefined\FRENCH
\chapter{Modèle de code}
\fi % FRENCH

\ifdefined\POLISH
\chapter{\PLph{}}
\fi % POLISH

% sections
\EN{\input{patterns/patterns_opt_dbg_EN}}
\ES{\input{patterns/patterns_opt_dbg_ES}}
\ITA{\input{patterns/patterns_opt_dbg_ITA}}
\PTBR{\input{patterns/patterns_opt_dbg_PTBR}}
\RU{\input{patterns/patterns_opt_dbg_RU}}
\THA{\input{patterns/patterns_opt_dbg_THA}}
\DE{\input{patterns/patterns_opt_dbg_DE}}
\FR{\input{patterns/patterns_opt_dbg_FR}}
\PL{\input{patterns/patterns_opt_dbg_PL}}

\RU{\section{Некоторые базовые понятия}}
\EN{\section{Some basics}}
\DE{\section{Einige Grundlagen}}
\FR{\section{Quelques bases}}
\ES{\section{\ESph{}}}
\ITA{\section{Alcune basi teoriche}}
\PTBR{\section{\PTBRph{}}}
\THA{\section{\THAph{}}}
\PL{\section{\PLph{}}}

% sections:
\EN{\input{patterns/intro_CPU_ISA_EN}}
\ES{\input{patterns/intro_CPU_ISA_ES}}
\ITA{\input{patterns/intro_CPU_ISA_ITA}}
\PTBR{\input{patterns/intro_CPU_ISA_PTBR}}
\RU{\input{patterns/intro_CPU_ISA_RU}}
\DE{\input{patterns/intro_CPU_ISA_DE}}
\FR{\input{patterns/intro_CPU_ISA_FR}}
\PL{\input{patterns/intro_CPU_ISA_PL}}

\EN{\input{patterns/numeral_EN}}
\RU{\input{patterns/numeral_RU}}
\ITA{\input{patterns/numeral_ITA}}
\DE{\input{patterns/numeral_DE}}
\FR{\input{patterns/numeral_FR}}
\PL{\input{patterns/numeral_PL}}

% chapters
\input{patterns/00_empty/main}
\input{patterns/011_ret/main}
\input{patterns/01_helloworld/main}
\input{patterns/015_prolog_epilogue/main}
\input{patterns/02_stack/main}
\input{patterns/03_printf/main}
\input{patterns/04_scanf/main}
\input{patterns/05_passing_arguments/main}
\input{patterns/06_return_results/main}
\input{patterns/061_pointers/main}
\input{patterns/065_GOTO/main}
\input{patterns/07_jcc/main}
\input{patterns/08_switch/main}
\input{patterns/09_loops/main}
\input{patterns/10_strings/main}
\input{patterns/11_arith_optimizations/main}
\input{patterns/12_FPU/main}
\input{patterns/13_arrays/main}
\input{patterns/14_bitfields/main}
\EN{\input{patterns/145_LCG/main_EN}}
\RU{\input{patterns/145_LCG/main_RU}}
\input{patterns/15_structs/main}
\input{patterns/17_unions/main}
\input{patterns/18_pointers_to_functions/main}
\input{patterns/185_64bit_in_32_env/main}

\EN{\input{patterns/19_SIMD/main_EN}}
\RU{\input{patterns/19_SIMD/main_RU}}
\DE{\input{patterns/19_SIMD/main_DE}}

\EN{\input{patterns/20_x64/main_EN}}
\RU{\input{patterns/20_x64/main_RU}}

\EN{\input{patterns/205_floating_SIMD/main_EN}}
\RU{\input{patterns/205_floating_SIMD/main_RU}}
\DE{\input{patterns/205_floating_SIMD/main_DE}}

\EN{\input{patterns/ARM/main_EN}}
\RU{\input{patterns/ARM/main_RU}}
\DE{\input{patterns/ARM/main_DE}}

\input{patterns/MIPS/main}

\EN{\section{Returning Values}
\label{ret_val_func}

Another simple function is the one that simply returns a constant value:

\lstinputlisting[caption=\EN{\CCpp Code},style=customc]{patterns/011_ret/1.c}

Let's compile it.

\subsection{x86}

Here's what both the GCC and MSVC compilers produce (with optimization) on the x86 platform:

\lstinputlisting[caption=\Optimizing GCC/MSVC (\assemblyOutput),style=customasmx86]{patterns/011_ret/1.s}

\myindex{x86!\Instructions!RET}
There are just two instructions: the first places the value 123 into the \EAX register,
which is used by convention for storing the return
value, and the second one is \RET, which returns execution to the \gls{caller}.

The caller will take the result from the \EAX register.

\subsection{ARM}

There are a few differences on the ARM platform:

\lstinputlisting[caption=\OptimizingKeilVI (\ARMMode) ASM Output,style=customasmARM]{patterns/011_ret/1_Keil_ARM_O3.s}

ARM uses the register \Reg{0} for returning the results of functions, so 123 is copied into \Reg{0}.

\myindex{ARM!\Instructions!MOV}
\myindex{x86!\Instructions!MOV}
It is worth noting that \MOV is a misleading name for the instruction in both the x86 and ARM \ac{ISA}s.

The data is not in fact \IT{moved}, but \IT{copied}.

\subsection{MIPS}

\label{MIPS_leaf_function_ex1}

The GCC assembly output below lists registers by number:

\lstinputlisting[caption=\Optimizing GCC 4.4.5 (\assemblyOutput),style=customasmMIPS]{patterns/011_ret/MIPS.s}

\dots while \IDA does it by their pseudo names:

\lstinputlisting[caption=\Optimizing GCC 4.4.5 (IDA),style=customasmMIPS]{patterns/011_ret/MIPS_IDA.lst}

The \$2 (or \$V0) register is used to store the function's return value.
\myindex{MIPS!\Pseudoinstructions!LI}
\INS{LI} stands for ``Load Immediate'' and is the MIPS equivalent to \MOV.

\myindex{MIPS!\Instructions!J}
The other instruction is the jump instruction (J or JR) which returns the execution flow to the \gls{caller}.

\myindex{MIPS!Branch delay slot}
You might be wondering why the positions of the load instruction (LI) and the jump instruction (J or JR) are swapped. This is due to a \ac{RISC} feature called ``branch delay slot''.

The reason this happens is a quirk in the architecture of some RISC \ac{ISA}s and isn't important for our
purposes---we must simply keep in mind that in MIPS, the instruction following a jump or branch instruction
is executed \IT{before} the jump/branch instruction itself.

As a consequence, branch instructions always swap places with the instruction executed immediately beforehand.


In practice, functions which merely return 1 (\IT{true}) or 0 (\IT{false}) are very frequent.

The smallest ever of the standard UNIX utilities, \IT{/bin/true} and \IT{/bin/false} return 0 and 1 respectively, as an exit code.
(Zero as an exit code usually means success, non-zero means error.)
}
\RU{\subsubsection{std::string}
\myindex{\Cpp!STL!std::string}
\label{std_string}

\myparagraph{Как устроена структура}

Многие строковые библиотеки \InSqBrackets{\CNotes 2.2} обеспечивают структуру содержащую ссылку 
на буфер собственно со строкой, переменная всегда содержащую длину строки 
(что очень удобно для массы функций \InSqBrackets{\CNotes 2.2.1}) и переменную содержащую текущий размер буфера.

Строка в буфере обыкновенно оканчивается нулем: это для того чтобы указатель на буфер можно было
передавать в функции требующие на вход обычную сишную \ac{ASCIIZ}-строку.

Стандарт \Cpp не описывает, как именно нужно реализовывать std::string,
но, как правило, они реализованы как описано выше, с небольшими дополнениями.

Строки в \Cpp это не класс (как, например, QString в Qt), а темплейт (basic\_string), 
это сделано для того чтобы поддерживать 
строки содержащие разного типа символы: как минимум \Tchar и \IT{wchar\_t}.

Так что, std::string это класс с базовым типом \Tchar.

А std::wstring это класс с базовым типом \IT{wchar\_t}.

\mysubparagraph{MSVC}

В реализации MSVC, вместо ссылки на буфер может содержаться сам буфер (если строка короче 16-и символов).

Это означает, что каждая короткая строка будет занимать в памяти по крайней мере $16 + 4 + 4 = 24$ 
байт для 32-битной среды либо $16 + 8 + 8 = 32$ 
байта в 64-битной, а если строка длиннее 16-и символов, то прибавьте еще длину самой строки.

\lstinputlisting[caption=пример для MSVC,style=customc]{\CURPATH/STL/string/MSVC_RU.cpp}

Собственно, из этого исходника почти всё ясно.

Несколько замечаний:

Если строка короче 16-и символов, 
то отдельный буфер для строки в \glslink{heap}{куче} выделяться не будет.

Это удобно потому что на практике, основная часть строк действительно короткие.
Вероятно, разработчики в Microsoft выбрали размер в 16 символов как разумный баланс.

Теперь очень важный момент в конце функции main(): мы не пользуемся методом c\_str(), тем не менее,
если это скомпилировать и запустить, то обе строки появятся в консоли!

Работает это вот почему.

В первом случае строка короче 16-и символов и в начале объекта std::string (его можно рассматривать
просто как структуру) расположен буфер с этой строкой.
\printf трактует указатель как указатель на массив символов оканчивающийся нулем и поэтому всё работает.

Вывод второй строки (длиннее 16-и символов) даже еще опаснее: это вообще типичная программистская ошибка 
(или опечатка), забыть дописать c\_str().
Это работает потому что в это время в начале структуры расположен указатель на буфер.
Это может надолго остаться незамеченным: до тех пока там не появится строка 
короче 16-и символов, тогда процесс упадет.

\mysubparagraph{GCC}

В реализации GCC в структуре есть еще одна переменная --- reference count.

Интересно, что указатель на экземпляр класса std::string в GCC указывает не на начало самой структуры, 
а на указатель на буфера.
В libstdc++-v3\textbackslash{}include\textbackslash{}bits\textbackslash{}basic\_string.h 
мы можем прочитать что это сделано для удобства отладки:

\begin{lstlisting}
   *  The reason you want _M_data pointing to the character %array and
   *  not the _Rep is so that the debugger can see the string
   *  contents. (Probably we should add a non-inline member to get
   *  the _Rep for the debugger to use, so users can check the actual
   *  string length.)
\end{lstlisting}

\href{http://go.yurichev.com/17085}{исходный код basic\_string.h}

В нашем примере мы учитываем это:

\lstinputlisting[caption=пример для GCC,style=customc]{\CURPATH/STL/string/GCC_RU.cpp}

Нужны еще небольшие хаки чтобы сымитировать типичную ошибку, которую мы уже видели выше, из-за
более ужесточенной проверки типов в GCC, тем не менее, printf() работает и здесь без c\_str().

\myparagraph{Чуть более сложный пример}

\lstinputlisting[style=customc]{\CURPATH/STL/string/3.cpp}

\lstinputlisting[caption=MSVC 2012,style=customasmx86]{\CURPATH/STL/string/3_MSVC_RU.asm}

Собственно, компилятор не конструирует строки статически: да в общем-то и как
это возможно, если буфер с ней нужно хранить в \glslink{heap}{куче}?

Вместо этого в сегменте данных хранятся обычные \ac{ASCIIZ}-строки, а позже, во время выполнения, 
при помощи метода \q{assign}, конструируются строки s1 и s2
.
При помощи \TT{operator+}, создается строка s3.

Обратите внимание на то что вызов метода c\_str() отсутствует,
потому что его код достаточно короткий и компилятор вставил его прямо здесь:
если строка короче 16-и байт, то в регистре EAX остается указатель на буфер,
а если длиннее, то из этого же места достается адрес на буфер расположенный в \glslink{heap}{куче}.

Далее следуют вызовы трех деструкторов, причем, они вызываются только если строка длиннее 16-и байт:
тогда нужно освободить буфера в \glslink{heap}{куче}.
В противном случае, так как все три объекта std::string хранятся в стеке,
они освобождаются автоматически после выхода из функции.

Следовательно, работа с короткими строками более быстрая из-за м\'{е}ньшего обращения к \glslink{heap}{куче}.

Код на GCC даже проще (из-за того, что в GCC, как мы уже видели, не реализована возможность хранить короткую
строку прямо в структуре):

% TODO1 comment each function meaning
\lstinputlisting[caption=GCC 4.8.1,style=customasmx86]{\CURPATH/STL/string/3_GCC_RU.s}

Можно заметить, что в деструкторы передается не указатель на объект,
а указатель на место за 12 байт (или 3 слова) перед ним, то есть, на настоящее начало структуры.

\myparagraph{std::string как глобальная переменная}
\label{sec:std_string_as_global_variable}

Опытные программисты на \Cpp знают, что глобальные переменные \ac{STL}-типов вполне можно объявлять.

Да, действительно:

\lstinputlisting[style=customc]{\CURPATH/STL/string/5.cpp}

Но как и где будет вызываться конструктор \TT{std::string}?

На самом деле, эта переменная будет инициализирована даже перед началом \main.

\lstinputlisting[caption=MSVC 2012: здесь конструируется глобальная переменная{,} а также регистрируется её деструктор,style=customasmx86]{\CURPATH/STL/string/5_MSVC_p2.asm}

\lstinputlisting[caption=MSVC 2012: здесь глобальная переменная используется в \main,style=customasmx86]{\CURPATH/STL/string/5_MSVC_p1.asm}

\lstinputlisting[caption=MSVC 2012: эта функция-деструктор вызывается перед выходом,style=customasmx86]{\CURPATH/STL/string/5_MSVC_p3.asm}

\myindex{\CStandardLibrary!atexit()}
В реальности, из \ac{CRT}, еще до вызова main(), вызывается специальная функция,
в которой перечислены все конструкторы подобных переменных.
Более того: при помощи atexit() регистрируется функция, которая будет вызвана в конце работы программы:
в этой функции компилятор собирает вызовы деструкторов всех подобных глобальных переменных.

GCC работает похожим образом:

\lstinputlisting[caption=GCC 4.8.1,style=customasmx86]{\CURPATH/STL/string/5_GCC.s}

Но он не выделяет отдельной функции в которой будут собраны деструкторы: 
каждый деструктор передается в atexit() по одному.

% TODO а если глобальная STL-переменная в другом модуле? надо проверить.

}
\ifdefined\SPANISH
\chapter{Patrones de código}
\fi % SPANISH

\ifdefined\GERMAN
\chapter{Code-Muster}
\fi % GERMAN

\ifdefined\ENGLISH
\chapter{Code Patterns}
\fi % ENGLISH

\ifdefined\ITALIAN
\chapter{Forme di codice}
\fi % ITALIAN

\ifdefined\RUSSIAN
\chapter{Образцы кода}
\fi % RUSSIAN

\ifdefined\BRAZILIAN
\chapter{Padrões de códigos}
\fi % BRAZILIAN

\ifdefined\THAI
\chapter{รูปแบบของโค้ด}
\fi % THAI

\ifdefined\FRENCH
\chapter{Modèle de code}
\fi % FRENCH

\ifdefined\POLISH
\chapter{\PLph{}}
\fi % POLISH

% sections
\EN{\input{patterns/patterns_opt_dbg_EN}}
\ES{\input{patterns/patterns_opt_dbg_ES}}
\ITA{\input{patterns/patterns_opt_dbg_ITA}}
\PTBR{\input{patterns/patterns_opt_dbg_PTBR}}
\RU{\input{patterns/patterns_opt_dbg_RU}}
\THA{\input{patterns/patterns_opt_dbg_THA}}
\DE{\input{patterns/patterns_opt_dbg_DE}}
\FR{\input{patterns/patterns_opt_dbg_FR}}
\PL{\input{patterns/patterns_opt_dbg_PL}}

\RU{\section{Некоторые базовые понятия}}
\EN{\section{Some basics}}
\DE{\section{Einige Grundlagen}}
\FR{\section{Quelques bases}}
\ES{\section{\ESph{}}}
\ITA{\section{Alcune basi teoriche}}
\PTBR{\section{\PTBRph{}}}
\THA{\section{\THAph{}}}
\PL{\section{\PLph{}}}

% sections:
\EN{\input{patterns/intro_CPU_ISA_EN}}
\ES{\input{patterns/intro_CPU_ISA_ES}}
\ITA{\input{patterns/intro_CPU_ISA_ITA}}
\PTBR{\input{patterns/intro_CPU_ISA_PTBR}}
\RU{\input{patterns/intro_CPU_ISA_RU}}
\DE{\input{patterns/intro_CPU_ISA_DE}}
\FR{\input{patterns/intro_CPU_ISA_FR}}
\PL{\input{patterns/intro_CPU_ISA_PL}}

\EN{\input{patterns/numeral_EN}}
\RU{\input{patterns/numeral_RU}}
\ITA{\input{patterns/numeral_ITA}}
\DE{\input{patterns/numeral_DE}}
\FR{\input{patterns/numeral_FR}}
\PL{\input{patterns/numeral_PL}}

% chapters
\input{patterns/00_empty/main}
\input{patterns/011_ret/main}
\input{patterns/01_helloworld/main}
\input{patterns/015_prolog_epilogue/main}
\input{patterns/02_stack/main}
\input{patterns/03_printf/main}
\input{patterns/04_scanf/main}
\input{patterns/05_passing_arguments/main}
\input{patterns/06_return_results/main}
\input{patterns/061_pointers/main}
\input{patterns/065_GOTO/main}
\input{patterns/07_jcc/main}
\input{patterns/08_switch/main}
\input{patterns/09_loops/main}
\input{patterns/10_strings/main}
\input{patterns/11_arith_optimizations/main}
\input{patterns/12_FPU/main}
\input{patterns/13_arrays/main}
\input{patterns/14_bitfields/main}
\EN{\input{patterns/145_LCG/main_EN}}
\RU{\input{patterns/145_LCG/main_RU}}
\input{patterns/15_structs/main}
\input{patterns/17_unions/main}
\input{patterns/18_pointers_to_functions/main}
\input{patterns/185_64bit_in_32_env/main}

\EN{\input{patterns/19_SIMD/main_EN}}
\RU{\input{patterns/19_SIMD/main_RU}}
\DE{\input{patterns/19_SIMD/main_DE}}

\EN{\input{patterns/20_x64/main_EN}}
\RU{\input{patterns/20_x64/main_RU}}

\EN{\input{patterns/205_floating_SIMD/main_EN}}
\RU{\input{patterns/205_floating_SIMD/main_RU}}
\DE{\input{patterns/205_floating_SIMD/main_DE}}

\EN{\input{patterns/ARM/main_EN}}
\RU{\input{patterns/ARM/main_RU}}
\DE{\input{patterns/ARM/main_DE}}

\input{patterns/MIPS/main}

\ifdefined\SPANISH
\chapter{Patrones de código}
\fi % SPANISH

\ifdefined\GERMAN
\chapter{Code-Muster}
\fi % GERMAN

\ifdefined\ENGLISH
\chapter{Code Patterns}
\fi % ENGLISH

\ifdefined\ITALIAN
\chapter{Forme di codice}
\fi % ITALIAN

\ifdefined\RUSSIAN
\chapter{Образцы кода}
\fi % RUSSIAN

\ifdefined\BRAZILIAN
\chapter{Padrões de códigos}
\fi % BRAZILIAN

\ifdefined\THAI
\chapter{รูปแบบของโค้ด}
\fi % THAI

\ifdefined\FRENCH
\chapter{Modèle de code}
\fi % FRENCH

\ifdefined\POLISH
\chapter{\PLph{}}
\fi % POLISH

% sections
\EN{\input{patterns/patterns_opt_dbg_EN}}
\ES{\input{patterns/patterns_opt_dbg_ES}}
\ITA{\input{patterns/patterns_opt_dbg_ITA}}
\PTBR{\input{patterns/patterns_opt_dbg_PTBR}}
\RU{\input{patterns/patterns_opt_dbg_RU}}
\THA{\input{patterns/patterns_opt_dbg_THA}}
\DE{\input{patterns/patterns_opt_dbg_DE}}
\FR{\input{patterns/patterns_opt_dbg_FR}}
\PL{\input{patterns/patterns_opt_dbg_PL}}

\RU{\section{Некоторые базовые понятия}}
\EN{\section{Some basics}}
\DE{\section{Einige Grundlagen}}
\FR{\section{Quelques bases}}
\ES{\section{\ESph{}}}
\ITA{\section{Alcune basi teoriche}}
\PTBR{\section{\PTBRph{}}}
\THA{\section{\THAph{}}}
\PL{\section{\PLph{}}}

% sections:
\EN{\input{patterns/intro_CPU_ISA_EN}}
\ES{\input{patterns/intro_CPU_ISA_ES}}
\ITA{\input{patterns/intro_CPU_ISA_ITA}}
\PTBR{\input{patterns/intro_CPU_ISA_PTBR}}
\RU{\input{patterns/intro_CPU_ISA_RU}}
\DE{\input{patterns/intro_CPU_ISA_DE}}
\FR{\input{patterns/intro_CPU_ISA_FR}}
\PL{\input{patterns/intro_CPU_ISA_PL}}

\EN{\input{patterns/numeral_EN}}
\RU{\input{patterns/numeral_RU}}
\ITA{\input{patterns/numeral_ITA}}
\DE{\input{patterns/numeral_DE}}
\FR{\input{patterns/numeral_FR}}
\PL{\input{patterns/numeral_PL}}

% chapters
\input{patterns/00_empty/main}
\input{patterns/011_ret/main}
\input{patterns/01_helloworld/main}
\input{patterns/015_prolog_epilogue/main}
\input{patterns/02_stack/main}
\input{patterns/03_printf/main}
\input{patterns/04_scanf/main}
\input{patterns/05_passing_arguments/main}
\input{patterns/06_return_results/main}
\input{patterns/061_pointers/main}
\input{patterns/065_GOTO/main}
\input{patterns/07_jcc/main}
\input{patterns/08_switch/main}
\input{patterns/09_loops/main}
\input{patterns/10_strings/main}
\input{patterns/11_arith_optimizations/main}
\input{patterns/12_FPU/main}
\input{patterns/13_arrays/main}
\input{patterns/14_bitfields/main}
\EN{\input{patterns/145_LCG/main_EN}}
\RU{\input{patterns/145_LCG/main_RU}}
\input{patterns/15_structs/main}
\input{patterns/17_unions/main}
\input{patterns/18_pointers_to_functions/main}
\input{patterns/185_64bit_in_32_env/main}

\EN{\input{patterns/19_SIMD/main_EN}}
\RU{\input{patterns/19_SIMD/main_RU}}
\DE{\input{patterns/19_SIMD/main_DE}}

\EN{\input{patterns/20_x64/main_EN}}
\RU{\input{patterns/20_x64/main_RU}}

\EN{\input{patterns/205_floating_SIMD/main_EN}}
\RU{\input{patterns/205_floating_SIMD/main_RU}}
\DE{\input{patterns/205_floating_SIMD/main_DE}}

\EN{\input{patterns/ARM/main_EN}}
\RU{\input{patterns/ARM/main_RU}}
\DE{\input{patterns/ARM/main_DE}}

\input{patterns/MIPS/main}

\ifdefined\SPANISH
\chapter{Patrones de código}
\fi % SPANISH

\ifdefined\GERMAN
\chapter{Code-Muster}
\fi % GERMAN

\ifdefined\ENGLISH
\chapter{Code Patterns}
\fi % ENGLISH

\ifdefined\ITALIAN
\chapter{Forme di codice}
\fi % ITALIAN

\ifdefined\RUSSIAN
\chapter{Образцы кода}
\fi % RUSSIAN

\ifdefined\BRAZILIAN
\chapter{Padrões de códigos}
\fi % BRAZILIAN

\ifdefined\THAI
\chapter{รูปแบบของโค้ด}
\fi % THAI

\ifdefined\FRENCH
\chapter{Modèle de code}
\fi % FRENCH

\ifdefined\POLISH
\chapter{\PLph{}}
\fi % POLISH

% sections
\EN{\input{patterns/patterns_opt_dbg_EN}}
\ES{\input{patterns/patterns_opt_dbg_ES}}
\ITA{\input{patterns/patterns_opt_dbg_ITA}}
\PTBR{\input{patterns/patterns_opt_dbg_PTBR}}
\RU{\input{patterns/patterns_opt_dbg_RU}}
\THA{\input{patterns/patterns_opt_dbg_THA}}
\DE{\input{patterns/patterns_opt_dbg_DE}}
\FR{\input{patterns/patterns_opt_dbg_FR}}
\PL{\input{patterns/patterns_opt_dbg_PL}}

\RU{\section{Некоторые базовые понятия}}
\EN{\section{Some basics}}
\DE{\section{Einige Grundlagen}}
\FR{\section{Quelques bases}}
\ES{\section{\ESph{}}}
\ITA{\section{Alcune basi teoriche}}
\PTBR{\section{\PTBRph{}}}
\THA{\section{\THAph{}}}
\PL{\section{\PLph{}}}

% sections:
\EN{\input{patterns/intro_CPU_ISA_EN}}
\ES{\input{patterns/intro_CPU_ISA_ES}}
\ITA{\input{patterns/intro_CPU_ISA_ITA}}
\PTBR{\input{patterns/intro_CPU_ISA_PTBR}}
\RU{\input{patterns/intro_CPU_ISA_RU}}
\DE{\input{patterns/intro_CPU_ISA_DE}}
\FR{\input{patterns/intro_CPU_ISA_FR}}
\PL{\input{patterns/intro_CPU_ISA_PL}}

\EN{\input{patterns/numeral_EN}}
\RU{\input{patterns/numeral_RU}}
\ITA{\input{patterns/numeral_ITA}}
\DE{\input{patterns/numeral_DE}}
\FR{\input{patterns/numeral_FR}}
\PL{\input{patterns/numeral_PL}}

% chapters
\input{patterns/00_empty/main}
\input{patterns/011_ret/main}
\input{patterns/01_helloworld/main}
\input{patterns/015_prolog_epilogue/main}
\input{patterns/02_stack/main}
\input{patterns/03_printf/main}
\input{patterns/04_scanf/main}
\input{patterns/05_passing_arguments/main}
\input{patterns/06_return_results/main}
\input{patterns/061_pointers/main}
\input{patterns/065_GOTO/main}
\input{patterns/07_jcc/main}
\input{patterns/08_switch/main}
\input{patterns/09_loops/main}
\input{patterns/10_strings/main}
\input{patterns/11_arith_optimizations/main}
\input{patterns/12_FPU/main}
\input{patterns/13_arrays/main}
\input{patterns/14_bitfields/main}
\EN{\input{patterns/145_LCG/main_EN}}
\RU{\input{patterns/145_LCG/main_RU}}
\input{patterns/15_structs/main}
\input{patterns/17_unions/main}
\input{patterns/18_pointers_to_functions/main}
\input{patterns/185_64bit_in_32_env/main}

\EN{\input{patterns/19_SIMD/main_EN}}
\RU{\input{patterns/19_SIMD/main_RU}}
\DE{\input{patterns/19_SIMD/main_DE}}

\EN{\input{patterns/20_x64/main_EN}}
\RU{\input{patterns/20_x64/main_RU}}

\EN{\input{patterns/205_floating_SIMD/main_EN}}
\RU{\input{patterns/205_floating_SIMD/main_RU}}
\DE{\input{patterns/205_floating_SIMD/main_DE}}

\EN{\input{patterns/ARM/main_EN}}
\RU{\input{patterns/ARM/main_RU}}
\DE{\input{patterns/ARM/main_DE}}

\input{patterns/MIPS/main}

\ifdefined\SPANISH
\chapter{Patrones de código}
\fi % SPANISH

\ifdefined\GERMAN
\chapter{Code-Muster}
\fi % GERMAN

\ifdefined\ENGLISH
\chapter{Code Patterns}
\fi % ENGLISH

\ifdefined\ITALIAN
\chapter{Forme di codice}
\fi % ITALIAN

\ifdefined\RUSSIAN
\chapter{Образцы кода}
\fi % RUSSIAN

\ifdefined\BRAZILIAN
\chapter{Padrões de códigos}
\fi % BRAZILIAN

\ifdefined\THAI
\chapter{รูปแบบของโค้ด}
\fi % THAI

\ifdefined\FRENCH
\chapter{Modèle de code}
\fi % FRENCH

\ifdefined\POLISH
\chapter{\PLph{}}
\fi % POLISH

% sections
\EN{\input{patterns/patterns_opt_dbg_EN}}
\ES{\input{patterns/patterns_opt_dbg_ES}}
\ITA{\input{patterns/patterns_opt_dbg_ITA}}
\PTBR{\input{patterns/patterns_opt_dbg_PTBR}}
\RU{\input{patterns/patterns_opt_dbg_RU}}
\THA{\input{patterns/patterns_opt_dbg_THA}}
\DE{\input{patterns/patterns_opt_dbg_DE}}
\FR{\input{patterns/patterns_opt_dbg_FR}}
\PL{\input{patterns/patterns_opt_dbg_PL}}

\RU{\section{Некоторые базовые понятия}}
\EN{\section{Some basics}}
\DE{\section{Einige Grundlagen}}
\FR{\section{Quelques bases}}
\ES{\section{\ESph{}}}
\ITA{\section{Alcune basi teoriche}}
\PTBR{\section{\PTBRph{}}}
\THA{\section{\THAph{}}}
\PL{\section{\PLph{}}}

% sections:
\EN{\input{patterns/intro_CPU_ISA_EN}}
\ES{\input{patterns/intro_CPU_ISA_ES}}
\ITA{\input{patterns/intro_CPU_ISA_ITA}}
\PTBR{\input{patterns/intro_CPU_ISA_PTBR}}
\RU{\input{patterns/intro_CPU_ISA_RU}}
\DE{\input{patterns/intro_CPU_ISA_DE}}
\FR{\input{patterns/intro_CPU_ISA_FR}}
\PL{\input{patterns/intro_CPU_ISA_PL}}

\EN{\input{patterns/numeral_EN}}
\RU{\input{patterns/numeral_RU}}
\ITA{\input{patterns/numeral_ITA}}
\DE{\input{patterns/numeral_DE}}
\FR{\input{patterns/numeral_FR}}
\PL{\input{patterns/numeral_PL}}

% chapters
\input{patterns/00_empty/main}
\input{patterns/011_ret/main}
\input{patterns/01_helloworld/main}
\input{patterns/015_prolog_epilogue/main}
\input{patterns/02_stack/main}
\input{patterns/03_printf/main}
\input{patterns/04_scanf/main}
\input{patterns/05_passing_arguments/main}
\input{patterns/06_return_results/main}
\input{patterns/061_pointers/main}
\input{patterns/065_GOTO/main}
\input{patterns/07_jcc/main}
\input{patterns/08_switch/main}
\input{patterns/09_loops/main}
\input{patterns/10_strings/main}
\input{patterns/11_arith_optimizations/main}
\input{patterns/12_FPU/main}
\input{patterns/13_arrays/main}
\input{patterns/14_bitfields/main}
\EN{\input{patterns/145_LCG/main_EN}}
\RU{\input{patterns/145_LCG/main_RU}}
\input{patterns/15_structs/main}
\input{patterns/17_unions/main}
\input{patterns/18_pointers_to_functions/main}
\input{patterns/185_64bit_in_32_env/main}

\EN{\input{patterns/19_SIMD/main_EN}}
\RU{\input{patterns/19_SIMD/main_RU}}
\DE{\input{patterns/19_SIMD/main_DE}}

\EN{\input{patterns/20_x64/main_EN}}
\RU{\input{patterns/20_x64/main_RU}}

\EN{\input{patterns/205_floating_SIMD/main_EN}}
\RU{\input{patterns/205_floating_SIMD/main_RU}}
\DE{\input{patterns/205_floating_SIMD/main_DE}}

\EN{\input{patterns/ARM/main_EN}}
\RU{\input{patterns/ARM/main_RU}}
\DE{\input{patterns/ARM/main_DE}}

\input{patterns/MIPS/main}


\EN{\section{Returning Values}
\label{ret_val_func}

Another simple function is the one that simply returns a constant value:

\lstinputlisting[caption=\EN{\CCpp Code},style=customc]{patterns/011_ret/1.c}

Let's compile it.

\subsection{x86}

Here's what both the GCC and MSVC compilers produce (with optimization) on the x86 platform:

\lstinputlisting[caption=\Optimizing GCC/MSVC (\assemblyOutput),style=customasmx86]{patterns/011_ret/1.s}

\myindex{x86!\Instructions!RET}
There are just two instructions: the first places the value 123 into the \EAX register,
which is used by convention for storing the return
value, and the second one is \RET, which returns execution to the \gls{caller}.

The caller will take the result from the \EAX register.

\subsection{ARM}

There are a few differences on the ARM platform:

\lstinputlisting[caption=\OptimizingKeilVI (\ARMMode) ASM Output,style=customasmARM]{patterns/011_ret/1_Keil_ARM_O3.s}

ARM uses the register \Reg{0} for returning the results of functions, so 123 is copied into \Reg{0}.

\myindex{ARM!\Instructions!MOV}
\myindex{x86!\Instructions!MOV}
It is worth noting that \MOV is a misleading name for the instruction in both the x86 and ARM \ac{ISA}s.

The data is not in fact \IT{moved}, but \IT{copied}.

\subsection{MIPS}

\label{MIPS_leaf_function_ex1}

The GCC assembly output below lists registers by number:

\lstinputlisting[caption=\Optimizing GCC 4.4.5 (\assemblyOutput),style=customasmMIPS]{patterns/011_ret/MIPS.s}

\dots while \IDA does it by their pseudo names:

\lstinputlisting[caption=\Optimizing GCC 4.4.5 (IDA),style=customasmMIPS]{patterns/011_ret/MIPS_IDA.lst}

The \$2 (or \$V0) register is used to store the function's return value.
\myindex{MIPS!\Pseudoinstructions!LI}
\INS{LI} stands for ``Load Immediate'' and is the MIPS equivalent to \MOV.

\myindex{MIPS!\Instructions!J}
The other instruction is the jump instruction (J or JR) which returns the execution flow to the \gls{caller}.

\myindex{MIPS!Branch delay slot}
You might be wondering why the positions of the load instruction (LI) and the jump instruction (J or JR) are swapped. This is due to a \ac{RISC} feature called ``branch delay slot''.

The reason this happens is a quirk in the architecture of some RISC \ac{ISA}s and isn't important for our
purposes---we must simply keep in mind that in MIPS, the instruction following a jump or branch instruction
is executed \IT{before} the jump/branch instruction itself.

As a consequence, branch instructions always swap places with the instruction executed immediately beforehand.


In practice, functions which merely return 1 (\IT{true}) or 0 (\IT{false}) are very frequent.

The smallest ever of the standard UNIX utilities, \IT{/bin/true} and \IT{/bin/false} return 0 and 1 respectively, as an exit code.
(Zero as an exit code usually means success, non-zero means error.)
}
\RU{\subsubsection{std::string}
\myindex{\Cpp!STL!std::string}
\label{std_string}

\myparagraph{Как устроена структура}

Многие строковые библиотеки \InSqBrackets{\CNotes 2.2} обеспечивают структуру содержащую ссылку 
на буфер собственно со строкой, переменная всегда содержащую длину строки 
(что очень удобно для массы функций \InSqBrackets{\CNotes 2.2.1}) и переменную содержащую текущий размер буфера.

Строка в буфере обыкновенно оканчивается нулем: это для того чтобы указатель на буфер можно было
передавать в функции требующие на вход обычную сишную \ac{ASCIIZ}-строку.

Стандарт \Cpp не описывает, как именно нужно реализовывать std::string,
но, как правило, они реализованы как описано выше, с небольшими дополнениями.

Строки в \Cpp это не класс (как, например, QString в Qt), а темплейт (basic\_string), 
это сделано для того чтобы поддерживать 
строки содержащие разного типа символы: как минимум \Tchar и \IT{wchar\_t}.

Так что, std::string это класс с базовым типом \Tchar.

А std::wstring это класс с базовым типом \IT{wchar\_t}.

\mysubparagraph{MSVC}

В реализации MSVC, вместо ссылки на буфер может содержаться сам буфер (если строка короче 16-и символов).

Это означает, что каждая короткая строка будет занимать в памяти по крайней мере $16 + 4 + 4 = 24$ 
байт для 32-битной среды либо $16 + 8 + 8 = 32$ 
байта в 64-битной, а если строка длиннее 16-и символов, то прибавьте еще длину самой строки.

\lstinputlisting[caption=пример для MSVC,style=customc]{\CURPATH/STL/string/MSVC_RU.cpp}

Собственно, из этого исходника почти всё ясно.

Несколько замечаний:

Если строка короче 16-и символов, 
то отдельный буфер для строки в \glslink{heap}{куче} выделяться не будет.

Это удобно потому что на практике, основная часть строк действительно короткие.
Вероятно, разработчики в Microsoft выбрали размер в 16 символов как разумный баланс.

Теперь очень важный момент в конце функции main(): мы не пользуемся методом c\_str(), тем не менее,
если это скомпилировать и запустить, то обе строки появятся в консоли!

Работает это вот почему.

В первом случае строка короче 16-и символов и в начале объекта std::string (его можно рассматривать
просто как структуру) расположен буфер с этой строкой.
\printf трактует указатель как указатель на массив символов оканчивающийся нулем и поэтому всё работает.

Вывод второй строки (длиннее 16-и символов) даже еще опаснее: это вообще типичная программистская ошибка 
(или опечатка), забыть дописать c\_str().
Это работает потому что в это время в начале структуры расположен указатель на буфер.
Это может надолго остаться незамеченным: до тех пока там не появится строка 
короче 16-и символов, тогда процесс упадет.

\mysubparagraph{GCC}

В реализации GCC в структуре есть еще одна переменная --- reference count.

Интересно, что указатель на экземпляр класса std::string в GCC указывает не на начало самой структуры, 
а на указатель на буфера.
В libstdc++-v3\textbackslash{}include\textbackslash{}bits\textbackslash{}basic\_string.h 
мы можем прочитать что это сделано для удобства отладки:

\begin{lstlisting}
   *  The reason you want _M_data pointing to the character %array and
   *  not the _Rep is so that the debugger can see the string
   *  contents. (Probably we should add a non-inline member to get
   *  the _Rep for the debugger to use, so users can check the actual
   *  string length.)
\end{lstlisting}

\href{http://go.yurichev.com/17085}{исходный код basic\_string.h}

В нашем примере мы учитываем это:

\lstinputlisting[caption=пример для GCC,style=customc]{\CURPATH/STL/string/GCC_RU.cpp}

Нужны еще небольшие хаки чтобы сымитировать типичную ошибку, которую мы уже видели выше, из-за
более ужесточенной проверки типов в GCC, тем не менее, printf() работает и здесь без c\_str().

\myparagraph{Чуть более сложный пример}

\lstinputlisting[style=customc]{\CURPATH/STL/string/3.cpp}

\lstinputlisting[caption=MSVC 2012,style=customasmx86]{\CURPATH/STL/string/3_MSVC_RU.asm}

Собственно, компилятор не конструирует строки статически: да в общем-то и как
это возможно, если буфер с ней нужно хранить в \glslink{heap}{куче}?

Вместо этого в сегменте данных хранятся обычные \ac{ASCIIZ}-строки, а позже, во время выполнения, 
при помощи метода \q{assign}, конструируются строки s1 и s2
.
При помощи \TT{operator+}, создается строка s3.

Обратите внимание на то что вызов метода c\_str() отсутствует,
потому что его код достаточно короткий и компилятор вставил его прямо здесь:
если строка короче 16-и байт, то в регистре EAX остается указатель на буфер,
а если длиннее, то из этого же места достается адрес на буфер расположенный в \glslink{heap}{куче}.

Далее следуют вызовы трех деструкторов, причем, они вызываются только если строка длиннее 16-и байт:
тогда нужно освободить буфера в \glslink{heap}{куче}.
В противном случае, так как все три объекта std::string хранятся в стеке,
они освобождаются автоматически после выхода из функции.

Следовательно, работа с короткими строками более быстрая из-за м\'{е}ньшего обращения к \glslink{heap}{куче}.

Код на GCC даже проще (из-за того, что в GCC, как мы уже видели, не реализована возможность хранить короткую
строку прямо в структуре):

% TODO1 comment each function meaning
\lstinputlisting[caption=GCC 4.8.1,style=customasmx86]{\CURPATH/STL/string/3_GCC_RU.s}

Можно заметить, что в деструкторы передается не указатель на объект,
а указатель на место за 12 байт (или 3 слова) перед ним, то есть, на настоящее начало структуры.

\myparagraph{std::string как глобальная переменная}
\label{sec:std_string_as_global_variable}

Опытные программисты на \Cpp знают, что глобальные переменные \ac{STL}-типов вполне можно объявлять.

Да, действительно:

\lstinputlisting[style=customc]{\CURPATH/STL/string/5.cpp}

Но как и где будет вызываться конструктор \TT{std::string}?

На самом деле, эта переменная будет инициализирована даже перед началом \main.

\lstinputlisting[caption=MSVC 2012: здесь конструируется глобальная переменная{,} а также регистрируется её деструктор,style=customasmx86]{\CURPATH/STL/string/5_MSVC_p2.asm}

\lstinputlisting[caption=MSVC 2012: здесь глобальная переменная используется в \main,style=customasmx86]{\CURPATH/STL/string/5_MSVC_p1.asm}

\lstinputlisting[caption=MSVC 2012: эта функция-деструктор вызывается перед выходом,style=customasmx86]{\CURPATH/STL/string/5_MSVC_p3.asm}

\myindex{\CStandardLibrary!atexit()}
В реальности, из \ac{CRT}, еще до вызова main(), вызывается специальная функция,
в которой перечислены все конструкторы подобных переменных.
Более того: при помощи atexit() регистрируется функция, которая будет вызвана в конце работы программы:
в этой функции компилятор собирает вызовы деструкторов всех подобных глобальных переменных.

GCC работает похожим образом:

\lstinputlisting[caption=GCC 4.8.1,style=customasmx86]{\CURPATH/STL/string/5_GCC.s}

Но он не выделяет отдельной функции в которой будут собраны деструкторы: 
каждый деструктор передается в atexit() по одному.

% TODO а если глобальная STL-переменная в другом модуле? надо проверить.

}
\DE{\subsection{Einfachste XOR-Verschlüsselung überhaupt}

Ich habe einmal eine Software gesehen, bei der alle Debugging-Ausgaben mit XOR mit dem Wert 3
verschlüsselt wurden. Mit anderen Worten, die beiden niedrigsten Bits aller Buchstaben wurden invertiert.

``Hello, world'' wurde zu ``Kfool/\#tlqog'':

\begin{lstlisting}
#!/usr/bin/python

msg="Hello, world!"

print "".join(map(lambda x: chr(ord(x)^3), msg))
\end{lstlisting}

Das ist eine ziemlich interessante Verschlüsselung (oder besser eine Verschleierung),
weil sie zwei wichtige Eigenschaften hat:
1) es ist eine einzige Funktion zum Verschlüsseln und entschlüsseln, sie muss nur wiederholt angewendet werden
2) die entstehenden Buchstaben befinden sich im druckbaren Bereich, also die ganze Zeichenkette kann ohne
Escape-Symbole im Code verwendet werden.

Die zweite Eigenschaft nutzt die Tatsache, dass alle druckbaren Zeichen in Reihen organisiert sind: 0x2x-0x7x,
und wenn die beiden niederwertigsten Bits invertiert werden, wird der Buchstabe um eine oder drei Stellen nach
links oder rechts \IT{verschoben}, aber niemals in eine andere Reihe:

\begin{figure}[H]
\centering
\includegraphics[width=0.7\textwidth]{ascii_clean.png}
\caption{7-Bit \ac{ASCII} Tabelle in Emacs}
\end{figure}

\dots mit dem Zeichen 0x7F als einziger Ausnahme.

Im Folgenden werden also beispielsweise die Zeichen A-Z \IT{verschlüsselt}:

\begin{lstlisting}
#!/usr/bin/python

msg="@ABCDEFGHIJKLMNO"

print "".join(map(lambda x: chr(ord(x)^3), msg))
\end{lstlisting}

Ergebnis:
% FIXME \verb  --  relevant comment for German?
\begin{lstlisting}
CBA@GFEDKJIHONML
\end{lstlisting}

Es sieht so aus als würden die Zeichen ``@'' und ``C'' sowie ``B'' und ``A'' vertauscht werden.

Hier ist noch ein interessantes Beispiel, in dem gezeigt wird, wie die Eigenschaften von XOR
ausgenutzt werden können: Exakt den gleichen Effekt, dass druckbare Zeichen auch druckbar bleiben,
kann man dadurch erzielen, dass irgendeine Kombination der niedrigsten vier Bits invertiert wird.
}

\EN{\section{Returning Values}
\label{ret_val_func}

Another simple function is the one that simply returns a constant value:

\lstinputlisting[caption=\EN{\CCpp Code},style=customc]{patterns/011_ret/1.c}

Let's compile it.

\subsection{x86}

Here's what both the GCC and MSVC compilers produce (with optimization) on the x86 platform:

\lstinputlisting[caption=\Optimizing GCC/MSVC (\assemblyOutput),style=customasmx86]{patterns/011_ret/1.s}

\myindex{x86!\Instructions!RET}
There are just two instructions: the first places the value 123 into the \EAX register,
which is used by convention for storing the return
value, and the second one is \RET, which returns execution to the \gls{caller}.

The caller will take the result from the \EAX register.

\subsection{ARM}

There are a few differences on the ARM platform:

\lstinputlisting[caption=\OptimizingKeilVI (\ARMMode) ASM Output,style=customasmARM]{patterns/011_ret/1_Keil_ARM_O3.s}

ARM uses the register \Reg{0} for returning the results of functions, so 123 is copied into \Reg{0}.

\myindex{ARM!\Instructions!MOV}
\myindex{x86!\Instructions!MOV}
It is worth noting that \MOV is a misleading name for the instruction in both the x86 and ARM \ac{ISA}s.

The data is not in fact \IT{moved}, but \IT{copied}.

\subsection{MIPS}

\label{MIPS_leaf_function_ex1}

The GCC assembly output below lists registers by number:

\lstinputlisting[caption=\Optimizing GCC 4.4.5 (\assemblyOutput),style=customasmMIPS]{patterns/011_ret/MIPS.s}

\dots while \IDA does it by their pseudo names:

\lstinputlisting[caption=\Optimizing GCC 4.4.5 (IDA),style=customasmMIPS]{patterns/011_ret/MIPS_IDA.lst}

The \$2 (or \$V0) register is used to store the function's return value.
\myindex{MIPS!\Pseudoinstructions!LI}
\INS{LI} stands for ``Load Immediate'' and is the MIPS equivalent to \MOV.

\myindex{MIPS!\Instructions!J}
The other instruction is the jump instruction (J or JR) which returns the execution flow to the \gls{caller}.

\myindex{MIPS!Branch delay slot}
You might be wondering why the positions of the load instruction (LI) and the jump instruction (J or JR) are swapped. This is due to a \ac{RISC} feature called ``branch delay slot''.

The reason this happens is a quirk in the architecture of some RISC \ac{ISA}s and isn't important for our
purposes---we must simply keep in mind that in MIPS, the instruction following a jump or branch instruction
is executed \IT{before} the jump/branch instruction itself.

As a consequence, branch instructions always swap places with the instruction executed immediately beforehand.


In practice, functions which merely return 1 (\IT{true}) or 0 (\IT{false}) are very frequent.

The smallest ever of the standard UNIX utilities, \IT{/bin/true} and \IT{/bin/false} return 0 and 1 respectively, as an exit code.
(Zero as an exit code usually means success, non-zero means error.)
}
\RU{\subsubsection{std::string}
\myindex{\Cpp!STL!std::string}
\label{std_string}

\myparagraph{Как устроена структура}

Многие строковые библиотеки \InSqBrackets{\CNotes 2.2} обеспечивают структуру содержащую ссылку 
на буфер собственно со строкой, переменная всегда содержащую длину строки 
(что очень удобно для массы функций \InSqBrackets{\CNotes 2.2.1}) и переменную содержащую текущий размер буфера.

Строка в буфере обыкновенно оканчивается нулем: это для того чтобы указатель на буфер можно было
передавать в функции требующие на вход обычную сишную \ac{ASCIIZ}-строку.

Стандарт \Cpp не описывает, как именно нужно реализовывать std::string,
но, как правило, они реализованы как описано выше, с небольшими дополнениями.

Строки в \Cpp это не класс (как, например, QString в Qt), а темплейт (basic\_string), 
это сделано для того чтобы поддерживать 
строки содержащие разного типа символы: как минимум \Tchar и \IT{wchar\_t}.

Так что, std::string это класс с базовым типом \Tchar.

А std::wstring это класс с базовым типом \IT{wchar\_t}.

\mysubparagraph{MSVC}

В реализации MSVC, вместо ссылки на буфер может содержаться сам буфер (если строка короче 16-и символов).

Это означает, что каждая короткая строка будет занимать в памяти по крайней мере $16 + 4 + 4 = 24$ 
байт для 32-битной среды либо $16 + 8 + 8 = 32$ 
байта в 64-битной, а если строка длиннее 16-и символов, то прибавьте еще длину самой строки.

\lstinputlisting[caption=пример для MSVC,style=customc]{\CURPATH/STL/string/MSVC_RU.cpp}

Собственно, из этого исходника почти всё ясно.

Несколько замечаний:

Если строка короче 16-и символов, 
то отдельный буфер для строки в \glslink{heap}{куче} выделяться не будет.

Это удобно потому что на практике, основная часть строк действительно короткие.
Вероятно, разработчики в Microsoft выбрали размер в 16 символов как разумный баланс.

Теперь очень важный момент в конце функции main(): мы не пользуемся методом c\_str(), тем не менее,
если это скомпилировать и запустить, то обе строки появятся в консоли!

Работает это вот почему.

В первом случае строка короче 16-и символов и в начале объекта std::string (его можно рассматривать
просто как структуру) расположен буфер с этой строкой.
\printf трактует указатель как указатель на массив символов оканчивающийся нулем и поэтому всё работает.

Вывод второй строки (длиннее 16-и символов) даже еще опаснее: это вообще типичная программистская ошибка 
(или опечатка), забыть дописать c\_str().
Это работает потому что в это время в начале структуры расположен указатель на буфер.
Это может надолго остаться незамеченным: до тех пока там не появится строка 
короче 16-и символов, тогда процесс упадет.

\mysubparagraph{GCC}

В реализации GCC в структуре есть еще одна переменная --- reference count.

Интересно, что указатель на экземпляр класса std::string в GCC указывает не на начало самой структуры, 
а на указатель на буфера.
В libstdc++-v3\textbackslash{}include\textbackslash{}bits\textbackslash{}basic\_string.h 
мы можем прочитать что это сделано для удобства отладки:

\begin{lstlisting}
   *  The reason you want _M_data pointing to the character %array and
   *  not the _Rep is so that the debugger can see the string
   *  contents. (Probably we should add a non-inline member to get
   *  the _Rep for the debugger to use, so users can check the actual
   *  string length.)
\end{lstlisting}

\href{http://go.yurichev.com/17085}{исходный код basic\_string.h}

В нашем примере мы учитываем это:

\lstinputlisting[caption=пример для GCC,style=customc]{\CURPATH/STL/string/GCC_RU.cpp}

Нужны еще небольшие хаки чтобы сымитировать типичную ошибку, которую мы уже видели выше, из-за
более ужесточенной проверки типов в GCC, тем не менее, printf() работает и здесь без c\_str().

\myparagraph{Чуть более сложный пример}

\lstinputlisting[style=customc]{\CURPATH/STL/string/3.cpp}

\lstinputlisting[caption=MSVC 2012,style=customasmx86]{\CURPATH/STL/string/3_MSVC_RU.asm}

Собственно, компилятор не конструирует строки статически: да в общем-то и как
это возможно, если буфер с ней нужно хранить в \glslink{heap}{куче}?

Вместо этого в сегменте данных хранятся обычные \ac{ASCIIZ}-строки, а позже, во время выполнения, 
при помощи метода \q{assign}, конструируются строки s1 и s2
.
При помощи \TT{operator+}, создается строка s3.

Обратите внимание на то что вызов метода c\_str() отсутствует,
потому что его код достаточно короткий и компилятор вставил его прямо здесь:
если строка короче 16-и байт, то в регистре EAX остается указатель на буфер,
а если длиннее, то из этого же места достается адрес на буфер расположенный в \glslink{heap}{куче}.

Далее следуют вызовы трех деструкторов, причем, они вызываются только если строка длиннее 16-и байт:
тогда нужно освободить буфера в \glslink{heap}{куче}.
В противном случае, так как все три объекта std::string хранятся в стеке,
они освобождаются автоматически после выхода из функции.

Следовательно, работа с короткими строками более быстрая из-за м\'{е}ньшего обращения к \glslink{heap}{куче}.

Код на GCC даже проще (из-за того, что в GCC, как мы уже видели, не реализована возможность хранить короткую
строку прямо в структуре):

% TODO1 comment each function meaning
\lstinputlisting[caption=GCC 4.8.1,style=customasmx86]{\CURPATH/STL/string/3_GCC_RU.s}

Можно заметить, что в деструкторы передается не указатель на объект,
а указатель на место за 12 байт (или 3 слова) перед ним, то есть, на настоящее начало структуры.

\myparagraph{std::string как глобальная переменная}
\label{sec:std_string_as_global_variable}

Опытные программисты на \Cpp знают, что глобальные переменные \ac{STL}-типов вполне можно объявлять.

Да, действительно:

\lstinputlisting[style=customc]{\CURPATH/STL/string/5.cpp}

Но как и где будет вызываться конструктор \TT{std::string}?

На самом деле, эта переменная будет инициализирована даже перед началом \main.

\lstinputlisting[caption=MSVC 2012: здесь конструируется глобальная переменная{,} а также регистрируется её деструктор,style=customasmx86]{\CURPATH/STL/string/5_MSVC_p2.asm}

\lstinputlisting[caption=MSVC 2012: здесь глобальная переменная используется в \main,style=customasmx86]{\CURPATH/STL/string/5_MSVC_p1.asm}

\lstinputlisting[caption=MSVC 2012: эта функция-деструктор вызывается перед выходом,style=customasmx86]{\CURPATH/STL/string/5_MSVC_p3.asm}

\myindex{\CStandardLibrary!atexit()}
В реальности, из \ac{CRT}, еще до вызова main(), вызывается специальная функция,
в которой перечислены все конструкторы подобных переменных.
Более того: при помощи atexit() регистрируется функция, которая будет вызвана в конце работы программы:
в этой функции компилятор собирает вызовы деструкторов всех подобных глобальных переменных.

GCC работает похожим образом:

\lstinputlisting[caption=GCC 4.8.1,style=customasmx86]{\CURPATH/STL/string/5_GCC.s}

Но он не выделяет отдельной функции в которой будут собраны деструкторы: 
каждый деструктор передается в atexit() по одному.

% TODO а если глобальная STL-переменная в другом модуле? надо проверить.

}

\EN{\section{Returning Values}
\label{ret_val_func}

Another simple function is the one that simply returns a constant value:

\lstinputlisting[caption=\EN{\CCpp Code},style=customc]{patterns/011_ret/1.c}

Let's compile it.

\subsection{x86}

Here's what both the GCC and MSVC compilers produce (with optimization) on the x86 platform:

\lstinputlisting[caption=\Optimizing GCC/MSVC (\assemblyOutput),style=customasmx86]{patterns/011_ret/1.s}

\myindex{x86!\Instructions!RET}
There are just two instructions: the first places the value 123 into the \EAX register,
which is used by convention for storing the return
value, and the second one is \RET, which returns execution to the \gls{caller}.

The caller will take the result from the \EAX register.

\subsection{ARM}

There are a few differences on the ARM platform:

\lstinputlisting[caption=\OptimizingKeilVI (\ARMMode) ASM Output,style=customasmARM]{patterns/011_ret/1_Keil_ARM_O3.s}

ARM uses the register \Reg{0} for returning the results of functions, so 123 is copied into \Reg{0}.

\myindex{ARM!\Instructions!MOV}
\myindex{x86!\Instructions!MOV}
It is worth noting that \MOV is a misleading name for the instruction in both the x86 and ARM \ac{ISA}s.

The data is not in fact \IT{moved}, but \IT{copied}.

\subsection{MIPS}

\label{MIPS_leaf_function_ex1}

The GCC assembly output below lists registers by number:

\lstinputlisting[caption=\Optimizing GCC 4.4.5 (\assemblyOutput),style=customasmMIPS]{patterns/011_ret/MIPS.s}

\dots while \IDA does it by their pseudo names:

\lstinputlisting[caption=\Optimizing GCC 4.4.5 (IDA),style=customasmMIPS]{patterns/011_ret/MIPS_IDA.lst}

The \$2 (or \$V0) register is used to store the function's return value.
\myindex{MIPS!\Pseudoinstructions!LI}
\INS{LI} stands for ``Load Immediate'' and is the MIPS equivalent to \MOV.

\myindex{MIPS!\Instructions!J}
The other instruction is the jump instruction (J or JR) which returns the execution flow to the \gls{caller}.

\myindex{MIPS!Branch delay slot}
You might be wondering why the positions of the load instruction (LI) and the jump instruction (J or JR) are swapped. This is due to a \ac{RISC} feature called ``branch delay slot''.

The reason this happens is a quirk in the architecture of some RISC \ac{ISA}s and isn't important for our
purposes---we must simply keep in mind that in MIPS, the instruction following a jump or branch instruction
is executed \IT{before} the jump/branch instruction itself.

As a consequence, branch instructions always swap places with the instruction executed immediately beforehand.


In practice, functions which merely return 1 (\IT{true}) or 0 (\IT{false}) are very frequent.

The smallest ever of the standard UNIX utilities, \IT{/bin/true} and \IT{/bin/false} return 0 and 1 respectively, as an exit code.
(Zero as an exit code usually means success, non-zero means error.)
}
\RU{\subsubsection{std::string}
\myindex{\Cpp!STL!std::string}
\label{std_string}

\myparagraph{Как устроена структура}

Многие строковые библиотеки \InSqBrackets{\CNotes 2.2} обеспечивают структуру содержащую ссылку 
на буфер собственно со строкой, переменная всегда содержащую длину строки 
(что очень удобно для массы функций \InSqBrackets{\CNotes 2.2.1}) и переменную содержащую текущий размер буфера.

Строка в буфере обыкновенно оканчивается нулем: это для того чтобы указатель на буфер можно было
передавать в функции требующие на вход обычную сишную \ac{ASCIIZ}-строку.

Стандарт \Cpp не описывает, как именно нужно реализовывать std::string,
но, как правило, они реализованы как описано выше, с небольшими дополнениями.

Строки в \Cpp это не класс (как, например, QString в Qt), а темплейт (basic\_string), 
это сделано для того чтобы поддерживать 
строки содержащие разного типа символы: как минимум \Tchar и \IT{wchar\_t}.

Так что, std::string это класс с базовым типом \Tchar.

А std::wstring это класс с базовым типом \IT{wchar\_t}.

\mysubparagraph{MSVC}

В реализации MSVC, вместо ссылки на буфер может содержаться сам буфер (если строка короче 16-и символов).

Это означает, что каждая короткая строка будет занимать в памяти по крайней мере $16 + 4 + 4 = 24$ 
байт для 32-битной среды либо $16 + 8 + 8 = 32$ 
байта в 64-битной, а если строка длиннее 16-и символов, то прибавьте еще длину самой строки.

\lstinputlisting[caption=пример для MSVC,style=customc]{\CURPATH/STL/string/MSVC_RU.cpp}

Собственно, из этого исходника почти всё ясно.

Несколько замечаний:

Если строка короче 16-и символов, 
то отдельный буфер для строки в \glslink{heap}{куче} выделяться не будет.

Это удобно потому что на практике, основная часть строк действительно короткие.
Вероятно, разработчики в Microsoft выбрали размер в 16 символов как разумный баланс.

Теперь очень важный момент в конце функции main(): мы не пользуемся методом c\_str(), тем не менее,
если это скомпилировать и запустить, то обе строки появятся в консоли!

Работает это вот почему.

В первом случае строка короче 16-и символов и в начале объекта std::string (его можно рассматривать
просто как структуру) расположен буфер с этой строкой.
\printf трактует указатель как указатель на массив символов оканчивающийся нулем и поэтому всё работает.

Вывод второй строки (длиннее 16-и символов) даже еще опаснее: это вообще типичная программистская ошибка 
(или опечатка), забыть дописать c\_str().
Это работает потому что в это время в начале структуры расположен указатель на буфер.
Это может надолго остаться незамеченным: до тех пока там не появится строка 
короче 16-и символов, тогда процесс упадет.

\mysubparagraph{GCC}

В реализации GCC в структуре есть еще одна переменная --- reference count.

Интересно, что указатель на экземпляр класса std::string в GCC указывает не на начало самой структуры, 
а на указатель на буфера.
В libstdc++-v3\textbackslash{}include\textbackslash{}bits\textbackslash{}basic\_string.h 
мы можем прочитать что это сделано для удобства отладки:

\begin{lstlisting}
   *  The reason you want _M_data pointing to the character %array and
   *  not the _Rep is so that the debugger can see the string
   *  contents. (Probably we should add a non-inline member to get
   *  the _Rep for the debugger to use, so users can check the actual
   *  string length.)
\end{lstlisting}

\href{http://go.yurichev.com/17085}{исходный код basic\_string.h}

В нашем примере мы учитываем это:

\lstinputlisting[caption=пример для GCC,style=customc]{\CURPATH/STL/string/GCC_RU.cpp}

Нужны еще небольшие хаки чтобы сымитировать типичную ошибку, которую мы уже видели выше, из-за
более ужесточенной проверки типов в GCC, тем не менее, printf() работает и здесь без c\_str().

\myparagraph{Чуть более сложный пример}

\lstinputlisting[style=customc]{\CURPATH/STL/string/3.cpp}

\lstinputlisting[caption=MSVC 2012,style=customasmx86]{\CURPATH/STL/string/3_MSVC_RU.asm}

Собственно, компилятор не конструирует строки статически: да в общем-то и как
это возможно, если буфер с ней нужно хранить в \glslink{heap}{куче}?

Вместо этого в сегменте данных хранятся обычные \ac{ASCIIZ}-строки, а позже, во время выполнения, 
при помощи метода \q{assign}, конструируются строки s1 и s2
.
При помощи \TT{operator+}, создается строка s3.

Обратите внимание на то что вызов метода c\_str() отсутствует,
потому что его код достаточно короткий и компилятор вставил его прямо здесь:
если строка короче 16-и байт, то в регистре EAX остается указатель на буфер,
а если длиннее, то из этого же места достается адрес на буфер расположенный в \glslink{heap}{куче}.

Далее следуют вызовы трех деструкторов, причем, они вызываются только если строка длиннее 16-и байт:
тогда нужно освободить буфера в \glslink{heap}{куче}.
В противном случае, так как все три объекта std::string хранятся в стеке,
они освобождаются автоматически после выхода из функции.

Следовательно, работа с короткими строками более быстрая из-за м\'{е}ньшего обращения к \glslink{heap}{куче}.

Код на GCC даже проще (из-за того, что в GCC, как мы уже видели, не реализована возможность хранить короткую
строку прямо в структуре):

% TODO1 comment each function meaning
\lstinputlisting[caption=GCC 4.8.1,style=customasmx86]{\CURPATH/STL/string/3_GCC_RU.s}

Можно заметить, что в деструкторы передается не указатель на объект,
а указатель на место за 12 байт (или 3 слова) перед ним, то есть, на настоящее начало структуры.

\myparagraph{std::string как глобальная переменная}
\label{sec:std_string_as_global_variable}

Опытные программисты на \Cpp знают, что глобальные переменные \ac{STL}-типов вполне можно объявлять.

Да, действительно:

\lstinputlisting[style=customc]{\CURPATH/STL/string/5.cpp}

Но как и где будет вызываться конструктор \TT{std::string}?

На самом деле, эта переменная будет инициализирована даже перед началом \main.

\lstinputlisting[caption=MSVC 2012: здесь конструируется глобальная переменная{,} а также регистрируется её деструктор,style=customasmx86]{\CURPATH/STL/string/5_MSVC_p2.asm}

\lstinputlisting[caption=MSVC 2012: здесь глобальная переменная используется в \main,style=customasmx86]{\CURPATH/STL/string/5_MSVC_p1.asm}

\lstinputlisting[caption=MSVC 2012: эта функция-деструктор вызывается перед выходом,style=customasmx86]{\CURPATH/STL/string/5_MSVC_p3.asm}

\myindex{\CStandardLibrary!atexit()}
В реальности, из \ac{CRT}, еще до вызова main(), вызывается специальная функция,
в которой перечислены все конструкторы подобных переменных.
Более того: при помощи atexit() регистрируется функция, которая будет вызвана в конце работы программы:
в этой функции компилятор собирает вызовы деструкторов всех подобных глобальных переменных.

GCC работает похожим образом:

\lstinputlisting[caption=GCC 4.8.1,style=customasmx86]{\CURPATH/STL/string/5_GCC.s}

Но он не выделяет отдельной функции в которой будут собраны деструкторы: 
каждый деструктор передается в atexit() по одному.

% TODO а если глобальная STL-переменная в другом модуле? надо проверить.

}
\DE{\subsection{Einfachste XOR-Verschlüsselung überhaupt}

Ich habe einmal eine Software gesehen, bei der alle Debugging-Ausgaben mit XOR mit dem Wert 3
verschlüsselt wurden. Mit anderen Worten, die beiden niedrigsten Bits aller Buchstaben wurden invertiert.

``Hello, world'' wurde zu ``Kfool/\#tlqog'':

\begin{lstlisting}
#!/usr/bin/python

msg="Hello, world!"

print "".join(map(lambda x: chr(ord(x)^3), msg))
\end{lstlisting}

Das ist eine ziemlich interessante Verschlüsselung (oder besser eine Verschleierung),
weil sie zwei wichtige Eigenschaften hat:
1) es ist eine einzige Funktion zum Verschlüsseln und entschlüsseln, sie muss nur wiederholt angewendet werden
2) die entstehenden Buchstaben befinden sich im druckbaren Bereich, also die ganze Zeichenkette kann ohne
Escape-Symbole im Code verwendet werden.

Die zweite Eigenschaft nutzt die Tatsache, dass alle druckbaren Zeichen in Reihen organisiert sind: 0x2x-0x7x,
und wenn die beiden niederwertigsten Bits invertiert werden, wird der Buchstabe um eine oder drei Stellen nach
links oder rechts \IT{verschoben}, aber niemals in eine andere Reihe:

\begin{figure}[H]
\centering
\includegraphics[width=0.7\textwidth]{ascii_clean.png}
\caption{7-Bit \ac{ASCII} Tabelle in Emacs}
\end{figure}

\dots mit dem Zeichen 0x7F als einziger Ausnahme.

Im Folgenden werden also beispielsweise die Zeichen A-Z \IT{verschlüsselt}:

\begin{lstlisting}
#!/usr/bin/python

msg="@ABCDEFGHIJKLMNO"

print "".join(map(lambda x: chr(ord(x)^3), msg))
\end{lstlisting}

Ergebnis:
% FIXME \verb  --  relevant comment for German?
\begin{lstlisting}
CBA@GFEDKJIHONML
\end{lstlisting}

Es sieht so aus als würden die Zeichen ``@'' und ``C'' sowie ``B'' und ``A'' vertauscht werden.

Hier ist noch ein interessantes Beispiel, in dem gezeigt wird, wie die Eigenschaften von XOR
ausgenutzt werden können: Exakt den gleichen Effekt, dass druckbare Zeichen auch druckbar bleiben,
kann man dadurch erzielen, dass irgendeine Kombination der niedrigsten vier Bits invertiert wird.
}

\EN{\section{Returning Values}
\label{ret_val_func}

Another simple function is the one that simply returns a constant value:

\lstinputlisting[caption=\EN{\CCpp Code},style=customc]{patterns/011_ret/1.c}

Let's compile it.

\subsection{x86}

Here's what both the GCC and MSVC compilers produce (with optimization) on the x86 platform:

\lstinputlisting[caption=\Optimizing GCC/MSVC (\assemblyOutput),style=customasmx86]{patterns/011_ret/1.s}

\myindex{x86!\Instructions!RET}
There are just two instructions: the first places the value 123 into the \EAX register,
which is used by convention for storing the return
value, and the second one is \RET, which returns execution to the \gls{caller}.

The caller will take the result from the \EAX register.

\subsection{ARM}

There are a few differences on the ARM platform:

\lstinputlisting[caption=\OptimizingKeilVI (\ARMMode) ASM Output,style=customasmARM]{patterns/011_ret/1_Keil_ARM_O3.s}

ARM uses the register \Reg{0} for returning the results of functions, so 123 is copied into \Reg{0}.

\myindex{ARM!\Instructions!MOV}
\myindex{x86!\Instructions!MOV}
It is worth noting that \MOV is a misleading name for the instruction in both the x86 and ARM \ac{ISA}s.

The data is not in fact \IT{moved}, but \IT{copied}.

\subsection{MIPS}

\label{MIPS_leaf_function_ex1}

The GCC assembly output below lists registers by number:

\lstinputlisting[caption=\Optimizing GCC 4.4.5 (\assemblyOutput),style=customasmMIPS]{patterns/011_ret/MIPS.s}

\dots while \IDA does it by their pseudo names:

\lstinputlisting[caption=\Optimizing GCC 4.4.5 (IDA),style=customasmMIPS]{patterns/011_ret/MIPS_IDA.lst}

The \$2 (or \$V0) register is used to store the function's return value.
\myindex{MIPS!\Pseudoinstructions!LI}
\INS{LI} stands for ``Load Immediate'' and is the MIPS equivalent to \MOV.

\myindex{MIPS!\Instructions!J}
The other instruction is the jump instruction (J or JR) which returns the execution flow to the \gls{caller}.

\myindex{MIPS!Branch delay slot}
You might be wondering why the positions of the load instruction (LI) and the jump instruction (J or JR) are swapped. This is due to a \ac{RISC} feature called ``branch delay slot''.

The reason this happens is a quirk in the architecture of some RISC \ac{ISA}s and isn't important for our
purposes---we must simply keep in mind that in MIPS, the instruction following a jump or branch instruction
is executed \IT{before} the jump/branch instruction itself.

As a consequence, branch instructions always swap places with the instruction executed immediately beforehand.


In practice, functions which merely return 1 (\IT{true}) or 0 (\IT{false}) are very frequent.

The smallest ever of the standard UNIX utilities, \IT{/bin/true} and \IT{/bin/false} return 0 and 1 respectively, as an exit code.
(Zero as an exit code usually means success, non-zero means error.)
}
\RU{\subsubsection{std::string}
\myindex{\Cpp!STL!std::string}
\label{std_string}

\myparagraph{Как устроена структура}

Многие строковые библиотеки \InSqBrackets{\CNotes 2.2} обеспечивают структуру содержащую ссылку 
на буфер собственно со строкой, переменная всегда содержащую длину строки 
(что очень удобно для массы функций \InSqBrackets{\CNotes 2.2.1}) и переменную содержащую текущий размер буфера.

Строка в буфере обыкновенно оканчивается нулем: это для того чтобы указатель на буфер можно было
передавать в функции требующие на вход обычную сишную \ac{ASCIIZ}-строку.

Стандарт \Cpp не описывает, как именно нужно реализовывать std::string,
но, как правило, они реализованы как описано выше, с небольшими дополнениями.

Строки в \Cpp это не класс (как, например, QString в Qt), а темплейт (basic\_string), 
это сделано для того чтобы поддерживать 
строки содержащие разного типа символы: как минимум \Tchar и \IT{wchar\_t}.

Так что, std::string это класс с базовым типом \Tchar.

А std::wstring это класс с базовым типом \IT{wchar\_t}.

\mysubparagraph{MSVC}

В реализации MSVC, вместо ссылки на буфер может содержаться сам буфер (если строка короче 16-и символов).

Это означает, что каждая короткая строка будет занимать в памяти по крайней мере $16 + 4 + 4 = 24$ 
байт для 32-битной среды либо $16 + 8 + 8 = 32$ 
байта в 64-битной, а если строка длиннее 16-и символов, то прибавьте еще длину самой строки.

\lstinputlisting[caption=пример для MSVC,style=customc]{\CURPATH/STL/string/MSVC_RU.cpp}

Собственно, из этого исходника почти всё ясно.

Несколько замечаний:

Если строка короче 16-и символов, 
то отдельный буфер для строки в \glslink{heap}{куче} выделяться не будет.

Это удобно потому что на практике, основная часть строк действительно короткие.
Вероятно, разработчики в Microsoft выбрали размер в 16 символов как разумный баланс.

Теперь очень важный момент в конце функции main(): мы не пользуемся методом c\_str(), тем не менее,
если это скомпилировать и запустить, то обе строки появятся в консоли!

Работает это вот почему.

В первом случае строка короче 16-и символов и в начале объекта std::string (его можно рассматривать
просто как структуру) расположен буфер с этой строкой.
\printf трактует указатель как указатель на массив символов оканчивающийся нулем и поэтому всё работает.

Вывод второй строки (длиннее 16-и символов) даже еще опаснее: это вообще типичная программистская ошибка 
(или опечатка), забыть дописать c\_str().
Это работает потому что в это время в начале структуры расположен указатель на буфер.
Это может надолго остаться незамеченным: до тех пока там не появится строка 
короче 16-и символов, тогда процесс упадет.

\mysubparagraph{GCC}

В реализации GCC в структуре есть еще одна переменная --- reference count.

Интересно, что указатель на экземпляр класса std::string в GCC указывает не на начало самой структуры, 
а на указатель на буфера.
В libstdc++-v3\textbackslash{}include\textbackslash{}bits\textbackslash{}basic\_string.h 
мы можем прочитать что это сделано для удобства отладки:

\begin{lstlisting}
   *  The reason you want _M_data pointing to the character %array and
   *  not the _Rep is so that the debugger can see the string
   *  contents. (Probably we should add a non-inline member to get
   *  the _Rep for the debugger to use, so users can check the actual
   *  string length.)
\end{lstlisting}

\href{http://go.yurichev.com/17085}{исходный код basic\_string.h}

В нашем примере мы учитываем это:

\lstinputlisting[caption=пример для GCC,style=customc]{\CURPATH/STL/string/GCC_RU.cpp}

Нужны еще небольшие хаки чтобы сымитировать типичную ошибку, которую мы уже видели выше, из-за
более ужесточенной проверки типов в GCC, тем не менее, printf() работает и здесь без c\_str().

\myparagraph{Чуть более сложный пример}

\lstinputlisting[style=customc]{\CURPATH/STL/string/3.cpp}

\lstinputlisting[caption=MSVC 2012,style=customasmx86]{\CURPATH/STL/string/3_MSVC_RU.asm}

Собственно, компилятор не конструирует строки статически: да в общем-то и как
это возможно, если буфер с ней нужно хранить в \glslink{heap}{куче}?

Вместо этого в сегменте данных хранятся обычные \ac{ASCIIZ}-строки, а позже, во время выполнения, 
при помощи метода \q{assign}, конструируются строки s1 и s2
.
При помощи \TT{operator+}, создается строка s3.

Обратите внимание на то что вызов метода c\_str() отсутствует,
потому что его код достаточно короткий и компилятор вставил его прямо здесь:
если строка короче 16-и байт, то в регистре EAX остается указатель на буфер,
а если длиннее, то из этого же места достается адрес на буфер расположенный в \glslink{heap}{куче}.

Далее следуют вызовы трех деструкторов, причем, они вызываются только если строка длиннее 16-и байт:
тогда нужно освободить буфера в \glslink{heap}{куче}.
В противном случае, так как все три объекта std::string хранятся в стеке,
они освобождаются автоматически после выхода из функции.

Следовательно, работа с короткими строками более быстрая из-за м\'{е}ньшего обращения к \glslink{heap}{куче}.

Код на GCC даже проще (из-за того, что в GCC, как мы уже видели, не реализована возможность хранить короткую
строку прямо в структуре):

% TODO1 comment each function meaning
\lstinputlisting[caption=GCC 4.8.1,style=customasmx86]{\CURPATH/STL/string/3_GCC_RU.s}

Можно заметить, что в деструкторы передается не указатель на объект,
а указатель на место за 12 байт (или 3 слова) перед ним, то есть, на настоящее начало структуры.

\myparagraph{std::string как глобальная переменная}
\label{sec:std_string_as_global_variable}

Опытные программисты на \Cpp знают, что глобальные переменные \ac{STL}-типов вполне можно объявлять.

Да, действительно:

\lstinputlisting[style=customc]{\CURPATH/STL/string/5.cpp}

Но как и где будет вызываться конструктор \TT{std::string}?

На самом деле, эта переменная будет инициализирована даже перед началом \main.

\lstinputlisting[caption=MSVC 2012: здесь конструируется глобальная переменная{,} а также регистрируется её деструктор,style=customasmx86]{\CURPATH/STL/string/5_MSVC_p2.asm}

\lstinputlisting[caption=MSVC 2012: здесь глобальная переменная используется в \main,style=customasmx86]{\CURPATH/STL/string/5_MSVC_p1.asm}

\lstinputlisting[caption=MSVC 2012: эта функция-деструктор вызывается перед выходом,style=customasmx86]{\CURPATH/STL/string/5_MSVC_p3.asm}

\myindex{\CStandardLibrary!atexit()}
В реальности, из \ac{CRT}, еще до вызова main(), вызывается специальная функция,
в которой перечислены все конструкторы подобных переменных.
Более того: при помощи atexit() регистрируется функция, которая будет вызвана в конце работы программы:
в этой функции компилятор собирает вызовы деструкторов всех подобных глобальных переменных.

GCC работает похожим образом:

\lstinputlisting[caption=GCC 4.8.1,style=customasmx86]{\CURPATH/STL/string/5_GCC.s}

Но он не выделяет отдельной функции в которой будут собраны деструкторы: 
каждый деструктор передается в atexit() по одному.

% TODO а если глобальная STL-переменная в другом модуле? надо проверить.

}
\DE{\subsection{Einfachste XOR-Verschlüsselung überhaupt}

Ich habe einmal eine Software gesehen, bei der alle Debugging-Ausgaben mit XOR mit dem Wert 3
verschlüsselt wurden. Mit anderen Worten, die beiden niedrigsten Bits aller Buchstaben wurden invertiert.

``Hello, world'' wurde zu ``Kfool/\#tlqog'':

\begin{lstlisting}
#!/usr/bin/python

msg="Hello, world!"

print "".join(map(lambda x: chr(ord(x)^3), msg))
\end{lstlisting}

Das ist eine ziemlich interessante Verschlüsselung (oder besser eine Verschleierung),
weil sie zwei wichtige Eigenschaften hat:
1) es ist eine einzige Funktion zum Verschlüsseln und entschlüsseln, sie muss nur wiederholt angewendet werden
2) die entstehenden Buchstaben befinden sich im druckbaren Bereich, also die ganze Zeichenkette kann ohne
Escape-Symbole im Code verwendet werden.

Die zweite Eigenschaft nutzt die Tatsache, dass alle druckbaren Zeichen in Reihen organisiert sind: 0x2x-0x7x,
und wenn die beiden niederwertigsten Bits invertiert werden, wird der Buchstabe um eine oder drei Stellen nach
links oder rechts \IT{verschoben}, aber niemals in eine andere Reihe:

\begin{figure}[H]
\centering
\includegraphics[width=0.7\textwidth]{ascii_clean.png}
\caption{7-Bit \ac{ASCII} Tabelle in Emacs}
\end{figure}

\dots mit dem Zeichen 0x7F als einziger Ausnahme.

Im Folgenden werden also beispielsweise die Zeichen A-Z \IT{verschlüsselt}:

\begin{lstlisting}
#!/usr/bin/python

msg="@ABCDEFGHIJKLMNO"

print "".join(map(lambda x: chr(ord(x)^3), msg))
\end{lstlisting}

Ergebnis:
% FIXME \verb  --  relevant comment for German?
\begin{lstlisting}
CBA@GFEDKJIHONML
\end{lstlisting}

Es sieht so aus als würden die Zeichen ``@'' und ``C'' sowie ``B'' und ``A'' vertauscht werden.

Hier ist noch ein interessantes Beispiel, in dem gezeigt wird, wie die Eigenschaften von XOR
ausgenutzt werden können: Exakt den gleichen Effekt, dass druckbare Zeichen auch druckbar bleiben,
kann man dadurch erzielen, dass irgendeine Kombination der niedrigsten vier Bits invertiert wird.
}

\ifdefined\SPANISH
\chapter{Patrones de código}
\fi % SPANISH

\ifdefined\GERMAN
\chapter{Code-Muster}
\fi % GERMAN

\ifdefined\ENGLISH
\chapter{Code Patterns}
\fi % ENGLISH

\ifdefined\ITALIAN
\chapter{Forme di codice}
\fi % ITALIAN

\ifdefined\RUSSIAN
\chapter{Образцы кода}
\fi % RUSSIAN

\ifdefined\BRAZILIAN
\chapter{Padrões de códigos}
\fi % BRAZILIAN

\ifdefined\THAI
\chapter{รูปแบบของโค้ด}
\fi % THAI

\ifdefined\FRENCH
\chapter{Modèle de code}
\fi % FRENCH

\ifdefined\POLISH
\chapter{\PLph{}}
\fi % POLISH

% sections
\EN{\input{patterns/patterns_opt_dbg_EN}}
\ES{\input{patterns/patterns_opt_dbg_ES}}
\ITA{\input{patterns/patterns_opt_dbg_ITA}}
\PTBR{\input{patterns/patterns_opt_dbg_PTBR}}
\RU{\input{patterns/patterns_opt_dbg_RU}}
\THA{\input{patterns/patterns_opt_dbg_THA}}
\DE{\input{patterns/patterns_opt_dbg_DE}}
\FR{\input{patterns/patterns_opt_dbg_FR}}
\PL{\input{patterns/patterns_opt_dbg_PL}}

\RU{\section{Некоторые базовые понятия}}
\EN{\section{Some basics}}
\DE{\section{Einige Grundlagen}}
\FR{\section{Quelques bases}}
\ES{\section{\ESph{}}}
\ITA{\section{Alcune basi teoriche}}
\PTBR{\section{\PTBRph{}}}
\THA{\section{\THAph{}}}
\PL{\section{\PLph{}}}

% sections:
\EN{\input{patterns/intro_CPU_ISA_EN}}
\ES{\input{patterns/intro_CPU_ISA_ES}}
\ITA{\input{patterns/intro_CPU_ISA_ITA}}
\PTBR{\input{patterns/intro_CPU_ISA_PTBR}}
\RU{\input{patterns/intro_CPU_ISA_RU}}
\DE{\input{patterns/intro_CPU_ISA_DE}}
\FR{\input{patterns/intro_CPU_ISA_FR}}
\PL{\input{patterns/intro_CPU_ISA_PL}}

\EN{\input{patterns/numeral_EN}}
\RU{\input{patterns/numeral_RU}}
\ITA{\input{patterns/numeral_ITA}}
\DE{\input{patterns/numeral_DE}}
\FR{\input{patterns/numeral_FR}}
\PL{\input{patterns/numeral_PL}}

% chapters
\input{patterns/00_empty/main}
\input{patterns/011_ret/main}
\input{patterns/01_helloworld/main}
\input{patterns/015_prolog_epilogue/main}
\input{patterns/02_stack/main}
\input{patterns/03_printf/main}
\input{patterns/04_scanf/main}
\input{patterns/05_passing_arguments/main}
\input{patterns/06_return_results/main}
\input{patterns/061_pointers/main}
\input{patterns/065_GOTO/main}
\input{patterns/07_jcc/main}
\input{patterns/08_switch/main}
\input{patterns/09_loops/main}
\input{patterns/10_strings/main}
\input{patterns/11_arith_optimizations/main}
\input{patterns/12_FPU/main}
\input{patterns/13_arrays/main}
\input{patterns/14_bitfields/main}
\EN{\input{patterns/145_LCG/main_EN}}
\RU{\input{patterns/145_LCG/main_RU}}
\input{patterns/15_structs/main}
\input{patterns/17_unions/main}
\input{patterns/18_pointers_to_functions/main}
\input{patterns/185_64bit_in_32_env/main}

\EN{\input{patterns/19_SIMD/main_EN}}
\RU{\input{patterns/19_SIMD/main_RU}}
\DE{\input{patterns/19_SIMD/main_DE}}

\EN{\input{patterns/20_x64/main_EN}}
\RU{\input{patterns/20_x64/main_RU}}

\EN{\input{patterns/205_floating_SIMD/main_EN}}
\RU{\input{patterns/205_floating_SIMD/main_RU}}
\DE{\input{patterns/205_floating_SIMD/main_DE}}

\EN{\input{patterns/ARM/main_EN}}
\RU{\input{patterns/ARM/main_RU}}
\DE{\input{patterns/ARM/main_DE}}

\input{patterns/MIPS/main}


\ifdefined\SPANISH
\chapter{Patrones de código}
\fi % SPANISH

\ifdefined\GERMAN
\chapter{Code-Muster}
\fi % GERMAN

\ifdefined\ENGLISH
\chapter{Code Patterns}
\fi % ENGLISH

\ifdefined\ITALIAN
\chapter{Forme di codice}
\fi % ITALIAN

\ifdefined\RUSSIAN
\chapter{Образцы кода}
\fi % RUSSIAN

\ifdefined\BRAZILIAN
\chapter{Padrões de códigos}
\fi % BRAZILIAN

\ifdefined\THAI
\chapter{รูปแบบของโค้ด}
\fi % THAI

\ifdefined\FRENCH
\chapter{Modèle de code}
\fi % FRENCH

\ifdefined\POLISH
\chapter{\PLph{}}
\fi % POLISH

% sections
\EN{\section{The method}

When the author of this book first started learning C and, later, \Cpp, he used to write small pieces of code, compile them,
and then look at the assembly language output. This made it very easy for him to understand what was going on in the code that he had written.
\footnote{In fact, he still does this when he can't understand what a particular bit of code does.}.
He did this so many times that the relationship between the \CCpp code and what the compiler produced was imprinted deeply in his mind.
It's now easy for him to imagine instantly a rough outline of a C code's appearance and function.
Perhaps this technique could be helpful for others.

%There are a lot of examples for both x86/x64 and ARM.
%Those who already familiar with one of architectures, may freely skim over pages.

By the way, there is a great website where you can do the same, with various compilers, instead of installing them on your box.
You can use it as well: \url{https://gcc.godbolt.org/}.

\section*{\Exercises}

When the author of this book studied assembly language, he also often compiled small C functions and then rewrote
them gradually to assembly, trying to make their code as short as possible.
This probably is not worth doing in real-world scenarios today,
because it's hard to compete with the latest compilers in terms of efficiency. It is, however, a very good way to gain a better understanding of assembly.
Feel free, therefore, to take any assembly code from this book and try to make it shorter.
However, don't forget to test what you have written.

% rewrote to show that debug\release and optimisations levels are orthogonal concepts.
\section*{Optimization levels and debug information}

Source code can be compiled by different compilers with various optimization levels.
A typical compiler has about three such levels, where level zero means that optimization is completely disabled.
Optimization can also be targeted towards code size or code speed.
A non-optimizing compiler is faster and produces more understandable (albeit verbose) code,
whereas an optimizing compiler is slower and tries to produce code that runs faster (but is not necessarily more compact).
In addition to optimization levels, a compiler can include some debug information in the resulting file,
producing code that is easy to debug.
One of the important features of the ´debug' code is that it might contain links
between each line of the source code and its respective machine code address.
Optimizing compilers, on the other hand, tend to produce output where entire lines of source code
can be optimized away and thus not even be present in the resulting machine code.
Reverse engineers can encounter either version, simply because some developers turn on the compiler's optimization flags and others do not.
Because of this, we'll try to work on examples of both debug and release versions of the code featured in this book, wherever possible.

Sometimes some pretty ancient compilers are used in this book, in order to get the shortest (or simplest) possible code snippet.
}
\ES{\input{patterns/patterns_opt_dbg_ES}}
\ITA{\input{patterns/patterns_opt_dbg_ITA}}
\PTBR{\input{patterns/patterns_opt_dbg_PTBR}}
\RU{\input{patterns/patterns_opt_dbg_RU}}
\THA{\input{patterns/patterns_opt_dbg_THA}}
\DE{\input{patterns/patterns_opt_dbg_DE}}
\FR{\input{patterns/patterns_opt_dbg_FR}}
\PL{\input{patterns/patterns_opt_dbg_PL}}

\RU{\section{Некоторые базовые понятия}}
\EN{\section{Some basics}}
\DE{\section{Einige Grundlagen}}
\FR{\section{Quelques bases}}
\ES{\section{\ESph{}}}
\ITA{\section{Alcune basi teoriche}}
\PTBR{\section{\PTBRph{}}}
\THA{\section{\THAph{}}}
\PL{\section{\PLph{}}}

% sections:
\EN{\input{patterns/intro_CPU_ISA_EN}}
\ES{\input{patterns/intro_CPU_ISA_ES}}
\ITA{\input{patterns/intro_CPU_ISA_ITA}}
\PTBR{\input{patterns/intro_CPU_ISA_PTBR}}
\RU{\input{patterns/intro_CPU_ISA_RU}}
\DE{\input{patterns/intro_CPU_ISA_DE}}
\FR{\input{patterns/intro_CPU_ISA_FR}}
\PL{\input{patterns/intro_CPU_ISA_PL}}

\EN{\subsection{Numeral Systems}

Humans have become accustomed to a decimal numeral system, probably because almost everyone has 10 fingers.
Nevertheless, the number \q{10} has no significant meaning in science and mathematics.
The natural numeral system in digital electronics is binary: 0 is for an absence of current in the wire, and 1 for presence.
10 in binary is 2 in decimal, 100 in binary is 4 in decimal, and so on.

% This sentence is a bit unweildy - maybe try 'Our ten-digit system would be described as having a radix...' - Renaissance
If the numeral system has 10 digits, it has a \IT{radix} (or \IT{base}) of 10.
The binary numeral system has a \IT{radix} of 2.

Important things to recall:

1) A \IT{number} is a number, while a \IT{digit} is a term from writing systems, and is usually one character

% The original is 'number' is not changed; I think the intent is value, and changed it - Renaissance
2) The value of a number does not change when converted to another radix; only the writing notation for that value has changed (and therefore the way of representing it in \ac{RAM}).

\subsection{Converting From One Radix To Another}

Positional notation is used almost every numerical system. This means that a digit has weight relative to where it is placed inside of the larger number.
If 2 is placed at the rightmost place, it's 2, but if it's placed one digit before rightmost, it's 20.

What does $1234$ stand for?

$10^3 \cdot 1 + 10^2 \cdot 2 + 10^1 \cdot 3 + 1 \cdot 4 = 1234$ or
$1000 \cdot 1 + 100 \cdot 2 + 10 \cdot 3 + 4 = 1234$

It's the same story for binary numbers, but the base is 2 instead of 10.
What does 0b101011 stand for?

$2^5 \cdot 1 + 2^4 \cdot 0 + 2^3 \cdot 1 + 2^2 \cdot 0 + 2^1 \cdot 1 + 2^0 \cdot 1 = 43$ or
$32 \cdot 1 + 16 \cdot 0 + 8 \cdot 1 + 4 \cdot 0 + 2 \cdot 1 + 1 = 43$

There is such a thing as non-positional notation, such as the Roman numeral system.
\footnote{About numeric system evolution, see \InSqBrackets{\TAOCPvolII{}, 195--213.}}.
% Maybe add a sentence to fill in that X is always 10, and is therefore non-positional, even though putting an I before subtracts and after adds, and is in that sense positional
Perhaps, humankind switched to positional notation because it's easier to do basic operations (addition, multiplication, etc.) on paper by hand.

Binary numbers can be added, subtracted and so on in the very same as taught in schools, but only 2 digits are available.

Binary numbers are bulky when represented in source code and dumps, so that is where the hexadecimal numeral system can be useful.
A hexadecimal radix uses the digits 0..9, and also 6 Latin characters: A..F.
Each hexadecimal digit takes 4 bits or 4 binary digits, so it's very easy to convert from binary number to hexadecimal and back, even manually, in one's mind.

\begin{center}
\begin{longtable}{ | l | l | l | }
\hline
\HeaderColor hexadecimal & \HeaderColor binary & \HeaderColor decimal \\
\hline
0	&0000	&0 \\
1	&0001	&1 \\
2	&0010	&2 \\
3	&0011	&3 \\
4	&0100	&4 \\
5	&0101	&5 \\
6	&0110	&6 \\
7	&0111	&7 \\
8	&1000	&8 \\
9	&1001	&9 \\
A	&1010	&10 \\
B	&1011	&11 \\
C	&1100	&12 \\
D	&1101	&13 \\
E	&1110	&14 \\
F	&1111	&15 \\
\hline
\end{longtable}
\end{center}

How can one tell which radix is being used in a specific instance?

Decimal numbers are usually written as is, i.e., 1234. Some assemblers allow an identifier on decimal radix numbers, in which the number would be written with a "d" suffix: 1234d.

Binary numbers are sometimes prepended with the "0b" prefix: 0b100110111 (\ac{GCC} has a non-standard language extension for this\footnote{\url{https://gcc.gnu.org/onlinedocs/gcc/Binary-constants.html}}).
There is also another way: using a "b" suffix, for example: 100110111b.
This book tries to use the "0b" prefix consistently throughout the book for binary numbers.

Hexadecimal numbers are prepended with "0x" prefix in \CCpp and other \ac{PL}s: 0x1234ABCD.
Alternatively, they are given a "h" suffix: 1234ABCDh. This is common way of representing them in assemblers and debuggers.
In this convention, if the number is started with a Latin (A..F) digit, a 0 is added at the beginning: 0ABCDEFh.
There was also convention that was popular in 8-bit home computers era, using \$ prefix, like \$ABCD.
The book will try to stick to "0x" prefix throughout the book for hexadecimal numbers.

Should one learn to convert numbers mentally? A table of 1-digit hexadecimal numbers can easily be memorized.
As for larger numbers, it's probably not worth tormenting yourself.

Perhaps the most visible hexadecimal numbers are in \ac{URL}s.
This is the way that non-Latin characters are encoded.
For example:
\url{https://en.wiktionary.org/wiki/na\%C3\%AFvet\%C3\%A9} is the \ac{URL} of Wiktionary article about \q{naïveté} word.

\subsubsection{Octal Radix}

Another numeral system heavily used in the past of computer programming is octal. In octal there are 8 digits (0..7), and each is mapped to 3 bits, so it's easy to convert numbers back and forth.
It has been superseded by the hexadecimal system almost everywhere, but, surprisingly, there is a *NIX utility, used often by many people, which takes octal numbers as argument: \TT{chmod}.

\myindex{UNIX!chmod}
As many *NIX users know, \TT{chmod} argument can be a number of 3 digits. The first digit represents the rights of the owner of the file (read, write and/or execute), the second is the rights for the group to which the file belongs, and the third is for everyone else.
Each digit that \TT{chmod} takes can be represented in binary form:

\begin{center}
\begin{longtable}{ | l | l | l | }
\hline
\HeaderColor decimal & \HeaderColor binary & \HeaderColor meaning \\
\hline
7	&111	&\textbf{rwx} \\
6	&110	&\textbf{rw-} \\
5	&101	&\textbf{r-x} \\
4	&100	&\textbf{r-{}-} \\
3	&011	&\textbf{-wx} \\
2	&010	&\textbf{-w-} \\
1	&001	&\textbf{-{}-x} \\
0	&000	&\textbf{-{}-{}-} \\
\hline
\end{longtable}
\end{center}

So each bit is mapped to a flag: read/write/execute.

The importance of \TT{chmod} here is that the whole number in argument can be represented as octal number.
Let's take, for example, 644.
When you run \TT{chmod 644 file}, you set read/write permissions for owner, read permissions for group and again, read permissions for everyone else.
If we convert the octal number 644 to binary, it would be \TT{110100100}, or, in groups of 3 bits, \TT{110 100 100}.

Now we see that each triplet describe permissions for owner/group/others: first is \TT{rw-}, second is \TT{r--} and third is \TT{r--}.

The octal numeral system was also popular on old computers like PDP-8, because word there could be 12, 24 or 36 bits, and these numbers are all divisible by 3, so the octal system was natural in that environment.
Nowadays, all popular computers employ word/address sizes of 16, 32 or 64 bits, and these numbers are all divisible by 4, so the hexadecimal system is more natural there.

The octal numeral system is supported by all standard \CCpp compilers.
This is a source of confusion sometimes, because octal numbers are encoded with a zero prepended, for example, 0377 is 255.
Sometimes, you might make a typo and write "09" instead of 9, and the compiler would report an error.
GCC might report something like this:\\
\TT{error: invalid digit "9" in octal constant}.

Also, the octal system is somewhat popular in Java. When the IDA shows Java strings with non-printable characters,
they are encoded in the octal system instead of hexadecimal.
\myindex{JAD}
The JAD Java decompiler behaves the same way.

\subsubsection{Divisibility}

When you see a decimal number like 120, you can quickly deduce that it's divisible by 10, because the last digit is zero.
In the same way, 123400 is divisible by 100, because the two last digits are zeros.

Likewise, the hexadecimal number 0x1230 is divisible by 0x10 (or 16), 0x123000 is divisible by 0x1000 (or 4096), etc.

The binary number 0b1000101000 is divisible by 0b1000 (8), etc.

This property can often be used to quickly realize if the size of some block in memory is padded to some boundary.
For example, sections in \ac{PE} files are almost always started at addresses ending with 3 hexadecimal zeros: 0x41000, 0x10001000, etc.
The reason behind this is the fact that almost all \ac{PE} sections are padded to a boundary of 0x1000 (4096) bytes.

\subsubsection{Multi-Precision Arithmetic and Radix}

\index{RSA}
Multi-precision arithmetic can use huge numbers, and each one may be stored in several bytes.
For example, RSA keys, both public and private, span up to 4096 bits, and maybe even more.

% I'm not sure how to change this, but the normal format for quoting would be just to mention the author or book, and footnote to the full reference
In \InSqBrackets{\TAOCPvolII, 265} we find the following idea: when you store a multi-precision number in several bytes,
the whole number can be represented as having a radix of $2^8=256$, and each digit goes to the corresponding byte.
Likewise, if you store a multi-precision number in several 32-bit integer values, each digit goes to each 32-bit slot,
and you may think about this number as stored in radix of $2^{32}$.

\subsubsection{How to Pronounce Non-Decimal Numbers}

Numbers in a non-decimal base are usually pronounced by digit by digit: ``one-zero-zero-one-one-...''.
Words like ``ten'' and ``thousand'' are usually not pronounced, to prevent confusion with the decimal base system.

\subsubsection{Floating point numbers}

To distinguish floating point numbers from integers, they are usually written with ``.0'' at the end,
like $0.0$, $123.0$, etc.
}
\RU{\subsection{Представление чисел}

Люди привыкли к десятичной системе счисления вероятно потому что почти у каждого есть по 10 пальцев.
Тем не менее, число 10 не имеет особого значения в науке и математике.
Двоичная система естествена для цифровой электроники: 0 означает отсутствие тока в проводе и 1 --- его присутствие.
10 в двоичной системе это 2 в десятичной; 100 в двоичной это 4 в десятичной, итд.

Если в системе счисления есть 10 цифр, её \IT{основание} или \IT{radix} это 10.
Двоичная система имеет \IT{основание} 2.

Важные вещи, которые полезно вспомнить:
1) \IT{число} это число, в то время как \IT{цифра} это термин из системы письменности, и это обычно один символ;
2) само число не меняется, когда конвертируется из одного основания в другое: меняется способ его записи (или представления
в памяти).

Как сконвертировать число из одного основания в другое?

Позиционная нотация используется почти везде, это означает, что всякая цифра имеет свой вес, в зависимости от её расположения
внутри числа.
Если 2 расположена в самом последнем месте справа, это 2.
Если она расположена в месте перед последним, это 20.

Что означает $1234$?

$10^3 \cdot 1 + 10^2 \cdot 2 + 10^1 \cdot 3 + 1 \cdot 4$ = 1234 или
$1000 \cdot 1 + 100 \cdot 2 + 10 \cdot 3 + 4 = 1234$

Та же история и для двоичных чисел, только основание там 2 вместо 10.
Что означает 0b101011?

$2^5 \cdot 1 + 2^4 \cdot 0 + 2^3 \cdot 1 + 2^2 \cdot 0 + 2^1 \cdot 1 + 2^0 \cdot 1 = 43$ или
$32 \cdot 1 + 16 \cdot 0 + 8 \cdot 1 + 4 \cdot 0 + 2 \cdot 1 + 1 = 43$

Позиционную нотацию можно противопоставить непозиционной нотации, такой как римская система записи чисел
\footnote{Об эволюции способов записи чисел, см.также: \InSqBrackets{\TAOCPvolII{}, 195--213.}}.
Вероятно, человечество перешло на позиционную нотацию, потому что так проще работать с числами (сложение, умножение, итд)
на бумаге, в ручную.

Действительно, двоичные числа можно складывать, вычитать, итд, точно также, как этому обычно обучают в школах,
только доступны лишь 2 цифры.

Двоичные числа громоздки, когда их используют в исходных кодах и дампах, так что в этих случаях применяется шестнадцатеричная
система.
Используются цифры 0..9 и еще 6 латинских букв: A..F.
Каждая шестнадцатеричная цифра занимает 4 бита или 4 двоичных цифры, так что конвертировать из двоичной системы в
шестнадцатеричную и назад, можно легко вручную, или даже в уме.

\begin{center}
\begin{longtable}{ | l | l | l | }
\hline
\HeaderColor шестнадцатеричная & \HeaderColor двоичная & \HeaderColor десятичная \\
\hline
0	&0000	&0 \\
1	&0001	&1 \\
2	&0010	&2 \\
3	&0011	&3 \\
4	&0100	&4 \\
5	&0101	&5 \\
6	&0110	&6 \\
7	&0111	&7 \\
8	&1000	&8 \\
9	&1001	&9 \\
A	&1010	&10 \\
B	&1011	&11 \\
C	&1100	&12 \\
D	&1101	&13 \\
E	&1110	&14 \\
F	&1111	&15 \\
\hline
\end{longtable}
\end{center}

Как понять, какое основание используется в конкретном месте?

Десятичные числа обычно записываются как есть, т.е., 1234. Но некоторые ассемблеры позволяют подчеркивать
этот факт для ясности, и это число может быть дополнено суффиксом "d": 1234d.

К двоичным числам иногда спереди добавляют префикс "0b": 0b100110111
(В \ac{GCC} для этого есть нестандартное расширение языка
\footnote{\url{https://gcc.gnu.org/onlinedocs/gcc/Binary-constants.html}}).
Есть также еще один способ: суффикс "b", например: 100110111b.
В этой книге я буду пытаться придерживаться префикса "0b" для двоичных чисел.

Шестнадцатеричные числа имеют префикс "0x" в \CCpp и некоторых других \ac{PL}: 0x1234ABCD.
Либо они имеют суффикс "h": 1234ABCDh --- обычно так они представляются в ассемблерах и отладчиках.
Если число начинается с цифры A..F, перед ним добавляется 0: 0ABCDEFh.
Во времена 8-битных домашних компьютеров, был также способ записи чисел используя префикс \$, например, \$ABCD.
В книге я попытаюсь придерживаться префикса "0x" для шестнадцатеричных чисел.

Нужно ли учиться конвертировать числа в уме? Таблицу шестнадцатеричных чисел из одной цифры легко запомнить.
А запоминать б\'{о}льшие числа, наверное, не стоит.

Наверное, чаще всего шестнадцатеричные числа можно увидеть в \ac{URL}-ах.
Так кодируются буквы не из числа латинских.
Например:
\url{https://en.wiktionary.org/wiki/na\%C3\%AFvet\%C3\%A9} это \ac{URL} страницы в Wiktionary о слове \q{naïveté}.

\subsubsection{Восьмеричная система}

Еще одна система, которая в прошлом много использовалась в программировании это восьмеричная: есть 8 цифр (0..7) и каждая
описывает 3 бита, так что легко конвертировать числа туда и назад.
Она почти везде была заменена шестнадцатеричной, но удивительно, в *NIX имеется утилита использующаяся многими людьми,
которая принимает на вход восьмеричное число: \TT{chmod}.

\myindex{UNIX!chmod}
Как знают многие пользователи *NIX, аргумент \TT{chmod} это число из трех цифр. Первая цифра это права владельца файла,
вторая это права группы (которой файл принадлежит), третья для всех остальных.
И каждая цифра может быть представлена в двоичном виде:

\begin{center}
\begin{longtable}{ | l | l | l | }
\hline
\HeaderColor десятичная & \HeaderColor двоичная & \HeaderColor значение \\
\hline
7	&111	&\textbf{rwx} \\
6	&110	&\textbf{rw-} \\
5	&101	&\textbf{r-x} \\
4	&100	&\textbf{r-{}-} \\
3	&011	&\textbf{-wx} \\
2	&010	&\textbf{-w-} \\
1	&001	&\textbf{-{}-x} \\
0	&000	&\textbf{-{}-{}-} \\
\hline
\end{longtable}
\end{center}

Так что каждый бит привязан к флагу: read/write/execute (чтение/запись/исполнение).

И вот почему я вспомнил здесь о \TT{chmod}, это потому что всё число может быть представлено как число в восьмеричной системе.
Для примера возьмем 644.
Когда вы запускаете \TT{chmod 644 file}, вы выставляете права read/write для владельца, права read для группы, и снова,
read для всех остальных.
Сконвертируем число 644 из восьмеричной системы в двоичную, это будет \TT{110100100}, или (в группах по 3 бита) \TT{110 100 100}.

Теперь мы видим, что каждая тройка описывает права для владельца/группы/остальных:
первая это \TT{rw-}, вторая это \TT{r--} и третья это \TT{r--}.

Восьмеричная система была также популярная на старых компьютерах вроде PDP-8, потому что слово там могло содержать 12, 24 или
36 бит, и эти числа делятся на 3, так что выбор восьмеричной системы в той среде был логичен.
Сейчас, все популярные компьютеры имеют размер слова/адреса 16, 32 или 64 бита, и эти числа делятся на 4,
так что шестнадцатеричная система здесь удобнее.

Восьмеричная система поддерживается всеми стандартными компиляторами \CCpp{}.
Это иногда источник недоумения, потому что восьмеричные числа кодируются с нулем вперед, например, 0377 это 255.
И иногда, вы можете сделать опечатку, и написать "09" вместо 9, и компилятор выдаст ошибку.
GCC может выдать что-то вроде:\\
\TT{error: invalid digit "9" in octal constant}.

Также, восьмеричная система популярна в Java: когда IDA показывает строку с непечатаемыми символами,
они кодируются в восьмеричной системе вместо шестнадцатеричной.
\myindex{JAD}
Точно также себя ведет декомпилятор с Java JAD.

\subsubsection{Делимость}

Когда вы видите десятичное число вроде 120, вы можете быстро понять что оно делится на 10, потому что последняя цифра это 0.
Точно также, 123400 делится на 100, потому что две последних цифры это нули.

Точно также, шестнадцатеричное число 0x1230 делится на 0x10 (или 16), 0x123000 делится на 0x1000 (или 4096), итд.

Двоичное число 0b1000101000 делится на 0b1000 (8), итд.

Это свойство можно часто использовать, чтобы быстро понять,
что длина какого-либо блока в памяти выровнена по некоторой границе.
Например, секции в \ac{PE}-файлах почти всегда начинаются с адресов заканчивающихся 3 шестнадцатеричными нулями:
0x41000, 0x10001000, итд.
Причина в том, что почти все секции в \ac{PE} выровнены по границе 0x1000 (4096) байт.

\subsubsection{Арифметика произвольной точности и основание}

\index{RSA}
Арифметика произвольной точности (multi-precision arithmetic) может использовать огромные числа,
которые могут храниться в нескольких байтах.
Например, ключи RSA, и открытые и закрытые, могут занимать до 4096 бит и даже больше.

В \InSqBrackets{\TAOCPvolII, 265} можно найти такую идею: когда вы сохраняете число произвольной точности в нескольких байтах,
всё число может быть представлено как имеющую систему счисления по основанию $2^8=256$, и каждая цифра находится
в соответствующем байте.
Точно также, если вы сохраняете число произвольной точности в нескольких 32-битных целочисленных значениях,
каждая цифра отправляется в каждый 32-битный слот, и вы можете считать что это число записано в системе с основанием $2^{32}$.

\subsubsection{Произношение}

Числа в недесятичных системах счислениях обычно произносятся по одной цифре: ``один-ноль-ноль-один-один-...''.
Слова вроде ``десять'', ``тысяча'', итд, обычно не произносятся, потому что тогда можно спутать с десятичной системой.

\subsubsection{Числа с плавающей запятой}

Чтобы отличать числа с плавающей запятой от целочисленных, часто, в конце добавляют ``.0'',
например $0.0$, $123.0$, итд.

}
\ITA{\input{patterns/numeral_ITA}}
\DE{\input{patterns/numeral_DE}}
\FR{\input{patterns/numeral_FR}}
\PL{\input{patterns/numeral_PL}}

% chapters
\ifdefined\SPANISH
\chapter{Patrones de código}
\fi % SPANISH

\ifdefined\GERMAN
\chapter{Code-Muster}
\fi % GERMAN

\ifdefined\ENGLISH
\chapter{Code Patterns}
\fi % ENGLISH

\ifdefined\ITALIAN
\chapter{Forme di codice}
\fi % ITALIAN

\ifdefined\RUSSIAN
\chapter{Образцы кода}
\fi % RUSSIAN

\ifdefined\BRAZILIAN
\chapter{Padrões de códigos}
\fi % BRAZILIAN

\ifdefined\THAI
\chapter{รูปแบบของโค้ด}
\fi % THAI

\ifdefined\FRENCH
\chapter{Modèle de code}
\fi % FRENCH

\ifdefined\POLISH
\chapter{\PLph{}}
\fi % POLISH

% sections
\EN{\input{patterns/patterns_opt_dbg_EN}}
\ES{\input{patterns/patterns_opt_dbg_ES}}
\ITA{\input{patterns/patterns_opt_dbg_ITA}}
\PTBR{\input{patterns/patterns_opt_dbg_PTBR}}
\RU{\input{patterns/patterns_opt_dbg_RU}}
\THA{\input{patterns/patterns_opt_dbg_THA}}
\DE{\input{patterns/patterns_opt_dbg_DE}}
\FR{\input{patterns/patterns_opt_dbg_FR}}
\PL{\input{patterns/patterns_opt_dbg_PL}}

\RU{\section{Некоторые базовые понятия}}
\EN{\section{Some basics}}
\DE{\section{Einige Grundlagen}}
\FR{\section{Quelques bases}}
\ES{\section{\ESph{}}}
\ITA{\section{Alcune basi teoriche}}
\PTBR{\section{\PTBRph{}}}
\THA{\section{\THAph{}}}
\PL{\section{\PLph{}}}

% sections:
\EN{\input{patterns/intro_CPU_ISA_EN}}
\ES{\input{patterns/intro_CPU_ISA_ES}}
\ITA{\input{patterns/intro_CPU_ISA_ITA}}
\PTBR{\input{patterns/intro_CPU_ISA_PTBR}}
\RU{\input{patterns/intro_CPU_ISA_RU}}
\DE{\input{patterns/intro_CPU_ISA_DE}}
\FR{\input{patterns/intro_CPU_ISA_FR}}
\PL{\input{patterns/intro_CPU_ISA_PL}}

\EN{\input{patterns/numeral_EN}}
\RU{\input{patterns/numeral_RU}}
\ITA{\input{patterns/numeral_ITA}}
\DE{\input{patterns/numeral_DE}}
\FR{\input{patterns/numeral_FR}}
\PL{\input{patterns/numeral_PL}}

% chapters
\input{patterns/00_empty/main}
\input{patterns/011_ret/main}
\input{patterns/01_helloworld/main}
\input{patterns/015_prolog_epilogue/main}
\input{patterns/02_stack/main}
\input{patterns/03_printf/main}
\input{patterns/04_scanf/main}
\input{patterns/05_passing_arguments/main}
\input{patterns/06_return_results/main}
\input{patterns/061_pointers/main}
\input{patterns/065_GOTO/main}
\input{patterns/07_jcc/main}
\input{patterns/08_switch/main}
\input{patterns/09_loops/main}
\input{patterns/10_strings/main}
\input{patterns/11_arith_optimizations/main}
\input{patterns/12_FPU/main}
\input{patterns/13_arrays/main}
\input{patterns/14_bitfields/main}
\EN{\input{patterns/145_LCG/main_EN}}
\RU{\input{patterns/145_LCG/main_RU}}
\input{patterns/15_structs/main}
\input{patterns/17_unions/main}
\input{patterns/18_pointers_to_functions/main}
\input{patterns/185_64bit_in_32_env/main}

\EN{\input{patterns/19_SIMD/main_EN}}
\RU{\input{patterns/19_SIMD/main_RU}}
\DE{\input{patterns/19_SIMD/main_DE}}

\EN{\input{patterns/20_x64/main_EN}}
\RU{\input{patterns/20_x64/main_RU}}

\EN{\input{patterns/205_floating_SIMD/main_EN}}
\RU{\input{patterns/205_floating_SIMD/main_RU}}
\DE{\input{patterns/205_floating_SIMD/main_DE}}

\EN{\input{patterns/ARM/main_EN}}
\RU{\input{patterns/ARM/main_RU}}
\DE{\input{patterns/ARM/main_DE}}

\input{patterns/MIPS/main}

\ifdefined\SPANISH
\chapter{Patrones de código}
\fi % SPANISH

\ifdefined\GERMAN
\chapter{Code-Muster}
\fi % GERMAN

\ifdefined\ENGLISH
\chapter{Code Patterns}
\fi % ENGLISH

\ifdefined\ITALIAN
\chapter{Forme di codice}
\fi % ITALIAN

\ifdefined\RUSSIAN
\chapter{Образцы кода}
\fi % RUSSIAN

\ifdefined\BRAZILIAN
\chapter{Padrões de códigos}
\fi % BRAZILIAN

\ifdefined\THAI
\chapter{รูปแบบของโค้ด}
\fi % THAI

\ifdefined\FRENCH
\chapter{Modèle de code}
\fi % FRENCH

\ifdefined\POLISH
\chapter{\PLph{}}
\fi % POLISH

% sections
\EN{\input{patterns/patterns_opt_dbg_EN}}
\ES{\input{patterns/patterns_opt_dbg_ES}}
\ITA{\input{patterns/patterns_opt_dbg_ITA}}
\PTBR{\input{patterns/patterns_opt_dbg_PTBR}}
\RU{\input{patterns/patterns_opt_dbg_RU}}
\THA{\input{patterns/patterns_opt_dbg_THA}}
\DE{\input{patterns/patterns_opt_dbg_DE}}
\FR{\input{patterns/patterns_opt_dbg_FR}}
\PL{\input{patterns/patterns_opt_dbg_PL}}

\RU{\section{Некоторые базовые понятия}}
\EN{\section{Some basics}}
\DE{\section{Einige Grundlagen}}
\FR{\section{Quelques bases}}
\ES{\section{\ESph{}}}
\ITA{\section{Alcune basi teoriche}}
\PTBR{\section{\PTBRph{}}}
\THA{\section{\THAph{}}}
\PL{\section{\PLph{}}}

% sections:
\EN{\input{patterns/intro_CPU_ISA_EN}}
\ES{\input{patterns/intro_CPU_ISA_ES}}
\ITA{\input{patterns/intro_CPU_ISA_ITA}}
\PTBR{\input{patterns/intro_CPU_ISA_PTBR}}
\RU{\input{patterns/intro_CPU_ISA_RU}}
\DE{\input{patterns/intro_CPU_ISA_DE}}
\FR{\input{patterns/intro_CPU_ISA_FR}}
\PL{\input{patterns/intro_CPU_ISA_PL}}

\EN{\input{patterns/numeral_EN}}
\RU{\input{patterns/numeral_RU}}
\ITA{\input{patterns/numeral_ITA}}
\DE{\input{patterns/numeral_DE}}
\FR{\input{patterns/numeral_FR}}
\PL{\input{patterns/numeral_PL}}

% chapters
\input{patterns/00_empty/main}
\input{patterns/011_ret/main}
\input{patterns/01_helloworld/main}
\input{patterns/015_prolog_epilogue/main}
\input{patterns/02_stack/main}
\input{patterns/03_printf/main}
\input{patterns/04_scanf/main}
\input{patterns/05_passing_arguments/main}
\input{patterns/06_return_results/main}
\input{patterns/061_pointers/main}
\input{patterns/065_GOTO/main}
\input{patterns/07_jcc/main}
\input{patterns/08_switch/main}
\input{patterns/09_loops/main}
\input{patterns/10_strings/main}
\input{patterns/11_arith_optimizations/main}
\input{patterns/12_FPU/main}
\input{patterns/13_arrays/main}
\input{patterns/14_bitfields/main}
\EN{\input{patterns/145_LCG/main_EN}}
\RU{\input{patterns/145_LCG/main_RU}}
\input{patterns/15_structs/main}
\input{patterns/17_unions/main}
\input{patterns/18_pointers_to_functions/main}
\input{patterns/185_64bit_in_32_env/main}

\EN{\input{patterns/19_SIMD/main_EN}}
\RU{\input{patterns/19_SIMD/main_RU}}
\DE{\input{patterns/19_SIMD/main_DE}}

\EN{\input{patterns/20_x64/main_EN}}
\RU{\input{patterns/20_x64/main_RU}}

\EN{\input{patterns/205_floating_SIMD/main_EN}}
\RU{\input{patterns/205_floating_SIMD/main_RU}}
\DE{\input{patterns/205_floating_SIMD/main_DE}}

\EN{\input{patterns/ARM/main_EN}}
\RU{\input{patterns/ARM/main_RU}}
\DE{\input{patterns/ARM/main_DE}}

\input{patterns/MIPS/main}

\ifdefined\SPANISH
\chapter{Patrones de código}
\fi % SPANISH

\ifdefined\GERMAN
\chapter{Code-Muster}
\fi % GERMAN

\ifdefined\ENGLISH
\chapter{Code Patterns}
\fi % ENGLISH

\ifdefined\ITALIAN
\chapter{Forme di codice}
\fi % ITALIAN

\ifdefined\RUSSIAN
\chapter{Образцы кода}
\fi % RUSSIAN

\ifdefined\BRAZILIAN
\chapter{Padrões de códigos}
\fi % BRAZILIAN

\ifdefined\THAI
\chapter{รูปแบบของโค้ด}
\fi % THAI

\ifdefined\FRENCH
\chapter{Modèle de code}
\fi % FRENCH

\ifdefined\POLISH
\chapter{\PLph{}}
\fi % POLISH

% sections
\EN{\input{patterns/patterns_opt_dbg_EN}}
\ES{\input{patterns/patterns_opt_dbg_ES}}
\ITA{\input{patterns/patterns_opt_dbg_ITA}}
\PTBR{\input{patterns/patterns_opt_dbg_PTBR}}
\RU{\input{patterns/patterns_opt_dbg_RU}}
\THA{\input{patterns/patterns_opt_dbg_THA}}
\DE{\input{patterns/patterns_opt_dbg_DE}}
\FR{\input{patterns/patterns_opt_dbg_FR}}
\PL{\input{patterns/patterns_opt_dbg_PL}}

\RU{\section{Некоторые базовые понятия}}
\EN{\section{Some basics}}
\DE{\section{Einige Grundlagen}}
\FR{\section{Quelques bases}}
\ES{\section{\ESph{}}}
\ITA{\section{Alcune basi teoriche}}
\PTBR{\section{\PTBRph{}}}
\THA{\section{\THAph{}}}
\PL{\section{\PLph{}}}

% sections:
\EN{\input{patterns/intro_CPU_ISA_EN}}
\ES{\input{patterns/intro_CPU_ISA_ES}}
\ITA{\input{patterns/intro_CPU_ISA_ITA}}
\PTBR{\input{patterns/intro_CPU_ISA_PTBR}}
\RU{\input{patterns/intro_CPU_ISA_RU}}
\DE{\input{patterns/intro_CPU_ISA_DE}}
\FR{\input{patterns/intro_CPU_ISA_FR}}
\PL{\input{patterns/intro_CPU_ISA_PL}}

\EN{\input{patterns/numeral_EN}}
\RU{\input{patterns/numeral_RU}}
\ITA{\input{patterns/numeral_ITA}}
\DE{\input{patterns/numeral_DE}}
\FR{\input{patterns/numeral_FR}}
\PL{\input{patterns/numeral_PL}}

% chapters
\input{patterns/00_empty/main}
\input{patterns/011_ret/main}
\input{patterns/01_helloworld/main}
\input{patterns/015_prolog_epilogue/main}
\input{patterns/02_stack/main}
\input{patterns/03_printf/main}
\input{patterns/04_scanf/main}
\input{patterns/05_passing_arguments/main}
\input{patterns/06_return_results/main}
\input{patterns/061_pointers/main}
\input{patterns/065_GOTO/main}
\input{patterns/07_jcc/main}
\input{patterns/08_switch/main}
\input{patterns/09_loops/main}
\input{patterns/10_strings/main}
\input{patterns/11_arith_optimizations/main}
\input{patterns/12_FPU/main}
\input{patterns/13_arrays/main}
\input{patterns/14_bitfields/main}
\EN{\input{patterns/145_LCG/main_EN}}
\RU{\input{patterns/145_LCG/main_RU}}
\input{patterns/15_structs/main}
\input{patterns/17_unions/main}
\input{patterns/18_pointers_to_functions/main}
\input{patterns/185_64bit_in_32_env/main}

\EN{\input{patterns/19_SIMD/main_EN}}
\RU{\input{patterns/19_SIMD/main_RU}}
\DE{\input{patterns/19_SIMD/main_DE}}

\EN{\input{patterns/20_x64/main_EN}}
\RU{\input{patterns/20_x64/main_RU}}

\EN{\input{patterns/205_floating_SIMD/main_EN}}
\RU{\input{patterns/205_floating_SIMD/main_RU}}
\DE{\input{patterns/205_floating_SIMD/main_DE}}

\EN{\input{patterns/ARM/main_EN}}
\RU{\input{patterns/ARM/main_RU}}
\DE{\input{patterns/ARM/main_DE}}

\input{patterns/MIPS/main}

\ifdefined\SPANISH
\chapter{Patrones de código}
\fi % SPANISH

\ifdefined\GERMAN
\chapter{Code-Muster}
\fi % GERMAN

\ifdefined\ENGLISH
\chapter{Code Patterns}
\fi % ENGLISH

\ifdefined\ITALIAN
\chapter{Forme di codice}
\fi % ITALIAN

\ifdefined\RUSSIAN
\chapter{Образцы кода}
\fi % RUSSIAN

\ifdefined\BRAZILIAN
\chapter{Padrões de códigos}
\fi % BRAZILIAN

\ifdefined\THAI
\chapter{รูปแบบของโค้ด}
\fi % THAI

\ifdefined\FRENCH
\chapter{Modèle de code}
\fi % FRENCH

\ifdefined\POLISH
\chapter{\PLph{}}
\fi % POLISH

% sections
\EN{\input{patterns/patterns_opt_dbg_EN}}
\ES{\input{patterns/patterns_opt_dbg_ES}}
\ITA{\input{patterns/patterns_opt_dbg_ITA}}
\PTBR{\input{patterns/patterns_opt_dbg_PTBR}}
\RU{\input{patterns/patterns_opt_dbg_RU}}
\THA{\input{patterns/patterns_opt_dbg_THA}}
\DE{\input{patterns/patterns_opt_dbg_DE}}
\FR{\input{patterns/patterns_opt_dbg_FR}}
\PL{\input{patterns/patterns_opt_dbg_PL}}

\RU{\section{Некоторые базовые понятия}}
\EN{\section{Some basics}}
\DE{\section{Einige Grundlagen}}
\FR{\section{Quelques bases}}
\ES{\section{\ESph{}}}
\ITA{\section{Alcune basi teoriche}}
\PTBR{\section{\PTBRph{}}}
\THA{\section{\THAph{}}}
\PL{\section{\PLph{}}}

% sections:
\EN{\input{patterns/intro_CPU_ISA_EN}}
\ES{\input{patterns/intro_CPU_ISA_ES}}
\ITA{\input{patterns/intro_CPU_ISA_ITA}}
\PTBR{\input{patterns/intro_CPU_ISA_PTBR}}
\RU{\input{patterns/intro_CPU_ISA_RU}}
\DE{\input{patterns/intro_CPU_ISA_DE}}
\FR{\input{patterns/intro_CPU_ISA_FR}}
\PL{\input{patterns/intro_CPU_ISA_PL}}

\EN{\input{patterns/numeral_EN}}
\RU{\input{patterns/numeral_RU}}
\ITA{\input{patterns/numeral_ITA}}
\DE{\input{patterns/numeral_DE}}
\FR{\input{patterns/numeral_FR}}
\PL{\input{patterns/numeral_PL}}

% chapters
\input{patterns/00_empty/main}
\input{patterns/011_ret/main}
\input{patterns/01_helloworld/main}
\input{patterns/015_prolog_epilogue/main}
\input{patterns/02_stack/main}
\input{patterns/03_printf/main}
\input{patterns/04_scanf/main}
\input{patterns/05_passing_arguments/main}
\input{patterns/06_return_results/main}
\input{patterns/061_pointers/main}
\input{patterns/065_GOTO/main}
\input{patterns/07_jcc/main}
\input{patterns/08_switch/main}
\input{patterns/09_loops/main}
\input{patterns/10_strings/main}
\input{patterns/11_arith_optimizations/main}
\input{patterns/12_FPU/main}
\input{patterns/13_arrays/main}
\input{patterns/14_bitfields/main}
\EN{\input{patterns/145_LCG/main_EN}}
\RU{\input{patterns/145_LCG/main_RU}}
\input{patterns/15_structs/main}
\input{patterns/17_unions/main}
\input{patterns/18_pointers_to_functions/main}
\input{patterns/185_64bit_in_32_env/main}

\EN{\input{patterns/19_SIMD/main_EN}}
\RU{\input{patterns/19_SIMD/main_RU}}
\DE{\input{patterns/19_SIMD/main_DE}}

\EN{\input{patterns/20_x64/main_EN}}
\RU{\input{patterns/20_x64/main_RU}}

\EN{\input{patterns/205_floating_SIMD/main_EN}}
\RU{\input{patterns/205_floating_SIMD/main_RU}}
\DE{\input{patterns/205_floating_SIMD/main_DE}}

\EN{\input{patterns/ARM/main_EN}}
\RU{\input{patterns/ARM/main_RU}}
\DE{\input{patterns/ARM/main_DE}}

\input{patterns/MIPS/main}

\ifdefined\SPANISH
\chapter{Patrones de código}
\fi % SPANISH

\ifdefined\GERMAN
\chapter{Code-Muster}
\fi % GERMAN

\ifdefined\ENGLISH
\chapter{Code Patterns}
\fi % ENGLISH

\ifdefined\ITALIAN
\chapter{Forme di codice}
\fi % ITALIAN

\ifdefined\RUSSIAN
\chapter{Образцы кода}
\fi % RUSSIAN

\ifdefined\BRAZILIAN
\chapter{Padrões de códigos}
\fi % BRAZILIAN

\ifdefined\THAI
\chapter{รูปแบบของโค้ด}
\fi % THAI

\ifdefined\FRENCH
\chapter{Modèle de code}
\fi % FRENCH

\ifdefined\POLISH
\chapter{\PLph{}}
\fi % POLISH

% sections
\EN{\input{patterns/patterns_opt_dbg_EN}}
\ES{\input{patterns/patterns_opt_dbg_ES}}
\ITA{\input{patterns/patterns_opt_dbg_ITA}}
\PTBR{\input{patterns/patterns_opt_dbg_PTBR}}
\RU{\input{patterns/patterns_opt_dbg_RU}}
\THA{\input{patterns/patterns_opt_dbg_THA}}
\DE{\input{patterns/patterns_opt_dbg_DE}}
\FR{\input{patterns/patterns_opt_dbg_FR}}
\PL{\input{patterns/patterns_opt_dbg_PL}}

\RU{\section{Некоторые базовые понятия}}
\EN{\section{Some basics}}
\DE{\section{Einige Grundlagen}}
\FR{\section{Quelques bases}}
\ES{\section{\ESph{}}}
\ITA{\section{Alcune basi teoriche}}
\PTBR{\section{\PTBRph{}}}
\THA{\section{\THAph{}}}
\PL{\section{\PLph{}}}

% sections:
\EN{\input{patterns/intro_CPU_ISA_EN}}
\ES{\input{patterns/intro_CPU_ISA_ES}}
\ITA{\input{patterns/intro_CPU_ISA_ITA}}
\PTBR{\input{patterns/intro_CPU_ISA_PTBR}}
\RU{\input{patterns/intro_CPU_ISA_RU}}
\DE{\input{patterns/intro_CPU_ISA_DE}}
\FR{\input{patterns/intro_CPU_ISA_FR}}
\PL{\input{patterns/intro_CPU_ISA_PL}}

\EN{\input{patterns/numeral_EN}}
\RU{\input{patterns/numeral_RU}}
\ITA{\input{patterns/numeral_ITA}}
\DE{\input{patterns/numeral_DE}}
\FR{\input{patterns/numeral_FR}}
\PL{\input{patterns/numeral_PL}}

% chapters
\input{patterns/00_empty/main}
\input{patterns/011_ret/main}
\input{patterns/01_helloworld/main}
\input{patterns/015_prolog_epilogue/main}
\input{patterns/02_stack/main}
\input{patterns/03_printf/main}
\input{patterns/04_scanf/main}
\input{patterns/05_passing_arguments/main}
\input{patterns/06_return_results/main}
\input{patterns/061_pointers/main}
\input{patterns/065_GOTO/main}
\input{patterns/07_jcc/main}
\input{patterns/08_switch/main}
\input{patterns/09_loops/main}
\input{patterns/10_strings/main}
\input{patterns/11_arith_optimizations/main}
\input{patterns/12_FPU/main}
\input{patterns/13_arrays/main}
\input{patterns/14_bitfields/main}
\EN{\input{patterns/145_LCG/main_EN}}
\RU{\input{patterns/145_LCG/main_RU}}
\input{patterns/15_structs/main}
\input{patterns/17_unions/main}
\input{patterns/18_pointers_to_functions/main}
\input{patterns/185_64bit_in_32_env/main}

\EN{\input{patterns/19_SIMD/main_EN}}
\RU{\input{patterns/19_SIMD/main_RU}}
\DE{\input{patterns/19_SIMD/main_DE}}

\EN{\input{patterns/20_x64/main_EN}}
\RU{\input{patterns/20_x64/main_RU}}

\EN{\input{patterns/205_floating_SIMD/main_EN}}
\RU{\input{patterns/205_floating_SIMD/main_RU}}
\DE{\input{patterns/205_floating_SIMD/main_DE}}

\EN{\input{patterns/ARM/main_EN}}
\RU{\input{patterns/ARM/main_RU}}
\DE{\input{patterns/ARM/main_DE}}

\input{patterns/MIPS/main}

\ifdefined\SPANISH
\chapter{Patrones de código}
\fi % SPANISH

\ifdefined\GERMAN
\chapter{Code-Muster}
\fi % GERMAN

\ifdefined\ENGLISH
\chapter{Code Patterns}
\fi % ENGLISH

\ifdefined\ITALIAN
\chapter{Forme di codice}
\fi % ITALIAN

\ifdefined\RUSSIAN
\chapter{Образцы кода}
\fi % RUSSIAN

\ifdefined\BRAZILIAN
\chapter{Padrões de códigos}
\fi % BRAZILIAN

\ifdefined\THAI
\chapter{รูปแบบของโค้ด}
\fi % THAI

\ifdefined\FRENCH
\chapter{Modèle de code}
\fi % FRENCH

\ifdefined\POLISH
\chapter{\PLph{}}
\fi % POLISH

% sections
\EN{\input{patterns/patterns_opt_dbg_EN}}
\ES{\input{patterns/patterns_opt_dbg_ES}}
\ITA{\input{patterns/patterns_opt_dbg_ITA}}
\PTBR{\input{patterns/patterns_opt_dbg_PTBR}}
\RU{\input{patterns/patterns_opt_dbg_RU}}
\THA{\input{patterns/patterns_opt_dbg_THA}}
\DE{\input{patterns/patterns_opt_dbg_DE}}
\FR{\input{patterns/patterns_opt_dbg_FR}}
\PL{\input{patterns/patterns_opt_dbg_PL}}

\RU{\section{Некоторые базовые понятия}}
\EN{\section{Some basics}}
\DE{\section{Einige Grundlagen}}
\FR{\section{Quelques bases}}
\ES{\section{\ESph{}}}
\ITA{\section{Alcune basi teoriche}}
\PTBR{\section{\PTBRph{}}}
\THA{\section{\THAph{}}}
\PL{\section{\PLph{}}}

% sections:
\EN{\input{patterns/intro_CPU_ISA_EN}}
\ES{\input{patterns/intro_CPU_ISA_ES}}
\ITA{\input{patterns/intro_CPU_ISA_ITA}}
\PTBR{\input{patterns/intro_CPU_ISA_PTBR}}
\RU{\input{patterns/intro_CPU_ISA_RU}}
\DE{\input{patterns/intro_CPU_ISA_DE}}
\FR{\input{patterns/intro_CPU_ISA_FR}}
\PL{\input{patterns/intro_CPU_ISA_PL}}

\EN{\input{patterns/numeral_EN}}
\RU{\input{patterns/numeral_RU}}
\ITA{\input{patterns/numeral_ITA}}
\DE{\input{patterns/numeral_DE}}
\FR{\input{patterns/numeral_FR}}
\PL{\input{patterns/numeral_PL}}

% chapters
\input{patterns/00_empty/main}
\input{patterns/011_ret/main}
\input{patterns/01_helloworld/main}
\input{patterns/015_prolog_epilogue/main}
\input{patterns/02_stack/main}
\input{patterns/03_printf/main}
\input{patterns/04_scanf/main}
\input{patterns/05_passing_arguments/main}
\input{patterns/06_return_results/main}
\input{patterns/061_pointers/main}
\input{patterns/065_GOTO/main}
\input{patterns/07_jcc/main}
\input{patterns/08_switch/main}
\input{patterns/09_loops/main}
\input{patterns/10_strings/main}
\input{patterns/11_arith_optimizations/main}
\input{patterns/12_FPU/main}
\input{patterns/13_arrays/main}
\input{patterns/14_bitfields/main}
\EN{\input{patterns/145_LCG/main_EN}}
\RU{\input{patterns/145_LCG/main_RU}}
\input{patterns/15_structs/main}
\input{patterns/17_unions/main}
\input{patterns/18_pointers_to_functions/main}
\input{patterns/185_64bit_in_32_env/main}

\EN{\input{patterns/19_SIMD/main_EN}}
\RU{\input{patterns/19_SIMD/main_RU}}
\DE{\input{patterns/19_SIMD/main_DE}}

\EN{\input{patterns/20_x64/main_EN}}
\RU{\input{patterns/20_x64/main_RU}}

\EN{\input{patterns/205_floating_SIMD/main_EN}}
\RU{\input{patterns/205_floating_SIMD/main_RU}}
\DE{\input{patterns/205_floating_SIMD/main_DE}}

\EN{\input{patterns/ARM/main_EN}}
\RU{\input{patterns/ARM/main_RU}}
\DE{\input{patterns/ARM/main_DE}}

\input{patterns/MIPS/main}

\ifdefined\SPANISH
\chapter{Patrones de código}
\fi % SPANISH

\ifdefined\GERMAN
\chapter{Code-Muster}
\fi % GERMAN

\ifdefined\ENGLISH
\chapter{Code Patterns}
\fi % ENGLISH

\ifdefined\ITALIAN
\chapter{Forme di codice}
\fi % ITALIAN

\ifdefined\RUSSIAN
\chapter{Образцы кода}
\fi % RUSSIAN

\ifdefined\BRAZILIAN
\chapter{Padrões de códigos}
\fi % BRAZILIAN

\ifdefined\THAI
\chapter{รูปแบบของโค้ด}
\fi % THAI

\ifdefined\FRENCH
\chapter{Modèle de code}
\fi % FRENCH

\ifdefined\POLISH
\chapter{\PLph{}}
\fi % POLISH

% sections
\EN{\input{patterns/patterns_opt_dbg_EN}}
\ES{\input{patterns/patterns_opt_dbg_ES}}
\ITA{\input{patterns/patterns_opt_dbg_ITA}}
\PTBR{\input{patterns/patterns_opt_dbg_PTBR}}
\RU{\input{patterns/patterns_opt_dbg_RU}}
\THA{\input{patterns/patterns_opt_dbg_THA}}
\DE{\input{patterns/patterns_opt_dbg_DE}}
\FR{\input{patterns/patterns_opt_dbg_FR}}
\PL{\input{patterns/patterns_opt_dbg_PL}}

\RU{\section{Некоторые базовые понятия}}
\EN{\section{Some basics}}
\DE{\section{Einige Grundlagen}}
\FR{\section{Quelques bases}}
\ES{\section{\ESph{}}}
\ITA{\section{Alcune basi teoriche}}
\PTBR{\section{\PTBRph{}}}
\THA{\section{\THAph{}}}
\PL{\section{\PLph{}}}

% sections:
\EN{\input{patterns/intro_CPU_ISA_EN}}
\ES{\input{patterns/intro_CPU_ISA_ES}}
\ITA{\input{patterns/intro_CPU_ISA_ITA}}
\PTBR{\input{patterns/intro_CPU_ISA_PTBR}}
\RU{\input{patterns/intro_CPU_ISA_RU}}
\DE{\input{patterns/intro_CPU_ISA_DE}}
\FR{\input{patterns/intro_CPU_ISA_FR}}
\PL{\input{patterns/intro_CPU_ISA_PL}}

\EN{\input{patterns/numeral_EN}}
\RU{\input{patterns/numeral_RU}}
\ITA{\input{patterns/numeral_ITA}}
\DE{\input{patterns/numeral_DE}}
\FR{\input{patterns/numeral_FR}}
\PL{\input{patterns/numeral_PL}}

% chapters
\input{patterns/00_empty/main}
\input{patterns/011_ret/main}
\input{patterns/01_helloworld/main}
\input{patterns/015_prolog_epilogue/main}
\input{patterns/02_stack/main}
\input{patterns/03_printf/main}
\input{patterns/04_scanf/main}
\input{patterns/05_passing_arguments/main}
\input{patterns/06_return_results/main}
\input{patterns/061_pointers/main}
\input{patterns/065_GOTO/main}
\input{patterns/07_jcc/main}
\input{patterns/08_switch/main}
\input{patterns/09_loops/main}
\input{patterns/10_strings/main}
\input{patterns/11_arith_optimizations/main}
\input{patterns/12_FPU/main}
\input{patterns/13_arrays/main}
\input{patterns/14_bitfields/main}
\EN{\input{patterns/145_LCG/main_EN}}
\RU{\input{patterns/145_LCG/main_RU}}
\input{patterns/15_structs/main}
\input{patterns/17_unions/main}
\input{patterns/18_pointers_to_functions/main}
\input{patterns/185_64bit_in_32_env/main}

\EN{\input{patterns/19_SIMD/main_EN}}
\RU{\input{patterns/19_SIMD/main_RU}}
\DE{\input{patterns/19_SIMD/main_DE}}

\EN{\input{patterns/20_x64/main_EN}}
\RU{\input{patterns/20_x64/main_RU}}

\EN{\input{patterns/205_floating_SIMD/main_EN}}
\RU{\input{patterns/205_floating_SIMD/main_RU}}
\DE{\input{patterns/205_floating_SIMD/main_DE}}

\EN{\input{patterns/ARM/main_EN}}
\RU{\input{patterns/ARM/main_RU}}
\DE{\input{patterns/ARM/main_DE}}

\input{patterns/MIPS/main}

\ifdefined\SPANISH
\chapter{Patrones de código}
\fi % SPANISH

\ifdefined\GERMAN
\chapter{Code-Muster}
\fi % GERMAN

\ifdefined\ENGLISH
\chapter{Code Patterns}
\fi % ENGLISH

\ifdefined\ITALIAN
\chapter{Forme di codice}
\fi % ITALIAN

\ifdefined\RUSSIAN
\chapter{Образцы кода}
\fi % RUSSIAN

\ifdefined\BRAZILIAN
\chapter{Padrões de códigos}
\fi % BRAZILIAN

\ifdefined\THAI
\chapter{รูปแบบของโค้ด}
\fi % THAI

\ifdefined\FRENCH
\chapter{Modèle de code}
\fi % FRENCH

\ifdefined\POLISH
\chapter{\PLph{}}
\fi % POLISH

% sections
\EN{\input{patterns/patterns_opt_dbg_EN}}
\ES{\input{patterns/patterns_opt_dbg_ES}}
\ITA{\input{patterns/patterns_opt_dbg_ITA}}
\PTBR{\input{patterns/patterns_opt_dbg_PTBR}}
\RU{\input{patterns/patterns_opt_dbg_RU}}
\THA{\input{patterns/patterns_opt_dbg_THA}}
\DE{\input{patterns/patterns_opt_dbg_DE}}
\FR{\input{patterns/patterns_opt_dbg_FR}}
\PL{\input{patterns/patterns_opt_dbg_PL}}

\RU{\section{Некоторые базовые понятия}}
\EN{\section{Some basics}}
\DE{\section{Einige Grundlagen}}
\FR{\section{Quelques bases}}
\ES{\section{\ESph{}}}
\ITA{\section{Alcune basi teoriche}}
\PTBR{\section{\PTBRph{}}}
\THA{\section{\THAph{}}}
\PL{\section{\PLph{}}}

% sections:
\EN{\input{patterns/intro_CPU_ISA_EN}}
\ES{\input{patterns/intro_CPU_ISA_ES}}
\ITA{\input{patterns/intro_CPU_ISA_ITA}}
\PTBR{\input{patterns/intro_CPU_ISA_PTBR}}
\RU{\input{patterns/intro_CPU_ISA_RU}}
\DE{\input{patterns/intro_CPU_ISA_DE}}
\FR{\input{patterns/intro_CPU_ISA_FR}}
\PL{\input{patterns/intro_CPU_ISA_PL}}

\EN{\input{patterns/numeral_EN}}
\RU{\input{patterns/numeral_RU}}
\ITA{\input{patterns/numeral_ITA}}
\DE{\input{patterns/numeral_DE}}
\FR{\input{patterns/numeral_FR}}
\PL{\input{patterns/numeral_PL}}

% chapters
\input{patterns/00_empty/main}
\input{patterns/011_ret/main}
\input{patterns/01_helloworld/main}
\input{patterns/015_prolog_epilogue/main}
\input{patterns/02_stack/main}
\input{patterns/03_printf/main}
\input{patterns/04_scanf/main}
\input{patterns/05_passing_arguments/main}
\input{patterns/06_return_results/main}
\input{patterns/061_pointers/main}
\input{patterns/065_GOTO/main}
\input{patterns/07_jcc/main}
\input{patterns/08_switch/main}
\input{patterns/09_loops/main}
\input{patterns/10_strings/main}
\input{patterns/11_arith_optimizations/main}
\input{patterns/12_FPU/main}
\input{patterns/13_arrays/main}
\input{patterns/14_bitfields/main}
\EN{\input{patterns/145_LCG/main_EN}}
\RU{\input{patterns/145_LCG/main_RU}}
\input{patterns/15_structs/main}
\input{patterns/17_unions/main}
\input{patterns/18_pointers_to_functions/main}
\input{patterns/185_64bit_in_32_env/main}

\EN{\input{patterns/19_SIMD/main_EN}}
\RU{\input{patterns/19_SIMD/main_RU}}
\DE{\input{patterns/19_SIMD/main_DE}}

\EN{\input{patterns/20_x64/main_EN}}
\RU{\input{patterns/20_x64/main_RU}}

\EN{\input{patterns/205_floating_SIMD/main_EN}}
\RU{\input{patterns/205_floating_SIMD/main_RU}}
\DE{\input{patterns/205_floating_SIMD/main_DE}}

\EN{\input{patterns/ARM/main_EN}}
\RU{\input{patterns/ARM/main_RU}}
\DE{\input{patterns/ARM/main_DE}}

\input{patterns/MIPS/main}

\ifdefined\SPANISH
\chapter{Patrones de código}
\fi % SPANISH

\ifdefined\GERMAN
\chapter{Code-Muster}
\fi % GERMAN

\ifdefined\ENGLISH
\chapter{Code Patterns}
\fi % ENGLISH

\ifdefined\ITALIAN
\chapter{Forme di codice}
\fi % ITALIAN

\ifdefined\RUSSIAN
\chapter{Образцы кода}
\fi % RUSSIAN

\ifdefined\BRAZILIAN
\chapter{Padrões de códigos}
\fi % BRAZILIAN

\ifdefined\THAI
\chapter{รูปแบบของโค้ด}
\fi % THAI

\ifdefined\FRENCH
\chapter{Modèle de code}
\fi % FRENCH

\ifdefined\POLISH
\chapter{\PLph{}}
\fi % POLISH

% sections
\EN{\input{patterns/patterns_opt_dbg_EN}}
\ES{\input{patterns/patterns_opt_dbg_ES}}
\ITA{\input{patterns/patterns_opt_dbg_ITA}}
\PTBR{\input{patterns/patterns_opt_dbg_PTBR}}
\RU{\input{patterns/patterns_opt_dbg_RU}}
\THA{\input{patterns/patterns_opt_dbg_THA}}
\DE{\input{patterns/patterns_opt_dbg_DE}}
\FR{\input{patterns/patterns_opt_dbg_FR}}
\PL{\input{patterns/patterns_opt_dbg_PL}}

\RU{\section{Некоторые базовые понятия}}
\EN{\section{Some basics}}
\DE{\section{Einige Grundlagen}}
\FR{\section{Quelques bases}}
\ES{\section{\ESph{}}}
\ITA{\section{Alcune basi teoriche}}
\PTBR{\section{\PTBRph{}}}
\THA{\section{\THAph{}}}
\PL{\section{\PLph{}}}

% sections:
\EN{\input{patterns/intro_CPU_ISA_EN}}
\ES{\input{patterns/intro_CPU_ISA_ES}}
\ITA{\input{patterns/intro_CPU_ISA_ITA}}
\PTBR{\input{patterns/intro_CPU_ISA_PTBR}}
\RU{\input{patterns/intro_CPU_ISA_RU}}
\DE{\input{patterns/intro_CPU_ISA_DE}}
\FR{\input{patterns/intro_CPU_ISA_FR}}
\PL{\input{patterns/intro_CPU_ISA_PL}}

\EN{\input{patterns/numeral_EN}}
\RU{\input{patterns/numeral_RU}}
\ITA{\input{patterns/numeral_ITA}}
\DE{\input{patterns/numeral_DE}}
\FR{\input{patterns/numeral_FR}}
\PL{\input{patterns/numeral_PL}}

% chapters
\input{patterns/00_empty/main}
\input{patterns/011_ret/main}
\input{patterns/01_helloworld/main}
\input{patterns/015_prolog_epilogue/main}
\input{patterns/02_stack/main}
\input{patterns/03_printf/main}
\input{patterns/04_scanf/main}
\input{patterns/05_passing_arguments/main}
\input{patterns/06_return_results/main}
\input{patterns/061_pointers/main}
\input{patterns/065_GOTO/main}
\input{patterns/07_jcc/main}
\input{patterns/08_switch/main}
\input{patterns/09_loops/main}
\input{patterns/10_strings/main}
\input{patterns/11_arith_optimizations/main}
\input{patterns/12_FPU/main}
\input{patterns/13_arrays/main}
\input{patterns/14_bitfields/main}
\EN{\input{patterns/145_LCG/main_EN}}
\RU{\input{patterns/145_LCG/main_RU}}
\input{patterns/15_structs/main}
\input{patterns/17_unions/main}
\input{patterns/18_pointers_to_functions/main}
\input{patterns/185_64bit_in_32_env/main}

\EN{\input{patterns/19_SIMD/main_EN}}
\RU{\input{patterns/19_SIMD/main_RU}}
\DE{\input{patterns/19_SIMD/main_DE}}

\EN{\input{patterns/20_x64/main_EN}}
\RU{\input{patterns/20_x64/main_RU}}

\EN{\input{patterns/205_floating_SIMD/main_EN}}
\RU{\input{patterns/205_floating_SIMD/main_RU}}
\DE{\input{patterns/205_floating_SIMD/main_DE}}

\EN{\input{patterns/ARM/main_EN}}
\RU{\input{patterns/ARM/main_RU}}
\DE{\input{patterns/ARM/main_DE}}

\input{patterns/MIPS/main}

\ifdefined\SPANISH
\chapter{Patrones de código}
\fi % SPANISH

\ifdefined\GERMAN
\chapter{Code-Muster}
\fi % GERMAN

\ifdefined\ENGLISH
\chapter{Code Patterns}
\fi % ENGLISH

\ifdefined\ITALIAN
\chapter{Forme di codice}
\fi % ITALIAN

\ifdefined\RUSSIAN
\chapter{Образцы кода}
\fi % RUSSIAN

\ifdefined\BRAZILIAN
\chapter{Padrões de códigos}
\fi % BRAZILIAN

\ifdefined\THAI
\chapter{รูปแบบของโค้ด}
\fi % THAI

\ifdefined\FRENCH
\chapter{Modèle de code}
\fi % FRENCH

\ifdefined\POLISH
\chapter{\PLph{}}
\fi % POLISH

% sections
\EN{\input{patterns/patterns_opt_dbg_EN}}
\ES{\input{patterns/patterns_opt_dbg_ES}}
\ITA{\input{patterns/patterns_opt_dbg_ITA}}
\PTBR{\input{patterns/patterns_opt_dbg_PTBR}}
\RU{\input{patterns/patterns_opt_dbg_RU}}
\THA{\input{patterns/patterns_opt_dbg_THA}}
\DE{\input{patterns/patterns_opt_dbg_DE}}
\FR{\input{patterns/patterns_opt_dbg_FR}}
\PL{\input{patterns/patterns_opt_dbg_PL}}

\RU{\section{Некоторые базовые понятия}}
\EN{\section{Some basics}}
\DE{\section{Einige Grundlagen}}
\FR{\section{Quelques bases}}
\ES{\section{\ESph{}}}
\ITA{\section{Alcune basi teoriche}}
\PTBR{\section{\PTBRph{}}}
\THA{\section{\THAph{}}}
\PL{\section{\PLph{}}}

% sections:
\EN{\input{patterns/intro_CPU_ISA_EN}}
\ES{\input{patterns/intro_CPU_ISA_ES}}
\ITA{\input{patterns/intro_CPU_ISA_ITA}}
\PTBR{\input{patterns/intro_CPU_ISA_PTBR}}
\RU{\input{patterns/intro_CPU_ISA_RU}}
\DE{\input{patterns/intro_CPU_ISA_DE}}
\FR{\input{patterns/intro_CPU_ISA_FR}}
\PL{\input{patterns/intro_CPU_ISA_PL}}

\EN{\input{patterns/numeral_EN}}
\RU{\input{patterns/numeral_RU}}
\ITA{\input{patterns/numeral_ITA}}
\DE{\input{patterns/numeral_DE}}
\FR{\input{patterns/numeral_FR}}
\PL{\input{patterns/numeral_PL}}

% chapters
\input{patterns/00_empty/main}
\input{patterns/011_ret/main}
\input{patterns/01_helloworld/main}
\input{patterns/015_prolog_epilogue/main}
\input{patterns/02_stack/main}
\input{patterns/03_printf/main}
\input{patterns/04_scanf/main}
\input{patterns/05_passing_arguments/main}
\input{patterns/06_return_results/main}
\input{patterns/061_pointers/main}
\input{patterns/065_GOTO/main}
\input{patterns/07_jcc/main}
\input{patterns/08_switch/main}
\input{patterns/09_loops/main}
\input{patterns/10_strings/main}
\input{patterns/11_arith_optimizations/main}
\input{patterns/12_FPU/main}
\input{patterns/13_arrays/main}
\input{patterns/14_bitfields/main}
\EN{\input{patterns/145_LCG/main_EN}}
\RU{\input{patterns/145_LCG/main_RU}}
\input{patterns/15_structs/main}
\input{patterns/17_unions/main}
\input{patterns/18_pointers_to_functions/main}
\input{patterns/185_64bit_in_32_env/main}

\EN{\input{patterns/19_SIMD/main_EN}}
\RU{\input{patterns/19_SIMD/main_RU}}
\DE{\input{patterns/19_SIMD/main_DE}}

\EN{\input{patterns/20_x64/main_EN}}
\RU{\input{patterns/20_x64/main_RU}}

\EN{\input{patterns/205_floating_SIMD/main_EN}}
\RU{\input{patterns/205_floating_SIMD/main_RU}}
\DE{\input{patterns/205_floating_SIMD/main_DE}}

\EN{\input{patterns/ARM/main_EN}}
\RU{\input{patterns/ARM/main_RU}}
\DE{\input{patterns/ARM/main_DE}}

\input{patterns/MIPS/main}

\ifdefined\SPANISH
\chapter{Patrones de código}
\fi % SPANISH

\ifdefined\GERMAN
\chapter{Code-Muster}
\fi % GERMAN

\ifdefined\ENGLISH
\chapter{Code Patterns}
\fi % ENGLISH

\ifdefined\ITALIAN
\chapter{Forme di codice}
\fi % ITALIAN

\ifdefined\RUSSIAN
\chapter{Образцы кода}
\fi % RUSSIAN

\ifdefined\BRAZILIAN
\chapter{Padrões de códigos}
\fi % BRAZILIAN

\ifdefined\THAI
\chapter{รูปแบบของโค้ด}
\fi % THAI

\ifdefined\FRENCH
\chapter{Modèle de code}
\fi % FRENCH

\ifdefined\POLISH
\chapter{\PLph{}}
\fi % POLISH

% sections
\EN{\input{patterns/patterns_opt_dbg_EN}}
\ES{\input{patterns/patterns_opt_dbg_ES}}
\ITA{\input{patterns/patterns_opt_dbg_ITA}}
\PTBR{\input{patterns/patterns_opt_dbg_PTBR}}
\RU{\input{patterns/patterns_opt_dbg_RU}}
\THA{\input{patterns/patterns_opt_dbg_THA}}
\DE{\input{patterns/patterns_opt_dbg_DE}}
\FR{\input{patterns/patterns_opt_dbg_FR}}
\PL{\input{patterns/patterns_opt_dbg_PL}}

\RU{\section{Некоторые базовые понятия}}
\EN{\section{Some basics}}
\DE{\section{Einige Grundlagen}}
\FR{\section{Quelques bases}}
\ES{\section{\ESph{}}}
\ITA{\section{Alcune basi teoriche}}
\PTBR{\section{\PTBRph{}}}
\THA{\section{\THAph{}}}
\PL{\section{\PLph{}}}

% sections:
\EN{\input{patterns/intro_CPU_ISA_EN}}
\ES{\input{patterns/intro_CPU_ISA_ES}}
\ITA{\input{patterns/intro_CPU_ISA_ITA}}
\PTBR{\input{patterns/intro_CPU_ISA_PTBR}}
\RU{\input{patterns/intro_CPU_ISA_RU}}
\DE{\input{patterns/intro_CPU_ISA_DE}}
\FR{\input{patterns/intro_CPU_ISA_FR}}
\PL{\input{patterns/intro_CPU_ISA_PL}}

\EN{\input{patterns/numeral_EN}}
\RU{\input{patterns/numeral_RU}}
\ITA{\input{patterns/numeral_ITA}}
\DE{\input{patterns/numeral_DE}}
\FR{\input{patterns/numeral_FR}}
\PL{\input{patterns/numeral_PL}}

% chapters
\input{patterns/00_empty/main}
\input{patterns/011_ret/main}
\input{patterns/01_helloworld/main}
\input{patterns/015_prolog_epilogue/main}
\input{patterns/02_stack/main}
\input{patterns/03_printf/main}
\input{patterns/04_scanf/main}
\input{patterns/05_passing_arguments/main}
\input{patterns/06_return_results/main}
\input{patterns/061_pointers/main}
\input{patterns/065_GOTO/main}
\input{patterns/07_jcc/main}
\input{patterns/08_switch/main}
\input{patterns/09_loops/main}
\input{patterns/10_strings/main}
\input{patterns/11_arith_optimizations/main}
\input{patterns/12_FPU/main}
\input{patterns/13_arrays/main}
\input{patterns/14_bitfields/main}
\EN{\input{patterns/145_LCG/main_EN}}
\RU{\input{patterns/145_LCG/main_RU}}
\input{patterns/15_structs/main}
\input{patterns/17_unions/main}
\input{patterns/18_pointers_to_functions/main}
\input{patterns/185_64bit_in_32_env/main}

\EN{\input{patterns/19_SIMD/main_EN}}
\RU{\input{patterns/19_SIMD/main_RU}}
\DE{\input{patterns/19_SIMD/main_DE}}

\EN{\input{patterns/20_x64/main_EN}}
\RU{\input{patterns/20_x64/main_RU}}

\EN{\input{patterns/205_floating_SIMD/main_EN}}
\RU{\input{patterns/205_floating_SIMD/main_RU}}
\DE{\input{patterns/205_floating_SIMD/main_DE}}

\EN{\input{patterns/ARM/main_EN}}
\RU{\input{patterns/ARM/main_RU}}
\DE{\input{patterns/ARM/main_DE}}

\input{patterns/MIPS/main}

\ifdefined\SPANISH
\chapter{Patrones de código}
\fi % SPANISH

\ifdefined\GERMAN
\chapter{Code-Muster}
\fi % GERMAN

\ifdefined\ENGLISH
\chapter{Code Patterns}
\fi % ENGLISH

\ifdefined\ITALIAN
\chapter{Forme di codice}
\fi % ITALIAN

\ifdefined\RUSSIAN
\chapter{Образцы кода}
\fi % RUSSIAN

\ifdefined\BRAZILIAN
\chapter{Padrões de códigos}
\fi % BRAZILIAN

\ifdefined\THAI
\chapter{รูปแบบของโค้ด}
\fi % THAI

\ifdefined\FRENCH
\chapter{Modèle de code}
\fi % FRENCH

\ifdefined\POLISH
\chapter{\PLph{}}
\fi % POLISH

% sections
\EN{\input{patterns/patterns_opt_dbg_EN}}
\ES{\input{patterns/patterns_opt_dbg_ES}}
\ITA{\input{patterns/patterns_opt_dbg_ITA}}
\PTBR{\input{patterns/patterns_opt_dbg_PTBR}}
\RU{\input{patterns/patterns_opt_dbg_RU}}
\THA{\input{patterns/patterns_opt_dbg_THA}}
\DE{\input{patterns/patterns_opt_dbg_DE}}
\FR{\input{patterns/patterns_opt_dbg_FR}}
\PL{\input{patterns/patterns_opt_dbg_PL}}

\RU{\section{Некоторые базовые понятия}}
\EN{\section{Some basics}}
\DE{\section{Einige Grundlagen}}
\FR{\section{Quelques bases}}
\ES{\section{\ESph{}}}
\ITA{\section{Alcune basi teoriche}}
\PTBR{\section{\PTBRph{}}}
\THA{\section{\THAph{}}}
\PL{\section{\PLph{}}}

% sections:
\EN{\input{patterns/intro_CPU_ISA_EN}}
\ES{\input{patterns/intro_CPU_ISA_ES}}
\ITA{\input{patterns/intro_CPU_ISA_ITA}}
\PTBR{\input{patterns/intro_CPU_ISA_PTBR}}
\RU{\input{patterns/intro_CPU_ISA_RU}}
\DE{\input{patterns/intro_CPU_ISA_DE}}
\FR{\input{patterns/intro_CPU_ISA_FR}}
\PL{\input{patterns/intro_CPU_ISA_PL}}

\EN{\input{patterns/numeral_EN}}
\RU{\input{patterns/numeral_RU}}
\ITA{\input{patterns/numeral_ITA}}
\DE{\input{patterns/numeral_DE}}
\FR{\input{patterns/numeral_FR}}
\PL{\input{patterns/numeral_PL}}

% chapters
\input{patterns/00_empty/main}
\input{patterns/011_ret/main}
\input{patterns/01_helloworld/main}
\input{patterns/015_prolog_epilogue/main}
\input{patterns/02_stack/main}
\input{patterns/03_printf/main}
\input{patterns/04_scanf/main}
\input{patterns/05_passing_arguments/main}
\input{patterns/06_return_results/main}
\input{patterns/061_pointers/main}
\input{patterns/065_GOTO/main}
\input{patterns/07_jcc/main}
\input{patterns/08_switch/main}
\input{patterns/09_loops/main}
\input{patterns/10_strings/main}
\input{patterns/11_arith_optimizations/main}
\input{patterns/12_FPU/main}
\input{patterns/13_arrays/main}
\input{patterns/14_bitfields/main}
\EN{\input{patterns/145_LCG/main_EN}}
\RU{\input{patterns/145_LCG/main_RU}}
\input{patterns/15_structs/main}
\input{patterns/17_unions/main}
\input{patterns/18_pointers_to_functions/main}
\input{patterns/185_64bit_in_32_env/main}

\EN{\input{patterns/19_SIMD/main_EN}}
\RU{\input{patterns/19_SIMD/main_RU}}
\DE{\input{patterns/19_SIMD/main_DE}}

\EN{\input{patterns/20_x64/main_EN}}
\RU{\input{patterns/20_x64/main_RU}}

\EN{\input{patterns/205_floating_SIMD/main_EN}}
\RU{\input{patterns/205_floating_SIMD/main_RU}}
\DE{\input{patterns/205_floating_SIMD/main_DE}}

\EN{\input{patterns/ARM/main_EN}}
\RU{\input{patterns/ARM/main_RU}}
\DE{\input{patterns/ARM/main_DE}}

\input{patterns/MIPS/main}

\ifdefined\SPANISH
\chapter{Patrones de código}
\fi % SPANISH

\ifdefined\GERMAN
\chapter{Code-Muster}
\fi % GERMAN

\ifdefined\ENGLISH
\chapter{Code Patterns}
\fi % ENGLISH

\ifdefined\ITALIAN
\chapter{Forme di codice}
\fi % ITALIAN

\ifdefined\RUSSIAN
\chapter{Образцы кода}
\fi % RUSSIAN

\ifdefined\BRAZILIAN
\chapter{Padrões de códigos}
\fi % BRAZILIAN

\ifdefined\THAI
\chapter{รูปแบบของโค้ด}
\fi % THAI

\ifdefined\FRENCH
\chapter{Modèle de code}
\fi % FRENCH

\ifdefined\POLISH
\chapter{\PLph{}}
\fi % POLISH

% sections
\EN{\input{patterns/patterns_opt_dbg_EN}}
\ES{\input{patterns/patterns_opt_dbg_ES}}
\ITA{\input{patterns/patterns_opt_dbg_ITA}}
\PTBR{\input{patterns/patterns_opt_dbg_PTBR}}
\RU{\input{patterns/patterns_opt_dbg_RU}}
\THA{\input{patterns/patterns_opt_dbg_THA}}
\DE{\input{patterns/patterns_opt_dbg_DE}}
\FR{\input{patterns/patterns_opt_dbg_FR}}
\PL{\input{patterns/patterns_opt_dbg_PL}}

\RU{\section{Некоторые базовые понятия}}
\EN{\section{Some basics}}
\DE{\section{Einige Grundlagen}}
\FR{\section{Quelques bases}}
\ES{\section{\ESph{}}}
\ITA{\section{Alcune basi teoriche}}
\PTBR{\section{\PTBRph{}}}
\THA{\section{\THAph{}}}
\PL{\section{\PLph{}}}

% sections:
\EN{\input{patterns/intro_CPU_ISA_EN}}
\ES{\input{patterns/intro_CPU_ISA_ES}}
\ITA{\input{patterns/intro_CPU_ISA_ITA}}
\PTBR{\input{patterns/intro_CPU_ISA_PTBR}}
\RU{\input{patterns/intro_CPU_ISA_RU}}
\DE{\input{patterns/intro_CPU_ISA_DE}}
\FR{\input{patterns/intro_CPU_ISA_FR}}
\PL{\input{patterns/intro_CPU_ISA_PL}}

\EN{\input{patterns/numeral_EN}}
\RU{\input{patterns/numeral_RU}}
\ITA{\input{patterns/numeral_ITA}}
\DE{\input{patterns/numeral_DE}}
\FR{\input{patterns/numeral_FR}}
\PL{\input{patterns/numeral_PL}}

% chapters
\input{patterns/00_empty/main}
\input{patterns/011_ret/main}
\input{patterns/01_helloworld/main}
\input{patterns/015_prolog_epilogue/main}
\input{patterns/02_stack/main}
\input{patterns/03_printf/main}
\input{patterns/04_scanf/main}
\input{patterns/05_passing_arguments/main}
\input{patterns/06_return_results/main}
\input{patterns/061_pointers/main}
\input{patterns/065_GOTO/main}
\input{patterns/07_jcc/main}
\input{patterns/08_switch/main}
\input{patterns/09_loops/main}
\input{patterns/10_strings/main}
\input{patterns/11_arith_optimizations/main}
\input{patterns/12_FPU/main}
\input{patterns/13_arrays/main}
\input{patterns/14_bitfields/main}
\EN{\input{patterns/145_LCG/main_EN}}
\RU{\input{patterns/145_LCG/main_RU}}
\input{patterns/15_structs/main}
\input{patterns/17_unions/main}
\input{patterns/18_pointers_to_functions/main}
\input{patterns/185_64bit_in_32_env/main}

\EN{\input{patterns/19_SIMD/main_EN}}
\RU{\input{patterns/19_SIMD/main_RU}}
\DE{\input{patterns/19_SIMD/main_DE}}

\EN{\input{patterns/20_x64/main_EN}}
\RU{\input{patterns/20_x64/main_RU}}

\EN{\input{patterns/205_floating_SIMD/main_EN}}
\RU{\input{patterns/205_floating_SIMD/main_RU}}
\DE{\input{patterns/205_floating_SIMD/main_DE}}

\EN{\input{patterns/ARM/main_EN}}
\RU{\input{patterns/ARM/main_RU}}
\DE{\input{patterns/ARM/main_DE}}

\input{patterns/MIPS/main}

\ifdefined\SPANISH
\chapter{Patrones de código}
\fi % SPANISH

\ifdefined\GERMAN
\chapter{Code-Muster}
\fi % GERMAN

\ifdefined\ENGLISH
\chapter{Code Patterns}
\fi % ENGLISH

\ifdefined\ITALIAN
\chapter{Forme di codice}
\fi % ITALIAN

\ifdefined\RUSSIAN
\chapter{Образцы кода}
\fi % RUSSIAN

\ifdefined\BRAZILIAN
\chapter{Padrões de códigos}
\fi % BRAZILIAN

\ifdefined\THAI
\chapter{รูปแบบของโค้ด}
\fi % THAI

\ifdefined\FRENCH
\chapter{Modèle de code}
\fi % FRENCH

\ifdefined\POLISH
\chapter{\PLph{}}
\fi % POLISH

% sections
\EN{\input{patterns/patterns_opt_dbg_EN}}
\ES{\input{patterns/patterns_opt_dbg_ES}}
\ITA{\input{patterns/patterns_opt_dbg_ITA}}
\PTBR{\input{patterns/patterns_opt_dbg_PTBR}}
\RU{\input{patterns/patterns_opt_dbg_RU}}
\THA{\input{patterns/patterns_opt_dbg_THA}}
\DE{\input{patterns/patterns_opt_dbg_DE}}
\FR{\input{patterns/patterns_opt_dbg_FR}}
\PL{\input{patterns/patterns_opt_dbg_PL}}

\RU{\section{Некоторые базовые понятия}}
\EN{\section{Some basics}}
\DE{\section{Einige Grundlagen}}
\FR{\section{Quelques bases}}
\ES{\section{\ESph{}}}
\ITA{\section{Alcune basi teoriche}}
\PTBR{\section{\PTBRph{}}}
\THA{\section{\THAph{}}}
\PL{\section{\PLph{}}}

% sections:
\EN{\input{patterns/intro_CPU_ISA_EN}}
\ES{\input{patterns/intro_CPU_ISA_ES}}
\ITA{\input{patterns/intro_CPU_ISA_ITA}}
\PTBR{\input{patterns/intro_CPU_ISA_PTBR}}
\RU{\input{patterns/intro_CPU_ISA_RU}}
\DE{\input{patterns/intro_CPU_ISA_DE}}
\FR{\input{patterns/intro_CPU_ISA_FR}}
\PL{\input{patterns/intro_CPU_ISA_PL}}

\EN{\input{patterns/numeral_EN}}
\RU{\input{patterns/numeral_RU}}
\ITA{\input{patterns/numeral_ITA}}
\DE{\input{patterns/numeral_DE}}
\FR{\input{patterns/numeral_FR}}
\PL{\input{patterns/numeral_PL}}

% chapters
\input{patterns/00_empty/main}
\input{patterns/011_ret/main}
\input{patterns/01_helloworld/main}
\input{patterns/015_prolog_epilogue/main}
\input{patterns/02_stack/main}
\input{patterns/03_printf/main}
\input{patterns/04_scanf/main}
\input{patterns/05_passing_arguments/main}
\input{patterns/06_return_results/main}
\input{patterns/061_pointers/main}
\input{patterns/065_GOTO/main}
\input{patterns/07_jcc/main}
\input{patterns/08_switch/main}
\input{patterns/09_loops/main}
\input{patterns/10_strings/main}
\input{patterns/11_arith_optimizations/main}
\input{patterns/12_FPU/main}
\input{patterns/13_arrays/main}
\input{patterns/14_bitfields/main}
\EN{\input{patterns/145_LCG/main_EN}}
\RU{\input{patterns/145_LCG/main_RU}}
\input{patterns/15_structs/main}
\input{patterns/17_unions/main}
\input{patterns/18_pointers_to_functions/main}
\input{patterns/185_64bit_in_32_env/main}

\EN{\input{patterns/19_SIMD/main_EN}}
\RU{\input{patterns/19_SIMD/main_RU}}
\DE{\input{patterns/19_SIMD/main_DE}}

\EN{\input{patterns/20_x64/main_EN}}
\RU{\input{patterns/20_x64/main_RU}}

\EN{\input{patterns/205_floating_SIMD/main_EN}}
\RU{\input{patterns/205_floating_SIMD/main_RU}}
\DE{\input{patterns/205_floating_SIMD/main_DE}}

\EN{\input{patterns/ARM/main_EN}}
\RU{\input{patterns/ARM/main_RU}}
\DE{\input{patterns/ARM/main_DE}}

\input{patterns/MIPS/main}

\ifdefined\SPANISH
\chapter{Patrones de código}
\fi % SPANISH

\ifdefined\GERMAN
\chapter{Code-Muster}
\fi % GERMAN

\ifdefined\ENGLISH
\chapter{Code Patterns}
\fi % ENGLISH

\ifdefined\ITALIAN
\chapter{Forme di codice}
\fi % ITALIAN

\ifdefined\RUSSIAN
\chapter{Образцы кода}
\fi % RUSSIAN

\ifdefined\BRAZILIAN
\chapter{Padrões de códigos}
\fi % BRAZILIAN

\ifdefined\THAI
\chapter{รูปแบบของโค้ด}
\fi % THAI

\ifdefined\FRENCH
\chapter{Modèle de code}
\fi % FRENCH

\ifdefined\POLISH
\chapter{\PLph{}}
\fi % POLISH

% sections
\EN{\input{patterns/patterns_opt_dbg_EN}}
\ES{\input{patterns/patterns_opt_dbg_ES}}
\ITA{\input{patterns/patterns_opt_dbg_ITA}}
\PTBR{\input{patterns/patterns_opt_dbg_PTBR}}
\RU{\input{patterns/patterns_opt_dbg_RU}}
\THA{\input{patterns/patterns_opt_dbg_THA}}
\DE{\input{patterns/patterns_opt_dbg_DE}}
\FR{\input{patterns/patterns_opt_dbg_FR}}
\PL{\input{patterns/patterns_opt_dbg_PL}}

\RU{\section{Некоторые базовые понятия}}
\EN{\section{Some basics}}
\DE{\section{Einige Grundlagen}}
\FR{\section{Quelques bases}}
\ES{\section{\ESph{}}}
\ITA{\section{Alcune basi teoriche}}
\PTBR{\section{\PTBRph{}}}
\THA{\section{\THAph{}}}
\PL{\section{\PLph{}}}

% sections:
\EN{\input{patterns/intro_CPU_ISA_EN}}
\ES{\input{patterns/intro_CPU_ISA_ES}}
\ITA{\input{patterns/intro_CPU_ISA_ITA}}
\PTBR{\input{patterns/intro_CPU_ISA_PTBR}}
\RU{\input{patterns/intro_CPU_ISA_RU}}
\DE{\input{patterns/intro_CPU_ISA_DE}}
\FR{\input{patterns/intro_CPU_ISA_FR}}
\PL{\input{patterns/intro_CPU_ISA_PL}}

\EN{\input{patterns/numeral_EN}}
\RU{\input{patterns/numeral_RU}}
\ITA{\input{patterns/numeral_ITA}}
\DE{\input{patterns/numeral_DE}}
\FR{\input{patterns/numeral_FR}}
\PL{\input{patterns/numeral_PL}}

% chapters
\input{patterns/00_empty/main}
\input{patterns/011_ret/main}
\input{patterns/01_helloworld/main}
\input{patterns/015_prolog_epilogue/main}
\input{patterns/02_stack/main}
\input{patterns/03_printf/main}
\input{patterns/04_scanf/main}
\input{patterns/05_passing_arguments/main}
\input{patterns/06_return_results/main}
\input{patterns/061_pointers/main}
\input{patterns/065_GOTO/main}
\input{patterns/07_jcc/main}
\input{patterns/08_switch/main}
\input{patterns/09_loops/main}
\input{patterns/10_strings/main}
\input{patterns/11_arith_optimizations/main}
\input{patterns/12_FPU/main}
\input{patterns/13_arrays/main}
\input{patterns/14_bitfields/main}
\EN{\input{patterns/145_LCG/main_EN}}
\RU{\input{patterns/145_LCG/main_RU}}
\input{patterns/15_structs/main}
\input{patterns/17_unions/main}
\input{patterns/18_pointers_to_functions/main}
\input{patterns/185_64bit_in_32_env/main}

\EN{\input{patterns/19_SIMD/main_EN}}
\RU{\input{patterns/19_SIMD/main_RU}}
\DE{\input{patterns/19_SIMD/main_DE}}

\EN{\input{patterns/20_x64/main_EN}}
\RU{\input{patterns/20_x64/main_RU}}

\EN{\input{patterns/205_floating_SIMD/main_EN}}
\RU{\input{patterns/205_floating_SIMD/main_RU}}
\DE{\input{patterns/205_floating_SIMD/main_DE}}

\EN{\input{patterns/ARM/main_EN}}
\RU{\input{patterns/ARM/main_RU}}
\DE{\input{patterns/ARM/main_DE}}

\input{patterns/MIPS/main}

\ifdefined\SPANISH
\chapter{Patrones de código}
\fi % SPANISH

\ifdefined\GERMAN
\chapter{Code-Muster}
\fi % GERMAN

\ifdefined\ENGLISH
\chapter{Code Patterns}
\fi % ENGLISH

\ifdefined\ITALIAN
\chapter{Forme di codice}
\fi % ITALIAN

\ifdefined\RUSSIAN
\chapter{Образцы кода}
\fi % RUSSIAN

\ifdefined\BRAZILIAN
\chapter{Padrões de códigos}
\fi % BRAZILIAN

\ifdefined\THAI
\chapter{รูปแบบของโค้ด}
\fi % THAI

\ifdefined\FRENCH
\chapter{Modèle de code}
\fi % FRENCH

\ifdefined\POLISH
\chapter{\PLph{}}
\fi % POLISH

% sections
\EN{\input{patterns/patterns_opt_dbg_EN}}
\ES{\input{patterns/patterns_opt_dbg_ES}}
\ITA{\input{patterns/patterns_opt_dbg_ITA}}
\PTBR{\input{patterns/patterns_opt_dbg_PTBR}}
\RU{\input{patterns/patterns_opt_dbg_RU}}
\THA{\input{patterns/patterns_opt_dbg_THA}}
\DE{\input{patterns/patterns_opt_dbg_DE}}
\FR{\input{patterns/patterns_opt_dbg_FR}}
\PL{\input{patterns/patterns_opt_dbg_PL}}

\RU{\section{Некоторые базовые понятия}}
\EN{\section{Some basics}}
\DE{\section{Einige Grundlagen}}
\FR{\section{Quelques bases}}
\ES{\section{\ESph{}}}
\ITA{\section{Alcune basi teoriche}}
\PTBR{\section{\PTBRph{}}}
\THA{\section{\THAph{}}}
\PL{\section{\PLph{}}}

% sections:
\EN{\input{patterns/intro_CPU_ISA_EN}}
\ES{\input{patterns/intro_CPU_ISA_ES}}
\ITA{\input{patterns/intro_CPU_ISA_ITA}}
\PTBR{\input{patterns/intro_CPU_ISA_PTBR}}
\RU{\input{patterns/intro_CPU_ISA_RU}}
\DE{\input{patterns/intro_CPU_ISA_DE}}
\FR{\input{patterns/intro_CPU_ISA_FR}}
\PL{\input{patterns/intro_CPU_ISA_PL}}

\EN{\input{patterns/numeral_EN}}
\RU{\input{patterns/numeral_RU}}
\ITA{\input{patterns/numeral_ITA}}
\DE{\input{patterns/numeral_DE}}
\FR{\input{patterns/numeral_FR}}
\PL{\input{patterns/numeral_PL}}

% chapters
\input{patterns/00_empty/main}
\input{patterns/011_ret/main}
\input{patterns/01_helloworld/main}
\input{patterns/015_prolog_epilogue/main}
\input{patterns/02_stack/main}
\input{patterns/03_printf/main}
\input{patterns/04_scanf/main}
\input{patterns/05_passing_arguments/main}
\input{patterns/06_return_results/main}
\input{patterns/061_pointers/main}
\input{patterns/065_GOTO/main}
\input{patterns/07_jcc/main}
\input{patterns/08_switch/main}
\input{patterns/09_loops/main}
\input{patterns/10_strings/main}
\input{patterns/11_arith_optimizations/main}
\input{patterns/12_FPU/main}
\input{patterns/13_arrays/main}
\input{patterns/14_bitfields/main}
\EN{\input{patterns/145_LCG/main_EN}}
\RU{\input{patterns/145_LCG/main_RU}}
\input{patterns/15_structs/main}
\input{patterns/17_unions/main}
\input{patterns/18_pointers_to_functions/main}
\input{patterns/185_64bit_in_32_env/main}

\EN{\input{patterns/19_SIMD/main_EN}}
\RU{\input{patterns/19_SIMD/main_RU}}
\DE{\input{patterns/19_SIMD/main_DE}}

\EN{\input{patterns/20_x64/main_EN}}
\RU{\input{patterns/20_x64/main_RU}}

\EN{\input{patterns/205_floating_SIMD/main_EN}}
\RU{\input{patterns/205_floating_SIMD/main_RU}}
\DE{\input{patterns/205_floating_SIMD/main_DE}}

\EN{\input{patterns/ARM/main_EN}}
\RU{\input{patterns/ARM/main_RU}}
\DE{\input{patterns/ARM/main_DE}}

\input{patterns/MIPS/main}

\EN{\input{patterns/12_FPU/main_EN}}
\RU{\input{patterns/12_FPU/main_RU}}
\DE{\input{patterns/12_FPU/main_DE}}
\FR{\input{patterns/12_FPU/main_FR}}


\ifdefined\SPANISH
\chapter{Patrones de código}
\fi % SPANISH

\ifdefined\GERMAN
\chapter{Code-Muster}
\fi % GERMAN

\ifdefined\ENGLISH
\chapter{Code Patterns}
\fi % ENGLISH

\ifdefined\ITALIAN
\chapter{Forme di codice}
\fi % ITALIAN

\ifdefined\RUSSIAN
\chapter{Образцы кода}
\fi % RUSSIAN

\ifdefined\BRAZILIAN
\chapter{Padrões de códigos}
\fi % BRAZILIAN

\ifdefined\THAI
\chapter{รูปแบบของโค้ด}
\fi % THAI

\ifdefined\FRENCH
\chapter{Modèle de code}
\fi % FRENCH

\ifdefined\POLISH
\chapter{\PLph{}}
\fi % POLISH

% sections
\EN{\input{patterns/patterns_opt_dbg_EN}}
\ES{\input{patterns/patterns_opt_dbg_ES}}
\ITA{\input{patterns/patterns_opt_dbg_ITA}}
\PTBR{\input{patterns/patterns_opt_dbg_PTBR}}
\RU{\input{patterns/patterns_opt_dbg_RU}}
\THA{\input{patterns/patterns_opt_dbg_THA}}
\DE{\input{patterns/patterns_opt_dbg_DE}}
\FR{\input{patterns/patterns_opt_dbg_FR}}
\PL{\input{patterns/patterns_opt_dbg_PL}}

\RU{\section{Некоторые базовые понятия}}
\EN{\section{Some basics}}
\DE{\section{Einige Grundlagen}}
\FR{\section{Quelques bases}}
\ES{\section{\ESph{}}}
\ITA{\section{Alcune basi teoriche}}
\PTBR{\section{\PTBRph{}}}
\THA{\section{\THAph{}}}
\PL{\section{\PLph{}}}

% sections:
\EN{\input{patterns/intro_CPU_ISA_EN}}
\ES{\input{patterns/intro_CPU_ISA_ES}}
\ITA{\input{patterns/intro_CPU_ISA_ITA}}
\PTBR{\input{patterns/intro_CPU_ISA_PTBR}}
\RU{\input{patterns/intro_CPU_ISA_RU}}
\DE{\input{patterns/intro_CPU_ISA_DE}}
\FR{\input{patterns/intro_CPU_ISA_FR}}
\PL{\input{patterns/intro_CPU_ISA_PL}}

\EN{\input{patterns/numeral_EN}}
\RU{\input{patterns/numeral_RU}}
\ITA{\input{patterns/numeral_ITA}}
\DE{\input{patterns/numeral_DE}}
\FR{\input{patterns/numeral_FR}}
\PL{\input{patterns/numeral_PL}}

% chapters
\input{patterns/00_empty/main}
\input{patterns/011_ret/main}
\input{patterns/01_helloworld/main}
\input{patterns/015_prolog_epilogue/main}
\input{patterns/02_stack/main}
\input{patterns/03_printf/main}
\input{patterns/04_scanf/main}
\input{patterns/05_passing_arguments/main}
\input{patterns/06_return_results/main}
\input{patterns/061_pointers/main}
\input{patterns/065_GOTO/main}
\input{patterns/07_jcc/main}
\input{patterns/08_switch/main}
\input{patterns/09_loops/main}
\input{patterns/10_strings/main}
\input{patterns/11_arith_optimizations/main}
\input{patterns/12_FPU/main}
\input{patterns/13_arrays/main}
\input{patterns/14_bitfields/main}
\EN{\input{patterns/145_LCG/main_EN}}
\RU{\input{patterns/145_LCG/main_RU}}
\input{patterns/15_structs/main}
\input{patterns/17_unions/main}
\input{patterns/18_pointers_to_functions/main}
\input{patterns/185_64bit_in_32_env/main}

\EN{\input{patterns/19_SIMD/main_EN}}
\RU{\input{patterns/19_SIMD/main_RU}}
\DE{\input{patterns/19_SIMD/main_DE}}

\EN{\input{patterns/20_x64/main_EN}}
\RU{\input{patterns/20_x64/main_RU}}

\EN{\input{patterns/205_floating_SIMD/main_EN}}
\RU{\input{patterns/205_floating_SIMD/main_RU}}
\DE{\input{patterns/205_floating_SIMD/main_DE}}

\EN{\input{patterns/ARM/main_EN}}
\RU{\input{patterns/ARM/main_RU}}
\DE{\input{patterns/ARM/main_DE}}

\input{patterns/MIPS/main}

\ifdefined\SPANISH
\chapter{Patrones de código}
\fi % SPANISH

\ifdefined\GERMAN
\chapter{Code-Muster}
\fi % GERMAN

\ifdefined\ENGLISH
\chapter{Code Patterns}
\fi % ENGLISH

\ifdefined\ITALIAN
\chapter{Forme di codice}
\fi % ITALIAN

\ifdefined\RUSSIAN
\chapter{Образцы кода}
\fi % RUSSIAN

\ifdefined\BRAZILIAN
\chapter{Padrões de códigos}
\fi % BRAZILIAN

\ifdefined\THAI
\chapter{รูปแบบของโค้ด}
\fi % THAI

\ifdefined\FRENCH
\chapter{Modèle de code}
\fi % FRENCH

\ifdefined\POLISH
\chapter{\PLph{}}
\fi % POLISH

% sections
\EN{\input{patterns/patterns_opt_dbg_EN}}
\ES{\input{patterns/patterns_opt_dbg_ES}}
\ITA{\input{patterns/patterns_opt_dbg_ITA}}
\PTBR{\input{patterns/patterns_opt_dbg_PTBR}}
\RU{\input{patterns/patterns_opt_dbg_RU}}
\THA{\input{patterns/patterns_opt_dbg_THA}}
\DE{\input{patterns/patterns_opt_dbg_DE}}
\FR{\input{patterns/patterns_opt_dbg_FR}}
\PL{\input{patterns/patterns_opt_dbg_PL}}

\RU{\section{Некоторые базовые понятия}}
\EN{\section{Some basics}}
\DE{\section{Einige Grundlagen}}
\FR{\section{Quelques bases}}
\ES{\section{\ESph{}}}
\ITA{\section{Alcune basi teoriche}}
\PTBR{\section{\PTBRph{}}}
\THA{\section{\THAph{}}}
\PL{\section{\PLph{}}}

% sections:
\EN{\input{patterns/intro_CPU_ISA_EN}}
\ES{\input{patterns/intro_CPU_ISA_ES}}
\ITA{\input{patterns/intro_CPU_ISA_ITA}}
\PTBR{\input{patterns/intro_CPU_ISA_PTBR}}
\RU{\input{patterns/intro_CPU_ISA_RU}}
\DE{\input{patterns/intro_CPU_ISA_DE}}
\FR{\input{patterns/intro_CPU_ISA_FR}}
\PL{\input{patterns/intro_CPU_ISA_PL}}

\EN{\input{patterns/numeral_EN}}
\RU{\input{patterns/numeral_RU}}
\ITA{\input{patterns/numeral_ITA}}
\DE{\input{patterns/numeral_DE}}
\FR{\input{patterns/numeral_FR}}
\PL{\input{patterns/numeral_PL}}

% chapters
\input{patterns/00_empty/main}
\input{patterns/011_ret/main}
\input{patterns/01_helloworld/main}
\input{patterns/015_prolog_epilogue/main}
\input{patterns/02_stack/main}
\input{patterns/03_printf/main}
\input{patterns/04_scanf/main}
\input{patterns/05_passing_arguments/main}
\input{patterns/06_return_results/main}
\input{patterns/061_pointers/main}
\input{patterns/065_GOTO/main}
\input{patterns/07_jcc/main}
\input{patterns/08_switch/main}
\input{patterns/09_loops/main}
\input{patterns/10_strings/main}
\input{patterns/11_arith_optimizations/main}
\input{patterns/12_FPU/main}
\input{patterns/13_arrays/main}
\input{patterns/14_bitfields/main}
\EN{\input{patterns/145_LCG/main_EN}}
\RU{\input{patterns/145_LCG/main_RU}}
\input{patterns/15_structs/main}
\input{patterns/17_unions/main}
\input{patterns/18_pointers_to_functions/main}
\input{patterns/185_64bit_in_32_env/main}

\EN{\input{patterns/19_SIMD/main_EN}}
\RU{\input{patterns/19_SIMD/main_RU}}
\DE{\input{patterns/19_SIMD/main_DE}}

\EN{\input{patterns/20_x64/main_EN}}
\RU{\input{patterns/20_x64/main_RU}}

\EN{\input{patterns/205_floating_SIMD/main_EN}}
\RU{\input{patterns/205_floating_SIMD/main_RU}}
\DE{\input{patterns/205_floating_SIMD/main_DE}}

\EN{\input{patterns/ARM/main_EN}}
\RU{\input{patterns/ARM/main_RU}}
\DE{\input{patterns/ARM/main_DE}}

\input{patterns/MIPS/main}

\EN{\section{Returning Values}
\label{ret_val_func}

Another simple function is the one that simply returns a constant value:

\lstinputlisting[caption=\EN{\CCpp Code},style=customc]{patterns/011_ret/1.c}

Let's compile it.

\subsection{x86}

Here's what both the GCC and MSVC compilers produce (with optimization) on the x86 platform:

\lstinputlisting[caption=\Optimizing GCC/MSVC (\assemblyOutput),style=customasmx86]{patterns/011_ret/1.s}

\myindex{x86!\Instructions!RET}
There are just two instructions: the first places the value 123 into the \EAX register,
which is used by convention for storing the return
value, and the second one is \RET, which returns execution to the \gls{caller}.

The caller will take the result from the \EAX register.

\subsection{ARM}

There are a few differences on the ARM platform:

\lstinputlisting[caption=\OptimizingKeilVI (\ARMMode) ASM Output,style=customasmARM]{patterns/011_ret/1_Keil_ARM_O3.s}

ARM uses the register \Reg{0} for returning the results of functions, so 123 is copied into \Reg{0}.

\myindex{ARM!\Instructions!MOV}
\myindex{x86!\Instructions!MOV}
It is worth noting that \MOV is a misleading name for the instruction in both the x86 and ARM \ac{ISA}s.

The data is not in fact \IT{moved}, but \IT{copied}.

\subsection{MIPS}

\label{MIPS_leaf_function_ex1}

The GCC assembly output below lists registers by number:

\lstinputlisting[caption=\Optimizing GCC 4.4.5 (\assemblyOutput),style=customasmMIPS]{patterns/011_ret/MIPS.s}

\dots while \IDA does it by their pseudo names:

\lstinputlisting[caption=\Optimizing GCC 4.4.5 (IDA),style=customasmMIPS]{patterns/011_ret/MIPS_IDA.lst}

The \$2 (or \$V0) register is used to store the function's return value.
\myindex{MIPS!\Pseudoinstructions!LI}
\INS{LI} stands for ``Load Immediate'' and is the MIPS equivalent to \MOV.

\myindex{MIPS!\Instructions!J}
The other instruction is the jump instruction (J or JR) which returns the execution flow to the \gls{caller}.

\myindex{MIPS!Branch delay slot}
You might be wondering why the positions of the load instruction (LI) and the jump instruction (J or JR) are swapped. This is due to a \ac{RISC} feature called ``branch delay slot''.

The reason this happens is a quirk in the architecture of some RISC \ac{ISA}s and isn't important for our
purposes---we must simply keep in mind that in MIPS, the instruction following a jump or branch instruction
is executed \IT{before} the jump/branch instruction itself.

As a consequence, branch instructions always swap places with the instruction executed immediately beforehand.


In practice, functions which merely return 1 (\IT{true}) or 0 (\IT{false}) are very frequent.

The smallest ever of the standard UNIX utilities, \IT{/bin/true} and \IT{/bin/false} return 0 and 1 respectively, as an exit code.
(Zero as an exit code usually means success, non-zero means error.)
}
\RU{\subsubsection{std::string}
\myindex{\Cpp!STL!std::string}
\label{std_string}

\myparagraph{Как устроена структура}

Многие строковые библиотеки \InSqBrackets{\CNotes 2.2} обеспечивают структуру содержащую ссылку 
на буфер собственно со строкой, переменная всегда содержащую длину строки 
(что очень удобно для массы функций \InSqBrackets{\CNotes 2.2.1}) и переменную содержащую текущий размер буфера.

Строка в буфере обыкновенно оканчивается нулем: это для того чтобы указатель на буфер можно было
передавать в функции требующие на вход обычную сишную \ac{ASCIIZ}-строку.

Стандарт \Cpp не описывает, как именно нужно реализовывать std::string,
но, как правило, они реализованы как описано выше, с небольшими дополнениями.

Строки в \Cpp это не класс (как, например, QString в Qt), а темплейт (basic\_string), 
это сделано для того чтобы поддерживать 
строки содержащие разного типа символы: как минимум \Tchar и \IT{wchar\_t}.

Так что, std::string это класс с базовым типом \Tchar.

А std::wstring это класс с базовым типом \IT{wchar\_t}.

\mysubparagraph{MSVC}

В реализации MSVC, вместо ссылки на буфер может содержаться сам буфер (если строка короче 16-и символов).

Это означает, что каждая короткая строка будет занимать в памяти по крайней мере $16 + 4 + 4 = 24$ 
байт для 32-битной среды либо $16 + 8 + 8 = 32$ 
байта в 64-битной, а если строка длиннее 16-и символов, то прибавьте еще длину самой строки.

\lstinputlisting[caption=пример для MSVC,style=customc]{\CURPATH/STL/string/MSVC_RU.cpp}

Собственно, из этого исходника почти всё ясно.

Несколько замечаний:

Если строка короче 16-и символов, 
то отдельный буфер для строки в \glslink{heap}{куче} выделяться не будет.

Это удобно потому что на практике, основная часть строк действительно короткие.
Вероятно, разработчики в Microsoft выбрали размер в 16 символов как разумный баланс.

Теперь очень важный момент в конце функции main(): мы не пользуемся методом c\_str(), тем не менее,
если это скомпилировать и запустить, то обе строки появятся в консоли!

Работает это вот почему.

В первом случае строка короче 16-и символов и в начале объекта std::string (его можно рассматривать
просто как структуру) расположен буфер с этой строкой.
\printf трактует указатель как указатель на массив символов оканчивающийся нулем и поэтому всё работает.

Вывод второй строки (длиннее 16-и символов) даже еще опаснее: это вообще типичная программистская ошибка 
(или опечатка), забыть дописать c\_str().
Это работает потому что в это время в начале структуры расположен указатель на буфер.
Это может надолго остаться незамеченным: до тех пока там не появится строка 
короче 16-и символов, тогда процесс упадет.

\mysubparagraph{GCC}

В реализации GCC в структуре есть еще одна переменная --- reference count.

Интересно, что указатель на экземпляр класса std::string в GCC указывает не на начало самой структуры, 
а на указатель на буфера.
В libstdc++-v3\textbackslash{}include\textbackslash{}bits\textbackslash{}basic\_string.h 
мы можем прочитать что это сделано для удобства отладки:

\begin{lstlisting}
   *  The reason you want _M_data pointing to the character %array and
   *  not the _Rep is so that the debugger can see the string
   *  contents. (Probably we should add a non-inline member to get
   *  the _Rep for the debugger to use, so users can check the actual
   *  string length.)
\end{lstlisting}

\href{http://go.yurichev.com/17085}{исходный код basic\_string.h}

В нашем примере мы учитываем это:

\lstinputlisting[caption=пример для GCC,style=customc]{\CURPATH/STL/string/GCC_RU.cpp}

Нужны еще небольшие хаки чтобы сымитировать типичную ошибку, которую мы уже видели выше, из-за
более ужесточенной проверки типов в GCC, тем не менее, printf() работает и здесь без c\_str().

\myparagraph{Чуть более сложный пример}

\lstinputlisting[style=customc]{\CURPATH/STL/string/3.cpp}

\lstinputlisting[caption=MSVC 2012,style=customasmx86]{\CURPATH/STL/string/3_MSVC_RU.asm}

Собственно, компилятор не конструирует строки статически: да в общем-то и как
это возможно, если буфер с ней нужно хранить в \glslink{heap}{куче}?

Вместо этого в сегменте данных хранятся обычные \ac{ASCIIZ}-строки, а позже, во время выполнения, 
при помощи метода \q{assign}, конструируются строки s1 и s2
.
При помощи \TT{operator+}, создается строка s3.

Обратите внимание на то что вызов метода c\_str() отсутствует,
потому что его код достаточно короткий и компилятор вставил его прямо здесь:
если строка короче 16-и байт, то в регистре EAX остается указатель на буфер,
а если длиннее, то из этого же места достается адрес на буфер расположенный в \glslink{heap}{куче}.

Далее следуют вызовы трех деструкторов, причем, они вызываются только если строка длиннее 16-и байт:
тогда нужно освободить буфера в \glslink{heap}{куче}.
В противном случае, так как все три объекта std::string хранятся в стеке,
они освобождаются автоматически после выхода из функции.

Следовательно, работа с короткими строками более быстрая из-за м\'{е}ньшего обращения к \glslink{heap}{куче}.

Код на GCC даже проще (из-за того, что в GCC, как мы уже видели, не реализована возможность хранить короткую
строку прямо в структуре):

% TODO1 comment each function meaning
\lstinputlisting[caption=GCC 4.8.1,style=customasmx86]{\CURPATH/STL/string/3_GCC_RU.s}

Можно заметить, что в деструкторы передается не указатель на объект,
а указатель на место за 12 байт (или 3 слова) перед ним, то есть, на настоящее начало структуры.

\myparagraph{std::string как глобальная переменная}
\label{sec:std_string_as_global_variable}

Опытные программисты на \Cpp знают, что глобальные переменные \ac{STL}-типов вполне можно объявлять.

Да, действительно:

\lstinputlisting[style=customc]{\CURPATH/STL/string/5.cpp}

Но как и где будет вызываться конструктор \TT{std::string}?

На самом деле, эта переменная будет инициализирована даже перед началом \main.

\lstinputlisting[caption=MSVC 2012: здесь конструируется глобальная переменная{,} а также регистрируется её деструктор,style=customasmx86]{\CURPATH/STL/string/5_MSVC_p2.asm}

\lstinputlisting[caption=MSVC 2012: здесь глобальная переменная используется в \main,style=customasmx86]{\CURPATH/STL/string/5_MSVC_p1.asm}

\lstinputlisting[caption=MSVC 2012: эта функция-деструктор вызывается перед выходом,style=customasmx86]{\CURPATH/STL/string/5_MSVC_p3.asm}

\myindex{\CStandardLibrary!atexit()}
В реальности, из \ac{CRT}, еще до вызова main(), вызывается специальная функция,
в которой перечислены все конструкторы подобных переменных.
Более того: при помощи atexit() регистрируется функция, которая будет вызвана в конце работы программы:
в этой функции компилятор собирает вызовы деструкторов всех подобных глобальных переменных.

GCC работает похожим образом:

\lstinputlisting[caption=GCC 4.8.1,style=customasmx86]{\CURPATH/STL/string/5_GCC.s}

Но он не выделяет отдельной функции в которой будут собраны деструкторы: 
каждый деструктор передается в atexit() по одному.

% TODO а если глобальная STL-переменная в другом модуле? надо проверить.

}
\ifdefined\SPANISH
\chapter{Patrones de código}
\fi % SPANISH

\ifdefined\GERMAN
\chapter{Code-Muster}
\fi % GERMAN

\ifdefined\ENGLISH
\chapter{Code Patterns}
\fi % ENGLISH

\ifdefined\ITALIAN
\chapter{Forme di codice}
\fi % ITALIAN

\ifdefined\RUSSIAN
\chapter{Образцы кода}
\fi % RUSSIAN

\ifdefined\BRAZILIAN
\chapter{Padrões de códigos}
\fi % BRAZILIAN

\ifdefined\THAI
\chapter{รูปแบบของโค้ด}
\fi % THAI

\ifdefined\FRENCH
\chapter{Modèle de code}
\fi % FRENCH

\ifdefined\POLISH
\chapter{\PLph{}}
\fi % POLISH

% sections
\EN{\input{patterns/patterns_opt_dbg_EN}}
\ES{\input{patterns/patterns_opt_dbg_ES}}
\ITA{\input{patterns/patterns_opt_dbg_ITA}}
\PTBR{\input{patterns/patterns_opt_dbg_PTBR}}
\RU{\input{patterns/patterns_opt_dbg_RU}}
\THA{\input{patterns/patterns_opt_dbg_THA}}
\DE{\input{patterns/patterns_opt_dbg_DE}}
\FR{\input{patterns/patterns_opt_dbg_FR}}
\PL{\input{patterns/patterns_opt_dbg_PL}}

\RU{\section{Некоторые базовые понятия}}
\EN{\section{Some basics}}
\DE{\section{Einige Grundlagen}}
\FR{\section{Quelques bases}}
\ES{\section{\ESph{}}}
\ITA{\section{Alcune basi teoriche}}
\PTBR{\section{\PTBRph{}}}
\THA{\section{\THAph{}}}
\PL{\section{\PLph{}}}

% sections:
\EN{\input{patterns/intro_CPU_ISA_EN}}
\ES{\input{patterns/intro_CPU_ISA_ES}}
\ITA{\input{patterns/intro_CPU_ISA_ITA}}
\PTBR{\input{patterns/intro_CPU_ISA_PTBR}}
\RU{\input{patterns/intro_CPU_ISA_RU}}
\DE{\input{patterns/intro_CPU_ISA_DE}}
\FR{\input{patterns/intro_CPU_ISA_FR}}
\PL{\input{patterns/intro_CPU_ISA_PL}}

\EN{\input{patterns/numeral_EN}}
\RU{\input{patterns/numeral_RU}}
\ITA{\input{patterns/numeral_ITA}}
\DE{\input{patterns/numeral_DE}}
\FR{\input{patterns/numeral_FR}}
\PL{\input{patterns/numeral_PL}}

% chapters
\input{patterns/00_empty/main}
\input{patterns/011_ret/main}
\input{patterns/01_helloworld/main}
\input{patterns/015_prolog_epilogue/main}
\input{patterns/02_stack/main}
\input{patterns/03_printf/main}
\input{patterns/04_scanf/main}
\input{patterns/05_passing_arguments/main}
\input{patterns/06_return_results/main}
\input{patterns/061_pointers/main}
\input{patterns/065_GOTO/main}
\input{patterns/07_jcc/main}
\input{patterns/08_switch/main}
\input{patterns/09_loops/main}
\input{patterns/10_strings/main}
\input{patterns/11_arith_optimizations/main}
\input{patterns/12_FPU/main}
\input{patterns/13_arrays/main}
\input{patterns/14_bitfields/main}
\EN{\input{patterns/145_LCG/main_EN}}
\RU{\input{patterns/145_LCG/main_RU}}
\input{patterns/15_structs/main}
\input{patterns/17_unions/main}
\input{patterns/18_pointers_to_functions/main}
\input{patterns/185_64bit_in_32_env/main}

\EN{\input{patterns/19_SIMD/main_EN}}
\RU{\input{patterns/19_SIMD/main_RU}}
\DE{\input{patterns/19_SIMD/main_DE}}

\EN{\input{patterns/20_x64/main_EN}}
\RU{\input{patterns/20_x64/main_RU}}

\EN{\input{patterns/205_floating_SIMD/main_EN}}
\RU{\input{patterns/205_floating_SIMD/main_RU}}
\DE{\input{patterns/205_floating_SIMD/main_DE}}

\EN{\input{patterns/ARM/main_EN}}
\RU{\input{patterns/ARM/main_RU}}
\DE{\input{patterns/ARM/main_DE}}

\input{patterns/MIPS/main}

\ifdefined\SPANISH
\chapter{Patrones de código}
\fi % SPANISH

\ifdefined\GERMAN
\chapter{Code-Muster}
\fi % GERMAN

\ifdefined\ENGLISH
\chapter{Code Patterns}
\fi % ENGLISH

\ifdefined\ITALIAN
\chapter{Forme di codice}
\fi % ITALIAN

\ifdefined\RUSSIAN
\chapter{Образцы кода}
\fi % RUSSIAN

\ifdefined\BRAZILIAN
\chapter{Padrões de códigos}
\fi % BRAZILIAN

\ifdefined\THAI
\chapter{รูปแบบของโค้ด}
\fi % THAI

\ifdefined\FRENCH
\chapter{Modèle de code}
\fi % FRENCH

\ifdefined\POLISH
\chapter{\PLph{}}
\fi % POLISH

% sections
\EN{\input{patterns/patterns_opt_dbg_EN}}
\ES{\input{patterns/patterns_opt_dbg_ES}}
\ITA{\input{patterns/patterns_opt_dbg_ITA}}
\PTBR{\input{patterns/patterns_opt_dbg_PTBR}}
\RU{\input{patterns/patterns_opt_dbg_RU}}
\THA{\input{patterns/patterns_opt_dbg_THA}}
\DE{\input{patterns/patterns_opt_dbg_DE}}
\FR{\input{patterns/patterns_opt_dbg_FR}}
\PL{\input{patterns/patterns_opt_dbg_PL}}

\RU{\section{Некоторые базовые понятия}}
\EN{\section{Some basics}}
\DE{\section{Einige Grundlagen}}
\FR{\section{Quelques bases}}
\ES{\section{\ESph{}}}
\ITA{\section{Alcune basi teoriche}}
\PTBR{\section{\PTBRph{}}}
\THA{\section{\THAph{}}}
\PL{\section{\PLph{}}}

% sections:
\EN{\input{patterns/intro_CPU_ISA_EN}}
\ES{\input{patterns/intro_CPU_ISA_ES}}
\ITA{\input{patterns/intro_CPU_ISA_ITA}}
\PTBR{\input{patterns/intro_CPU_ISA_PTBR}}
\RU{\input{patterns/intro_CPU_ISA_RU}}
\DE{\input{patterns/intro_CPU_ISA_DE}}
\FR{\input{patterns/intro_CPU_ISA_FR}}
\PL{\input{patterns/intro_CPU_ISA_PL}}

\EN{\input{patterns/numeral_EN}}
\RU{\input{patterns/numeral_RU}}
\ITA{\input{patterns/numeral_ITA}}
\DE{\input{patterns/numeral_DE}}
\FR{\input{patterns/numeral_FR}}
\PL{\input{patterns/numeral_PL}}

% chapters
\input{patterns/00_empty/main}
\input{patterns/011_ret/main}
\input{patterns/01_helloworld/main}
\input{patterns/015_prolog_epilogue/main}
\input{patterns/02_stack/main}
\input{patterns/03_printf/main}
\input{patterns/04_scanf/main}
\input{patterns/05_passing_arguments/main}
\input{patterns/06_return_results/main}
\input{patterns/061_pointers/main}
\input{patterns/065_GOTO/main}
\input{patterns/07_jcc/main}
\input{patterns/08_switch/main}
\input{patterns/09_loops/main}
\input{patterns/10_strings/main}
\input{patterns/11_arith_optimizations/main}
\input{patterns/12_FPU/main}
\input{patterns/13_arrays/main}
\input{patterns/14_bitfields/main}
\EN{\input{patterns/145_LCG/main_EN}}
\RU{\input{patterns/145_LCG/main_RU}}
\input{patterns/15_structs/main}
\input{patterns/17_unions/main}
\input{patterns/18_pointers_to_functions/main}
\input{patterns/185_64bit_in_32_env/main}

\EN{\input{patterns/19_SIMD/main_EN}}
\RU{\input{patterns/19_SIMD/main_RU}}
\DE{\input{patterns/19_SIMD/main_DE}}

\EN{\input{patterns/20_x64/main_EN}}
\RU{\input{patterns/20_x64/main_RU}}

\EN{\input{patterns/205_floating_SIMD/main_EN}}
\RU{\input{patterns/205_floating_SIMD/main_RU}}
\DE{\input{patterns/205_floating_SIMD/main_DE}}

\EN{\input{patterns/ARM/main_EN}}
\RU{\input{patterns/ARM/main_RU}}
\DE{\input{patterns/ARM/main_DE}}

\input{patterns/MIPS/main}

\ifdefined\SPANISH
\chapter{Patrones de código}
\fi % SPANISH

\ifdefined\GERMAN
\chapter{Code-Muster}
\fi % GERMAN

\ifdefined\ENGLISH
\chapter{Code Patterns}
\fi % ENGLISH

\ifdefined\ITALIAN
\chapter{Forme di codice}
\fi % ITALIAN

\ifdefined\RUSSIAN
\chapter{Образцы кода}
\fi % RUSSIAN

\ifdefined\BRAZILIAN
\chapter{Padrões de códigos}
\fi % BRAZILIAN

\ifdefined\THAI
\chapter{รูปแบบของโค้ด}
\fi % THAI

\ifdefined\FRENCH
\chapter{Modèle de code}
\fi % FRENCH

\ifdefined\POLISH
\chapter{\PLph{}}
\fi % POLISH

% sections
\EN{\input{patterns/patterns_opt_dbg_EN}}
\ES{\input{patterns/patterns_opt_dbg_ES}}
\ITA{\input{patterns/patterns_opt_dbg_ITA}}
\PTBR{\input{patterns/patterns_opt_dbg_PTBR}}
\RU{\input{patterns/patterns_opt_dbg_RU}}
\THA{\input{patterns/patterns_opt_dbg_THA}}
\DE{\input{patterns/patterns_opt_dbg_DE}}
\FR{\input{patterns/patterns_opt_dbg_FR}}
\PL{\input{patterns/patterns_opt_dbg_PL}}

\RU{\section{Некоторые базовые понятия}}
\EN{\section{Some basics}}
\DE{\section{Einige Grundlagen}}
\FR{\section{Quelques bases}}
\ES{\section{\ESph{}}}
\ITA{\section{Alcune basi teoriche}}
\PTBR{\section{\PTBRph{}}}
\THA{\section{\THAph{}}}
\PL{\section{\PLph{}}}

% sections:
\EN{\input{patterns/intro_CPU_ISA_EN}}
\ES{\input{patterns/intro_CPU_ISA_ES}}
\ITA{\input{patterns/intro_CPU_ISA_ITA}}
\PTBR{\input{patterns/intro_CPU_ISA_PTBR}}
\RU{\input{patterns/intro_CPU_ISA_RU}}
\DE{\input{patterns/intro_CPU_ISA_DE}}
\FR{\input{patterns/intro_CPU_ISA_FR}}
\PL{\input{patterns/intro_CPU_ISA_PL}}

\EN{\input{patterns/numeral_EN}}
\RU{\input{patterns/numeral_RU}}
\ITA{\input{patterns/numeral_ITA}}
\DE{\input{patterns/numeral_DE}}
\FR{\input{patterns/numeral_FR}}
\PL{\input{patterns/numeral_PL}}

% chapters
\input{patterns/00_empty/main}
\input{patterns/011_ret/main}
\input{patterns/01_helloworld/main}
\input{patterns/015_prolog_epilogue/main}
\input{patterns/02_stack/main}
\input{patterns/03_printf/main}
\input{patterns/04_scanf/main}
\input{patterns/05_passing_arguments/main}
\input{patterns/06_return_results/main}
\input{patterns/061_pointers/main}
\input{patterns/065_GOTO/main}
\input{patterns/07_jcc/main}
\input{patterns/08_switch/main}
\input{patterns/09_loops/main}
\input{patterns/10_strings/main}
\input{patterns/11_arith_optimizations/main}
\input{patterns/12_FPU/main}
\input{patterns/13_arrays/main}
\input{patterns/14_bitfields/main}
\EN{\input{patterns/145_LCG/main_EN}}
\RU{\input{patterns/145_LCG/main_RU}}
\input{patterns/15_structs/main}
\input{patterns/17_unions/main}
\input{patterns/18_pointers_to_functions/main}
\input{patterns/185_64bit_in_32_env/main}

\EN{\input{patterns/19_SIMD/main_EN}}
\RU{\input{patterns/19_SIMD/main_RU}}
\DE{\input{patterns/19_SIMD/main_DE}}

\EN{\input{patterns/20_x64/main_EN}}
\RU{\input{patterns/20_x64/main_RU}}

\EN{\input{patterns/205_floating_SIMD/main_EN}}
\RU{\input{patterns/205_floating_SIMD/main_RU}}
\DE{\input{patterns/205_floating_SIMD/main_DE}}

\EN{\input{patterns/ARM/main_EN}}
\RU{\input{patterns/ARM/main_RU}}
\DE{\input{patterns/ARM/main_DE}}

\input{patterns/MIPS/main}

\ifdefined\SPANISH
\chapter{Patrones de código}
\fi % SPANISH

\ifdefined\GERMAN
\chapter{Code-Muster}
\fi % GERMAN

\ifdefined\ENGLISH
\chapter{Code Patterns}
\fi % ENGLISH

\ifdefined\ITALIAN
\chapter{Forme di codice}
\fi % ITALIAN

\ifdefined\RUSSIAN
\chapter{Образцы кода}
\fi % RUSSIAN

\ifdefined\BRAZILIAN
\chapter{Padrões de códigos}
\fi % BRAZILIAN

\ifdefined\THAI
\chapter{รูปแบบของโค้ด}
\fi % THAI

\ifdefined\FRENCH
\chapter{Modèle de code}
\fi % FRENCH

\ifdefined\POLISH
\chapter{\PLph{}}
\fi % POLISH

% sections
\EN{\input{patterns/patterns_opt_dbg_EN}}
\ES{\input{patterns/patterns_opt_dbg_ES}}
\ITA{\input{patterns/patterns_opt_dbg_ITA}}
\PTBR{\input{patterns/patterns_opt_dbg_PTBR}}
\RU{\input{patterns/patterns_opt_dbg_RU}}
\THA{\input{patterns/patterns_opt_dbg_THA}}
\DE{\input{patterns/patterns_opt_dbg_DE}}
\FR{\input{patterns/patterns_opt_dbg_FR}}
\PL{\input{patterns/patterns_opt_dbg_PL}}

\RU{\section{Некоторые базовые понятия}}
\EN{\section{Some basics}}
\DE{\section{Einige Grundlagen}}
\FR{\section{Quelques bases}}
\ES{\section{\ESph{}}}
\ITA{\section{Alcune basi teoriche}}
\PTBR{\section{\PTBRph{}}}
\THA{\section{\THAph{}}}
\PL{\section{\PLph{}}}

% sections:
\EN{\input{patterns/intro_CPU_ISA_EN}}
\ES{\input{patterns/intro_CPU_ISA_ES}}
\ITA{\input{patterns/intro_CPU_ISA_ITA}}
\PTBR{\input{patterns/intro_CPU_ISA_PTBR}}
\RU{\input{patterns/intro_CPU_ISA_RU}}
\DE{\input{patterns/intro_CPU_ISA_DE}}
\FR{\input{patterns/intro_CPU_ISA_FR}}
\PL{\input{patterns/intro_CPU_ISA_PL}}

\EN{\input{patterns/numeral_EN}}
\RU{\input{patterns/numeral_RU}}
\ITA{\input{patterns/numeral_ITA}}
\DE{\input{patterns/numeral_DE}}
\FR{\input{patterns/numeral_FR}}
\PL{\input{patterns/numeral_PL}}

% chapters
\input{patterns/00_empty/main}
\input{patterns/011_ret/main}
\input{patterns/01_helloworld/main}
\input{patterns/015_prolog_epilogue/main}
\input{patterns/02_stack/main}
\input{patterns/03_printf/main}
\input{patterns/04_scanf/main}
\input{patterns/05_passing_arguments/main}
\input{patterns/06_return_results/main}
\input{patterns/061_pointers/main}
\input{patterns/065_GOTO/main}
\input{patterns/07_jcc/main}
\input{patterns/08_switch/main}
\input{patterns/09_loops/main}
\input{patterns/10_strings/main}
\input{patterns/11_arith_optimizations/main}
\input{patterns/12_FPU/main}
\input{patterns/13_arrays/main}
\input{patterns/14_bitfields/main}
\EN{\input{patterns/145_LCG/main_EN}}
\RU{\input{patterns/145_LCG/main_RU}}
\input{patterns/15_structs/main}
\input{patterns/17_unions/main}
\input{patterns/18_pointers_to_functions/main}
\input{patterns/185_64bit_in_32_env/main}

\EN{\input{patterns/19_SIMD/main_EN}}
\RU{\input{patterns/19_SIMD/main_RU}}
\DE{\input{patterns/19_SIMD/main_DE}}

\EN{\input{patterns/20_x64/main_EN}}
\RU{\input{patterns/20_x64/main_RU}}

\EN{\input{patterns/205_floating_SIMD/main_EN}}
\RU{\input{patterns/205_floating_SIMD/main_RU}}
\DE{\input{patterns/205_floating_SIMD/main_DE}}

\EN{\input{patterns/ARM/main_EN}}
\RU{\input{patterns/ARM/main_RU}}
\DE{\input{patterns/ARM/main_DE}}

\input{patterns/MIPS/main}


\EN{\section{Returning Values}
\label{ret_val_func}

Another simple function is the one that simply returns a constant value:

\lstinputlisting[caption=\EN{\CCpp Code},style=customc]{patterns/011_ret/1.c}

Let's compile it.

\subsection{x86}

Here's what both the GCC and MSVC compilers produce (with optimization) on the x86 platform:

\lstinputlisting[caption=\Optimizing GCC/MSVC (\assemblyOutput),style=customasmx86]{patterns/011_ret/1.s}

\myindex{x86!\Instructions!RET}
There are just two instructions: the first places the value 123 into the \EAX register,
which is used by convention for storing the return
value, and the second one is \RET, which returns execution to the \gls{caller}.

The caller will take the result from the \EAX register.

\subsection{ARM}

There are a few differences on the ARM platform:

\lstinputlisting[caption=\OptimizingKeilVI (\ARMMode) ASM Output,style=customasmARM]{patterns/011_ret/1_Keil_ARM_O3.s}

ARM uses the register \Reg{0} for returning the results of functions, so 123 is copied into \Reg{0}.

\myindex{ARM!\Instructions!MOV}
\myindex{x86!\Instructions!MOV}
It is worth noting that \MOV is a misleading name for the instruction in both the x86 and ARM \ac{ISA}s.

The data is not in fact \IT{moved}, but \IT{copied}.

\subsection{MIPS}

\label{MIPS_leaf_function_ex1}

The GCC assembly output below lists registers by number:

\lstinputlisting[caption=\Optimizing GCC 4.4.5 (\assemblyOutput),style=customasmMIPS]{patterns/011_ret/MIPS.s}

\dots while \IDA does it by their pseudo names:

\lstinputlisting[caption=\Optimizing GCC 4.4.5 (IDA),style=customasmMIPS]{patterns/011_ret/MIPS_IDA.lst}

The \$2 (or \$V0) register is used to store the function's return value.
\myindex{MIPS!\Pseudoinstructions!LI}
\INS{LI} stands for ``Load Immediate'' and is the MIPS equivalent to \MOV.

\myindex{MIPS!\Instructions!J}
The other instruction is the jump instruction (J or JR) which returns the execution flow to the \gls{caller}.

\myindex{MIPS!Branch delay slot}
You might be wondering why the positions of the load instruction (LI) and the jump instruction (J or JR) are swapped. This is due to a \ac{RISC} feature called ``branch delay slot''.

The reason this happens is a quirk in the architecture of some RISC \ac{ISA}s and isn't important for our
purposes---we must simply keep in mind that in MIPS, the instruction following a jump or branch instruction
is executed \IT{before} the jump/branch instruction itself.

As a consequence, branch instructions always swap places with the instruction executed immediately beforehand.


In practice, functions which merely return 1 (\IT{true}) or 0 (\IT{false}) are very frequent.

The smallest ever of the standard UNIX utilities, \IT{/bin/true} and \IT{/bin/false} return 0 and 1 respectively, as an exit code.
(Zero as an exit code usually means success, non-zero means error.)
}
\RU{\subsubsection{std::string}
\myindex{\Cpp!STL!std::string}
\label{std_string}

\myparagraph{Как устроена структура}

Многие строковые библиотеки \InSqBrackets{\CNotes 2.2} обеспечивают структуру содержащую ссылку 
на буфер собственно со строкой, переменная всегда содержащую длину строки 
(что очень удобно для массы функций \InSqBrackets{\CNotes 2.2.1}) и переменную содержащую текущий размер буфера.

Строка в буфере обыкновенно оканчивается нулем: это для того чтобы указатель на буфер можно было
передавать в функции требующие на вход обычную сишную \ac{ASCIIZ}-строку.

Стандарт \Cpp не описывает, как именно нужно реализовывать std::string,
но, как правило, они реализованы как описано выше, с небольшими дополнениями.

Строки в \Cpp это не класс (как, например, QString в Qt), а темплейт (basic\_string), 
это сделано для того чтобы поддерживать 
строки содержащие разного типа символы: как минимум \Tchar и \IT{wchar\_t}.

Так что, std::string это класс с базовым типом \Tchar.

А std::wstring это класс с базовым типом \IT{wchar\_t}.

\mysubparagraph{MSVC}

В реализации MSVC, вместо ссылки на буфер может содержаться сам буфер (если строка короче 16-и символов).

Это означает, что каждая короткая строка будет занимать в памяти по крайней мере $16 + 4 + 4 = 24$ 
байт для 32-битной среды либо $16 + 8 + 8 = 32$ 
байта в 64-битной, а если строка длиннее 16-и символов, то прибавьте еще длину самой строки.

\lstinputlisting[caption=пример для MSVC,style=customc]{\CURPATH/STL/string/MSVC_RU.cpp}

Собственно, из этого исходника почти всё ясно.

Несколько замечаний:

Если строка короче 16-и символов, 
то отдельный буфер для строки в \glslink{heap}{куче} выделяться не будет.

Это удобно потому что на практике, основная часть строк действительно короткие.
Вероятно, разработчики в Microsoft выбрали размер в 16 символов как разумный баланс.

Теперь очень важный момент в конце функции main(): мы не пользуемся методом c\_str(), тем не менее,
если это скомпилировать и запустить, то обе строки появятся в консоли!

Работает это вот почему.

В первом случае строка короче 16-и символов и в начале объекта std::string (его можно рассматривать
просто как структуру) расположен буфер с этой строкой.
\printf трактует указатель как указатель на массив символов оканчивающийся нулем и поэтому всё работает.

Вывод второй строки (длиннее 16-и символов) даже еще опаснее: это вообще типичная программистская ошибка 
(или опечатка), забыть дописать c\_str().
Это работает потому что в это время в начале структуры расположен указатель на буфер.
Это может надолго остаться незамеченным: до тех пока там не появится строка 
короче 16-и символов, тогда процесс упадет.

\mysubparagraph{GCC}

В реализации GCC в структуре есть еще одна переменная --- reference count.

Интересно, что указатель на экземпляр класса std::string в GCC указывает не на начало самой структуры, 
а на указатель на буфера.
В libstdc++-v3\textbackslash{}include\textbackslash{}bits\textbackslash{}basic\_string.h 
мы можем прочитать что это сделано для удобства отладки:

\begin{lstlisting}
   *  The reason you want _M_data pointing to the character %array and
   *  not the _Rep is so that the debugger can see the string
   *  contents. (Probably we should add a non-inline member to get
   *  the _Rep for the debugger to use, so users can check the actual
   *  string length.)
\end{lstlisting}

\href{http://go.yurichev.com/17085}{исходный код basic\_string.h}

В нашем примере мы учитываем это:

\lstinputlisting[caption=пример для GCC,style=customc]{\CURPATH/STL/string/GCC_RU.cpp}

Нужны еще небольшие хаки чтобы сымитировать типичную ошибку, которую мы уже видели выше, из-за
более ужесточенной проверки типов в GCC, тем не менее, printf() работает и здесь без c\_str().

\myparagraph{Чуть более сложный пример}

\lstinputlisting[style=customc]{\CURPATH/STL/string/3.cpp}

\lstinputlisting[caption=MSVC 2012,style=customasmx86]{\CURPATH/STL/string/3_MSVC_RU.asm}

Собственно, компилятор не конструирует строки статически: да в общем-то и как
это возможно, если буфер с ней нужно хранить в \glslink{heap}{куче}?

Вместо этого в сегменте данных хранятся обычные \ac{ASCIIZ}-строки, а позже, во время выполнения, 
при помощи метода \q{assign}, конструируются строки s1 и s2
.
При помощи \TT{operator+}, создается строка s3.

Обратите внимание на то что вызов метода c\_str() отсутствует,
потому что его код достаточно короткий и компилятор вставил его прямо здесь:
если строка короче 16-и байт, то в регистре EAX остается указатель на буфер,
а если длиннее, то из этого же места достается адрес на буфер расположенный в \glslink{heap}{куче}.

Далее следуют вызовы трех деструкторов, причем, они вызываются только если строка длиннее 16-и байт:
тогда нужно освободить буфера в \glslink{heap}{куче}.
В противном случае, так как все три объекта std::string хранятся в стеке,
они освобождаются автоматически после выхода из функции.

Следовательно, работа с короткими строками более быстрая из-за м\'{е}ньшего обращения к \glslink{heap}{куче}.

Код на GCC даже проще (из-за того, что в GCC, как мы уже видели, не реализована возможность хранить короткую
строку прямо в структуре):

% TODO1 comment each function meaning
\lstinputlisting[caption=GCC 4.8.1,style=customasmx86]{\CURPATH/STL/string/3_GCC_RU.s}

Можно заметить, что в деструкторы передается не указатель на объект,
а указатель на место за 12 байт (или 3 слова) перед ним, то есть, на настоящее начало структуры.

\myparagraph{std::string как глобальная переменная}
\label{sec:std_string_as_global_variable}

Опытные программисты на \Cpp знают, что глобальные переменные \ac{STL}-типов вполне можно объявлять.

Да, действительно:

\lstinputlisting[style=customc]{\CURPATH/STL/string/5.cpp}

Но как и где будет вызываться конструктор \TT{std::string}?

На самом деле, эта переменная будет инициализирована даже перед началом \main.

\lstinputlisting[caption=MSVC 2012: здесь конструируется глобальная переменная{,} а также регистрируется её деструктор,style=customasmx86]{\CURPATH/STL/string/5_MSVC_p2.asm}

\lstinputlisting[caption=MSVC 2012: здесь глобальная переменная используется в \main,style=customasmx86]{\CURPATH/STL/string/5_MSVC_p1.asm}

\lstinputlisting[caption=MSVC 2012: эта функция-деструктор вызывается перед выходом,style=customasmx86]{\CURPATH/STL/string/5_MSVC_p3.asm}

\myindex{\CStandardLibrary!atexit()}
В реальности, из \ac{CRT}, еще до вызова main(), вызывается специальная функция,
в которой перечислены все конструкторы подобных переменных.
Более того: при помощи atexit() регистрируется функция, которая будет вызвана в конце работы программы:
в этой функции компилятор собирает вызовы деструкторов всех подобных глобальных переменных.

GCC работает похожим образом:

\lstinputlisting[caption=GCC 4.8.1,style=customasmx86]{\CURPATH/STL/string/5_GCC.s}

Но он не выделяет отдельной функции в которой будут собраны деструкторы: 
каждый деструктор передается в atexit() по одному.

% TODO а если глобальная STL-переменная в другом модуле? надо проверить.

}
\DE{\subsection{Einfachste XOR-Verschlüsselung überhaupt}

Ich habe einmal eine Software gesehen, bei der alle Debugging-Ausgaben mit XOR mit dem Wert 3
verschlüsselt wurden. Mit anderen Worten, die beiden niedrigsten Bits aller Buchstaben wurden invertiert.

``Hello, world'' wurde zu ``Kfool/\#tlqog'':

\begin{lstlisting}
#!/usr/bin/python

msg="Hello, world!"

print "".join(map(lambda x: chr(ord(x)^3), msg))
\end{lstlisting}

Das ist eine ziemlich interessante Verschlüsselung (oder besser eine Verschleierung),
weil sie zwei wichtige Eigenschaften hat:
1) es ist eine einzige Funktion zum Verschlüsseln und entschlüsseln, sie muss nur wiederholt angewendet werden
2) die entstehenden Buchstaben befinden sich im druckbaren Bereich, also die ganze Zeichenkette kann ohne
Escape-Symbole im Code verwendet werden.

Die zweite Eigenschaft nutzt die Tatsache, dass alle druckbaren Zeichen in Reihen organisiert sind: 0x2x-0x7x,
und wenn die beiden niederwertigsten Bits invertiert werden, wird der Buchstabe um eine oder drei Stellen nach
links oder rechts \IT{verschoben}, aber niemals in eine andere Reihe:

\begin{figure}[H]
\centering
\includegraphics[width=0.7\textwidth]{ascii_clean.png}
\caption{7-Bit \ac{ASCII} Tabelle in Emacs}
\end{figure}

\dots mit dem Zeichen 0x7F als einziger Ausnahme.

Im Folgenden werden also beispielsweise die Zeichen A-Z \IT{verschlüsselt}:

\begin{lstlisting}
#!/usr/bin/python

msg="@ABCDEFGHIJKLMNO"

print "".join(map(lambda x: chr(ord(x)^3), msg))
\end{lstlisting}

Ergebnis:
% FIXME \verb  --  relevant comment for German?
\begin{lstlisting}
CBA@GFEDKJIHONML
\end{lstlisting}

Es sieht so aus als würden die Zeichen ``@'' und ``C'' sowie ``B'' und ``A'' vertauscht werden.

Hier ist noch ein interessantes Beispiel, in dem gezeigt wird, wie die Eigenschaften von XOR
ausgenutzt werden können: Exakt den gleichen Effekt, dass druckbare Zeichen auch druckbar bleiben,
kann man dadurch erzielen, dass irgendeine Kombination der niedrigsten vier Bits invertiert wird.
}

\EN{\section{Returning Values}
\label{ret_val_func}

Another simple function is the one that simply returns a constant value:

\lstinputlisting[caption=\EN{\CCpp Code},style=customc]{patterns/011_ret/1.c}

Let's compile it.

\subsection{x86}

Here's what both the GCC and MSVC compilers produce (with optimization) on the x86 platform:

\lstinputlisting[caption=\Optimizing GCC/MSVC (\assemblyOutput),style=customasmx86]{patterns/011_ret/1.s}

\myindex{x86!\Instructions!RET}
There are just two instructions: the first places the value 123 into the \EAX register,
which is used by convention for storing the return
value, and the second one is \RET, which returns execution to the \gls{caller}.

The caller will take the result from the \EAX register.

\subsection{ARM}

There are a few differences on the ARM platform:

\lstinputlisting[caption=\OptimizingKeilVI (\ARMMode) ASM Output,style=customasmARM]{patterns/011_ret/1_Keil_ARM_O3.s}

ARM uses the register \Reg{0} for returning the results of functions, so 123 is copied into \Reg{0}.

\myindex{ARM!\Instructions!MOV}
\myindex{x86!\Instructions!MOV}
It is worth noting that \MOV is a misleading name for the instruction in both the x86 and ARM \ac{ISA}s.

The data is not in fact \IT{moved}, but \IT{copied}.

\subsection{MIPS}

\label{MIPS_leaf_function_ex1}

The GCC assembly output below lists registers by number:

\lstinputlisting[caption=\Optimizing GCC 4.4.5 (\assemblyOutput),style=customasmMIPS]{patterns/011_ret/MIPS.s}

\dots while \IDA does it by their pseudo names:

\lstinputlisting[caption=\Optimizing GCC 4.4.5 (IDA),style=customasmMIPS]{patterns/011_ret/MIPS_IDA.lst}

The \$2 (or \$V0) register is used to store the function's return value.
\myindex{MIPS!\Pseudoinstructions!LI}
\INS{LI} stands for ``Load Immediate'' and is the MIPS equivalent to \MOV.

\myindex{MIPS!\Instructions!J}
The other instruction is the jump instruction (J or JR) which returns the execution flow to the \gls{caller}.

\myindex{MIPS!Branch delay slot}
You might be wondering why the positions of the load instruction (LI) and the jump instruction (J or JR) are swapped. This is due to a \ac{RISC} feature called ``branch delay slot''.

The reason this happens is a quirk in the architecture of some RISC \ac{ISA}s and isn't important for our
purposes---we must simply keep in mind that in MIPS, the instruction following a jump or branch instruction
is executed \IT{before} the jump/branch instruction itself.

As a consequence, branch instructions always swap places with the instruction executed immediately beforehand.


In practice, functions which merely return 1 (\IT{true}) or 0 (\IT{false}) are very frequent.

The smallest ever of the standard UNIX utilities, \IT{/bin/true} and \IT{/bin/false} return 0 and 1 respectively, as an exit code.
(Zero as an exit code usually means success, non-zero means error.)
}
\RU{\subsubsection{std::string}
\myindex{\Cpp!STL!std::string}
\label{std_string}

\myparagraph{Как устроена структура}

Многие строковые библиотеки \InSqBrackets{\CNotes 2.2} обеспечивают структуру содержащую ссылку 
на буфер собственно со строкой, переменная всегда содержащую длину строки 
(что очень удобно для массы функций \InSqBrackets{\CNotes 2.2.1}) и переменную содержащую текущий размер буфера.

Строка в буфере обыкновенно оканчивается нулем: это для того чтобы указатель на буфер можно было
передавать в функции требующие на вход обычную сишную \ac{ASCIIZ}-строку.

Стандарт \Cpp не описывает, как именно нужно реализовывать std::string,
но, как правило, они реализованы как описано выше, с небольшими дополнениями.

Строки в \Cpp это не класс (как, например, QString в Qt), а темплейт (basic\_string), 
это сделано для того чтобы поддерживать 
строки содержащие разного типа символы: как минимум \Tchar и \IT{wchar\_t}.

Так что, std::string это класс с базовым типом \Tchar.

А std::wstring это класс с базовым типом \IT{wchar\_t}.

\mysubparagraph{MSVC}

В реализации MSVC, вместо ссылки на буфер может содержаться сам буфер (если строка короче 16-и символов).

Это означает, что каждая короткая строка будет занимать в памяти по крайней мере $16 + 4 + 4 = 24$ 
байт для 32-битной среды либо $16 + 8 + 8 = 32$ 
байта в 64-битной, а если строка длиннее 16-и символов, то прибавьте еще длину самой строки.

\lstinputlisting[caption=пример для MSVC,style=customc]{\CURPATH/STL/string/MSVC_RU.cpp}

Собственно, из этого исходника почти всё ясно.

Несколько замечаний:

Если строка короче 16-и символов, 
то отдельный буфер для строки в \glslink{heap}{куче} выделяться не будет.

Это удобно потому что на практике, основная часть строк действительно короткие.
Вероятно, разработчики в Microsoft выбрали размер в 16 символов как разумный баланс.

Теперь очень важный момент в конце функции main(): мы не пользуемся методом c\_str(), тем не менее,
если это скомпилировать и запустить, то обе строки появятся в консоли!

Работает это вот почему.

В первом случае строка короче 16-и символов и в начале объекта std::string (его можно рассматривать
просто как структуру) расположен буфер с этой строкой.
\printf трактует указатель как указатель на массив символов оканчивающийся нулем и поэтому всё работает.

Вывод второй строки (длиннее 16-и символов) даже еще опаснее: это вообще типичная программистская ошибка 
(или опечатка), забыть дописать c\_str().
Это работает потому что в это время в начале структуры расположен указатель на буфер.
Это может надолго остаться незамеченным: до тех пока там не появится строка 
короче 16-и символов, тогда процесс упадет.

\mysubparagraph{GCC}

В реализации GCC в структуре есть еще одна переменная --- reference count.

Интересно, что указатель на экземпляр класса std::string в GCC указывает не на начало самой структуры, 
а на указатель на буфера.
В libstdc++-v3\textbackslash{}include\textbackslash{}bits\textbackslash{}basic\_string.h 
мы можем прочитать что это сделано для удобства отладки:

\begin{lstlisting}
   *  The reason you want _M_data pointing to the character %array and
   *  not the _Rep is so that the debugger can see the string
   *  contents. (Probably we should add a non-inline member to get
   *  the _Rep for the debugger to use, so users can check the actual
   *  string length.)
\end{lstlisting}

\href{http://go.yurichev.com/17085}{исходный код basic\_string.h}

В нашем примере мы учитываем это:

\lstinputlisting[caption=пример для GCC,style=customc]{\CURPATH/STL/string/GCC_RU.cpp}

Нужны еще небольшие хаки чтобы сымитировать типичную ошибку, которую мы уже видели выше, из-за
более ужесточенной проверки типов в GCC, тем не менее, printf() работает и здесь без c\_str().

\myparagraph{Чуть более сложный пример}

\lstinputlisting[style=customc]{\CURPATH/STL/string/3.cpp}

\lstinputlisting[caption=MSVC 2012,style=customasmx86]{\CURPATH/STL/string/3_MSVC_RU.asm}

Собственно, компилятор не конструирует строки статически: да в общем-то и как
это возможно, если буфер с ней нужно хранить в \glslink{heap}{куче}?

Вместо этого в сегменте данных хранятся обычные \ac{ASCIIZ}-строки, а позже, во время выполнения, 
при помощи метода \q{assign}, конструируются строки s1 и s2
.
При помощи \TT{operator+}, создается строка s3.

Обратите внимание на то что вызов метода c\_str() отсутствует,
потому что его код достаточно короткий и компилятор вставил его прямо здесь:
если строка короче 16-и байт, то в регистре EAX остается указатель на буфер,
а если длиннее, то из этого же места достается адрес на буфер расположенный в \glslink{heap}{куче}.

Далее следуют вызовы трех деструкторов, причем, они вызываются только если строка длиннее 16-и байт:
тогда нужно освободить буфера в \glslink{heap}{куче}.
В противном случае, так как все три объекта std::string хранятся в стеке,
они освобождаются автоматически после выхода из функции.

Следовательно, работа с короткими строками более быстрая из-за м\'{е}ньшего обращения к \glslink{heap}{куче}.

Код на GCC даже проще (из-за того, что в GCC, как мы уже видели, не реализована возможность хранить короткую
строку прямо в структуре):

% TODO1 comment each function meaning
\lstinputlisting[caption=GCC 4.8.1,style=customasmx86]{\CURPATH/STL/string/3_GCC_RU.s}

Можно заметить, что в деструкторы передается не указатель на объект,
а указатель на место за 12 байт (или 3 слова) перед ним, то есть, на настоящее начало структуры.

\myparagraph{std::string как глобальная переменная}
\label{sec:std_string_as_global_variable}

Опытные программисты на \Cpp знают, что глобальные переменные \ac{STL}-типов вполне можно объявлять.

Да, действительно:

\lstinputlisting[style=customc]{\CURPATH/STL/string/5.cpp}

Но как и где будет вызываться конструктор \TT{std::string}?

На самом деле, эта переменная будет инициализирована даже перед началом \main.

\lstinputlisting[caption=MSVC 2012: здесь конструируется глобальная переменная{,} а также регистрируется её деструктор,style=customasmx86]{\CURPATH/STL/string/5_MSVC_p2.asm}

\lstinputlisting[caption=MSVC 2012: здесь глобальная переменная используется в \main,style=customasmx86]{\CURPATH/STL/string/5_MSVC_p1.asm}

\lstinputlisting[caption=MSVC 2012: эта функция-деструктор вызывается перед выходом,style=customasmx86]{\CURPATH/STL/string/5_MSVC_p3.asm}

\myindex{\CStandardLibrary!atexit()}
В реальности, из \ac{CRT}, еще до вызова main(), вызывается специальная функция,
в которой перечислены все конструкторы подобных переменных.
Более того: при помощи atexit() регистрируется функция, которая будет вызвана в конце работы программы:
в этой функции компилятор собирает вызовы деструкторов всех подобных глобальных переменных.

GCC работает похожим образом:

\lstinputlisting[caption=GCC 4.8.1,style=customasmx86]{\CURPATH/STL/string/5_GCC.s}

Но он не выделяет отдельной функции в которой будут собраны деструкторы: 
каждый деструктор передается в atexit() по одному.

% TODO а если глобальная STL-переменная в другом модуле? надо проверить.

}

\EN{\section{Returning Values}
\label{ret_val_func}

Another simple function is the one that simply returns a constant value:

\lstinputlisting[caption=\EN{\CCpp Code},style=customc]{patterns/011_ret/1.c}

Let's compile it.

\subsection{x86}

Here's what both the GCC and MSVC compilers produce (with optimization) on the x86 platform:

\lstinputlisting[caption=\Optimizing GCC/MSVC (\assemblyOutput),style=customasmx86]{patterns/011_ret/1.s}

\myindex{x86!\Instructions!RET}
There are just two instructions: the first places the value 123 into the \EAX register,
which is used by convention for storing the return
value, and the second one is \RET, which returns execution to the \gls{caller}.

The caller will take the result from the \EAX register.

\subsection{ARM}

There are a few differences on the ARM platform:

\lstinputlisting[caption=\OptimizingKeilVI (\ARMMode) ASM Output,style=customasmARM]{patterns/011_ret/1_Keil_ARM_O3.s}

ARM uses the register \Reg{0} for returning the results of functions, so 123 is copied into \Reg{0}.

\myindex{ARM!\Instructions!MOV}
\myindex{x86!\Instructions!MOV}
It is worth noting that \MOV is a misleading name for the instruction in both the x86 and ARM \ac{ISA}s.

The data is not in fact \IT{moved}, but \IT{copied}.

\subsection{MIPS}

\label{MIPS_leaf_function_ex1}

The GCC assembly output below lists registers by number:

\lstinputlisting[caption=\Optimizing GCC 4.4.5 (\assemblyOutput),style=customasmMIPS]{patterns/011_ret/MIPS.s}

\dots while \IDA does it by their pseudo names:

\lstinputlisting[caption=\Optimizing GCC 4.4.5 (IDA),style=customasmMIPS]{patterns/011_ret/MIPS_IDA.lst}

The \$2 (or \$V0) register is used to store the function's return value.
\myindex{MIPS!\Pseudoinstructions!LI}
\INS{LI} stands for ``Load Immediate'' and is the MIPS equivalent to \MOV.

\myindex{MIPS!\Instructions!J}
The other instruction is the jump instruction (J or JR) which returns the execution flow to the \gls{caller}.

\myindex{MIPS!Branch delay slot}
You might be wondering why the positions of the load instruction (LI) and the jump instruction (J or JR) are swapped. This is due to a \ac{RISC} feature called ``branch delay slot''.

The reason this happens is a quirk in the architecture of some RISC \ac{ISA}s and isn't important for our
purposes---we must simply keep in mind that in MIPS, the instruction following a jump or branch instruction
is executed \IT{before} the jump/branch instruction itself.

As a consequence, branch instructions always swap places with the instruction executed immediately beforehand.


In practice, functions which merely return 1 (\IT{true}) or 0 (\IT{false}) are very frequent.

The smallest ever of the standard UNIX utilities, \IT{/bin/true} and \IT{/bin/false} return 0 and 1 respectively, as an exit code.
(Zero as an exit code usually means success, non-zero means error.)
}
\RU{\subsubsection{std::string}
\myindex{\Cpp!STL!std::string}
\label{std_string}

\myparagraph{Как устроена структура}

Многие строковые библиотеки \InSqBrackets{\CNotes 2.2} обеспечивают структуру содержащую ссылку 
на буфер собственно со строкой, переменная всегда содержащую длину строки 
(что очень удобно для массы функций \InSqBrackets{\CNotes 2.2.1}) и переменную содержащую текущий размер буфера.

Строка в буфере обыкновенно оканчивается нулем: это для того чтобы указатель на буфер можно было
передавать в функции требующие на вход обычную сишную \ac{ASCIIZ}-строку.

Стандарт \Cpp не описывает, как именно нужно реализовывать std::string,
но, как правило, они реализованы как описано выше, с небольшими дополнениями.

Строки в \Cpp это не класс (как, например, QString в Qt), а темплейт (basic\_string), 
это сделано для того чтобы поддерживать 
строки содержащие разного типа символы: как минимум \Tchar и \IT{wchar\_t}.

Так что, std::string это класс с базовым типом \Tchar.

А std::wstring это класс с базовым типом \IT{wchar\_t}.

\mysubparagraph{MSVC}

В реализации MSVC, вместо ссылки на буфер может содержаться сам буфер (если строка короче 16-и символов).

Это означает, что каждая короткая строка будет занимать в памяти по крайней мере $16 + 4 + 4 = 24$ 
байт для 32-битной среды либо $16 + 8 + 8 = 32$ 
байта в 64-битной, а если строка длиннее 16-и символов, то прибавьте еще длину самой строки.

\lstinputlisting[caption=пример для MSVC,style=customc]{\CURPATH/STL/string/MSVC_RU.cpp}

Собственно, из этого исходника почти всё ясно.

Несколько замечаний:

Если строка короче 16-и символов, 
то отдельный буфер для строки в \glslink{heap}{куче} выделяться не будет.

Это удобно потому что на практике, основная часть строк действительно короткие.
Вероятно, разработчики в Microsoft выбрали размер в 16 символов как разумный баланс.

Теперь очень важный момент в конце функции main(): мы не пользуемся методом c\_str(), тем не менее,
если это скомпилировать и запустить, то обе строки появятся в консоли!

Работает это вот почему.

В первом случае строка короче 16-и символов и в начале объекта std::string (его можно рассматривать
просто как структуру) расположен буфер с этой строкой.
\printf трактует указатель как указатель на массив символов оканчивающийся нулем и поэтому всё работает.

Вывод второй строки (длиннее 16-и символов) даже еще опаснее: это вообще типичная программистская ошибка 
(или опечатка), забыть дописать c\_str().
Это работает потому что в это время в начале структуры расположен указатель на буфер.
Это может надолго остаться незамеченным: до тех пока там не появится строка 
короче 16-и символов, тогда процесс упадет.

\mysubparagraph{GCC}

В реализации GCC в структуре есть еще одна переменная --- reference count.

Интересно, что указатель на экземпляр класса std::string в GCC указывает не на начало самой структуры, 
а на указатель на буфера.
В libstdc++-v3\textbackslash{}include\textbackslash{}bits\textbackslash{}basic\_string.h 
мы можем прочитать что это сделано для удобства отладки:

\begin{lstlisting}
   *  The reason you want _M_data pointing to the character %array and
   *  not the _Rep is so that the debugger can see the string
   *  contents. (Probably we should add a non-inline member to get
   *  the _Rep for the debugger to use, so users can check the actual
   *  string length.)
\end{lstlisting}

\href{http://go.yurichev.com/17085}{исходный код basic\_string.h}

В нашем примере мы учитываем это:

\lstinputlisting[caption=пример для GCC,style=customc]{\CURPATH/STL/string/GCC_RU.cpp}

Нужны еще небольшие хаки чтобы сымитировать типичную ошибку, которую мы уже видели выше, из-за
более ужесточенной проверки типов в GCC, тем не менее, printf() работает и здесь без c\_str().

\myparagraph{Чуть более сложный пример}

\lstinputlisting[style=customc]{\CURPATH/STL/string/3.cpp}

\lstinputlisting[caption=MSVC 2012,style=customasmx86]{\CURPATH/STL/string/3_MSVC_RU.asm}

Собственно, компилятор не конструирует строки статически: да в общем-то и как
это возможно, если буфер с ней нужно хранить в \glslink{heap}{куче}?

Вместо этого в сегменте данных хранятся обычные \ac{ASCIIZ}-строки, а позже, во время выполнения, 
при помощи метода \q{assign}, конструируются строки s1 и s2
.
При помощи \TT{operator+}, создается строка s3.

Обратите внимание на то что вызов метода c\_str() отсутствует,
потому что его код достаточно короткий и компилятор вставил его прямо здесь:
если строка короче 16-и байт, то в регистре EAX остается указатель на буфер,
а если длиннее, то из этого же места достается адрес на буфер расположенный в \glslink{heap}{куче}.

Далее следуют вызовы трех деструкторов, причем, они вызываются только если строка длиннее 16-и байт:
тогда нужно освободить буфера в \glslink{heap}{куче}.
В противном случае, так как все три объекта std::string хранятся в стеке,
они освобождаются автоматически после выхода из функции.

Следовательно, работа с короткими строками более быстрая из-за м\'{е}ньшего обращения к \glslink{heap}{куче}.

Код на GCC даже проще (из-за того, что в GCC, как мы уже видели, не реализована возможность хранить короткую
строку прямо в структуре):

% TODO1 comment each function meaning
\lstinputlisting[caption=GCC 4.8.1,style=customasmx86]{\CURPATH/STL/string/3_GCC_RU.s}

Можно заметить, что в деструкторы передается не указатель на объект,
а указатель на место за 12 байт (или 3 слова) перед ним, то есть, на настоящее начало структуры.

\myparagraph{std::string как глобальная переменная}
\label{sec:std_string_as_global_variable}

Опытные программисты на \Cpp знают, что глобальные переменные \ac{STL}-типов вполне можно объявлять.

Да, действительно:

\lstinputlisting[style=customc]{\CURPATH/STL/string/5.cpp}

Но как и где будет вызываться конструктор \TT{std::string}?

На самом деле, эта переменная будет инициализирована даже перед началом \main.

\lstinputlisting[caption=MSVC 2012: здесь конструируется глобальная переменная{,} а также регистрируется её деструктор,style=customasmx86]{\CURPATH/STL/string/5_MSVC_p2.asm}

\lstinputlisting[caption=MSVC 2012: здесь глобальная переменная используется в \main,style=customasmx86]{\CURPATH/STL/string/5_MSVC_p1.asm}

\lstinputlisting[caption=MSVC 2012: эта функция-деструктор вызывается перед выходом,style=customasmx86]{\CURPATH/STL/string/5_MSVC_p3.asm}

\myindex{\CStandardLibrary!atexit()}
В реальности, из \ac{CRT}, еще до вызова main(), вызывается специальная функция,
в которой перечислены все конструкторы подобных переменных.
Более того: при помощи atexit() регистрируется функция, которая будет вызвана в конце работы программы:
в этой функции компилятор собирает вызовы деструкторов всех подобных глобальных переменных.

GCC работает похожим образом:

\lstinputlisting[caption=GCC 4.8.1,style=customasmx86]{\CURPATH/STL/string/5_GCC.s}

Но он не выделяет отдельной функции в которой будут собраны деструкторы: 
каждый деструктор передается в atexit() по одному.

% TODO а если глобальная STL-переменная в другом модуле? надо проверить.

}
\DE{\subsection{Einfachste XOR-Verschlüsselung überhaupt}

Ich habe einmal eine Software gesehen, bei der alle Debugging-Ausgaben mit XOR mit dem Wert 3
verschlüsselt wurden. Mit anderen Worten, die beiden niedrigsten Bits aller Buchstaben wurden invertiert.

``Hello, world'' wurde zu ``Kfool/\#tlqog'':

\begin{lstlisting}
#!/usr/bin/python

msg="Hello, world!"

print "".join(map(lambda x: chr(ord(x)^3), msg))
\end{lstlisting}

Das ist eine ziemlich interessante Verschlüsselung (oder besser eine Verschleierung),
weil sie zwei wichtige Eigenschaften hat:
1) es ist eine einzige Funktion zum Verschlüsseln und entschlüsseln, sie muss nur wiederholt angewendet werden
2) die entstehenden Buchstaben befinden sich im druckbaren Bereich, also die ganze Zeichenkette kann ohne
Escape-Symbole im Code verwendet werden.

Die zweite Eigenschaft nutzt die Tatsache, dass alle druckbaren Zeichen in Reihen organisiert sind: 0x2x-0x7x,
und wenn die beiden niederwertigsten Bits invertiert werden, wird der Buchstabe um eine oder drei Stellen nach
links oder rechts \IT{verschoben}, aber niemals in eine andere Reihe:

\begin{figure}[H]
\centering
\includegraphics[width=0.7\textwidth]{ascii_clean.png}
\caption{7-Bit \ac{ASCII} Tabelle in Emacs}
\end{figure}

\dots mit dem Zeichen 0x7F als einziger Ausnahme.

Im Folgenden werden also beispielsweise die Zeichen A-Z \IT{verschlüsselt}:

\begin{lstlisting}
#!/usr/bin/python

msg="@ABCDEFGHIJKLMNO"

print "".join(map(lambda x: chr(ord(x)^3), msg))
\end{lstlisting}

Ergebnis:
% FIXME \verb  --  relevant comment for German?
\begin{lstlisting}
CBA@GFEDKJIHONML
\end{lstlisting}

Es sieht so aus als würden die Zeichen ``@'' und ``C'' sowie ``B'' und ``A'' vertauscht werden.

Hier ist noch ein interessantes Beispiel, in dem gezeigt wird, wie die Eigenschaften von XOR
ausgenutzt werden können: Exakt den gleichen Effekt, dass druckbare Zeichen auch druckbar bleiben,
kann man dadurch erzielen, dass irgendeine Kombination der niedrigsten vier Bits invertiert wird.
}

\EN{\section{Returning Values}
\label{ret_val_func}

Another simple function is the one that simply returns a constant value:

\lstinputlisting[caption=\EN{\CCpp Code},style=customc]{patterns/011_ret/1.c}

Let's compile it.

\subsection{x86}

Here's what both the GCC and MSVC compilers produce (with optimization) on the x86 platform:

\lstinputlisting[caption=\Optimizing GCC/MSVC (\assemblyOutput),style=customasmx86]{patterns/011_ret/1.s}

\myindex{x86!\Instructions!RET}
There are just two instructions: the first places the value 123 into the \EAX register,
which is used by convention for storing the return
value, and the second one is \RET, which returns execution to the \gls{caller}.

The caller will take the result from the \EAX register.

\subsection{ARM}

There are a few differences on the ARM platform:

\lstinputlisting[caption=\OptimizingKeilVI (\ARMMode) ASM Output,style=customasmARM]{patterns/011_ret/1_Keil_ARM_O3.s}

ARM uses the register \Reg{0} for returning the results of functions, so 123 is copied into \Reg{0}.

\myindex{ARM!\Instructions!MOV}
\myindex{x86!\Instructions!MOV}
It is worth noting that \MOV is a misleading name for the instruction in both the x86 and ARM \ac{ISA}s.

The data is not in fact \IT{moved}, but \IT{copied}.

\subsection{MIPS}

\label{MIPS_leaf_function_ex1}

The GCC assembly output below lists registers by number:

\lstinputlisting[caption=\Optimizing GCC 4.4.5 (\assemblyOutput),style=customasmMIPS]{patterns/011_ret/MIPS.s}

\dots while \IDA does it by their pseudo names:

\lstinputlisting[caption=\Optimizing GCC 4.4.5 (IDA),style=customasmMIPS]{patterns/011_ret/MIPS_IDA.lst}

The \$2 (or \$V0) register is used to store the function's return value.
\myindex{MIPS!\Pseudoinstructions!LI}
\INS{LI} stands for ``Load Immediate'' and is the MIPS equivalent to \MOV.

\myindex{MIPS!\Instructions!J}
The other instruction is the jump instruction (J or JR) which returns the execution flow to the \gls{caller}.

\myindex{MIPS!Branch delay slot}
You might be wondering why the positions of the load instruction (LI) and the jump instruction (J or JR) are swapped. This is due to a \ac{RISC} feature called ``branch delay slot''.

The reason this happens is a quirk in the architecture of some RISC \ac{ISA}s and isn't important for our
purposes---we must simply keep in mind that in MIPS, the instruction following a jump or branch instruction
is executed \IT{before} the jump/branch instruction itself.

As a consequence, branch instructions always swap places with the instruction executed immediately beforehand.


In practice, functions which merely return 1 (\IT{true}) or 0 (\IT{false}) are very frequent.

The smallest ever of the standard UNIX utilities, \IT{/bin/true} and \IT{/bin/false} return 0 and 1 respectively, as an exit code.
(Zero as an exit code usually means success, non-zero means error.)
}
\RU{\subsubsection{std::string}
\myindex{\Cpp!STL!std::string}
\label{std_string}

\myparagraph{Как устроена структура}

Многие строковые библиотеки \InSqBrackets{\CNotes 2.2} обеспечивают структуру содержащую ссылку 
на буфер собственно со строкой, переменная всегда содержащую длину строки 
(что очень удобно для массы функций \InSqBrackets{\CNotes 2.2.1}) и переменную содержащую текущий размер буфера.

Строка в буфере обыкновенно оканчивается нулем: это для того чтобы указатель на буфер можно было
передавать в функции требующие на вход обычную сишную \ac{ASCIIZ}-строку.

Стандарт \Cpp не описывает, как именно нужно реализовывать std::string,
но, как правило, они реализованы как описано выше, с небольшими дополнениями.

Строки в \Cpp это не класс (как, например, QString в Qt), а темплейт (basic\_string), 
это сделано для того чтобы поддерживать 
строки содержащие разного типа символы: как минимум \Tchar и \IT{wchar\_t}.

Так что, std::string это класс с базовым типом \Tchar.

А std::wstring это класс с базовым типом \IT{wchar\_t}.

\mysubparagraph{MSVC}

В реализации MSVC, вместо ссылки на буфер может содержаться сам буфер (если строка короче 16-и символов).

Это означает, что каждая короткая строка будет занимать в памяти по крайней мере $16 + 4 + 4 = 24$ 
байт для 32-битной среды либо $16 + 8 + 8 = 32$ 
байта в 64-битной, а если строка длиннее 16-и символов, то прибавьте еще длину самой строки.

\lstinputlisting[caption=пример для MSVC,style=customc]{\CURPATH/STL/string/MSVC_RU.cpp}

Собственно, из этого исходника почти всё ясно.

Несколько замечаний:

Если строка короче 16-и символов, 
то отдельный буфер для строки в \glslink{heap}{куче} выделяться не будет.

Это удобно потому что на практике, основная часть строк действительно короткие.
Вероятно, разработчики в Microsoft выбрали размер в 16 символов как разумный баланс.

Теперь очень важный момент в конце функции main(): мы не пользуемся методом c\_str(), тем не менее,
если это скомпилировать и запустить, то обе строки появятся в консоли!

Работает это вот почему.

В первом случае строка короче 16-и символов и в начале объекта std::string (его можно рассматривать
просто как структуру) расположен буфер с этой строкой.
\printf трактует указатель как указатель на массив символов оканчивающийся нулем и поэтому всё работает.

Вывод второй строки (длиннее 16-и символов) даже еще опаснее: это вообще типичная программистская ошибка 
(или опечатка), забыть дописать c\_str().
Это работает потому что в это время в начале структуры расположен указатель на буфер.
Это может надолго остаться незамеченным: до тех пока там не появится строка 
короче 16-и символов, тогда процесс упадет.

\mysubparagraph{GCC}

В реализации GCC в структуре есть еще одна переменная --- reference count.

Интересно, что указатель на экземпляр класса std::string в GCC указывает не на начало самой структуры, 
а на указатель на буфера.
В libstdc++-v3\textbackslash{}include\textbackslash{}bits\textbackslash{}basic\_string.h 
мы можем прочитать что это сделано для удобства отладки:

\begin{lstlisting}
   *  The reason you want _M_data pointing to the character %array and
   *  not the _Rep is so that the debugger can see the string
   *  contents. (Probably we should add a non-inline member to get
   *  the _Rep for the debugger to use, so users can check the actual
   *  string length.)
\end{lstlisting}

\href{http://go.yurichev.com/17085}{исходный код basic\_string.h}

В нашем примере мы учитываем это:

\lstinputlisting[caption=пример для GCC,style=customc]{\CURPATH/STL/string/GCC_RU.cpp}

Нужны еще небольшие хаки чтобы сымитировать типичную ошибку, которую мы уже видели выше, из-за
более ужесточенной проверки типов в GCC, тем не менее, printf() работает и здесь без c\_str().

\myparagraph{Чуть более сложный пример}

\lstinputlisting[style=customc]{\CURPATH/STL/string/3.cpp}

\lstinputlisting[caption=MSVC 2012,style=customasmx86]{\CURPATH/STL/string/3_MSVC_RU.asm}

Собственно, компилятор не конструирует строки статически: да в общем-то и как
это возможно, если буфер с ней нужно хранить в \glslink{heap}{куче}?

Вместо этого в сегменте данных хранятся обычные \ac{ASCIIZ}-строки, а позже, во время выполнения, 
при помощи метода \q{assign}, конструируются строки s1 и s2
.
При помощи \TT{operator+}, создается строка s3.

Обратите внимание на то что вызов метода c\_str() отсутствует,
потому что его код достаточно короткий и компилятор вставил его прямо здесь:
если строка короче 16-и байт, то в регистре EAX остается указатель на буфер,
а если длиннее, то из этого же места достается адрес на буфер расположенный в \glslink{heap}{куче}.

Далее следуют вызовы трех деструкторов, причем, они вызываются только если строка длиннее 16-и байт:
тогда нужно освободить буфера в \glslink{heap}{куче}.
В противном случае, так как все три объекта std::string хранятся в стеке,
они освобождаются автоматически после выхода из функции.

Следовательно, работа с короткими строками более быстрая из-за м\'{е}ньшего обращения к \glslink{heap}{куче}.

Код на GCC даже проще (из-за того, что в GCC, как мы уже видели, не реализована возможность хранить короткую
строку прямо в структуре):

% TODO1 comment each function meaning
\lstinputlisting[caption=GCC 4.8.1,style=customasmx86]{\CURPATH/STL/string/3_GCC_RU.s}

Можно заметить, что в деструкторы передается не указатель на объект,
а указатель на место за 12 байт (или 3 слова) перед ним, то есть, на настоящее начало структуры.

\myparagraph{std::string как глобальная переменная}
\label{sec:std_string_as_global_variable}

Опытные программисты на \Cpp знают, что глобальные переменные \ac{STL}-типов вполне можно объявлять.

Да, действительно:

\lstinputlisting[style=customc]{\CURPATH/STL/string/5.cpp}

Но как и где будет вызываться конструктор \TT{std::string}?

На самом деле, эта переменная будет инициализирована даже перед началом \main.

\lstinputlisting[caption=MSVC 2012: здесь конструируется глобальная переменная{,} а также регистрируется её деструктор,style=customasmx86]{\CURPATH/STL/string/5_MSVC_p2.asm}

\lstinputlisting[caption=MSVC 2012: здесь глобальная переменная используется в \main,style=customasmx86]{\CURPATH/STL/string/5_MSVC_p1.asm}

\lstinputlisting[caption=MSVC 2012: эта функция-деструктор вызывается перед выходом,style=customasmx86]{\CURPATH/STL/string/5_MSVC_p3.asm}

\myindex{\CStandardLibrary!atexit()}
В реальности, из \ac{CRT}, еще до вызова main(), вызывается специальная функция,
в которой перечислены все конструкторы подобных переменных.
Более того: при помощи atexit() регистрируется функция, которая будет вызвана в конце работы программы:
в этой функции компилятор собирает вызовы деструкторов всех подобных глобальных переменных.

GCC работает похожим образом:

\lstinputlisting[caption=GCC 4.8.1,style=customasmx86]{\CURPATH/STL/string/5_GCC.s}

Но он не выделяет отдельной функции в которой будут собраны деструкторы: 
каждый деструктор передается в atexit() по одному.

% TODO а если глобальная STL-переменная в другом модуле? надо проверить.

}
\DE{\subsection{Einfachste XOR-Verschlüsselung überhaupt}

Ich habe einmal eine Software gesehen, bei der alle Debugging-Ausgaben mit XOR mit dem Wert 3
verschlüsselt wurden. Mit anderen Worten, die beiden niedrigsten Bits aller Buchstaben wurden invertiert.

``Hello, world'' wurde zu ``Kfool/\#tlqog'':

\begin{lstlisting}
#!/usr/bin/python

msg="Hello, world!"

print "".join(map(lambda x: chr(ord(x)^3), msg))
\end{lstlisting}

Das ist eine ziemlich interessante Verschlüsselung (oder besser eine Verschleierung),
weil sie zwei wichtige Eigenschaften hat:
1) es ist eine einzige Funktion zum Verschlüsseln und entschlüsseln, sie muss nur wiederholt angewendet werden
2) die entstehenden Buchstaben befinden sich im druckbaren Bereich, also die ganze Zeichenkette kann ohne
Escape-Symbole im Code verwendet werden.

Die zweite Eigenschaft nutzt die Tatsache, dass alle druckbaren Zeichen in Reihen organisiert sind: 0x2x-0x7x,
und wenn die beiden niederwertigsten Bits invertiert werden, wird der Buchstabe um eine oder drei Stellen nach
links oder rechts \IT{verschoben}, aber niemals in eine andere Reihe:

\begin{figure}[H]
\centering
\includegraphics[width=0.7\textwidth]{ascii_clean.png}
\caption{7-Bit \ac{ASCII} Tabelle in Emacs}
\end{figure}

\dots mit dem Zeichen 0x7F als einziger Ausnahme.

Im Folgenden werden also beispielsweise die Zeichen A-Z \IT{verschlüsselt}:

\begin{lstlisting}
#!/usr/bin/python

msg="@ABCDEFGHIJKLMNO"

print "".join(map(lambda x: chr(ord(x)^3), msg))
\end{lstlisting}

Ergebnis:
% FIXME \verb  --  relevant comment for German?
\begin{lstlisting}
CBA@GFEDKJIHONML
\end{lstlisting}

Es sieht so aus als würden die Zeichen ``@'' und ``C'' sowie ``B'' und ``A'' vertauscht werden.

Hier ist noch ein interessantes Beispiel, in dem gezeigt wird, wie die Eigenschaften von XOR
ausgenutzt werden können: Exakt den gleichen Effekt, dass druckbare Zeichen auch druckbar bleiben,
kann man dadurch erzielen, dass irgendeine Kombination der niedrigsten vier Bits invertiert wird.
}

\ifdefined\SPANISH
\chapter{Patrones de código}
\fi % SPANISH

\ifdefined\GERMAN
\chapter{Code-Muster}
\fi % GERMAN

\ifdefined\ENGLISH
\chapter{Code Patterns}
\fi % ENGLISH

\ifdefined\ITALIAN
\chapter{Forme di codice}
\fi % ITALIAN

\ifdefined\RUSSIAN
\chapter{Образцы кода}
\fi % RUSSIAN

\ifdefined\BRAZILIAN
\chapter{Padrões de códigos}
\fi % BRAZILIAN

\ifdefined\THAI
\chapter{รูปแบบของโค้ด}
\fi % THAI

\ifdefined\FRENCH
\chapter{Modèle de code}
\fi % FRENCH

\ifdefined\POLISH
\chapter{\PLph{}}
\fi % POLISH

% sections
\EN{\input{patterns/patterns_opt_dbg_EN}}
\ES{\input{patterns/patterns_opt_dbg_ES}}
\ITA{\input{patterns/patterns_opt_dbg_ITA}}
\PTBR{\input{patterns/patterns_opt_dbg_PTBR}}
\RU{\input{patterns/patterns_opt_dbg_RU}}
\THA{\input{patterns/patterns_opt_dbg_THA}}
\DE{\input{patterns/patterns_opt_dbg_DE}}
\FR{\input{patterns/patterns_opt_dbg_FR}}
\PL{\input{patterns/patterns_opt_dbg_PL}}

\RU{\section{Некоторые базовые понятия}}
\EN{\section{Some basics}}
\DE{\section{Einige Grundlagen}}
\FR{\section{Quelques bases}}
\ES{\section{\ESph{}}}
\ITA{\section{Alcune basi teoriche}}
\PTBR{\section{\PTBRph{}}}
\THA{\section{\THAph{}}}
\PL{\section{\PLph{}}}

% sections:
\EN{\input{patterns/intro_CPU_ISA_EN}}
\ES{\input{patterns/intro_CPU_ISA_ES}}
\ITA{\input{patterns/intro_CPU_ISA_ITA}}
\PTBR{\input{patterns/intro_CPU_ISA_PTBR}}
\RU{\input{patterns/intro_CPU_ISA_RU}}
\DE{\input{patterns/intro_CPU_ISA_DE}}
\FR{\input{patterns/intro_CPU_ISA_FR}}
\PL{\input{patterns/intro_CPU_ISA_PL}}

\EN{\input{patterns/numeral_EN}}
\RU{\input{patterns/numeral_RU}}
\ITA{\input{patterns/numeral_ITA}}
\DE{\input{patterns/numeral_DE}}
\FR{\input{patterns/numeral_FR}}
\PL{\input{patterns/numeral_PL}}

% chapters
\input{patterns/00_empty/main}
\input{patterns/011_ret/main}
\input{patterns/01_helloworld/main}
\input{patterns/015_prolog_epilogue/main}
\input{patterns/02_stack/main}
\input{patterns/03_printf/main}
\input{patterns/04_scanf/main}
\input{patterns/05_passing_arguments/main}
\input{patterns/06_return_results/main}
\input{patterns/061_pointers/main}
\input{patterns/065_GOTO/main}
\input{patterns/07_jcc/main}
\input{patterns/08_switch/main}
\input{patterns/09_loops/main}
\input{patterns/10_strings/main}
\input{patterns/11_arith_optimizations/main}
\input{patterns/12_FPU/main}
\input{patterns/13_arrays/main}
\input{patterns/14_bitfields/main}
\EN{\input{patterns/145_LCG/main_EN}}
\RU{\input{patterns/145_LCG/main_RU}}
\input{patterns/15_structs/main}
\input{patterns/17_unions/main}
\input{patterns/18_pointers_to_functions/main}
\input{patterns/185_64bit_in_32_env/main}

\EN{\input{patterns/19_SIMD/main_EN}}
\RU{\input{patterns/19_SIMD/main_RU}}
\DE{\input{patterns/19_SIMD/main_DE}}

\EN{\input{patterns/20_x64/main_EN}}
\RU{\input{patterns/20_x64/main_RU}}

\EN{\input{patterns/205_floating_SIMD/main_EN}}
\RU{\input{patterns/205_floating_SIMD/main_RU}}
\DE{\input{patterns/205_floating_SIMD/main_DE}}

\EN{\input{patterns/ARM/main_EN}}
\RU{\input{patterns/ARM/main_RU}}
\DE{\input{patterns/ARM/main_DE}}

\input{patterns/MIPS/main}


\ifdefined\SPANISH
\chapter{Patrones de código}
\fi % SPANISH

\ifdefined\GERMAN
\chapter{Code-Muster}
\fi % GERMAN

\ifdefined\ENGLISH
\chapter{Code Patterns}
\fi % ENGLISH

\ifdefined\ITALIAN
\chapter{Forme di codice}
\fi % ITALIAN

\ifdefined\RUSSIAN
\chapter{Образцы кода}
\fi % RUSSIAN

\ifdefined\BRAZILIAN
\chapter{Padrões de códigos}
\fi % BRAZILIAN

\ifdefined\THAI
\chapter{รูปแบบของโค้ด}
\fi % THAI

\ifdefined\FRENCH
\chapter{Modèle de code}
\fi % FRENCH

\ifdefined\POLISH
\chapter{\PLph{}}
\fi % POLISH

% sections
\EN{\section{The method}

When the author of this book first started learning C and, later, \Cpp, he used to write small pieces of code, compile them,
and then look at the assembly language output. This made it very easy for him to understand what was going on in the code that he had written.
\footnote{In fact, he still does this when he can't understand what a particular bit of code does.}.
He did this so many times that the relationship between the \CCpp code and what the compiler produced was imprinted deeply in his mind.
It's now easy for him to imagine instantly a rough outline of a C code's appearance and function.
Perhaps this technique could be helpful for others.

%There are a lot of examples for both x86/x64 and ARM.
%Those who already familiar with one of architectures, may freely skim over pages.

By the way, there is a great website where you can do the same, with various compilers, instead of installing them on your box.
You can use it as well: \url{https://gcc.godbolt.org/}.

\section*{\Exercises}

When the author of this book studied assembly language, he also often compiled small C functions and then rewrote
them gradually to assembly, trying to make their code as short as possible.
This probably is not worth doing in real-world scenarios today,
because it's hard to compete with the latest compilers in terms of efficiency. It is, however, a very good way to gain a better understanding of assembly.
Feel free, therefore, to take any assembly code from this book and try to make it shorter.
However, don't forget to test what you have written.

% rewrote to show that debug\release and optimisations levels are orthogonal concepts.
\section*{Optimization levels and debug information}

Source code can be compiled by different compilers with various optimization levels.
A typical compiler has about three such levels, where level zero means that optimization is completely disabled.
Optimization can also be targeted towards code size or code speed.
A non-optimizing compiler is faster and produces more understandable (albeit verbose) code,
whereas an optimizing compiler is slower and tries to produce code that runs faster (but is not necessarily more compact).
In addition to optimization levels, a compiler can include some debug information in the resulting file,
producing code that is easy to debug.
One of the important features of the ´debug' code is that it might contain links
between each line of the source code and its respective machine code address.
Optimizing compilers, on the other hand, tend to produce output where entire lines of source code
can be optimized away and thus not even be present in the resulting machine code.
Reverse engineers can encounter either version, simply because some developers turn on the compiler's optimization flags and others do not.
Because of this, we'll try to work on examples of both debug and release versions of the code featured in this book, wherever possible.

Sometimes some pretty ancient compilers are used in this book, in order to get the shortest (or simplest) possible code snippet.
}
\ES{\input{patterns/patterns_opt_dbg_ES}}
\ITA{\input{patterns/patterns_opt_dbg_ITA}}
\PTBR{\input{patterns/patterns_opt_dbg_PTBR}}
\RU{\input{patterns/patterns_opt_dbg_RU}}
\THA{\input{patterns/patterns_opt_dbg_THA}}
\DE{\input{patterns/patterns_opt_dbg_DE}}
\FR{\input{patterns/patterns_opt_dbg_FR}}
\PL{\input{patterns/patterns_opt_dbg_PL}}

\RU{\section{Некоторые базовые понятия}}
\EN{\section{Some basics}}
\DE{\section{Einige Grundlagen}}
\FR{\section{Quelques bases}}
\ES{\section{\ESph{}}}
\ITA{\section{Alcune basi teoriche}}
\PTBR{\section{\PTBRph{}}}
\THA{\section{\THAph{}}}
\PL{\section{\PLph{}}}

% sections:
\EN{\input{patterns/intro_CPU_ISA_EN}}
\ES{\input{patterns/intro_CPU_ISA_ES}}
\ITA{\input{patterns/intro_CPU_ISA_ITA}}
\PTBR{\input{patterns/intro_CPU_ISA_PTBR}}
\RU{\input{patterns/intro_CPU_ISA_RU}}
\DE{\input{patterns/intro_CPU_ISA_DE}}
\FR{\input{patterns/intro_CPU_ISA_FR}}
\PL{\input{patterns/intro_CPU_ISA_PL}}

\EN{\subsection{Numeral Systems}

Humans have become accustomed to a decimal numeral system, probably because almost everyone has 10 fingers.
Nevertheless, the number \q{10} has no significant meaning in science and mathematics.
The natural numeral system in digital electronics is binary: 0 is for an absence of current in the wire, and 1 for presence.
10 in binary is 2 in decimal, 100 in binary is 4 in decimal, and so on.

% This sentence is a bit unweildy - maybe try 'Our ten-digit system would be described as having a radix...' - Renaissance
If the numeral system has 10 digits, it has a \IT{radix} (or \IT{base}) of 10.
The binary numeral system has a \IT{radix} of 2.

Important things to recall:

1) A \IT{number} is a number, while a \IT{digit} is a term from writing systems, and is usually one character

% The original is 'number' is not changed; I think the intent is value, and changed it - Renaissance
2) The value of a number does not change when converted to another radix; only the writing notation for that value has changed (and therefore the way of representing it in \ac{RAM}).

\subsection{Converting From One Radix To Another}

Positional notation is used almost every numerical system. This means that a digit has weight relative to where it is placed inside of the larger number.
If 2 is placed at the rightmost place, it's 2, but if it's placed one digit before rightmost, it's 20.

What does $1234$ stand for?

$10^3 \cdot 1 + 10^2 \cdot 2 + 10^1 \cdot 3 + 1 \cdot 4 = 1234$ or
$1000 \cdot 1 + 100 \cdot 2 + 10 \cdot 3 + 4 = 1234$

It's the same story for binary numbers, but the base is 2 instead of 10.
What does 0b101011 stand for?

$2^5 \cdot 1 + 2^4 \cdot 0 + 2^3 \cdot 1 + 2^2 \cdot 0 + 2^1 \cdot 1 + 2^0 \cdot 1 = 43$ or
$32 \cdot 1 + 16 \cdot 0 + 8 \cdot 1 + 4 \cdot 0 + 2 \cdot 1 + 1 = 43$

There is such a thing as non-positional notation, such as the Roman numeral system.
\footnote{About numeric system evolution, see \InSqBrackets{\TAOCPvolII{}, 195--213.}}.
% Maybe add a sentence to fill in that X is always 10, and is therefore non-positional, even though putting an I before subtracts and after adds, and is in that sense positional
Perhaps, humankind switched to positional notation because it's easier to do basic operations (addition, multiplication, etc.) on paper by hand.

Binary numbers can be added, subtracted and so on in the very same as taught in schools, but only 2 digits are available.

Binary numbers are bulky when represented in source code and dumps, so that is where the hexadecimal numeral system can be useful.
A hexadecimal radix uses the digits 0..9, and also 6 Latin characters: A..F.
Each hexadecimal digit takes 4 bits or 4 binary digits, so it's very easy to convert from binary number to hexadecimal and back, even manually, in one's mind.

\begin{center}
\begin{longtable}{ | l | l | l | }
\hline
\HeaderColor hexadecimal & \HeaderColor binary & \HeaderColor decimal \\
\hline
0	&0000	&0 \\
1	&0001	&1 \\
2	&0010	&2 \\
3	&0011	&3 \\
4	&0100	&4 \\
5	&0101	&5 \\
6	&0110	&6 \\
7	&0111	&7 \\
8	&1000	&8 \\
9	&1001	&9 \\
A	&1010	&10 \\
B	&1011	&11 \\
C	&1100	&12 \\
D	&1101	&13 \\
E	&1110	&14 \\
F	&1111	&15 \\
\hline
\end{longtable}
\end{center}

How can one tell which radix is being used in a specific instance?

Decimal numbers are usually written as is, i.e., 1234. Some assemblers allow an identifier on decimal radix numbers, in which the number would be written with a "d" suffix: 1234d.

Binary numbers are sometimes prepended with the "0b" prefix: 0b100110111 (\ac{GCC} has a non-standard language extension for this\footnote{\url{https://gcc.gnu.org/onlinedocs/gcc/Binary-constants.html}}).
There is also another way: using a "b" suffix, for example: 100110111b.
This book tries to use the "0b" prefix consistently throughout the book for binary numbers.

Hexadecimal numbers are prepended with "0x" prefix in \CCpp and other \ac{PL}s: 0x1234ABCD.
Alternatively, they are given a "h" suffix: 1234ABCDh. This is common way of representing them in assemblers and debuggers.
In this convention, if the number is started with a Latin (A..F) digit, a 0 is added at the beginning: 0ABCDEFh.
There was also convention that was popular in 8-bit home computers era, using \$ prefix, like \$ABCD.
The book will try to stick to "0x" prefix throughout the book for hexadecimal numbers.

Should one learn to convert numbers mentally? A table of 1-digit hexadecimal numbers can easily be memorized.
As for larger numbers, it's probably not worth tormenting yourself.

Perhaps the most visible hexadecimal numbers are in \ac{URL}s.
This is the way that non-Latin characters are encoded.
For example:
\url{https://en.wiktionary.org/wiki/na\%C3\%AFvet\%C3\%A9} is the \ac{URL} of Wiktionary article about \q{naïveté} word.

\subsubsection{Octal Radix}

Another numeral system heavily used in the past of computer programming is octal. In octal there are 8 digits (0..7), and each is mapped to 3 bits, so it's easy to convert numbers back and forth.
It has been superseded by the hexadecimal system almost everywhere, but, surprisingly, there is a *NIX utility, used often by many people, which takes octal numbers as argument: \TT{chmod}.

\myindex{UNIX!chmod}
As many *NIX users know, \TT{chmod} argument can be a number of 3 digits. The first digit represents the rights of the owner of the file (read, write and/or execute), the second is the rights for the group to which the file belongs, and the third is for everyone else.
Each digit that \TT{chmod} takes can be represented in binary form:

\begin{center}
\begin{longtable}{ | l | l | l | }
\hline
\HeaderColor decimal & \HeaderColor binary & \HeaderColor meaning \\
\hline
7	&111	&\textbf{rwx} \\
6	&110	&\textbf{rw-} \\
5	&101	&\textbf{r-x} \\
4	&100	&\textbf{r-{}-} \\
3	&011	&\textbf{-wx} \\
2	&010	&\textbf{-w-} \\
1	&001	&\textbf{-{}-x} \\
0	&000	&\textbf{-{}-{}-} \\
\hline
\end{longtable}
\end{center}

So each bit is mapped to a flag: read/write/execute.

The importance of \TT{chmod} here is that the whole number in argument can be represented as octal number.
Let's take, for example, 644.
When you run \TT{chmod 644 file}, you set read/write permissions for owner, read permissions for group and again, read permissions for everyone else.
If we convert the octal number 644 to binary, it would be \TT{110100100}, or, in groups of 3 bits, \TT{110 100 100}.

Now we see that each triplet describe permissions for owner/group/others: first is \TT{rw-}, second is \TT{r--} and third is \TT{r--}.

The octal numeral system was also popular on old computers like PDP-8, because word there could be 12, 24 or 36 bits, and these numbers are all divisible by 3, so the octal system was natural in that environment.
Nowadays, all popular computers employ word/address sizes of 16, 32 or 64 bits, and these numbers are all divisible by 4, so the hexadecimal system is more natural there.

The octal numeral system is supported by all standard \CCpp compilers.
This is a source of confusion sometimes, because octal numbers are encoded with a zero prepended, for example, 0377 is 255.
Sometimes, you might make a typo and write "09" instead of 9, and the compiler would report an error.
GCC might report something like this:\\
\TT{error: invalid digit "9" in octal constant}.

Also, the octal system is somewhat popular in Java. When the IDA shows Java strings with non-printable characters,
they are encoded in the octal system instead of hexadecimal.
\myindex{JAD}
The JAD Java decompiler behaves the same way.

\subsubsection{Divisibility}

When you see a decimal number like 120, you can quickly deduce that it's divisible by 10, because the last digit is zero.
In the same way, 123400 is divisible by 100, because the two last digits are zeros.

Likewise, the hexadecimal number 0x1230 is divisible by 0x10 (or 16), 0x123000 is divisible by 0x1000 (or 4096), etc.

The binary number 0b1000101000 is divisible by 0b1000 (8), etc.

This property can often be used to quickly realize if the size of some block in memory is padded to some boundary.
For example, sections in \ac{PE} files are almost always started at addresses ending with 3 hexadecimal zeros: 0x41000, 0x10001000, etc.
The reason behind this is the fact that almost all \ac{PE} sections are padded to a boundary of 0x1000 (4096) bytes.

\subsubsection{Multi-Precision Arithmetic and Radix}

\index{RSA}
Multi-precision arithmetic can use huge numbers, and each one may be stored in several bytes.
For example, RSA keys, both public and private, span up to 4096 bits, and maybe even more.

% I'm not sure how to change this, but the normal format for quoting would be just to mention the author or book, and footnote to the full reference
In \InSqBrackets{\TAOCPvolII, 265} we find the following idea: when you store a multi-precision number in several bytes,
the whole number can be represented as having a radix of $2^8=256$, and each digit goes to the corresponding byte.
Likewise, if you store a multi-precision number in several 32-bit integer values, each digit goes to each 32-bit slot,
and you may think about this number as stored in radix of $2^{32}$.

\subsubsection{How to Pronounce Non-Decimal Numbers}

Numbers in a non-decimal base are usually pronounced by digit by digit: ``one-zero-zero-one-one-...''.
Words like ``ten'' and ``thousand'' are usually not pronounced, to prevent confusion with the decimal base system.

\subsubsection{Floating point numbers}

To distinguish floating point numbers from integers, they are usually written with ``.0'' at the end,
like $0.0$, $123.0$, etc.
}
\RU{\subsection{Представление чисел}

Люди привыкли к десятичной системе счисления вероятно потому что почти у каждого есть по 10 пальцев.
Тем не менее, число 10 не имеет особого значения в науке и математике.
Двоичная система естествена для цифровой электроники: 0 означает отсутствие тока в проводе и 1 --- его присутствие.
10 в двоичной системе это 2 в десятичной; 100 в двоичной это 4 в десятичной, итд.

Если в системе счисления есть 10 цифр, её \IT{основание} или \IT{radix} это 10.
Двоичная система имеет \IT{основание} 2.

Важные вещи, которые полезно вспомнить:
1) \IT{число} это число, в то время как \IT{цифра} это термин из системы письменности, и это обычно один символ;
2) само число не меняется, когда конвертируется из одного основания в другое: меняется способ его записи (или представления
в памяти).

Как сконвертировать число из одного основания в другое?

Позиционная нотация используется почти везде, это означает, что всякая цифра имеет свой вес, в зависимости от её расположения
внутри числа.
Если 2 расположена в самом последнем месте справа, это 2.
Если она расположена в месте перед последним, это 20.

Что означает $1234$?

$10^3 \cdot 1 + 10^2 \cdot 2 + 10^1 \cdot 3 + 1 \cdot 4$ = 1234 или
$1000 \cdot 1 + 100 \cdot 2 + 10 \cdot 3 + 4 = 1234$

Та же история и для двоичных чисел, только основание там 2 вместо 10.
Что означает 0b101011?

$2^5 \cdot 1 + 2^4 \cdot 0 + 2^3 \cdot 1 + 2^2 \cdot 0 + 2^1 \cdot 1 + 2^0 \cdot 1 = 43$ или
$32 \cdot 1 + 16 \cdot 0 + 8 \cdot 1 + 4 \cdot 0 + 2 \cdot 1 + 1 = 43$

Позиционную нотацию можно противопоставить непозиционной нотации, такой как римская система записи чисел
\footnote{Об эволюции способов записи чисел, см.также: \InSqBrackets{\TAOCPvolII{}, 195--213.}}.
Вероятно, человечество перешло на позиционную нотацию, потому что так проще работать с числами (сложение, умножение, итд)
на бумаге, в ручную.

Действительно, двоичные числа можно складывать, вычитать, итд, точно также, как этому обычно обучают в школах,
только доступны лишь 2 цифры.

Двоичные числа громоздки, когда их используют в исходных кодах и дампах, так что в этих случаях применяется шестнадцатеричная
система.
Используются цифры 0..9 и еще 6 латинских букв: A..F.
Каждая шестнадцатеричная цифра занимает 4 бита или 4 двоичных цифры, так что конвертировать из двоичной системы в
шестнадцатеричную и назад, можно легко вручную, или даже в уме.

\begin{center}
\begin{longtable}{ | l | l | l | }
\hline
\HeaderColor шестнадцатеричная & \HeaderColor двоичная & \HeaderColor десятичная \\
\hline
0	&0000	&0 \\
1	&0001	&1 \\
2	&0010	&2 \\
3	&0011	&3 \\
4	&0100	&4 \\
5	&0101	&5 \\
6	&0110	&6 \\
7	&0111	&7 \\
8	&1000	&8 \\
9	&1001	&9 \\
A	&1010	&10 \\
B	&1011	&11 \\
C	&1100	&12 \\
D	&1101	&13 \\
E	&1110	&14 \\
F	&1111	&15 \\
\hline
\end{longtable}
\end{center}

Как понять, какое основание используется в конкретном месте?

Десятичные числа обычно записываются как есть, т.е., 1234. Но некоторые ассемблеры позволяют подчеркивать
этот факт для ясности, и это число может быть дополнено суффиксом "d": 1234d.

К двоичным числам иногда спереди добавляют префикс "0b": 0b100110111
(В \ac{GCC} для этого есть нестандартное расширение языка
\footnote{\url{https://gcc.gnu.org/onlinedocs/gcc/Binary-constants.html}}).
Есть также еще один способ: суффикс "b", например: 100110111b.
В этой книге я буду пытаться придерживаться префикса "0b" для двоичных чисел.

Шестнадцатеричные числа имеют префикс "0x" в \CCpp и некоторых других \ac{PL}: 0x1234ABCD.
Либо они имеют суффикс "h": 1234ABCDh --- обычно так они представляются в ассемблерах и отладчиках.
Если число начинается с цифры A..F, перед ним добавляется 0: 0ABCDEFh.
Во времена 8-битных домашних компьютеров, был также способ записи чисел используя префикс \$, например, \$ABCD.
В книге я попытаюсь придерживаться префикса "0x" для шестнадцатеричных чисел.

Нужно ли учиться конвертировать числа в уме? Таблицу шестнадцатеричных чисел из одной цифры легко запомнить.
А запоминать б\'{о}льшие числа, наверное, не стоит.

Наверное, чаще всего шестнадцатеричные числа можно увидеть в \ac{URL}-ах.
Так кодируются буквы не из числа латинских.
Например:
\url{https://en.wiktionary.org/wiki/na\%C3\%AFvet\%C3\%A9} это \ac{URL} страницы в Wiktionary о слове \q{naïveté}.

\subsubsection{Восьмеричная система}

Еще одна система, которая в прошлом много использовалась в программировании это восьмеричная: есть 8 цифр (0..7) и каждая
описывает 3 бита, так что легко конвертировать числа туда и назад.
Она почти везде была заменена шестнадцатеричной, но удивительно, в *NIX имеется утилита использующаяся многими людьми,
которая принимает на вход восьмеричное число: \TT{chmod}.

\myindex{UNIX!chmod}
Как знают многие пользователи *NIX, аргумент \TT{chmod} это число из трех цифр. Первая цифра это права владельца файла,
вторая это права группы (которой файл принадлежит), третья для всех остальных.
И каждая цифра может быть представлена в двоичном виде:

\begin{center}
\begin{longtable}{ | l | l | l | }
\hline
\HeaderColor десятичная & \HeaderColor двоичная & \HeaderColor значение \\
\hline
7	&111	&\textbf{rwx} \\
6	&110	&\textbf{rw-} \\
5	&101	&\textbf{r-x} \\
4	&100	&\textbf{r-{}-} \\
3	&011	&\textbf{-wx} \\
2	&010	&\textbf{-w-} \\
1	&001	&\textbf{-{}-x} \\
0	&000	&\textbf{-{}-{}-} \\
\hline
\end{longtable}
\end{center}

Так что каждый бит привязан к флагу: read/write/execute (чтение/запись/исполнение).

И вот почему я вспомнил здесь о \TT{chmod}, это потому что всё число может быть представлено как число в восьмеричной системе.
Для примера возьмем 644.
Когда вы запускаете \TT{chmod 644 file}, вы выставляете права read/write для владельца, права read для группы, и снова,
read для всех остальных.
Сконвертируем число 644 из восьмеричной системы в двоичную, это будет \TT{110100100}, или (в группах по 3 бита) \TT{110 100 100}.

Теперь мы видим, что каждая тройка описывает права для владельца/группы/остальных:
первая это \TT{rw-}, вторая это \TT{r--} и третья это \TT{r--}.

Восьмеричная система была также популярная на старых компьютерах вроде PDP-8, потому что слово там могло содержать 12, 24 или
36 бит, и эти числа делятся на 3, так что выбор восьмеричной системы в той среде был логичен.
Сейчас, все популярные компьютеры имеют размер слова/адреса 16, 32 или 64 бита, и эти числа делятся на 4,
так что шестнадцатеричная система здесь удобнее.

Восьмеричная система поддерживается всеми стандартными компиляторами \CCpp{}.
Это иногда источник недоумения, потому что восьмеричные числа кодируются с нулем вперед, например, 0377 это 255.
И иногда, вы можете сделать опечатку, и написать "09" вместо 9, и компилятор выдаст ошибку.
GCC может выдать что-то вроде:\\
\TT{error: invalid digit "9" in octal constant}.

Также, восьмеричная система популярна в Java: когда IDA показывает строку с непечатаемыми символами,
они кодируются в восьмеричной системе вместо шестнадцатеричной.
\myindex{JAD}
Точно также себя ведет декомпилятор с Java JAD.

\subsubsection{Делимость}

Когда вы видите десятичное число вроде 120, вы можете быстро понять что оно делится на 10, потому что последняя цифра это 0.
Точно также, 123400 делится на 100, потому что две последних цифры это нули.

Точно также, шестнадцатеричное число 0x1230 делится на 0x10 (или 16), 0x123000 делится на 0x1000 (или 4096), итд.

Двоичное число 0b1000101000 делится на 0b1000 (8), итд.

Это свойство можно часто использовать, чтобы быстро понять,
что длина какого-либо блока в памяти выровнена по некоторой границе.
Например, секции в \ac{PE}-файлах почти всегда начинаются с адресов заканчивающихся 3 шестнадцатеричными нулями:
0x41000, 0x10001000, итд.
Причина в том, что почти все секции в \ac{PE} выровнены по границе 0x1000 (4096) байт.

\subsubsection{Арифметика произвольной точности и основание}

\index{RSA}
Арифметика произвольной точности (multi-precision arithmetic) может использовать огромные числа,
которые могут храниться в нескольких байтах.
Например, ключи RSA, и открытые и закрытые, могут занимать до 4096 бит и даже больше.

В \InSqBrackets{\TAOCPvolII, 265} можно найти такую идею: когда вы сохраняете число произвольной точности в нескольких байтах,
всё число может быть представлено как имеющую систему счисления по основанию $2^8=256$, и каждая цифра находится
в соответствующем байте.
Точно также, если вы сохраняете число произвольной точности в нескольких 32-битных целочисленных значениях,
каждая цифра отправляется в каждый 32-битный слот, и вы можете считать что это число записано в системе с основанием $2^{32}$.

\subsubsection{Произношение}

Числа в недесятичных системах счислениях обычно произносятся по одной цифре: ``один-ноль-ноль-один-один-...''.
Слова вроде ``десять'', ``тысяча'', итд, обычно не произносятся, потому что тогда можно спутать с десятичной системой.

\subsubsection{Числа с плавающей запятой}

Чтобы отличать числа с плавающей запятой от целочисленных, часто, в конце добавляют ``.0'',
например $0.0$, $123.0$, итд.

}
\ITA{\input{patterns/numeral_ITA}}
\DE{\input{patterns/numeral_DE}}
\FR{\input{patterns/numeral_FR}}
\PL{\input{patterns/numeral_PL}}

% chapters
\ifdefined\SPANISH
\chapter{Patrones de código}
\fi % SPANISH

\ifdefined\GERMAN
\chapter{Code-Muster}
\fi % GERMAN

\ifdefined\ENGLISH
\chapter{Code Patterns}
\fi % ENGLISH

\ifdefined\ITALIAN
\chapter{Forme di codice}
\fi % ITALIAN

\ifdefined\RUSSIAN
\chapter{Образцы кода}
\fi % RUSSIAN

\ifdefined\BRAZILIAN
\chapter{Padrões de códigos}
\fi % BRAZILIAN

\ifdefined\THAI
\chapter{รูปแบบของโค้ด}
\fi % THAI

\ifdefined\FRENCH
\chapter{Modèle de code}
\fi % FRENCH

\ifdefined\POLISH
\chapter{\PLph{}}
\fi % POLISH

% sections
\EN{\input{patterns/patterns_opt_dbg_EN}}
\ES{\input{patterns/patterns_opt_dbg_ES}}
\ITA{\input{patterns/patterns_opt_dbg_ITA}}
\PTBR{\input{patterns/patterns_opt_dbg_PTBR}}
\RU{\input{patterns/patterns_opt_dbg_RU}}
\THA{\input{patterns/patterns_opt_dbg_THA}}
\DE{\input{patterns/patterns_opt_dbg_DE}}
\FR{\input{patterns/patterns_opt_dbg_FR}}
\PL{\input{patterns/patterns_opt_dbg_PL}}

\RU{\section{Некоторые базовые понятия}}
\EN{\section{Some basics}}
\DE{\section{Einige Grundlagen}}
\FR{\section{Quelques bases}}
\ES{\section{\ESph{}}}
\ITA{\section{Alcune basi teoriche}}
\PTBR{\section{\PTBRph{}}}
\THA{\section{\THAph{}}}
\PL{\section{\PLph{}}}

% sections:
\EN{\input{patterns/intro_CPU_ISA_EN}}
\ES{\input{patterns/intro_CPU_ISA_ES}}
\ITA{\input{patterns/intro_CPU_ISA_ITA}}
\PTBR{\input{patterns/intro_CPU_ISA_PTBR}}
\RU{\input{patterns/intro_CPU_ISA_RU}}
\DE{\input{patterns/intro_CPU_ISA_DE}}
\FR{\input{patterns/intro_CPU_ISA_FR}}
\PL{\input{patterns/intro_CPU_ISA_PL}}

\EN{\input{patterns/numeral_EN}}
\RU{\input{patterns/numeral_RU}}
\ITA{\input{patterns/numeral_ITA}}
\DE{\input{patterns/numeral_DE}}
\FR{\input{patterns/numeral_FR}}
\PL{\input{patterns/numeral_PL}}

% chapters
\input{patterns/00_empty/main}
\input{patterns/011_ret/main}
\input{patterns/01_helloworld/main}
\input{patterns/015_prolog_epilogue/main}
\input{patterns/02_stack/main}
\input{patterns/03_printf/main}
\input{patterns/04_scanf/main}
\input{patterns/05_passing_arguments/main}
\input{patterns/06_return_results/main}
\input{patterns/061_pointers/main}
\input{patterns/065_GOTO/main}
\input{patterns/07_jcc/main}
\input{patterns/08_switch/main}
\input{patterns/09_loops/main}
\input{patterns/10_strings/main}
\input{patterns/11_arith_optimizations/main}
\input{patterns/12_FPU/main}
\input{patterns/13_arrays/main}
\input{patterns/14_bitfields/main}
\EN{\input{patterns/145_LCG/main_EN}}
\RU{\input{patterns/145_LCG/main_RU}}
\input{patterns/15_structs/main}
\input{patterns/17_unions/main}
\input{patterns/18_pointers_to_functions/main}
\input{patterns/185_64bit_in_32_env/main}

\EN{\input{patterns/19_SIMD/main_EN}}
\RU{\input{patterns/19_SIMD/main_RU}}
\DE{\input{patterns/19_SIMD/main_DE}}

\EN{\input{patterns/20_x64/main_EN}}
\RU{\input{patterns/20_x64/main_RU}}

\EN{\input{patterns/205_floating_SIMD/main_EN}}
\RU{\input{patterns/205_floating_SIMD/main_RU}}
\DE{\input{patterns/205_floating_SIMD/main_DE}}

\EN{\input{patterns/ARM/main_EN}}
\RU{\input{patterns/ARM/main_RU}}
\DE{\input{patterns/ARM/main_DE}}

\input{patterns/MIPS/main}

\ifdefined\SPANISH
\chapter{Patrones de código}
\fi % SPANISH

\ifdefined\GERMAN
\chapter{Code-Muster}
\fi % GERMAN

\ifdefined\ENGLISH
\chapter{Code Patterns}
\fi % ENGLISH

\ifdefined\ITALIAN
\chapter{Forme di codice}
\fi % ITALIAN

\ifdefined\RUSSIAN
\chapter{Образцы кода}
\fi % RUSSIAN

\ifdefined\BRAZILIAN
\chapter{Padrões de códigos}
\fi % BRAZILIAN

\ifdefined\THAI
\chapter{รูปแบบของโค้ด}
\fi % THAI

\ifdefined\FRENCH
\chapter{Modèle de code}
\fi % FRENCH

\ifdefined\POLISH
\chapter{\PLph{}}
\fi % POLISH

% sections
\EN{\input{patterns/patterns_opt_dbg_EN}}
\ES{\input{patterns/patterns_opt_dbg_ES}}
\ITA{\input{patterns/patterns_opt_dbg_ITA}}
\PTBR{\input{patterns/patterns_opt_dbg_PTBR}}
\RU{\input{patterns/patterns_opt_dbg_RU}}
\THA{\input{patterns/patterns_opt_dbg_THA}}
\DE{\input{patterns/patterns_opt_dbg_DE}}
\FR{\input{patterns/patterns_opt_dbg_FR}}
\PL{\input{patterns/patterns_opt_dbg_PL}}

\RU{\section{Некоторые базовые понятия}}
\EN{\section{Some basics}}
\DE{\section{Einige Grundlagen}}
\FR{\section{Quelques bases}}
\ES{\section{\ESph{}}}
\ITA{\section{Alcune basi teoriche}}
\PTBR{\section{\PTBRph{}}}
\THA{\section{\THAph{}}}
\PL{\section{\PLph{}}}

% sections:
\EN{\input{patterns/intro_CPU_ISA_EN}}
\ES{\input{patterns/intro_CPU_ISA_ES}}
\ITA{\input{patterns/intro_CPU_ISA_ITA}}
\PTBR{\input{patterns/intro_CPU_ISA_PTBR}}
\RU{\input{patterns/intro_CPU_ISA_RU}}
\DE{\input{patterns/intro_CPU_ISA_DE}}
\FR{\input{patterns/intro_CPU_ISA_FR}}
\PL{\input{patterns/intro_CPU_ISA_PL}}

\EN{\input{patterns/numeral_EN}}
\RU{\input{patterns/numeral_RU}}
\ITA{\input{patterns/numeral_ITA}}
\DE{\input{patterns/numeral_DE}}
\FR{\input{patterns/numeral_FR}}
\PL{\input{patterns/numeral_PL}}

% chapters
\input{patterns/00_empty/main}
\input{patterns/011_ret/main}
\input{patterns/01_helloworld/main}
\input{patterns/015_prolog_epilogue/main}
\input{patterns/02_stack/main}
\input{patterns/03_printf/main}
\input{patterns/04_scanf/main}
\input{patterns/05_passing_arguments/main}
\input{patterns/06_return_results/main}
\input{patterns/061_pointers/main}
\input{patterns/065_GOTO/main}
\input{patterns/07_jcc/main}
\input{patterns/08_switch/main}
\input{patterns/09_loops/main}
\input{patterns/10_strings/main}
\input{patterns/11_arith_optimizations/main}
\input{patterns/12_FPU/main}
\input{patterns/13_arrays/main}
\input{patterns/14_bitfields/main}
\EN{\input{patterns/145_LCG/main_EN}}
\RU{\input{patterns/145_LCG/main_RU}}
\input{patterns/15_structs/main}
\input{patterns/17_unions/main}
\input{patterns/18_pointers_to_functions/main}
\input{patterns/185_64bit_in_32_env/main}

\EN{\input{patterns/19_SIMD/main_EN}}
\RU{\input{patterns/19_SIMD/main_RU}}
\DE{\input{patterns/19_SIMD/main_DE}}

\EN{\input{patterns/20_x64/main_EN}}
\RU{\input{patterns/20_x64/main_RU}}

\EN{\input{patterns/205_floating_SIMD/main_EN}}
\RU{\input{patterns/205_floating_SIMD/main_RU}}
\DE{\input{patterns/205_floating_SIMD/main_DE}}

\EN{\input{patterns/ARM/main_EN}}
\RU{\input{patterns/ARM/main_RU}}
\DE{\input{patterns/ARM/main_DE}}

\input{patterns/MIPS/main}

\ifdefined\SPANISH
\chapter{Patrones de código}
\fi % SPANISH

\ifdefined\GERMAN
\chapter{Code-Muster}
\fi % GERMAN

\ifdefined\ENGLISH
\chapter{Code Patterns}
\fi % ENGLISH

\ifdefined\ITALIAN
\chapter{Forme di codice}
\fi % ITALIAN

\ifdefined\RUSSIAN
\chapter{Образцы кода}
\fi % RUSSIAN

\ifdefined\BRAZILIAN
\chapter{Padrões de códigos}
\fi % BRAZILIAN

\ifdefined\THAI
\chapter{รูปแบบของโค้ด}
\fi % THAI

\ifdefined\FRENCH
\chapter{Modèle de code}
\fi % FRENCH

\ifdefined\POLISH
\chapter{\PLph{}}
\fi % POLISH

% sections
\EN{\input{patterns/patterns_opt_dbg_EN}}
\ES{\input{patterns/patterns_opt_dbg_ES}}
\ITA{\input{patterns/patterns_opt_dbg_ITA}}
\PTBR{\input{patterns/patterns_opt_dbg_PTBR}}
\RU{\input{patterns/patterns_opt_dbg_RU}}
\THA{\input{patterns/patterns_opt_dbg_THA}}
\DE{\input{patterns/patterns_opt_dbg_DE}}
\FR{\input{patterns/patterns_opt_dbg_FR}}
\PL{\input{patterns/patterns_opt_dbg_PL}}

\RU{\section{Некоторые базовые понятия}}
\EN{\section{Some basics}}
\DE{\section{Einige Grundlagen}}
\FR{\section{Quelques bases}}
\ES{\section{\ESph{}}}
\ITA{\section{Alcune basi teoriche}}
\PTBR{\section{\PTBRph{}}}
\THA{\section{\THAph{}}}
\PL{\section{\PLph{}}}

% sections:
\EN{\input{patterns/intro_CPU_ISA_EN}}
\ES{\input{patterns/intro_CPU_ISA_ES}}
\ITA{\input{patterns/intro_CPU_ISA_ITA}}
\PTBR{\input{patterns/intro_CPU_ISA_PTBR}}
\RU{\input{patterns/intro_CPU_ISA_RU}}
\DE{\input{patterns/intro_CPU_ISA_DE}}
\FR{\input{patterns/intro_CPU_ISA_FR}}
\PL{\input{patterns/intro_CPU_ISA_PL}}

\EN{\input{patterns/numeral_EN}}
\RU{\input{patterns/numeral_RU}}
\ITA{\input{patterns/numeral_ITA}}
\DE{\input{patterns/numeral_DE}}
\FR{\input{patterns/numeral_FR}}
\PL{\input{patterns/numeral_PL}}

% chapters
\input{patterns/00_empty/main}
\input{patterns/011_ret/main}
\input{patterns/01_helloworld/main}
\input{patterns/015_prolog_epilogue/main}
\input{patterns/02_stack/main}
\input{patterns/03_printf/main}
\input{patterns/04_scanf/main}
\input{patterns/05_passing_arguments/main}
\input{patterns/06_return_results/main}
\input{patterns/061_pointers/main}
\input{patterns/065_GOTO/main}
\input{patterns/07_jcc/main}
\input{patterns/08_switch/main}
\input{patterns/09_loops/main}
\input{patterns/10_strings/main}
\input{patterns/11_arith_optimizations/main}
\input{patterns/12_FPU/main}
\input{patterns/13_arrays/main}
\input{patterns/14_bitfields/main}
\EN{\input{patterns/145_LCG/main_EN}}
\RU{\input{patterns/145_LCG/main_RU}}
\input{patterns/15_structs/main}
\input{patterns/17_unions/main}
\input{patterns/18_pointers_to_functions/main}
\input{patterns/185_64bit_in_32_env/main}

\EN{\input{patterns/19_SIMD/main_EN}}
\RU{\input{patterns/19_SIMD/main_RU}}
\DE{\input{patterns/19_SIMD/main_DE}}

\EN{\input{patterns/20_x64/main_EN}}
\RU{\input{patterns/20_x64/main_RU}}

\EN{\input{patterns/205_floating_SIMD/main_EN}}
\RU{\input{patterns/205_floating_SIMD/main_RU}}
\DE{\input{patterns/205_floating_SIMD/main_DE}}

\EN{\input{patterns/ARM/main_EN}}
\RU{\input{patterns/ARM/main_RU}}
\DE{\input{patterns/ARM/main_DE}}

\input{patterns/MIPS/main}

\ifdefined\SPANISH
\chapter{Patrones de código}
\fi % SPANISH

\ifdefined\GERMAN
\chapter{Code-Muster}
\fi % GERMAN

\ifdefined\ENGLISH
\chapter{Code Patterns}
\fi % ENGLISH

\ifdefined\ITALIAN
\chapter{Forme di codice}
\fi % ITALIAN

\ifdefined\RUSSIAN
\chapter{Образцы кода}
\fi % RUSSIAN

\ifdefined\BRAZILIAN
\chapter{Padrões de códigos}
\fi % BRAZILIAN

\ifdefined\THAI
\chapter{รูปแบบของโค้ด}
\fi % THAI

\ifdefined\FRENCH
\chapter{Modèle de code}
\fi % FRENCH

\ifdefined\POLISH
\chapter{\PLph{}}
\fi % POLISH

% sections
\EN{\input{patterns/patterns_opt_dbg_EN}}
\ES{\input{patterns/patterns_opt_dbg_ES}}
\ITA{\input{patterns/patterns_opt_dbg_ITA}}
\PTBR{\input{patterns/patterns_opt_dbg_PTBR}}
\RU{\input{patterns/patterns_opt_dbg_RU}}
\THA{\input{patterns/patterns_opt_dbg_THA}}
\DE{\input{patterns/patterns_opt_dbg_DE}}
\FR{\input{patterns/patterns_opt_dbg_FR}}
\PL{\input{patterns/patterns_opt_dbg_PL}}

\RU{\section{Некоторые базовые понятия}}
\EN{\section{Some basics}}
\DE{\section{Einige Grundlagen}}
\FR{\section{Quelques bases}}
\ES{\section{\ESph{}}}
\ITA{\section{Alcune basi teoriche}}
\PTBR{\section{\PTBRph{}}}
\THA{\section{\THAph{}}}
\PL{\section{\PLph{}}}

% sections:
\EN{\input{patterns/intro_CPU_ISA_EN}}
\ES{\input{patterns/intro_CPU_ISA_ES}}
\ITA{\input{patterns/intro_CPU_ISA_ITA}}
\PTBR{\input{patterns/intro_CPU_ISA_PTBR}}
\RU{\input{patterns/intro_CPU_ISA_RU}}
\DE{\input{patterns/intro_CPU_ISA_DE}}
\FR{\input{patterns/intro_CPU_ISA_FR}}
\PL{\input{patterns/intro_CPU_ISA_PL}}

\EN{\input{patterns/numeral_EN}}
\RU{\input{patterns/numeral_RU}}
\ITA{\input{patterns/numeral_ITA}}
\DE{\input{patterns/numeral_DE}}
\FR{\input{patterns/numeral_FR}}
\PL{\input{patterns/numeral_PL}}

% chapters
\input{patterns/00_empty/main}
\input{patterns/011_ret/main}
\input{patterns/01_helloworld/main}
\input{patterns/015_prolog_epilogue/main}
\input{patterns/02_stack/main}
\input{patterns/03_printf/main}
\input{patterns/04_scanf/main}
\input{patterns/05_passing_arguments/main}
\input{patterns/06_return_results/main}
\input{patterns/061_pointers/main}
\input{patterns/065_GOTO/main}
\input{patterns/07_jcc/main}
\input{patterns/08_switch/main}
\input{patterns/09_loops/main}
\input{patterns/10_strings/main}
\input{patterns/11_arith_optimizations/main}
\input{patterns/12_FPU/main}
\input{patterns/13_arrays/main}
\input{patterns/14_bitfields/main}
\EN{\input{patterns/145_LCG/main_EN}}
\RU{\input{patterns/145_LCG/main_RU}}
\input{patterns/15_structs/main}
\input{patterns/17_unions/main}
\input{patterns/18_pointers_to_functions/main}
\input{patterns/185_64bit_in_32_env/main}

\EN{\input{patterns/19_SIMD/main_EN}}
\RU{\input{patterns/19_SIMD/main_RU}}
\DE{\input{patterns/19_SIMD/main_DE}}

\EN{\input{patterns/20_x64/main_EN}}
\RU{\input{patterns/20_x64/main_RU}}

\EN{\input{patterns/205_floating_SIMD/main_EN}}
\RU{\input{patterns/205_floating_SIMD/main_RU}}
\DE{\input{patterns/205_floating_SIMD/main_DE}}

\EN{\input{patterns/ARM/main_EN}}
\RU{\input{patterns/ARM/main_RU}}
\DE{\input{patterns/ARM/main_DE}}

\input{patterns/MIPS/main}

\ifdefined\SPANISH
\chapter{Patrones de código}
\fi % SPANISH

\ifdefined\GERMAN
\chapter{Code-Muster}
\fi % GERMAN

\ifdefined\ENGLISH
\chapter{Code Patterns}
\fi % ENGLISH

\ifdefined\ITALIAN
\chapter{Forme di codice}
\fi % ITALIAN

\ifdefined\RUSSIAN
\chapter{Образцы кода}
\fi % RUSSIAN

\ifdefined\BRAZILIAN
\chapter{Padrões de códigos}
\fi % BRAZILIAN

\ifdefined\THAI
\chapter{รูปแบบของโค้ด}
\fi % THAI

\ifdefined\FRENCH
\chapter{Modèle de code}
\fi % FRENCH

\ifdefined\POLISH
\chapter{\PLph{}}
\fi % POLISH

% sections
\EN{\input{patterns/patterns_opt_dbg_EN}}
\ES{\input{patterns/patterns_opt_dbg_ES}}
\ITA{\input{patterns/patterns_opt_dbg_ITA}}
\PTBR{\input{patterns/patterns_opt_dbg_PTBR}}
\RU{\input{patterns/patterns_opt_dbg_RU}}
\THA{\input{patterns/patterns_opt_dbg_THA}}
\DE{\input{patterns/patterns_opt_dbg_DE}}
\FR{\input{patterns/patterns_opt_dbg_FR}}
\PL{\input{patterns/patterns_opt_dbg_PL}}

\RU{\section{Некоторые базовые понятия}}
\EN{\section{Some basics}}
\DE{\section{Einige Grundlagen}}
\FR{\section{Quelques bases}}
\ES{\section{\ESph{}}}
\ITA{\section{Alcune basi teoriche}}
\PTBR{\section{\PTBRph{}}}
\THA{\section{\THAph{}}}
\PL{\section{\PLph{}}}

% sections:
\EN{\input{patterns/intro_CPU_ISA_EN}}
\ES{\input{patterns/intro_CPU_ISA_ES}}
\ITA{\input{patterns/intro_CPU_ISA_ITA}}
\PTBR{\input{patterns/intro_CPU_ISA_PTBR}}
\RU{\input{patterns/intro_CPU_ISA_RU}}
\DE{\input{patterns/intro_CPU_ISA_DE}}
\FR{\input{patterns/intro_CPU_ISA_FR}}
\PL{\input{patterns/intro_CPU_ISA_PL}}

\EN{\input{patterns/numeral_EN}}
\RU{\input{patterns/numeral_RU}}
\ITA{\input{patterns/numeral_ITA}}
\DE{\input{patterns/numeral_DE}}
\FR{\input{patterns/numeral_FR}}
\PL{\input{patterns/numeral_PL}}

% chapters
\input{patterns/00_empty/main}
\input{patterns/011_ret/main}
\input{patterns/01_helloworld/main}
\input{patterns/015_prolog_epilogue/main}
\input{patterns/02_stack/main}
\input{patterns/03_printf/main}
\input{patterns/04_scanf/main}
\input{patterns/05_passing_arguments/main}
\input{patterns/06_return_results/main}
\input{patterns/061_pointers/main}
\input{patterns/065_GOTO/main}
\input{patterns/07_jcc/main}
\input{patterns/08_switch/main}
\input{patterns/09_loops/main}
\input{patterns/10_strings/main}
\input{patterns/11_arith_optimizations/main}
\input{patterns/12_FPU/main}
\input{patterns/13_arrays/main}
\input{patterns/14_bitfields/main}
\EN{\input{patterns/145_LCG/main_EN}}
\RU{\input{patterns/145_LCG/main_RU}}
\input{patterns/15_structs/main}
\input{patterns/17_unions/main}
\input{patterns/18_pointers_to_functions/main}
\input{patterns/185_64bit_in_32_env/main}

\EN{\input{patterns/19_SIMD/main_EN}}
\RU{\input{patterns/19_SIMD/main_RU}}
\DE{\input{patterns/19_SIMD/main_DE}}

\EN{\input{patterns/20_x64/main_EN}}
\RU{\input{patterns/20_x64/main_RU}}

\EN{\input{patterns/205_floating_SIMD/main_EN}}
\RU{\input{patterns/205_floating_SIMD/main_RU}}
\DE{\input{patterns/205_floating_SIMD/main_DE}}

\EN{\input{patterns/ARM/main_EN}}
\RU{\input{patterns/ARM/main_RU}}
\DE{\input{patterns/ARM/main_DE}}

\input{patterns/MIPS/main}

\ifdefined\SPANISH
\chapter{Patrones de código}
\fi % SPANISH

\ifdefined\GERMAN
\chapter{Code-Muster}
\fi % GERMAN

\ifdefined\ENGLISH
\chapter{Code Patterns}
\fi % ENGLISH

\ifdefined\ITALIAN
\chapter{Forme di codice}
\fi % ITALIAN

\ifdefined\RUSSIAN
\chapter{Образцы кода}
\fi % RUSSIAN

\ifdefined\BRAZILIAN
\chapter{Padrões de códigos}
\fi % BRAZILIAN

\ifdefined\THAI
\chapter{รูปแบบของโค้ด}
\fi % THAI

\ifdefined\FRENCH
\chapter{Modèle de code}
\fi % FRENCH

\ifdefined\POLISH
\chapter{\PLph{}}
\fi % POLISH

% sections
\EN{\input{patterns/patterns_opt_dbg_EN}}
\ES{\input{patterns/patterns_opt_dbg_ES}}
\ITA{\input{patterns/patterns_opt_dbg_ITA}}
\PTBR{\input{patterns/patterns_opt_dbg_PTBR}}
\RU{\input{patterns/patterns_opt_dbg_RU}}
\THA{\input{patterns/patterns_opt_dbg_THA}}
\DE{\input{patterns/patterns_opt_dbg_DE}}
\FR{\input{patterns/patterns_opt_dbg_FR}}
\PL{\input{patterns/patterns_opt_dbg_PL}}

\RU{\section{Некоторые базовые понятия}}
\EN{\section{Some basics}}
\DE{\section{Einige Grundlagen}}
\FR{\section{Quelques bases}}
\ES{\section{\ESph{}}}
\ITA{\section{Alcune basi teoriche}}
\PTBR{\section{\PTBRph{}}}
\THA{\section{\THAph{}}}
\PL{\section{\PLph{}}}

% sections:
\EN{\input{patterns/intro_CPU_ISA_EN}}
\ES{\input{patterns/intro_CPU_ISA_ES}}
\ITA{\input{patterns/intro_CPU_ISA_ITA}}
\PTBR{\input{patterns/intro_CPU_ISA_PTBR}}
\RU{\input{patterns/intro_CPU_ISA_RU}}
\DE{\input{patterns/intro_CPU_ISA_DE}}
\FR{\input{patterns/intro_CPU_ISA_FR}}
\PL{\input{patterns/intro_CPU_ISA_PL}}

\EN{\input{patterns/numeral_EN}}
\RU{\input{patterns/numeral_RU}}
\ITA{\input{patterns/numeral_ITA}}
\DE{\input{patterns/numeral_DE}}
\FR{\input{patterns/numeral_FR}}
\PL{\input{patterns/numeral_PL}}

% chapters
\input{patterns/00_empty/main}
\input{patterns/011_ret/main}
\input{patterns/01_helloworld/main}
\input{patterns/015_prolog_epilogue/main}
\input{patterns/02_stack/main}
\input{patterns/03_printf/main}
\input{patterns/04_scanf/main}
\input{patterns/05_passing_arguments/main}
\input{patterns/06_return_results/main}
\input{patterns/061_pointers/main}
\input{patterns/065_GOTO/main}
\input{patterns/07_jcc/main}
\input{patterns/08_switch/main}
\input{patterns/09_loops/main}
\input{patterns/10_strings/main}
\input{patterns/11_arith_optimizations/main}
\input{patterns/12_FPU/main}
\input{patterns/13_arrays/main}
\input{patterns/14_bitfields/main}
\EN{\input{patterns/145_LCG/main_EN}}
\RU{\input{patterns/145_LCG/main_RU}}
\input{patterns/15_structs/main}
\input{patterns/17_unions/main}
\input{patterns/18_pointers_to_functions/main}
\input{patterns/185_64bit_in_32_env/main}

\EN{\input{patterns/19_SIMD/main_EN}}
\RU{\input{patterns/19_SIMD/main_RU}}
\DE{\input{patterns/19_SIMD/main_DE}}

\EN{\input{patterns/20_x64/main_EN}}
\RU{\input{patterns/20_x64/main_RU}}

\EN{\input{patterns/205_floating_SIMD/main_EN}}
\RU{\input{patterns/205_floating_SIMD/main_RU}}
\DE{\input{patterns/205_floating_SIMD/main_DE}}

\EN{\input{patterns/ARM/main_EN}}
\RU{\input{patterns/ARM/main_RU}}
\DE{\input{patterns/ARM/main_DE}}

\input{patterns/MIPS/main}

\ifdefined\SPANISH
\chapter{Patrones de código}
\fi % SPANISH

\ifdefined\GERMAN
\chapter{Code-Muster}
\fi % GERMAN

\ifdefined\ENGLISH
\chapter{Code Patterns}
\fi % ENGLISH

\ifdefined\ITALIAN
\chapter{Forme di codice}
\fi % ITALIAN

\ifdefined\RUSSIAN
\chapter{Образцы кода}
\fi % RUSSIAN

\ifdefined\BRAZILIAN
\chapter{Padrões de códigos}
\fi % BRAZILIAN

\ifdefined\THAI
\chapter{รูปแบบของโค้ด}
\fi % THAI

\ifdefined\FRENCH
\chapter{Modèle de code}
\fi % FRENCH

\ifdefined\POLISH
\chapter{\PLph{}}
\fi % POLISH

% sections
\EN{\input{patterns/patterns_opt_dbg_EN}}
\ES{\input{patterns/patterns_opt_dbg_ES}}
\ITA{\input{patterns/patterns_opt_dbg_ITA}}
\PTBR{\input{patterns/patterns_opt_dbg_PTBR}}
\RU{\input{patterns/patterns_opt_dbg_RU}}
\THA{\input{patterns/patterns_opt_dbg_THA}}
\DE{\input{patterns/patterns_opt_dbg_DE}}
\FR{\input{patterns/patterns_opt_dbg_FR}}
\PL{\input{patterns/patterns_opt_dbg_PL}}

\RU{\section{Некоторые базовые понятия}}
\EN{\section{Some basics}}
\DE{\section{Einige Grundlagen}}
\FR{\section{Quelques bases}}
\ES{\section{\ESph{}}}
\ITA{\section{Alcune basi teoriche}}
\PTBR{\section{\PTBRph{}}}
\THA{\section{\THAph{}}}
\PL{\section{\PLph{}}}

% sections:
\EN{\input{patterns/intro_CPU_ISA_EN}}
\ES{\input{patterns/intro_CPU_ISA_ES}}
\ITA{\input{patterns/intro_CPU_ISA_ITA}}
\PTBR{\input{patterns/intro_CPU_ISA_PTBR}}
\RU{\input{patterns/intro_CPU_ISA_RU}}
\DE{\input{patterns/intro_CPU_ISA_DE}}
\FR{\input{patterns/intro_CPU_ISA_FR}}
\PL{\input{patterns/intro_CPU_ISA_PL}}

\EN{\input{patterns/numeral_EN}}
\RU{\input{patterns/numeral_RU}}
\ITA{\input{patterns/numeral_ITA}}
\DE{\input{patterns/numeral_DE}}
\FR{\input{patterns/numeral_FR}}
\PL{\input{patterns/numeral_PL}}

% chapters
\input{patterns/00_empty/main}
\input{patterns/011_ret/main}
\input{patterns/01_helloworld/main}
\input{patterns/015_prolog_epilogue/main}
\input{patterns/02_stack/main}
\input{patterns/03_printf/main}
\input{patterns/04_scanf/main}
\input{patterns/05_passing_arguments/main}
\input{patterns/06_return_results/main}
\input{patterns/061_pointers/main}
\input{patterns/065_GOTO/main}
\input{patterns/07_jcc/main}
\input{patterns/08_switch/main}
\input{patterns/09_loops/main}
\input{patterns/10_strings/main}
\input{patterns/11_arith_optimizations/main}
\input{patterns/12_FPU/main}
\input{patterns/13_arrays/main}
\input{patterns/14_bitfields/main}
\EN{\input{patterns/145_LCG/main_EN}}
\RU{\input{patterns/145_LCG/main_RU}}
\input{patterns/15_structs/main}
\input{patterns/17_unions/main}
\input{patterns/18_pointers_to_functions/main}
\input{patterns/185_64bit_in_32_env/main}

\EN{\input{patterns/19_SIMD/main_EN}}
\RU{\input{patterns/19_SIMD/main_RU}}
\DE{\input{patterns/19_SIMD/main_DE}}

\EN{\input{patterns/20_x64/main_EN}}
\RU{\input{patterns/20_x64/main_RU}}

\EN{\input{patterns/205_floating_SIMD/main_EN}}
\RU{\input{patterns/205_floating_SIMD/main_RU}}
\DE{\input{patterns/205_floating_SIMD/main_DE}}

\EN{\input{patterns/ARM/main_EN}}
\RU{\input{patterns/ARM/main_RU}}
\DE{\input{patterns/ARM/main_DE}}

\input{patterns/MIPS/main}

\ifdefined\SPANISH
\chapter{Patrones de código}
\fi % SPANISH

\ifdefined\GERMAN
\chapter{Code-Muster}
\fi % GERMAN

\ifdefined\ENGLISH
\chapter{Code Patterns}
\fi % ENGLISH

\ifdefined\ITALIAN
\chapter{Forme di codice}
\fi % ITALIAN

\ifdefined\RUSSIAN
\chapter{Образцы кода}
\fi % RUSSIAN

\ifdefined\BRAZILIAN
\chapter{Padrões de códigos}
\fi % BRAZILIAN

\ifdefined\THAI
\chapter{รูปแบบของโค้ด}
\fi % THAI

\ifdefined\FRENCH
\chapter{Modèle de code}
\fi % FRENCH

\ifdefined\POLISH
\chapter{\PLph{}}
\fi % POLISH

% sections
\EN{\input{patterns/patterns_opt_dbg_EN}}
\ES{\input{patterns/patterns_opt_dbg_ES}}
\ITA{\input{patterns/patterns_opt_dbg_ITA}}
\PTBR{\input{patterns/patterns_opt_dbg_PTBR}}
\RU{\input{patterns/patterns_opt_dbg_RU}}
\THA{\input{patterns/patterns_opt_dbg_THA}}
\DE{\input{patterns/patterns_opt_dbg_DE}}
\FR{\input{patterns/patterns_opt_dbg_FR}}
\PL{\input{patterns/patterns_opt_dbg_PL}}

\RU{\section{Некоторые базовые понятия}}
\EN{\section{Some basics}}
\DE{\section{Einige Grundlagen}}
\FR{\section{Quelques bases}}
\ES{\section{\ESph{}}}
\ITA{\section{Alcune basi teoriche}}
\PTBR{\section{\PTBRph{}}}
\THA{\section{\THAph{}}}
\PL{\section{\PLph{}}}

% sections:
\EN{\input{patterns/intro_CPU_ISA_EN}}
\ES{\input{patterns/intro_CPU_ISA_ES}}
\ITA{\input{patterns/intro_CPU_ISA_ITA}}
\PTBR{\input{patterns/intro_CPU_ISA_PTBR}}
\RU{\input{patterns/intro_CPU_ISA_RU}}
\DE{\input{patterns/intro_CPU_ISA_DE}}
\FR{\input{patterns/intro_CPU_ISA_FR}}
\PL{\input{patterns/intro_CPU_ISA_PL}}

\EN{\input{patterns/numeral_EN}}
\RU{\input{patterns/numeral_RU}}
\ITA{\input{patterns/numeral_ITA}}
\DE{\input{patterns/numeral_DE}}
\FR{\input{patterns/numeral_FR}}
\PL{\input{patterns/numeral_PL}}

% chapters
\input{patterns/00_empty/main}
\input{patterns/011_ret/main}
\input{patterns/01_helloworld/main}
\input{patterns/015_prolog_epilogue/main}
\input{patterns/02_stack/main}
\input{patterns/03_printf/main}
\input{patterns/04_scanf/main}
\input{patterns/05_passing_arguments/main}
\input{patterns/06_return_results/main}
\input{patterns/061_pointers/main}
\input{patterns/065_GOTO/main}
\input{patterns/07_jcc/main}
\input{patterns/08_switch/main}
\input{patterns/09_loops/main}
\input{patterns/10_strings/main}
\input{patterns/11_arith_optimizations/main}
\input{patterns/12_FPU/main}
\input{patterns/13_arrays/main}
\input{patterns/14_bitfields/main}
\EN{\input{patterns/145_LCG/main_EN}}
\RU{\input{patterns/145_LCG/main_RU}}
\input{patterns/15_structs/main}
\input{patterns/17_unions/main}
\input{patterns/18_pointers_to_functions/main}
\input{patterns/185_64bit_in_32_env/main}

\EN{\input{patterns/19_SIMD/main_EN}}
\RU{\input{patterns/19_SIMD/main_RU}}
\DE{\input{patterns/19_SIMD/main_DE}}

\EN{\input{patterns/20_x64/main_EN}}
\RU{\input{patterns/20_x64/main_RU}}

\EN{\input{patterns/205_floating_SIMD/main_EN}}
\RU{\input{patterns/205_floating_SIMD/main_RU}}
\DE{\input{patterns/205_floating_SIMD/main_DE}}

\EN{\input{patterns/ARM/main_EN}}
\RU{\input{patterns/ARM/main_RU}}
\DE{\input{patterns/ARM/main_DE}}

\input{patterns/MIPS/main}

\ifdefined\SPANISH
\chapter{Patrones de código}
\fi % SPANISH

\ifdefined\GERMAN
\chapter{Code-Muster}
\fi % GERMAN

\ifdefined\ENGLISH
\chapter{Code Patterns}
\fi % ENGLISH

\ifdefined\ITALIAN
\chapter{Forme di codice}
\fi % ITALIAN

\ifdefined\RUSSIAN
\chapter{Образцы кода}
\fi % RUSSIAN

\ifdefined\BRAZILIAN
\chapter{Padrões de códigos}
\fi % BRAZILIAN

\ifdefined\THAI
\chapter{รูปแบบของโค้ด}
\fi % THAI

\ifdefined\FRENCH
\chapter{Modèle de code}
\fi % FRENCH

\ifdefined\POLISH
\chapter{\PLph{}}
\fi % POLISH

% sections
\EN{\input{patterns/patterns_opt_dbg_EN}}
\ES{\input{patterns/patterns_opt_dbg_ES}}
\ITA{\input{patterns/patterns_opt_dbg_ITA}}
\PTBR{\input{patterns/patterns_opt_dbg_PTBR}}
\RU{\input{patterns/patterns_opt_dbg_RU}}
\THA{\input{patterns/patterns_opt_dbg_THA}}
\DE{\input{patterns/patterns_opt_dbg_DE}}
\FR{\input{patterns/patterns_opt_dbg_FR}}
\PL{\input{patterns/patterns_opt_dbg_PL}}

\RU{\section{Некоторые базовые понятия}}
\EN{\section{Some basics}}
\DE{\section{Einige Grundlagen}}
\FR{\section{Quelques bases}}
\ES{\section{\ESph{}}}
\ITA{\section{Alcune basi teoriche}}
\PTBR{\section{\PTBRph{}}}
\THA{\section{\THAph{}}}
\PL{\section{\PLph{}}}

% sections:
\EN{\input{patterns/intro_CPU_ISA_EN}}
\ES{\input{patterns/intro_CPU_ISA_ES}}
\ITA{\input{patterns/intro_CPU_ISA_ITA}}
\PTBR{\input{patterns/intro_CPU_ISA_PTBR}}
\RU{\input{patterns/intro_CPU_ISA_RU}}
\DE{\input{patterns/intro_CPU_ISA_DE}}
\FR{\input{patterns/intro_CPU_ISA_FR}}
\PL{\input{patterns/intro_CPU_ISA_PL}}

\EN{\input{patterns/numeral_EN}}
\RU{\input{patterns/numeral_RU}}
\ITA{\input{patterns/numeral_ITA}}
\DE{\input{patterns/numeral_DE}}
\FR{\input{patterns/numeral_FR}}
\PL{\input{patterns/numeral_PL}}

% chapters
\input{patterns/00_empty/main}
\input{patterns/011_ret/main}
\input{patterns/01_helloworld/main}
\input{patterns/015_prolog_epilogue/main}
\input{patterns/02_stack/main}
\input{patterns/03_printf/main}
\input{patterns/04_scanf/main}
\input{patterns/05_passing_arguments/main}
\input{patterns/06_return_results/main}
\input{patterns/061_pointers/main}
\input{patterns/065_GOTO/main}
\input{patterns/07_jcc/main}
\input{patterns/08_switch/main}
\input{patterns/09_loops/main}
\input{patterns/10_strings/main}
\input{patterns/11_arith_optimizations/main}
\input{patterns/12_FPU/main}
\input{patterns/13_arrays/main}
\input{patterns/14_bitfields/main}
\EN{\input{patterns/145_LCG/main_EN}}
\RU{\input{patterns/145_LCG/main_RU}}
\input{patterns/15_structs/main}
\input{patterns/17_unions/main}
\input{patterns/18_pointers_to_functions/main}
\input{patterns/185_64bit_in_32_env/main}

\EN{\input{patterns/19_SIMD/main_EN}}
\RU{\input{patterns/19_SIMD/main_RU}}
\DE{\input{patterns/19_SIMD/main_DE}}

\EN{\input{patterns/20_x64/main_EN}}
\RU{\input{patterns/20_x64/main_RU}}

\EN{\input{patterns/205_floating_SIMD/main_EN}}
\RU{\input{patterns/205_floating_SIMD/main_RU}}
\DE{\input{patterns/205_floating_SIMD/main_DE}}

\EN{\input{patterns/ARM/main_EN}}
\RU{\input{patterns/ARM/main_RU}}
\DE{\input{patterns/ARM/main_DE}}

\input{patterns/MIPS/main}

\ifdefined\SPANISH
\chapter{Patrones de código}
\fi % SPANISH

\ifdefined\GERMAN
\chapter{Code-Muster}
\fi % GERMAN

\ifdefined\ENGLISH
\chapter{Code Patterns}
\fi % ENGLISH

\ifdefined\ITALIAN
\chapter{Forme di codice}
\fi % ITALIAN

\ifdefined\RUSSIAN
\chapter{Образцы кода}
\fi % RUSSIAN

\ifdefined\BRAZILIAN
\chapter{Padrões de códigos}
\fi % BRAZILIAN

\ifdefined\THAI
\chapter{รูปแบบของโค้ด}
\fi % THAI

\ifdefined\FRENCH
\chapter{Modèle de code}
\fi % FRENCH

\ifdefined\POLISH
\chapter{\PLph{}}
\fi % POLISH

% sections
\EN{\input{patterns/patterns_opt_dbg_EN}}
\ES{\input{patterns/patterns_opt_dbg_ES}}
\ITA{\input{patterns/patterns_opt_dbg_ITA}}
\PTBR{\input{patterns/patterns_opt_dbg_PTBR}}
\RU{\input{patterns/patterns_opt_dbg_RU}}
\THA{\input{patterns/patterns_opt_dbg_THA}}
\DE{\input{patterns/patterns_opt_dbg_DE}}
\FR{\input{patterns/patterns_opt_dbg_FR}}
\PL{\input{patterns/patterns_opt_dbg_PL}}

\RU{\section{Некоторые базовые понятия}}
\EN{\section{Some basics}}
\DE{\section{Einige Grundlagen}}
\FR{\section{Quelques bases}}
\ES{\section{\ESph{}}}
\ITA{\section{Alcune basi teoriche}}
\PTBR{\section{\PTBRph{}}}
\THA{\section{\THAph{}}}
\PL{\section{\PLph{}}}

% sections:
\EN{\input{patterns/intro_CPU_ISA_EN}}
\ES{\input{patterns/intro_CPU_ISA_ES}}
\ITA{\input{patterns/intro_CPU_ISA_ITA}}
\PTBR{\input{patterns/intro_CPU_ISA_PTBR}}
\RU{\input{patterns/intro_CPU_ISA_RU}}
\DE{\input{patterns/intro_CPU_ISA_DE}}
\FR{\input{patterns/intro_CPU_ISA_FR}}
\PL{\input{patterns/intro_CPU_ISA_PL}}

\EN{\input{patterns/numeral_EN}}
\RU{\input{patterns/numeral_RU}}
\ITA{\input{patterns/numeral_ITA}}
\DE{\input{patterns/numeral_DE}}
\FR{\input{patterns/numeral_FR}}
\PL{\input{patterns/numeral_PL}}

% chapters
\input{patterns/00_empty/main}
\input{patterns/011_ret/main}
\input{patterns/01_helloworld/main}
\input{patterns/015_prolog_epilogue/main}
\input{patterns/02_stack/main}
\input{patterns/03_printf/main}
\input{patterns/04_scanf/main}
\input{patterns/05_passing_arguments/main}
\input{patterns/06_return_results/main}
\input{patterns/061_pointers/main}
\input{patterns/065_GOTO/main}
\input{patterns/07_jcc/main}
\input{patterns/08_switch/main}
\input{patterns/09_loops/main}
\input{patterns/10_strings/main}
\input{patterns/11_arith_optimizations/main}
\input{patterns/12_FPU/main}
\input{patterns/13_arrays/main}
\input{patterns/14_bitfields/main}
\EN{\input{patterns/145_LCG/main_EN}}
\RU{\input{patterns/145_LCG/main_RU}}
\input{patterns/15_structs/main}
\input{patterns/17_unions/main}
\input{patterns/18_pointers_to_functions/main}
\input{patterns/185_64bit_in_32_env/main}

\EN{\input{patterns/19_SIMD/main_EN}}
\RU{\input{patterns/19_SIMD/main_RU}}
\DE{\input{patterns/19_SIMD/main_DE}}

\EN{\input{patterns/20_x64/main_EN}}
\RU{\input{patterns/20_x64/main_RU}}

\EN{\input{patterns/205_floating_SIMD/main_EN}}
\RU{\input{patterns/205_floating_SIMD/main_RU}}
\DE{\input{patterns/205_floating_SIMD/main_DE}}

\EN{\input{patterns/ARM/main_EN}}
\RU{\input{patterns/ARM/main_RU}}
\DE{\input{patterns/ARM/main_DE}}

\input{patterns/MIPS/main}

\ifdefined\SPANISH
\chapter{Patrones de código}
\fi % SPANISH

\ifdefined\GERMAN
\chapter{Code-Muster}
\fi % GERMAN

\ifdefined\ENGLISH
\chapter{Code Patterns}
\fi % ENGLISH

\ifdefined\ITALIAN
\chapter{Forme di codice}
\fi % ITALIAN

\ifdefined\RUSSIAN
\chapter{Образцы кода}
\fi % RUSSIAN

\ifdefined\BRAZILIAN
\chapter{Padrões de códigos}
\fi % BRAZILIAN

\ifdefined\THAI
\chapter{รูปแบบของโค้ด}
\fi % THAI

\ifdefined\FRENCH
\chapter{Modèle de code}
\fi % FRENCH

\ifdefined\POLISH
\chapter{\PLph{}}
\fi % POLISH

% sections
\EN{\input{patterns/patterns_opt_dbg_EN}}
\ES{\input{patterns/patterns_opt_dbg_ES}}
\ITA{\input{patterns/patterns_opt_dbg_ITA}}
\PTBR{\input{patterns/patterns_opt_dbg_PTBR}}
\RU{\input{patterns/patterns_opt_dbg_RU}}
\THA{\input{patterns/patterns_opt_dbg_THA}}
\DE{\input{patterns/patterns_opt_dbg_DE}}
\FR{\input{patterns/patterns_opt_dbg_FR}}
\PL{\input{patterns/patterns_opt_dbg_PL}}

\RU{\section{Некоторые базовые понятия}}
\EN{\section{Some basics}}
\DE{\section{Einige Grundlagen}}
\FR{\section{Quelques bases}}
\ES{\section{\ESph{}}}
\ITA{\section{Alcune basi teoriche}}
\PTBR{\section{\PTBRph{}}}
\THA{\section{\THAph{}}}
\PL{\section{\PLph{}}}

% sections:
\EN{\input{patterns/intro_CPU_ISA_EN}}
\ES{\input{patterns/intro_CPU_ISA_ES}}
\ITA{\input{patterns/intro_CPU_ISA_ITA}}
\PTBR{\input{patterns/intro_CPU_ISA_PTBR}}
\RU{\input{patterns/intro_CPU_ISA_RU}}
\DE{\input{patterns/intro_CPU_ISA_DE}}
\FR{\input{patterns/intro_CPU_ISA_FR}}
\PL{\input{patterns/intro_CPU_ISA_PL}}

\EN{\input{patterns/numeral_EN}}
\RU{\input{patterns/numeral_RU}}
\ITA{\input{patterns/numeral_ITA}}
\DE{\input{patterns/numeral_DE}}
\FR{\input{patterns/numeral_FR}}
\PL{\input{patterns/numeral_PL}}

% chapters
\input{patterns/00_empty/main}
\input{patterns/011_ret/main}
\input{patterns/01_helloworld/main}
\input{patterns/015_prolog_epilogue/main}
\input{patterns/02_stack/main}
\input{patterns/03_printf/main}
\input{patterns/04_scanf/main}
\input{patterns/05_passing_arguments/main}
\input{patterns/06_return_results/main}
\input{patterns/061_pointers/main}
\input{patterns/065_GOTO/main}
\input{patterns/07_jcc/main}
\input{patterns/08_switch/main}
\input{patterns/09_loops/main}
\input{patterns/10_strings/main}
\input{patterns/11_arith_optimizations/main}
\input{patterns/12_FPU/main}
\input{patterns/13_arrays/main}
\input{patterns/14_bitfields/main}
\EN{\input{patterns/145_LCG/main_EN}}
\RU{\input{patterns/145_LCG/main_RU}}
\input{patterns/15_structs/main}
\input{patterns/17_unions/main}
\input{patterns/18_pointers_to_functions/main}
\input{patterns/185_64bit_in_32_env/main}

\EN{\input{patterns/19_SIMD/main_EN}}
\RU{\input{patterns/19_SIMD/main_RU}}
\DE{\input{patterns/19_SIMD/main_DE}}

\EN{\input{patterns/20_x64/main_EN}}
\RU{\input{patterns/20_x64/main_RU}}

\EN{\input{patterns/205_floating_SIMD/main_EN}}
\RU{\input{patterns/205_floating_SIMD/main_RU}}
\DE{\input{patterns/205_floating_SIMD/main_DE}}

\EN{\input{patterns/ARM/main_EN}}
\RU{\input{patterns/ARM/main_RU}}
\DE{\input{patterns/ARM/main_DE}}

\input{patterns/MIPS/main}

\ifdefined\SPANISH
\chapter{Patrones de código}
\fi % SPANISH

\ifdefined\GERMAN
\chapter{Code-Muster}
\fi % GERMAN

\ifdefined\ENGLISH
\chapter{Code Patterns}
\fi % ENGLISH

\ifdefined\ITALIAN
\chapter{Forme di codice}
\fi % ITALIAN

\ifdefined\RUSSIAN
\chapter{Образцы кода}
\fi % RUSSIAN

\ifdefined\BRAZILIAN
\chapter{Padrões de códigos}
\fi % BRAZILIAN

\ifdefined\THAI
\chapter{รูปแบบของโค้ด}
\fi % THAI

\ifdefined\FRENCH
\chapter{Modèle de code}
\fi % FRENCH

\ifdefined\POLISH
\chapter{\PLph{}}
\fi % POLISH

% sections
\EN{\input{patterns/patterns_opt_dbg_EN}}
\ES{\input{patterns/patterns_opt_dbg_ES}}
\ITA{\input{patterns/patterns_opt_dbg_ITA}}
\PTBR{\input{patterns/patterns_opt_dbg_PTBR}}
\RU{\input{patterns/patterns_opt_dbg_RU}}
\THA{\input{patterns/patterns_opt_dbg_THA}}
\DE{\input{patterns/patterns_opt_dbg_DE}}
\FR{\input{patterns/patterns_opt_dbg_FR}}
\PL{\input{patterns/patterns_opt_dbg_PL}}

\RU{\section{Некоторые базовые понятия}}
\EN{\section{Some basics}}
\DE{\section{Einige Grundlagen}}
\FR{\section{Quelques bases}}
\ES{\section{\ESph{}}}
\ITA{\section{Alcune basi teoriche}}
\PTBR{\section{\PTBRph{}}}
\THA{\section{\THAph{}}}
\PL{\section{\PLph{}}}

% sections:
\EN{\input{patterns/intro_CPU_ISA_EN}}
\ES{\input{patterns/intro_CPU_ISA_ES}}
\ITA{\input{patterns/intro_CPU_ISA_ITA}}
\PTBR{\input{patterns/intro_CPU_ISA_PTBR}}
\RU{\input{patterns/intro_CPU_ISA_RU}}
\DE{\input{patterns/intro_CPU_ISA_DE}}
\FR{\input{patterns/intro_CPU_ISA_FR}}
\PL{\input{patterns/intro_CPU_ISA_PL}}

\EN{\input{patterns/numeral_EN}}
\RU{\input{patterns/numeral_RU}}
\ITA{\input{patterns/numeral_ITA}}
\DE{\input{patterns/numeral_DE}}
\FR{\input{patterns/numeral_FR}}
\PL{\input{patterns/numeral_PL}}

% chapters
\input{patterns/00_empty/main}
\input{patterns/011_ret/main}
\input{patterns/01_helloworld/main}
\input{patterns/015_prolog_epilogue/main}
\input{patterns/02_stack/main}
\input{patterns/03_printf/main}
\input{patterns/04_scanf/main}
\input{patterns/05_passing_arguments/main}
\input{patterns/06_return_results/main}
\input{patterns/061_pointers/main}
\input{patterns/065_GOTO/main}
\input{patterns/07_jcc/main}
\input{patterns/08_switch/main}
\input{patterns/09_loops/main}
\input{patterns/10_strings/main}
\input{patterns/11_arith_optimizations/main}
\input{patterns/12_FPU/main}
\input{patterns/13_arrays/main}
\input{patterns/14_bitfields/main}
\EN{\input{patterns/145_LCG/main_EN}}
\RU{\input{patterns/145_LCG/main_RU}}
\input{patterns/15_structs/main}
\input{patterns/17_unions/main}
\input{patterns/18_pointers_to_functions/main}
\input{patterns/185_64bit_in_32_env/main}

\EN{\input{patterns/19_SIMD/main_EN}}
\RU{\input{patterns/19_SIMD/main_RU}}
\DE{\input{patterns/19_SIMD/main_DE}}

\EN{\input{patterns/20_x64/main_EN}}
\RU{\input{patterns/20_x64/main_RU}}

\EN{\input{patterns/205_floating_SIMD/main_EN}}
\RU{\input{patterns/205_floating_SIMD/main_RU}}
\DE{\input{patterns/205_floating_SIMD/main_DE}}

\EN{\input{patterns/ARM/main_EN}}
\RU{\input{patterns/ARM/main_RU}}
\DE{\input{patterns/ARM/main_DE}}

\input{patterns/MIPS/main}

\ifdefined\SPANISH
\chapter{Patrones de código}
\fi % SPANISH

\ifdefined\GERMAN
\chapter{Code-Muster}
\fi % GERMAN

\ifdefined\ENGLISH
\chapter{Code Patterns}
\fi % ENGLISH

\ifdefined\ITALIAN
\chapter{Forme di codice}
\fi % ITALIAN

\ifdefined\RUSSIAN
\chapter{Образцы кода}
\fi % RUSSIAN

\ifdefined\BRAZILIAN
\chapter{Padrões de códigos}
\fi % BRAZILIAN

\ifdefined\THAI
\chapter{รูปแบบของโค้ด}
\fi % THAI

\ifdefined\FRENCH
\chapter{Modèle de code}
\fi % FRENCH

\ifdefined\POLISH
\chapter{\PLph{}}
\fi % POLISH

% sections
\EN{\input{patterns/patterns_opt_dbg_EN}}
\ES{\input{patterns/patterns_opt_dbg_ES}}
\ITA{\input{patterns/patterns_opt_dbg_ITA}}
\PTBR{\input{patterns/patterns_opt_dbg_PTBR}}
\RU{\input{patterns/patterns_opt_dbg_RU}}
\THA{\input{patterns/patterns_opt_dbg_THA}}
\DE{\input{patterns/patterns_opt_dbg_DE}}
\FR{\input{patterns/patterns_opt_dbg_FR}}
\PL{\input{patterns/patterns_opt_dbg_PL}}

\RU{\section{Некоторые базовые понятия}}
\EN{\section{Some basics}}
\DE{\section{Einige Grundlagen}}
\FR{\section{Quelques bases}}
\ES{\section{\ESph{}}}
\ITA{\section{Alcune basi teoriche}}
\PTBR{\section{\PTBRph{}}}
\THA{\section{\THAph{}}}
\PL{\section{\PLph{}}}

% sections:
\EN{\input{patterns/intro_CPU_ISA_EN}}
\ES{\input{patterns/intro_CPU_ISA_ES}}
\ITA{\input{patterns/intro_CPU_ISA_ITA}}
\PTBR{\input{patterns/intro_CPU_ISA_PTBR}}
\RU{\input{patterns/intro_CPU_ISA_RU}}
\DE{\input{patterns/intro_CPU_ISA_DE}}
\FR{\input{patterns/intro_CPU_ISA_FR}}
\PL{\input{patterns/intro_CPU_ISA_PL}}

\EN{\input{patterns/numeral_EN}}
\RU{\input{patterns/numeral_RU}}
\ITA{\input{patterns/numeral_ITA}}
\DE{\input{patterns/numeral_DE}}
\FR{\input{patterns/numeral_FR}}
\PL{\input{patterns/numeral_PL}}

% chapters
\input{patterns/00_empty/main}
\input{patterns/011_ret/main}
\input{patterns/01_helloworld/main}
\input{patterns/015_prolog_epilogue/main}
\input{patterns/02_stack/main}
\input{patterns/03_printf/main}
\input{patterns/04_scanf/main}
\input{patterns/05_passing_arguments/main}
\input{patterns/06_return_results/main}
\input{patterns/061_pointers/main}
\input{patterns/065_GOTO/main}
\input{patterns/07_jcc/main}
\input{patterns/08_switch/main}
\input{patterns/09_loops/main}
\input{patterns/10_strings/main}
\input{patterns/11_arith_optimizations/main}
\input{patterns/12_FPU/main}
\input{patterns/13_arrays/main}
\input{patterns/14_bitfields/main}
\EN{\input{patterns/145_LCG/main_EN}}
\RU{\input{patterns/145_LCG/main_RU}}
\input{patterns/15_structs/main}
\input{patterns/17_unions/main}
\input{patterns/18_pointers_to_functions/main}
\input{patterns/185_64bit_in_32_env/main}

\EN{\input{patterns/19_SIMD/main_EN}}
\RU{\input{patterns/19_SIMD/main_RU}}
\DE{\input{patterns/19_SIMD/main_DE}}

\EN{\input{patterns/20_x64/main_EN}}
\RU{\input{patterns/20_x64/main_RU}}

\EN{\input{patterns/205_floating_SIMD/main_EN}}
\RU{\input{patterns/205_floating_SIMD/main_RU}}
\DE{\input{patterns/205_floating_SIMD/main_DE}}

\EN{\input{patterns/ARM/main_EN}}
\RU{\input{patterns/ARM/main_RU}}
\DE{\input{patterns/ARM/main_DE}}

\input{patterns/MIPS/main}

\ifdefined\SPANISH
\chapter{Patrones de código}
\fi % SPANISH

\ifdefined\GERMAN
\chapter{Code-Muster}
\fi % GERMAN

\ifdefined\ENGLISH
\chapter{Code Patterns}
\fi % ENGLISH

\ifdefined\ITALIAN
\chapter{Forme di codice}
\fi % ITALIAN

\ifdefined\RUSSIAN
\chapter{Образцы кода}
\fi % RUSSIAN

\ifdefined\BRAZILIAN
\chapter{Padrões de códigos}
\fi % BRAZILIAN

\ifdefined\THAI
\chapter{รูปแบบของโค้ด}
\fi % THAI

\ifdefined\FRENCH
\chapter{Modèle de code}
\fi % FRENCH

\ifdefined\POLISH
\chapter{\PLph{}}
\fi % POLISH

% sections
\EN{\input{patterns/patterns_opt_dbg_EN}}
\ES{\input{patterns/patterns_opt_dbg_ES}}
\ITA{\input{patterns/patterns_opt_dbg_ITA}}
\PTBR{\input{patterns/patterns_opt_dbg_PTBR}}
\RU{\input{patterns/patterns_opt_dbg_RU}}
\THA{\input{patterns/patterns_opt_dbg_THA}}
\DE{\input{patterns/patterns_opt_dbg_DE}}
\FR{\input{patterns/patterns_opt_dbg_FR}}
\PL{\input{patterns/patterns_opt_dbg_PL}}

\RU{\section{Некоторые базовые понятия}}
\EN{\section{Some basics}}
\DE{\section{Einige Grundlagen}}
\FR{\section{Quelques bases}}
\ES{\section{\ESph{}}}
\ITA{\section{Alcune basi teoriche}}
\PTBR{\section{\PTBRph{}}}
\THA{\section{\THAph{}}}
\PL{\section{\PLph{}}}

% sections:
\EN{\input{patterns/intro_CPU_ISA_EN}}
\ES{\input{patterns/intro_CPU_ISA_ES}}
\ITA{\input{patterns/intro_CPU_ISA_ITA}}
\PTBR{\input{patterns/intro_CPU_ISA_PTBR}}
\RU{\input{patterns/intro_CPU_ISA_RU}}
\DE{\input{patterns/intro_CPU_ISA_DE}}
\FR{\input{patterns/intro_CPU_ISA_FR}}
\PL{\input{patterns/intro_CPU_ISA_PL}}

\EN{\input{patterns/numeral_EN}}
\RU{\input{patterns/numeral_RU}}
\ITA{\input{patterns/numeral_ITA}}
\DE{\input{patterns/numeral_DE}}
\FR{\input{patterns/numeral_FR}}
\PL{\input{patterns/numeral_PL}}

% chapters
\input{patterns/00_empty/main}
\input{patterns/011_ret/main}
\input{patterns/01_helloworld/main}
\input{patterns/015_prolog_epilogue/main}
\input{patterns/02_stack/main}
\input{patterns/03_printf/main}
\input{patterns/04_scanf/main}
\input{patterns/05_passing_arguments/main}
\input{patterns/06_return_results/main}
\input{patterns/061_pointers/main}
\input{patterns/065_GOTO/main}
\input{patterns/07_jcc/main}
\input{patterns/08_switch/main}
\input{patterns/09_loops/main}
\input{patterns/10_strings/main}
\input{patterns/11_arith_optimizations/main}
\input{patterns/12_FPU/main}
\input{patterns/13_arrays/main}
\input{patterns/14_bitfields/main}
\EN{\input{patterns/145_LCG/main_EN}}
\RU{\input{patterns/145_LCG/main_RU}}
\input{patterns/15_structs/main}
\input{patterns/17_unions/main}
\input{patterns/18_pointers_to_functions/main}
\input{patterns/185_64bit_in_32_env/main}

\EN{\input{patterns/19_SIMD/main_EN}}
\RU{\input{patterns/19_SIMD/main_RU}}
\DE{\input{patterns/19_SIMD/main_DE}}

\EN{\input{patterns/20_x64/main_EN}}
\RU{\input{patterns/20_x64/main_RU}}

\EN{\input{patterns/205_floating_SIMD/main_EN}}
\RU{\input{patterns/205_floating_SIMD/main_RU}}
\DE{\input{patterns/205_floating_SIMD/main_DE}}

\EN{\input{patterns/ARM/main_EN}}
\RU{\input{patterns/ARM/main_RU}}
\DE{\input{patterns/ARM/main_DE}}

\input{patterns/MIPS/main}

\ifdefined\SPANISH
\chapter{Patrones de código}
\fi % SPANISH

\ifdefined\GERMAN
\chapter{Code-Muster}
\fi % GERMAN

\ifdefined\ENGLISH
\chapter{Code Patterns}
\fi % ENGLISH

\ifdefined\ITALIAN
\chapter{Forme di codice}
\fi % ITALIAN

\ifdefined\RUSSIAN
\chapter{Образцы кода}
\fi % RUSSIAN

\ifdefined\BRAZILIAN
\chapter{Padrões de códigos}
\fi % BRAZILIAN

\ifdefined\THAI
\chapter{รูปแบบของโค้ด}
\fi % THAI

\ifdefined\FRENCH
\chapter{Modèle de code}
\fi % FRENCH

\ifdefined\POLISH
\chapter{\PLph{}}
\fi % POLISH

% sections
\EN{\input{patterns/patterns_opt_dbg_EN}}
\ES{\input{patterns/patterns_opt_dbg_ES}}
\ITA{\input{patterns/patterns_opt_dbg_ITA}}
\PTBR{\input{patterns/patterns_opt_dbg_PTBR}}
\RU{\input{patterns/patterns_opt_dbg_RU}}
\THA{\input{patterns/patterns_opt_dbg_THA}}
\DE{\input{patterns/patterns_opt_dbg_DE}}
\FR{\input{patterns/patterns_opt_dbg_FR}}
\PL{\input{patterns/patterns_opt_dbg_PL}}

\RU{\section{Некоторые базовые понятия}}
\EN{\section{Some basics}}
\DE{\section{Einige Grundlagen}}
\FR{\section{Quelques bases}}
\ES{\section{\ESph{}}}
\ITA{\section{Alcune basi teoriche}}
\PTBR{\section{\PTBRph{}}}
\THA{\section{\THAph{}}}
\PL{\section{\PLph{}}}

% sections:
\EN{\input{patterns/intro_CPU_ISA_EN}}
\ES{\input{patterns/intro_CPU_ISA_ES}}
\ITA{\input{patterns/intro_CPU_ISA_ITA}}
\PTBR{\input{patterns/intro_CPU_ISA_PTBR}}
\RU{\input{patterns/intro_CPU_ISA_RU}}
\DE{\input{patterns/intro_CPU_ISA_DE}}
\FR{\input{patterns/intro_CPU_ISA_FR}}
\PL{\input{patterns/intro_CPU_ISA_PL}}

\EN{\input{patterns/numeral_EN}}
\RU{\input{patterns/numeral_RU}}
\ITA{\input{patterns/numeral_ITA}}
\DE{\input{patterns/numeral_DE}}
\FR{\input{patterns/numeral_FR}}
\PL{\input{patterns/numeral_PL}}

% chapters
\input{patterns/00_empty/main}
\input{patterns/011_ret/main}
\input{patterns/01_helloworld/main}
\input{patterns/015_prolog_epilogue/main}
\input{patterns/02_stack/main}
\input{patterns/03_printf/main}
\input{patterns/04_scanf/main}
\input{patterns/05_passing_arguments/main}
\input{patterns/06_return_results/main}
\input{patterns/061_pointers/main}
\input{patterns/065_GOTO/main}
\input{patterns/07_jcc/main}
\input{patterns/08_switch/main}
\input{patterns/09_loops/main}
\input{patterns/10_strings/main}
\input{patterns/11_arith_optimizations/main}
\input{patterns/12_FPU/main}
\input{patterns/13_arrays/main}
\input{patterns/14_bitfields/main}
\EN{\input{patterns/145_LCG/main_EN}}
\RU{\input{patterns/145_LCG/main_RU}}
\input{patterns/15_structs/main}
\input{patterns/17_unions/main}
\input{patterns/18_pointers_to_functions/main}
\input{patterns/185_64bit_in_32_env/main}

\EN{\input{patterns/19_SIMD/main_EN}}
\RU{\input{patterns/19_SIMD/main_RU}}
\DE{\input{patterns/19_SIMD/main_DE}}

\EN{\input{patterns/20_x64/main_EN}}
\RU{\input{patterns/20_x64/main_RU}}

\EN{\input{patterns/205_floating_SIMD/main_EN}}
\RU{\input{patterns/205_floating_SIMD/main_RU}}
\DE{\input{patterns/205_floating_SIMD/main_DE}}

\EN{\input{patterns/ARM/main_EN}}
\RU{\input{patterns/ARM/main_RU}}
\DE{\input{patterns/ARM/main_DE}}

\input{patterns/MIPS/main}

\ifdefined\SPANISH
\chapter{Patrones de código}
\fi % SPANISH

\ifdefined\GERMAN
\chapter{Code-Muster}
\fi % GERMAN

\ifdefined\ENGLISH
\chapter{Code Patterns}
\fi % ENGLISH

\ifdefined\ITALIAN
\chapter{Forme di codice}
\fi % ITALIAN

\ifdefined\RUSSIAN
\chapter{Образцы кода}
\fi % RUSSIAN

\ifdefined\BRAZILIAN
\chapter{Padrões de códigos}
\fi % BRAZILIAN

\ifdefined\THAI
\chapter{รูปแบบของโค้ด}
\fi % THAI

\ifdefined\FRENCH
\chapter{Modèle de code}
\fi % FRENCH

\ifdefined\POLISH
\chapter{\PLph{}}
\fi % POLISH

% sections
\EN{\input{patterns/patterns_opt_dbg_EN}}
\ES{\input{patterns/patterns_opt_dbg_ES}}
\ITA{\input{patterns/patterns_opt_dbg_ITA}}
\PTBR{\input{patterns/patterns_opt_dbg_PTBR}}
\RU{\input{patterns/patterns_opt_dbg_RU}}
\THA{\input{patterns/patterns_opt_dbg_THA}}
\DE{\input{patterns/patterns_opt_dbg_DE}}
\FR{\input{patterns/patterns_opt_dbg_FR}}
\PL{\input{patterns/patterns_opt_dbg_PL}}

\RU{\section{Некоторые базовые понятия}}
\EN{\section{Some basics}}
\DE{\section{Einige Grundlagen}}
\FR{\section{Quelques bases}}
\ES{\section{\ESph{}}}
\ITA{\section{Alcune basi teoriche}}
\PTBR{\section{\PTBRph{}}}
\THA{\section{\THAph{}}}
\PL{\section{\PLph{}}}

% sections:
\EN{\input{patterns/intro_CPU_ISA_EN}}
\ES{\input{patterns/intro_CPU_ISA_ES}}
\ITA{\input{patterns/intro_CPU_ISA_ITA}}
\PTBR{\input{patterns/intro_CPU_ISA_PTBR}}
\RU{\input{patterns/intro_CPU_ISA_RU}}
\DE{\input{patterns/intro_CPU_ISA_DE}}
\FR{\input{patterns/intro_CPU_ISA_FR}}
\PL{\input{patterns/intro_CPU_ISA_PL}}

\EN{\input{patterns/numeral_EN}}
\RU{\input{patterns/numeral_RU}}
\ITA{\input{patterns/numeral_ITA}}
\DE{\input{patterns/numeral_DE}}
\FR{\input{patterns/numeral_FR}}
\PL{\input{patterns/numeral_PL}}

% chapters
\input{patterns/00_empty/main}
\input{patterns/011_ret/main}
\input{patterns/01_helloworld/main}
\input{patterns/015_prolog_epilogue/main}
\input{patterns/02_stack/main}
\input{patterns/03_printf/main}
\input{patterns/04_scanf/main}
\input{patterns/05_passing_arguments/main}
\input{patterns/06_return_results/main}
\input{patterns/061_pointers/main}
\input{patterns/065_GOTO/main}
\input{patterns/07_jcc/main}
\input{patterns/08_switch/main}
\input{patterns/09_loops/main}
\input{patterns/10_strings/main}
\input{patterns/11_arith_optimizations/main}
\input{patterns/12_FPU/main}
\input{patterns/13_arrays/main}
\input{patterns/14_bitfields/main}
\EN{\input{patterns/145_LCG/main_EN}}
\RU{\input{patterns/145_LCG/main_RU}}
\input{patterns/15_structs/main}
\input{patterns/17_unions/main}
\input{patterns/18_pointers_to_functions/main}
\input{patterns/185_64bit_in_32_env/main}

\EN{\input{patterns/19_SIMD/main_EN}}
\RU{\input{patterns/19_SIMD/main_RU}}
\DE{\input{patterns/19_SIMD/main_DE}}

\EN{\input{patterns/20_x64/main_EN}}
\RU{\input{patterns/20_x64/main_RU}}

\EN{\input{patterns/205_floating_SIMD/main_EN}}
\RU{\input{patterns/205_floating_SIMD/main_RU}}
\DE{\input{patterns/205_floating_SIMD/main_DE}}

\EN{\input{patterns/ARM/main_EN}}
\RU{\input{patterns/ARM/main_RU}}
\DE{\input{patterns/ARM/main_DE}}

\input{patterns/MIPS/main}

\EN{\input{patterns/12_FPU/main_EN}}
\RU{\input{patterns/12_FPU/main_RU}}
\DE{\input{patterns/12_FPU/main_DE}}
\FR{\input{patterns/12_FPU/main_FR}}


\ifdefined\SPANISH
\chapter{Patrones de código}
\fi % SPANISH

\ifdefined\GERMAN
\chapter{Code-Muster}
\fi % GERMAN

\ifdefined\ENGLISH
\chapter{Code Patterns}
\fi % ENGLISH

\ifdefined\ITALIAN
\chapter{Forme di codice}
\fi % ITALIAN

\ifdefined\RUSSIAN
\chapter{Образцы кода}
\fi % RUSSIAN

\ifdefined\BRAZILIAN
\chapter{Padrões de códigos}
\fi % BRAZILIAN

\ifdefined\THAI
\chapter{รูปแบบของโค้ด}
\fi % THAI

\ifdefined\FRENCH
\chapter{Modèle de code}
\fi % FRENCH

\ifdefined\POLISH
\chapter{\PLph{}}
\fi % POLISH

% sections
\EN{\input{patterns/patterns_opt_dbg_EN}}
\ES{\input{patterns/patterns_opt_dbg_ES}}
\ITA{\input{patterns/patterns_opt_dbg_ITA}}
\PTBR{\input{patterns/patterns_opt_dbg_PTBR}}
\RU{\input{patterns/patterns_opt_dbg_RU}}
\THA{\input{patterns/patterns_opt_dbg_THA}}
\DE{\input{patterns/patterns_opt_dbg_DE}}
\FR{\input{patterns/patterns_opt_dbg_FR}}
\PL{\input{patterns/patterns_opt_dbg_PL}}

\RU{\section{Некоторые базовые понятия}}
\EN{\section{Some basics}}
\DE{\section{Einige Grundlagen}}
\FR{\section{Quelques bases}}
\ES{\section{\ESph{}}}
\ITA{\section{Alcune basi teoriche}}
\PTBR{\section{\PTBRph{}}}
\THA{\section{\THAph{}}}
\PL{\section{\PLph{}}}

% sections:
\EN{\input{patterns/intro_CPU_ISA_EN}}
\ES{\input{patterns/intro_CPU_ISA_ES}}
\ITA{\input{patterns/intro_CPU_ISA_ITA}}
\PTBR{\input{patterns/intro_CPU_ISA_PTBR}}
\RU{\input{patterns/intro_CPU_ISA_RU}}
\DE{\input{patterns/intro_CPU_ISA_DE}}
\FR{\input{patterns/intro_CPU_ISA_FR}}
\PL{\input{patterns/intro_CPU_ISA_PL}}

\EN{\input{patterns/numeral_EN}}
\RU{\input{patterns/numeral_RU}}
\ITA{\input{patterns/numeral_ITA}}
\DE{\input{patterns/numeral_DE}}
\FR{\input{patterns/numeral_FR}}
\PL{\input{patterns/numeral_PL}}

% chapters
\input{patterns/00_empty/main}
\input{patterns/011_ret/main}
\input{patterns/01_helloworld/main}
\input{patterns/015_prolog_epilogue/main}
\input{patterns/02_stack/main}
\input{patterns/03_printf/main}
\input{patterns/04_scanf/main}
\input{patterns/05_passing_arguments/main}
\input{patterns/06_return_results/main}
\input{patterns/061_pointers/main}
\input{patterns/065_GOTO/main}
\input{patterns/07_jcc/main}
\input{patterns/08_switch/main}
\input{patterns/09_loops/main}
\input{patterns/10_strings/main}
\input{patterns/11_arith_optimizations/main}
\input{patterns/12_FPU/main}
\input{patterns/13_arrays/main}
\input{patterns/14_bitfields/main}
\EN{\input{patterns/145_LCG/main_EN}}
\RU{\input{patterns/145_LCG/main_RU}}
\input{patterns/15_structs/main}
\input{patterns/17_unions/main}
\input{patterns/18_pointers_to_functions/main}
\input{patterns/185_64bit_in_32_env/main}

\EN{\input{patterns/19_SIMD/main_EN}}
\RU{\input{patterns/19_SIMD/main_RU}}
\DE{\input{patterns/19_SIMD/main_DE}}

\EN{\input{patterns/20_x64/main_EN}}
\RU{\input{patterns/20_x64/main_RU}}

\EN{\input{patterns/205_floating_SIMD/main_EN}}
\RU{\input{patterns/205_floating_SIMD/main_RU}}
\DE{\input{patterns/205_floating_SIMD/main_DE}}

\EN{\input{patterns/ARM/main_EN}}
\RU{\input{patterns/ARM/main_RU}}
\DE{\input{patterns/ARM/main_DE}}

\input{patterns/MIPS/main}

\ifdefined\SPANISH
\chapter{Patrones de código}
\fi % SPANISH

\ifdefined\GERMAN
\chapter{Code-Muster}
\fi % GERMAN

\ifdefined\ENGLISH
\chapter{Code Patterns}
\fi % ENGLISH

\ifdefined\ITALIAN
\chapter{Forme di codice}
\fi % ITALIAN

\ifdefined\RUSSIAN
\chapter{Образцы кода}
\fi % RUSSIAN

\ifdefined\BRAZILIAN
\chapter{Padrões de códigos}
\fi % BRAZILIAN

\ifdefined\THAI
\chapter{รูปแบบของโค้ด}
\fi % THAI

\ifdefined\FRENCH
\chapter{Modèle de code}
\fi % FRENCH

\ifdefined\POLISH
\chapter{\PLph{}}
\fi % POLISH

% sections
\EN{\input{patterns/patterns_opt_dbg_EN}}
\ES{\input{patterns/patterns_opt_dbg_ES}}
\ITA{\input{patterns/patterns_opt_dbg_ITA}}
\PTBR{\input{patterns/patterns_opt_dbg_PTBR}}
\RU{\input{patterns/patterns_opt_dbg_RU}}
\THA{\input{patterns/patterns_opt_dbg_THA}}
\DE{\input{patterns/patterns_opt_dbg_DE}}
\FR{\input{patterns/patterns_opt_dbg_FR}}
\PL{\input{patterns/patterns_opt_dbg_PL}}

\RU{\section{Некоторые базовые понятия}}
\EN{\section{Some basics}}
\DE{\section{Einige Grundlagen}}
\FR{\section{Quelques bases}}
\ES{\section{\ESph{}}}
\ITA{\section{Alcune basi teoriche}}
\PTBR{\section{\PTBRph{}}}
\THA{\section{\THAph{}}}
\PL{\section{\PLph{}}}

% sections:
\EN{\input{patterns/intro_CPU_ISA_EN}}
\ES{\input{patterns/intro_CPU_ISA_ES}}
\ITA{\input{patterns/intro_CPU_ISA_ITA}}
\PTBR{\input{patterns/intro_CPU_ISA_PTBR}}
\RU{\input{patterns/intro_CPU_ISA_RU}}
\DE{\input{patterns/intro_CPU_ISA_DE}}
\FR{\input{patterns/intro_CPU_ISA_FR}}
\PL{\input{patterns/intro_CPU_ISA_PL}}

\EN{\input{patterns/numeral_EN}}
\RU{\input{patterns/numeral_RU}}
\ITA{\input{patterns/numeral_ITA}}
\DE{\input{patterns/numeral_DE}}
\FR{\input{patterns/numeral_FR}}
\PL{\input{patterns/numeral_PL}}

% chapters
\input{patterns/00_empty/main}
\input{patterns/011_ret/main}
\input{patterns/01_helloworld/main}
\input{patterns/015_prolog_epilogue/main}
\input{patterns/02_stack/main}
\input{patterns/03_printf/main}
\input{patterns/04_scanf/main}
\input{patterns/05_passing_arguments/main}
\input{patterns/06_return_results/main}
\input{patterns/061_pointers/main}
\input{patterns/065_GOTO/main}
\input{patterns/07_jcc/main}
\input{patterns/08_switch/main}
\input{patterns/09_loops/main}
\input{patterns/10_strings/main}
\input{patterns/11_arith_optimizations/main}
\input{patterns/12_FPU/main}
\input{patterns/13_arrays/main}
\input{patterns/14_bitfields/main}
\EN{\input{patterns/145_LCG/main_EN}}
\RU{\input{patterns/145_LCG/main_RU}}
\input{patterns/15_structs/main}
\input{patterns/17_unions/main}
\input{patterns/18_pointers_to_functions/main}
\input{patterns/185_64bit_in_32_env/main}

\EN{\input{patterns/19_SIMD/main_EN}}
\RU{\input{patterns/19_SIMD/main_RU}}
\DE{\input{patterns/19_SIMD/main_DE}}

\EN{\input{patterns/20_x64/main_EN}}
\RU{\input{patterns/20_x64/main_RU}}

\EN{\input{patterns/205_floating_SIMD/main_EN}}
\RU{\input{patterns/205_floating_SIMD/main_RU}}
\DE{\input{patterns/205_floating_SIMD/main_DE}}

\EN{\input{patterns/ARM/main_EN}}
\RU{\input{patterns/ARM/main_RU}}
\DE{\input{patterns/ARM/main_DE}}

\input{patterns/MIPS/main}

\EN{\section{Returning Values}
\label{ret_val_func}

Another simple function is the one that simply returns a constant value:

\lstinputlisting[caption=\EN{\CCpp Code},style=customc]{patterns/011_ret/1.c}

Let's compile it.

\subsection{x86}

Here's what both the GCC and MSVC compilers produce (with optimization) on the x86 platform:

\lstinputlisting[caption=\Optimizing GCC/MSVC (\assemblyOutput),style=customasmx86]{patterns/011_ret/1.s}

\myindex{x86!\Instructions!RET}
There are just two instructions: the first places the value 123 into the \EAX register,
which is used by convention for storing the return
value, and the second one is \RET, which returns execution to the \gls{caller}.

The caller will take the result from the \EAX register.

\subsection{ARM}

There are a few differences on the ARM platform:

\lstinputlisting[caption=\OptimizingKeilVI (\ARMMode) ASM Output,style=customasmARM]{patterns/011_ret/1_Keil_ARM_O3.s}

ARM uses the register \Reg{0} for returning the results of functions, so 123 is copied into \Reg{0}.

\myindex{ARM!\Instructions!MOV}
\myindex{x86!\Instructions!MOV}
It is worth noting that \MOV is a misleading name for the instruction in both the x86 and ARM \ac{ISA}s.

The data is not in fact \IT{moved}, but \IT{copied}.

\subsection{MIPS}

\label{MIPS_leaf_function_ex1}

The GCC assembly output below lists registers by number:

\lstinputlisting[caption=\Optimizing GCC 4.4.5 (\assemblyOutput),style=customasmMIPS]{patterns/011_ret/MIPS.s}

\dots while \IDA does it by their pseudo names:

\lstinputlisting[caption=\Optimizing GCC 4.4.5 (IDA),style=customasmMIPS]{patterns/011_ret/MIPS_IDA.lst}

The \$2 (or \$V0) register is used to store the function's return value.
\myindex{MIPS!\Pseudoinstructions!LI}
\INS{LI} stands for ``Load Immediate'' and is the MIPS equivalent to \MOV.

\myindex{MIPS!\Instructions!J}
The other instruction is the jump instruction (J or JR) which returns the execution flow to the \gls{caller}.

\myindex{MIPS!Branch delay slot}
You might be wondering why the positions of the load instruction (LI) and the jump instruction (J or JR) are swapped. This is due to a \ac{RISC} feature called ``branch delay slot''.

The reason this happens is a quirk in the architecture of some RISC \ac{ISA}s and isn't important for our
purposes---we must simply keep in mind that in MIPS, the instruction following a jump or branch instruction
is executed \IT{before} the jump/branch instruction itself.

As a consequence, branch instructions always swap places with the instruction executed immediately beforehand.


In practice, functions which merely return 1 (\IT{true}) or 0 (\IT{false}) are very frequent.

The smallest ever of the standard UNIX utilities, \IT{/bin/true} and \IT{/bin/false} return 0 and 1 respectively, as an exit code.
(Zero as an exit code usually means success, non-zero means error.)
}
\RU{\subsubsection{std::string}
\myindex{\Cpp!STL!std::string}
\label{std_string}

\myparagraph{Как устроена структура}

Многие строковые библиотеки \InSqBrackets{\CNotes 2.2} обеспечивают структуру содержащую ссылку 
на буфер собственно со строкой, переменная всегда содержащую длину строки 
(что очень удобно для массы функций \InSqBrackets{\CNotes 2.2.1}) и переменную содержащую текущий размер буфера.

Строка в буфере обыкновенно оканчивается нулем: это для того чтобы указатель на буфер можно было
передавать в функции требующие на вход обычную сишную \ac{ASCIIZ}-строку.

Стандарт \Cpp не описывает, как именно нужно реализовывать std::string,
но, как правило, они реализованы как описано выше, с небольшими дополнениями.

Строки в \Cpp это не класс (как, например, QString в Qt), а темплейт (basic\_string), 
это сделано для того чтобы поддерживать 
строки содержащие разного типа символы: как минимум \Tchar и \IT{wchar\_t}.

Так что, std::string это класс с базовым типом \Tchar.

А std::wstring это класс с базовым типом \IT{wchar\_t}.

\mysubparagraph{MSVC}

В реализации MSVC, вместо ссылки на буфер может содержаться сам буфер (если строка короче 16-и символов).

Это означает, что каждая короткая строка будет занимать в памяти по крайней мере $16 + 4 + 4 = 24$ 
байт для 32-битной среды либо $16 + 8 + 8 = 32$ 
байта в 64-битной, а если строка длиннее 16-и символов, то прибавьте еще длину самой строки.

\lstinputlisting[caption=пример для MSVC,style=customc]{\CURPATH/STL/string/MSVC_RU.cpp}

Собственно, из этого исходника почти всё ясно.

Несколько замечаний:

Если строка короче 16-и символов, 
то отдельный буфер для строки в \glslink{heap}{куче} выделяться не будет.

Это удобно потому что на практике, основная часть строк действительно короткие.
Вероятно, разработчики в Microsoft выбрали размер в 16 символов как разумный баланс.

Теперь очень важный момент в конце функции main(): мы не пользуемся методом c\_str(), тем не менее,
если это скомпилировать и запустить, то обе строки появятся в консоли!

Работает это вот почему.

В первом случае строка короче 16-и символов и в начале объекта std::string (его можно рассматривать
просто как структуру) расположен буфер с этой строкой.
\printf трактует указатель как указатель на массив символов оканчивающийся нулем и поэтому всё работает.

Вывод второй строки (длиннее 16-и символов) даже еще опаснее: это вообще типичная программистская ошибка 
(или опечатка), забыть дописать c\_str().
Это работает потому что в это время в начале структуры расположен указатель на буфер.
Это может надолго остаться незамеченным: до тех пока там не появится строка 
короче 16-и символов, тогда процесс упадет.

\mysubparagraph{GCC}

В реализации GCC в структуре есть еще одна переменная --- reference count.

Интересно, что указатель на экземпляр класса std::string в GCC указывает не на начало самой структуры, 
а на указатель на буфера.
В libstdc++-v3\textbackslash{}include\textbackslash{}bits\textbackslash{}basic\_string.h 
мы можем прочитать что это сделано для удобства отладки:

\begin{lstlisting}
   *  The reason you want _M_data pointing to the character %array and
   *  not the _Rep is so that the debugger can see the string
   *  contents. (Probably we should add a non-inline member to get
   *  the _Rep for the debugger to use, so users can check the actual
   *  string length.)
\end{lstlisting}

\href{http://go.yurichev.com/17085}{исходный код basic\_string.h}

В нашем примере мы учитываем это:

\lstinputlisting[caption=пример для GCC,style=customc]{\CURPATH/STL/string/GCC_RU.cpp}

Нужны еще небольшие хаки чтобы сымитировать типичную ошибку, которую мы уже видели выше, из-за
более ужесточенной проверки типов в GCC, тем не менее, printf() работает и здесь без c\_str().

\myparagraph{Чуть более сложный пример}

\lstinputlisting[style=customc]{\CURPATH/STL/string/3.cpp}

\lstinputlisting[caption=MSVC 2012,style=customasmx86]{\CURPATH/STL/string/3_MSVC_RU.asm}

Собственно, компилятор не конструирует строки статически: да в общем-то и как
это возможно, если буфер с ней нужно хранить в \glslink{heap}{куче}?

Вместо этого в сегменте данных хранятся обычные \ac{ASCIIZ}-строки, а позже, во время выполнения, 
при помощи метода \q{assign}, конструируются строки s1 и s2
.
При помощи \TT{operator+}, создается строка s3.

Обратите внимание на то что вызов метода c\_str() отсутствует,
потому что его код достаточно короткий и компилятор вставил его прямо здесь:
если строка короче 16-и байт, то в регистре EAX остается указатель на буфер,
а если длиннее, то из этого же места достается адрес на буфер расположенный в \glslink{heap}{куче}.

Далее следуют вызовы трех деструкторов, причем, они вызываются только если строка длиннее 16-и байт:
тогда нужно освободить буфера в \glslink{heap}{куче}.
В противном случае, так как все три объекта std::string хранятся в стеке,
они освобождаются автоматически после выхода из функции.

Следовательно, работа с короткими строками более быстрая из-за м\'{е}ньшего обращения к \glslink{heap}{куче}.

Код на GCC даже проще (из-за того, что в GCC, как мы уже видели, не реализована возможность хранить короткую
строку прямо в структуре):

% TODO1 comment each function meaning
\lstinputlisting[caption=GCC 4.8.1,style=customasmx86]{\CURPATH/STL/string/3_GCC_RU.s}

Можно заметить, что в деструкторы передается не указатель на объект,
а указатель на место за 12 байт (или 3 слова) перед ним, то есть, на настоящее начало структуры.

\myparagraph{std::string как глобальная переменная}
\label{sec:std_string_as_global_variable}

Опытные программисты на \Cpp знают, что глобальные переменные \ac{STL}-типов вполне можно объявлять.

Да, действительно:

\lstinputlisting[style=customc]{\CURPATH/STL/string/5.cpp}

Но как и где будет вызываться конструктор \TT{std::string}?

На самом деле, эта переменная будет инициализирована даже перед началом \main.

\lstinputlisting[caption=MSVC 2012: здесь конструируется глобальная переменная{,} а также регистрируется её деструктор,style=customasmx86]{\CURPATH/STL/string/5_MSVC_p2.asm}

\lstinputlisting[caption=MSVC 2012: здесь глобальная переменная используется в \main,style=customasmx86]{\CURPATH/STL/string/5_MSVC_p1.asm}

\lstinputlisting[caption=MSVC 2012: эта функция-деструктор вызывается перед выходом,style=customasmx86]{\CURPATH/STL/string/5_MSVC_p3.asm}

\myindex{\CStandardLibrary!atexit()}
В реальности, из \ac{CRT}, еще до вызова main(), вызывается специальная функция,
в которой перечислены все конструкторы подобных переменных.
Более того: при помощи atexit() регистрируется функция, которая будет вызвана в конце работы программы:
в этой функции компилятор собирает вызовы деструкторов всех подобных глобальных переменных.

GCC работает похожим образом:

\lstinputlisting[caption=GCC 4.8.1,style=customasmx86]{\CURPATH/STL/string/5_GCC.s}

Но он не выделяет отдельной функции в которой будут собраны деструкторы: 
каждый деструктор передается в atexit() по одному.

% TODO а если глобальная STL-переменная в другом модуле? надо проверить.

}
\ifdefined\SPANISH
\chapter{Patrones de código}
\fi % SPANISH

\ifdefined\GERMAN
\chapter{Code-Muster}
\fi % GERMAN

\ifdefined\ENGLISH
\chapter{Code Patterns}
\fi % ENGLISH

\ifdefined\ITALIAN
\chapter{Forme di codice}
\fi % ITALIAN

\ifdefined\RUSSIAN
\chapter{Образцы кода}
\fi % RUSSIAN

\ifdefined\BRAZILIAN
\chapter{Padrões de códigos}
\fi % BRAZILIAN

\ifdefined\THAI
\chapter{รูปแบบของโค้ด}
\fi % THAI

\ifdefined\FRENCH
\chapter{Modèle de code}
\fi % FRENCH

\ifdefined\POLISH
\chapter{\PLph{}}
\fi % POLISH

% sections
\EN{\input{patterns/patterns_opt_dbg_EN}}
\ES{\input{patterns/patterns_opt_dbg_ES}}
\ITA{\input{patterns/patterns_opt_dbg_ITA}}
\PTBR{\input{patterns/patterns_opt_dbg_PTBR}}
\RU{\input{patterns/patterns_opt_dbg_RU}}
\THA{\input{patterns/patterns_opt_dbg_THA}}
\DE{\input{patterns/patterns_opt_dbg_DE}}
\FR{\input{patterns/patterns_opt_dbg_FR}}
\PL{\input{patterns/patterns_opt_dbg_PL}}

\RU{\section{Некоторые базовые понятия}}
\EN{\section{Some basics}}
\DE{\section{Einige Grundlagen}}
\FR{\section{Quelques bases}}
\ES{\section{\ESph{}}}
\ITA{\section{Alcune basi teoriche}}
\PTBR{\section{\PTBRph{}}}
\THA{\section{\THAph{}}}
\PL{\section{\PLph{}}}

% sections:
\EN{\input{patterns/intro_CPU_ISA_EN}}
\ES{\input{patterns/intro_CPU_ISA_ES}}
\ITA{\input{patterns/intro_CPU_ISA_ITA}}
\PTBR{\input{patterns/intro_CPU_ISA_PTBR}}
\RU{\input{patterns/intro_CPU_ISA_RU}}
\DE{\input{patterns/intro_CPU_ISA_DE}}
\FR{\input{patterns/intro_CPU_ISA_FR}}
\PL{\input{patterns/intro_CPU_ISA_PL}}

\EN{\input{patterns/numeral_EN}}
\RU{\input{patterns/numeral_RU}}
\ITA{\input{patterns/numeral_ITA}}
\DE{\input{patterns/numeral_DE}}
\FR{\input{patterns/numeral_FR}}
\PL{\input{patterns/numeral_PL}}

% chapters
\input{patterns/00_empty/main}
\input{patterns/011_ret/main}
\input{patterns/01_helloworld/main}
\input{patterns/015_prolog_epilogue/main}
\input{patterns/02_stack/main}
\input{patterns/03_printf/main}
\input{patterns/04_scanf/main}
\input{patterns/05_passing_arguments/main}
\input{patterns/06_return_results/main}
\input{patterns/061_pointers/main}
\input{patterns/065_GOTO/main}
\input{patterns/07_jcc/main}
\input{patterns/08_switch/main}
\input{patterns/09_loops/main}
\input{patterns/10_strings/main}
\input{patterns/11_arith_optimizations/main}
\input{patterns/12_FPU/main}
\input{patterns/13_arrays/main}
\input{patterns/14_bitfields/main}
\EN{\input{patterns/145_LCG/main_EN}}
\RU{\input{patterns/145_LCG/main_RU}}
\input{patterns/15_structs/main}
\input{patterns/17_unions/main}
\input{patterns/18_pointers_to_functions/main}
\input{patterns/185_64bit_in_32_env/main}

\EN{\input{patterns/19_SIMD/main_EN}}
\RU{\input{patterns/19_SIMD/main_RU}}
\DE{\input{patterns/19_SIMD/main_DE}}

\EN{\input{patterns/20_x64/main_EN}}
\RU{\input{patterns/20_x64/main_RU}}

\EN{\input{patterns/205_floating_SIMD/main_EN}}
\RU{\input{patterns/205_floating_SIMD/main_RU}}
\DE{\input{patterns/205_floating_SIMD/main_DE}}

\EN{\input{patterns/ARM/main_EN}}
\RU{\input{patterns/ARM/main_RU}}
\DE{\input{patterns/ARM/main_DE}}

\input{patterns/MIPS/main}

\ifdefined\SPANISH
\chapter{Patrones de código}
\fi % SPANISH

\ifdefined\GERMAN
\chapter{Code-Muster}
\fi % GERMAN

\ifdefined\ENGLISH
\chapter{Code Patterns}
\fi % ENGLISH

\ifdefined\ITALIAN
\chapter{Forme di codice}
\fi % ITALIAN

\ifdefined\RUSSIAN
\chapter{Образцы кода}
\fi % RUSSIAN

\ifdefined\BRAZILIAN
\chapter{Padrões de códigos}
\fi % BRAZILIAN

\ifdefined\THAI
\chapter{รูปแบบของโค้ด}
\fi % THAI

\ifdefined\FRENCH
\chapter{Modèle de code}
\fi % FRENCH

\ifdefined\POLISH
\chapter{\PLph{}}
\fi % POLISH

% sections
\EN{\input{patterns/patterns_opt_dbg_EN}}
\ES{\input{patterns/patterns_opt_dbg_ES}}
\ITA{\input{patterns/patterns_opt_dbg_ITA}}
\PTBR{\input{patterns/patterns_opt_dbg_PTBR}}
\RU{\input{patterns/patterns_opt_dbg_RU}}
\THA{\input{patterns/patterns_opt_dbg_THA}}
\DE{\input{patterns/patterns_opt_dbg_DE}}
\FR{\input{patterns/patterns_opt_dbg_FR}}
\PL{\input{patterns/patterns_opt_dbg_PL}}

\RU{\section{Некоторые базовые понятия}}
\EN{\section{Some basics}}
\DE{\section{Einige Grundlagen}}
\FR{\section{Quelques bases}}
\ES{\section{\ESph{}}}
\ITA{\section{Alcune basi teoriche}}
\PTBR{\section{\PTBRph{}}}
\THA{\section{\THAph{}}}
\PL{\section{\PLph{}}}

% sections:
\EN{\input{patterns/intro_CPU_ISA_EN}}
\ES{\input{patterns/intro_CPU_ISA_ES}}
\ITA{\input{patterns/intro_CPU_ISA_ITA}}
\PTBR{\input{patterns/intro_CPU_ISA_PTBR}}
\RU{\input{patterns/intro_CPU_ISA_RU}}
\DE{\input{patterns/intro_CPU_ISA_DE}}
\FR{\input{patterns/intro_CPU_ISA_FR}}
\PL{\input{patterns/intro_CPU_ISA_PL}}

\EN{\input{patterns/numeral_EN}}
\RU{\input{patterns/numeral_RU}}
\ITA{\input{patterns/numeral_ITA}}
\DE{\input{patterns/numeral_DE}}
\FR{\input{patterns/numeral_FR}}
\PL{\input{patterns/numeral_PL}}

% chapters
\input{patterns/00_empty/main}
\input{patterns/011_ret/main}
\input{patterns/01_helloworld/main}
\input{patterns/015_prolog_epilogue/main}
\input{patterns/02_stack/main}
\input{patterns/03_printf/main}
\input{patterns/04_scanf/main}
\input{patterns/05_passing_arguments/main}
\input{patterns/06_return_results/main}
\input{patterns/061_pointers/main}
\input{patterns/065_GOTO/main}
\input{patterns/07_jcc/main}
\input{patterns/08_switch/main}
\input{patterns/09_loops/main}
\input{patterns/10_strings/main}
\input{patterns/11_arith_optimizations/main}
\input{patterns/12_FPU/main}
\input{patterns/13_arrays/main}
\input{patterns/14_bitfields/main}
\EN{\input{patterns/145_LCG/main_EN}}
\RU{\input{patterns/145_LCG/main_RU}}
\input{patterns/15_structs/main}
\input{patterns/17_unions/main}
\input{patterns/18_pointers_to_functions/main}
\input{patterns/185_64bit_in_32_env/main}

\EN{\input{patterns/19_SIMD/main_EN}}
\RU{\input{patterns/19_SIMD/main_RU}}
\DE{\input{patterns/19_SIMD/main_DE}}

\EN{\input{patterns/20_x64/main_EN}}
\RU{\input{patterns/20_x64/main_RU}}

\EN{\input{patterns/205_floating_SIMD/main_EN}}
\RU{\input{patterns/205_floating_SIMD/main_RU}}
\DE{\input{patterns/205_floating_SIMD/main_DE}}

\EN{\input{patterns/ARM/main_EN}}
\RU{\input{patterns/ARM/main_RU}}
\DE{\input{patterns/ARM/main_DE}}

\input{patterns/MIPS/main}

\ifdefined\SPANISH
\chapter{Patrones de código}
\fi % SPANISH

\ifdefined\GERMAN
\chapter{Code-Muster}
\fi % GERMAN

\ifdefined\ENGLISH
\chapter{Code Patterns}
\fi % ENGLISH

\ifdefined\ITALIAN
\chapter{Forme di codice}
\fi % ITALIAN

\ifdefined\RUSSIAN
\chapter{Образцы кода}
\fi % RUSSIAN

\ifdefined\BRAZILIAN
\chapter{Padrões de códigos}
\fi % BRAZILIAN

\ifdefined\THAI
\chapter{รูปแบบของโค้ด}
\fi % THAI

\ifdefined\FRENCH
\chapter{Modèle de code}
\fi % FRENCH

\ifdefined\POLISH
\chapter{\PLph{}}
\fi % POLISH

% sections
\EN{\input{patterns/patterns_opt_dbg_EN}}
\ES{\input{patterns/patterns_opt_dbg_ES}}
\ITA{\input{patterns/patterns_opt_dbg_ITA}}
\PTBR{\input{patterns/patterns_opt_dbg_PTBR}}
\RU{\input{patterns/patterns_opt_dbg_RU}}
\THA{\input{patterns/patterns_opt_dbg_THA}}
\DE{\input{patterns/patterns_opt_dbg_DE}}
\FR{\input{patterns/patterns_opt_dbg_FR}}
\PL{\input{patterns/patterns_opt_dbg_PL}}

\RU{\section{Некоторые базовые понятия}}
\EN{\section{Some basics}}
\DE{\section{Einige Grundlagen}}
\FR{\section{Quelques bases}}
\ES{\section{\ESph{}}}
\ITA{\section{Alcune basi teoriche}}
\PTBR{\section{\PTBRph{}}}
\THA{\section{\THAph{}}}
\PL{\section{\PLph{}}}

% sections:
\EN{\input{patterns/intro_CPU_ISA_EN}}
\ES{\input{patterns/intro_CPU_ISA_ES}}
\ITA{\input{patterns/intro_CPU_ISA_ITA}}
\PTBR{\input{patterns/intro_CPU_ISA_PTBR}}
\RU{\input{patterns/intro_CPU_ISA_RU}}
\DE{\input{patterns/intro_CPU_ISA_DE}}
\FR{\input{patterns/intro_CPU_ISA_FR}}
\PL{\input{patterns/intro_CPU_ISA_PL}}

\EN{\input{patterns/numeral_EN}}
\RU{\input{patterns/numeral_RU}}
\ITA{\input{patterns/numeral_ITA}}
\DE{\input{patterns/numeral_DE}}
\FR{\input{patterns/numeral_FR}}
\PL{\input{patterns/numeral_PL}}

% chapters
\input{patterns/00_empty/main}
\input{patterns/011_ret/main}
\input{patterns/01_helloworld/main}
\input{patterns/015_prolog_epilogue/main}
\input{patterns/02_stack/main}
\input{patterns/03_printf/main}
\input{patterns/04_scanf/main}
\input{patterns/05_passing_arguments/main}
\input{patterns/06_return_results/main}
\input{patterns/061_pointers/main}
\input{patterns/065_GOTO/main}
\input{patterns/07_jcc/main}
\input{patterns/08_switch/main}
\input{patterns/09_loops/main}
\input{patterns/10_strings/main}
\input{patterns/11_arith_optimizations/main}
\input{patterns/12_FPU/main}
\input{patterns/13_arrays/main}
\input{patterns/14_bitfields/main}
\EN{\input{patterns/145_LCG/main_EN}}
\RU{\input{patterns/145_LCG/main_RU}}
\input{patterns/15_structs/main}
\input{patterns/17_unions/main}
\input{patterns/18_pointers_to_functions/main}
\input{patterns/185_64bit_in_32_env/main}

\EN{\input{patterns/19_SIMD/main_EN}}
\RU{\input{patterns/19_SIMD/main_RU}}
\DE{\input{patterns/19_SIMD/main_DE}}

\EN{\input{patterns/20_x64/main_EN}}
\RU{\input{patterns/20_x64/main_RU}}

\EN{\input{patterns/205_floating_SIMD/main_EN}}
\RU{\input{patterns/205_floating_SIMD/main_RU}}
\DE{\input{patterns/205_floating_SIMD/main_DE}}

\EN{\input{patterns/ARM/main_EN}}
\RU{\input{patterns/ARM/main_RU}}
\DE{\input{patterns/ARM/main_DE}}

\input{patterns/MIPS/main}

\ifdefined\SPANISH
\chapter{Patrones de código}
\fi % SPANISH

\ifdefined\GERMAN
\chapter{Code-Muster}
\fi % GERMAN

\ifdefined\ENGLISH
\chapter{Code Patterns}
\fi % ENGLISH

\ifdefined\ITALIAN
\chapter{Forme di codice}
\fi % ITALIAN

\ifdefined\RUSSIAN
\chapter{Образцы кода}
\fi % RUSSIAN

\ifdefined\BRAZILIAN
\chapter{Padrões de códigos}
\fi % BRAZILIAN

\ifdefined\THAI
\chapter{รูปแบบของโค้ด}
\fi % THAI

\ifdefined\FRENCH
\chapter{Modèle de code}
\fi % FRENCH

\ifdefined\POLISH
\chapter{\PLph{}}
\fi % POLISH

% sections
\EN{\input{patterns/patterns_opt_dbg_EN}}
\ES{\input{patterns/patterns_opt_dbg_ES}}
\ITA{\input{patterns/patterns_opt_dbg_ITA}}
\PTBR{\input{patterns/patterns_opt_dbg_PTBR}}
\RU{\input{patterns/patterns_opt_dbg_RU}}
\THA{\input{patterns/patterns_opt_dbg_THA}}
\DE{\input{patterns/patterns_opt_dbg_DE}}
\FR{\input{patterns/patterns_opt_dbg_FR}}
\PL{\input{patterns/patterns_opt_dbg_PL}}

\RU{\section{Некоторые базовые понятия}}
\EN{\section{Some basics}}
\DE{\section{Einige Grundlagen}}
\FR{\section{Quelques bases}}
\ES{\section{\ESph{}}}
\ITA{\section{Alcune basi teoriche}}
\PTBR{\section{\PTBRph{}}}
\THA{\section{\THAph{}}}
\PL{\section{\PLph{}}}

% sections:
\EN{\input{patterns/intro_CPU_ISA_EN}}
\ES{\input{patterns/intro_CPU_ISA_ES}}
\ITA{\input{patterns/intro_CPU_ISA_ITA}}
\PTBR{\input{patterns/intro_CPU_ISA_PTBR}}
\RU{\input{patterns/intro_CPU_ISA_RU}}
\DE{\input{patterns/intro_CPU_ISA_DE}}
\FR{\input{patterns/intro_CPU_ISA_FR}}
\PL{\input{patterns/intro_CPU_ISA_PL}}

\EN{\input{patterns/numeral_EN}}
\RU{\input{patterns/numeral_RU}}
\ITA{\input{patterns/numeral_ITA}}
\DE{\input{patterns/numeral_DE}}
\FR{\input{patterns/numeral_FR}}
\PL{\input{patterns/numeral_PL}}

% chapters
\input{patterns/00_empty/main}
\input{patterns/011_ret/main}
\input{patterns/01_helloworld/main}
\input{patterns/015_prolog_epilogue/main}
\input{patterns/02_stack/main}
\input{patterns/03_printf/main}
\input{patterns/04_scanf/main}
\input{patterns/05_passing_arguments/main}
\input{patterns/06_return_results/main}
\input{patterns/061_pointers/main}
\input{patterns/065_GOTO/main}
\input{patterns/07_jcc/main}
\input{patterns/08_switch/main}
\input{patterns/09_loops/main}
\input{patterns/10_strings/main}
\input{patterns/11_arith_optimizations/main}
\input{patterns/12_FPU/main}
\input{patterns/13_arrays/main}
\input{patterns/14_bitfields/main}
\EN{\input{patterns/145_LCG/main_EN}}
\RU{\input{patterns/145_LCG/main_RU}}
\input{patterns/15_structs/main}
\input{patterns/17_unions/main}
\input{patterns/18_pointers_to_functions/main}
\input{patterns/185_64bit_in_32_env/main}

\EN{\input{patterns/19_SIMD/main_EN}}
\RU{\input{patterns/19_SIMD/main_RU}}
\DE{\input{patterns/19_SIMD/main_DE}}

\EN{\input{patterns/20_x64/main_EN}}
\RU{\input{patterns/20_x64/main_RU}}

\EN{\input{patterns/205_floating_SIMD/main_EN}}
\RU{\input{patterns/205_floating_SIMD/main_RU}}
\DE{\input{patterns/205_floating_SIMD/main_DE}}

\EN{\input{patterns/ARM/main_EN}}
\RU{\input{patterns/ARM/main_RU}}
\DE{\input{patterns/ARM/main_DE}}

\input{patterns/MIPS/main}


\EN{\section{Returning Values}
\label{ret_val_func}

Another simple function is the one that simply returns a constant value:

\lstinputlisting[caption=\EN{\CCpp Code},style=customc]{patterns/011_ret/1.c}

Let's compile it.

\subsection{x86}

Here's what both the GCC and MSVC compilers produce (with optimization) on the x86 platform:

\lstinputlisting[caption=\Optimizing GCC/MSVC (\assemblyOutput),style=customasmx86]{patterns/011_ret/1.s}

\myindex{x86!\Instructions!RET}
There are just two instructions: the first places the value 123 into the \EAX register,
which is used by convention for storing the return
value, and the second one is \RET, which returns execution to the \gls{caller}.

The caller will take the result from the \EAX register.

\subsection{ARM}

There are a few differences on the ARM platform:

\lstinputlisting[caption=\OptimizingKeilVI (\ARMMode) ASM Output,style=customasmARM]{patterns/011_ret/1_Keil_ARM_O3.s}

ARM uses the register \Reg{0} for returning the results of functions, so 123 is copied into \Reg{0}.

\myindex{ARM!\Instructions!MOV}
\myindex{x86!\Instructions!MOV}
It is worth noting that \MOV is a misleading name for the instruction in both the x86 and ARM \ac{ISA}s.

The data is not in fact \IT{moved}, but \IT{copied}.

\subsection{MIPS}

\label{MIPS_leaf_function_ex1}

The GCC assembly output below lists registers by number:

\lstinputlisting[caption=\Optimizing GCC 4.4.5 (\assemblyOutput),style=customasmMIPS]{patterns/011_ret/MIPS.s}

\dots while \IDA does it by their pseudo names:

\lstinputlisting[caption=\Optimizing GCC 4.4.5 (IDA),style=customasmMIPS]{patterns/011_ret/MIPS_IDA.lst}

The \$2 (or \$V0) register is used to store the function's return value.
\myindex{MIPS!\Pseudoinstructions!LI}
\INS{LI} stands for ``Load Immediate'' and is the MIPS equivalent to \MOV.

\myindex{MIPS!\Instructions!J}
The other instruction is the jump instruction (J or JR) which returns the execution flow to the \gls{caller}.

\myindex{MIPS!Branch delay slot}
You might be wondering why the positions of the load instruction (LI) and the jump instruction (J or JR) are swapped. This is due to a \ac{RISC} feature called ``branch delay slot''.

The reason this happens is a quirk in the architecture of some RISC \ac{ISA}s and isn't important for our
purposes---we must simply keep in mind that in MIPS, the instruction following a jump or branch instruction
is executed \IT{before} the jump/branch instruction itself.

As a consequence, branch instructions always swap places with the instruction executed immediately beforehand.


In practice, functions which merely return 1 (\IT{true}) or 0 (\IT{false}) are very frequent.

The smallest ever of the standard UNIX utilities, \IT{/bin/true} and \IT{/bin/false} return 0 and 1 respectively, as an exit code.
(Zero as an exit code usually means success, non-zero means error.)
}
\RU{\subsubsection{std::string}
\myindex{\Cpp!STL!std::string}
\label{std_string}

\myparagraph{Как устроена структура}

Многие строковые библиотеки \InSqBrackets{\CNotes 2.2} обеспечивают структуру содержащую ссылку 
на буфер собственно со строкой, переменная всегда содержащую длину строки 
(что очень удобно для массы функций \InSqBrackets{\CNotes 2.2.1}) и переменную содержащую текущий размер буфера.

Строка в буфере обыкновенно оканчивается нулем: это для того чтобы указатель на буфер можно было
передавать в функции требующие на вход обычную сишную \ac{ASCIIZ}-строку.

Стандарт \Cpp не описывает, как именно нужно реализовывать std::string,
но, как правило, они реализованы как описано выше, с небольшими дополнениями.

Строки в \Cpp это не класс (как, например, QString в Qt), а темплейт (basic\_string), 
это сделано для того чтобы поддерживать 
строки содержащие разного типа символы: как минимум \Tchar и \IT{wchar\_t}.

Так что, std::string это класс с базовым типом \Tchar.

А std::wstring это класс с базовым типом \IT{wchar\_t}.

\mysubparagraph{MSVC}

В реализации MSVC, вместо ссылки на буфер может содержаться сам буфер (если строка короче 16-и символов).

Это означает, что каждая короткая строка будет занимать в памяти по крайней мере $16 + 4 + 4 = 24$ 
байт для 32-битной среды либо $16 + 8 + 8 = 32$ 
байта в 64-битной, а если строка длиннее 16-и символов, то прибавьте еще длину самой строки.

\lstinputlisting[caption=пример для MSVC,style=customc]{\CURPATH/STL/string/MSVC_RU.cpp}

Собственно, из этого исходника почти всё ясно.

Несколько замечаний:

Если строка короче 16-и символов, 
то отдельный буфер для строки в \glslink{heap}{куче} выделяться не будет.

Это удобно потому что на практике, основная часть строк действительно короткие.
Вероятно, разработчики в Microsoft выбрали размер в 16 символов как разумный баланс.

Теперь очень важный момент в конце функции main(): мы не пользуемся методом c\_str(), тем не менее,
если это скомпилировать и запустить, то обе строки появятся в консоли!

Работает это вот почему.

В первом случае строка короче 16-и символов и в начале объекта std::string (его можно рассматривать
просто как структуру) расположен буфер с этой строкой.
\printf трактует указатель как указатель на массив символов оканчивающийся нулем и поэтому всё работает.

Вывод второй строки (длиннее 16-и символов) даже еще опаснее: это вообще типичная программистская ошибка 
(или опечатка), забыть дописать c\_str().
Это работает потому что в это время в начале структуры расположен указатель на буфер.
Это может надолго остаться незамеченным: до тех пока там не появится строка 
короче 16-и символов, тогда процесс упадет.

\mysubparagraph{GCC}

В реализации GCC в структуре есть еще одна переменная --- reference count.

Интересно, что указатель на экземпляр класса std::string в GCC указывает не на начало самой структуры, 
а на указатель на буфера.
В libstdc++-v3\textbackslash{}include\textbackslash{}bits\textbackslash{}basic\_string.h 
мы можем прочитать что это сделано для удобства отладки:

\begin{lstlisting}
   *  The reason you want _M_data pointing to the character %array and
   *  not the _Rep is so that the debugger can see the string
   *  contents. (Probably we should add a non-inline member to get
   *  the _Rep for the debugger to use, so users can check the actual
   *  string length.)
\end{lstlisting}

\href{http://go.yurichev.com/17085}{исходный код basic\_string.h}

В нашем примере мы учитываем это:

\lstinputlisting[caption=пример для GCC,style=customc]{\CURPATH/STL/string/GCC_RU.cpp}

Нужны еще небольшие хаки чтобы сымитировать типичную ошибку, которую мы уже видели выше, из-за
более ужесточенной проверки типов в GCC, тем не менее, printf() работает и здесь без c\_str().

\myparagraph{Чуть более сложный пример}

\lstinputlisting[style=customc]{\CURPATH/STL/string/3.cpp}

\lstinputlisting[caption=MSVC 2012,style=customasmx86]{\CURPATH/STL/string/3_MSVC_RU.asm}

Собственно, компилятор не конструирует строки статически: да в общем-то и как
это возможно, если буфер с ней нужно хранить в \glslink{heap}{куче}?

Вместо этого в сегменте данных хранятся обычные \ac{ASCIIZ}-строки, а позже, во время выполнения, 
при помощи метода \q{assign}, конструируются строки s1 и s2
.
При помощи \TT{operator+}, создается строка s3.

Обратите внимание на то что вызов метода c\_str() отсутствует,
потому что его код достаточно короткий и компилятор вставил его прямо здесь:
если строка короче 16-и байт, то в регистре EAX остается указатель на буфер,
а если длиннее, то из этого же места достается адрес на буфер расположенный в \glslink{heap}{куче}.

Далее следуют вызовы трех деструкторов, причем, они вызываются только если строка длиннее 16-и байт:
тогда нужно освободить буфера в \glslink{heap}{куче}.
В противном случае, так как все три объекта std::string хранятся в стеке,
они освобождаются автоматически после выхода из функции.

Следовательно, работа с короткими строками более быстрая из-за м\'{е}ньшего обращения к \glslink{heap}{куче}.

Код на GCC даже проще (из-за того, что в GCC, как мы уже видели, не реализована возможность хранить короткую
строку прямо в структуре):

% TODO1 comment each function meaning
\lstinputlisting[caption=GCC 4.8.1,style=customasmx86]{\CURPATH/STL/string/3_GCC_RU.s}

Можно заметить, что в деструкторы передается не указатель на объект,
а указатель на место за 12 байт (или 3 слова) перед ним, то есть, на настоящее начало структуры.

\myparagraph{std::string как глобальная переменная}
\label{sec:std_string_as_global_variable}

Опытные программисты на \Cpp знают, что глобальные переменные \ac{STL}-типов вполне можно объявлять.

Да, действительно:

\lstinputlisting[style=customc]{\CURPATH/STL/string/5.cpp}

Но как и где будет вызываться конструктор \TT{std::string}?

На самом деле, эта переменная будет инициализирована даже перед началом \main.

\lstinputlisting[caption=MSVC 2012: здесь конструируется глобальная переменная{,} а также регистрируется её деструктор,style=customasmx86]{\CURPATH/STL/string/5_MSVC_p2.asm}

\lstinputlisting[caption=MSVC 2012: здесь глобальная переменная используется в \main,style=customasmx86]{\CURPATH/STL/string/5_MSVC_p1.asm}

\lstinputlisting[caption=MSVC 2012: эта функция-деструктор вызывается перед выходом,style=customasmx86]{\CURPATH/STL/string/5_MSVC_p3.asm}

\myindex{\CStandardLibrary!atexit()}
В реальности, из \ac{CRT}, еще до вызова main(), вызывается специальная функция,
в которой перечислены все конструкторы подобных переменных.
Более того: при помощи atexit() регистрируется функция, которая будет вызвана в конце работы программы:
в этой функции компилятор собирает вызовы деструкторов всех подобных глобальных переменных.

GCC работает похожим образом:

\lstinputlisting[caption=GCC 4.8.1,style=customasmx86]{\CURPATH/STL/string/5_GCC.s}

Но он не выделяет отдельной функции в которой будут собраны деструкторы: 
каждый деструктор передается в atexit() по одному.

% TODO а если глобальная STL-переменная в другом модуле? надо проверить.

}
\DE{\subsection{Einfachste XOR-Verschlüsselung überhaupt}

Ich habe einmal eine Software gesehen, bei der alle Debugging-Ausgaben mit XOR mit dem Wert 3
verschlüsselt wurden. Mit anderen Worten, die beiden niedrigsten Bits aller Buchstaben wurden invertiert.

``Hello, world'' wurde zu ``Kfool/\#tlqog'':

\begin{lstlisting}
#!/usr/bin/python

msg="Hello, world!"

print "".join(map(lambda x: chr(ord(x)^3), msg))
\end{lstlisting}

Das ist eine ziemlich interessante Verschlüsselung (oder besser eine Verschleierung),
weil sie zwei wichtige Eigenschaften hat:
1) es ist eine einzige Funktion zum Verschlüsseln und entschlüsseln, sie muss nur wiederholt angewendet werden
2) die entstehenden Buchstaben befinden sich im druckbaren Bereich, also die ganze Zeichenkette kann ohne
Escape-Symbole im Code verwendet werden.

Die zweite Eigenschaft nutzt die Tatsache, dass alle druckbaren Zeichen in Reihen organisiert sind: 0x2x-0x7x,
und wenn die beiden niederwertigsten Bits invertiert werden, wird der Buchstabe um eine oder drei Stellen nach
links oder rechts \IT{verschoben}, aber niemals in eine andere Reihe:

\begin{figure}[H]
\centering
\includegraphics[width=0.7\textwidth]{ascii_clean.png}
\caption{7-Bit \ac{ASCII} Tabelle in Emacs}
\end{figure}

\dots mit dem Zeichen 0x7F als einziger Ausnahme.

Im Folgenden werden also beispielsweise die Zeichen A-Z \IT{verschlüsselt}:

\begin{lstlisting}
#!/usr/bin/python

msg="@ABCDEFGHIJKLMNO"

print "".join(map(lambda x: chr(ord(x)^3), msg))
\end{lstlisting}

Ergebnis:
% FIXME \verb  --  relevant comment for German?
\begin{lstlisting}
CBA@GFEDKJIHONML
\end{lstlisting}

Es sieht so aus als würden die Zeichen ``@'' und ``C'' sowie ``B'' und ``A'' vertauscht werden.

Hier ist noch ein interessantes Beispiel, in dem gezeigt wird, wie die Eigenschaften von XOR
ausgenutzt werden können: Exakt den gleichen Effekt, dass druckbare Zeichen auch druckbar bleiben,
kann man dadurch erzielen, dass irgendeine Kombination der niedrigsten vier Bits invertiert wird.
}

\EN{\section{Returning Values}
\label{ret_val_func}

Another simple function is the one that simply returns a constant value:

\lstinputlisting[caption=\EN{\CCpp Code},style=customc]{patterns/011_ret/1.c}

Let's compile it.

\subsection{x86}

Here's what both the GCC and MSVC compilers produce (with optimization) on the x86 platform:

\lstinputlisting[caption=\Optimizing GCC/MSVC (\assemblyOutput),style=customasmx86]{patterns/011_ret/1.s}

\myindex{x86!\Instructions!RET}
There are just two instructions: the first places the value 123 into the \EAX register,
which is used by convention for storing the return
value, and the second one is \RET, which returns execution to the \gls{caller}.

The caller will take the result from the \EAX register.

\subsection{ARM}

There are a few differences on the ARM platform:

\lstinputlisting[caption=\OptimizingKeilVI (\ARMMode) ASM Output,style=customasmARM]{patterns/011_ret/1_Keil_ARM_O3.s}

ARM uses the register \Reg{0} for returning the results of functions, so 123 is copied into \Reg{0}.

\myindex{ARM!\Instructions!MOV}
\myindex{x86!\Instructions!MOV}
It is worth noting that \MOV is a misleading name for the instruction in both the x86 and ARM \ac{ISA}s.

The data is not in fact \IT{moved}, but \IT{copied}.

\subsection{MIPS}

\label{MIPS_leaf_function_ex1}

The GCC assembly output below lists registers by number:

\lstinputlisting[caption=\Optimizing GCC 4.4.5 (\assemblyOutput),style=customasmMIPS]{patterns/011_ret/MIPS.s}

\dots while \IDA does it by their pseudo names:

\lstinputlisting[caption=\Optimizing GCC 4.4.5 (IDA),style=customasmMIPS]{patterns/011_ret/MIPS_IDA.lst}

The \$2 (or \$V0) register is used to store the function's return value.
\myindex{MIPS!\Pseudoinstructions!LI}
\INS{LI} stands for ``Load Immediate'' and is the MIPS equivalent to \MOV.

\myindex{MIPS!\Instructions!J}
The other instruction is the jump instruction (J or JR) which returns the execution flow to the \gls{caller}.

\myindex{MIPS!Branch delay slot}
You might be wondering why the positions of the load instruction (LI) and the jump instruction (J or JR) are swapped. This is due to a \ac{RISC} feature called ``branch delay slot''.

The reason this happens is a quirk in the architecture of some RISC \ac{ISA}s and isn't important for our
purposes---we must simply keep in mind that in MIPS, the instruction following a jump or branch instruction
is executed \IT{before} the jump/branch instruction itself.

As a consequence, branch instructions always swap places with the instruction executed immediately beforehand.


In practice, functions which merely return 1 (\IT{true}) or 0 (\IT{false}) are very frequent.

The smallest ever of the standard UNIX utilities, \IT{/bin/true} and \IT{/bin/false} return 0 and 1 respectively, as an exit code.
(Zero as an exit code usually means success, non-zero means error.)
}
\RU{\subsubsection{std::string}
\myindex{\Cpp!STL!std::string}
\label{std_string}

\myparagraph{Как устроена структура}

Многие строковые библиотеки \InSqBrackets{\CNotes 2.2} обеспечивают структуру содержащую ссылку 
на буфер собственно со строкой, переменная всегда содержащую длину строки 
(что очень удобно для массы функций \InSqBrackets{\CNotes 2.2.1}) и переменную содержащую текущий размер буфера.

Строка в буфере обыкновенно оканчивается нулем: это для того чтобы указатель на буфер можно было
передавать в функции требующие на вход обычную сишную \ac{ASCIIZ}-строку.

Стандарт \Cpp не описывает, как именно нужно реализовывать std::string,
но, как правило, они реализованы как описано выше, с небольшими дополнениями.

Строки в \Cpp это не класс (как, например, QString в Qt), а темплейт (basic\_string), 
это сделано для того чтобы поддерживать 
строки содержащие разного типа символы: как минимум \Tchar и \IT{wchar\_t}.

Так что, std::string это класс с базовым типом \Tchar.

А std::wstring это класс с базовым типом \IT{wchar\_t}.

\mysubparagraph{MSVC}

В реализации MSVC, вместо ссылки на буфер может содержаться сам буфер (если строка короче 16-и символов).

Это означает, что каждая короткая строка будет занимать в памяти по крайней мере $16 + 4 + 4 = 24$ 
байт для 32-битной среды либо $16 + 8 + 8 = 32$ 
байта в 64-битной, а если строка длиннее 16-и символов, то прибавьте еще длину самой строки.

\lstinputlisting[caption=пример для MSVC,style=customc]{\CURPATH/STL/string/MSVC_RU.cpp}

Собственно, из этого исходника почти всё ясно.

Несколько замечаний:

Если строка короче 16-и символов, 
то отдельный буфер для строки в \glslink{heap}{куче} выделяться не будет.

Это удобно потому что на практике, основная часть строк действительно короткие.
Вероятно, разработчики в Microsoft выбрали размер в 16 символов как разумный баланс.

Теперь очень важный момент в конце функции main(): мы не пользуемся методом c\_str(), тем не менее,
если это скомпилировать и запустить, то обе строки появятся в консоли!

Работает это вот почему.

В первом случае строка короче 16-и символов и в начале объекта std::string (его можно рассматривать
просто как структуру) расположен буфер с этой строкой.
\printf трактует указатель как указатель на массив символов оканчивающийся нулем и поэтому всё работает.

Вывод второй строки (длиннее 16-и символов) даже еще опаснее: это вообще типичная программистская ошибка 
(или опечатка), забыть дописать c\_str().
Это работает потому что в это время в начале структуры расположен указатель на буфер.
Это может надолго остаться незамеченным: до тех пока там не появится строка 
короче 16-и символов, тогда процесс упадет.

\mysubparagraph{GCC}

В реализации GCC в структуре есть еще одна переменная --- reference count.

Интересно, что указатель на экземпляр класса std::string в GCC указывает не на начало самой структуры, 
а на указатель на буфера.
В libstdc++-v3\textbackslash{}include\textbackslash{}bits\textbackslash{}basic\_string.h 
мы можем прочитать что это сделано для удобства отладки:

\begin{lstlisting}
   *  The reason you want _M_data pointing to the character %array and
   *  not the _Rep is so that the debugger can see the string
   *  contents. (Probably we should add a non-inline member to get
   *  the _Rep for the debugger to use, so users can check the actual
   *  string length.)
\end{lstlisting}

\href{http://go.yurichev.com/17085}{исходный код basic\_string.h}

В нашем примере мы учитываем это:

\lstinputlisting[caption=пример для GCC,style=customc]{\CURPATH/STL/string/GCC_RU.cpp}

Нужны еще небольшие хаки чтобы сымитировать типичную ошибку, которую мы уже видели выше, из-за
более ужесточенной проверки типов в GCC, тем не менее, printf() работает и здесь без c\_str().

\myparagraph{Чуть более сложный пример}

\lstinputlisting[style=customc]{\CURPATH/STL/string/3.cpp}

\lstinputlisting[caption=MSVC 2012,style=customasmx86]{\CURPATH/STL/string/3_MSVC_RU.asm}

Собственно, компилятор не конструирует строки статически: да в общем-то и как
это возможно, если буфер с ней нужно хранить в \glslink{heap}{куче}?

Вместо этого в сегменте данных хранятся обычные \ac{ASCIIZ}-строки, а позже, во время выполнения, 
при помощи метода \q{assign}, конструируются строки s1 и s2
.
При помощи \TT{operator+}, создается строка s3.

Обратите внимание на то что вызов метода c\_str() отсутствует,
потому что его код достаточно короткий и компилятор вставил его прямо здесь:
если строка короче 16-и байт, то в регистре EAX остается указатель на буфер,
а если длиннее, то из этого же места достается адрес на буфер расположенный в \glslink{heap}{куче}.

Далее следуют вызовы трех деструкторов, причем, они вызываются только если строка длиннее 16-и байт:
тогда нужно освободить буфера в \glslink{heap}{куче}.
В противном случае, так как все три объекта std::string хранятся в стеке,
они освобождаются автоматически после выхода из функции.

Следовательно, работа с короткими строками более быстрая из-за м\'{е}ньшего обращения к \glslink{heap}{куче}.

Код на GCC даже проще (из-за того, что в GCC, как мы уже видели, не реализована возможность хранить короткую
строку прямо в структуре):

% TODO1 comment each function meaning
\lstinputlisting[caption=GCC 4.8.1,style=customasmx86]{\CURPATH/STL/string/3_GCC_RU.s}

Можно заметить, что в деструкторы передается не указатель на объект,
а указатель на место за 12 байт (или 3 слова) перед ним, то есть, на настоящее начало структуры.

\myparagraph{std::string как глобальная переменная}
\label{sec:std_string_as_global_variable}

Опытные программисты на \Cpp знают, что глобальные переменные \ac{STL}-типов вполне можно объявлять.

Да, действительно:

\lstinputlisting[style=customc]{\CURPATH/STL/string/5.cpp}

Но как и где будет вызываться конструктор \TT{std::string}?

На самом деле, эта переменная будет инициализирована даже перед началом \main.

\lstinputlisting[caption=MSVC 2012: здесь конструируется глобальная переменная{,} а также регистрируется её деструктор,style=customasmx86]{\CURPATH/STL/string/5_MSVC_p2.asm}

\lstinputlisting[caption=MSVC 2012: здесь глобальная переменная используется в \main,style=customasmx86]{\CURPATH/STL/string/5_MSVC_p1.asm}

\lstinputlisting[caption=MSVC 2012: эта функция-деструктор вызывается перед выходом,style=customasmx86]{\CURPATH/STL/string/5_MSVC_p3.asm}

\myindex{\CStandardLibrary!atexit()}
В реальности, из \ac{CRT}, еще до вызова main(), вызывается специальная функция,
в которой перечислены все конструкторы подобных переменных.
Более того: при помощи atexit() регистрируется функция, которая будет вызвана в конце работы программы:
в этой функции компилятор собирает вызовы деструкторов всех подобных глобальных переменных.

GCC работает похожим образом:

\lstinputlisting[caption=GCC 4.8.1,style=customasmx86]{\CURPATH/STL/string/5_GCC.s}

Но он не выделяет отдельной функции в которой будут собраны деструкторы: 
каждый деструктор передается в atexit() по одному.

% TODO а если глобальная STL-переменная в другом модуле? надо проверить.

}

\EN{\section{Returning Values}
\label{ret_val_func}

Another simple function is the one that simply returns a constant value:

\lstinputlisting[caption=\EN{\CCpp Code},style=customc]{patterns/011_ret/1.c}

Let's compile it.

\subsection{x86}

Here's what both the GCC and MSVC compilers produce (with optimization) on the x86 platform:

\lstinputlisting[caption=\Optimizing GCC/MSVC (\assemblyOutput),style=customasmx86]{patterns/011_ret/1.s}

\myindex{x86!\Instructions!RET}
There are just two instructions: the first places the value 123 into the \EAX register,
which is used by convention for storing the return
value, and the second one is \RET, which returns execution to the \gls{caller}.

The caller will take the result from the \EAX register.

\subsection{ARM}

There are a few differences on the ARM platform:

\lstinputlisting[caption=\OptimizingKeilVI (\ARMMode) ASM Output,style=customasmARM]{patterns/011_ret/1_Keil_ARM_O3.s}

ARM uses the register \Reg{0} for returning the results of functions, so 123 is copied into \Reg{0}.

\myindex{ARM!\Instructions!MOV}
\myindex{x86!\Instructions!MOV}
It is worth noting that \MOV is a misleading name for the instruction in both the x86 and ARM \ac{ISA}s.

The data is not in fact \IT{moved}, but \IT{copied}.

\subsection{MIPS}

\label{MIPS_leaf_function_ex1}

The GCC assembly output below lists registers by number:

\lstinputlisting[caption=\Optimizing GCC 4.4.5 (\assemblyOutput),style=customasmMIPS]{patterns/011_ret/MIPS.s}

\dots while \IDA does it by their pseudo names:

\lstinputlisting[caption=\Optimizing GCC 4.4.5 (IDA),style=customasmMIPS]{patterns/011_ret/MIPS_IDA.lst}

The \$2 (or \$V0) register is used to store the function's return value.
\myindex{MIPS!\Pseudoinstructions!LI}
\INS{LI} stands for ``Load Immediate'' and is the MIPS equivalent to \MOV.

\myindex{MIPS!\Instructions!J}
The other instruction is the jump instruction (J or JR) which returns the execution flow to the \gls{caller}.

\myindex{MIPS!Branch delay slot}
You might be wondering why the positions of the load instruction (LI) and the jump instruction (J or JR) are swapped. This is due to a \ac{RISC} feature called ``branch delay slot''.

The reason this happens is a quirk in the architecture of some RISC \ac{ISA}s and isn't important for our
purposes---we must simply keep in mind that in MIPS, the instruction following a jump or branch instruction
is executed \IT{before} the jump/branch instruction itself.

As a consequence, branch instructions always swap places with the instruction executed immediately beforehand.


In practice, functions which merely return 1 (\IT{true}) or 0 (\IT{false}) are very frequent.

The smallest ever of the standard UNIX utilities, \IT{/bin/true} and \IT{/bin/false} return 0 and 1 respectively, as an exit code.
(Zero as an exit code usually means success, non-zero means error.)
}
\RU{\subsubsection{std::string}
\myindex{\Cpp!STL!std::string}
\label{std_string}

\myparagraph{Как устроена структура}

Многие строковые библиотеки \InSqBrackets{\CNotes 2.2} обеспечивают структуру содержащую ссылку 
на буфер собственно со строкой, переменная всегда содержащую длину строки 
(что очень удобно для массы функций \InSqBrackets{\CNotes 2.2.1}) и переменную содержащую текущий размер буфера.

Строка в буфере обыкновенно оканчивается нулем: это для того чтобы указатель на буфер можно было
передавать в функции требующие на вход обычную сишную \ac{ASCIIZ}-строку.

Стандарт \Cpp не описывает, как именно нужно реализовывать std::string,
но, как правило, они реализованы как описано выше, с небольшими дополнениями.

Строки в \Cpp это не класс (как, например, QString в Qt), а темплейт (basic\_string), 
это сделано для того чтобы поддерживать 
строки содержащие разного типа символы: как минимум \Tchar и \IT{wchar\_t}.

Так что, std::string это класс с базовым типом \Tchar.

А std::wstring это класс с базовым типом \IT{wchar\_t}.

\mysubparagraph{MSVC}

В реализации MSVC, вместо ссылки на буфер может содержаться сам буфер (если строка короче 16-и символов).

Это означает, что каждая короткая строка будет занимать в памяти по крайней мере $16 + 4 + 4 = 24$ 
байт для 32-битной среды либо $16 + 8 + 8 = 32$ 
байта в 64-битной, а если строка длиннее 16-и символов, то прибавьте еще длину самой строки.

\lstinputlisting[caption=пример для MSVC,style=customc]{\CURPATH/STL/string/MSVC_RU.cpp}

Собственно, из этого исходника почти всё ясно.

Несколько замечаний:

Если строка короче 16-и символов, 
то отдельный буфер для строки в \glslink{heap}{куче} выделяться не будет.

Это удобно потому что на практике, основная часть строк действительно короткие.
Вероятно, разработчики в Microsoft выбрали размер в 16 символов как разумный баланс.

Теперь очень важный момент в конце функции main(): мы не пользуемся методом c\_str(), тем не менее,
если это скомпилировать и запустить, то обе строки появятся в консоли!

Работает это вот почему.

В первом случае строка короче 16-и символов и в начале объекта std::string (его можно рассматривать
просто как структуру) расположен буфер с этой строкой.
\printf трактует указатель как указатель на массив символов оканчивающийся нулем и поэтому всё работает.

Вывод второй строки (длиннее 16-и символов) даже еще опаснее: это вообще типичная программистская ошибка 
(или опечатка), забыть дописать c\_str().
Это работает потому что в это время в начале структуры расположен указатель на буфер.
Это может надолго остаться незамеченным: до тех пока там не появится строка 
короче 16-и символов, тогда процесс упадет.

\mysubparagraph{GCC}

В реализации GCC в структуре есть еще одна переменная --- reference count.

Интересно, что указатель на экземпляр класса std::string в GCC указывает не на начало самой структуры, 
а на указатель на буфера.
В libstdc++-v3\textbackslash{}include\textbackslash{}bits\textbackslash{}basic\_string.h 
мы можем прочитать что это сделано для удобства отладки:

\begin{lstlisting}
   *  The reason you want _M_data pointing to the character %array and
   *  not the _Rep is so that the debugger can see the string
   *  contents. (Probably we should add a non-inline member to get
   *  the _Rep for the debugger to use, so users can check the actual
   *  string length.)
\end{lstlisting}

\href{http://go.yurichev.com/17085}{исходный код basic\_string.h}

В нашем примере мы учитываем это:

\lstinputlisting[caption=пример для GCC,style=customc]{\CURPATH/STL/string/GCC_RU.cpp}

Нужны еще небольшие хаки чтобы сымитировать типичную ошибку, которую мы уже видели выше, из-за
более ужесточенной проверки типов в GCC, тем не менее, printf() работает и здесь без c\_str().

\myparagraph{Чуть более сложный пример}

\lstinputlisting[style=customc]{\CURPATH/STL/string/3.cpp}

\lstinputlisting[caption=MSVC 2012,style=customasmx86]{\CURPATH/STL/string/3_MSVC_RU.asm}

Собственно, компилятор не конструирует строки статически: да в общем-то и как
это возможно, если буфер с ней нужно хранить в \glslink{heap}{куче}?

Вместо этого в сегменте данных хранятся обычные \ac{ASCIIZ}-строки, а позже, во время выполнения, 
при помощи метода \q{assign}, конструируются строки s1 и s2
.
При помощи \TT{operator+}, создается строка s3.

Обратите внимание на то что вызов метода c\_str() отсутствует,
потому что его код достаточно короткий и компилятор вставил его прямо здесь:
если строка короче 16-и байт, то в регистре EAX остается указатель на буфер,
а если длиннее, то из этого же места достается адрес на буфер расположенный в \glslink{heap}{куче}.

Далее следуют вызовы трех деструкторов, причем, они вызываются только если строка длиннее 16-и байт:
тогда нужно освободить буфера в \glslink{heap}{куче}.
В противном случае, так как все три объекта std::string хранятся в стеке,
они освобождаются автоматически после выхода из функции.

Следовательно, работа с короткими строками более быстрая из-за м\'{е}ньшего обращения к \glslink{heap}{куче}.

Код на GCC даже проще (из-за того, что в GCC, как мы уже видели, не реализована возможность хранить короткую
строку прямо в структуре):

% TODO1 comment each function meaning
\lstinputlisting[caption=GCC 4.8.1,style=customasmx86]{\CURPATH/STL/string/3_GCC_RU.s}

Можно заметить, что в деструкторы передается не указатель на объект,
а указатель на место за 12 байт (или 3 слова) перед ним, то есть, на настоящее начало структуры.

\myparagraph{std::string как глобальная переменная}
\label{sec:std_string_as_global_variable}

Опытные программисты на \Cpp знают, что глобальные переменные \ac{STL}-типов вполне можно объявлять.

Да, действительно:

\lstinputlisting[style=customc]{\CURPATH/STL/string/5.cpp}

Но как и где будет вызываться конструктор \TT{std::string}?

На самом деле, эта переменная будет инициализирована даже перед началом \main.

\lstinputlisting[caption=MSVC 2012: здесь конструируется глобальная переменная{,} а также регистрируется её деструктор,style=customasmx86]{\CURPATH/STL/string/5_MSVC_p2.asm}

\lstinputlisting[caption=MSVC 2012: здесь глобальная переменная используется в \main,style=customasmx86]{\CURPATH/STL/string/5_MSVC_p1.asm}

\lstinputlisting[caption=MSVC 2012: эта функция-деструктор вызывается перед выходом,style=customasmx86]{\CURPATH/STL/string/5_MSVC_p3.asm}

\myindex{\CStandardLibrary!atexit()}
В реальности, из \ac{CRT}, еще до вызова main(), вызывается специальная функция,
в которой перечислены все конструкторы подобных переменных.
Более того: при помощи atexit() регистрируется функция, которая будет вызвана в конце работы программы:
в этой функции компилятор собирает вызовы деструкторов всех подобных глобальных переменных.

GCC работает похожим образом:

\lstinputlisting[caption=GCC 4.8.1,style=customasmx86]{\CURPATH/STL/string/5_GCC.s}

Но он не выделяет отдельной функции в которой будут собраны деструкторы: 
каждый деструктор передается в atexit() по одному.

% TODO а если глобальная STL-переменная в другом модуле? надо проверить.

}
\DE{\subsection{Einfachste XOR-Verschlüsselung überhaupt}

Ich habe einmal eine Software gesehen, bei der alle Debugging-Ausgaben mit XOR mit dem Wert 3
verschlüsselt wurden. Mit anderen Worten, die beiden niedrigsten Bits aller Buchstaben wurden invertiert.

``Hello, world'' wurde zu ``Kfool/\#tlqog'':

\begin{lstlisting}
#!/usr/bin/python

msg="Hello, world!"

print "".join(map(lambda x: chr(ord(x)^3), msg))
\end{lstlisting}

Das ist eine ziemlich interessante Verschlüsselung (oder besser eine Verschleierung),
weil sie zwei wichtige Eigenschaften hat:
1) es ist eine einzige Funktion zum Verschlüsseln und entschlüsseln, sie muss nur wiederholt angewendet werden
2) die entstehenden Buchstaben befinden sich im druckbaren Bereich, also die ganze Zeichenkette kann ohne
Escape-Symbole im Code verwendet werden.

Die zweite Eigenschaft nutzt die Tatsache, dass alle druckbaren Zeichen in Reihen organisiert sind: 0x2x-0x7x,
und wenn die beiden niederwertigsten Bits invertiert werden, wird der Buchstabe um eine oder drei Stellen nach
links oder rechts \IT{verschoben}, aber niemals in eine andere Reihe:

\begin{figure}[H]
\centering
\includegraphics[width=0.7\textwidth]{ascii_clean.png}
\caption{7-Bit \ac{ASCII} Tabelle in Emacs}
\end{figure}

\dots mit dem Zeichen 0x7F als einziger Ausnahme.

Im Folgenden werden also beispielsweise die Zeichen A-Z \IT{verschlüsselt}:

\begin{lstlisting}
#!/usr/bin/python

msg="@ABCDEFGHIJKLMNO"

print "".join(map(lambda x: chr(ord(x)^3), msg))
\end{lstlisting}

Ergebnis:
% FIXME \verb  --  relevant comment for German?
\begin{lstlisting}
CBA@GFEDKJIHONML
\end{lstlisting}

Es sieht so aus als würden die Zeichen ``@'' und ``C'' sowie ``B'' und ``A'' vertauscht werden.

Hier ist noch ein interessantes Beispiel, in dem gezeigt wird, wie die Eigenschaften von XOR
ausgenutzt werden können: Exakt den gleichen Effekt, dass druckbare Zeichen auch druckbar bleiben,
kann man dadurch erzielen, dass irgendeine Kombination der niedrigsten vier Bits invertiert wird.
}

\EN{\section{Returning Values}
\label{ret_val_func}

Another simple function is the one that simply returns a constant value:

\lstinputlisting[caption=\EN{\CCpp Code},style=customc]{patterns/011_ret/1.c}

Let's compile it.

\subsection{x86}

Here's what both the GCC and MSVC compilers produce (with optimization) on the x86 platform:

\lstinputlisting[caption=\Optimizing GCC/MSVC (\assemblyOutput),style=customasmx86]{patterns/011_ret/1.s}

\myindex{x86!\Instructions!RET}
There are just two instructions: the first places the value 123 into the \EAX register,
which is used by convention for storing the return
value, and the second one is \RET, which returns execution to the \gls{caller}.

The caller will take the result from the \EAX register.

\subsection{ARM}

There are a few differences on the ARM platform:

\lstinputlisting[caption=\OptimizingKeilVI (\ARMMode) ASM Output,style=customasmARM]{patterns/011_ret/1_Keil_ARM_O3.s}

ARM uses the register \Reg{0} for returning the results of functions, so 123 is copied into \Reg{0}.

\myindex{ARM!\Instructions!MOV}
\myindex{x86!\Instructions!MOV}
It is worth noting that \MOV is a misleading name for the instruction in both the x86 and ARM \ac{ISA}s.

The data is not in fact \IT{moved}, but \IT{copied}.

\subsection{MIPS}

\label{MIPS_leaf_function_ex1}

The GCC assembly output below lists registers by number:

\lstinputlisting[caption=\Optimizing GCC 4.4.5 (\assemblyOutput),style=customasmMIPS]{patterns/011_ret/MIPS.s}

\dots while \IDA does it by their pseudo names:

\lstinputlisting[caption=\Optimizing GCC 4.4.5 (IDA),style=customasmMIPS]{patterns/011_ret/MIPS_IDA.lst}

The \$2 (or \$V0) register is used to store the function's return value.
\myindex{MIPS!\Pseudoinstructions!LI}
\INS{LI} stands for ``Load Immediate'' and is the MIPS equivalent to \MOV.

\myindex{MIPS!\Instructions!J}
The other instruction is the jump instruction (J or JR) which returns the execution flow to the \gls{caller}.

\myindex{MIPS!Branch delay slot}
You might be wondering why the positions of the load instruction (LI) and the jump instruction (J or JR) are swapped. This is due to a \ac{RISC} feature called ``branch delay slot''.

The reason this happens is a quirk in the architecture of some RISC \ac{ISA}s and isn't important for our
purposes---we must simply keep in mind that in MIPS, the instruction following a jump or branch instruction
is executed \IT{before} the jump/branch instruction itself.

As a consequence, branch instructions always swap places with the instruction executed immediately beforehand.


In practice, functions which merely return 1 (\IT{true}) or 0 (\IT{false}) are very frequent.

The smallest ever of the standard UNIX utilities, \IT{/bin/true} and \IT{/bin/false} return 0 and 1 respectively, as an exit code.
(Zero as an exit code usually means success, non-zero means error.)
}
\RU{\subsubsection{std::string}
\myindex{\Cpp!STL!std::string}
\label{std_string}

\myparagraph{Как устроена структура}

Многие строковые библиотеки \InSqBrackets{\CNotes 2.2} обеспечивают структуру содержащую ссылку 
на буфер собственно со строкой, переменная всегда содержащую длину строки 
(что очень удобно для массы функций \InSqBrackets{\CNotes 2.2.1}) и переменную содержащую текущий размер буфера.

Строка в буфере обыкновенно оканчивается нулем: это для того чтобы указатель на буфер можно было
передавать в функции требующие на вход обычную сишную \ac{ASCIIZ}-строку.

Стандарт \Cpp не описывает, как именно нужно реализовывать std::string,
но, как правило, они реализованы как описано выше, с небольшими дополнениями.

Строки в \Cpp это не класс (как, например, QString в Qt), а темплейт (basic\_string), 
это сделано для того чтобы поддерживать 
строки содержащие разного типа символы: как минимум \Tchar и \IT{wchar\_t}.

Так что, std::string это класс с базовым типом \Tchar.

А std::wstring это класс с базовым типом \IT{wchar\_t}.

\mysubparagraph{MSVC}

В реализации MSVC, вместо ссылки на буфер может содержаться сам буфер (если строка короче 16-и символов).

Это означает, что каждая короткая строка будет занимать в памяти по крайней мере $16 + 4 + 4 = 24$ 
байт для 32-битной среды либо $16 + 8 + 8 = 32$ 
байта в 64-битной, а если строка длиннее 16-и символов, то прибавьте еще длину самой строки.

\lstinputlisting[caption=пример для MSVC,style=customc]{\CURPATH/STL/string/MSVC_RU.cpp}

Собственно, из этого исходника почти всё ясно.

Несколько замечаний:

Если строка короче 16-и символов, 
то отдельный буфер для строки в \glslink{heap}{куче} выделяться не будет.

Это удобно потому что на практике, основная часть строк действительно короткие.
Вероятно, разработчики в Microsoft выбрали размер в 16 символов как разумный баланс.

Теперь очень важный момент в конце функции main(): мы не пользуемся методом c\_str(), тем не менее,
если это скомпилировать и запустить, то обе строки появятся в консоли!

Работает это вот почему.

В первом случае строка короче 16-и символов и в начале объекта std::string (его можно рассматривать
просто как структуру) расположен буфер с этой строкой.
\printf трактует указатель как указатель на массив символов оканчивающийся нулем и поэтому всё работает.

Вывод второй строки (длиннее 16-и символов) даже еще опаснее: это вообще типичная программистская ошибка 
(или опечатка), забыть дописать c\_str().
Это работает потому что в это время в начале структуры расположен указатель на буфер.
Это может надолго остаться незамеченным: до тех пока там не появится строка 
короче 16-и символов, тогда процесс упадет.

\mysubparagraph{GCC}

В реализации GCC в структуре есть еще одна переменная --- reference count.

Интересно, что указатель на экземпляр класса std::string в GCC указывает не на начало самой структуры, 
а на указатель на буфера.
В libstdc++-v3\textbackslash{}include\textbackslash{}bits\textbackslash{}basic\_string.h 
мы можем прочитать что это сделано для удобства отладки:

\begin{lstlisting}
   *  The reason you want _M_data pointing to the character %array and
   *  not the _Rep is so that the debugger can see the string
   *  contents. (Probably we should add a non-inline member to get
   *  the _Rep for the debugger to use, so users can check the actual
   *  string length.)
\end{lstlisting}

\href{http://go.yurichev.com/17085}{исходный код basic\_string.h}

В нашем примере мы учитываем это:

\lstinputlisting[caption=пример для GCC,style=customc]{\CURPATH/STL/string/GCC_RU.cpp}

Нужны еще небольшие хаки чтобы сымитировать типичную ошибку, которую мы уже видели выше, из-за
более ужесточенной проверки типов в GCC, тем не менее, printf() работает и здесь без c\_str().

\myparagraph{Чуть более сложный пример}

\lstinputlisting[style=customc]{\CURPATH/STL/string/3.cpp}

\lstinputlisting[caption=MSVC 2012,style=customasmx86]{\CURPATH/STL/string/3_MSVC_RU.asm}

Собственно, компилятор не конструирует строки статически: да в общем-то и как
это возможно, если буфер с ней нужно хранить в \glslink{heap}{куче}?

Вместо этого в сегменте данных хранятся обычные \ac{ASCIIZ}-строки, а позже, во время выполнения, 
при помощи метода \q{assign}, конструируются строки s1 и s2
.
При помощи \TT{operator+}, создается строка s3.

Обратите внимание на то что вызов метода c\_str() отсутствует,
потому что его код достаточно короткий и компилятор вставил его прямо здесь:
если строка короче 16-и байт, то в регистре EAX остается указатель на буфер,
а если длиннее, то из этого же места достается адрес на буфер расположенный в \glslink{heap}{куче}.

Далее следуют вызовы трех деструкторов, причем, они вызываются только если строка длиннее 16-и байт:
тогда нужно освободить буфера в \glslink{heap}{куче}.
В противном случае, так как все три объекта std::string хранятся в стеке,
они освобождаются автоматически после выхода из функции.

Следовательно, работа с короткими строками более быстрая из-за м\'{е}ньшего обращения к \glslink{heap}{куче}.

Код на GCC даже проще (из-за того, что в GCC, как мы уже видели, не реализована возможность хранить короткую
строку прямо в структуре):

% TODO1 comment each function meaning
\lstinputlisting[caption=GCC 4.8.1,style=customasmx86]{\CURPATH/STL/string/3_GCC_RU.s}

Можно заметить, что в деструкторы передается не указатель на объект,
а указатель на место за 12 байт (или 3 слова) перед ним, то есть, на настоящее начало структуры.

\myparagraph{std::string как глобальная переменная}
\label{sec:std_string_as_global_variable}

Опытные программисты на \Cpp знают, что глобальные переменные \ac{STL}-типов вполне можно объявлять.

Да, действительно:

\lstinputlisting[style=customc]{\CURPATH/STL/string/5.cpp}

Но как и где будет вызываться конструктор \TT{std::string}?

На самом деле, эта переменная будет инициализирована даже перед началом \main.

\lstinputlisting[caption=MSVC 2012: здесь конструируется глобальная переменная{,} а также регистрируется её деструктор,style=customasmx86]{\CURPATH/STL/string/5_MSVC_p2.asm}

\lstinputlisting[caption=MSVC 2012: здесь глобальная переменная используется в \main,style=customasmx86]{\CURPATH/STL/string/5_MSVC_p1.asm}

\lstinputlisting[caption=MSVC 2012: эта функция-деструктор вызывается перед выходом,style=customasmx86]{\CURPATH/STL/string/5_MSVC_p3.asm}

\myindex{\CStandardLibrary!atexit()}
В реальности, из \ac{CRT}, еще до вызова main(), вызывается специальная функция,
в которой перечислены все конструкторы подобных переменных.
Более того: при помощи atexit() регистрируется функция, которая будет вызвана в конце работы программы:
в этой функции компилятор собирает вызовы деструкторов всех подобных глобальных переменных.

GCC работает похожим образом:

\lstinputlisting[caption=GCC 4.8.1,style=customasmx86]{\CURPATH/STL/string/5_GCC.s}

Но он не выделяет отдельной функции в которой будут собраны деструкторы: 
каждый деструктор передается в atexit() по одному.

% TODO а если глобальная STL-переменная в другом модуле? надо проверить.

}
\DE{\subsection{Einfachste XOR-Verschlüsselung überhaupt}

Ich habe einmal eine Software gesehen, bei der alle Debugging-Ausgaben mit XOR mit dem Wert 3
verschlüsselt wurden. Mit anderen Worten, die beiden niedrigsten Bits aller Buchstaben wurden invertiert.

``Hello, world'' wurde zu ``Kfool/\#tlqog'':

\begin{lstlisting}
#!/usr/bin/python

msg="Hello, world!"

print "".join(map(lambda x: chr(ord(x)^3), msg))
\end{lstlisting}

Das ist eine ziemlich interessante Verschlüsselung (oder besser eine Verschleierung),
weil sie zwei wichtige Eigenschaften hat:
1) es ist eine einzige Funktion zum Verschlüsseln und entschlüsseln, sie muss nur wiederholt angewendet werden
2) die entstehenden Buchstaben befinden sich im druckbaren Bereich, also die ganze Zeichenkette kann ohne
Escape-Symbole im Code verwendet werden.

Die zweite Eigenschaft nutzt die Tatsache, dass alle druckbaren Zeichen in Reihen organisiert sind: 0x2x-0x7x,
und wenn die beiden niederwertigsten Bits invertiert werden, wird der Buchstabe um eine oder drei Stellen nach
links oder rechts \IT{verschoben}, aber niemals in eine andere Reihe:

\begin{figure}[H]
\centering
\includegraphics[width=0.7\textwidth]{ascii_clean.png}
\caption{7-Bit \ac{ASCII} Tabelle in Emacs}
\end{figure}

\dots mit dem Zeichen 0x7F als einziger Ausnahme.

Im Folgenden werden also beispielsweise die Zeichen A-Z \IT{verschlüsselt}:

\begin{lstlisting}
#!/usr/bin/python

msg="@ABCDEFGHIJKLMNO"

print "".join(map(lambda x: chr(ord(x)^3), msg))
\end{lstlisting}

Ergebnis:
% FIXME \verb  --  relevant comment for German?
\begin{lstlisting}
CBA@GFEDKJIHONML
\end{lstlisting}

Es sieht so aus als würden die Zeichen ``@'' und ``C'' sowie ``B'' und ``A'' vertauscht werden.

Hier ist noch ein interessantes Beispiel, in dem gezeigt wird, wie die Eigenschaften von XOR
ausgenutzt werden können: Exakt den gleichen Effekt, dass druckbare Zeichen auch druckbar bleiben,
kann man dadurch erzielen, dass irgendeine Kombination der niedrigsten vier Bits invertiert wird.
}

\ifdefined\SPANISH
\chapter{Patrones de código}
\fi % SPANISH

\ifdefined\GERMAN
\chapter{Code-Muster}
\fi % GERMAN

\ifdefined\ENGLISH
\chapter{Code Patterns}
\fi % ENGLISH

\ifdefined\ITALIAN
\chapter{Forme di codice}
\fi % ITALIAN

\ifdefined\RUSSIAN
\chapter{Образцы кода}
\fi % RUSSIAN

\ifdefined\BRAZILIAN
\chapter{Padrões de códigos}
\fi % BRAZILIAN

\ifdefined\THAI
\chapter{รูปแบบของโค้ด}
\fi % THAI

\ifdefined\FRENCH
\chapter{Modèle de code}
\fi % FRENCH

\ifdefined\POLISH
\chapter{\PLph{}}
\fi % POLISH

% sections
\EN{\input{patterns/patterns_opt_dbg_EN}}
\ES{\input{patterns/patterns_opt_dbg_ES}}
\ITA{\input{patterns/patterns_opt_dbg_ITA}}
\PTBR{\input{patterns/patterns_opt_dbg_PTBR}}
\RU{\input{patterns/patterns_opt_dbg_RU}}
\THA{\input{patterns/patterns_opt_dbg_THA}}
\DE{\input{patterns/patterns_opt_dbg_DE}}
\FR{\input{patterns/patterns_opt_dbg_FR}}
\PL{\input{patterns/patterns_opt_dbg_PL}}

\RU{\section{Некоторые базовые понятия}}
\EN{\section{Some basics}}
\DE{\section{Einige Grundlagen}}
\FR{\section{Quelques bases}}
\ES{\section{\ESph{}}}
\ITA{\section{Alcune basi teoriche}}
\PTBR{\section{\PTBRph{}}}
\THA{\section{\THAph{}}}
\PL{\section{\PLph{}}}

% sections:
\EN{\input{patterns/intro_CPU_ISA_EN}}
\ES{\input{patterns/intro_CPU_ISA_ES}}
\ITA{\input{patterns/intro_CPU_ISA_ITA}}
\PTBR{\input{patterns/intro_CPU_ISA_PTBR}}
\RU{\input{patterns/intro_CPU_ISA_RU}}
\DE{\input{patterns/intro_CPU_ISA_DE}}
\FR{\input{patterns/intro_CPU_ISA_FR}}
\PL{\input{patterns/intro_CPU_ISA_PL}}

\EN{\input{patterns/numeral_EN}}
\RU{\input{patterns/numeral_RU}}
\ITA{\input{patterns/numeral_ITA}}
\DE{\input{patterns/numeral_DE}}
\FR{\input{patterns/numeral_FR}}
\PL{\input{patterns/numeral_PL}}

% chapters
\input{patterns/00_empty/main}
\input{patterns/011_ret/main}
\input{patterns/01_helloworld/main}
\input{patterns/015_prolog_epilogue/main}
\input{patterns/02_stack/main}
\input{patterns/03_printf/main}
\input{patterns/04_scanf/main}
\input{patterns/05_passing_arguments/main}
\input{patterns/06_return_results/main}
\input{patterns/061_pointers/main}
\input{patterns/065_GOTO/main}
\input{patterns/07_jcc/main}
\input{patterns/08_switch/main}
\input{patterns/09_loops/main}
\input{patterns/10_strings/main}
\input{patterns/11_arith_optimizations/main}
\input{patterns/12_FPU/main}
\input{patterns/13_arrays/main}
\input{patterns/14_bitfields/main}
\EN{\input{patterns/145_LCG/main_EN}}
\RU{\input{patterns/145_LCG/main_RU}}
\input{patterns/15_structs/main}
\input{patterns/17_unions/main}
\input{patterns/18_pointers_to_functions/main}
\input{patterns/185_64bit_in_32_env/main}

\EN{\input{patterns/19_SIMD/main_EN}}
\RU{\input{patterns/19_SIMD/main_RU}}
\DE{\input{patterns/19_SIMD/main_DE}}

\EN{\input{patterns/20_x64/main_EN}}
\RU{\input{patterns/20_x64/main_RU}}

\EN{\input{patterns/205_floating_SIMD/main_EN}}
\RU{\input{patterns/205_floating_SIMD/main_RU}}
\DE{\input{patterns/205_floating_SIMD/main_DE}}

\EN{\input{patterns/ARM/main_EN}}
\RU{\input{patterns/ARM/main_RU}}
\DE{\input{patterns/ARM/main_DE}}

\input{patterns/MIPS/main}


\ifdefined\SPANISH
\chapter{Patrones de código}
\fi % SPANISH

\ifdefined\GERMAN
\chapter{Code-Muster}
\fi % GERMAN

\ifdefined\ENGLISH
\chapter{Code Patterns}
\fi % ENGLISH

\ifdefined\ITALIAN
\chapter{Forme di codice}
\fi % ITALIAN

\ifdefined\RUSSIAN
\chapter{Образцы кода}
\fi % RUSSIAN

\ifdefined\BRAZILIAN
\chapter{Padrões de códigos}
\fi % BRAZILIAN

\ifdefined\THAI
\chapter{รูปแบบของโค้ด}
\fi % THAI

\ifdefined\FRENCH
\chapter{Modèle de code}
\fi % FRENCH

\ifdefined\POLISH
\chapter{\PLph{}}
\fi % POLISH

% sections
\EN{\section{The method}

When the author of this book first started learning C and, later, \Cpp, he used to write small pieces of code, compile them,
and then look at the assembly language output. This made it very easy for him to understand what was going on in the code that he had written.
\footnote{In fact, he still does this when he can't understand what a particular bit of code does.}.
He did this so many times that the relationship between the \CCpp code and what the compiler produced was imprinted deeply in his mind.
It's now easy for him to imagine instantly a rough outline of a C code's appearance and function.
Perhaps this technique could be helpful for others.

%There are a lot of examples for both x86/x64 and ARM.
%Those who already familiar with one of architectures, may freely skim over pages.

By the way, there is a great website where you can do the same, with various compilers, instead of installing them on your box.
You can use it as well: \url{https://gcc.godbolt.org/}.

\section*{\Exercises}

When the author of this book studied assembly language, he also often compiled small C functions and then rewrote
them gradually to assembly, trying to make their code as short as possible.
This probably is not worth doing in real-world scenarios today,
because it's hard to compete with the latest compilers in terms of efficiency. It is, however, a very good way to gain a better understanding of assembly.
Feel free, therefore, to take any assembly code from this book and try to make it shorter.
However, don't forget to test what you have written.

% rewrote to show that debug\release and optimisations levels are orthogonal concepts.
\section*{Optimization levels and debug information}

Source code can be compiled by different compilers with various optimization levels.
A typical compiler has about three such levels, where level zero means that optimization is completely disabled.
Optimization can also be targeted towards code size or code speed.
A non-optimizing compiler is faster and produces more understandable (albeit verbose) code,
whereas an optimizing compiler is slower and tries to produce code that runs faster (but is not necessarily more compact).
In addition to optimization levels, a compiler can include some debug information in the resulting file,
producing code that is easy to debug.
One of the important features of the ´debug' code is that it might contain links
between each line of the source code and its respective machine code address.
Optimizing compilers, on the other hand, tend to produce output where entire lines of source code
can be optimized away and thus not even be present in the resulting machine code.
Reverse engineers can encounter either version, simply because some developers turn on the compiler's optimization flags and others do not.
Because of this, we'll try to work on examples of both debug and release versions of the code featured in this book, wherever possible.

Sometimes some pretty ancient compilers are used in this book, in order to get the shortest (or simplest) possible code snippet.
}
\ES{\input{patterns/patterns_opt_dbg_ES}}
\ITA{\input{patterns/patterns_opt_dbg_ITA}}
\PTBR{\input{patterns/patterns_opt_dbg_PTBR}}
\RU{\input{patterns/patterns_opt_dbg_RU}}
\THA{\input{patterns/patterns_opt_dbg_THA}}
\DE{\input{patterns/patterns_opt_dbg_DE}}
\FR{\input{patterns/patterns_opt_dbg_FR}}
\PL{\input{patterns/patterns_opt_dbg_PL}}

\RU{\section{Некоторые базовые понятия}}
\EN{\section{Some basics}}
\DE{\section{Einige Grundlagen}}
\FR{\section{Quelques bases}}
\ES{\section{\ESph{}}}
\ITA{\section{Alcune basi teoriche}}
\PTBR{\section{\PTBRph{}}}
\THA{\section{\THAph{}}}
\PL{\section{\PLph{}}}

% sections:
\EN{\input{patterns/intro_CPU_ISA_EN}}
\ES{\input{patterns/intro_CPU_ISA_ES}}
\ITA{\input{patterns/intro_CPU_ISA_ITA}}
\PTBR{\input{patterns/intro_CPU_ISA_PTBR}}
\RU{\input{patterns/intro_CPU_ISA_RU}}
\DE{\input{patterns/intro_CPU_ISA_DE}}
\FR{\input{patterns/intro_CPU_ISA_FR}}
\PL{\input{patterns/intro_CPU_ISA_PL}}

\EN{\subsection{Numeral Systems}

Humans have become accustomed to a decimal numeral system, probably because almost everyone has 10 fingers.
Nevertheless, the number \q{10} has no significant meaning in science and mathematics.
The natural numeral system in digital electronics is binary: 0 is for an absence of current in the wire, and 1 for presence.
10 in binary is 2 in decimal, 100 in binary is 4 in decimal, and so on.

% This sentence is a bit unweildy - maybe try 'Our ten-digit system would be described as having a radix...' - Renaissance
If the numeral system has 10 digits, it has a \IT{radix} (or \IT{base}) of 10.
The binary numeral system has a \IT{radix} of 2.

Important things to recall:

1) A \IT{number} is a number, while a \IT{digit} is a term from writing systems, and is usually one character

% The original is 'number' is not changed; I think the intent is value, and changed it - Renaissance
2) The value of a number does not change when converted to another radix; only the writing notation for that value has changed (and therefore the way of representing it in \ac{RAM}).

\subsection{Converting From One Radix To Another}

Positional notation is used almost every numerical system. This means that a digit has weight relative to where it is placed inside of the larger number.
If 2 is placed at the rightmost place, it's 2, but if it's placed one digit before rightmost, it's 20.

What does $1234$ stand for?

$10^3 \cdot 1 + 10^2 \cdot 2 + 10^1 \cdot 3 + 1 \cdot 4 = 1234$ or
$1000 \cdot 1 + 100 \cdot 2 + 10 \cdot 3 + 4 = 1234$

It's the same story for binary numbers, but the base is 2 instead of 10.
What does 0b101011 stand for?

$2^5 \cdot 1 + 2^4 \cdot 0 + 2^3 \cdot 1 + 2^2 \cdot 0 + 2^1 \cdot 1 + 2^0 \cdot 1 = 43$ or
$32 \cdot 1 + 16 \cdot 0 + 8 \cdot 1 + 4 \cdot 0 + 2 \cdot 1 + 1 = 43$

There is such a thing as non-positional notation, such as the Roman numeral system.
\footnote{About numeric system evolution, see \InSqBrackets{\TAOCPvolII{}, 195--213.}}.
% Maybe add a sentence to fill in that X is always 10, and is therefore non-positional, even though putting an I before subtracts and after adds, and is in that sense positional
Perhaps, humankind switched to positional notation because it's easier to do basic operations (addition, multiplication, etc.) on paper by hand.

Binary numbers can be added, subtracted and so on in the very same as taught in schools, but only 2 digits are available.

Binary numbers are bulky when represented in source code and dumps, so that is where the hexadecimal numeral system can be useful.
A hexadecimal radix uses the digits 0..9, and also 6 Latin characters: A..F.
Each hexadecimal digit takes 4 bits or 4 binary digits, so it's very easy to convert from binary number to hexadecimal and back, even manually, in one's mind.

\begin{center}
\begin{longtable}{ | l | l | l | }
\hline
\HeaderColor hexadecimal & \HeaderColor binary & \HeaderColor decimal \\
\hline
0	&0000	&0 \\
1	&0001	&1 \\
2	&0010	&2 \\
3	&0011	&3 \\
4	&0100	&4 \\
5	&0101	&5 \\
6	&0110	&6 \\
7	&0111	&7 \\
8	&1000	&8 \\
9	&1001	&9 \\
A	&1010	&10 \\
B	&1011	&11 \\
C	&1100	&12 \\
D	&1101	&13 \\
E	&1110	&14 \\
F	&1111	&15 \\
\hline
\end{longtable}
\end{center}

How can one tell which radix is being used in a specific instance?

Decimal numbers are usually written as is, i.e., 1234. Some assemblers allow an identifier on decimal radix numbers, in which the number would be written with a "d" suffix: 1234d.

Binary numbers are sometimes prepended with the "0b" prefix: 0b100110111 (\ac{GCC} has a non-standard language extension for this\footnote{\url{https://gcc.gnu.org/onlinedocs/gcc/Binary-constants.html}}).
There is also another way: using a "b" suffix, for example: 100110111b.
This book tries to use the "0b" prefix consistently throughout the book for binary numbers.

Hexadecimal numbers are prepended with "0x" prefix in \CCpp and other \ac{PL}s: 0x1234ABCD.
Alternatively, they are given a "h" suffix: 1234ABCDh. This is common way of representing them in assemblers and debuggers.
In this convention, if the number is started with a Latin (A..F) digit, a 0 is added at the beginning: 0ABCDEFh.
There was also convention that was popular in 8-bit home computers era, using \$ prefix, like \$ABCD.
The book will try to stick to "0x" prefix throughout the book for hexadecimal numbers.

Should one learn to convert numbers mentally? A table of 1-digit hexadecimal numbers can easily be memorized.
As for larger numbers, it's probably not worth tormenting yourself.

Perhaps the most visible hexadecimal numbers are in \ac{URL}s.
This is the way that non-Latin characters are encoded.
For example:
\url{https://en.wiktionary.org/wiki/na\%C3\%AFvet\%C3\%A9} is the \ac{URL} of Wiktionary article about \q{naïveté} word.

\subsubsection{Octal Radix}

Another numeral system heavily used in the past of computer programming is octal. In octal there are 8 digits (0..7), and each is mapped to 3 bits, so it's easy to convert numbers back and forth.
It has been superseded by the hexadecimal system almost everywhere, but, surprisingly, there is a *NIX utility, used often by many people, which takes octal numbers as argument: \TT{chmod}.

\myindex{UNIX!chmod}
As many *NIX users know, \TT{chmod} argument can be a number of 3 digits. The first digit represents the rights of the owner of the file (read, write and/or execute), the second is the rights for the group to which the file belongs, and the third is for everyone else.
Each digit that \TT{chmod} takes can be represented in binary form:

\begin{center}
\begin{longtable}{ | l | l | l | }
\hline
\HeaderColor decimal & \HeaderColor binary & \HeaderColor meaning \\
\hline
7	&111	&\textbf{rwx} \\
6	&110	&\textbf{rw-} \\
5	&101	&\textbf{r-x} \\
4	&100	&\textbf{r-{}-} \\
3	&011	&\textbf{-wx} \\
2	&010	&\textbf{-w-} \\
1	&001	&\textbf{-{}-x} \\
0	&000	&\textbf{-{}-{}-} \\
\hline
\end{longtable}
\end{center}

So each bit is mapped to a flag: read/write/execute.

The importance of \TT{chmod} here is that the whole number in argument can be represented as octal number.
Let's take, for example, 644.
When you run \TT{chmod 644 file}, you set read/write permissions for owner, read permissions for group and again, read permissions for everyone else.
If we convert the octal number 644 to binary, it would be \TT{110100100}, or, in groups of 3 bits, \TT{110 100 100}.

Now we see that each triplet describe permissions for owner/group/others: first is \TT{rw-}, second is \TT{r--} and third is \TT{r--}.

The octal numeral system was also popular on old computers like PDP-8, because word there could be 12, 24 or 36 bits, and these numbers are all divisible by 3, so the octal system was natural in that environment.
Nowadays, all popular computers employ word/address sizes of 16, 32 or 64 bits, and these numbers are all divisible by 4, so the hexadecimal system is more natural there.

The octal numeral system is supported by all standard \CCpp compilers.
This is a source of confusion sometimes, because octal numbers are encoded with a zero prepended, for example, 0377 is 255.
Sometimes, you might make a typo and write "09" instead of 9, and the compiler would report an error.
GCC might report something like this:\\
\TT{error: invalid digit "9" in octal constant}.

Also, the octal system is somewhat popular in Java. When the IDA shows Java strings with non-printable characters,
they are encoded in the octal system instead of hexadecimal.
\myindex{JAD}
The JAD Java decompiler behaves the same way.

\subsubsection{Divisibility}

When you see a decimal number like 120, you can quickly deduce that it's divisible by 10, because the last digit is zero.
In the same way, 123400 is divisible by 100, because the two last digits are zeros.

Likewise, the hexadecimal number 0x1230 is divisible by 0x10 (or 16), 0x123000 is divisible by 0x1000 (or 4096), etc.

The binary number 0b1000101000 is divisible by 0b1000 (8), etc.

This property can often be used to quickly realize if the size of some block in memory is padded to some boundary.
For example, sections in \ac{PE} files are almost always started at addresses ending with 3 hexadecimal zeros: 0x41000, 0x10001000, etc.
The reason behind this is the fact that almost all \ac{PE} sections are padded to a boundary of 0x1000 (4096) bytes.

\subsubsection{Multi-Precision Arithmetic and Radix}

\index{RSA}
Multi-precision arithmetic can use huge numbers, and each one may be stored in several bytes.
For example, RSA keys, both public and private, span up to 4096 bits, and maybe even more.

% I'm not sure how to change this, but the normal format for quoting would be just to mention the author or book, and footnote to the full reference
In \InSqBrackets{\TAOCPvolII, 265} we find the following idea: when you store a multi-precision number in several bytes,
the whole number can be represented as having a radix of $2^8=256$, and each digit goes to the corresponding byte.
Likewise, if you store a multi-precision number in several 32-bit integer values, each digit goes to each 32-bit slot,
and you may think about this number as stored in radix of $2^{32}$.

\subsubsection{How to Pronounce Non-Decimal Numbers}

Numbers in a non-decimal base are usually pronounced by digit by digit: ``one-zero-zero-one-one-...''.
Words like ``ten'' and ``thousand'' are usually not pronounced, to prevent confusion with the decimal base system.

\subsubsection{Floating point numbers}

To distinguish floating point numbers from integers, they are usually written with ``.0'' at the end,
like $0.0$, $123.0$, etc.
}
\RU{\subsection{Представление чисел}

Люди привыкли к десятичной системе счисления вероятно потому что почти у каждого есть по 10 пальцев.
Тем не менее, число 10 не имеет особого значения в науке и математике.
Двоичная система естествена для цифровой электроники: 0 означает отсутствие тока в проводе и 1 --- его присутствие.
10 в двоичной системе это 2 в десятичной; 100 в двоичной это 4 в десятичной, итд.

Если в системе счисления есть 10 цифр, её \IT{основание} или \IT{radix} это 10.
Двоичная система имеет \IT{основание} 2.

Важные вещи, которые полезно вспомнить:
1) \IT{число} это число, в то время как \IT{цифра} это термин из системы письменности, и это обычно один символ;
2) само число не меняется, когда конвертируется из одного основания в другое: меняется способ его записи (или представления
в памяти).

Как сконвертировать число из одного основания в другое?

Позиционная нотация используется почти везде, это означает, что всякая цифра имеет свой вес, в зависимости от её расположения
внутри числа.
Если 2 расположена в самом последнем месте справа, это 2.
Если она расположена в месте перед последним, это 20.

Что означает $1234$?

$10^3 \cdot 1 + 10^2 \cdot 2 + 10^1 \cdot 3 + 1 \cdot 4$ = 1234 или
$1000 \cdot 1 + 100 \cdot 2 + 10 \cdot 3 + 4 = 1234$

Та же история и для двоичных чисел, только основание там 2 вместо 10.
Что означает 0b101011?

$2^5 \cdot 1 + 2^4 \cdot 0 + 2^3 \cdot 1 + 2^2 \cdot 0 + 2^1 \cdot 1 + 2^0 \cdot 1 = 43$ или
$32 \cdot 1 + 16 \cdot 0 + 8 \cdot 1 + 4 \cdot 0 + 2 \cdot 1 + 1 = 43$

Позиционную нотацию можно противопоставить непозиционной нотации, такой как римская система записи чисел
\footnote{Об эволюции способов записи чисел, см.также: \InSqBrackets{\TAOCPvolII{}, 195--213.}}.
Вероятно, человечество перешло на позиционную нотацию, потому что так проще работать с числами (сложение, умножение, итд)
на бумаге, в ручную.

Действительно, двоичные числа можно складывать, вычитать, итд, точно также, как этому обычно обучают в школах,
только доступны лишь 2 цифры.

Двоичные числа громоздки, когда их используют в исходных кодах и дампах, так что в этих случаях применяется шестнадцатеричная
система.
Используются цифры 0..9 и еще 6 латинских букв: A..F.
Каждая шестнадцатеричная цифра занимает 4 бита или 4 двоичных цифры, так что конвертировать из двоичной системы в
шестнадцатеричную и назад, можно легко вручную, или даже в уме.

\begin{center}
\begin{longtable}{ | l | l | l | }
\hline
\HeaderColor шестнадцатеричная & \HeaderColor двоичная & \HeaderColor десятичная \\
\hline
0	&0000	&0 \\
1	&0001	&1 \\
2	&0010	&2 \\
3	&0011	&3 \\
4	&0100	&4 \\
5	&0101	&5 \\
6	&0110	&6 \\
7	&0111	&7 \\
8	&1000	&8 \\
9	&1001	&9 \\
A	&1010	&10 \\
B	&1011	&11 \\
C	&1100	&12 \\
D	&1101	&13 \\
E	&1110	&14 \\
F	&1111	&15 \\
\hline
\end{longtable}
\end{center}

Как понять, какое основание используется в конкретном месте?

Десятичные числа обычно записываются как есть, т.е., 1234. Но некоторые ассемблеры позволяют подчеркивать
этот факт для ясности, и это число может быть дополнено суффиксом "d": 1234d.

К двоичным числам иногда спереди добавляют префикс "0b": 0b100110111
(В \ac{GCC} для этого есть нестандартное расширение языка
\footnote{\url{https://gcc.gnu.org/onlinedocs/gcc/Binary-constants.html}}).
Есть также еще один способ: суффикс "b", например: 100110111b.
В этой книге я буду пытаться придерживаться префикса "0b" для двоичных чисел.

Шестнадцатеричные числа имеют префикс "0x" в \CCpp и некоторых других \ac{PL}: 0x1234ABCD.
Либо они имеют суффикс "h": 1234ABCDh --- обычно так они представляются в ассемблерах и отладчиках.
Если число начинается с цифры A..F, перед ним добавляется 0: 0ABCDEFh.
Во времена 8-битных домашних компьютеров, был также способ записи чисел используя префикс \$, например, \$ABCD.
В книге я попытаюсь придерживаться префикса "0x" для шестнадцатеричных чисел.

Нужно ли учиться конвертировать числа в уме? Таблицу шестнадцатеричных чисел из одной цифры легко запомнить.
А запоминать б\'{о}льшие числа, наверное, не стоит.

Наверное, чаще всего шестнадцатеричные числа можно увидеть в \ac{URL}-ах.
Так кодируются буквы не из числа латинских.
Например:
\url{https://en.wiktionary.org/wiki/na\%C3\%AFvet\%C3\%A9} это \ac{URL} страницы в Wiktionary о слове \q{naïveté}.

\subsubsection{Восьмеричная система}

Еще одна система, которая в прошлом много использовалась в программировании это восьмеричная: есть 8 цифр (0..7) и каждая
описывает 3 бита, так что легко конвертировать числа туда и назад.
Она почти везде была заменена шестнадцатеричной, но удивительно, в *NIX имеется утилита использующаяся многими людьми,
которая принимает на вход восьмеричное число: \TT{chmod}.

\myindex{UNIX!chmod}
Как знают многие пользователи *NIX, аргумент \TT{chmod} это число из трех цифр. Первая цифра это права владельца файла,
вторая это права группы (которой файл принадлежит), третья для всех остальных.
И каждая цифра может быть представлена в двоичном виде:

\begin{center}
\begin{longtable}{ | l | l | l | }
\hline
\HeaderColor десятичная & \HeaderColor двоичная & \HeaderColor значение \\
\hline
7	&111	&\textbf{rwx} \\
6	&110	&\textbf{rw-} \\
5	&101	&\textbf{r-x} \\
4	&100	&\textbf{r-{}-} \\
3	&011	&\textbf{-wx} \\
2	&010	&\textbf{-w-} \\
1	&001	&\textbf{-{}-x} \\
0	&000	&\textbf{-{}-{}-} \\
\hline
\end{longtable}
\end{center}

Так что каждый бит привязан к флагу: read/write/execute (чтение/запись/исполнение).

И вот почему я вспомнил здесь о \TT{chmod}, это потому что всё число может быть представлено как число в восьмеричной системе.
Для примера возьмем 644.
Когда вы запускаете \TT{chmod 644 file}, вы выставляете права read/write для владельца, права read для группы, и снова,
read для всех остальных.
Сконвертируем число 644 из восьмеричной системы в двоичную, это будет \TT{110100100}, или (в группах по 3 бита) \TT{110 100 100}.

Теперь мы видим, что каждая тройка описывает права для владельца/группы/остальных:
первая это \TT{rw-}, вторая это \TT{r--} и третья это \TT{r--}.

Восьмеричная система была также популярная на старых компьютерах вроде PDP-8, потому что слово там могло содержать 12, 24 или
36 бит, и эти числа делятся на 3, так что выбор восьмеричной системы в той среде был логичен.
Сейчас, все популярные компьютеры имеют размер слова/адреса 16, 32 или 64 бита, и эти числа делятся на 4,
так что шестнадцатеричная система здесь удобнее.

Восьмеричная система поддерживается всеми стандартными компиляторами \CCpp{}.
Это иногда источник недоумения, потому что восьмеричные числа кодируются с нулем вперед, например, 0377 это 255.
И иногда, вы можете сделать опечатку, и написать "09" вместо 9, и компилятор выдаст ошибку.
GCC может выдать что-то вроде:\\
\TT{error: invalid digit "9" in octal constant}.

Также, восьмеричная система популярна в Java: когда IDA показывает строку с непечатаемыми символами,
они кодируются в восьмеричной системе вместо шестнадцатеричной.
\myindex{JAD}
Точно также себя ведет декомпилятор с Java JAD.

\subsubsection{Делимость}

Когда вы видите десятичное число вроде 120, вы можете быстро понять что оно делится на 10, потому что последняя цифра это 0.
Точно также, 123400 делится на 100, потому что две последних цифры это нули.

Точно также, шестнадцатеричное число 0x1230 делится на 0x10 (или 16), 0x123000 делится на 0x1000 (или 4096), итд.

Двоичное число 0b1000101000 делится на 0b1000 (8), итд.

Это свойство можно часто использовать, чтобы быстро понять,
что длина какого-либо блока в памяти выровнена по некоторой границе.
Например, секции в \ac{PE}-файлах почти всегда начинаются с адресов заканчивающихся 3 шестнадцатеричными нулями:
0x41000, 0x10001000, итд.
Причина в том, что почти все секции в \ac{PE} выровнены по границе 0x1000 (4096) байт.

\subsubsection{Арифметика произвольной точности и основание}

\index{RSA}
Арифметика произвольной точности (multi-precision arithmetic) может использовать огромные числа,
которые могут храниться в нескольких байтах.
Например, ключи RSA, и открытые и закрытые, могут занимать до 4096 бит и даже больше.

В \InSqBrackets{\TAOCPvolII, 265} можно найти такую идею: когда вы сохраняете число произвольной точности в нескольких байтах,
всё число может быть представлено как имеющую систему счисления по основанию $2^8=256$, и каждая цифра находится
в соответствующем байте.
Точно также, если вы сохраняете число произвольной точности в нескольких 32-битных целочисленных значениях,
каждая цифра отправляется в каждый 32-битный слот, и вы можете считать что это число записано в системе с основанием $2^{32}$.

\subsubsection{Произношение}

Числа в недесятичных системах счислениях обычно произносятся по одной цифре: ``один-ноль-ноль-один-один-...''.
Слова вроде ``десять'', ``тысяча'', итд, обычно не произносятся, потому что тогда можно спутать с десятичной системой.

\subsubsection{Числа с плавающей запятой}

Чтобы отличать числа с плавающей запятой от целочисленных, часто, в конце добавляют ``.0'',
например $0.0$, $123.0$, итд.

}
\ITA{\input{patterns/numeral_ITA}}
\DE{\input{patterns/numeral_DE}}
\FR{\input{patterns/numeral_FR}}
\PL{\input{patterns/numeral_PL}}

% chapters
\ifdefined\SPANISH
\chapter{Patrones de código}
\fi % SPANISH

\ifdefined\GERMAN
\chapter{Code-Muster}
\fi % GERMAN

\ifdefined\ENGLISH
\chapter{Code Patterns}
\fi % ENGLISH

\ifdefined\ITALIAN
\chapter{Forme di codice}
\fi % ITALIAN

\ifdefined\RUSSIAN
\chapter{Образцы кода}
\fi % RUSSIAN

\ifdefined\BRAZILIAN
\chapter{Padrões de códigos}
\fi % BRAZILIAN

\ifdefined\THAI
\chapter{รูปแบบของโค้ด}
\fi % THAI

\ifdefined\FRENCH
\chapter{Modèle de code}
\fi % FRENCH

\ifdefined\POLISH
\chapter{\PLph{}}
\fi % POLISH

% sections
\EN{\input{patterns/patterns_opt_dbg_EN}}
\ES{\input{patterns/patterns_opt_dbg_ES}}
\ITA{\input{patterns/patterns_opt_dbg_ITA}}
\PTBR{\input{patterns/patterns_opt_dbg_PTBR}}
\RU{\input{patterns/patterns_opt_dbg_RU}}
\THA{\input{patterns/patterns_opt_dbg_THA}}
\DE{\input{patterns/patterns_opt_dbg_DE}}
\FR{\input{patterns/patterns_opt_dbg_FR}}
\PL{\input{patterns/patterns_opt_dbg_PL}}

\RU{\section{Некоторые базовые понятия}}
\EN{\section{Some basics}}
\DE{\section{Einige Grundlagen}}
\FR{\section{Quelques bases}}
\ES{\section{\ESph{}}}
\ITA{\section{Alcune basi teoriche}}
\PTBR{\section{\PTBRph{}}}
\THA{\section{\THAph{}}}
\PL{\section{\PLph{}}}

% sections:
\EN{\input{patterns/intro_CPU_ISA_EN}}
\ES{\input{patterns/intro_CPU_ISA_ES}}
\ITA{\input{patterns/intro_CPU_ISA_ITA}}
\PTBR{\input{patterns/intro_CPU_ISA_PTBR}}
\RU{\input{patterns/intro_CPU_ISA_RU}}
\DE{\input{patterns/intro_CPU_ISA_DE}}
\FR{\input{patterns/intro_CPU_ISA_FR}}
\PL{\input{patterns/intro_CPU_ISA_PL}}

\EN{\input{patterns/numeral_EN}}
\RU{\input{patterns/numeral_RU}}
\ITA{\input{patterns/numeral_ITA}}
\DE{\input{patterns/numeral_DE}}
\FR{\input{patterns/numeral_FR}}
\PL{\input{patterns/numeral_PL}}

% chapters
\input{patterns/00_empty/main}
\input{patterns/011_ret/main}
\input{patterns/01_helloworld/main}
\input{patterns/015_prolog_epilogue/main}
\input{patterns/02_stack/main}
\input{patterns/03_printf/main}
\input{patterns/04_scanf/main}
\input{patterns/05_passing_arguments/main}
\input{patterns/06_return_results/main}
\input{patterns/061_pointers/main}
\input{patterns/065_GOTO/main}
\input{patterns/07_jcc/main}
\input{patterns/08_switch/main}
\input{patterns/09_loops/main}
\input{patterns/10_strings/main}
\input{patterns/11_arith_optimizations/main}
\input{patterns/12_FPU/main}
\input{patterns/13_arrays/main}
\input{patterns/14_bitfields/main}
\EN{\input{patterns/145_LCG/main_EN}}
\RU{\input{patterns/145_LCG/main_RU}}
\input{patterns/15_structs/main}
\input{patterns/17_unions/main}
\input{patterns/18_pointers_to_functions/main}
\input{patterns/185_64bit_in_32_env/main}

\EN{\input{patterns/19_SIMD/main_EN}}
\RU{\input{patterns/19_SIMD/main_RU}}
\DE{\input{patterns/19_SIMD/main_DE}}

\EN{\input{patterns/20_x64/main_EN}}
\RU{\input{patterns/20_x64/main_RU}}

\EN{\input{patterns/205_floating_SIMD/main_EN}}
\RU{\input{patterns/205_floating_SIMD/main_RU}}
\DE{\input{patterns/205_floating_SIMD/main_DE}}

\EN{\input{patterns/ARM/main_EN}}
\RU{\input{patterns/ARM/main_RU}}
\DE{\input{patterns/ARM/main_DE}}

\input{patterns/MIPS/main}

\ifdefined\SPANISH
\chapter{Patrones de código}
\fi % SPANISH

\ifdefined\GERMAN
\chapter{Code-Muster}
\fi % GERMAN

\ifdefined\ENGLISH
\chapter{Code Patterns}
\fi % ENGLISH

\ifdefined\ITALIAN
\chapter{Forme di codice}
\fi % ITALIAN

\ifdefined\RUSSIAN
\chapter{Образцы кода}
\fi % RUSSIAN

\ifdefined\BRAZILIAN
\chapter{Padrões de códigos}
\fi % BRAZILIAN

\ifdefined\THAI
\chapter{รูปแบบของโค้ด}
\fi % THAI

\ifdefined\FRENCH
\chapter{Modèle de code}
\fi % FRENCH

\ifdefined\POLISH
\chapter{\PLph{}}
\fi % POLISH

% sections
\EN{\input{patterns/patterns_opt_dbg_EN}}
\ES{\input{patterns/patterns_opt_dbg_ES}}
\ITA{\input{patterns/patterns_opt_dbg_ITA}}
\PTBR{\input{patterns/patterns_opt_dbg_PTBR}}
\RU{\input{patterns/patterns_opt_dbg_RU}}
\THA{\input{patterns/patterns_opt_dbg_THA}}
\DE{\input{patterns/patterns_opt_dbg_DE}}
\FR{\input{patterns/patterns_opt_dbg_FR}}
\PL{\input{patterns/patterns_opt_dbg_PL}}

\RU{\section{Некоторые базовые понятия}}
\EN{\section{Some basics}}
\DE{\section{Einige Grundlagen}}
\FR{\section{Quelques bases}}
\ES{\section{\ESph{}}}
\ITA{\section{Alcune basi teoriche}}
\PTBR{\section{\PTBRph{}}}
\THA{\section{\THAph{}}}
\PL{\section{\PLph{}}}

% sections:
\EN{\input{patterns/intro_CPU_ISA_EN}}
\ES{\input{patterns/intro_CPU_ISA_ES}}
\ITA{\input{patterns/intro_CPU_ISA_ITA}}
\PTBR{\input{patterns/intro_CPU_ISA_PTBR}}
\RU{\input{patterns/intro_CPU_ISA_RU}}
\DE{\input{patterns/intro_CPU_ISA_DE}}
\FR{\input{patterns/intro_CPU_ISA_FR}}
\PL{\input{patterns/intro_CPU_ISA_PL}}

\EN{\input{patterns/numeral_EN}}
\RU{\input{patterns/numeral_RU}}
\ITA{\input{patterns/numeral_ITA}}
\DE{\input{patterns/numeral_DE}}
\FR{\input{patterns/numeral_FR}}
\PL{\input{patterns/numeral_PL}}

% chapters
\input{patterns/00_empty/main}
\input{patterns/011_ret/main}
\input{patterns/01_helloworld/main}
\input{patterns/015_prolog_epilogue/main}
\input{patterns/02_stack/main}
\input{patterns/03_printf/main}
\input{patterns/04_scanf/main}
\input{patterns/05_passing_arguments/main}
\input{patterns/06_return_results/main}
\input{patterns/061_pointers/main}
\input{patterns/065_GOTO/main}
\input{patterns/07_jcc/main}
\input{patterns/08_switch/main}
\input{patterns/09_loops/main}
\input{patterns/10_strings/main}
\input{patterns/11_arith_optimizations/main}
\input{patterns/12_FPU/main}
\input{patterns/13_arrays/main}
\input{patterns/14_bitfields/main}
\EN{\input{patterns/145_LCG/main_EN}}
\RU{\input{patterns/145_LCG/main_RU}}
\input{patterns/15_structs/main}
\input{patterns/17_unions/main}
\input{patterns/18_pointers_to_functions/main}
\input{patterns/185_64bit_in_32_env/main}

\EN{\input{patterns/19_SIMD/main_EN}}
\RU{\input{patterns/19_SIMD/main_RU}}
\DE{\input{patterns/19_SIMD/main_DE}}

\EN{\input{patterns/20_x64/main_EN}}
\RU{\input{patterns/20_x64/main_RU}}

\EN{\input{patterns/205_floating_SIMD/main_EN}}
\RU{\input{patterns/205_floating_SIMD/main_RU}}
\DE{\input{patterns/205_floating_SIMD/main_DE}}

\EN{\input{patterns/ARM/main_EN}}
\RU{\input{patterns/ARM/main_RU}}
\DE{\input{patterns/ARM/main_DE}}

\input{patterns/MIPS/main}

\ifdefined\SPANISH
\chapter{Patrones de código}
\fi % SPANISH

\ifdefined\GERMAN
\chapter{Code-Muster}
\fi % GERMAN

\ifdefined\ENGLISH
\chapter{Code Patterns}
\fi % ENGLISH

\ifdefined\ITALIAN
\chapter{Forme di codice}
\fi % ITALIAN

\ifdefined\RUSSIAN
\chapter{Образцы кода}
\fi % RUSSIAN

\ifdefined\BRAZILIAN
\chapter{Padrões de códigos}
\fi % BRAZILIAN

\ifdefined\THAI
\chapter{รูปแบบของโค้ด}
\fi % THAI

\ifdefined\FRENCH
\chapter{Modèle de code}
\fi % FRENCH

\ifdefined\POLISH
\chapter{\PLph{}}
\fi % POLISH

% sections
\EN{\input{patterns/patterns_opt_dbg_EN}}
\ES{\input{patterns/patterns_opt_dbg_ES}}
\ITA{\input{patterns/patterns_opt_dbg_ITA}}
\PTBR{\input{patterns/patterns_opt_dbg_PTBR}}
\RU{\input{patterns/patterns_opt_dbg_RU}}
\THA{\input{patterns/patterns_opt_dbg_THA}}
\DE{\input{patterns/patterns_opt_dbg_DE}}
\FR{\input{patterns/patterns_opt_dbg_FR}}
\PL{\input{patterns/patterns_opt_dbg_PL}}

\RU{\section{Некоторые базовые понятия}}
\EN{\section{Some basics}}
\DE{\section{Einige Grundlagen}}
\FR{\section{Quelques bases}}
\ES{\section{\ESph{}}}
\ITA{\section{Alcune basi teoriche}}
\PTBR{\section{\PTBRph{}}}
\THA{\section{\THAph{}}}
\PL{\section{\PLph{}}}

% sections:
\EN{\input{patterns/intro_CPU_ISA_EN}}
\ES{\input{patterns/intro_CPU_ISA_ES}}
\ITA{\input{patterns/intro_CPU_ISA_ITA}}
\PTBR{\input{patterns/intro_CPU_ISA_PTBR}}
\RU{\input{patterns/intro_CPU_ISA_RU}}
\DE{\input{patterns/intro_CPU_ISA_DE}}
\FR{\input{patterns/intro_CPU_ISA_FR}}
\PL{\input{patterns/intro_CPU_ISA_PL}}

\EN{\input{patterns/numeral_EN}}
\RU{\input{patterns/numeral_RU}}
\ITA{\input{patterns/numeral_ITA}}
\DE{\input{patterns/numeral_DE}}
\FR{\input{patterns/numeral_FR}}
\PL{\input{patterns/numeral_PL}}

% chapters
\input{patterns/00_empty/main}
\input{patterns/011_ret/main}
\input{patterns/01_helloworld/main}
\input{patterns/015_prolog_epilogue/main}
\input{patterns/02_stack/main}
\input{patterns/03_printf/main}
\input{patterns/04_scanf/main}
\input{patterns/05_passing_arguments/main}
\input{patterns/06_return_results/main}
\input{patterns/061_pointers/main}
\input{patterns/065_GOTO/main}
\input{patterns/07_jcc/main}
\input{patterns/08_switch/main}
\input{patterns/09_loops/main}
\input{patterns/10_strings/main}
\input{patterns/11_arith_optimizations/main}
\input{patterns/12_FPU/main}
\input{patterns/13_arrays/main}
\input{patterns/14_bitfields/main}
\EN{\input{patterns/145_LCG/main_EN}}
\RU{\input{patterns/145_LCG/main_RU}}
\input{patterns/15_structs/main}
\input{patterns/17_unions/main}
\input{patterns/18_pointers_to_functions/main}
\input{patterns/185_64bit_in_32_env/main}

\EN{\input{patterns/19_SIMD/main_EN}}
\RU{\input{patterns/19_SIMD/main_RU}}
\DE{\input{patterns/19_SIMD/main_DE}}

\EN{\input{patterns/20_x64/main_EN}}
\RU{\input{patterns/20_x64/main_RU}}

\EN{\input{patterns/205_floating_SIMD/main_EN}}
\RU{\input{patterns/205_floating_SIMD/main_RU}}
\DE{\input{patterns/205_floating_SIMD/main_DE}}

\EN{\input{patterns/ARM/main_EN}}
\RU{\input{patterns/ARM/main_RU}}
\DE{\input{patterns/ARM/main_DE}}

\input{patterns/MIPS/main}

\ifdefined\SPANISH
\chapter{Patrones de código}
\fi % SPANISH

\ifdefined\GERMAN
\chapter{Code-Muster}
\fi % GERMAN

\ifdefined\ENGLISH
\chapter{Code Patterns}
\fi % ENGLISH

\ifdefined\ITALIAN
\chapter{Forme di codice}
\fi % ITALIAN

\ifdefined\RUSSIAN
\chapter{Образцы кода}
\fi % RUSSIAN

\ifdefined\BRAZILIAN
\chapter{Padrões de códigos}
\fi % BRAZILIAN

\ifdefined\THAI
\chapter{รูปแบบของโค้ด}
\fi % THAI

\ifdefined\FRENCH
\chapter{Modèle de code}
\fi % FRENCH

\ifdefined\POLISH
\chapter{\PLph{}}
\fi % POLISH

% sections
\EN{\input{patterns/patterns_opt_dbg_EN}}
\ES{\input{patterns/patterns_opt_dbg_ES}}
\ITA{\input{patterns/patterns_opt_dbg_ITA}}
\PTBR{\input{patterns/patterns_opt_dbg_PTBR}}
\RU{\input{patterns/patterns_opt_dbg_RU}}
\THA{\input{patterns/patterns_opt_dbg_THA}}
\DE{\input{patterns/patterns_opt_dbg_DE}}
\FR{\input{patterns/patterns_opt_dbg_FR}}
\PL{\input{patterns/patterns_opt_dbg_PL}}

\RU{\section{Некоторые базовые понятия}}
\EN{\section{Some basics}}
\DE{\section{Einige Grundlagen}}
\FR{\section{Quelques bases}}
\ES{\section{\ESph{}}}
\ITA{\section{Alcune basi teoriche}}
\PTBR{\section{\PTBRph{}}}
\THA{\section{\THAph{}}}
\PL{\section{\PLph{}}}

% sections:
\EN{\input{patterns/intro_CPU_ISA_EN}}
\ES{\input{patterns/intro_CPU_ISA_ES}}
\ITA{\input{patterns/intro_CPU_ISA_ITA}}
\PTBR{\input{patterns/intro_CPU_ISA_PTBR}}
\RU{\input{patterns/intro_CPU_ISA_RU}}
\DE{\input{patterns/intro_CPU_ISA_DE}}
\FR{\input{patterns/intro_CPU_ISA_FR}}
\PL{\input{patterns/intro_CPU_ISA_PL}}

\EN{\input{patterns/numeral_EN}}
\RU{\input{patterns/numeral_RU}}
\ITA{\input{patterns/numeral_ITA}}
\DE{\input{patterns/numeral_DE}}
\FR{\input{patterns/numeral_FR}}
\PL{\input{patterns/numeral_PL}}

% chapters
\input{patterns/00_empty/main}
\input{patterns/011_ret/main}
\input{patterns/01_helloworld/main}
\input{patterns/015_prolog_epilogue/main}
\input{patterns/02_stack/main}
\input{patterns/03_printf/main}
\input{patterns/04_scanf/main}
\input{patterns/05_passing_arguments/main}
\input{patterns/06_return_results/main}
\input{patterns/061_pointers/main}
\input{patterns/065_GOTO/main}
\input{patterns/07_jcc/main}
\input{patterns/08_switch/main}
\input{patterns/09_loops/main}
\input{patterns/10_strings/main}
\input{patterns/11_arith_optimizations/main}
\input{patterns/12_FPU/main}
\input{patterns/13_arrays/main}
\input{patterns/14_bitfields/main}
\EN{\input{patterns/145_LCG/main_EN}}
\RU{\input{patterns/145_LCG/main_RU}}
\input{patterns/15_structs/main}
\input{patterns/17_unions/main}
\input{patterns/18_pointers_to_functions/main}
\input{patterns/185_64bit_in_32_env/main}

\EN{\input{patterns/19_SIMD/main_EN}}
\RU{\input{patterns/19_SIMD/main_RU}}
\DE{\input{patterns/19_SIMD/main_DE}}

\EN{\input{patterns/20_x64/main_EN}}
\RU{\input{patterns/20_x64/main_RU}}

\EN{\input{patterns/205_floating_SIMD/main_EN}}
\RU{\input{patterns/205_floating_SIMD/main_RU}}
\DE{\input{patterns/205_floating_SIMD/main_DE}}

\EN{\input{patterns/ARM/main_EN}}
\RU{\input{patterns/ARM/main_RU}}
\DE{\input{patterns/ARM/main_DE}}

\input{patterns/MIPS/main}

\ifdefined\SPANISH
\chapter{Patrones de código}
\fi % SPANISH

\ifdefined\GERMAN
\chapter{Code-Muster}
\fi % GERMAN

\ifdefined\ENGLISH
\chapter{Code Patterns}
\fi % ENGLISH

\ifdefined\ITALIAN
\chapter{Forme di codice}
\fi % ITALIAN

\ifdefined\RUSSIAN
\chapter{Образцы кода}
\fi % RUSSIAN

\ifdefined\BRAZILIAN
\chapter{Padrões de códigos}
\fi % BRAZILIAN

\ifdefined\THAI
\chapter{รูปแบบของโค้ด}
\fi % THAI

\ifdefined\FRENCH
\chapter{Modèle de code}
\fi % FRENCH

\ifdefined\POLISH
\chapter{\PLph{}}
\fi % POLISH

% sections
\EN{\input{patterns/patterns_opt_dbg_EN}}
\ES{\input{patterns/patterns_opt_dbg_ES}}
\ITA{\input{patterns/patterns_opt_dbg_ITA}}
\PTBR{\input{patterns/patterns_opt_dbg_PTBR}}
\RU{\input{patterns/patterns_opt_dbg_RU}}
\THA{\input{patterns/patterns_opt_dbg_THA}}
\DE{\input{patterns/patterns_opt_dbg_DE}}
\FR{\input{patterns/patterns_opt_dbg_FR}}
\PL{\input{patterns/patterns_opt_dbg_PL}}

\RU{\section{Некоторые базовые понятия}}
\EN{\section{Some basics}}
\DE{\section{Einige Grundlagen}}
\FR{\section{Quelques bases}}
\ES{\section{\ESph{}}}
\ITA{\section{Alcune basi teoriche}}
\PTBR{\section{\PTBRph{}}}
\THA{\section{\THAph{}}}
\PL{\section{\PLph{}}}

% sections:
\EN{\input{patterns/intro_CPU_ISA_EN}}
\ES{\input{patterns/intro_CPU_ISA_ES}}
\ITA{\input{patterns/intro_CPU_ISA_ITA}}
\PTBR{\input{patterns/intro_CPU_ISA_PTBR}}
\RU{\input{patterns/intro_CPU_ISA_RU}}
\DE{\input{patterns/intro_CPU_ISA_DE}}
\FR{\input{patterns/intro_CPU_ISA_FR}}
\PL{\input{patterns/intro_CPU_ISA_PL}}

\EN{\input{patterns/numeral_EN}}
\RU{\input{patterns/numeral_RU}}
\ITA{\input{patterns/numeral_ITA}}
\DE{\input{patterns/numeral_DE}}
\FR{\input{patterns/numeral_FR}}
\PL{\input{patterns/numeral_PL}}

% chapters
\input{patterns/00_empty/main}
\input{patterns/011_ret/main}
\input{patterns/01_helloworld/main}
\input{patterns/015_prolog_epilogue/main}
\input{patterns/02_stack/main}
\input{patterns/03_printf/main}
\input{patterns/04_scanf/main}
\input{patterns/05_passing_arguments/main}
\input{patterns/06_return_results/main}
\input{patterns/061_pointers/main}
\input{patterns/065_GOTO/main}
\input{patterns/07_jcc/main}
\input{patterns/08_switch/main}
\input{patterns/09_loops/main}
\input{patterns/10_strings/main}
\input{patterns/11_arith_optimizations/main}
\input{patterns/12_FPU/main}
\input{patterns/13_arrays/main}
\input{patterns/14_bitfields/main}
\EN{\input{patterns/145_LCG/main_EN}}
\RU{\input{patterns/145_LCG/main_RU}}
\input{patterns/15_structs/main}
\input{patterns/17_unions/main}
\input{patterns/18_pointers_to_functions/main}
\input{patterns/185_64bit_in_32_env/main}

\EN{\input{patterns/19_SIMD/main_EN}}
\RU{\input{patterns/19_SIMD/main_RU}}
\DE{\input{patterns/19_SIMD/main_DE}}

\EN{\input{patterns/20_x64/main_EN}}
\RU{\input{patterns/20_x64/main_RU}}

\EN{\input{patterns/205_floating_SIMD/main_EN}}
\RU{\input{patterns/205_floating_SIMD/main_RU}}
\DE{\input{patterns/205_floating_SIMD/main_DE}}

\EN{\input{patterns/ARM/main_EN}}
\RU{\input{patterns/ARM/main_RU}}
\DE{\input{patterns/ARM/main_DE}}

\input{patterns/MIPS/main}

\ifdefined\SPANISH
\chapter{Patrones de código}
\fi % SPANISH

\ifdefined\GERMAN
\chapter{Code-Muster}
\fi % GERMAN

\ifdefined\ENGLISH
\chapter{Code Patterns}
\fi % ENGLISH

\ifdefined\ITALIAN
\chapter{Forme di codice}
\fi % ITALIAN

\ifdefined\RUSSIAN
\chapter{Образцы кода}
\fi % RUSSIAN

\ifdefined\BRAZILIAN
\chapter{Padrões de códigos}
\fi % BRAZILIAN

\ifdefined\THAI
\chapter{รูปแบบของโค้ด}
\fi % THAI

\ifdefined\FRENCH
\chapter{Modèle de code}
\fi % FRENCH

\ifdefined\POLISH
\chapter{\PLph{}}
\fi % POLISH

% sections
\EN{\input{patterns/patterns_opt_dbg_EN}}
\ES{\input{patterns/patterns_opt_dbg_ES}}
\ITA{\input{patterns/patterns_opt_dbg_ITA}}
\PTBR{\input{patterns/patterns_opt_dbg_PTBR}}
\RU{\input{patterns/patterns_opt_dbg_RU}}
\THA{\input{patterns/patterns_opt_dbg_THA}}
\DE{\input{patterns/patterns_opt_dbg_DE}}
\FR{\input{patterns/patterns_opt_dbg_FR}}
\PL{\input{patterns/patterns_opt_dbg_PL}}

\RU{\section{Некоторые базовые понятия}}
\EN{\section{Some basics}}
\DE{\section{Einige Grundlagen}}
\FR{\section{Quelques bases}}
\ES{\section{\ESph{}}}
\ITA{\section{Alcune basi teoriche}}
\PTBR{\section{\PTBRph{}}}
\THA{\section{\THAph{}}}
\PL{\section{\PLph{}}}

% sections:
\EN{\input{patterns/intro_CPU_ISA_EN}}
\ES{\input{patterns/intro_CPU_ISA_ES}}
\ITA{\input{patterns/intro_CPU_ISA_ITA}}
\PTBR{\input{patterns/intro_CPU_ISA_PTBR}}
\RU{\input{patterns/intro_CPU_ISA_RU}}
\DE{\input{patterns/intro_CPU_ISA_DE}}
\FR{\input{patterns/intro_CPU_ISA_FR}}
\PL{\input{patterns/intro_CPU_ISA_PL}}

\EN{\input{patterns/numeral_EN}}
\RU{\input{patterns/numeral_RU}}
\ITA{\input{patterns/numeral_ITA}}
\DE{\input{patterns/numeral_DE}}
\FR{\input{patterns/numeral_FR}}
\PL{\input{patterns/numeral_PL}}

% chapters
\input{patterns/00_empty/main}
\input{patterns/011_ret/main}
\input{patterns/01_helloworld/main}
\input{patterns/015_prolog_epilogue/main}
\input{patterns/02_stack/main}
\input{patterns/03_printf/main}
\input{patterns/04_scanf/main}
\input{patterns/05_passing_arguments/main}
\input{patterns/06_return_results/main}
\input{patterns/061_pointers/main}
\input{patterns/065_GOTO/main}
\input{patterns/07_jcc/main}
\input{patterns/08_switch/main}
\input{patterns/09_loops/main}
\input{patterns/10_strings/main}
\input{patterns/11_arith_optimizations/main}
\input{patterns/12_FPU/main}
\input{patterns/13_arrays/main}
\input{patterns/14_bitfields/main}
\EN{\input{patterns/145_LCG/main_EN}}
\RU{\input{patterns/145_LCG/main_RU}}
\input{patterns/15_structs/main}
\input{patterns/17_unions/main}
\input{patterns/18_pointers_to_functions/main}
\input{patterns/185_64bit_in_32_env/main}

\EN{\input{patterns/19_SIMD/main_EN}}
\RU{\input{patterns/19_SIMD/main_RU}}
\DE{\input{patterns/19_SIMD/main_DE}}

\EN{\input{patterns/20_x64/main_EN}}
\RU{\input{patterns/20_x64/main_RU}}

\EN{\input{patterns/205_floating_SIMD/main_EN}}
\RU{\input{patterns/205_floating_SIMD/main_RU}}
\DE{\input{patterns/205_floating_SIMD/main_DE}}

\EN{\input{patterns/ARM/main_EN}}
\RU{\input{patterns/ARM/main_RU}}
\DE{\input{patterns/ARM/main_DE}}

\input{patterns/MIPS/main}

\ifdefined\SPANISH
\chapter{Patrones de código}
\fi % SPANISH

\ifdefined\GERMAN
\chapter{Code-Muster}
\fi % GERMAN

\ifdefined\ENGLISH
\chapter{Code Patterns}
\fi % ENGLISH

\ifdefined\ITALIAN
\chapter{Forme di codice}
\fi % ITALIAN

\ifdefined\RUSSIAN
\chapter{Образцы кода}
\fi % RUSSIAN

\ifdefined\BRAZILIAN
\chapter{Padrões de códigos}
\fi % BRAZILIAN

\ifdefined\THAI
\chapter{รูปแบบของโค้ด}
\fi % THAI

\ifdefined\FRENCH
\chapter{Modèle de code}
\fi % FRENCH

\ifdefined\POLISH
\chapter{\PLph{}}
\fi % POLISH

% sections
\EN{\input{patterns/patterns_opt_dbg_EN}}
\ES{\input{patterns/patterns_opt_dbg_ES}}
\ITA{\input{patterns/patterns_opt_dbg_ITA}}
\PTBR{\input{patterns/patterns_opt_dbg_PTBR}}
\RU{\input{patterns/patterns_opt_dbg_RU}}
\THA{\input{patterns/patterns_opt_dbg_THA}}
\DE{\input{patterns/patterns_opt_dbg_DE}}
\FR{\input{patterns/patterns_opt_dbg_FR}}
\PL{\input{patterns/patterns_opt_dbg_PL}}

\RU{\section{Некоторые базовые понятия}}
\EN{\section{Some basics}}
\DE{\section{Einige Grundlagen}}
\FR{\section{Quelques bases}}
\ES{\section{\ESph{}}}
\ITA{\section{Alcune basi teoriche}}
\PTBR{\section{\PTBRph{}}}
\THA{\section{\THAph{}}}
\PL{\section{\PLph{}}}

% sections:
\EN{\input{patterns/intro_CPU_ISA_EN}}
\ES{\input{patterns/intro_CPU_ISA_ES}}
\ITA{\input{patterns/intro_CPU_ISA_ITA}}
\PTBR{\input{patterns/intro_CPU_ISA_PTBR}}
\RU{\input{patterns/intro_CPU_ISA_RU}}
\DE{\input{patterns/intro_CPU_ISA_DE}}
\FR{\input{patterns/intro_CPU_ISA_FR}}
\PL{\input{patterns/intro_CPU_ISA_PL}}

\EN{\input{patterns/numeral_EN}}
\RU{\input{patterns/numeral_RU}}
\ITA{\input{patterns/numeral_ITA}}
\DE{\input{patterns/numeral_DE}}
\FR{\input{patterns/numeral_FR}}
\PL{\input{patterns/numeral_PL}}

% chapters
\input{patterns/00_empty/main}
\input{patterns/011_ret/main}
\input{patterns/01_helloworld/main}
\input{patterns/015_prolog_epilogue/main}
\input{patterns/02_stack/main}
\input{patterns/03_printf/main}
\input{patterns/04_scanf/main}
\input{patterns/05_passing_arguments/main}
\input{patterns/06_return_results/main}
\input{patterns/061_pointers/main}
\input{patterns/065_GOTO/main}
\input{patterns/07_jcc/main}
\input{patterns/08_switch/main}
\input{patterns/09_loops/main}
\input{patterns/10_strings/main}
\input{patterns/11_arith_optimizations/main}
\input{patterns/12_FPU/main}
\input{patterns/13_arrays/main}
\input{patterns/14_bitfields/main}
\EN{\input{patterns/145_LCG/main_EN}}
\RU{\input{patterns/145_LCG/main_RU}}
\input{patterns/15_structs/main}
\input{patterns/17_unions/main}
\input{patterns/18_pointers_to_functions/main}
\input{patterns/185_64bit_in_32_env/main}

\EN{\input{patterns/19_SIMD/main_EN}}
\RU{\input{patterns/19_SIMD/main_RU}}
\DE{\input{patterns/19_SIMD/main_DE}}

\EN{\input{patterns/20_x64/main_EN}}
\RU{\input{patterns/20_x64/main_RU}}

\EN{\input{patterns/205_floating_SIMD/main_EN}}
\RU{\input{patterns/205_floating_SIMD/main_RU}}
\DE{\input{patterns/205_floating_SIMD/main_DE}}

\EN{\input{patterns/ARM/main_EN}}
\RU{\input{patterns/ARM/main_RU}}
\DE{\input{patterns/ARM/main_DE}}

\input{patterns/MIPS/main}

\ifdefined\SPANISH
\chapter{Patrones de código}
\fi % SPANISH

\ifdefined\GERMAN
\chapter{Code-Muster}
\fi % GERMAN

\ifdefined\ENGLISH
\chapter{Code Patterns}
\fi % ENGLISH

\ifdefined\ITALIAN
\chapter{Forme di codice}
\fi % ITALIAN

\ifdefined\RUSSIAN
\chapter{Образцы кода}
\fi % RUSSIAN

\ifdefined\BRAZILIAN
\chapter{Padrões de códigos}
\fi % BRAZILIAN

\ifdefined\THAI
\chapter{รูปแบบของโค้ด}
\fi % THAI

\ifdefined\FRENCH
\chapter{Modèle de code}
\fi % FRENCH

\ifdefined\POLISH
\chapter{\PLph{}}
\fi % POLISH

% sections
\EN{\input{patterns/patterns_opt_dbg_EN}}
\ES{\input{patterns/patterns_opt_dbg_ES}}
\ITA{\input{patterns/patterns_opt_dbg_ITA}}
\PTBR{\input{patterns/patterns_opt_dbg_PTBR}}
\RU{\input{patterns/patterns_opt_dbg_RU}}
\THA{\input{patterns/patterns_opt_dbg_THA}}
\DE{\input{patterns/patterns_opt_dbg_DE}}
\FR{\input{patterns/patterns_opt_dbg_FR}}
\PL{\input{patterns/patterns_opt_dbg_PL}}

\RU{\section{Некоторые базовые понятия}}
\EN{\section{Some basics}}
\DE{\section{Einige Grundlagen}}
\FR{\section{Quelques bases}}
\ES{\section{\ESph{}}}
\ITA{\section{Alcune basi teoriche}}
\PTBR{\section{\PTBRph{}}}
\THA{\section{\THAph{}}}
\PL{\section{\PLph{}}}

% sections:
\EN{\input{patterns/intro_CPU_ISA_EN}}
\ES{\input{patterns/intro_CPU_ISA_ES}}
\ITA{\input{patterns/intro_CPU_ISA_ITA}}
\PTBR{\input{patterns/intro_CPU_ISA_PTBR}}
\RU{\input{patterns/intro_CPU_ISA_RU}}
\DE{\input{patterns/intro_CPU_ISA_DE}}
\FR{\input{patterns/intro_CPU_ISA_FR}}
\PL{\input{patterns/intro_CPU_ISA_PL}}

\EN{\input{patterns/numeral_EN}}
\RU{\input{patterns/numeral_RU}}
\ITA{\input{patterns/numeral_ITA}}
\DE{\input{patterns/numeral_DE}}
\FR{\input{patterns/numeral_FR}}
\PL{\input{patterns/numeral_PL}}

% chapters
\input{patterns/00_empty/main}
\input{patterns/011_ret/main}
\input{patterns/01_helloworld/main}
\input{patterns/015_prolog_epilogue/main}
\input{patterns/02_stack/main}
\input{patterns/03_printf/main}
\input{patterns/04_scanf/main}
\input{patterns/05_passing_arguments/main}
\input{patterns/06_return_results/main}
\input{patterns/061_pointers/main}
\input{patterns/065_GOTO/main}
\input{patterns/07_jcc/main}
\input{patterns/08_switch/main}
\input{patterns/09_loops/main}
\input{patterns/10_strings/main}
\input{patterns/11_arith_optimizations/main}
\input{patterns/12_FPU/main}
\input{patterns/13_arrays/main}
\input{patterns/14_bitfields/main}
\EN{\input{patterns/145_LCG/main_EN}}
\RU{\input{patterns/145_LCG/main_RU}}
\input{patterns/15_structs/main}
\input{patterns/17_unions/main}
\input{patterns/18_pointers_to_functions/main}
\input{patterns/185_64bit_in_32_env/main}

\EN{\input{patterns/19_SIMD/main_EN}}
\RU{\input{patterns/19_SIMD/main_RU}}
\DE{\input{patterns/19_SIMD/main_DE}}

\EN{\input{patterns/20_x64/main_EN}}
\RU{\input{patterns/20_x64/main_RU}}

\EN{\input{patterns/205_floating_SIMD/main_EN}}
\RU{\input{patterns/205_floating_SIMD/main_RU}}
\DE{\input{patterns/205_floating_SIMD/main_DE}}

\EN{\input{patterns/ARM/main_EN}}
\RU{\input{patterns/ARM/main_RU}}
\DE{\input{patterns/ARM/main_DE}}

\input{patterns/MIPS/main}

\ifdefined\SPANISH
\chapter{Patrones de código}
\fi % SPANISH

\ifdefined\GERMAN
\chapter{Code-Muster}
\fi % GERMAN

\ifdefined\ENGLISH
\chapter{Code Patterns}
\fi % ENGLISH

\ifdefined\ITALIAN
\chapter{Forme di codice}
\fi % ITALIAN

\ifdefined\RUSSIAN
\chapter{Образцы кода}
\fi % RUSSIAN

\ifdefined\BRAZILIAN
\chapter{Padrões de códigos}
\fi % BRAZILIAN

\ifdefined\THAI
\chapter{รูปแบบของโค้ด}
\fi % THAI

\ifdefined\FRENCH
\chapter{Modèle de code}
\fi % FRENCH

\ifdefined\POLISH
\chapter{\PLph{}}
\fi % POLISH

% sections
\EN{\input{patterns/patterns_opt_dbg_EN}}
\ES{\input{patterns/patterns_opt_dbg_ES}}
\ITA{\input{patterns/patterns_opt_dbg_ITA}}
\PTBR{\input{patterns/patterns_opt_dbg_PTBR}}
\RU{\input{patterns/patterns_opt_dbg_RU}}
\THA{\input{patterns/patterns_opt_dbg_THA}}
\DE{\input{patterns/patterns_opt_dbg_DE}}
\FR{\input{patterns/patterns_opt_dbg_FR}}
\PL{\input{patterns/patterns_opt_dbg_PL}}

\RU{\section{Некоторые базовые понятия}}
\EN{\section{Some basics}}
\DE{\section{Einige Grundlagen}}
\FR{\section{Quelques bases}}
\ES{\section{\ESph{}}}
\ITA{\section{Alcune basi teoriche}}
\PTBR{\section{\PTBRph{}}}
\THA{\section{\THAph{}}}
\PL{\section{\PLph{}}}

% sections:
\EN{\input{patterns/intro_CPU_ISA_EN}}
\ES{\input{patterns/intro_CPU_ISA_ES}}
\ITA{\input{patterns/intro_CPU_ISA_ITA}}
\PTBR{\input{patterns/intro_CPU_ISA_PTBR}}
\RU{\input{patterns/intro_CPU_ISA_RU}}
\DE{\input{patterns/intro_CPU_ISA_DE}}
\FR{\input{patterns/intro_CPU_ISA_FR}}
\PL{\input{patterns/intro_CPU_ISA_PL}}

\EN{\input{patterns/numeral_EN}}
\RU{\input{patterns/numeral_RU}}
\ITA{\input{patterns/numeral_ITA}}
\DE{\input{patterns/numeral_DE}}
\FR{\input{patterns/numeral_FR}}
\PL{\input{patterns/numeral_PL}}

% chapters
\input{patterns/00_empty/main}
\input{patterns/011_ret/main}
\input{patterns/01_helloworld/main}
\input{patterns/015_prolog_epilogue/main}
\input{patterns/02_stack/main}
\input{patterns/03_printf/main}
\input{patterns/04_scanf/main}
\input{patterns/05_passing_arguments/main}
\input{patterns/06_return_results/main}
\input{patterns/061_pointers/main}
\input{patterns/065_GOTO/main}
\input{patterns/07_jcc/main}
\input{patterns/08_switch/main}
\input{patterns/09_loops/main}
\input{patterns/10_strings/main}
\input{patterns/11_arith_optimizations/main}
\input{patterns/12_FPU/main}
\input{patterns/13_arrays/main}
\input{patterns/14_bitfields/main}
\EN{\input{patterns/145_LCG/main_EN}}
\RU{\input{patterns/145_LCG/main_RU}}
\input{patterns/15_structs/main}
\input{patterns/17_unions/main}
\input{patterns/18_pointers_to_functions/main}
\input{patterns/185_64bit_in_32_env/main}

\EN{\input{patterns/19_SIMD/main_EN}}
\RU{\input{patterns/19_SIMD/main_RU}}
\DE{\input{patterns/19_SIMD/main_DE}}

\EN{\input{patterns/20_x64/main_EN}}
\RU{\input{patterns/20_x64/main_RU}}

\EN{\input{patterns/205_floating_SIMD/main_EN}}
\RU{\input{patterns/205_floating_SIMD/main_RU}}
\DE{\input{patterns/205_floating_SIMD/main_DE}}

\EN{\input{patterns/ARM/main_EN}}
\RU{\input{patterns/ARM/main_RU}}
\DE{\input{patterns/ARM/main_DE}}

\input{patterns/MIPS/main}

\ifdefined\SPANISH
\chapter{Patrones de código}
\fi % SPANISH

\ifdefined\GERMAN
\chapter{Code-Muster}
\fi % GERMAN

\ifdefined\ENGLISH
\chapter{Code Patterns}
\fi % ENGLISH

\ifdefined\ITALIAN
\chapter{Forme di codice}
\fi % ITALIAN

\ifdefined\RUSSIAN
\chapter{Образцы кода}
\fi % RUSSIAN

\ifdefined\BRAZILIAN
\chapter{Padrões de códigos}
\fi % BRAZILIAN

\ifdefined\THAI
\chapter{รูปแบบของโค้ด}
\fi % THAI

\ifdefined\FRENCH
\chapter{Modèle de code}
\fi % FRENCH

\ifdefined\POLISH
\chapter{\PLph{}}
\fi % POLISH

% sections
\EN{\input{patterns/patterns_opt_dbg_EN}}
\ES{\input{patterns/patterns_opt_dbg_ES}}
\ITA{\input{patterns/patterns_opt_dbg_ITA}}
\PTBR{\input{patterns/patterns_opt_dbg_PTBR}}
\RU{\input{patterns/patterns_opt_dbg_RU}}
\THA{\input{patterns/patterns_opt_dbg_THA}}
\DE{\input{patterns/patterns_opt_dbg_DE}}
\FR{\input{patterns/patterns_opt_dbg_FR}}
\PL{\input{patterns/patterns_opt_dbg_PL}}

\RU{\section{Некоторые базовые понятия}}
\EN{\section{Some basics}}
\DE{\section{Einige Grundlagen}}
\FR{\section{Quelques bases}}
\ES{\section{\ESph{}}}
\ITA{\section{Alcune basi teoriche}}
\PTBR{\section{\PTBRph{}}}
\THA{\section{\THAph{}}}
\PL{\section{\PLph{}}}

% sections:
\EN{\input{patterns/intro_CPU_ISA_EN}}
\ES{\input{patterns/intro_CPU_ISA_ES}}
\ITA{\input{patterns/intro_CPU_ISA_ITA}}
\PTBR{\input{patterns/intro_CPU_ISA_PTBR}}
\RU{\input{patterns/intro_CPU_ISA_RU}}
\DE{\input{patterns/intro_CPU_ISA_DE}}
\FR{\input{patterns/intro_CPU_ISA_FR}}
\PL{\input{patterns/intro_CPU_ISA_PL}}

\EN{\input{patterns/numeral_EN}}
\RU{\input{patterns/numeral_RU}}
\ITA{\input{patterns/numeral_ITA}}
\DE{\input{patterns/numeral_DE}}
\FR{\input{patterns/numeral_FR}}
\PL{\input{patterns/numeral_PL}}

% chapters
\input{patterns/00_empty/main}
\input{patterns/011_ret/main}
\input{patterns/01_helloworld/main}
\input{patterns/015_prolog_epilogue/main}
\input{patterns/02_stack/main}
\input{patterns/03_printf/main}
\input{patterns/04_scanf/main}
\input{patterns/05_passing_arguments/main}
\input{patterns/06_return_results/main}
\input{patterns/061_pointers/main}
\input{patterns/065_GOTO/main}
\input{patterns/07_jcc/main}
\input{patterns/08_switch/main}
\input{patterns/09_loops/main}
\input{patterns/10_strings/main}
\input{patterns/11_arith_optimizations/main}
\input{patterns/12_FPU/main}
\input{patterns/13_arrays/main}
\input{patterns/14_bitfields/main}
\EN{\input{patterns/145_LCG/main_EN}}
\RU{\input{patterns/145_LCG/main_RU}}
\input{patterns/15_structs/main}
\input{patterns/17_unions/main}
\input{patterns/18_pointers_to_functions/main}
\input{patterns/185_64bit_in_32_env/main}

\EN{\input{patterns/19_SIMD/main_EN}}
\RU{\input{patterns/19_SIMD/main_RU}}
\DE{\input{patterns/19_SIMD/main_DE}}

\EN{\input{patterns/20_x64/main_EN}}
\RU{\input{patterns/20_x64/main_RU}}

\EN{\input{patterns/205_floating_SIMD/main_EN}}
\RU{\input{patterns/205_floating_SIMD/main_RU}}
\DE{\input{patterns/205_floating_SIMD/main_DE}}

\EN{\input{patterns/ARM/main_EN}}
\RU{\input{patterns/ARM/main_RU}}
\DE{\input{patterns/ARM/main_DE}}

\input{patterns/MIPS/main}

\ifdefined\SPANISH
\chapter{Patrones de código}
\fi % SPANISH

\ifdefined\GERMAN
\chapter{Code-Muster}
\fi % GERMAN

\ifdefined\ENGLISH
\chapter{Code Patterns}
\fi % ENGLISH

\ifdefined\ITALIAN
\chapter{Forme di codice}
\fi % ITALIAN

\ifdefined\RUSSIAN
\chapter{Образцы кода}
\fi % RUSSIAN

\ifdefined\BRAZILIAN
\chapter{Padrões de códigos}
\fi % BRAZILIAN

\ifdefined\THAI
\chapter{รูปแบบของโค้ด}
\fi % THAI

\ifdefined\FRENCH
\chapter{Modèle de code}
\fi % FRENCH

\ifdefined\POLISH
\chapter{\PLph{}}
\fi % POLISH

% sections
\EN{\input{patterns/patterns_opt_dbg_EN}}
\ES{\input{patterns/patterns_opt_dbg_ES}}
\ITA{\input{patterns/patterns_opt_dbg_ITA}}
\PTBR{\input{patterns/patterns_opt_dbg_PTBR}}
\RU{\input{patterns/patterns_opt_dbg_RU}}
\THA{\input{patterns/patterns_opt_dbg_THA}}
\DE{\input{patterns/patterns_opt_dbg_DE}}
\FR{\input{patterns/patterns_opt_dbg_FR}}
\PL{\input{patterns/patterns_opt_dbg_PL}}

\RU{\section{Некоторые базовые понятия}}
\EN{\section{Some basics}}
\DE{\section{Einige Grundlagen}}
\FR{\section{Quelques bases}}
\ES{\section{\ESph{}}}
\ITA{\section{Alcune basi teoriche}}
\PTBR{\section{\PTBRph{}}}
\THA{\section{\THAph{}}}
\PL{\section{\PLph{}}}

% sections:
\EN{\input{patterns/intro_CPU_ISA_EN}}
\ES{\input{patterns/intro_CPU_ISA_ES}}
\ITA{\input{patterns/intro_CPU_ISA_ITA}}
\PTBR{\input{patterns/intro_CPU_ISA_PTBR}}
\RU{\input{patterns/intro_CPU_ISA_RU}}
\DE{\input{patterns/intro_CPU_ISA_DE}}
\FR{\input{patterns/intro_CPU_ISA_FR}}
\PL{\input{patterns/intro_CPU_ISA_PL}}

\EN{\input{patterns/numeral_EN}}
\RU{\input{patterns/numeral_RU}}
\ITA{\input{patterns/numeral_ITA}}
\DE{\input{patterns/numeral_DE}}
\FR{\input{patterns/numeral_FR}}
\PL{\input{patterns/numeral_PL}}

% chapters
\input{patterns/00_empty/main}
\input{patterns/011_ret/main}
\input{patterns/01_helloworld/main}
\input{patterns/015_prolog_epilogue/main}
\input{patterns/02_stack/main}
\input{patterns/03_printf/main}
\input{patterns/04_scanf/main}
\input{patterns/05_passing_arguments/main}
\input{patterns/06_return_results/main}
\input{patterns/061_pointers/main}
\input{patterns/065_GOTO/main}
\input{patterns/07_jcc/main}
\input{patterns/08_switch/main}
\input{patterns/09_loops/main}
\input{patterns/10_strings/main}
\input{patterns/11_arith_optimizations/main}
\input{patterns/12_FPU/main}
\input{patterns/13_arrays/main}
\input{patterns/14_bitfields/main}
\EN{\input{patterns/145_LCG/main_EN}}
\RU{\input{patterns/145_LCG/main_RU}}
\input{patterns/15_structs/main}
\input{patterns/17_unions/main}
\input{patterns/18_pointers_to_functions/main}
\input{patterns/185_64bit_in_32_env/main}

\EN{\input{patterns/19_SIMD/main_EN}}
\RU{\input{patterns/19_SIMD/main_RU}}
\DE{\input{patterns/19_SIMD/main_DE}}

\EN{\input{patterns/20_x64/main_EN}}
\RU{\input{patterns/20_x64/main_RU}}

\EN{\input{patterns/205_floating_SIMD/main_EN}}
\RU{\input{patterns/205_floating_SIMD/main_RU}}
\DE{\input{patterns/205_floating_SIMD/main_DE}}

\EN{\input{patterns/ARM/main_EN}}
\RU{\input{patterns/ARM/main_RU}}
\DE{\input{patterns/ARM/main_DE}}

\input{patterns/MIPS/main}

\ifdefined\SPANISH
\chapter{Patrones de código}
\fi % SPANISH

\ifdefined\GERMAN
\chapter{Code-Muster}
\fi % GERMAN

\ifdefined\ENGLISH
\chapter{Code Patterns}
\fi % ENGLISH

\ifdefined\ITALIAN
\chapter{Forme di codice}
\fi % ITALIAN

\ifdefined\RUSSIAN
\chapter{Образцы кода}
\fi % RUSSIAN

\ifdefined\BRAZILIAN
\chapter{Padrões de códigos}
\fi % BRAZILIAN

\ifdefined\THAI
\chapter{รูปแบบของโค้ด}
\fi % THAI

\ifdefined\FRENCH
\chapter{Modèle de code}
\fi % FRENCH

\ifdefined\POLISH
\chapter{\PLph{}}
\fi % POLISH

% sections
\EN{\input{patterns/patterns_opt_dbg_EN}}
\ES{\input{patterns/patterns_opt_dbg_ES}}
\ITA{\input{patterns/patterns_opt_dbg_ITA}}
\PTBR{\input{patterns/patterns_opt_dbg_PTBR}}
\RU{\input{patterns/patterns_opt_dbg_RU}}
\THA{\input{patterns/patterns_opt_dbg_THA}}
\DE{\input{patterns/patterns_opt_dbg_DE}}
\FR{\input{patterns/patterns_opt_dbg_FR}}
\PL{\input{patterns/patterns_opt_dbg_PL}}

\RU{\section{Некоторые базовые понятия}}
\EN{\section{Some basics}}
\DE{\section{Einige Grundlagen}}
\FR{\section{Quelques bases}}
\ES{\section{\ESph{}}}
\ITA{\section{Alcune basi teoriche}}
\PTBR{\section{\PTBRph{}}}
\THA{\section{\THAph{}}}
\PL{\section{\PLph{}}}

% sections:
\EN{\input{patterns/intro_CPU_ISA_EN}}
\ES{\input{patterns/intro_CPU_ISA_ES}}
\ITA{\input{patterns/intro_CPU_ISA_ITA}}
\PTBR{\input{patterns/intro_CPU_ISA_PTBR}}
\RU{\input{patterns/intro_CPU_ISA_RU}}
\DE{\input{patterns/intro_CPU_ISA_DE}}
\FR{\input{patterns/intro_CPU_ISA_FR}}
\PL{\input{patterns/intro_CPU_ISA_PL}}

\EN{\input{patterns/numeral_EN}}
\RU{\input{patterns/numeral_RU}}
\ITA{\input{patterns/numeral_ITA}}
\DE{\input{patterns/numeral_DE}}
\FR{\input{patterns/numeral_FR}}
\PL{\input{patterns/numeral_PL}}

% chapters
\input{patterns/00_empty/main}
\input{patterns/011_ret/main}
\input{patterns/01_helloworld/main}
\input{patterns/015_prolog_epilogue/main}
\input{patterns/02_stack/main}
\input{patterns/03_printf/main}
\input{patterns/04_scanf/main}
\input{patterns/05_passing_arguments/main}
\input{patterns/06_return_results/main}
\input{patterns/061_pointers/main}
\input{patterns/065_GOTO/main}
\input{patterns/07_jcc/main}
\input{patterns/08_switch/main}
\input{patterns/09_loops/main}
\input{patterns/10_strings/main}
\input{patterns/11_arith_optimizations/main}
\input{patterns/12_FPU/main}
\input{patterns/13_arrays/main}
\input{patterns/14_bitfields/main}
\EN{\input{patterns/145_LCG/main_EN}}
\RU{\input{patterns/145_LCG/main_RU}}
\input{patterns/15_structs/main}
\input{patterns/17_unions/main}
\input{patterns/18_pointers_to_functions/main}
\input{patterns/185_64bit_in_32_env/main}

\EN{\input{patterns/19_SIMD/main_EN}}
\RU{\input{patterns/19_SIMD/main_RU}}
\DE{\input{patterns/19_SIMD/main_DE}}

\EN{\input{patterns/20_x64/main_EN}}
\RU{\input{patterns/20_x64/main_RU}}

\EN{\input{patterns/205_floating_SIMD/main_EN}}
\RU{\input{patterns/205_floating_SIMD/main_RU}}
\DE{\input{patterns/205_floating_SIMD/main_DE}}

\EN{\input{patterns/ARM/main_EN}}
\RU{\input{patterns/ARM/main_RU}}
\DE{\input{patterns/ARM/main_DE}}

\input{patterns/MIPS/main}

\ifdefined\SPANISH
\chapter{Patrones de código}
\fi % SPANISH

\ifdefined\GERMAN
\chapter{Code-Muster}
\fi % GERMAN

\ifdefined\ENGLISH
\chapter{Code Patterns}
\fi % ENGLISH

\ifdefined\ITALIAN
\chapter{Forme di codice}
\fi % ITALIAN

\ifdefined\RUSSIAN
\chapter{Образцы кода}
\fi % RUSSIAN

\ifdefined\BRAZILIAN
\chapter{Padrões de códigos}
\fi % BRAZILIAN

\ifdefined\THAI
\chapter{รูปแบบของโค้ด}
\fi % THAI

\ifdefined\FRENCH
\chapter{Modèle de code}
\fi % FRENCH

\ifdefined\POLISH
\chapter{\PLph{}}
\fi % POLISH

% sections
\EN{\input{patterns/patterns_opt_dbg_EN}}
\ES{\input{patterns/patterns_opt_dbg_ES}}
\ITA{\input{patterns/patterns_opt_dbg_ITA}}
\PTBR{\input{patterns/patterns_opt_dbg_PTBR}}
\RU{\input{patterns/patterns_opt_dbg_RU}}
\THA{\input{patterns/patterns_opt_dbg_THA}}
\DE{\input{patterns/patterns_opt_dbg_DE}}
\FR{\input{patterns/patterns_opt_dbg_FR}}
\PL{\input{patterns/patterns_opt_dbg_PL}}

\RU{\section{Некоторые базовые понятия}}
\EN{\section{Some basics}}
\DE{\section{Einige Grundlagen}}
\FR{\section{Quelques bases}}
\ES{\section{\ESph{}}}
\ITA{\section{Alcune basi teoriche}}
\PTBR{\section{\PTBRph{}}}
\THA{\section{\THAph{}}}
\PL{\section{\PLph{}}}

% sections:
\EN{\input{patterns/intro_CPU_ISA_EN}}
\ES{\input{patterns/intro_CPU_ISA_ES}}
\ITA{\input{patterns/intro_CPU_ISA_ITA}}
\PTBR{\input{patterns/intro_CPU_ISA_PTBR}}
\RU{\input{patterns/intro_CPU_ISA_RU}}
\DE{\input{patterns/intro_CPU_ISA_DE}}
\FR{\input{patterns/intro_CPU_ISA_FR}}
\PL{\input{patterns/intro_CPU_ISA_PL}}

\EN{\input{patterns/numeral_EN}}
\RU{\input{patterns/numeral_RU}}
\ITA{\input{patterns/numeral_ITA}}
\DE{\input{patterns/numeral_DE}}
\FR{\input{patterns/numeral_FR}}
\PL{\input{patterns/numeral_PL}}

% chapters
\input{patterns/00_empty/main}
\input{patterns/011_ret/main}
\input{patterns/01_helloworld/main}
\input{patterns/015_prolog_epilogue/main}
\input{patterns/02_stack/main}
\input{patterns/03_printf/main}
\input{patterns/04_scanf/main}
\input{patterns/05_passing_arguments/main}
\input{patterns/06_return_results/main}
\input{patterns/061_pointers/main}
\input{patterns/065_GOTO/main}
\input{patterns/07_jcc/main}
\input{patterns/08_switch/main}
\input{patterns/09_loops/main}
\input{patterns/10_strings/main}
\input{patterns/11_arith_optimizations/main}
\input{patterns/12_FPU/main}
\input{patterns/13_arrays/main}
\input{patterns/14_bitfields/main}
\EN{\input{patterns/145_LCG/main_EN}}
\RU{\input{patterns/145_LCG/main_RU}}
\input{patterns/15_structs/main}
\input{patterns/17_unions/main}
\input{patterns/18_pointers_to_functions/main}
\input{patterns/185_64bit_in_32_env/main}

\EN{\input{patterns/19_SIMD/main_EN}}
\RU{\input{patterns/19_SIMD/main_RU}}
\DE{\input{patterns/19_SIMD/main_DE}}

\EN{\input{patterns/20_x64/main_EN}}
\RU{\input{patterns/20_x64/main_RU}}

\EN{\input{patterns/205_floating_SIMD/main_EN}}
\RU{\input{patterns/205_floating_SIMD/main_RU}}
\DE{\input{patterns/205_floating_SIMD/main_DE}}

\EN{\input{patterns/ARM/main_EN}}
\RU{\input{patterns/ARM/main_RU}}
\DE{\input{patterns/ARM/main_DE}}

\input{patterns/MIPS/main}

\ifdefined\SPANISH
\chapter{Patrones de código}
\fi % SPANISH

\ifdefined\GERMAN
\chapter{Code-Muster}
\fi % GERMAN

\ifdefined\ENGLISH
\chapter{Code Patterns}
\fi % ENGLISH

\ifdefined\ITALIAN
\chapter{Forme di codice}
\fi % ITALIAN

\ifdefined\RUSSIAN
\chapter{Образцы кода}
\fi % RUSSIAN

\ifdefined\BRAZILIAN
\chapter{Padrões de códigos}
\fi % BRAZILIAN

\ifdefined\THAI
\chapter{รูปแบบของโค้ด}
\fi % THAI

\ifdefined\FRENCH
\chapter{Modèle de code}
\fi % FRENCH

\ifdefined\POLISH
\chapter{\PLph{}}
\fi % POLISH

% sections
\EN{\input{patterns/patterns_opt_dbg_EN}}
\ES{\input{patterns/patterns_opt_dbg_ES}}
\ITA{\input{patterns/patterns_opt_dbg_ITA}}
\PTBR{\input{patterns/patterns_opt_dbg_PTBR}}
\RU{\input{patterns/patterns_opt_dbg_RU}}
\THA{\input{patterns/patterns_opt_dbg_THA}}
\DE{\input{patterns/patterns_opt_dbg_DE}}
\FR{\input{patterns/patterns_opt_dbg_FR}}
\PL{\input{patterns/patterns_opt_dbg_PL}}

\RU{\section{Некоторые базовые понятия}}
\EN{\section{Some basics}}
\DE{\section{Einige Grundlagen}}
\FR{\section{Quelques bases}}
\ES{\section{\ESph{}}}
\ITA{\section{Alcune basi teoriche}}
\PTBR{\section{\PTBRph{}}}
\THA{\section{\THAph{}}}
\PL{\section{\PLph{}}}

% sections:
\EN{\input{patterns/intro_CPU_ISA_EN}}
\ES{\input{patterns/intro_CPU_ISA_ES}}
\ITA{\input{patterns/intro_CPU_ISA_ITA}}
\PTBR{\input{patterns/intro_CPU_ISA_PTBR}}
\RU{\input{patterns/intro_CPU_ISA_RU}}
\DE{\input{patterns/intro_CPU_ISA_DE}}
\FR{\input{patterns/intro_CPU_ISA_FR}}
\PL{\input{patterns/intro_CPU_ISA_PL}}

\EN{\input{patterns/numeral_EN}}
\RU{\input{patterns/numeral_RU}}
\ITA{\input{patterns/numeral_ITA}}
\DE{\input{patterns/numeral_DE}}
\FR{\input{patterns/numeral_FR}}
\PL{\input{patterns/numeral_PL}}

% chapters
\input{patterns/00_empty/main}
\input{patterns/011_ret/main}
\input{patterns/01_helloworld/main}
\input{patterns/015_prolog_epilogue/main}
\input{patterns/02_stack/main}
\input{patterns/03_printf/main}
\input{patterns/04_scanf/main}
\input{patterns/05_passing_arguments/main}
\input{patterns/06_return_results/main}
\input{patterns/061_pointers/main}
\input{patterns/065_GOTO/main}
\input{patterns/07_jcc/main}
\input{patterns/08_switch/main}
\input{patterns/09_loops/main}
\input{patterns/10_strings/main}
\input{patterns/11_arith_optimizations/main}
\input{patterns/12_FPU/main}
\input{patterns/13_arrays/main}
\input{patterns/14_bitfields/main}
\EN{\input{patterns/145_LCG/main_EN}}
\RU{\input{patterns/145_LCG/main_RU}}
\input{patterns/15_structs/main}
\input{patterns/17_unions/main}
\input{patterns/18_pointers_to_functions/main}
\input{patterns/185_64bit_in_32_env/main}

\EN{\input{patterns/19_SIMD/main_EN}}
\RU{\input{patterns/19_SIMD/main_RU}}
\DE{\input{patterns/19_SIMD/main_DE}}

\EN{\input{patterns/20_x64/main_EN}}
\RU{\input{patterns/20_x64/main_RU}}

\EN{\input{patterns/205_floating_SIMD/main_EN}}
\RU{\input{patterns/205_floating_SIMD/main_RU}}
\DE{\input{patterns/205_floating_SIMD/main_DE}}

\EN{\input{patterns/ARM/main_EN}}
\RU{\input{patterns/ARM/main_RU}}
\DE{\input{patterns/ARM/main_DE}}

\input{patterns/MIPS/main}

\ifdefined\SPANISH
\chapter{Patrones de código}
\fi % SPANISH

\ifdefined\GERMAN
\chapter{Code-Muster}
\fi % GERMAN

\ifdefined\ENGLISH
\chapter{Code Patterns}
\fi % ENGLISH

\ifdefined\ITALIAN
\chapter{Forme di codice}
\fi % ITALIAN

\ifdefined\RUSSIAN
\chapter{Образцы кода}
\fi % RUSSIAN

\ifdefined\BRAZILIAN
\chapter{Padrões de códigos}
\fi % BRAZILIAN

\ifdefined\THAI
\chapter{รูปแบบของโค้ด}
\fi % THAI

\ifdefined\FRENCH
\chapter{Modèle de code}
\fi % FRENCH

\ifdefined\POLISH
\chapter{\PLph{}}
\fi % POLISH

% sections
\EN{\input{patterns/patterns_opt_dbg_EN}}
\ES{\input{patterns/patterns_opt_dbg_ES}}
\ITA{\input{patterns/patterns_opt_dbg_ITA}}
\PTBR{\input{patterns/patterns_opt_dbg_PTBR}}
\RU{\input{patterns/patterns_opt_dbg_RU}}
\THA{\input{patterns/patterns_opt_dbg_THA}}
\DE{\input{patterns/patterns_opt_dbg_DE}}
\FR{\input{patterns/patterns_opt_dbg_FR}}
\PL{\input{patterns/patterns_opt_dbg_PL}}

\RU{\section{Некоторые базовые понятия}}
\EN{\section{Some basics}}
\DE{\section{Einige Grundlagen}}
\FR{\section{Quelques bases}}
\ES{\section{\ESph{}}}
\ITA{\section{Alcune basi teoriche}}
\PTBR{\section{\PTBRph{}}}
\THA{\section{\THAph{}}}
\PL{\section{\PLph{}}}

% sections:
\EN{\input{patterns/intro_CPU_ISA_EN}}
\ES{\input{patterns/intro_CPU_ISA_ES}}
\ITA{\input{patterns/intro_CPU_ISA_ITA}}
\PTBR{\input{patterns/intro_CPU_ISA_PTBR}}
\RU{\input{patterns/intro_CPU_ISA_RU}}
\DE{\input{patterns/intro_CPU_ISA_DE}}
\FR{\input{patterns/intro_CPU_ISA_FR}}
\PL{\input{patterns/intro_CPU_ISA_PL}}

\EN{\input{patterns/numeral_EN}}
\RU{\input{patterns/numeral_RU}}
\ITA{\input{patterns/numeral_ITA}}
\DE{\input{patterns/numeral_DE}}
\FR{\input{patterns/numeral_FR}}
\PL{\input{patterns/numeral_PL}}

% chapters
\input{patterns/00_empty/main}
\input{patterns/011_ret/main}
\input{patterns/01_helloworld/main}
\input{patterns/015_prolog_epilogue/main}
\input{patterns/02_stack/main}
\input{patterns/03_printf/main}
\input{patterns/04_scanf/main}
\input{patterns/05_passing_arguments/main}
\input{patterns/06_return_results/main}
\input{patterns/061_pointers/main}
\input{patterns/065_GOTO/main}
\input{patterns/07_jcc/main}
\input{patterns/08_switch/main}
\input{patterns/09_loops/main}
\input{patterns/10_strings/main}
\input{patterns/11_arith_optimizations/main}
\input{patterns/12_FPU/main}
\input{patterns/13_arrays/main}
\input{patterns/14_bitfields/main}
\EN{\input{patterns/145_LCG/main_EN}}
\RU{\input{patterns/145_LCG/main_RU}}
\input{patterns/15_structs/main}
\input{patterns/17_unions/main}
\input{patterns/18_pointers_to_functions/main}
\input{patterns/185_64bit_in_32_env/main}

\EN{\input{patterns/19_SIMD/main_EN}}
\RU{\input{patterns/19_SIMD/main_RU}}
\DE{\input{patterns/19_SIMD/main_DE}}

\EN{\input{patterns/20_x64/main_EN}}
\RU{\input{patterns/20_x64/main_RU}}

\EN{\input{patterns/205_floating_SIMD/main_EN}}
\RU{\input{patterns/205_floating_SIMD/main_RU}}
\DE{\input{patterns/205_floating_SIMD/main_DE}}

\EN{\input{patterns/ARM/main_EN}}
\RU{\input{patterns/ARM/main_RU}}
\DE{\input{patterns/ARM/main_DE}}

\input{patterns/MIPS/main}

\ifdefined\SPANISH
\chapter{Patrones de código}
\fi % SPANISH

\ifdefined\GERMAN
\chapter{Code-Muster}
\fi % GERMAN

\ifdefined\ENGLISH
\chapter{Code Patterns}
\fi % ENGLISH

\ifdefined\ITALIAN
\chapter{Forme di codice}
\fi % ITALIAN

\ifdefined\RUSSIAN
\chapter{Образцы кода}
\fi % RUSSIAN

\ifdefined\BRAZILIAN
\chapter{Padrões de códigos}
\fi % BRAZILIAN

\ifdefined\THAI
\chapter{รูปแบบของโค้ด}
\fi % THAI

\ifdefined\FRENCH
\chapter{Modèle de code}
\fi % FRENCH

\ifdefined\POLISH
\chapter{\PLph{}}
\fi % POLISH

% sections
\EN{\input{patterns/patterns_opt_dbg_EN}}
\ES{\input{patterns/patterns_opt_dbg_ES}}
\ITA{\input{patterns/patterns_opt_dbg_ITA}}
\PTBR{\input{patterns/patterns_opt_dbg_PTBR}}
\RU{\input{patterns/patterns_opt_dbg_RU}}
\THA{\input{patterns/patterns_opt_dbg_THA}}
\DE{\input{patterns/patterns_opt_dbg_DE}}
\FR{\input{patterns/patterns_opt_dbg_FR}}
\PL{\input{patterns/patterns_opt_dbg_PL}}

\RU{\section{Некоторые базовые понятия}}
\EN{\section{Some basics}}
\DE{\section{Einige Grundlagen}}
\FR{\section{Quelques bases}}
\ES{\section{\ESph{}}}
\ITA{\section{Alcune basi teoriche}}
\PTBR{\section{\PTBRph{}}}
\THA{\section{\THAph{}}}
\PL{\section{\PLph{}}}

% sections:
\EN{\input{patterns/intro_CPU_ISA_EN}}
\ES{\input{patterns/intro_CPU_ISA_ES}}
\ITA{\input{patterns/intro_CPU_ISA_ITA}}
\PTBR{\input{patterns/intro_CPU_ISA_PTBR}}
\RU{\input{patterns/intro_CPU_ISA_RU}}
\DE{\input{patterns/intro_CPU_ISA_DE}}
\FR{\input{patterns/intro_CPU_ISA_FR}}
\PL{\input{patterns/intro_CPU_ISA_PL}}

\EN{\input{patterns/numeral_EN}}
\RU{\input{patterns/numeral_RU}}
\ITA{\input{patterns/numeral_ITA}}
\DE{\input{patterns/numeral_DE}}
\FR{\input{patterns/numeral_FR}}
\PL{\input{patterns/numeral_PL}}

% chapters
\input{patterns/00_empty/main}
\input{patterns/011_ret/main}
\input{patterns/01_helloworld/main}
\input{patterns/015_prolog_epilogue/main}
\input{patterns/02_stack/main}
\input{patterns/03_printf/main}
\input{patterns/04_scanf/main}
\input{patterns/05_passing_arguments/main}
\input{patterns/06_return_results/main}
\input{patterns/061_pointers/main}
\input{patterns/065_GOTO/main}
\input{patterns/07_jcc/main}
\input{patterns/08_switch/main}
\input{patterns/09_loops/main}
\input{patterns/10_strings/main}
\input{patterns/11_arith_optimizations/main}
\input{patterns/12_FPU/main}
\input{patterns/13_arrays/main}
\input{patterns/14_bitfields/main}
\EN{\input{patterns/145_LCG/main_EN}}
\RU{\input{patterns/145_LCG/main_RU}}
\input{patterns/15_structs/main}
\input{patterns/17_unions/main}
\input{patterns/18_pointers_to_functions/main}
\input{patterns/185_64bit_in_32_env/main}

\EN{\input{patterns/19_SIMD/main_EN}}
\RU{\input{patterns/19_SIMD/main_RU}}
\DE{\input{patterns/19_SIMD/main_DE}}

\EN{\input{patterns/20_x64/main_EN}}
\RU{\input{patterns/20_x64/main_RU}}

\EN{\input{patterns/205_floating_SIMD/main_EN}}
\RU{\input{patterns/205_floating_SIMD/main_RU}}
\DE{\input{patterns/205_floating_SIMD/main_DE}}

\EN{\input{patterns/ARM/main_EN}}
\RU{\input{patterns/ARM/main_RU}}
\DE{\input{patterns/ARM/main_DE}}

\input{patterns/MIPS/main}

\EN{\input{patterns/12_FPU/main_EN}}
\RU{\input{patterns/12_FPU/main_RU}}
\DE{\input{patterns/12_FPU/main_DE}}
\FR{\input{patterns/12_FPU/main_FR}}


\ifdefined\SPANISH
\chapter{Patrones de código}
\fi % SPANISH

\ifdefined\GERMAN
\chapter{Code-Muster}
\fi % GERMAN

\ifdefined\ENGLISH
\chapter{Code Patterns}
\fi % ENGLISH

\ifdefined\ITALIAN
\chapter{Forme di codice}
\fi % ITALIAN

\ifdefined\RUSSIAN
\chapter{Образцы кода}
\fi % RUSSIAN

\ifdefined\BRAZILIAN
\chapter{Padrões de códigos}
\fi % BRAZILIAN

\ifdefined\THAI
\chapter{รูปแบบของโค้ด}
\fi % THAI

\ifdefined\FRENCH
\chapter{Modèle de code}
\fi % FRENCH

\ifdefined\POLISH
\chapter{\PLph{}}
\fi % POLISH

% sections
\EN{\input{patterns/patterns_opt_dbg_EN}}
\ES{\input{patterns/patterns_opt_dbg_ES}}
\ITA{\input{patterns/patterns_opt_dbg_ITA}}
\PTBR{\input{patterns/patterns_opt_dbg_PTBR}}
\RU{\input{patterns/patterns_opt_dbg_RU}}
\THA{\input{patterns/patterns_opt_dbg_THA}}
\DE{\input{patterns/patterns_opt_dbg_DE}}
\FR{\input{patterns/patterns_opt_dbg_FR}}
\PL{\input{patterns/patterns_opt_dbg_PL}}

\RU{\section{Некоторые базовые понятия}}
\EN{\section{Some basics}}
\DE{\section{Einige Grundlagen}}
\FR{\section{Quelques bases}}
\ES{\section{\ESph{}}}
\ITA{\section{Alcune basi teoriche}}
\PTBR{\section{\PTBRph{}}}
\THA{\section{\THAph{}}}
\PL{\section{\PLph{}}}

% sections:
\EN{\input{patterns/intro_CPU_ISA_EN}}
\ES{\input{patterns/intro_CPU_ISA_ES}}
\ITA{\input{patterns/intro_CPU_ISA_ITA}}
\PTBR{\input{patterns/intro_CPU_ISA_PTBR}}
\RU{\input{patterns/intro_CPU_ISA_RU}}
\DE{\input{patterns/intro_CPU_ISA_DE}}
\FR{\input{patterns/intro_CPU_ISA_FR}}
\PL{\input{patterns/intro_CPU_ISA_PL}}

\EN{\input{patterns/numeral_EN}}
\RU{\input{patterns/numeral_RU}}
\ITA{\input{patterns/numeral_ITA}}
\DE{\input{patterns/numeral_DE}}
\FR{\input{patterns/numeral_FR}}
\PL{\input{patterns/numeral_PL}}

% chapters
\input{patterns/00_empty/main}
\input{patterns/011_ret/main}
\input{patterns/01_helloworld/main}
\input{patterns/015_prolog_epilogue/main}
\input{patterns/02_stack/main}
\input{patterns/03_printf/main}
\input{patterns/04_scanf/main}
\input{patterns/05_passing_arguments/main}
\input{patterns/06_return_results/main}
\input{patterns/061_pointers/main}
\input{patterns/065_GOTO/main}
\input{patterns/07_jcc/main}
\input{patterns/08_switch/main}
\input{patterns/09_loops/main}
\input{patterns/10_strings/main}
\input{patterns/11_arith_optimizations/main}
\input{patterns/12_FPU/main}
\input{patterns/13_arrays/main}
\input{patterns/14_bitfields/main}
\EN{\input{patterns/145_LCG/main_EN}}
\RU{\input{patterns/145_LCG/main_RU}}
\input{patterns/15_structs/main}
\input{patterns/17_unions/main}
\input{patterns/18_pointers_to_functions/main}
\input{patterns/185_64bit_in_32_env/main}

\EN{\input{patterns/19_SIMD/main_EN}}
\RU{\input{patterns/19_SIMD/main_RU}}
\DE{\input{patterns/19_SIMD/main_DE}}

\EN{\input{patterns/20_x64/main_EN}}
\RU{\input{patterns/20_x64/main_RU}}

\EN{\input{patterns/205_floating_SIMD/main_EN}}
\RU{\input{patterns/205_floating_SIMD/main_RU}}
\DE{\input{patterns/205_floating_SIMD/main_DE}}

\EN{\input{patterns/ARM/main_EN}}
\RU{\input{patterns/ARM/main_RU}}
\DE{\input{patterns/ARM/main_DE}}

\input{patterns/MIPS/main}

\ifdefined\SPANISH
\chapter{Patrones de código}
\fi % SPANISH

\ifdefined\GERMAN
\chapter{Code-Muster}
\fi % GERMAN

\ifdefined\ENGLISH
\chapter{Code Patterns}
\fi % ENGLISH

\ifdefined\ITALIAN
\chapter{Forme di codice}
\fi % ITALIAN

\ifdefined\RUSSIAN
\chapter{Образцы кода}
\fi % RUSSIAN

\ifdefined\BRAZILIAN
\chapter{Padrões de códigos}
\fi % BRAZILIAN

\ifdefined\THAI
\chapter{รูปแบบของโค้ด}
\fi % THAI

\ifdefined\FRENCH
\chapter{Modèle de code}
\fi % FRENCH

\ifdefined\POLISH
\chapter{\PLph{}}
\fi % POLISH

% sections
\EN{\input{patterns/patterns_opt_dbg_EN}}
\ES{\input{patterns/patterns_opt_dbg_ES}}
\ITA{\input{patterns/patterns_opt_dbg_ITA}}
\PTBR{\input{patterns/patterns_opt_dbg_PTBR}}
\RU{\input{patterns/patterns_opt_dbg_RU}}
\THA{\input{patterns/patterns_opt_dbg_THA}}
\DE{\input{patterns/patterns_opt_dbg_DE}}
\FR{\input{patterns/patterns_opt_dbg_FR}}
\PL{\input{patterns/patterns_opt_dbg_PL}}

\RU{\section{Некоторые базовые понятия}}
\EN{\section{Some basics}}
\DE{\section{Einige Grundlagen}}
\FR{\section{Quelques bases}}
\ES{\section{\ESph{}}}
\ITA{\section{Alcune basi teoriche}}
\PTBR{\section{\PTBRph{}}}
\THA{\section{\THAph{}}}
\PL{\section{\PLph{}}}

% sections:
\EN{\input{patterns/intro_CPU_ISA_EN}}
\ES{\input{patterns/intro_CPU_ISA_ES}}
\ITA{\input{patterns/intro_CPU_ISA_ITA}}
\PTBR{\input{patterns/intro_CPU_ISA_PTBR}}
\RU{\input{patterns/intro_CPU_ISA_RU}}
\DE{\input{patterns/intro_CPU_ISA_DE}}
\FR{\input{patterns/intro_CPU_ISA_FR}}
\PL{\input{patterns/intro_CPU_ISA_PL}}

\EN{\input{patterns/numeral_EN}}
\RU{\input{patterns/numeral_RU}}
\ITA{\input{patterns/numeral_ITA}}
\DE{\input{patterns/numeral_DE}}
\FR{\input{patterns/numeral_FR}}
\PL{\input{patterns/numeral_PL}}

% chapters
\input{patterns/00_empty/main}
\input{patterns/011_ret/main}
\input{patterns/01_helloworld/main}
\input{patterns/015_prolog_epilogue/main}
\input{patterns/02_stack/main}
\input{patterns/03_printf/main}
\input{patterns/04_scanf/main}
\input{patterns/05_passing_arguments/main}
\input{patterns/06_return_results/main}
\input{patterns/061_pointers/main}
\input{patterns/065_GOTO/main}
\input{patterns/07_jcc/main}
\input{patterns/08_switch/main}
\input{patterns/09_loops/main}
\input{patterns/10_strings/main}
\input{patterns/11_arith_optimizations/main}
\input{patterns/12_FPU/main}
\input{patterns/13_arrays/main}
\input{patterns/14_bitfields/main}
\EN{\input{patterns/145_LCG/main_EN}}
\RU{\input{patterns/145_LCG/main_RU}}
\input{patterns/15_structs/main}
\input{patterns/17_unions/main}
\input{patterns/18_pointers_to_functions/main}
\input{patterns/185_64bit_in_32_env/main}

\EN{\input{patterns/19_SIMD/main_EN}}
\RU{\input{patterns/19_SIMD/main_RU}}
\DE{\input{patterns/19_SIMD/main_DE}}

\EN{\input{patterns/20_x64/main_EN}}
\RU{\input{patterns/20_x64/main_RU}}

\EN{\input{patterns/205_floating_SIMD/main_EN}}
\RU{\input{patterns/205_floating_SIMD/main_RU}}
\DE{\input{patterns/205_floating_SIMD/main_DE}}

\EN{\input{patterns/ARM/main_EN}}
\RU{\input{patterns/ARM/main_RU}}
\DE{\input{patterns/ARM/main_DE}}

\input{patterns/MIPS/main}

\EN{\section{Returning Values}
\label{ret_val_func}

Another simple function is the one that simply returns a constant value:

\lstinputlisting[caption=\EN{\CCpp Code},style=customc]{patterns/011_ret/1.c}

Let's compile it.

\subsection{x86}

Here's what both the GCC and MSVC compilers produce (with optimization) on the x86 platform:

\lstinputlisting[caption=\Optimizing GCC/MSVC (\assemblyOutput),style=customasmx86]{patterns/011_ret/1.s}

\myindex{x86!\Instructions!RET}
There are just two instructions: the first places the value 123 into the \EAX register,
which is used by convention for storing the return
value, and the second one is \RET, which returns execution to the \gls{caller}.

The caller will take the result from the \EAX register.

\subsection{ARM}

There are a few differences on the ARM platform:

\lstinputlisting[caption=\OptimizingKeilVI (\ARMMode) ASM Output,style=customasmARM]{patterns/011_ret/1_Keil_ARM_O3.s}

ARM uses the register \Reg{0} for returning the results of functions, so 123 is copied into \Reg{0}.

\myindex{ARM!\Instructions!MOV}
\myindex{x86!\Instructions!MOV}
It is worth noting that \MOV is a misleading name for the instruction in both the x86 and ARM \ac{ISA}s.

The data is not in fact \IT{moved}, but \IT{copied}.

\subsection{MIPS}

\label{MIPS_leaf_function_ex1}

The GCC assembly output below lists registers by number:

\lstinputlisting[caption=\Optimizing GCC 4.4.5 (\assemblyOutput),style=customasmMIPS]{patterns/011_ret/MIPS.s}

\dots while \IDA does it by their pseudo names:

\lstinputlisting[caption=\Optimizing GCC 4.4.5 (IDA),style=customasmMIPS]{patterns/011_ret/MIPS_IDA.lst}

The \$2 (or \$V0) register is used to store the function's return value.
\myindex{MIPS!\Pseudoinstructions!LI}
\INS{LI} stands for ``Load Immediate'' and is the MIPS equivalent to \MOV.

\myindex{MIPS!\Instructions!J}
The other instruction is the jump instruction (J or JR) which returns the execution flow to the \gls{caller}.

\myindex{MIPS!Branch delay slot}
You might be wondering why the positions of the load instruction (LI) and the jump instruction (J or JR) are swapped. This is due to a \ac{RISC} feature called ``branch delay slot''.

The reason this happens is a quirk in the architecture of some RISC \ac{ISA}s and isn't important for our
purposes---we must simply keep in mind that in MIPS, the instruction following a jump or branch instruction
is executed \IT{before} the jump/branch instruction itself.

As a consequence, branch instructions always swap places with the instruction executed immediately beforehand.


In practice, functions which merely return 1 (\IT{true}) or 0 (\IT{false}) are very frequent.

The smallest ever of the standard UNIX utilities, \IT{/bin/true} and \IT{/bin/false} return 0 and 1 respectively, as an exit code.
(Zero as an exit code usually means success, non-zero means error.)
}
\RU{\subsubsection{std::string}
\myindex{\Cpp!STL!std::string}
\label{std_string}

\myparagraph{Как устроена структура}

Многие строковые библиотеки \InSqBrackets{\CNotes 2.2} обеспечивают структуру содержащую ссылку 
на буфер собственно со строкой, переменная всегда содержащую длину строки 
(что очень удобно для массы функций \InSqBrackets{\CNotes 2.2.1}) и переменную содержащую текущий размер буфера.

Строка в буфере обыкновенно оканчивается нулем: это для того чтобы указатель на буфер можно было
передавать в функции требующие на вход обычную сишную \ac{ASCIIZ}-строку.

Стандарт \Cpp не описывает, как именно нужно реализовывать std::string,
но, как правило, они реализованы как описано выше, с небольшими дополнениями.

Строки в \Cpp это не класс (как, например, QString в Qt), а темплейт (basic\_string), 
это сделано для того чтобы поддерживать 
строки содержащие разного типа символы: как минимум \Tchar и \IT{wchar\_t}.

Так что, std::string это класс с базовым типом \Tchar.

А std::wstring это класс с базовым типом \IT{wchar\_t}.

\mysubparagraph{MSVC}

В реализации MSVC, вместо ссылки на буфер может содержаться сам буфер (если строка короче 16-и символов).

Это означает, что каждая короткая строка будет занимать в памяти по крайней мере $16 + 4 + 4 = 24$ 
байт для 32-битной среды либо $16 + 8 + 8 = 32$ 
байта в 64-битной, а если строка длиннее 16-и символов, то прибавьте еще длину самой строки.

\lstinputlisting[caption=пример для MSVC,style=customc]{\CURPATH/STL/string/MSVC_RU.cpp}

Собственно, из этого исходника почти всё ясно.

Несколько замечаний:

Если строка короче 16-и символов, 
то отдельный буфер для строки в \glslink{heap}{куче} выделяться не будет.

Это удобно потому что на практике, основная часть строк действительно короткие.
Вероятно, разработчики в Microsoft выбрали размер в 16 символов как разумный баланс.

Теперь очень важный момент в конце функции main(): мы не пользуемся методом c\_str(), тем не менее,
если это скомпилировать и запустить, то обе строки появятся в консоли!

Работает это вот почему.

В первом случае строка короче 16-и символов и в начале объекта std::string (его можно рассматривать
просто как структуру) расположен буфер с этой строкой.
\printf трактует указатель как указатель на массив символов оканчивающийся нулем и поэтому всё работает.

Вывод второй строки (длиннее 16-и символов) даже еще опаснее: это вообще типичная программистская ошибка 
(или опечатка), забыть дописать c\_str().
Это работает потому что в это время в начале структуры расположен указатель на буфер.
Это может надолго остаться незамеченным: до тех пока там не появится строка 
короче 16-и символов, тогда процесс упадет.

\mysubparagraph{GCC}

В реализации GCC в структуре есть еще одна переменная --- reference count.

Интересно, что указатель на экземпляр класса std::string в GCC указывает не на начало самой структуры, 
а на указатель на буфера.
В libstdc++-v3\textbackslash{}include\textbackslash{}bits\textbackslash{}basic\_string.h 
мы можем прочитать что это сделано для удобства отладки:

\begin{lstlisting}
   *  The reason you want _M_data pointing to the character %array and
   *  not the _Rep is so that the debugger can see the string
   *  contents. (Probably we should add a non-inline member to get
   *  the _Rep for the debugger to use, so users can check the actual
   *  string length.)
\end{lstlisting}

\href{http://go.yurichev.com/17085}{исходный код basic\_string.h}

В нашем примере мы учитываем это:

\lstinputlisting[caption=пример для GCC,style=customc]{\CURPATH/STL/string/GCC_RU.cpp}

Нужны еще небольшие хаки чтобы сымитировать типичную ошибку, которую мы уже видели выше, из-за
более ужесточенной проверки типов в GCC, тем не менее, printf() работает и здесь без c\_str().

\myparagraph{Чуть более сложный пример}

\lstinputlisting[style=customc]{\CURPATH/STL/string/3.cpp}

\lstinputlisting[caption=MSVC 2012,style=customasmx86]{\CURPATH/STL/string/3_MSVC_RU.asm}

Собственно, компилятор не конструирует строки статически: да в общем-то и как
это возможно, если буфер с ней нужно хранить в \glslink{heap}{куче}?

Вместо этого в сегменте данных хранятся обычные \ac{ASCIIZ}-строки, а позже, во время выполнения, 
при помощи метода \q{assign}, конструируются строки s1 и s2
.
При помощи \TT{operator+}, создается строка s3.

Обратите внимание на то что вызов метода c\_str() отсутствует,
потому что его код достаточно короткий и компилятор вставил его прямо здесь:
если строка короче 16-и байт, то в регистре EAX остается указатель на буфер,
а если длиннее, то из этого же места достается адрес на буфер расположенный в \glslink{heap}{куче}.

Далее следуют вызовы трех деструкторов, причем, они вызываются только если строка длиннее 16-и байт:
тогда нужно освободить буфера в \glslink{heap}{куче}.
В противном случае, так как все три объекта std::string хранятся в стеке,
они освобождаются автоматически после выхода из функции.

Следовательно, работа с короткими строками более быстрая из-за м\'{е}ньшего обращения к \glslink{heap}{куче}.

Код на GCC даже проще (из-за того, что в GCC, как мы уже видели, не реализована возможность хранить короткую
строку прямо в структуре):

% TODO1 comment each function meaning
\lstinputlisting[caption=GCC 4.8.1,style=customasmx86]{\CURPATH/STL/string/3_GCC_RU.s}

Можно заметить, что в деструкторы передается не указатель на объект,
а указатель на место за 12 байт (или 3 слова) перед ним, то есть, на настоящее начало структуры.

\myparagraph{std::string как глобальная переменная}
\label{sec:std_string_as_global_variable}

Опытные программисты на \Cpp знают, что глобальные переменные \ac{STL}-типов вполне можно объявлять.

Да, действительно:

\lstinputlisting[style=customc]{\CURPATH/STL/string/5.cpp}

Но как и где будет вызываться конструктор \TT{std::string}?

На самом деле, эта переменная будет инициализирована даже перед началом \main.

\lstinputlisting[caption=MSVC 2012: здесь конструируется глобальная переменная{,} а также регистрируется её деструктор,style=customasmx86]{\CURPATH/STL/string/5_MSVC_p2.asm}

\lstinputlisting[caption=MSVC 2012: здесь глобальная переменная используется в \main,style=customasmx86]{\CURPATH/STL/string/5_MSVC_p1.asm}

\lstinputlisting[caption=MSVC 2012: эта функция-деструктор вызывается перед выходом,style=customasmx86]{\CURPATH/STL/string/5_MSVC_p3.asm}

\myindex{\CStandardLibrary!atexit()}
В реальности, из \ac{CRT}, еще до вызова main(), вызывается специальная функция,
в которой перечислены все конструкторы подобных переменных.
Более того: при помощи atexit() регистрируется функция, которая будет вызвана в конце работы программы:
в этой функции компилятор собирает вызовы деструкторов всех подобных глобальных переменных.

GCC работает похожим образом:

\lstinputlisting[caption=GCC 4.8.1,style=customasmx86]{\CURPATH/STL/string/5_GCC.s}

Но он не выделяет отдельной функции в которой будут собраны деструкторы: 
каждый деструктор передается в atexit() по одному.

% TODO а если глобальная STL-переменная в другом модуле? надо проверить.

}
\ifdefined\SPANISH
\chapter{Patrones de código}
\fi % SPANISH

\ifdefined\GERMAN
\chapter{Code-Muster}
\fi % GERMAN

\ifdefined\ENGLISH
\chapter{Code Patterns}
\fi % ENGLISH

\ifdefined\ITALIAN
\chapter{Forme di codice}
\fi % ITALIAN

\ifdefined\RUSSIAN
\chapter{Образцы кода}
\fi % RUSSIAN

\ifdefined\BRAZILIAN
\chapter{Padrões de códigos}
\fi % BRAZILIAN

\ifdefined\THAI
\chapter{รูปแบบของโค้ด}
\fi % THAI

\ifdefined\FRENCH
\chapter{Modèle de code}
\fi % FRENCH

\ifdefined\POLISH
\chapter{\PLph{}}
\fi % POLISH

% sections
\EN{\input{patterns/patterns_opt_dbg_EN}}
\ES{\input{patterns/patterns_opt_dbg_ES}}
\ITA{\input{patterns/patterns_opt_dbg_ITA}}
\PTBR{\input{patterns/patterns_opt_dbg_PTBR}}
\RU{\input{patterns/patterns_opt_dbg_RU}}
\THA{\input{patterns/patterns_opt_dbg_THA}}
\DE{\input{patterns/patterns_opt_dbg_DE}}
\FR{\input{patterns/patterns_opt_dbg_FR}}
\PL{\input{patterns/patterns_opt_dbg_PL}}

\RU{\section{Некоторые базовые понятия}}
\EN{\section{Some basics}}
\DE{\section{Einige Grundlagen}}
\FR{\section{Quelques bases}}
\ES{\section{\ESph{}}}
\ITA{\section{Alcune basi teoriche}}
\PTBR{\section{\PTBRph{}}}
\THA{\section{\THAph{}}}
\PL{\section{\PLph{}}}

% sections:
\EN{\input{patterns/intro_CPU_ISA_EN}}
\ES{\input{patterns/intro_CPU_ISA_ES}}
\ITA{\input{patterns/intro_CPU_ISA_ITA}}
\PTBR{\input{patterns/intro_CPU_ISA_PTBR}}
\RU{\input{patterns/intro_CPU_ISA_RU}}
\DE{\input{patterns/intro_CPU_ISA_DE}}
\FR{\input{patterns/intro_CPU_ISA_FR}}
\PL{\input{patterns/intro_CPU_ISA_PL}}

\EN{\input{patterns/numeral_EN}}
\RU{\input{patterns/numeral_RU}}
\ITA{\input{patterns/numeral_ITA}}
\DE{\input{patterns/numeral_DE}}
\FR{\input{patterns/numeral_FR}}
\PL{\input{patterns/numeral_PL}}

% chapters
\input{patterns/00_empty/main}
\input{patterns/011_ret/main}
\input{patterns/01_helloworld/main}
\input{patterns/015_prolog_epilogue/main}
\input{patterns/02_stack/main}
\input{patterns/03_printf/main}
\input{patterns/04_scanf/main}
\input{patterns/05_passing_arguments/main}
\input{patterns/06_return_results/main}
\input{patterns/061_pointers/main}
\input{patterns/065_GOTO/main}
\input{patterns/07_jcc/main}
\input{patterns/08_switch/main}
\input{patterns/09_loops/main}
\input{patterns/10_strings/main}
\input{patterns/11_arith_optimizations/main}
\input{patterns/12_FPU/main}
\input{patterns/13_arrays/main}
\input{patterns/14_bitfields/main}
\EN{\input{patterns/145_LCG/main_EN}}
\RU{\input{patterns/145_LCG/main_RU}}
\input{patterns/15_structs/main}
\input{patterns/17_unions/main}
\input{patterns/18_pointers_to_functions/main}
\input{patterns/185_64bit_in_32_env/main}

\EN{\input{patterns/19_SIMD/main_EN}}
\RU{\input{patterns/19_SIMD/main_RU}}
\DE{\input{patterns/19_SIMD/main_DE}}

\EN{\input{patterns/20_x64/main_EN}}
\RU{\input{patterns/20_x64/main_RU}}

\EN{\input{patterns/205_floating_SIMD/main_EN}}
\RU{\input{patterns/205_floating_SIMD/main_RU}}
\DE{\input{patterns/205_floating_SIMD/main_DE}}

\EN{\input{patterns/ARM/main_EN}}
\RU{\input{patterns/ARM/main_RU}}
\DE{\input{patterns/ARM/main_DE}}

\input{patterns/MIPS/main}

\ifdefined\SPANISH
\chapter{Patrones de código}
\fi % SPANISH

\ifdefined\GERMAN
\chapter{Code-Muster}
\fi % GERMAN

\ifdefined\ENGLISH
\chapter{Code Patterns}
\fi % ENGLISH

\ifdefined\ITALIAN
\chapter{Forme di codice}
\fi % ITALIAN

\ifdefined\RUSSIAN
\chapter{Образцы кода}
\fi % RUSSIAN

\ifdefined\BRAZILIAN
\chapter{Padrões de códigos}
\fi % BRAZILIAN

\ifdefined\THAI
\chapter{รูปแบบของโค้ด}
\fi % THAI

\ifdefined\FRENCH
\chapter{Modèle de code}
\fi % FRENCH

\ifdefined\POLISH
\chapter{\PLph{}}
\fi % POLISH

% sections
\EN{\input{patterns/patterns_opt_dbg_EN}}
\ES{\input{patterns/patterns_opt_dbg_ES}}
\ITA{\input{patterns/patterns_opt_dbg_ITA}}
\PTBR{\input{patterns/patterns_opt_dbg_PTBR}}
\RU{\input{patterns/patterns_opt_dbg_RU}}
\THA{\input{patterns/patterns_opt_dbg_THA}}
\DE{\input{patterns/patterns_opt_dbg_DE}}
\FR{\input{patterns/patterns_opt_dbg_FR}}
\PL{\input{patterns/patterns_opt_dbg_PL}}

\RU{\section{Некоторые базовые понятия}}
\EN{\section{Some basics}}
\DE{\section{Einige Grundlagen}}
\FR{\section{Quelques bases}}
\ES{\section{\ESph{}}}
\ITA{\section{Alcune basi teoriche}}
\PTBR{\section{\PTBRph{}}}
\THA{\section{\THAph{}}}
\PL{\section{\PLph{}}}

% sections:
\EN{\input{patterns/intro_CPU_ISA_EN}}
\ES{\input{patterns/intro_CPU_ISA_ES}}
\ITA{\input{patterns/intro_CPU_ISA_ITA}}
\PTBR{\input{patterns/intro_CPU_ISA_PTBR}}
\RU{\input{patterns/intro_CPU_ISA_RU}}
\DE{\input{patterns/intro_CPU_ISA_DE}}
\FR{\input{patterns/intro_CPU_ISA_FR}}
\PL{\input{patterns/intro_CPU_ISA_PL}}

\EN{\input{patterns/numeral_EN}}
\RU{\input{patterns/numeral_RU}}
\ITA{\input{patterns/numeral_ITA}}
\DE{\input{patterns/numeral_DE}}
\FR{\input{patterns/numeral_FR}}
\PL{\input{patterns/numeral_PL}}

% chapters
\input{patterns/00_empty/main}
\input{patterns/011_ret/main}
\input{patterns/01_helloworld/main}
\input{patterns/015_prolog_epilogue/main}
\input{patterns/02_stack/main}
\input{patterns/03_printf/main}
\input{patterns/04_scanf/main}
\input{patterns/05_passing_arguments/main}
\input{patterns/06_return_results/main}
\input{patterns/061_pointers/main}
\input{patterns/065_GOTO/main}
\input{patterns/07_jcc/main}
\input{patterns/08_switch/main}
\input{patterns/09_loops/main}
\input{patterns/10_strings/main}
\input{patterns/11_arith_optimizations/main}
\input{patterns/12_FPU/main}
\input{patterns/13_arrays/main}
\input{patterns/14_bitfields/main}
\EN{\input{patterns/145_LCG/main_EN}}
\RU{\input{patterns/145_LCG/main_RU}}
\input{patterns/15_structs/main}
\input{patterns/17_unions/main}
\input{patterns/18_pointers_to_functions/main}
\input{patterns/185_64bit_in_32_env/main}

\EN{\input{patterns/19_SIMD/main_EN}}
\RU{\input{patterns/19_SIMD/main_RU}}
\DE{\input{patterns/19_SIMD/main_DE}}

\EN{\input{patterns/20_x64/main_EN}}
\RU{\input{patterns/20_x64/main_RU}}

\EN{\input{patterns/205_floating_SIMD/main_EN}}
\RU{\input{patterns/205_floating_SIMD/main_RU}}
\DE{\input{patterns/205_floating_SIMD/main_DE}}

\EN{\input{patterns/ARM/main_EN}}
\RU{\input{patterns/ARM/main_RU}}
\DE{\input{patterns/ARM/main_DE}}

\input{patterns/MIPS/main}

\ifdefined\SPANISH
\chapter{Patrones de código}
\fi % SPANISH

\ifdefined\GERMAN
\chapter{Code-Muster}
\fi % GERMAN

\ifdefined\ENGLISH
\chapter{Code Patterns}
\fi % ENGLISH

\ifdefined\ITALIAN
\chapter{Forme di codice}
\fi % ITALIAN

\ifdefined\RUSSIAN
\chapter{Образцы кода}
\fi % RUSSIAN

\ifdefined\BRAZILIAN
\chapter{Padrões de códigos}
\fi % BRAZILIAN

\ifdefined\THAI
\chapter{รูปแบบของโค้ด}
\fi % THAI

\ifdefined\FRENCH
\chapter{Modèle de code}
\fi % FRENCH

\ifdefined\POLISH
\chapter{\PLph{}}
\fi % POLISH

% sections
\EN{\input{patterns/patterns_opt_dbg_EN}}
\ES{\input{patterns/patterns_opt_dbg_ES}}
\ITA{\input{patterns/patterns_opt_dbg_ITA}}
\PTBR{\input{patterns/patterns_opt_dbg_PTBR}}
\RU{\input{patterns/patterns_opt_dbg_RU}}
\THA{\input{patterns/patterns_opt_dbg_THA}}
\DE{\input{patterns/patterns_opt_dbg_DE}}
\FR{\input{patterns/patterns_opt_dbg_FR}}
\PL{\input{patterns/patterns_opt_dbg_PL}}

\RU{\section{Некоторые базовые понятия}}
\EN{\section{Some basics}}
\DE{\section{Einige Grundlagen}}
\FR{\section{Quelques bases}}
\ES{\section{\ESph{}}}
\ITA{\section{Alcune basi teoriche}}
\PTBR{\section{\PTBRph{}}}
\THA{\section{\THAph{}}}
\PL{\section{\PLph{}}}

% sections:
\EN{\input{patterns/intro_CPU_ISA_EN}}
\ES{\input{patterns/intro_CPU_ISA_ES}}
\ITA{\input{patterns/intro_CPU_ISA_ITA}}
\PTBR{\input{patterns/intro_CPU_ISA_PTBR}}
\RU{\input{patterns/intro_CPU_ISA_RU}}
\DE{\input{patterns/intro_CPU_ISA_DE}}
\FR{\input{patterns/intro_CPU_ISA_FR}}
\PL{\input{patterns/intro_CPU_ISA_PL}}

\EN{\input{patterns/numeral_EN}}
\RU{\input{patterns/numeral_RU}}
\ITA{\input{patterns/numeral_ITA}}
\DE{\input{patterns/numeral_DE}}
\FR{\input{patterns/numeral_FR}}
\PL{\input{patterns/numeral_PL}}

% chapters
\input{patterns/00_empty/main}
\input{patterns/011_ret/main}
\input{patterns/01_helloworld/main}
\input{patterns/015_prolog_epilogue/main}
\input{patterns/02_stack/main}
\input{patterns/03_printf/main}
\input{patterns/04_scanf/main}
\input{patterns/05_passing_arguments/main}
\input{patterns/06_return_results/main}
\input{patterns/061_pointers/main}
\input{patterns/065_GOTO/main}
\input{patterns/07_jcc/main}
\input{patterns/08_switch/main}
\input{patterns/09_loops/main}
\input{patterns/10_strings/main}
\input{patterns/11_arith_optimizations/main}
\input{patterns/12_FPU/main}
\input{patterns/13_arrays/main}
\input{patterns/14_bitfields/main}
\EN{\input{patterns/145_LCG/main_EN}}
\RU{\input{patterns/145_LCG/main_RU}}
\input{patterns/15_structs/main}
\input{patterns/17_unions/main}
\input{patterns/18_pointers_to_functions/main}
\input{patterns/185_64bit_in_32_env/main}

\EN{\input{patterns/19_SIMD/main_EN}}
\RU{\input{patterns/19_SIMD/main_RU}}
\DE{\input{patterns/19_SIMD/main_DE}}

\EN{\input{patterns/20_x64/main_EN}}
\RU{\input{patterns/20_x64/main_RU}}

\EN{\input{patterns/205_floating_SIMD/main_EN}}
\RU{\input{patterns/205_floating_SIMD/main_RU}}
\DE{\input{patterns/205_floating_SIMD/main_DE}}

\EN{\input{patterns/ARM/main_EN}}
\RU{\input{patterns/ARM/main_RU}}
\DE{\input{patterns/ARM/main_DE}}

\input{patterns/MIPS/main}

\ifdefined\SPANISH
\chapter{Patrones de código}
\fi % SPANISH

\ifdefined\GERMAN
\chapter{Code-Muster}
\fi % GERMAN

\ifdefined\ENGLISH
\chapter{Code Patterns}
\fi % ENGLISH

\ifdefined\ITALIAN
\chapter{Forme di codice}
\fi % ITALIAN

\ifdefined\RUSSIAN
\chapter{Образцы кода}
\fi % RUSSIAN

\ifdefined\BRAZILIAN
\chapter{Padrões de códigos}
\fi % BRAZILIAN

\ifdefined\THAI
\chapter{รูปแบบของโค้ด}
\fi % THAI

\ifdefined\FRENCH
\chapter{Modèle de code}
\fi % FRENCH

\ifdefined\POLISH
\chapter{\PLph{}}
\fi % POLISH

% sections
\EN{\input{patterns/patterns_opt_dbg_EN}}
\ES{\input{patterns/patterns_opt_dbg_ES}}
\ITA{\input{patterns/patterns_opt_dbg_ITA}}
\PTBR{\input{patterns/patterns_opt_dbg_PTBR}}
\RU{\input{patterns/patterns_opt_dbg_RU}}
\THA{\input{patterns/patterns_opt_dbg_THA}}
\DE{\input{patterns/patterns_opt_dbg_DE}}
\FR{\input{patterns/patterns_opt_dbg_FR}}
\PL{\input{patterns/patterns_opt_dbg_PL}}

\RU{\section{Некоторые базовые понятия}}
\EN{\section{Some basics}}
\DE{\section{Einige Grundlagen}}
\FR{\section{Quelques bases}}
\ES{\section{\ESph{}}}
\ITA{\section{Alcune basi teoriche}}
\PTBR{\section{\PTBRph{}}}
\THA{\section{\THAph{}}}
\PL{\section{\PLph{}}}

% sections:
\EN{\input{patterns/intro_CPU_ISA_EN}}
\ES{\input{patterns/intro_CPU_ISA_ES}}
\ITA{\input{patterns/intro_CPU_ISA_ITA}}
\PTBR{\input{patterns/intro_CPU_ISA_PTBR}}
\RU{\input{patterns/intro_CPU_ISA_RU}}
\DE{\input{patterns/intro_CPU_ISA_DE}}
\FR{\input{patterns/intro_CPU_ISA_FR}}
\PL{\input{patterns/intro_CPU_ISA_PL}}

\EN{\input{patterns/numeral_EN}}
\RU{\input{patterns/numeral_RU}}
\ITA{\input{patterns/numeral_ITA}}
\DE{\input{patterns/numeral_DE}}
\FR{\input{patterns/numeral_FR}}
\PL{\input{patterns/numeral_PL}}

% chapters
\input{patterns/00_empty/main}
\input{patterns/011_ret/main}
\input{patterns/01_helloworld/main}
\input{patterns/015_prolog_epilogue/main}
\input{patterns/02_stack/main}
\input{patterns/03_printf/main}
\input{patterns/04_scanf/main}
\input{patterns/05_passing_arguments/main}
\input{patterns/06_return_results/main}
\input{patterns/061_pointers/main}
\input{patterns/065_GOTO/main}
\input{patterns/07_jcc/main}
\input{patterns/08_switch/main}
\input{patterns/09_loops/main}
\input{patterns/10_strings/main}
\input{patterns/11_arith_optimizations/main}
\input{patterns/12_FPU/main}
\input{patterns/13_arrays/main}
\input{patterns/14_bitfields/main}
\EN{\input{patterns/145_LCG/main_EN}}
\RU{\input{patterns/145_LCG/main_RU}}
\input{patterns/15_structs/main}
\input{patterns/17_unions/main}
\input{patterns/18_pointers_to_functions/main}
\input{patterns/185_64bit_in_32_env/main}

\EN{\input{patterns/19_SIMD/main_EN}}
\RU{\input{patterns/19_SIMD/main_RU}}
\DE{\input{patterns/19_SIMD/main_DE}}

\EN{\input{patterns/20_x64/main_EN}}
\RU{\input{patterns/20_x64/main_RU}}

\EN{\input{patterns/205_floating_SIMD/main_EN}}
\RU{\input{patterns/205_floating_SIMD/main_RU}}
\DE{\input{patterns/205_floating_SIMD/main_DE}}

\EN{\input{patterns/ARM/main_EN}}
\RU{\input{patterns/ARM/main_RU}}
\DE{\input{patterns/ARM/main_DE}}

\input{patterns/MIPS/main}


\EN{\section{Returning Values}
\label{ret_val_func}

Another simple function is the one that simply returns a constant value:

\lstinputlisting[caption=\EN{\CCpp Code},style=customc]{patterns/011_ret/1.c}

Let's compile it.

\subsection{x86}

Here's what both the GCC and MSVC compilers produce (with optimization) on the x86 platform:

\lstinputlisting[caption=\Optimizing GCC/MSVC (\assemblyOutput),style=customasmx86]{patterns/011_ret/1.s}

\myindex{x86!\Instructions!RET}
There are just two instructions: the first places the value 123 into the \EAX register,
which is used by convention for storing the return
value, and the second one is \RET, which returns execution to the \gls{caller}.

The caller will take the result from the \EAX register.

\subsection{ARM}

There are a few differences on the ARM platform:

\lstinputlisting[caption=\OptimizingKeilVI (\ARMMode) ASM Output,style=customasmARM]{patterns/011_ret/1_Keil_ARM_O3.s}

ARM uses the register \Reg{0} for returning the results of functions, so 123 is copied into \Reg{0}.

\myindex{ARM!\Instructions!MOV}
\myindex{x86!\Instructions!MOV}
It is worth noting that \MOV is a misleading name for the instruction in both the x86 and ARM \ac{ISA}s.

The data is not in fact \IT{moved}, but \IT{copied}.

\subsection{MIPS}

\label{MIPS_leaf_function_ex1}

The GCC assembly output below lists registers by number:

\lstinputlisting[caption=\Optimizing GCC 4.4.5 (\assemblyOutput),style=customasmMIPS]{patterns/011_ret/MIPS.s}

\dots while \IDA does it by their pseudo names:

\lstinputlisting[caption=\Optimizing GCC 4.4.5 (IDA),style=customasmMIPS]{patterns/011_ret/MIPS_IDA.lst}

The \$2 (or \$V0) register is used to store the function's return value.
\myindex{MIPS!\Pseudoinstructions!LI}
\INS{LI} stands for ``Load Immediate'' and is the MIPS equivalent to \MOV.

\myindex{MIPS!\Instructions!J}
The other instruction is the jump instruction (J or JR) which returns the execution flow to the \gls{caller}.

\myindex{MIPS!Branch delay slot}
You might be wondering why the positions of the load instruction (LI) and the jump instruction (J or JR) are swapped. This is due to a \ac{RISC} feature called ``branch delay slot''.

The reason this happens is a quirk in the architecture of some RISC \ac{ISA}s and isn't important for our
purposes---we must simply keep in mind that in MIPS, the instruction following a jump or branch instruction
is executed \IT{before} the jump/branch instruction itself.

As a consequence, branch instructions always swap places with the instruction executed immediately beforehand.


In practice, functions which merely return 1 (\IT{true}) or 0 (\IT{false}) are very frequent.

The smallest ever of the standard UNIX utilities, \IT{/bin/true} and \IT{/bin/false} return 0 and 1 respectively, as an exit code.
(Zero as an exit code usually means success, non-zero means error.)
}
\RU{\subsubsection{std::string}
\myindex{\Cpp!STL!std::string}
\label{std_string}

\myparagraph{Как устроена структура}

Многие строковые библиотеки \InSqBrackets{\CNotes 2.2} обеспечивают структуру содержащую ссылку 
на буфер собственно со строкой, переменная всегда содержащую длину строки 
(что очень удобно для массы функций \InSqBrackets{\CNotes 2.2.1}) и переменную содержащую текущий размер буфера.

Строка в буфере обыкновенно оканчивается нулем: это для того чтобы указатель на буфер можно было
передавать в функции требующие на вход обычную сишную \ac{ASCIIZ}-строку.

Стандарт \Cpp не описывает, как именно нужно реализовывать std::string,
но, как правило, они реализованы как описано выше, с небольшими дополнениями.

Строки в \Cpp это не класс (как, например, QString в Qt), а темплейт (basic\_string), 
это сделано для того чтобы поддерживать 
строки содержащие разного типа символы: как минимум \Tchar и \IT{wchar\_t}.

Так что, std::string это класс с базовым типом \Tchar.

А std::wstring это класс с базовым типом \IT{wchar\_t}.

\mysubparagraph{MSVC}

В реализации MSVC, вместо ссылки на буфер может содержаться сам буфер (если строка короче 16-и символов).

Это означает, что каждая короткая строка будет занимать в памяти по крайней мере $16 + 4 + 4 = 24$ 
байт для 32-битной среды либо $16 + 8 + 8 = 32$ 
байта в 64-битной, а если строка длиннее 16-и символов, то прибавьте еще длину самой строки.

\lstinputlisting[caption=пример для MSVC,style=customc]{\CURPATH/STL/string/MSVC_RU.cpp}

Собственно, из этого исходника почти всё ясно.

Несколько замечаний:

Если строка короче 16-и символов, 
то отдельный буфер для строки в \glslink{heap}{куче} выделяться не будет.

Это удобно потому что на практике, основная часть строк действительно короткие.
Вероятно, разработчики в Microsoft выбрали размер в 16 символов как разумный баланс.

Теперь очень важный момент в конце функции main(): мы не пользуемся методом c\_str(), тем не менее,
если это скомпилировать и запустить, то обе строки появятся в консоли!

Работает это вот почему.

В первом случае строка короче 16-и символов и в начале объекта std::string (его можно рассматривать
просто как структуру) расположен буфер с этой строкой.
\printf трактует указатель как указатель на массив символов оканчивающийся нулем и поэтому всё работает.

Вывод второй строки (длиннее 16-и символов) даже еще опаснее: это вообще типичная программистская ошибка 
(или опечатка), забыть дописать c\_str().
Это работает потому что в это время в начале структуры расположен указатель на буфер.
Это может надолго остаться незамеченным: до тех пока там не появится строка 
короче 16-и символов, тогда процесс упадет.

\mysubparagraph{GCC}

В реализации GCC в структуре есть еще одна переменная --- reference count.

Интересно, что указатель на экземпляр класса std::string в GCC указывает не на начало самой структуры, 
а на указатель на буфера.
В libstdc++-v3\textbackslash{}include\textbackslash{}bits\textbackslash{}basic\_string.h 
мы можем прочитать что это сделано для удобства отладки:

\begin{lstlisting}
   *  The reason you want _M_data pointing to the character %array and
   *  not the _Rep is so that the debugger can see the string
   *  contents. (Probably we should add a non-inline member to get
   *  the _Rep for the debugger to use, so users can check the actual
   *  string length.)
\end{lstlisting}

\href{http://go.yurichev.com/17085}{исходный код basic\_string.h}

В нашем примере мы учитываем это:

\lstinputlisting[caption=пример для GCC,style=customc]{\CURPATH/STL/string/GCC_RU.cpp}

Нужны еще небольшие хаки чтобы сымитировать типичную ошибку, которую мы уже видели выше, из-за
более ужесточенной проверки типов в GCC, тем не менее, printf() работает и здесь без c\_str().

\myparagraph{Чуть более сложный пример}

\lstinputlisting[style=customc]{\CURPATH/STL/string/3.cpp}

\lstinputlisting[caption=MSVC 2012,style=customasmx86]{\CURPATH/STL/string/3_MSVC_RU.asm}

Собственно, компилятор не конструирует строки статически: да в общем-то и как
это возможно, если буфер с ней нужно хранить в \glslink{heap}{куче}?

Вместо этого в сегменте данных хранятся обычные \ac{ASCIIZ}-строки, а позже, во время выполнения, 
при помощи метода \q{assign}, конструируются строки s1 и s2
.
При помощи \TT{operator+}, создается строка s3.

Обратите внимание на то что вызов метода c\_str() отсутствует,
потому что его код достаточно короткий и компилятор вставил его прямо здесь:
если строка короче 16-и байт, то в регистре EAX остается указатель на буфер,
а если длиннее, то из этого же места достается адрес на буфер расположенный в \glslink{heap}{куче}.

Далее следуют вызовы трех деструкторов, причем, они вызываются только если строка длиннее 16-и байт:
тогда нужно освободить буфера в \glslink{heap}{куче}.
В противном случае, так как все три объекта std::string хранятся в стеке,
они освобождаются автоматически после выхода из функции.

Следовательно, работа с короткими строками более быстрая из-за м\'{е}ньшего обращения к \glslink{heap}{куче}.

Код на GCC даже проще (из-за того, что в GCC, как мы уже видели, не реализована возможность хранить короткую
строку прямо в структуре):

% TODO1 comment each function meaning
\lstinputlisting[caption=GCC 4.8.1,style=customasmx86]{\CURPATH/STL/string/3_GCC_RU.s}

Можно заметить, что в деструкторы передается не указатель на объект,
а указатель на место за 12 байт (или 3 слова) перед ним, то есть, на настоящее начало структуры.

\myparagraph{std::string как глобальная переменная}
\label{sec:std_string_as_global_variable}

Опытные программисты на \Cpp знают, что глобальные переменные \ac{STL}-типов вполне можно объявлять.

Да, действительно:

\lstinputlisting[style=customc]{\CURPATH/STL/string/5.cpp}

Но как и где будет вызываться конструктор \TT{std::string}?

На самом деле, эта переменная будет инициализирована даже перед началом \main.

\lstinputlisting[caption=MSVC 2012: здесь конструируется глобальная переменная{,} а также регистрируется её деструктор,style=customasmx86]{\CURPATH/STL/string/5_MSVC_p2.asm}

\lstinputlisting[caption=MSVC 2012: здесь глобальная переменная используется в \main,style=customasmx86]{\CURPATH/STL/string/5_MSVC_p1.asm}

\lstinputlisting[caption=MSVC 2012: эта функция-деструктор вызывается перед выходом,style=customasmx86]{\CURPATH/STL/string/5_MSVC_p3.asm}

\myindex{\CStandardLibrary!atexit()}
В реальности, из \ac{CRT}, еще до вызова main(), вызывается специальная функция,
в которой перечислены все конструкторы подобных переменных.
Более того: при помощи atexit() регистрируется функция, которая будет вызвана в конце работы программы:
в этой функции компилятор собирает вызовы деструкторов всех подобных глобальных переменных.

GCC работает похожим образом:

\lstinputlisting[caption=GCC 4.8.1,style=customasmx86]{\CURPATH/STL/string/5_GCC.s}

Но он не выделяет отдельной функции в которой будут собраны деструкторы: 
каждый деструктор передается в atexit() по одному.

% TODO а если глобальная STL-переменная в другом модуле? надо проверить.

}
\DE{\subsection{Einfachste XOR-Verschlüsselung überhaupt}

Ich habe einmal eine Software gesehen, bei der alle Debugging-Ausgaben mit XOR mit dem Wert 3
verschlüsselt wurden. Mit anderen Worten, die beiden niedrigsten Bits aller Buchstaben wurden invertiert.

``Hello, world'' wurde zu ``Kfool/\#tlqog'':

\begin{lstlisting}
#!/usr/bin/python

msg="Hello, world!"

print "".join(map(lambda x: chr(ord(x)^3), msg))
\end{lstlisting}

Das ist eine ziemlich interessante Verschlüsselung (oder besser eine Verschleierung),
weil sie zwei wichtige Eigenschaften hat:
1) es ist eine einzige Funktion zum Verschlüsseln und entschlüsseln, sie muss nur wiederholt angewendet werden
2) die entstehenden Buchstaben befinden sich im druckbaren Bereich, also die ganze Zeichenkette kann ohne
Escape-Symbole im Code verwendet werden.

Die zweite Eigenschaft nutzt die Tatsache, dass alle druckbaren Zeichen in Reihen organisiert sind: 0x2x-0x7x,
und wenn die beiden niederwertigsten Bits invertiert werden, wird der Buchstabe um eine oder drei Stellen nach
links oder rechts \IT{verschoben}, aber niemals in eine andere Reihe:

\begin{figure}[H]
\centering
\includegraphics[width=0.7\textwidth]{ascii_clean.png}
\caption{7-Bit \ac{ASCII} Tabelle in Emacs}
\end{figure}

\dots mit dem Zeichen 0x7F als einziger Ausnahme.

Im Folgenden werden also beispielsweise die Zeichen A-Z \IT{verschlüsselt}:

\begin{lstlisting}
#!/usr/bin/python

msg="@ABCDEFGHIJKLMNO"

print "".join(map(lambda x: chr(ord(x)^3), msg))
\end{lstlisting}

Ergebnis:
% FIXME \verb  --  relevant comment for German?
\begin{lstlisting}
CBA@GFEDKJIHONML
\end{lstlisting}

Es sieht so aus als würden die Zeichen ``@'' und ``C'' sowie ``B'' und ``A'' vertauscht werden.

Hier ist noch ein interessantes Beispiel, in dem gezeigt wird, wie die Eigenschaften von XOR
ausgenutzt werden können: Exakt den gleichen Effekt, dass druckbare Zeichen auch druckbar bleiben,
kann man dadurch erzielen, dass irgendeine Kombination der niedrigsten vier Bits invertiert wird.
}

\EN{\section{Returning Values}
\label{ret_val_func}

Another simple function is the one that simply returns a constant value:

\lstinputlisting[caption=\EN{\CCpp Code},style=customc]{patterns/011_ret/1.c}

Let's compile it.

\subsection{x86}

Here's what both the GCC and MSVC compilers produce (with optimization) on the x86 platform:

\lstinputlisting[caption=\Optimizing GCC/MSVC (\assemblyOutput),style=customasmx86]{patterns/011_ret/1.s}

\myindex{x86!\Instructions!RET}
There are just two instructions: the first places the value 123 into the \EAX register,
which is used by convention for storing the return
value, and the second one is \RET, which returns execution to the \gls{caller}.

The caller will take the result from the \EAX register.

\subsection{ARM}

There are a few differences on the ARM platform:

\lstinputlisting[caption=\OptimizingKeilVI (\ARMMode) ASM Output,style=customasmARM]{patterns/011_ret/1_Keil_ARM_O3.s}

ARM uses the register \Reg{0} for returning the results of functions, so 123 is copied into \Reg{0}.

\myindex{ARM!\Instructions!MOV}
\myindex{x86!\Instructions!MOV}
It is worth noting that \MOV is a misleading name for the instruction in both the x86 and ARM \ac{ISA}s.

The data is not in fact \IT{moved}, but \IT{copied}.

\subsection{MIPS}

\label{MIPS_leaf_function_ex1}

The GCC assembly output below lists registers by number:

\lstinputlisting[caption=\Optimizing GCC 4.4.5 (\assemblyOutput),style=customasmMIPS]{patterns/011_ret/MIPS.s}

\dots while \IDA does it by their pseudo names:

\lstinputlisting[caption=\Optimizing GCC 4.4.5 (IDA),style=customasmMIPS]{patterns/011_ret/MIPS_IDA.lst}

The \$2 (or \$V0) register is used to store the function's return value.
\myindex{MIPS!\Pseudoinstructions!LI}
\INS{LI} stands for ``Load Immediate'' and is the MIPS equivalent to \MOV.

\myindex{MIPS!\Instructions!J}
The other instruction is the jump instruction (J or JR) which returns the execution flow to the \gls{caller}.

\myindex{MIPS!Branch delay slot}
You might be wondering why the positions of the load instruction (LI) and the jump instruction (J or JR) are swapped. This is due to a \ac{RISC} feature called ``branch delay slot''.

The reason this happens is a quirk in the architecture of some RISC \ac{ISA}s and isn't important for our
purposes---we must simply keep in mind that in MIPS, the instruction following a jump or branch instruction
is executed \IT{before} the jump/branch instruction itself.

As a consequence, branch instructions always swap places with the instruction executed immediately beforehand.


In practice, functions which merely return 1 (\IT{true}) or 0 (\IT{false}) are very frequent.

The smallest ever of the standard UNIX utilities, \IT{/bin/true} and \IT{/bin/false} return 0 and 1 respectively, as an exit code.
(Zero as an exit code usually means success, non-zero means error.)
}
\RU{\subsubsection{std::string}
\myindex{\Cpp!STL!std::string}
\label{std_string}

\myparagraph{Как устроена структура}

Многие строковые библиотеки \InSqBrackets{\CNotes 2.2} обеспечивают структуру содержащую ссылку 
на буфер собственно со строкой, переменная всегда содержащую длину строки 
(что очень удобно для массы функций \InSqBrackets{\CNotes 2.2.1}) и переменную содержащую текущий размер буфера.

Строка в буфере обыкновенно оканчивается нулем: это для того чтобы указатель на буфер можно было
передавать в функции требующие на вход обычную сишную \ac{ASCIIZ}-строку.

Стандарт \Cpp не описывает, как именно нужно реализовывать std::string,
но, как правило, они реализованы как описано выше, с небольшими дополнениями.

Строки в \Cpp это не класс (как, например, QString в Qt), а темплейт (basic\_string), 
это сделано для того чтобы поддерживать 
строки содержащие разного типа символы: как минимум \Tchar и \IT{wchar\_t}.

Так что, std::string это класс с базовым типом \Tchar.

А std::wstring это класс с базовым типом \IT{wchar\_t}.

\mysubparagraph{MSVC}

В реализации MSVC, вместо ссылки на буфер может содержаться сам буфер (если строка короче 16-и символов).

Это означает, что каждая короткая строка будет занимать в памяти по крайней мере $16 + 4 + 4 = 24$ 
байт для 32-битной среды либо $16 + 8 + 8 = 32$ 
байта в 64-битной, а если строка длиннее 16-и символов, то прибавьте еще длину самой строки.

\lstinputlisting[caption=пример для MSVC,style=customc]{\CURPATH/STL/string/MSVC_RU.cpp}

Собственно, из этого исходника почти всё ясно.

Несколько замечаний:

Если строка короче 16-и символов, 
то отдельный буфер для строки в \glslink{heap}{куче} выделяться не будет.

Это удобно потому что на практике, основная часть строк действительно короткие.
Вероятно, разработчики в Microsoft выбрали размер в 16 символов как разумный баланс.

Теперь очень важный момент в конце функции main(): мы не пользуемся методом c\_str(), тем не менее,
если это скомпилировать и запустить, то обе строки появятся в консоли!

Работает это вот почему.

В первом случае строка короче 16-и символов и в начале объекта std::string (его можно рассматривать
просто как структуру) расположен буфер с этой строкой.
\printf трактует указатель как указатель на массив символов оканчивающийся нулем и поэтому всё работает.

Вывод второй строки (длиннее 16-и символов) даже еще опаснее: это вообще типичная программистская ошибка 
(или опечатка), забыть дописать c\_str().
Это работает потому что в это время в начале структуры расположен указатель на буфер.
Это может надолго остаться незамеченным: до тех пока там не появится строка 
короче 16-и символов, тогда процесс упадет.

\mysubparagraph{GCC}

В реализации GCC в структуре есть еще одна переменная --- reference count.

Интересно, что указатель на экземпляр класса std::string в GCC указывает не на начало самой структуры, 
а на указатель на буфера.
В libstdc++-v3\textbackslash{}include\textbackslash{}bits\textbackslash{}basic\_string.h 
мы можем прочитать что это сделано для удобства отладки:

\begin{lstlisting}
   *  The reason you want _M_data pointing to the character %array and
   *  not the _Rep is so that the debugger can see the string
   *  contents. (Probably we should add a non-inline member to get
   *  the _Rep for the debugger to use, so users can check the actual
   *  string length.)
\end{lstlisting}

\href{http://go.yurichev.com/17085}{исходный код basic\_string.h}

В нашем примере мы учитываем это:

\lstinputlisting[caption=пример для GCC,style=customc]{\CURPATH/STL/string/GCC_RU.cpp}

Нужны еще небольшие хаки чтобы сымитировать типичную ошибку, которую мы уже видели выше, из-за
более ужесточенной проверки типов в GCC, тем не менее, printf() работает и здесь без c\_str().

\myparagraph{Чуть более сложный пример}

\lstinputlisting[style=customc]{\CURPATH/STL/string/3.cpp}

\lstinputlisting[caption=MSVC 2012,style=customasmx86]{\CURPATH/STL/string/3_MSVC_RU.asm}

Собственно, компилятор не конструирует строки статически: да в общем-то и как
это возможно, если буфер с ней нужно хранить в \glslink{heap}{куче}?

Вместо этого в сегменте данных хранятся обычные \ac{ASCIIZ}-строки, а позже, во время выполнения, 
при помощи метода \q{assign}, конструируются строки s1 и s2
.
При помощи \TT{operator+}, создается строка s3.

Обратите внимание на то что вызов метода c\_str() отсутствует,
потому что его код достаточно короткий и компилятор вставил его прямо здесь:
если строка короче 16-и байт, то в регистре EAX остается указатель на буфер,
а если длиннее, то из этого же места достается адрес на буфер расположенный в \glslink{heap}{куче}.

Далее следуют вызовы трех деструкторов, причем, они вызываются только если строка длиннее 16-и байт:
тогда нужно освободить буфера в \glslink{heap}{куче}.
В противном случае, так как все три объекта std::string хранятся в стеке,
они освобождаются автоматически после выхода из функции.

Следовательно, работа с короткими строками более быстрая из-за м\'{е}ньшего обращения к \glslink{heap}{куче}.

Код на GCC даже проще (из-за того, что в GCC, как мы уже видели, не реализована возможность хранить короткую
строку прямо в структуре):

% TODO1 comment each function meaning
\lstinputlisting[caption=GCC 4.8.1,style=customasmx86]{\CURPATH/STL/string/3_GCC_RU.s}

Можно заметить, что в деструкторы передается не указатель на объект,
а указатель на место за 12 байт (или 3 слова) перед ним, то есть, на настоящее начало структуры.

\myparagraph{std::string как глобальная переменная}
\label{sec:std_string_as_global_variable}

Опытные программисты на \Cpp знают, что глобальные переменные \ac{STL}-типов вполне можно объявлять.

Да, действительно:

\lstinputlisting[style=customc]{\CURPATH/STL/string/5.cpp}

Но как и где будет вызываться конструктор \TT{std::string}?

На самом деле, эта переменная будет инициализирована даже перед началом \main.

\lstinputlisting[caption=MSVC 2012: здесь конструируется глобальная переменная{,} а также регистрируется её деструктор,style=customasmx86]{\CURPATH/STL/string/5_MSVC_p2.asm}

\lstinputlisting[caption=MSVC 2012: здесь глобальная переменная используется в \main,style=customasmx86]{\CURPATH/STL/string/5_MSVC_p1.asm}

\lstinputlisting[caption=MSVC 2012: эта функция-деструктор вызывается перед выходом,style=customasmx86]{\CURPATH/STL/string/5_MSVC_p3.asm}

\myindex{\CStandardLibrary!atexit()}
В реальности, из \ac{CRT}, еще до вызова main(), вызывается специальная функция,
в которой перечислены все конструкторы подобных переменных.
Более того: при помощи atexit() регистрируется функция, которая будет вызвана в конце работы программы:
в этой функции компилятор собирает вызовы деструкторов всех подобных глобальных переменных.

GCC работает похожим образом:

\lstinputlisting[caption=GCC 4.8.1,style=customasmx86]{\CURPATH/STL/string/5_GCC.s}

Но он не выделяет отдельной функции в которой будут собраны деструкторы: 
каждый деструктор передается в atexit() по одному.

% TODO а если глобальная STL-переменная в другом модуле? надо проверить.

}

\EN{\section{Returning Values}
\label{ret_val_func}

Another simple function is the one that simply returns a constant value:

\lstinputlisting[caption=\EN{\CCpp Code},style=customc]{patterns/011_ret/1.c}

Let's compile it.

\subsection{x86}

Here's what both the GCC and MSVC compilers produce (with optimization) on the x86 platform:

\lstinputlisting[caption=\Optimizing GCC/MSVC (\assemblyOutput),style=customasmx86]{patterns/011_ret/1.s}

\myindex{x86!\Instructions!RET}
There are just two instructions: the first places the value 123 into the \EAX register,
which is used by convention for storing the return
value, and the second one is \RET, which returns execution to the \gls{caller}.

The caller will take the result from the \EAX register.

\subsection{ARM}

There are a few differences on the ARM platform:

\lstinputlisting[caption=\OptimizingKeilVI (\ARMMode) ASM Output,style=customasmARM]{patterns/011_ret/1_Keil_ARM_O3.s}

ARM uses the register \Reg{0} for returning the results of functions, so 123 is copied into \Reg{0}.

\myindex{ARM!\Instructions!MOV}
\myindex{x86!\Instructions!MOV}
It is worth noting that \MOV is a misleading name for the instruction in both the x86 and ARM \ac{ISA}s.

The data is not in fact \IT{moved}, but \IT{copied}.

\subsection{MIPS}

\label{MIPS_leaf_function_ex1}

The GCC assembly output below lists registers by number:

\lstinputlisting[caption=\Optimizing GCC 4.4.5 (\assemblyOutput),style=customasmMIPS]{patterns/011_ret/MIPS.s}

\dots while \IDA does it by their pseudo names:

\lstinputlisting[caption=\Optimizing GCC 4.4.5 (IDA),style=customasmMIPS]{patterns/011_ret/MIPS_IDA.lst}

The \$2 (or \$V0) register is used to store the function's return value.
\myindex{MIPS!\Pseudoinstructions!LI}
\INS{LI} stands for ``Load Immediate'' and is the MIPS equivalent to \MOV.

\myindex{MIPS!\Instructions!J}
The other instruction is the jump instruction (J or JR) which returns the execution flow to the \gls{caller}.

\myindex{MIPS!Branch delay slot}
You might be wondering why the positions of the load instruction (LI) and the jump instruction (J or JR) are swapped. This is due to a \ac{RISC} feature called ``branch delay slot''.

The reason this happens is a quirk in the architecture of some RISC \ac{ISA}s and isn't important for our
purposes---we must simply keep in mind that in MIPS, the instruction following a jump or branch instruction
is executed \IT{before} the jump/branch instruction itself.

As a consequence, branch instructions always swap places with the instruction executed immediately beforehand.


In practice, functions which merely return 1 (\IT{true}) or 0 (\IT{false}) are very frequent.

The smallest ever of the standard UNIX utilities, \IT{/bin/true} and \IT{/bin/false} return 0 and 1 respectively, as an exit code.
(Zero as an exit code usually means success, non-zero means error.)
}
\RU{\subsubsection{std::string}
\myindex{\Cpp!STL!std::string}
\label{std_string}

\myparagraph{Как устроена структура}

Многие строковые библиотеки \InSqBrackets{\CNotes 2.2} обеспечивают структуру содержащую ссылку 
на буфер собственно со строкой, переменная всегда содержащую длину строки 
(что очень удобно для массы функций \InSqBrackets{\CNotes 2.2.1}) и переменную содержащую текущий размер буфера.

Строка в буфере обыкновенно оканчивается нулем: это для того чтобы указатель на буфер можно было
передавать в функции требующие на вход обычную сишную \ac{ASCIIZ}-строку.

Стандарт \Cpp не описывает, как именно нужно реализовывать std::string,
но, как правило, они реализованы как описано выше, с небольшими дополнениями.

Строки в \Cpp это не класс (как, например, QString в Qt), а темплейт (basic\_string), 
это сделано для того чтобы поддерживать 
строки содержащие разного типа символы: как минимум \Tchar и \IT{wchar\_t}.

Так что, std::string это класс с базовым типом \Tchar.

А std::wstring это класс с базовым типом \IT{wchar\_t}.

\mysubparagraph{MSVC}

В реализации MSVC, вместо ссылки на буфер может содержаться сам буфер (если строка короче 16-и символов).

Это означает, что каждая короткая строка будет занимать в памяти по крайней мере $16 + 4 + 4 = 24$ 
байт для 32-битной среды либо $16 + 8 + 8 = 32$ 
байта в 64-битной, а если строка длиннее 16-и символов, то прибавьте еще длину самой строки.

\lstinputlisting[caption=пример для MSVC,style=customc]{\CURPATH/STL/string/MSVC_RU.cpp}

Собственно, из этого исходника почти всё ясно.

Несколько замечаний:

Если строка короче 16-и символов, 
то отдельный буфер для строки в \glslink{heap}{куче} выделяться не будет.

Это удобно потому что на практике, основная часть строк действительно короткие.
Вероятно, разработчики в Microsoft выбрали размер в 16 символов как разумный баланс.

Теперь очень важный момент в конце функции main(): мы не пользуемся методом c\_str(), тем не менее,
если это скомпилировать и запустить, то обе строки появятся в консоли!

Работает это вот почему.

В первом случае строка короче 16-и символов и в начале объекта std::string (его можно рассматривать
просто как структуру) расположен буфер с этой строкой.
\printf трактует указатель как указатель на массив символов оканчивающийся нулем и поэтому всё работает.

Вывод второй строки (длиннее 16-и символов) даже еще опаснее: это вообще типичная программистская ошибка 
(или опечатка), забыть дописать c\_str().
Это работает потому что в это время в начале структуры расположен указатель на буфер.
Это может надолго остаться незамеченным: до тех пока там не появится строка 
короче 16-и символов, тогда процесс упадет.

\mysubparagraph{GCC}

В реализации GCC в структуре есть еще одна переменная --- reference count.

Интересно, что указатель на экземпляр класса std::string в GCC указывает не на начало самой структуры, 
а на указатель на буфера.
В libstdc++-v3\textbackslash{}include\textbackslash{}bits\textbackslash{}basic\_string.h 
мы можем прочитать что это сделано для удобства отладки:

\begin{lstlisting}
   *  The reason you want _M_data pointing to the character %array and
   *  not the _Rep is so that the debugger can see the string
   *  contents. (Probably we should add a non-inline member to get
   *  the _Rep for the debugger to use, so users can check the actual
   *  string length.)
\end{lstlisting}

\href{http://go.yurichev.com/17085}{исходный код basic\_string.h}

В нашем примере мы учитываем это:

\lstinputlisting[caption=пример для GCC,style=customc]{\CURPATH/STL/string/GCC_RU.cpp}

Нужны еще небольшие хаки чтобы сымитировать типичную ошибку, которую мы уже видели выше, из-за
более ужесточенной проверки типов в GCC, тем не менее, printf() работает и здесь без c\_str().

\myparagraph{Чуть более сложный пример}

\lstinputlisting[style=customc]{\CURPATH/STL/string/3.cpp}

\lstinputlisting[caption=MSVC 2012,style=customasmx86]{\CURPATH/STL/string/3_MSVC_RU.asm}

Собственно, компилятор не конструирует строки статически: да в общем-то и как
это возможно, если буфер с ней нужно хранить в \glslink{heap}{куче}?

Вместо этого в сегменте данных хранятся обычные \ac{ASCIIZ}-строки, а позже, во время выполнения, 
при помощи метода \q{assign}, конструируются строки s1 и s2
.
При помощи \TT{operator+}, создается строка s3.

Обратите внимание на то что вызов метода c\_str() отсутствует,
потому что его код достаточно короткий и компилятор вставил его прямо здесь:
если строка короче 16-и байт, то в регистре EAX остается указатель на буфер,
а если длиннее, то из этого же места достается адрес на буфер расположенный в \glslink{heap}{куче}.

Далее следуют вызовы трех деструкторов, причем, они вызываются только если строка длиннее 16-и байт:
тогда нужно освободить буфера в \glslink{heap}{куче}.
В противном случае, так как все три объекта std::string хранятся в стеке,
они освобождаются автоматически после выхода из функции.

Следовательно, работа с короткими строками более быстрая из-за м\'{е}ньшего обращения к \glslink{heap}{куче}.

Код на GCC даже проще (из-за того, что в GCC, как мы уже видели, не реализована возможность хранить короткую
строку прямо в структуре):

% TODO1 comment each function meaning
\lstinputlisting[caption=GCC 4.8.1,style=customasmx86]{\CURPATH/STL/string/3_GCC_RU.s}

Можно заметить, что в деструкторы передается не указатель на объект,
а указатель на место за 12 байт (или 3 слова) перед ним, то есть, на настоящее начало структуры.

\myparagraph{std::string как глобальная переменная}
\label{sec:std_string_as_global_variable}

Опытные программисты на \Cpp знают, что глобальные переменные \ac{STL}-типов вполне можно объявлять.

Да, действительно:

\lstinputlisting[style=customc]{\CURPATH/STL/string/5.cpp}

Но как и где будет вызываться конструктор \TT{std::string}?

На самом деле, эта переменная будет инициализирована даже перед началом \main.

\lstinputlisting[caption=MSVC 2012: здесь конструируется глобальная переменная{,} а также регистрируется её деструктор,style=customasmx86]{\CURPATH/STL/string/5_MSVC_p2.asm}

\lstinputlisting[caption=MSVC 2012: здесь глобальная переменная используется в \main,style=customasmx86]{\CURPATH/STL/string/5_MSVC_p1.asm}

\lstinputlisting[caption=MSVC 2012: эта функция-деструктор вызывается перед выходом,style=customasmx86]{\CURPATH/STL/string/5_MSVC_p3.asm}

\myindex{\CStandardLibrary!atexit()}
В реальности, из \ac{CRT}, еще до вызова main(), вызывается специальная функция,
в которой перечислены все конструкторы подобных переменных.
Более того: при помощи atexit() регистрируется функция, которая будет вызвана в конце работы программы:
в этой функции компилятор собирает вызовы деструкторов всех подобных глобальных переменных.

GCC работает похожим образом:

\lstinputlisting[caption=GCC 4.8.1,style=customasmx86]{\CURPATH/STL/string/5_GCC.s}

Но он не выделяет отдельной функции в которой будут собраны деструкторы: 
каждый деструктор передается в atexit() по одному.

% TODO а если глобальная STL-переменная в другом модуле? надо проверить.

}
\DE{\subsection{Einfachste XOR-Verschlüsselung überhaupt}

Ich habe einmal eine Software gesehen, bei der alle Debugging-Ausgaben mit XOR mit dem Wert 3
verschlüsselt wurden. Mit anderen Worten, die beiden niedrigsten Bits aller Buchstaben wurden invertiert.

``Hello, world'' wurde zu ``Kfool/\#tlqog'':

\begin{lstlisting}
#!/usr/bin/python

msg="Hello, world!"

print "".join(map(lambda x: chr(ord(x)^3), msg))
\end{lstlisting}

Das ist eine ziemlich interessante Verschlüsselung (oder besser eine Verschleierung),
weil sie zwei wichtige Eigenschaften hat:
1) es ist eine einzige Funktion zum Verschlüsseln und entschlüsseln, sie muss nur wiederholt angewendet werden
2) die entstehenden Buchstaben befinden sich im druckbaren Bereich, also die ganze Zeichenkette kann ohne
Escape-Symbole im Code verwendet werden.

Die zweite Eigenschaft nutzt die Tatsache, dass alle druckbaren Zeichen in Reihen organisiert sind: 0x2x-0x7x,
und wenn die beiden niederwertigsten Bits invertiert werden, wird der Buchstabe um eine oder drei Stellen nach
links oder rechts \IT{verschoben}, aber niemals in eine andere Reihe:

\begin{figure}[H]
\centering
\includegraphics[width=0.7\textwidth]{ascii_clean.png}
\caption{7-Bit \ac{ASCII} Tabelle in Emacs}
\end{figure}

\dots mit dem Zeichen 0x7F als einziger Ausnahme.

Im Folgenden werden also beispielsweise die Zeichen A-Z \IT{verschlüsselt}:

\begin{lstlisting}
#!/usr/bin/python

msg="@ABCDEFGHIJKLMNO"

print "".join(map(lambda x: chr(ord(x)^3), msg))
\end{lstlisting}

Ergebnis:
% FIXME \verb  --  relevant comment for German?
\begin{lstlisting}
CBA@GFEDKJIHONML
\end{lstlisting}

Es sieht so aus als würden die Zeichen ``@'' und ``C'' sowie ``B'' und ``A'' vertauscht werden.

Hier ist noch ein interessantes Beispiel, in dem gezeigt wird, wie die Eigenschaften von XOR
ausgenutzt werden können: Exakt den gleichen Effekt, dass druckbare Zeichen auch druckbar bleiben,
kann man dadurch erzielen, dass irgendeine Kombination der niedrigsten vier Bits invertiert wird.
}

\EN{\section{Returning Values}
\label{ret_val_func}

Another simple function is the one that simply returns a constant value:

\lstinputlisting[caption=\EN{\CCpp Code},style=customc]{patterns/011_ret/1.c}

Let's compile it.

\subsection{x86}

Here's what both the GCC and MSVC compilers produce (with optimization) on the x86 platform:

\lstinputlisting[caption=\Optimizing GCC/MSVC (\assemblyOutput),style=customasmx86]{patterns/011_ret/1.s}

\myindex{x86!\Instructions!RET}
There are just two instructions: the first places the value 123 into the \EAX register,
which is used by convention for storing the return
value, and the second one is \RET, which returns execution to the \gls{caller}.

The caller will take the result from the \EAX register.

\subsection{ARM}

There are a few differences on the ARM platform:

\lstinputlisting[caption=\OptimizingKeilVI (\ARMMode) ASM Output,style=customasmARM]{patterns/011_ret/1_Keil_ARM_O3.s}

ARM uses the register \Reg{0} for returning the results of functions, so 123 is copied into \Reg{0}.

\myindex{ARM!\Instructions!MOV}
\myindex{x86!\Instructions!MOV}
It is worth noting that \MOV is a misleading name for the instruction in both the x86 and ARM \ac{ISA}s.

The data is not in fact \IT{moved}, but \IT{copied}.

\subsection{MIPS}

\label{MIPS_leaf_function_ex1}

The GCC assembly output below lists registers by number:

\lstinputlisting[caption=\Optimizing GCC 4.4.5 (\assemblyOutput),style=customasmMIPS]{patterns/011_ret/MIPS.s}

\dots while \IDA does it by their pseudo names:

\lstinputlisting[caption=\Optimizing GCC 4.4.5 (IDA),style=customasmMIPS]{patterns/011_ret/MIPS_IDA.lst}

The \$2 (or \$V0) register is used to store the function's return value.
\myindex{MIPS!\Pseudoinstructions!LI}
\INS{LI} stands for ``Load Immediate'' and is the MIPS equivalent to \MOV.

\myindex{MIPS!\Instructions!J}
The other instruction is the jump instruction (J or JR) which returns the execution flow to the \gls{caller}.

\myindex{MIPS!Branch delay slot}
You might be wondering why the positions of the load instruction (LI) and the jump instruction (J or JR) are swapped. This is due to a \ac{RISC} feature called ``branch delay slot''.

The reason this happens is a quirk in the architecture of some RISC \ac{ISA}s and isn't important for our
purposes---we must simply keep in mind that in MIPS, the instruction following a jump or branch instruction
is executed \IT{before} the jump/branch instruction itself.

As a consequence, branch instructions always swap places with the instruction executed immediately beforehand.


In practice, functions which merely return 1 (\IT{true}) or 0 (\IT{false}) are very frequent.

The smallest ever of the standard UNIX utilities, \IT{/bin/true} and \IT{/bin/false} return 0 and 1 respectively, as an exit code.
(Zero as an exit code usually means success, non-zero means error.)
}
\RU{\subsubsection{std::string}
\myindex{\Cpp!STL!std::string}
\label{std_string}

\myparagraph{Как устроена структура}

Многие строковые библиотеки \InSqBrackets{\CNotes 2.2} обеспечивают структуру содержащую ссылку 
на буфер собственно со строкой, переменная всегда содержащую длину строки 
(что очень удобно для массы функций \InSqBrackets{\CNotes 2.2.1}) и переменную содержащую текущий размер буфера.

Строка в буфере обыкновенно оканчивается нулем: это для того чтобы указатель на буфер можно было
передавать в функции требующие на вход обычную сишную \ac{ASCIIZ}-строку.

Стандарт \Cpp не описывает, как именно нужно реализовывать std::string,
но, как правило, они реализованы как описано выше, с небольшими дополнениями.

Строки в \Cpp это не класс (как, например, QString в Qt), а темплейт (basic\_string), 
это сделано для того чтобы поддерживать 
строки содержащие разного типа символы: как минимум \Tchar и \IT{wchar\_t}.

Так что, std::string это класс с базовым типом \Tchar.

А std::wstring это класс с базовым типом \IT{wchar\_t}.

\mysubparagraph{MSVC}

В реализации MSVC, вместо ссылки на буфер может содержаться сам буфер (если строка короче 16-и символов).

Это означает, что каждая короткая строка будет занимать в памяти по крайней мере $16 + 4 + 4 = 24$ 
байт для 32-битной среды либо $16 + 8 + 8 = 32$ 
байта в 64-битной, а если строка длиннее 16-и символов, то прибавьте еще длину самой строки.

\lstinputlisting[caption=пример для MSVC,style=customc]{\CURPATH/STL/string/MSVC_RU.cpp}

Собственно, из этого исходника почти всё ясно.

Несколько замечаний:

Если строка короче 16-и символов, 
то отдельный буфер для строки в \glslink{heap}{куче} выделяться не будет.

Это удобно потому что на практике, основная часть строк действительно короткие.
Вероятно, разработчики в Microsoft выбрали размер в 16 символов как разумный баланс.

Теперь очень важный момент в конце функции main(): мы не пользуемся методом c\_str(), тем не менее,
если это скомпилировать и запустить, то обе строки появятся в консоли!

Работает это вот почему.

В первом случае строка короче 16-и символов и в начале объекта std::string (его можно рассматривать
просто как структуру) расположен буфер с этой строкой.
\printf трактует указатель как указатель на массив символов оканчивающийся нулем и поэтому всё работает.

Вывод второй строки (длиннее 16-и символов) даже еще опаснее: это вообще типичная программистская ошибка 
(или опечатка), забыть дописать c\_str().
Это работает потому что в это время в начале структуры расположен указатель на буфер.
Это может надолго остаться незамеченным: до тех пока там не появится строка 
короче 16-и символов, тогда процесс упадет.

\mysubparagraph{GCC}

В реализации GCC в структуре есть еще одна переменная --- reference count.

Интересно, что указатель на экземпляр класса std::string в GCC указывает не на начало самой структуры, 
а на указатель на буфера.
В libstdc++-v3\textbackslash{}include\textbackslash{}bits\textbackslash{}basic\_string.h 
мы можем прочитать что это сделано для удобства отладки:

\begin{lstlisting}
   *  The reason you want _M_data pointing to the character %array and
   *  not the _Rep is so that the debugger can see the string
   *  contents. (Probably we should add a non-inline member to get
   *  the _Rep for the debugger to use, so users can check the actual
   *  string length.)
\end{lstlisting}

\href{http://go.yurichev.com/17085}{исходный код basic\_string.h}

В нашем примере мы учитываем это:

\lstinputlisting[caption=пример для GCC,style=customc]{\CURPATH/STL/string/GCC_RU.cpp}

Нужны еще небольшие хаки чтобы сымитировать типичную ошибку, которую мы уже видели выше, из-за
более ужесточенной проверки типов в GCC, тем не менее, printf() работает и здесь без c\_str().

\myparagraph{Чуть более сложный пример}

\lstinputlisting[style=customc]{\CURPATH/STL/string/3.cpp}

\lstinputlisting[caption=MSVC 2012,style=customasmx86]{\CURPATH/STL/string/3_MSVC_RU.asm}

Собственно, компилятор не конструирует строки статически: да в общем-то и как
это возможно, если буфер с ней нужно хранить в \glslink{heap}{куче}?

Вместо этого в сегменте данных хранятся обычные \ac{ASCIIZ}-строки, а позже, во время выполнения, 
при помощи метода \q{assign}, конструируются строки s1 и s2
.
При помощи \TT{operator+}, создается строка s3.

Обратите внимание на то что вызов метода c\_str() отсутствует,
потому что его код достаточно короткий и компилятор вставил его прямо здесь:
если строка короче 16-и байт, то в регистре EAX остается указатель на буфер,
а если длиннее, то из этого же места достается адрес на буфер расположенный в \glslink{heap}{куче}.

Далее следуют вызовы трех деструкторов, причем, они вызываются только если строка длиннее 16-и байт:
тогда нужно освободить буфера в \glslink{heap}{куче}.
В противном случае, так как все три объекта std::string хранятся в стеке,
они освобождаются автоматически после выхода из функции.

Следовательно, работа с короткими строками более быстрая из-за м\'{е}ньшего обращения к \glslink{heap}{куче}.

Код на GCC даже проще (из-за того, что в GCC, как мы уже видели, не реализована возможность хранить короткую
строку прямо в структуре):

% TODO1 comment each function meaning
\lstinputlisting[caption=GCC 4.8.1,style=customasmx86]{\CURPATH/STL/string/3_GCC_RU.s}

Можно заметить, что в деструкторы передается не указатель на объект,
а указатель на место за 12 байт (или 3 слова) перед ним, то есть, на настоящее начало структуры.

\myparagraph{std::string как глобальная переменная}
\label{sec:std_string_as_global_variable}

Опытные программисты на \Cpp знают, что глобальные переменные \ac{STL}-типов вполне можно объявлять.

Да, действительно:

\lstinputlisting[style=customc]{\CURPATH/STL/string/5.cpp}

Но как и где будет вызываться конструктор \TT{std::string}?

На самом деле, эта переменная будет инициализирована даже перед началом \main.

\lstinputlisting[caption=MSVC 2012: здесь конструируется глобальная переменная{,} а также регистрируется её деструктор,style=customasmx86]{\CURPATH/STL/string/5_MSVC_p2.asm}

\lstinputlisting[caption=MSVC 2012: здесь глобальная переменная используется в \main,style=customasmx86]{\CURPATH/STL/string/5_MSVC_p1.asm}

\lstinputlisting[caption=MSVC 2012: эта функция-деструктор вызывается перед выходом,style=customasmx86]{\CURPATH/STL/string/5_MSVC_p3.asm}

\myindex{\CStandardLibrary!atexit()}
В реальности, из \ac{CRT}, еще до вызова main(), вызывается специальная функция,
в которой перечислены все конструкторы подобных переменных.
Более того: при помощи atexit() регистрируется функция, которая будет вызвана в конце работы программы:
в этой функции компилятор собирает вызовы деструкторов всех подобных глобальных переменных.

GCC работает похожим образом:

\lstinputlisting[caption=GCC 4.8.1,style=customasmx86]{\CURPATH/STL/string/5_GCC.s}

Но он не выделяет отдельной функции в которой будут собраны деструкторы: 
каждый деструктор передается в atexit() по одному.

% TODO а если глобальная STL-переменная в другом модуле? надо проверить.

}
\DE{\subsection{Einfachste XOR-Verschlüsselung überhaupt}

Ich habe einmal eine Software gesehen, bei der alle Debugging-Ausgaben mit XOR mit dem Wert 3
verschlüsselt wurden. Mit anderen Worten, die beiden niedrigsten Bits aller Buchstaben wurden invertiert.

``Hello, world'' wurde zu ``Kfool/\#tlqog'':

\begin{lstlisting}
#!/usr/bin/python

msg="Hello, world!"

print "".join(map(lambda x: chr(ord(x)^3), msg))
\end{lstlisting}

Das ist eine ziemlich interessante Verschlüsselung (oder besser eine Verschleierung),
weil sie zwei wichtige Eigenschaften hat:
1) es ist eine einzige Funktion zum Verschlüsseln und entschlüsseln, sie muss nur wiederholt angewendet werden
2) die entstehenden Buchstaben befinden sich im druckbaren Bereich, also die ganze Zeichenkette kann ohne
Escape-Symbole im Code verwendet werden.

Die zweite Eigenschaft nutzt die Tatsache, dass alle druckbaren Zeichen in Reihen organisiert sind: 0x2x-0x7x,
und wenn die beiden niederwertigsten Bits invertiert werden, wird der Buchstabe um eine oder drei Stellen nach
links oder rechts \IT{verschoben}, aber niemals in eine andere Reihe:

\begin{figure}[H]
\centering
\includegraphics[width=0.7\textwidth]{ascii_clean.png}
\caption{7-Bit \ac{ASCII} Tabelle in Emacs}
\end{figure}

\dots mit dem Zeichen 0x7F als einziger Ausnahme.

Im Folgenden werden also beispielsweise die Zeichen A-Z \IT{verschlüsselt}:

\begin{lstlisting}
#!/usr/bin/python

msg="@ABCDEFGHIJKLMNO"

print "".join(map(lambda x: chr(ord(x)^3), msg))
\end{lstlisting}

Ergebnis:
% FIXME \verb  --  relevant comment for German?
\begin{lstlisting}
CBA@GFEDKJIHONML
\end{lstlisting}

Es sieht so aus als würden die Zeichen ``@'' und ``C'' sowie ``B'' und ``A'' vertauscht werden.

Hier ist noch ein interessantes Beispiel, in dem gezeigt wird, wie die Eigenschaften von XOR
ausgenutzt werden können: Exakt den gleichen Effekt, dass druckbare Zeichen auch druckbar bleiben,
kann man dadurch erzielen, dass irgendeine Kombination der niedrigsten vier Bits invertiert wird.
}

\ifdefined\SPANISH
\chapter{Patrones de código}
\fi % SPANISH

\ifdefined\GERMAN
\chapter{Code-Muster}
\fi % GERMAN

\ifdefined\ENGLISH
\chapter{Code Patterns}
\fi % ENGLISH

\ifdefined\ITALIAN
\chapter{Forme di codice}
\fi % ITALIAN

\ifdefined\RUSSIAN
\chapter{Образцы кода}
\fi % RUSSIAN

\ifdefined\BRAZILIAN
\chapter{Padrões de códigos}
\fi % BRAZILIAN

\ifdefined\THAI
\chapter{รูปแบบของโค้ด}
\fi % THAI

\ifdefined\FRENCH
\chapter{Modèle de code}
\fi % FRENCH

\ifdefined\POLISH
\chapter{\PLph{}}
\fi % POLISH

% sections
\EN{\input{patterns/patterns_opt_dbg_EN}}
\ES{\input{patterns/patterns_opt_dbg_ES}}
\ITA{\input{patterns/patterns_opt_dbg_ITA}}
\PTBR{\input{patterns/patterns_opt_dbg_PTBR}}
\RU{\input{patterns/patterns_opt_dbg_RU}}
\THA{\input{patterns/patterns_opt_dbg_THA}}
\DE{\input{patterns/patterns_opt_dbg_DE}}
\FR{\input{patterns/patterns_opt_dbg_FR}}
\PL{\input{patterns/patterns_opt_dbg_PL}}

\RU{\section{Некоторые базовые понятия}}
\EN{\section{Some basics}}
\DE{\section{Einige Grundlagen}}
\FR{\section{Quelques bases}}
\ES{\section{\ESph{}}}
\ITA{\section{Alcune basi teoriche}}
\PTBR{\section{\PTBRph{}}}
\THA{\section{\THAph{}}}
\PL{\section{\PLph{}}}

% sections:
\EN{\input{patterns/intro_CPU_ISA_EN}}
\ES{\input{patterns/intro_CPU_ISA_ES}}
\ITA{\input{patterns/intro_CPU_ISA_ITA}}
\PTBR{\input{patterns/intro_CPU_ISA_PTBR}}
\RU{\input{patterns/intro_CPU_ISA_RU}}
\DE{\input{patterns/intro_CPU_ISA_DE}}
\FR{\input{patterns/intro_CPU_ISA_FR}}
\PL{\input{patterns/intro_CPU_ISA_PL}}

\EN{\input{patterns/numeral_EN}}
\RU{\input{patterns/numeral_RU}}
\ITA{\input{patterns/numeral_ITA}}
\DE{\input{patterns/numeral_DE}}
\FR{\input{patterns/numeral_FR}}
\PL{\input{patterns/numeral_PL}}

% chapters
\input{patterns/00_empty/main}
\input{patterns/011_ret/main}
\input{patterns/01_helloworld/main}
\input{patterns/015_prolog_epilogue/main}
\input{patterns/02_stack/main}
\input{patterns/03_printf/main}
\input{patterns/04_scanf/main}
\input{patterns/05_passing_arguments/main}
\input{patterns/06_return_results/main}
\input{patterns/061_pointers/main}
\input{patterns/065_GOTO/main}
\input{patterns/07_jcc/main}
\input{patterns/08_switch/main}
\input{patterns/09_loops/main}
\input{patterns/10_strings/main}
\input{patterns/11_arith_optimizations/main}
\input{patterns/12_FPU/main}
\input{patterns/13_arrays/main}
\input{patterns/14_bitfields/main}
\EN{\input{patterns/145_LCG/main_EN}}
\RU{\input{patterns/145_LCG/main_RU}}
\input{patterns/15_structs/main}
\input{patterns/17_unions/main}
\input{patterns/18_pointers_to_functions/main}
\input{patterns/185_64bit_in_32_env/main}

\EN{\input{patterns/19_SIMD/main_EN}}
\RU{\input{patterns/19_SIMD/main_RU}}
\DE{\input{patterns/19_SIMD/main_DE}}

\EN{\input{patterns/20_x64/main_EN}}
\RU{\input{patterns/20_x64/main_RU}}

\EN{\input{patterns/205_floating_SIMD/main_EN}}
\RU{\input{patterns/205_floating_SIMD/main_RU}}
\DE{\input{patterns/205_floating_SIMD/main_DE}}

\EN{\input{patterns/ARM/main_EN}}
\RU{\input{patterns/ARM/main_RU}}
\DE{\input{patterns/ARM/main_DE}}

\input{patterns/MIPS/main}


\EN{\section{\FPUChapterName}
\label{sec:FPU}

\newcommand{\FNURLSTACK}{\footnote{\href{http://go.yurichev.com/17123}{wikipedia.org/wiki/Stack\_machine}}}
\newcommand{\FNURLFORTH}{\footnote{\href{http://go.yurichev.com/17124}{wikipedia.org/wiki/Forth\_(programming\_language)}}}
\newcommand{\FNURLIEEE}{\footnote{\href{http://go.yurichev.com/17125}{wikipedia.org/wiki/IEEE\_floating\_point}}}
\newcommand{\FNURLSP}{\footnote{\href{http://go.yurichev.com/17126}{wikipedia.org/wiki/Single-precision\_floating-point\_format}}}
\newcommand{\FNURLDP}{\footnote{\href{http://go.yurichev.com/17127}{wikipedia.org/wiki/Double-precision\_floating-point\_format}}}
\newcommand{\FNURLEP}{\footnote{\href{http://go.yurichev.com/17128}{wikipedia.org/wiki/Extended\_precision}}}

The \ac{FPU} is a device within the main \ac{CPU}, specially designed to deal with floating point numbers.

It was called \q{coprocessor} in the past and it stays somewhat aside of the main \ac{CPU}.

\subsection{IEEE 754}

A number in the IEEE 754 format consists of a \IT{sign}, a \IT{significand} (also called \IT{fraction}) and an \IT{exponent}.

\subsection{x86}

It is worth looking into stack machines\FNURLSTACK or learning the basics of the Forth language\FNURLFORTH,
before studying the \ac{FPU} in x86.

\myindex{Intel!80486}
\myindex{Intel!FPU}
It is interesting to know that in the past (before the 80486 CPU) the coprocessor was a separate chip 
and it was not always pre-installed on the motherboard. It was possible to buy it separately and install it
\footnote{For example, John Carmack used fixed-point arithmetic 
(\href{http://go.yurichev.com/17356}{wikipedia.org/wiki/Fixed-point\_arithmetic}) values in his Doom video game, stored in 
32-bit \ac{GPR} registers (16 bit for integral part and another 16 bit for fractional part), so Doom
could work on 32-bit computers without FPU, i.e., 80386 and 80486 SX.}.

Starting with the 80486 DX CPU, the \ac{FPU} is integrated in the \ac{CPU}.

\myindex{x86!\Instructions!FWAIT}
The \INS{FWAIT} instruction reminds us of that fact---it switches the \ac{CPU} to a waiting state, so it can wait until the \ac{FPU} has finished with its work.

Another rudiment is the fact that the \ac{FPU} instruction 
opcodes start with the so called \q{escape}-opcodes (\GTT{D8..DF}), i.e., 
opcodes passed to a separate coprocessor.

\myindex{IEEE 754}
\label{FPU_is_stack}

The FPU has a stack capable to holding 8 80-bit registers, and each register can hold a number 
in the IEEE 754\FNURLIEEE format.

They are \ST{0}..\ST{7}. For brevity, \IDA and \olly show \ST{0} as \GTT{ST}, 
which is represented in some textbooks and manuals as \q{Stack Top}.

\subsection{ARM, MIPS, x86/x64 SIMD}

In ARM and MIPS the FPU is not a stack, but a set of registers, which can be accessed randomly, like \ac{GPR}.

The same ideology is used in the SIMD extensions of x86/x64 CPUs.

\subsection{\CCpp}

\myindex{float}
\myindex{double}

The standard \CCpp languages offer at least two floating number types, \Tfloat (\IT{single-precision}\FNURLSP, 32 bits)
\footnote{the single precision floating point number format is also addressed in 
the \IT{\WorkingWithFloatAsWithStructSubSubSectionName}~(\myref{sec:floatasstruct}) section}
and \Tdouble (\IT{double-precision}\FNURLDP, 64 bits).

In \InSqBrackets{\TAOCPvolII 246} we can find the \IT{single-precision} means that the floating point value can be placed into a single
[32-bit] machine word, \IT{double-precision} means it can be stored in two words (64 bits).

\myindex{long double}

GCC also supports the \IT{long double} type (\IT{extended precision}\FNURLEP, 80 bit), which MSVC doesn't.

The \Tfloat type requires the same number of bits as the \Tint type in 32-bit environments, 
but the number representation is completely different.

\input{patterns/12_FPU/1_simple/main}
\input{patterns/12_FPU/2_passing_floats/main}
\input{patterns/12_FPU/3_comparison/main}

\subsection{Some constants}

It's easy to find representations of some constants in Wikipedia for IEEE 754 encoded numbers.
It's interesting to know that 0.0 in IEEE 754 is represented as 32 zero bits (for single precision) or 64 zero bits
(for double).
So in order to set a floating point variable to 0.0 in register or memory, one can use \MOV or \TT{XOR reg, reg} instruction.
\myindex{\CStandardLibrary!memset()}
This is suitable for structures where many variables present of various data types.
With usual memset() function one can set all integer variables to 0, all boolean variables to \IT{false}, all pointers
to NULL, and all floating point variables (of any precision) to 0.0.

\subsection{Copying}

One may think inertially that \INS{FLD}/\INS{FST} instructions must be used to load and store (and hence, copy) IEEE 754 values.
Nevertheless, same can be achieved easier by usual \INS{MOV} instruction, which, of course, copies values bitwisely.

\subsection{Stack, calculators and reverse Polish notation}

\myindex{Reverse Polish notation}

Now we understand why some old calculators use reverse Polish notation
\footnote{\href{http://go.yurichev.com/17354}{wikipedia.org/wiki/Reverse\_Polish\_notation}}.

For example, for addition of 12 and 34 one has to enter 12, then 34, then press \q{plus} sign.

It's because old calculators were just stack machine implementations, and this was much simpler
than to handle complex parenthesized expressions.

\subsection{80 bits?}

\myindex{Punched card}
Internal numbers representation in FPU --- 80-bit.
Strange number, because the number not in $2^n$ form.
There is a hypothesis that this is probably due to historical reasons---the standard IBM puched card can encode 12 rows of 80 bits.
$80\cdot 25$ text mode resolution was also popular in past.
If you know better, please a drop email to the author: \EMAIL{}.

\subsection{x64}

On how floating point numbers are processed in x86-64, read more here: \myref{floating_SIMD}.

% sections
\input{patterns/12_FPU/exercises}
}
\RU{\section{\FPUChapterName}
\label{sec:FPU}

\newcommand{\FNURLSTACK}{\footnote{\href{http://go.yurichev.com/17123}{wikipedia.org/wiki/Stack\_machine}}}
\newcommand{\FNURLFORTH}{\footnote{\href{http://go.yurichev.com/17124}{wikipedia.org/wiki/Forth\_(programming\_language)}}}
\newcommand{\FNURLIEEE}{\footnote{\href{http://go.yurichev.com/17125}{wikipedia.org/wiki/IEEE\_floating\_point}}}
\newcommand{\FNURLSP}{\footnote{\href{http://go.yurichev.com/17126}{wikipedia.org/wiki/Single-precision\_floating-point\_format}}}
\newcommand{\FNURLDP}{\footnote{\href{http://go.yurichev.com/17127}{wikipedia.org/wiki/Double-precision\_floating-point\_format}}}
\newcommand{\FNURLEP}{\footnote{\href{http://go.yurichev.com/17128}{wikipedia.org/wiki/Extended\_precision}}}

\ac{FPU}~--- блок в процессоре работающий с числами с плавающей запятой.

Раньше он назывался \q{сопроцессором} и он стоит немного в стороне от \ac{CPU}.

\subsection{IEEE 754}

Число с плавающей точкой в формате IEEE 754 состоит из \IT{знака}, \IT{мантиссы}\footnote{\IT{significand} или \IT{fraction} 
в англоязычной литературе} и \IT{экспоненты}.

\subsection{x86}

Перед изучением \ac{FPU} в x86 полезно ознакомиться с тем как работают стековые машины\FNURLSTACK 
или ознакомиться с основами языка Forth\FNURLFORTH.

\myindex{Intel!80486}
\myindex{Intel!FPU}
Интересен факт, что в свое время (до 80486) сопроцессор был отдельным чипом на материнской плате, 
и вследствие его высокой цены, он не всегда присутствовал. Его можно было докупить и установить отдельно
\footnote{Например, Джон Кармак использовал в своей игре Doom числа с фиксированной запятой 
(\href{http://go.yurichev.com/17357}{ru.wikipedia.org/wiki/Число\_с\_фиксированной\_запятой}), хранящиеся
в обычных 32-битных \ac{GPR} (16 бит на целую часть и 16 на дробную),
чтобы Doom работал на 32-битных компьютерах без FPU, т.е. 80386 и 80486 SX.}.
Начиная с 80486 DX в состав процессора всегда входит FPU.

\myindex{x86!\Instructions!FWAIT}
Этот факт может напоминать такой рудимент как наличие инструкции \INS{FWAIT}, которая заставляет
\ac{CPU} ожидать, пока \ac{FPU} закончит работу.
Другой рудимент это тот факт, что опкоды \ac{FPU}-инструкций начинаются с т.н. \q{escape}-опкодов 
(\GTT{D8..DF}) как опкоды, передающиеся в отдельный сопроцессор.

\myindex{IEEE 754}
\label{FPU_is_stack}
FPU имеет стек из восьми 80-битных регистров: \ST{0}..\ST{7}.
Для краткости, \IDA и \olly отображают \ST{0} как \GTT{ST},
что в некоторых учебниках и документациях означает \q{Stack Top} (\q{вершина стека}).
Каждый регистр может содержать число в формате IEEE 754\FNURLIEEE.

\subsection{ARM, MIPS, x86/x64 SIMD}

В ARM и MIPS FPU это не стек, а просто набор регистров, к которым можно обращаться произвольно, как к \ac{GPR}.

Такая же идеология применяется в расширениях SIMD в процессорах x86/x64.

\subsection{\CCpp}

\myindex{float}
\myindex{double}
В стандартных \CCpp имеются два типа для работы с числами с плавающей запятой: 
\Tfloat (\IT{число одинарной точности}\FNURLSP, 32 бита)
\footnote{Формат представления чисел с плавающей точкой одинарной точности затрагивается в разделе 
\IT{\WorkingWithFloatAsWithStructSubSubSectionName}~(\myref{sec:floatasstruct}).}
и \Tdouble (\IT{число двойной точности}\FNURLDP, 64 бита).

В \InSqBrackets{\TAOCPvolII 246} мы можем найти что \IT{single-precision} означает, что значение с плавающей точкой может быть
помещено в одно [32-битное] машинное слово, а \IT{doulbe-precision} означает, что оно размещено в двух словах (64 бита).

\myindex{long double}
GCC также поддерживает тип \IT{long double} (\IT{extended precision}\FNURLEP, 80 бит), но MSVC~--- нет.

Несмотря на то, что \Tfloat занимает столько же места, сколько и \Tint на 32-битной архитектуре, 
представление чисел, разумеется, совершенно другое.

\input{patterns/12_FPU/1_simple/main}
\input{patterns/12_FPU/2_passing_floats/main}
\input{patterns/12_FPU/3_comparison/main}

\subsection{Некоторые константы}

В Wikipedia легко найти представление некоторых констант в IEEE 754.
Любопытно узнать, что 0.0 в IEEE 754 представляется как 32 нулевых бита (для одинарной точности) или 64 нулевых бита
(для двойной).
Так что, для записи числа 0.0 в переменную в памяти или регистр, можно пользоваться инструкцией \MOV, или \TT{XOR reg, reg}.
\myindex{\CStandardLibrary!memset()}
Это тем может быть удобно, что если в структуре есть много переменных разных типов, то обычной ф-ций memset()
можно установить все целочисленные переменные в 0, все булевы переменные в \IT{false}, все указатели в NULL,
и все переменные с плавающей точкой (любой точности) в 0.0.

\subsection{Копирование}

По инерции можно подумать, что для загрузки и сохранения (и, следовательно, копирования) чисел в формате
IEEE 754 нужно использовать пару инструкций \INS{FLD}/\INS{FST}.
Тем не менее, этого куда легче достичь используя обычную инструкцию \INS{MOV},
которая, конечно же, просто копирует значения побитово.

\subsection{Стек, калькуляторы и обратная польская запись}

\myindex{Обратная польская запись}
Теперь понятно, почему некоторые старые калькуляторы используют обратную польскую запись
\footnote{\href{http://go.yurichev.com/17355}{ru.wikipedia.org/wiki/Обратная\_польская\_запись}}.

Например для сложения 12 и 34 нужно было набрать 12, потом 34, потом нажать знак \q{плюс}.

Это потому что старые калькуляторы просто реализовали стековую машину и это было куда проще, чем обрабатывать сложные выражения со скобками.

\subsection{80 бит?}

\myindex{Перфокарты}
Внутреннее представление чисел с FPU --- 80-битное.
Странное число, потому как не является числом вида $2^n$.
Имеется гипотеза, что причина, возможно, историческая --- стандартные IBM-овские перфокарты могли кодировать 12 строк по 80 бит.
Раньше было также популярно текстовое разрешение $80 \cdot 25$.
Если вы знаете более точную причину, просьба сообщить автору: \EMAIL{}.

\subsection{x64}

О том, как происходит работа с числами с плавающей запятой в x86-64, читайте здесь: \myref{floating_SIMD}.

% sections
\input{patterns/12_FPU/exercises}

}
\DE{\subsection{Einfachste XOR-Verschlüsselung überhaupt}

Ich habe einmal eine Software gesehen, bei der alle Debugging-Ausgaben mit XOR mit dem Wert 3
verschlüsselt wurden. Mit anderen Worten, die beiden niedrigsten Bits aller Buchstaben wurden invertiert.

``Hello, world'' wurde zu ``Kfool/\#tlqog'':

\begin{lstlisting}
#!/usr/bin/python

msg="Hello, world!"

print "".join(map(lambda x: chr(ord(x)^3), msg))
\end{lstlisting}

Das ist eine ziemlich interessante Verschlüsselung (oder besser eine Verschleierung),
weil sie zwei wichtige Eigenschaften hat:
1) es ist eine einzige Funktion zum Verschlüsseln und entschlüsseln, sie muss nur wiederholt angewendet werden
2) die entstehenden Buchstaben befinden sich im druckbaren Bereich, also die ganze Zeichenkette kann ohne
Escape-Symbole im Code verwendet werden.

Die zweite Eigenschaft nutzt die Tatsache, dass alle druckbaren Zeichen in Reihen organisiert sind: 0x2x-0x7x,
und wenn die beiden niederwertigsten Bits invertiert werden, wird der Buchstabe um eine oder drei Stellen nach
links oder rechts \IT{verschoben}, aber niemals in eine andere Reihe:

\begin{figure}[H]
\centering
\includegraphics[width=0.7\textwidth]{ascii_clean.png}
\caption{7-Bit \ac{ASCII} Tabelle in Emacs}
\end{figure}

\dots mit dem Zeichen 0x7F als einziger Ausnahme.

Im Folgenden werden also beispielsweise die Zeichen A-Z \IT{verschlüsselt}:

\begin{lstlisting}
#!/usr/bin/python

msg="@ABCDEFGHIJKLMNO"

print "".join(map(lambda x: chr(ord(x)^3), msg))
\end{lstlisting}

Ergebnis:
% FIXME \verb  --  relevant comment for German?
\begin{lstlisting}
CBA@GFEDKJIHONML
\end{lstlisting}

Es sieht so aus als würden die Zeichen ``@'' und ``C'' sowie ``B'' und ``A'' vertauscht werden.

Hier ist noch ein interessantes Beispiel, in dem gezeigt wird, wie die Eigenschaften von XOR
ausgenutzt werden können: Exakt den gleichen Effekt, dass druckbare Zeichen auch druckbar bleiben,
kann man dadurch erzielen, dass irgendeine Kombination der niedrigsten vier Bits invertiert wird.
}
\FR{\section{\FPUChapterName}
\label{sec:FPU}

\newcommand{\FNURLSTACK}{\footnote{\href{http://go.yurichev.com/17123}{wikipedia.org/wiki/Stack\_machine}}}
\newcommand{\FNURLFORTH}{\footnote{\href{http://go.yurichev.com/17124}{wikipedia.org/wiki/Forth\_(programming\_language)}}}
\newcommand{\FNURLIEEE}{\footnote{\href{http://go.yurichev.com/17125}{wikipedia.org/wiki/IEEE\_floating\_point}}}
\newcommand{\FNURLSP}{\footnote{\href{http://go.yurichev.com/17126}{wikipedia.org/wiki/Single-precision\_floating-point\_format}}}
\newcommand{\FNURLDP}{\footnote{\href{http://go.yurichev.com/17127}{wikipedia.org/wiki/Double-precision\_floating-point\_format}}}
\newcommand{\FNURLEP}{\footnote{\href{http://go.yurichev.com/17128}{wikipedia.org/wiki/Extended\_precision}}}

Le \ac{FPU} est un dispositif à l'intérieur du \ac{CPU}, spécialement conçu pour
traiter les nombres à virgules flottantes.

Il était appelé \q{coprocesseur} dans le passé et il était en dehors du \ac{CPU}.

\subsection{IEEE 754}

Un nombre au format IEEE 754 consiste en un \IT{signe}, un \IT{significande} (aussi
appelé \IT{fraction}) et un \IT{exposant}.

\subsection{x86}

Ca vaut la peine de jetter un oeil sur les machines à base de piles ou d'apprendre
les bases du langage Forth\FNURLFORTH, avant d'étudier le \ac{FPU} en x86.

\myindex{Intel!80486}
\myindex{Intel!FPU}
Il est intéressant de savoir que dans le passé (avant le CPU 80486) le coprocesseur
était une puce séparée et n'était pas toujours pré-installé sur la carte mère. Il
était possible de l'acheter séparemment et de l'installer\footnote{Par exemple, John
Carmack a utilisé des valeurs arithmétiques à virgule fixe
(\href{http://go.yurichev.com/17356}{wikipedia.org/wiki/Fixed-point\_arithmetic})
dans son jeu vidéo Doom, stockées dans des registres 32-bit \ac{GPR} (16 bit pour
la partie entière et 16 bit pour la partie fractionnaire), donc Doom pouvait fonctionner
sur des ordinateurs 32-bit sans FPU, i.e., 80386 et 80486 SX.}.

A partir du CPU 80486 DX, le \ac{FPU} est intégré dans le \ac{CPU}.

\myindex{x86!\Instructions!FWAIT}
L'instruction \INS{FWAIT} nous rapelle le fait qu'elle passe le \ac{CPU} dans un
état d'attente, afin d'attendre que le \ac{FPU} ait fini son traitement.

Un autre rudiment est le fait que les opcodes d'instruction \ac{FPU} commencent
avec ce qui est appelé l'opcode-\q{d'échappement} (\GTT{D8..DF}), i.e., opcodes
passés à un coprocesseur séparé.

\myindex{IEEE 754}
\label{FPU_is_stack}

Le FPU a une pile capable de contenir 8 registres de 80-bit, et chaque registre peut
contenir un nombre au format IEEE 754\FNURLIEEE.

Ce sont \ST{0}..\ST{7}. Par concision, \IDA et \olly montrent \ST{0} comme \GTT{ST},
qui est représenté dans certains livres et manuels comme \q{Stack Top}.

\subsection{ARM, MIPS, x86/x64 SIMD}

En ARM et MIPS le FPU n'a pas de pile, mais un ensemble de registres.

La même idéologie est utilisée dans l'extension SIMD des CPUs x86/x64.

\subsection{\CCpp}

\myindex{float}
\myindex{double}

Le standard des langages \CCpp offre au moins deux types de nombres à virgule flottante,
\Tfloat (\IT{simple-précision}\FNURLSP, 32 bits) \footnote{le format des nombres
à virgule flottante simple précision est aussi abordé dans la section \IT{\WorkingWithFloatAsWithStructSubSubSectionName}~(\myref{sec:floatasstruct})}
et \Tdouble  (\IT{double-précision}\FNURLDP, 64 bits).

Dans \InSqBrackets{\TAOCPvolII 246} nous pouvons trouver que \IT{simple-précision}
signifie que la valeur flottante peut être stockée dans un simple mot machine [32-bit],
\IT{double-précision} signifie qu'elle peut être stockée dans deux mots (64 bits).

\myindex{long double}

GCC supporte également le type \IT{long double} (\IT{précision étendue}\FNURLEP,
80 bit), que MSVC ne supporte pas.

Le type \Tfloat nécessite le même nombre de bits que le type \Tint dans les environnements
32-bit, mais la représentation du nombre est complètement différente.

\input{patterns/12_FPU/1_simple/main}
\input{patterns/12_FPU/2_passing_floats/main}
\input{patterns/12_FPU/3_comparison/main}

\subsection{Quelques constantes}

Il est facile de trouver la représentation de certaines constantes pour des nombres
encodés au format IEEE 754 sur Wikipedia.
Il est intéressant de savoir que 0,0 en IEEE 754 est représenté par 32 bits à zéro
(pour la simple précision) ou 64 bits à zéro (pour la double).
Donc pour mettre une variable flottante à 0,0 dans un registre ou en mémoire, on
peut utiliser l'instruction \MOV ou \TT{XOR reg, reg}.
\myindex{\CStandardLibrary!memset()}
Ceci est utilisable pour les structures où des variables de types variés sont présentes.
Avec la fonction usuelle memset() il est possible de mettre toutes les variables
entières à 0, toutes les variables booléennes à \IT{false}, tous les pointeurs à
NULL, et toutes les variables flottantes (de n'importe quelle précision) à 0,0.

\subsection{Copie}

On peut tout d'abord penser qu'il faut utiliser les instructions \INS{FLD}/\INS{FST}
pour charger et stocker (et donc, copier) des valeurs IEEE 754.
Néanmoins, la même chose peut-être effectuée plus facilement avec l'instruction usuelle
\INS{MOV}, qui, bien sûr, copie les valeurs au niveau binaire.

\subsection{Pile, calculateurs et notation polonaise inverse}

\myindex{Notation polonaise inverse}

Maintenant nous comprenons pourquoi certains anciens calculateurs utilisent la notation
Polonaise inverse
\footnote{\href{http://go.yurichev.com/17354}{wikipedia.org/wiki/Reverse\_Polish\_notation}}.

Par exemple, pour aditionner 12 et 34, on doit entrer 12, puis 34, et presser le signe
\q{plus}.

C'est parce que les anciens calculateurs étaient juste des implémentations de machine
à pile, et c'était bien plus simple que de manipuler des expressions complexes avec
parenthèses.
\subsection{x64}

Sur la manière dont sont traités les nombres à virgules flottante en x86-64, lire ici: \myref{floating_SIMD}.

% sections
\input{patterns/12_FPU/exercises}
}


\ifdefined\SPANISH
\chapter{Patrones de código}
\fi % SPANISH

\ifdefined\GERMAN
\chapter{Code-Muster}
\fi % GERMAN

\ifdefined\ENGLISH
\chapter{Code Patterns}
\fi % ENGLISH

\ifdefined\ITALIAN
\chapter{Forme di codice}
\fi % ITALIAN

\ifdefined\RUSSIAN
\chapter{Образцы кода}
\fi % RUSSIAN

\ifdefined\BRAZILIAN
\chapter{Padrões de códigos}
\fi % BRAZILIAN

\ifdefined\THAI
\chapter{รูปแบบของโค้ด}
\fi % THAI

\ifdefined\FRENCH
\chapter{Modèle de code}
\fi % FRENCH

\ifdefined\POLISH
\chapter{\PLph{}}
\fi % POLISH

% sections
\EN{\section{The method}

When the author of this book first started learning C and, later, \Cpp, he used to write small pieces of code, compile them,
and then look at the assembly language output. This made it very easy for him to understand what was going on in the code that he had written.
\footnote{In fact, he still does this when he can't understand what a particular bit of code does.}.
He did this so many times that the relationship between the \CCpp code and what the compiler produced was imprinted deeply in his mind.
It's now easy for him to imagine instantly a rough outline of a C code's appearance and function.
Perhaps this technique could be helpful for others.

%There are a lot of examples for both x86/x64 and ARM.
%Those who already familiar with one of architectures, may freely skim over pages.

By the way, there is a great website where you can do the same, with various compilers, instead of installing them on your box.
You can use it as well: \url{https://gcc.godbolt.org/}.

\section*{\Exercises}

When the author of this book studied assembly language, he also often compiled small C functions and then rewrote
them gradually to assembly, trying to make their code as short as possible.
This probably is not worth doing in real-world scenarios today,
because it's hard to compete with the latest compilers in terms of efficiency. It is, however, a very good way to gain a better understanding of assembly.
Feel free, therefore, to take any assembly code from this book and try to make it shorter.
However, don't forget to test what you have written.

% rewrote to show that debug\release and optimisations levels are orthogonal concepts.
\section*{Optimization levels and debug information}

Source code can be compiled by different compilers with various optimization levels.
A typical compiler has about three such levels, where level zero means that optimization is completely disabled.
Optimization can also be targeted towards code size or code speed.
A non-optimizing compiler is faster and produces more understandable (albeit verbose) code,
whereas an optimizing compiler is slower and tries to produce code that runs faster (but is not necessarily more compact).
In addition to optimization levels, a compiler can include some debug information in the resulting file,
producing code that is easy to debug.
One of the important features of the ´debug' code is that it might contain links
between each line of the source code and its respective machine code address.
Optimizing compilers, on the other hand, tend to produce output where entire lines of source code
can be optimized away and thus not even be present in the resulting machine code.
Reverse engineers can encounter either version, simply because some developers turn on the compiler's optimization flags and others do not.
Because of this, we'll try to work on examples of both debug and release versions of the code featured in this book, wherever possible.

Sometimes some pretty ancient compilers are used in this book, in order to get the shortest (or simplest) possible code snippet.
}
\ES{\input{patterns/patterns_opt_dbg_ES}}
\ITA{\input{patterns/patterns_opt_dbg_ITA}}
\PTBR{\input{patterns/patterns_opt_dbg_PTBR}}
\RU{\input{patterns/patterns_opt_dbg_RU}}
\THA{\input{patterns/patterns_opt_dbg_THA}}
\DE{\input{patterns/patterns_opt_dbg_DE}}
\FR{\input{patterns/patterns_opt_dbg_FR}}
\PL{\input{patterns/patterns_opt_dbg_PL}}

\RU{\section{Некоторые базовые понятия}}
\EN{\section{Some basics}}
\DE{\section{Einige Grundlagen}}
\FR{\section{Quelques bases}}
\ES{\section{\ESph{}}}
\ITA{\section{Alcune basi teoriche}}
\PTBR{\section{\PTBRph{}}}
\THA{\section{\THAph{}}}
\PL{\section{\PLph{}}}

% sections:
\EN{\input{patterns/intro_CPU_ISA_EN}}
\ES{\input{patterns/intro_CPU_ISA_ES}}
\ITA{\input{patterns/intro_CPU_ISA_ITA}}
\PTBR{\input{patterns/intro_CPU_ISA_PTBR}}
\RU{\input{patterns/intro_CPU_ISA_RU}}
\DE{\input{patterns/intro_CPU_ISA_DE}}
\FR{\input{patterns/intro_CPU_ISA_FR}}
\PL{\input{patterns/intro_CPU_ISA_PL}}

\EN{\subsection{Numeral Systems}

Humans have become accustomed to a decimal numeral system, probably because almost everyone has 10 fingers.
Nevertheless, the number \q{10} has no significant meaning in science and mathematics.
The natural numeral system in digital electronics is binary: 0 is for an absence of current in the wire, and 1 for presence.
10 in binary is 2 in decimal, 100 in binary is 4 in decimal, and so on.

% This sentence is a bit unweildy - maybe try 'Our ten-digit system would be described as having a radix...' - Renaissance
If the numeral system has 10 digits, it has a \IT{radix} (or \IT{base}) of 10.
The binary numeral system has a \IT{radix} of 2.

Important things to recall:

1) A \IT{number} is a number, while a \IT{digit} is a term from writing systems, and is usually one character

% The original is 'number' is not changed; I think the intent is value, and changed it - Renaissance
2) The value of a number does not change when converted to another radix; only the writing notation for that value has changed (and therefore the way of representing it in \ac{RAM}).

\subsection{Converting From One Radix To Another}

Positional notation is used almost every numerical system. This means that a digit has weight relative to where it is placed inside of the larger number.
If 2 is placed at the rightmost place, it's 2, but if it's placed one digit before rightmost, it's 20.

What does $1234$ stand for?

$10^3 \cdot 1 + 10^2 \cdot 2 + 10^1 \cdot 3 + 1 \cdot 4 = 1234$ or
$1000 \cdot 1 + 100 \cdot 2 + 10 \cdot 3 + 4 = 1234$

It's the same story for binary numbers, but the base is 2 instead of 10.
What does 0b101011 stand for?

$2^5 \cdot 1 + 2^4 \cdot 0 + 2^3 \cdot 1 + 2^2 \cdot 0 + 2^1 \cdot 1 + 2^0 \cdot 1 = 43$ or
$32 \cdot 1 + 16 \cdot 0 + 8 \cdot 1 + 4 \cdot 0 + 2 \cdot 1 + 1 = 43$

There is such a thing as non-positional notation, such as the Roman numeral system.
\footnote{About numeric system evolution, see \InSqBrackets{\TAOCPvolII{}, 195--213.}}.
% Maybe add a sentence to fill in that X is always 10, and is therefore non-positional, even though putting an I before subtracts and after adds, and is in that sense positional
Perhaps, humankind switched to positional notation because it's easier to do basic operations (addition, multiplication, etc.) on paper by hand.

Binary numbers can be added, subtracted and so on in the very same as taught in schools, but only 2 digits are available.

Binary numbers are bulky when represented in source code and dumps, so that is where the hexadecimal numeral system can be useful.
A hexadecimal radix uses the digits 0..9, and also 6 Latin characters: A..F.
Each hexadecimal digit takes 4 bits or 4 binary digits, so it's very easy to convert from binary number to hexadecimal and back, even manually, in one's mind.

\begin{center}
\begin{longtable}{ | l | l | l | }
\hline
\HeaderColor hexadecimal & \HeaderColor binary & \HeaderColor decimal \\
\hline
0	&0000	&0 \\
1	&0001	&1 \\
2	&0010	&2 \\
3	&0011	&3 \\
4	&0100	&4 \\
5	&0101	&5 \\
6	&0110	&6 \\
7	&0111	&7 \\
8	&1000	&8 \\
9	&1001	&9 \\
A	&1010	&10 \\
B	&1011	&11 \\
C	&1100	&12 \\
D	&1101	&13 \\
E	&1110	&14 \\
F	&1111	&15 \\
\hline
\end{longtable}
\end{center}

How can one tell which radix is being used in a specific instance?

Decimal numbers are usually written as is, i.e., 1234. Some assemblers allow an identifier on decimal radix numbers, in which the number would be written with a "d" suffix: 1234d.

Binary numbers are sometimes prepended with the "0b" prefix: 0b100110111 (\ac{GCC} has a non-standard language extension for this\footnote{\url{https://gcc.gnu.org/onlinedocs/gcc/Binary-constants.html}}).
There is also another way: using a "b" suffix, for example: 100110111b.
This book tries to use the "0b" prefix consistently throughout the book for binary numbers.

Hexadecimal numbers are prepended with "0x" prefix in \CCpp and other \ac{PL}s: 0x1234ABCD.
Alternatively, they are given a "h" suffix: 1234ABCDh. This is common way of representing them in assemblers and debuggers.
In this convention, if the number is started with a Latin (A..F) digit, a 0 is added at the beginning: 0ABCDEFh.
There was also convention that was popular in 8-bit home computers era, using \$ prefix, like \$ABCD.
The book will try to stick to "0x" prefix throughout the book for hexadecimal numbers.

Should one learn to convert numbers mentally? A table of 1-digit hexadecimal numbers can easily be memorized.
As for larger numbers, it's probably not worth tormenting yourself.

Perhaps the most visible hexadecimal numbers are in \ac{URL}s.
This is the way that non-Latin characters are encoded.
For example:
\url{https://en.wiktionary.org/wiki/na\%C3\%AFvet\%C3\%A9} is the \ac{URL} of Wiktionary article about \q{naïveté} word.

\subsubsection{Octal Radix}

Another numeral system heavily used in the past of computer programming is octal. In octal there are 8 digits (0..7), and each is mapped to 3 bits, so it's easy to convert numbers back and forth.
It has been superseded by the hexadecimal system almost everywhere, but, surprisingly, there is a *NIX utility, used often by many people, which takes octal numbers as argument: \TT{chmod}.

\myindex{UNIX!chmod}
As many *NIX users know, \TT{chmod} argument can be a number of 3 digits. The first digit represents the rights of the owner of the file (read, write and/or execute), the second is the rights for the group to which the file belongs, and the third is for everyone else.
Each digit that \TT{chmod} takes can be represented in binary form:

\begin{center}
\begin{longtable}{ | l | l | l | }
\hline
\HeaderColor decimal & \HeaderColor binary & \HeaderColor meaning \\
\hline
7	&111	&\textbf{rwx} \\
6	&110	&\textbf{rw-} \\
5	&101	&\textbf{r-x} \\
4	&100	&\textbf{r-{}-} \\
3	&011	&\textbf{-wx} \\
2	&010	&\textbf{-w-} \\
1	&001	&\textbf{-{}-x} \\
0	&000	&\textbf{-{}-{}-} \\
\hline
\end{longtable}
\end{center}

So each bit is mapped to a flag: read/write/execute.

The importance of \TT{chmod} here is that the whole number in argument can be represented as octal number.
Let's take, for example, 644.
When you run \TT{chmod 644 file}, you set read/write permissions for owner, read permissions for group and again, read permissions for everyone else.
If we convert the octal number 644 to binary, it would be \TT{110100100}, or, in groups of 3 bits, \TT{110 100 100}.

Now we see that each triplet describe permissions for owner/group/others: first is \TT{rw-}, second is \TT{r--} and third is \TT{r--}.

The octal numeral system was also popular on old computers like PDP-8, because word there could be 12, 24 or 36 bits, and these numbers are all divisible by 3, so the octal system was natural in that environment.
Nowadays, all popular computers employ word/address sizes of 16, 32 or 64 bits, and these numbers are all divisible by 4, so the hexadecimal system is more natural there.

The octal numeral system is supported by all standard \CCpp compilers.
This is a source of confusion sometimes, because octal numbers are encoded with a zero prepended, for example, 0377 is 255.
Sometimes, you might make a typo and write "09" instead of 9, and the compiler would report an error.
GCC might report something like this:\\
\TT{error: invalid digit "9" in octal constant}.

Also, the octal system is somewhat popular in Java. When the IDA shows Java strings with non-printable characters,
they are encoded in the octal system instead of hexadecimal.
\myindex{JAD}
The JAD Java decompiler behaves the same way.

\subsubsection{Divisibility}

When you see a decimal number like 120, you can quickly deduce that it's divisible by 10, because the last digit is zero.
In the same way, 123400 is divisible by 100, because the two last digits are zeros.

Likewise, the hexadecimal number 0x1230 is divisible by 0x10 (or 16), 0x123000 is divisible by 0x1000 (or 4096), etc.

The binary number 0b1000101000 is divisible by 0b1000 (8), etc.

This property can often be used to quickly realize if the size of some block in memory is padded to some boundary.
For example, sections in \ac{PE} files are almost always started at addresses ending with 3 hexadecimal zeros: 0x41000, 0x10001000, etc.
The reason behind this is the fact that almost all \ac{PE} sections are padded to a boundary of 0x1000 (4096) bytes.

\subsubsection{Multi-Precision Arithmetic and Radix}

\index{RSA}
Multi-precision arithmetic can use huge numbers, and each one may be stored in several bytes.
For example, RSA keys, both public and private, span up to 4096 bits, and maybe even more.

% I'm not sure how to change this, but the normal format for quoting would be just to mention the author or book, and footnote to the full reference
In \InSqBrackets{\TAOCPvolII, 265} we find the following idea: when you store a multi-precision number in several bytes,
the whole number can be represented as having a radix of $2^8=256$, and each digit goes to the corresponding byte.
Likewise, if you store a multi-precision number in several 32-bit integer values, each digit goes to each 32-bit slot,
and you may think about this number as stored in radix of $2^{32}$.

\subsubsection{How to Pronounce Non-Decimal Numbers}

Numbers in a non-decimal base are usually pronounced by digit by digit: ``one-zero-zero-one-one-...''.
Words like ``ten'' and ``thousand'' are usually not pronounced, to prevent confusion with the decimal base system.

\subsubsection{Floating point numbers}

To distinguish floating point numbers from integers, they are usually written with ``.0'' at the end,
like $0.0$, $123.0$, etc.
}
\RU{\subsection{Представление чисел}

Люди привыкли к десятичной системе счисления вероятно потому что почти у каждого есть по 10 пальцев.
Тем не менее, число 10 не имеет особого значения в науке и математике.
Двоичная система естествена для цифровой электроники: 0 означает отсутствие тока в проводе и 1 --- его присутствие.
10 в двоичной системе это 2 в десятичной; 100 в двоичной это 4 в десятичной, итд.

Если в системе счисления есть 10 цифр, её \IT{основание} или \IT{radix} это 10.
Двоичная система имеет \IT{основание} 2.

Важные вещи, которые полезно вспомнить:
1) \IT{число} это число, в то время как \IT{цифра} это термин из системы письменности, и это обычно один символ;
2) само число не меняется, когда конвертируется из одного основания в другое: меняется способ его записи (или представления
в памяти).

Как сконвертировать число из одного основания в другое?

Позиционная нотация используется почти везде, это означает, что всякая цифра имеет свой вес, в зависимости от её расположения
внутри числа.
Если 2 расположена в самом последнем месте справа, это 2.
Если она расположена в месте перед последним, это 20.

Что означает $1234$?

$10^3 \cdot 1 + 10^2 \cdot 2 + 10^1 \cdot 3 + 1 \cdot 4$ = 1234 или
$1000 \cdot 1 + 100 \cdot 2 + 10 \cdot 3 + 4 = 1234$

Та же история и для двоичных чисел, только основание там 2 вместо 10.
Что означает 0b101011?

$2^5 \cdot 1 + 2^4 \cdot 0 + 2^3 \cdot 1 + 2^2 \cdot 0 + 2^1 \cdot 1 + 2^0 \cdot 1 = 43$ или
$32 \cdot 1 + 16 \cdot 0 + 8 \cdot 1 + 4 \cdot 0 + 2 \cdot 1 + 1 = 43$

Позиционную нотацию можно противопоставить непозиционной нотации, такой как римская система записи чисел
\footnote{Об эволюции способов записи чисел, см.также: \InSqBrackets{\TAOCPvolII{}, 195--213.}}.
Вероятно, человечество перешло на позиционную нотацию, потому что так проще работать с числами (сложение, умножение, итд)
на бумаге, в ручную.

Действительно, двоичные числа можно складывать, вычитать, итд, точно также, как этому обычно обучают в школах,
только доступны лишь 2 цифры.

Двоичные числа громоздки, когда их используют в исходных кодах и дампах, так что в этих случаях применяется шестнадцатеричная
система.
Используются цифры 0..9 и еще 6 латинских букв: A..F.
Каждая шестнадцатеричная цифра занимает 4 бита или 4 двоичных цифры, так что конвертировать из двоичной системы в
шестнадцатеричную и назад, можно легко вручную, или даже в уме.

\begin{center}
\begin{longtable}{ | l | l | l | }
\hline
\HeaderColor шестнадцатеричная & \HeaderColor двоичная & \HeaderColor десятичная \\
\hline
0	&0000	&0 \\
1	&0001	&1 \\
2	&0010	&2 \\
3	&0011	&3 \\
4	&0100	&4 \\
5	&0101	&5 \\
6	&0110	&6 \\
7	&0111	&7 \\
8	&1000	&8 \\
9	&1001	&9 \\
A	&1010	&10 \\
B	&1011	&11 \\
C	&1100	&12 \\
D	&1101	&13 \\
E	&1110	&14 \\
F	&1111	&15 \\
\hline
\end{longtable}
\end{center}

Как понять, какое основание используется в конкретном месте?

Десятичные числа обычно записываются как есть, т.е., 1234. Но некоторые ассемблеры позволяют подчеркивать
этот факт для ясности, и это число может быть дополнено суффиксом "d": 1234d.

К двоичным числам иногда спереди добавляют префикс "0b": 0b100110111
(В \ac{GCC} для этого есть нестандартное расширение языка
\footnote{\url{https://gcc.gnu.org/onlinedocs/gcc/Binary-constants.html}}).
Есть также еще один способ: суффикс "b", например: 100110111b.
В этой книге я буду пытаться придерживаться префикса "0b" для двоичных чисел.

Шестнадцатеричные числа имеют префикс "0x" в \CCpp и некоторых других \ac{PL}: 0x1234ABCD.
Либо они имеют суффикс "h": 1234ABCDh --- обычно так они представляются в ассемблерах и отладчиках.
Если число начинается с цифры A..F, перед ним добавляется 0: 0ABCDEFh.
Во времена 8-битных домашних компьютеров, был также способ записи чисел используя префикс \$, например, \$ABCD.
В книге я попытаюсь придерживаться префикса "0x" для шестнадцатеричных чисел.

Нужно ли учиться конвертировать числа в уме? Таблицу шестнадцатеричных чисел из одной цифры легко запомнить.
А запоминать б\'{о}льшие числа, наверное, не стоит.

Наверное, чаще всего шестнадцатеричные числа можно увидеть в \ac{URL}-ах.
Так кодируются буквы не из числа латинских.
Например:
\url{https://en.wiktionary.org/wiki/na\%C3\%AFvet\%C3\%A9} это \ac{URL} страницы в Wiktionary о слове \q{naïveté}.

\subsubsection{Восьмеричная система}

Еще одна система, которая в прошлом много использовалась в программировании это восьмеричная: есть 8 цифр (0..7) и каждая
описывает 3 бита, так что легко конвертировать числа туда и назад.
Она почти везде была заменена шестнадцатеричной, но удивительно, в *NIX имеется утилита использующаяся многими людьми,
которая принимает на вход восьмеричное число: \TT{chmod}.

\myindex{UNIX!chmod}
Как знают многие пользователи *NIX, аргумент \TT{chmod} это число из трех цифр. Первая цифра это права владельца файла,
вторая это права группы (которой файл принадлежит), третья для всех остальных.
И каждая цифра может быть представлена в двоичном виде:

\begin{center}
\begin{longtable}{ | l | l | l | }
\hline
\HeaderColor десятичная & \HeaderColor двоичная & \HeaderColor значение \\
\hline
7	&111	&\textbf{rwx} \\
6	&110	&\textbf{rw-} \\
5	&101	&\textbf{r-x} \\
4	&100	&\textbf{r-{}-} \\
3	&011	&\textbf{-wx} \\
2	&010	&\textbf{-w-} \\
1	&001	&\textbf{-{}-x} \\
0	&000	&\textbf{-{}-{}-} \\
\hline
\end{longtable}
\end{center}

Так что каждый бит привязан к флагу: read/write/execute (чтение/запись/исполнение).

И вот почему я вспомнил здесь о \TT{chmod}, это потому что всё число может быть представлено как число в восьмеричной системе.
Для примера возьмем 644.
Когда вы запускаете \TT{chmod 644 file}, вы выставляете права read/write для владельца, права read для группы, и снова,
read для всех остальных.
Сконвертируем число 644 из восьмеричной системы в двоичную, это будет \TT{110100100}, или (в группах по 3 бита) \TT{110 100 100}.

Теперь мы видим, что каждая тройка описывает права для владельца/группы/остальных:
первая это \TT{rw-}, вторая это \TT{r--} и третья это \TT{r--}.

Восьмеричная система была также популярная на старых компьютерах вроде PDP-8, потому что слово там могло содержать 12, 24 или
36 бит, и эти числа делятся на 3, так что выбор восьмеричной системы в той среде был логичен.
Сейчас, все популярные компьютеры имеют размер слова/адреса 16, 32 или 64 бита, и эти числа делятся на 4,
так что шестнадцатеричная система здесь удобнее.

Восьмеричная система поддерживается всеми стандартными компиляторами \CCpp{}.
Это иногда источник недоумения, потому что восьмеричные числа кодируются с нулем вперед, например, 0377 это 255.
И иногда, вы можете сделать опечатку, и написать "09" вместо 9, и компилятор выдаст ошибку.
GCC может выдать что-то вроде:\\
\TT{error: invalid digit "9" in octal constant}.

Также, восьмеричная система популярна в Java: когда IDA показывает строку с непечатаемыми символами,
они кодируются в восьмеричной системе вместо шестнадцатеричной.
\myindex{JAD}
Точно также себя ведет декомпилятор с Java JAD.

\subsubsection{Делимость}

Когда вы видите десятичное число вроде 120, вы можете быстро понять что оно делится на 10, потому что последняя цифра это 0.
Точно также, 123400 делится на 100, потому что две последних цифры это нули.

Точно также, шестнадцатеричное число 0x1230 делится на 0x10 (или 16), 0x123000 делится на 0x1000 (или 4096), итд.

Двоичное число 0b1000101000 делится на 0b1000 (8), итд.

Это свойство можно часто использовать, чтобы быстро понять,
что длина какого-либо блока в памяти выровнена по некоторой границе.
Например, секции в \ac{PE}-файлах почти всегда начинаются с адресов заканчивающихся 3 шестнадцатеричными нулями:
0x41000, 0x10001000, итд.
Причина в том, что почти все секции в \ac{PE} выровнены по границе 0x1000 (4096) байт.

\subsubsection{Арифметика произвольной точности и основание}

\index{RSA}
Арифметика произвольной точности (multi-precision arithmetic) может использовать огромные числа,
которые могут храниться в нескольких байтах.
Например, ключи RSA, и открытые и закрытые, могут занимать до 4096 бит и даже больше.

В \InSqBrackets{\TAOCPvolII, 265} можно найти такую идею: когда вы сохраняете число произвольной точности в нескольких байтах,
всё число может быть представлено как имеющую систему счисления по основанию $2^8=256$, и каждая цифра находится
в соответствующем байте.
Точно также, если вы сохраняете число произвольной точности в нескольких 32-битных целочисленных значениях,
каждая цифра отправляется в каждый 32-битный слот, и вы можете считать что это число записано в системе с основанием $2^{32}$.

\subsubsection{Произношение}

Числа в недесятичных системах счислениях обычно произносятся по одной цифре: ``один-ноль-ноль-один-один-...''.
Слова вроде ``десять'', ``тысяча'', итд, обычно не произносятся, потому что тогда можно спутать с десятичной системой.

\subsubsection{Числа с плавающей запятой}

Чтобы отличать числа с плавающей запятой от целочисленных, часто, в конце добавляют ``.0'',
например $0.0$, $123.0$, итд.

}
\ITA{\input{patterns/numeral_ITA}}
\DE{\input{patterns/numeral_DE}}
\FR{\input{patterns/numeral_FR}}
\PL{\input{patterns/numeral_PL}}

% chapters
\ifdefined\SPANISH
\chapter{Patrones de código}
\fi % SPANISH

\ifdefined\GERMAN
\chapter{Code-Muster}
\fi % GERMAN

\ifdefined\ENGLISH
\chapter{Code Patterns}
\fi % ENGLISH

\ifdefined\ITALIAN
\chapter{Forme di codice}
\fi % ITALIAN

\ifdefined\RUSSIAN
\chapter{Образцы кода}
\fi % RUSSIAN

\ifdefined\BRAZILIAN
\chapter{Padrões de códigos}
\fi % BRAZILIAN

\ifdefined\THAI
\chapter{รูปแบบของโค้ด}
\fi % THAI

\ifdefined\FRENCH
\chapter{Modèle de code}
\fi % FRENCH

\ifdefined\POLISH
\chapter{\PLph{}}
\fi % POLISH

% sections
\EN{\input{patterns/patterns_opt_dbg_EN}}
\ES{\input{patterns/patterns_opt_dbg_ES}}
\ITA{\input{patterns/patterns_opt_dbg_ITA}}
\PTBR{\input{patterns/patterns_opt_dbg_PTBR}}
\RU{\input{patterns/patterns_opt_dbg_RU}}
\THA{\input{patterns/patterns_opt_dbg_THA}}
\DE{\input{patterns/patterns_opt_dbg_DE}}
\FR{\input{patterns/patterns_opt_dbg_FR}}
\PL{\input{patterns/patterns_opt_dbg_PL}}

\RU{\section{Некоторые базовые понятия}}
\EN{\section{Some basics}}
\DE{\section{Einige Grundlagen}}
\FR{\section{Quelques bases}}
\ES{\section{\ESph{}}}
\ITA{\section{Alcune basi teoriche}}
\PTBR{\section{\PTBRph{}}}
\THA{\section{\THAph{}}}
\PL{\section{\PLph{}}}

% sections:
\EN{\input{patterns/intro_CPU_ISA_EN}}
\ES{\input{patterns/intro_CPU_ISA_ES}}
\ITA{\input{patterns/intro_CPU_ISA_ITA}}
\PTBR{\input{patterns/intro_CPU_ISA_PTBR}}
\RU{\input{patterns/intro_CPU_ISA_RU}}
\DE{\input{patterns/intro_CPU_ISA_DE}}
\FR{\input{patterns/intro_CPU_ISA_FR}}
\PL{\input{patterns/intro_CPU_ISA_PL}}

\EN{\input{patterns/numeral_EN}}
\RU{\input{patterns/numeral_RU}}
\ITA{\input{patterns/numeral_ITA}}
\DE{\input{patterns/numeral_DE}}
\FR{\input{patterns/numeral_FR}}
\PL{\input{patterns/numeral_PL}}

% chapters
\input{patterns/00_empty/main}
\input{patterns/011_ret/main}
\input{patterns/01_helloworld/main}
\input{patterns/015_prolog_epilogue/main}
\input{patterns/02_stack/main}
\input{patterns/03_printf/main}
\input{patterns/04_scanf/main}
\input{patterns/05_passing_arguments/main}
\input{patterns/06_return_results/main}
\input{patterns/061_pointers/main}
\input{patterns/065_GOTO/main}
\input{patterns/07_jcc/main}
\input{patterns/08_switch/main}
\input{patterns/09_loops/main}
\input{patterns/10_strings/main}
\input{patterns/11_arith_optimizations/main}
\input{patterns/12_FPU/main}
\input{patterns/13_arrays/main}
\input{patterns/14_bitfields/main}
\EN{\input{patterns/145_LCG/main_EN}}
\RU{\input{patterns/145_LCG/main_RU}}
\input{patterns/15_structs/main}
\input{patterns/17_unions/main}
\input{patterns/18_pointers_to_functions/main}
\input{patterns/185_64bit_in_32_env/main}

\EN{\input{patterns/19_SIMD/main_EN}}
\RU{\input{patterns/19_SIMD/main_RU}}
\DE{\input{patterns/19_SIMD/main_DE}}

\EN{\input{patterns/20_x64/main_EN}}
\RU{\input{patterns/20_x64/main_RU}}

\EN{\input{patterns/205_floating_SIMD/main_EN}}
\RU{\input{patterns/205_floating_SIMD/main_RU}}
\DE{\input{patterns/205_floating_SIMD/main_DE}}

\EN{\input{patterns/ARM/main_EN}}
\RU{\input{patterns/ARM/main_RU}}
\DE{\input{patterns/ARM/main_DE}}

\input{patterns/MIPS/main}

\ifdefined\SPANISH
\chapter{Patrones de código}
\fi % SPANISH

\ifdefined\GERMAN
\chapter{Code-Muster}
\fi % GERMAN

\ifdefined\ENGLISH
\chapter{Code Patterns}
\fi % ENGLISH

\ifdefined\ITALIAN
\chapter{Forme di codice}
\fi % ITALIAN

\ifdefined\RUSSIAN
\chapter{Образцы кода}
\fi % RUSSIAN

\ifdefined\BRAZILIAN
\chapter{Padrões de códigos}
\fi % BRAZILIAN

\ifdefined\THAI
\chapter{รูปแบบของโค้ด}
\fi % THAI

\ifdefined\FRENCH
\chapter{Modèle de code}
\fi % FRENCH

\ifdefined\POLISH
\chapter{\PLph{}}
\fi % POLISH

% sections
\EN{\input{patterns/patterns_opt_dbg_EN}}
\ES{\input{patterns/patterns_opt_dbg_ES}}
\ITA{\input{patterns/patterns_opt_dbg_ITA}}
\PTBR{\input{patterns/patterns_opt_dbg_PTBR}}
\RU{\input{patterns/patterns_opt_dbg_RU}}
\THA{\input{patterns/patterns_opt_dbg_THA}}
\DE{\input{patterns/patterns_opt_dbg_DE}}
\FR{\input{patterns/patterns_opt_dbg_FR}}
\PL{\input{patterns/patterns_opt_dbg_PL}}

\RU{\section{Некоторые базовые понятия}}
\EN{\section{Some basics}}
\DE{\section{Einige Grundlagen}}
\FR{\section{Quelques bases}}
\ES{\section{\ESph{}}}
\ITA{\section{Alcune basi teoriche}}
\PTBR{\section{\PTBRph{}}}
\THA{\section{\THAph{}}}
\PL{\section{\PLph{}}}

% sections:
\EN{\input{patterns/intro_CPU_ISA_EN}}
\ES{\input{patterns/intro_CPU_ISA_ES}}
\ITA{\input{patterns/intro_CPU_ISA_ITA}}
\PTBR{\input{patterns/intro_CPU_ISA_PTBR}}
\RU{\input{patterns/intro_CPU_ISA_RU}}
\DE{\input{patterns/intro_CPU_ISA_DE}}
\FR{\input{patterns/intro_CPU_ISA_FR}}
\PL{\input{patterns/intro_CPU_ISA_PL}}

\EN{\input{patterns/numeral_EN}}
\RU{\input{patterns/numeral_RU}}
\ITA{\input{patterns/numeral_ITA}}
\DE{\input{patterns/numeral_DE}}
\FR{\input{patterns/numeral_FR}}
\PL{\input{patterns/numeral_PL}}

% chapters
\input{patterns/00_empty/main}
\input{patterns/011_ret/main}
\input{patterns/01_helloworld/main}
\input{patterns/015_prolog_epilogue/main}
\input{patterns/02_stack/main}
\input{patterns/03_printf/main}
\input{patterns/04_scanf/main}
\input{patterns/05_passing_arguments/main}
\input{patterns/06_return_results/main}
\input{patterns/061_pointers/main}
\input{patterns/065_GOTO/main}
\input{patterns/07_jcc/main}
\input{patterns/08_switch/main}
\input{patterns/09_loops/main}
\input{patterns/10_strings/main}
\input{patterns/11_arith_optimizations/main}
\input{patterns/12_FPU/main}
\input{patterns/13_arrays/main}
\input{patterns/14_bitfields/main}
\EN{\input{patterns/145_LCG/main_EN}}
\RU{\input{patterns/145_LCG/main_RU}}
\input{patterns/15_structs/main}
\input{patterns/17_unions/main}
\input{patterns/18_pointers_to_functions/main}
\input{patterns/185_64bit_in_32_env/main}

\EN{\input{patterns/19_SIMD/main_EN}}
\RU{\input{patterns/19_SIMD/main_RU}}
\DE{\input{patterns/19_SIMD/main_DE}}

\EN{\input{patterns/20_x64/main_EN}}
\RU{\input{patterns/20_x64/main_RU}}

\EN{\input{patterns/205_floating_SIMD/main_EN}}
\RU{\input{patterns/205_floating_SIMD/main_RU}}
\DE{\input{patterns/205_floating_SIMD/main_DE}}

\EN{\input{patterns/ARM/main_EN}}
\RU{\input{patterns/ARM/main_RU}}
\DE{\input{patterns/ARM/main_DE}}

\input{patterns/MIPS/main}

\ifdefined\SPANISH
\chapter{Patrones de código}
\fi % SPANISH

\ifdefined\GERMAN
\chapter{Code-Muster}
\fi % GERMAN

\ifdefined\ENGLISH
\chapter{Code Patterns}
\fi % ENGLISH

\ifdefined\ITALIAN
\chapter{Forme di codice}
\fi % ITALIAN

\ifdefined\RUSSIAN
\chapter{Образцы кода}
\fi % RUSSIAN

\ifdefined\BRAZILIAN
\chapter{Padrões de códigos}
\fi % BRAZILIAN

\ifdefined\THAI
\chapter{รูปแบบของโค้ด}
\fi % THAI

\ifdefined\FRENCH
\chapter{Modèle de code}
\fi % FRENCH

\ifdefined\POLISH
\chapter{\PLph{}}
\fi % POLISH

% sections
\EN{\input{patterns/patterns_opt_dbg_EN}}
\ES{\input{patterns/patterns_opt_dbg_ES}}
\ITA{\input{patterns/patterns_opt_dbg_ITA}}
\PTBR{\input{patterns/patterns_opt_dbg_PTBR}}
\RU{\input{patterns/patterns_opt_dbg_RU}}
\THA{\input{patterns/patterns_opt_dbg_THA}}
\DE{\input{patterns/patterns_opt_dbg_DE}}
\FR{\input{patterns/patterns_opt_dbg_FR}}
\PL{\input{patterns/patterns_opt_dbg_PL}}

\RU{\section{Некоторые базовые понятия}}
\EN{\section{Some basics}}
\DE{\section{Einige Grundlagen}}
\FR{\section{Quelques bases}}
\ES{\section{\ESph{}}}
\ITA{\section{Alcune basi teoriche}}
\PTBR{\section{\PTBRph{}}}
\THA{\section{\THAph{}}}
\PL{\section{\PLph{}}}

% sections:
\EN{\input{patterns/intro_CPU_ISA_EN}}
\ES{\input{patterns/intro_CPU_ISA_ES}}
\ITA{\input{patterns/intro_CPU_ISA_ITA}}
\PTBR{\input{patterns/intro_CPU_ISA_PTBR}}
\RU{\input{patterns/intro_CPU_ISA_RU}}
\DE{\input{patterns/intro_CPU_ISA_DE}}
\FR{\input{patterns/intro_CPU_ISA_FR}}
\PL{\input{patterns/intro_CPU_ISA_PL}}

\EN{\input{patterns/numeral_EN}}
\RU{\input{patterns/numeral_RU}}
\ITA{\input{patterns/numeral_ITA}}
\DE{\input{patterns/numeral_DE}}
\FR{\input{patterns/numeral_FR}}
\PL{\input{patterns/numeral_PL}}

% chapters
\input{patterns/00_empty/main}
\input{patterns/011_ret/main}
\input{patterns/01_helloworld/main}
\input{patterns/015_prolog_epilogue/main}
\input{patterns/02_stack/main}
\input{patterns/03_printf/main}
\input{patterns/04_scanf/main}
\input{patterns/05_passing_arguments/main}
\input{patterns/06_return_results/main}
\input{patterns/061_pointers/main}
\input{patterns/065_GOTO/main}
\input{patterns/07_jcc/main}
\input{patterns/08_switch/main}
\input{patterns/09_loops/main}
\input{patterns/10_strings/main}
\input{patterns/11_arith_optimizations/main}
\input{patterns/12_FPU/main}
\input{patterns/13_arrays/main}
\input{patterns/14_bitfields/main}
\EN{\input{patterns/145_LCG/main_EN}}
\RU{\input{patterns/145_LCG/main_RU}}
\input{patterns/15_structs/main}
\input{patterns/17_unions/main}
\input{patterns/18_pointers_to_functions/main}
\input{patterns/185_64bit_in_32_env/main}

\EN{\input{patterns/19_SIMD/main_EN}}
\RU{\input{patterns/19_SIMD/main_RU}}
\DE{\input{patterns/19_SIMD/main_DE}}

\EN{\input{patterns/20_x64/main_EN}}
\RU{\input{patterns/20_x64/main_RU}}

\EN{\input{patterns/205_floating_SIMD/main_EN}}
\RU{\input{patterns/205_floating_SIMD/main_RU}}
\DE{\input{patterns/205_floating_SIMD/main_DE}}

\EN{\input{patterns/ARM/main_EN}}
\RU{\input{patterns/ARM/main_RU}}
\DE{\input{patterns/ARM/main_DE}}

\input{patterns/MIPS/main}

\ifdefined\SPANISH
\chapter{Patrones de código}
\fi % SPANISH

\ifdefined\GERMAN
\chapter{Code-Muster}
\fi % GERMAN

\ifdefined\ENGLISH
\chapter{Code Patterns}
\fi % ENGLISH

\ifdefined\ITALIAN
\chapter{Forme di codice}
\fi % ITALIAN

\ifdefined\RUSSIAN
\chapter{Образцы кода}
\fi % RUSSIAN

\ifdefined\BRAZILIAN
\chapter{Padrões de códigos}
\fi % BRAZILIAN

\ifdefined\THAI
\chapter{รูปแบบของโค้ด}
\fi % THAI

\ifdefined\FRENCH
\chapter{Modèle de code}
\fi % FRENCH

\ifdefined\POLISH
\chapter{\PLph{}}
\fi % POLISH

% sections
\EN{\input{patterns/patterns_opt_dbg_EN}}
\ES{\input{patterns/patterns_opt_dbg_ES}}
\ITA{\input{patterns/patterns_opt_dbg_ITA}}
\PTBR{\input{patterns/patterns_opt_dbg_PTBR}}
\RU{\input{patterns/patterns_opt_dbg_RU}}
\THA{\input{patterns/patterns_opt_dbg_THA}}
\DE{\input{patterns/patterns_opt_dbg_DE}}
\FR{\input{patterns/patterns_opt_dbg_FR}}
\PL{\input{patterns/patterns_opt_dbg_PL}}

\RU{\section{Некоторые базовые понятия}}
\EN{\section{Some basics}}
\DE{\section{Einige Grundlagen}}
\FR{\section{Quelques bases}}
\ES{\section{\ESph{}}}
\ITA{\section{Alcune basi teoriche}}
\PTBR{\section{\PTBRph{}}}
\THA{\section{\THAph{}}}
\PL{\section{\PLph{}}}

% sections:
\EN{\input{patterns/intro_CPU_ISA_EN}}
\ES{\input{patterns/intro_CPU_ISA_ES}}
\ITA{\input{patterns/intro_CPU_ISA_ITA}}
\PTBR{\input{patterns/intro_CPU_ISA_PTBR}}
\RU{\input{patterns/intro_CPU_ISA_RU}}
\DE{\input{patterns/intro_CPU_ISA_DE}}
\FR{\input{patterns/intro_CPU_ISA_FR}}
\PL{\input{patterns/intro_CPU_ISA_PL}}

\EN{\input{patterns/numeral_EN}}
\RU{\input{patterns/numeral_RU}}
\ITA{\input{patterns/numeral_ITA}}
\DE{\input{patterns/numeral_DE}}
\FR{\input{patterns/numeral_FR}}
\PL{\input{patterns/numeral_PL}}

% chapters
\input{patterns/00_empty/main}
\input{patterns/011_ret/main}
\input{patterns/01_helloworld/main}
\input{patterns/015_prolog_epilogue/main}
\input{patterns/02_stack/main}
\input{patterns/03_printf/main}
\input{patterns/04_scanf/main}
\input{patterns/05_passing_arguments/main}
\input{patterns/06_return_results/main}
\input{patterns/061_pointers/main}
\input{patterns/065_GOTO/main}
\input{patterns/07_jcc/main}
\input{patterns/08_switch/main}
\input{patterns/09_loops/main}
\input{patterns/10_strings/main}
\input{patterns/11_arith_optimizations/main}
\input{patterns/12_FPU/main}
\input{patterns/13_arrays/main}
\input{patterns/14_bitfields/main}
\EN{\input{patterns/145_LCG/main_EN}}
\RU{\input{patterns/145_LCG/main_RU}}
\input{patterns/15_structs/main}
\input{patterns/17_unions/main}
\input{patterns/18_pointers_to_functions/main}
\input{patterns/185_64bit_in_32_env/main}

\EN{\input{patterns/19_SIMD/main_EN}}
\RU{\input{patterns/19_SIMD/main_RU}}
\DE{\input{patterns/19_SIMD/main_DE}}

\EN{\input{patterns/20_x64/main_EN}}
\RU{\input{patterns/20_x64/main_RU}}

\EN{\input{patterns/205_floating_SIMD/main_EN}}
\RU{\input{patterns/205_floating_SIMD/main_RU}}
\DE{\input{patterns/205_floating_SIMD/main_DE}}

\EN{\input{patterns/ARM/main_EN}}
\RU{\input{patterns/ARM/main_RU}}
\DE{\input{patterns/ARM/main_DE}}

\input{patterns/MIPS/main}

\ifdefined\SPANISH
\chapter{Patrones de código}
\fi % SPANISH

\ifdefined\GERMAN
\chapter{Code-Muster}
\fi % GERMAN

\ifdefined\ENGLISH
\chapter{Code Patterns}
\fi % ENGLISH

\ifdefined\ITALIAN
\chapter{Forme di codice}
\fi % ITALIAN

\ifdefined\RUSSIAN
\chapter{Образцы кода}
\fi % RUSSIAN

\ifdefined\BRAZILIAN
\chapter{Padrões de códigos}
\fi % BRAZILIAN

\ifdefined\THAI
\chapter{รูปแบบของโค้ด}
\fi % THAI

\ifdefined\FRENCH
\chapter{Modèle de code}
\fi % FRENCH

\ifdefined\POLISH
\chapter{\PLph{}}
\fi % POLISH

% sections
\EN{\input{patterns/patterns_opt_dbg_EN}}
\ES{\input{patterns/patterns_opt_dbg_ES}}
\ITA{\input{patterns/patterns_opt_dbg_ITA}}
\PTBR{\input{patterns/patterns_opt_dbg_PTBR}}
\RU{\input{patterns/patterns_opt_dbg_RU}}
\THA{\input{patterns/patterns_opt_dbg_THA}}
\DE{\input{patterns/patterns_opt_dbg_DE}}
\FR{\input{patterns/patterns_opt_dbg_FR}}
\PL{\input{patterns/patterns_opt_dbg_PL}}

\RU{\section{Некоторые базовые понятия}}
\EN{\section{Some basics}}
\DE{\section{Einige Grundlagen}}
\FR{\section{Quelques bases}}
\ES{\section{\ESph{}}}
\ITA{\section{Alcune basi teoriche}}
\PTBR{\section{\PTBRph{}}}
\THA{\section{\THAph{}}}
\PL{\section{\PLph{}}}

% sections:
\EN{\input{patterns/intro_CPU_ISA_EN}}
\ES{\input{patterns/intro_CPU_ISA_ES}}
\ITA{\input{patterns/intro_CPU_ISA_ITA}}
\PTBR{\input{patterns/intro_CPU_ISA_PTBR}}
\RU{\input{patterns/intro_CPU_ISA_RU}}
\DE{\input{patterns/intro_CPU_ISA_DE}}
\FR{\input{patterns/intro_CPU_ISA_FR}}
\PL{\input{patterns/intro_CPU_ISA_PL}}

\EN{\input{patterns/numeral_EN}}
\RU{\input{patterns/numeral_RU}}
\ITA{\input{patterns/numeral_ITA}}
\DE{\input{patterns/numeral_DE}}
\FR{\input{patterns/numeral_FR}}
\PL{\input{patterns/numeral_PL}}

% chapters
\input{patterns/00_empty/main}
\input{patterns/011_ret/main}
\input{patterns/01_helloworld/main}
\input{patterns/015_prolog_epilogue/main}
\input{patterns/02_stack/main}
\input{patterns/03_printf/main}
\input{patterns/04_scanf/main}
\input{patterns/05_passing_arguments/main}
\input{patterns/06_return_results/main}
\input{patterns/061_pointers/main}
\input{patterns/065_GOTO/main}
\input{patterns/07_jcc/main}
\input{patterns/08_switch/main}
\input{patterns/09_loops/main}
\input{patterns/10_strings/main}
\input{patterns/11_arith_optimizations/main}
\input{patterns/12_FPU/main}
\input{patterns/13_arrays/main}
\input{patterns/14_bitfields/main}
\EN{\input{patterns/145_LCG/main_EN}}
\RU{\input{patterns/145_LCG/main_RU}}
\input{patterns/15_structs/main}
\input{patterns/17_unions/main}
\input{patterns/18_pointers_to_functions/main}
\input{patterns/185_64bit_in_32_env/main}

\EN{\input{patterns/19_SIMD/main_EN}}
\RU{\input{patterns/19_SIMD/main_RU}}
\DE{\input{patterns/19_SIMD/main_DE}}

\EN{\input{patterns/20_x64/main_EN}}
\RU{\input{patterns/20_x64/main_RU}}

\EN{\input{patterns/205_floating_SIMD/main_EN}}
\RU{\input{patterns/205_floating_SIMD/main_RU}}
\DE{\input{patterns/205_floating_SIMD/main_DE}}

\EN{\input{patterns/ARM/main_EN}}
\RU{\input{patterns/ARM/main_RU}}
\DE{\input{patterns/ARM/main_DE}}

\input{patterns/MIPS/main}

\ifdefined\SPANISH
\chapter{Patrones de código}
\fi % SPANISH

\ifdefined\GERMAN
\chapter{Code-Muster}
\fi % GERMAN

\ifdefined\ENGLISH
\chapter{Code Patterns}
\fi % ENGLISH

\ifdefined\ITALIAN
\chapter{Forme di codice}
\fi % ITALIAN

\ifdefined\RUSSIAN
\chapter{Образцы кода}
\fi % RUSSIAN

\ifdefined\BRAZILIAN
\chapter{Padrões de códigos}
\fi % BRAZILIAN

\ifdefined\THAI
\chapter{รูปแบบของโค้ด}
\fi % THAI

\ifdefined\FRENCH
\chapter{Modèle de code}
\fi % FRENCH

\ifdefined\POLISH
\chapter{\PLph{}}
\fi % POLISH

% sections
\EN{\input{patterns/patterns_opt_dbg_EN}}
\ES{\input{patterns/patterns_opt_dbg_ES}}
\ITA{\input{patterns/patterns_opt_dbg_ITA}}
\PTBR{\input{patterns/patterns_opt_dbg_PTBR}}
\RU{\input{patterns/patterns_opt_dbg_RU}}
\THA{\input{patterns/patterns_opt_dbg_THA}}
\DE{\input{patterns/patterns_opt_dbg_DE}}
\FR{\input{patterns/patterns_opt_dbg_FR}}
\PL{\input{patterns/patterns_opt_dbg_PL}}

\RU{\section{Некоторые базовые понятия}}
\EN{\section{Some basics}}
\DE{\section{Einige Grundlagen}}
\FR{\section{Quelques bases}}
\ES{\section{\ESph{}}}
\ITA{\section{Alcune basi teoriche}}
\PTBR{\section{\PTBRph{}}}
\THA{\section{\THAph{}}}
\PL{\section{\PLph{}}}

% sections:
\EN{\input{patterns/intro_CPU_ISA_EN}}
\ES{\input{patterns/intro_CPU_ISA_ES}}
\ITA{\input{patterns/intro_CPU_ISA_ITA}}
\PTBR{\input{patterns/intro_CPU_ISA_PTBR}}
\RU{\input{patterns/intro_CPU_ISA_RU}}
\DE{\input{patterns/intro_CPU_ISA_DE}}
\FR{\input{patterns/intro_CPU_ISA_FR}}
\PL{\input{patterns/intro_CPU_ISA_PL}}

\EN{\input{patterns/numeral_EN}}
\RU{\input{patterns/numeral_RU}}
\ITA{\input{patterns/numeral_ITA}}
\DE{\input{patterns/numeral_DE}}
\FR{\input{patterns/numeral_FR}}
\PL{\input{patterns/numeral_PL}}

% chapters
\input{patterns/00_empty/main}
\input{patterns/011_ret/main}
\input{patterns/01_helloworld/main}
\input{patterns/015_prolog_epilogue/main}
\input{patterns/02_stack/main}
\input{patterns/03_printf/main}
\input{patterns/04_scanf/main}
\input{patterns/05_passing_arguments/main}
\input{patterns/06_return_results/main}
\input{patterns/061_pointers/main}
\input{patterns/065_GOTO/main}
\input{patterns/07_jcc/main}
\input{patterns/08_switch/main}
\input{patterns/09_loops/main}
\input{patterns/10_strings/main}
\input{patterns/11_arith_optimizations/main}
\input{patterns/12_FPU/main}
\input{patterns/13_arrays/main}
\input{patterns/14_bitfields/main}
\EN{\input{patterns/145_LCG/main_EN}}
\RU{\input{patterns/145_LCG/main_RU}}
\input{patterns/15_structs/main}
\input{patterns/17_unions/main}
\input{patterns/18_pointers_to_functions/main}
\input{patterns/185_64bit_in_32_env/main}

\EN{\input{patterns/19_SIMD/main_EN}}
\RU{\input{patterns/19_SIMD/main_RU}}
\DE{\input{patterns/19_SIMD/main_DE}}

\EN{\input{patterns/20_x64/main_EN}}
\RU{\input{patterns/20_x64/main_RU}}

\EN{\input{patterns/205_floating_SIMD/main_EN}}
\RU{\input{patterns/205_floating_SIMD/main_RU}}
\DE{\input{patterns/205_floating_SIMD/main_DE}}

\EN{\input{patterns/ARM/main_EN}}
\RU{\input{patterns/ARM/main_RU}}
\DE{\input{patterns/ARM/main_DE}}

\input{patterns/MIPS/main}

\ifdefined\SPANISH
\chapter{Patrones de código}
\fi % SPANISH

\ifdefined\GERMAN
\chapter{Code-Muster}
\fi % GERMAN

\ifdefined\ENGLISH
\chapter{Code Patterns}
\fi % ENGLISH

\ifdefined\ITALIAN
\chapter{Forme di codice}
\fi % ITALIAN

\ifdefined\RUSSIAN
\chapter{Образцы кода}
\fi % RUSSIAN

\ifdefined\BRAZILIAN
\chapter{Padrões de códigos}
\fi % BRAZILIAN

\ifdefined\THAI
\chapter{รูปแบบของโค้ด}
\fi % THAI

\ifdefined\FRENCH
\chapter{Modèle de code}
\fi % FRENCH

\ifdefined\POLISH
\chapter{\PLph{}}
\fi % POLISH

% sections
\EN{\input{patterns/patterns_opt_dbg_EN}}
\ES{\input{patterns/patterns_opt_dbg_ES}}
\ITA{\input{patterns/patterns_opt_dbg_ITA}}
\PTBR{\input{patterns/patterns_opt_dbg_PTBR}}
\RU{\input{patterns/patterns_opt_dbg_RU}}
\THA{\input{patterns/patterns_opt_dbg_THA}}
\DE{\input{patterns/patterns_opt_dbg_DE}}
\FR{\input{patterns/patterns_opt_dbg_FR}}
\PL{\input{patterns/patterns_opt_dbg_PL}}

\RU{\section{Некоторые базовые понятия}}
\EN{\section{Some basics}}
\DE{\section{Einige Grundlagen}}
\FR{\section{Quelques bases}}
\ES{\section{\ESph{}}}
\ITA{\section{Alcune basi teoriche}}
\PTBR{\section{\PTBRph{}}}
\THA{\section{\THAph{}}}
\PL{\section{\PLph{}}}

% sections:
\EN{\input{patterns/intro_CPU_ISA_EN}}
\ES{\input{patterns/intro_CPU_ISA_ES}}
\ITA{\input{patterns/intro_CPU_ISA_ITA}}
\PTBR{\input{patterns/intro_CPU_ISA_PTBR}}
\RU{\input{patterns/intro_CPU_ISA_RU}}
\DE{\input{patterns/intro_CPU_ISA_DE}}
\FR{\input{patterns/intro_CPU_ISA_FR}}
\PL{\input{patterns/intro_CPU_ISA_PL}}

\EN{\input{patterns/numeral_EN}}
\RU{\input{patterns/numeral_RU}}
\ITA{\input{patterns/numeral_ITA}}
\DE{\input{patterns/numeral_DE}}
\FR{\input{patterns/numeral_FR}}
\PL{\input{patterns/numeral_PL}}

% chapters
\input{patterns/00_empty/main}
\input{patterns/011_ret/main}
\input{patterns/01_helloworld/main}
\input{patterns/015_prolog_epilogue/main}
\input{patterns/02_stack/main}
\input{patterns/03_printf/main}
\input{patterns/04_scanf/main}
\input{patterns/05_passing_arguments/main}
\input{patterns/06_return_results/main}
\input{patterns/061_pointers/main}
\input{patterns/065_GOTO/main}
\input{patterns/07_jcc/main}
\input{patterns/08_switch/main}
\input{patterns/09_loops/main}
\input{patterns/10_strings/main}
\input{patterns/11_arith_optimizations/main}
\input{patterns/12_FPU/main}
\input{patterns/13_arrays/main}
\input{patterns/14_bitfields/main}
\EN{\input{patterns/145_LCG/main_EN}}
\RU{\input{patterns/145_LCG/main_RU}}
\input{patterns/15_structs/main}
\input{patterns/17_unions/main}
\input{patterns/18_pointers_to_functions/main}
\input{patterns/185_64bit_in_32_env/main}

\EN{\input{patterns/19_SIMD/main_EN}}
\RU{\input{patterns/19_SIMD/main_RU}}
\DE{\input{patterns/19_SIMD/main_DE}}

\EN{\input{patterns/20_x64/main_EN}}
\RU{\input{patterns/20_x64/main_RU}}

\EN{\input{patterns/205_floating_SIMD/main_EN}}
\RU{\input{patterns/205_floating_SIMD/main_RU}}
\DE{\input{patterns/205_floating_SIMD/main_DE}}

\EN{\input{patterns/ARM/main_EN}}
\RU{\input{patterns/ARM/main_RU}}
\DE{\input{patterns/ARM/main_DE}}

\input{patterns/MIPS/main}

\ifdefined\SPANISH
\chapter{Patrones de código}
\fi % SPANISH

\ifdefined\GERMAN
\chapter{Code-Muster}
\fi % GERMAN

\ifdefined\ENGLISH
\chapter{Code Patterns}
\fi % ENGLISH

\ifdefined\ITALIAN
\chapter{Forme di codice}
\fi % ITALIAN

\ifdefined\RUSSIAN
\chapter{Образцы кода}
\fi % RUSSIAN

\ifdefined\BRAZILIAN
\chapter{Padrões de códigos}
\fi % BRAZILIAN

\ifdefined\THAI
\chapter{รูปแบบของโค้ด}
\fi % THAI

\ifdefined\FRENCH
\chapter{Modèle de code}
\fi % FRENCH

\ifdefined\POLISH
\chapter{\PLph{}}
\fi % POLISH

% sections
\EN{\input{patterns/patterns_opt_dbg_EN}}
\ES{\input{patterns/patterns_opt_dbg_ES}}
\ITA{\input{patterns/patterns_opt_dbg_ITA}}
\PTBR{\input{patterns/patterns_opt_dbg_PTBR}}
\RU{\input{patterns/patterns_opt_dbg_RU}}
\THA{\input{patterns/patterns_opt_dbg_THA}}
\DE{\input{patterns/patterns_opt_dbg_DE}}
\FR{\input{patterns/patterns_opt_dbg_FR}}
\PL{\input{patterns/patterns_opt_dbg_PL}}

\RU{\section{Некоторые базовые понятия}}
\EN{\section{Some basics}}
\DE{\section{Einige Grundlagen}}
\FR{\section{Quelques bases}}
\ES{\section{\ESph{}}}
\ITA{\section{Alcune basi teoriche}}
\PTBR{\section{\PTBRph{}}}
\THA{\section{\THAph{}}}
\PL{\section{\PLph{}}}

% sections:
\EN{\input{patterns/intro_CPU_ISA_EN}}
\ES{\input{patterns/intro_CPU_ISA_ES}}
\ITA{\input{patterns/intro_CPU_ISA_ITA}}
\PTBR{\input{patterns/intro_CPU_ISA_PTBR}}
\RU{\input{patterns/intro_CPU_ISA_RU}}
\DE{\input{patterns/intro_CPU_ISA_DE}}
\FR{\input{patterns/intro_CPU_ISA_FR}}
\PL{\input{patterns/intro_CPU_ISA_PL}}

\EN{\input{patterns/numeral_EN}}
\RU{\input{patterns/numeral_RU}}
\ITA{\input{patterns/numeral_ITA}}
\DE{\input{patterns/numeral_DE}}
\FR{\input{patterns/numeral_FR}}
\PL{\input{patterns/numeral_PL}}

% chapters
\input{patterns/00_empty/main}
\input{patterns/011_ret/main}
\input{patterns/01_helloworld/main}
\input{patterns/015_prolog_epilogue/main}
\input{patterns/02_stack/main}
\input{patterns/03_printf/main}
\input{patterns/04_scanf/main}
\input{patterns/05_passing_arguments/main}
\input{patterns/06_return_results/main}
\input{patterns/061_pointers/main}
\input{patterns/065_GOTO/main}
\input{patterns/07_jcc/main}
\input{patterns/08_switch/main}
\input{patterns/09_loops/main}
\input{patterns/10_strings/main}
\input{patterns/11_arith_optimizations/main}
\input{patterns/12_FPU/main}
\input{patterns/13_arrays/main}
\input{patterns/14_bitfields/main}
\EN{\input{patterns/145_LCG/main_EN}}
\RU{\input{patterns/145_LCG/main_RU}}
\input{patterns/15_structs/main}
\input{patterns/17_unions/main}
\input{patterns/18_pointers_to_functions/main}
\input{patterns/185_64bit_in_32_env/main}

\EN{\input{patterns/19_SIMD/main_EN}}
\RU{\input{patterns/19_SIMD/main_RU}}
\DE{\input{patterns/19_SIMD/main_DE}}

\EN{\input{patterns/20_x64/main_EN}}
\RU{\input{patterns/20_x64/main_RU}}

\EN{\input{patterns/205_floating_SIMD/main_EN}}
\RU{\input{patterns/205_floating_SIMD/main_RU}}
\DE{\input{patterns/205_floating_SIMD/main_DE}}

\EN{\input{patterns/ARM/main_EN}}
\RU{\input{patterns/ARM/main_RU}}
\DE{\input{patterns/ARM/main_DE}}

\input{patterns/MIPS/main}

\ifdefined\SPANISH
\chapter{Patrones de código}
\fi % SPANISH

\ifdefined\GERMAN
\chapter{Code-Muster}
\fi % GERMAN

\ifdefined\ENGLISH
\chapter{Code Patterns}
\fi % ENGLISH

\ifdefined\ITALIAN
\chapter{Forme di codice}
\fi % ITALIAN

\ifdefined\RUSSIAN
\chapter{Образцы кода}
\fi % RUSSIAN

\ifdefined\BRAZILIAN
\chapter{Padrões de códigos}
\fi % BRAZILIAN

\ifdefined\THAI
\chapter{รูปแบบของโค้ด}
\fi % THAI

\ifdefined\FRENCH
\chapter{Modèle de code}
\fi % FRENCH

\ifdefined\POLISH
\chapter{\PLph{}}
\fi % POLISH

% sections
\EN{\input{patterns/patterns_opt_dbg_EN}}
\ES{\input{patterns/patterns_opt_dbg_ES}}
\ITA{\input{patterns/patterns_opt_dbg_ITA}}
\PTBR{\input{patterns/patterns_opt_dbg_PTBR}}
\RU{\input{patterns/patterns_opt_dbg_RU}}
\THA{\input{patterns/patterns_opt_dbg_THA}}
\DE{\input{patterns/patterns_opt_dbg_DE}}
\FR{\input{patterns/patterns_opt_dbg_FR}}
\PL{\input{patterns/patterns_opt_dbg_PL}}

\RU{\section{Некоторые базовые понятия}}
\EN{\section{Some basics}}
\DE{\section{Einige Grundlagen}}
\FR{\section{Quelques bases}}
\ES{\section{\ESph{}}}
\ITA{\section{Alcune basi teoriche}}
\PTBR{\section{\PTBRph{}}}
\THA{\section{\THAph{}}}
\PL{\section{\PLph{}}}

% sections:
\EN{\input{patterns/intro_CPU_ISA_EN}}
\ES{\input{patterns/intro_CPU_ISA_ES}}
\ITA{\input{patterns/intro_CPU_ISA_ITA}}
\PTBR{\input{patterns/intro_CPU_ISA_PTBR}}
\RU{\input{patterns/intro_CPU_ISA_RU}}
\DE{\input{patterns/intro_CPU_ISA_DE}}
\FR{\input{patterns/intro_CPU_ISA_FR}}
\PL{\input{patterns/intro_CPU_ISA_PL}}

\EN{\input{patterns/numeral_EN}}
\RU{\input{patterns/numeral_RU}}
\ITA{\input{patterns/numeral_ITA}}
\DE{\input{patterns/numeral_DE}}
\FR{\input{patterns/numeral_FR}}
\PL{\input{patterns/numeral_PL}}

% chapters
\input{patterns/00_empty/main}
\input{patterns/011_ret/main}
\input{patterns/01_helloworld/main}
\input{patterns/015_prolog_epilogue/main}
\input{patterns/02_stack/main}
\input{patterns/03_printf/main}
\input{patterns/04_scanf/main}
\input{patterns/05_passing_arguments/main}
\input{patterns/06_return_results/main}
\input{patterns/061_pointers/main}
\input{patterns/065_GOTO/main}
\input{patterns/07_jcc/main}
\input{patterns/08_switch/main}
\input{patterns/09_loops/main}
\input{patterns/10_strings/main}
\input{patterns/11_arith_optimizations/main}
\input{patterns/12_FPU/main}
\input{patterns/13_arrays/main}
\input{patterns/14_bitfields/main}
\EN{\input{patterns/145_LCG/main_EN}}
\RU{\input{patterns/145_LCG/main_RU}}
\input{patterns/15_structs/main}
\input{patterns/17_unions/main}
\input{patterns/18_pointers_to_functions/main}
\input{patterns/185_64bit_in_32_env/main}

\EN{\input{patterns/19_SIMD/main_EN}}
\RU{\input{patterns/19_SIMD/main_RU}}
\DE{\input{patterns/19_SIMD/main_DE}}

\EN{\input{patterns/20_x64/main_EN}}
\RU{\input{patterns/20_x64/main_RU}}

\EN{\input{patterns/205_floating_SIMD/main_EN}}
\RU{\input{patterns/205_floating_SIMD/main_RU}}
\DE{\input{patterns/205_floating_SIMD/main_DE}}

\EN{\input{patterns/ARM/main_EN}}
\RU{\input{patterns/ARM/main_RU}}
\DE{\input{patterns/ARM/main_DE}}

\input{patterns/MIPS/main}

\ifdefined\SPANISH
\chapter{Patrones de código}
\fi % SPANISH

\ifdefined\GERMAN
\chapter{Code-Muster}
\fi % GERMAN

\ifdefined\ENGLISH
\chapter{Code Patterns}
\fi % ENGLISH

\ifdefined\ITALIAN
\chapter{Forme di codice}
\fi % ITALIAN

\ifdefined\RUSSIAN
\chapter{Образцы кода}
\fi % RUSSIAN

\ifdefined\BRAZILIAN
\chapter{Padrões de códigos}
\fi % BRAZILIAN

\ifdefined\THAI
\chapter{รูปแบบของโค้ด}
\fi % THAI

\ifdefined\FRENCH
\chapter{Modèle de code}
\fi % FRENCH

\ifdefined\POLISH
\chapter{\PLph{}}
\fi % POLISH

% sections
\EN{\input{patterns/patterns_opt_dbg_EN}}
\ES{\input{patterns/patterns_opt_dbg_ES}}
\ITA{\input{patterns/patterns_opt_dbg_ITA}}
\PTBR{\input{patterns/patterns_opt_dbg_PTBR}}
\RU{\input{patterns/patterns_opt_dbg_RU}}
\THA{\input{patterns/patterns_opt_dbg_THA}}
\DE{\input{patterns/patterns_opt_dbg_DE}}
\FR{\input{patterns/patterns_opt_dbg_FR}}
\PL{\input{patterns/patterns_opt_dbg_PL}}

\RU{\section{Некоторые базовые понятия}}
\EN{\section{Some basics}}
\DE{\section{Einige Grundlagen}}
\FR{\section{Quelques bases}}
\ES{\section{\ESph{}}}
\ITA{\section{Alcune basi teoriche}}
\PTBR{\section{\PTBRph{}}}
\THA{\section{\THAph{}}}
\PL{\section{\PLph{}}}

% sections:
\EN{\input{patterns/intro_CPU_ISA_EN}}
\ES{\input{patterns/intro_CPU_ISA_ES}}
\ITA{\input{patterns/intro_CPU_ISA_ITA}}
\PTBR{\input{patterns/intro_CPU_ISA_PTBR}}
\RU{\input{patterns/intro_CPU_ISA_RU}}
\DE{\input{patterns/intro_CPU_ISA_DE}}
\FR{\input{patterns/intro_CPU_ISA_FR}}
\PL{\input{patterns/intro_CPU_ISA_PL}}

\EN{\input{patterns/numeral_EN}}
\RU{\input{patterns/numeral_RU}}
\ITA{\input{patterns/numeral_ITA}}
\DE{\input{patterns/numeral_DE}}
\FR{\input{patterns/numeral_FR}}
\PL{\input{patterns/numeral_PL}}

% chapters
\input{patterns/00_empty/main}
\input{patterns/011_ret/main}
\input{patterns/01_helloworld/main}
\input{patterns/015_prolog_epilogue/main}
\input{patterns/02_stack/main}
\input{patterns/03_printf/main}
\input{patterns/04_scanf/main}
\input{patterns/05_passing_arguments/main}
\input{patterns/06_return_results/main}
\input{patterns/061_pointers/main}
\input{patterns/065_GOTO/main}
\input{patterns/07_jcc/main}
\input{patterns/08_switch/main}
\input{patterns/09_loops/main}
\input{patterns/10_strings/main}
\input{patterns/11_arith_optimizations/main}
\input{patterns/12_FPU/main}
\input{patterns/13_arrays/main}
\input{patterns/14_bitfields/main}
\EN{\input{patterns/145_LCG/main_EN}}
\RU{\input{patterns/145_LCG/main_RU}}
\input{patterns/15_structs/main}
\input{patterns/17_unions/main}
\input{patterns/18_pointers_to_functions/main}
\input{patterns/185_64bit_in_32_env/main}

\EN{\input{patterns/19_SIMD/main_EN}}
\RU{\input{patterns/19_SIMD/main_RU}}
\DE{\input{patterns/19_SIMD/main_DE}}

\EN{\input{patterns/20_x64/main_EN}}
\RU{\input{patterns/20_x64/main_RU}}

\EN{\input{patterns/205_floating_SIMD/main_EN}}
\RU{\input{patterns/205_floating_SIMD/main_RU}}
\DE{\input{patterns/205_floating_SIMD/main_DE}}

\EN{\input{patterns/ARM/main_EN}}
\RU{\input{patterns/ARM/main_RU}}
\DE{\input{patterns/ARM/main_DE}}

\input{patterns/MIPS/main}

\ifdefined\SPANISH
\chapter{Patrones de código}
\fi % SPANISH

\ifdefined\GERMAN
\chapter{Code-Muster}
\fi % GERMAN

\ifdefined\ENGLISH
\chapter{Code Patterns}
\fi % ENGLISH

\ifdefined\ITALIAN
\chapter{Forme di codice}
\fi % ITALIAN

\ifdefined\RUSSIAN
\chapter{Образцы кода}
\fi % RUSSIAN

\ifdefined\BRAZILIAN
\chapter{Padrões de códigos}
\fi % BRAZILIAN

\ifdefined\THAI
\chapter{รูปแบบของโค้ด}
\fi % THAI

\ifdefined\FRENCH
\chapter{Modèle de code}
\fi % FRENCH

\ifdefined\POLISH
\chapter{\PLph{}}
\fi % POLISH

% sections
\EN{\input{patterns/patterns_opt_dbg_EN}}
\ES{\input{patterns/patterns_opt_dbg_ES}}
\ITA{\input{patterns/patterns_opt_dbg_ITA}}
\PTBR{\input{patterns/patterns_opt_dbg_PTBR}}
\RU{\input{patterns/patterns_opt_dbg_RU}}
\THA{\input{patterns/patterns_opt_dbg_THA}}
\DE{\input{patterns/patterns_opt_dbg_DE}}
\FR{\input{patterns/patterns_opt_dbg_FR}}
\PL{\input{patterns/patterns_opt_dbg_PL}}

\RU{\section{Некоторые базовые понятия}}
\EN{\section{Some basics}}
\DE{\section{Einige Grundlagen}}
\FR{\section{Quelques bases}}
\ES{\section{\ESph{}}}
\ITA{\section{Alcune basi teoriche}}
\PTBR{\section{\PTBRph{}}}
\THA{\section{\THAph{}}}
\PL{\section{\PLph{}}}

% sections:
\EN{\input{patterns/intro_CPU_ISA_EN}}
\ES{\input{patterns/intro_CPU_ISA_ES}}
\ITA{\input{patterns/intro_CPU_ISA_ITA}}
\PTBR{\input{patterns/intro_CPU_ISA_PTBR}}
\RU{\input{patterns/intro_CPU_ISA_RU}}
\DE{\input{patterns/intro_CPU_ISA_DE}}
\FR{\input{patterns/intro_CPU_ISA_FR}}
\PL{\input{patterns/intro_CPU_ISA_PL}}

\EN{\input{patterns/numeral_EN}}
\RU{\input{patterns/numeral_RU}}
\ITA{\input{patterns/numeral_ITA}}
\DE{\input{patterns/numeral_DE}}
\FR{\input{patterns/numeral_FR}}
\PL{\input{patterns/numeral_PL}}

% chapters
\input{patterns/00_empty/main}
\input{patterns/011_ret/main}
\input{patterns/01_helloworld/main}
\input{patterns/015_prolog_epilogue/main}
\input{patterns/02_stack/main}
\input{patterns/03_printf/main}
\input{patterns/04_scanf/main}
\input{patterns/05_passing_arguments/main}
\input{patterns/06_return_results/main}
\input{patterns/061_pointers/main}
\input{patterns/065_GOTO/main}
\input{patterns/07_jcc/main}
\input{patterns/08_switch/main}
\input{patterns/09_loops/main}
\input{patterns/10_strings/main}
\input{patterns/11_arith_optimizations/main}
\input{patterns/12_FPU/main}
\input{patterns/13_arrays/main}
\input{patterns/14_bitfields/main}
\EN{\input{patterns/145_LCG/main_EN}}
\RU{\input{patterns/145_LCG/main_RU}}
\input{patterns/15_structs/main}
\input{patterns/17_unions/main}
\input{patterns/18_pointers_to_functions/main}
\input{patterns/185_64bit_in_32_env/main}

\EN{\input{patterns/19_SIMD/main_EN}}
\RU{\input{patterns/19_SIMD/main_RU}}
\DE{\input{patterns/19_SIMD/main_DE}}

\EN{\input{patterns/20_x64/main_EN}}
\RU{\input{patterns/20_x64/main_RU}}

\EN{\input{patterns/205_floating_SIMD/main_EN}}
\RU{\input{patterns/205_floating_SIMD/main_RU}}
\DE{\input{patterns/205_floating_SIMD/main_DE}}

\EN{\input{patterns/ARM/main_EN}}
\RU{\input{patterns/ARM/main_RU}}
\DE{\input{patterns/ARM/main_DE}}

\input{patterns/MIPS/main}

\ifdefined\SPANISH
\chapter{Patrones de código}
\fi % SPANISH

\ifdefined\GERMAN
\chapter{Code-Muster}
\fi % GERMAN

\ifdefined\ENGLISH
\chapter{Code Patterns}
\fi % ENGLISH

\ifdefined\ITALIAN
\chapter{Forme di codice}
\fi % ITALIAN

\ifdefined\RUSSIAN
\chapter{Образцы кода}
\fi % RUSSIAN

\ifdefined\BRAZILIAN
\chapter{Padrões de códigos}
\fi % BRAZILIAN

\ifdefined\THAI
\chapter{รูปแบบของโค้ด}
\fi % THAI

\ifdefined\FRENCH
\chapter{Modèle de code}
\fi % FRENCH

\ifdefined\POLISH
\chapter{\PLph{}}
\fi % POLISH

% sections
\EN{\input{patterns/patterns_opt_dbg_EN}}
\ES{\input{patterns/patterns_opt_dbg_ES}}
\ITA{\input{patterns/patterns_opt_dbg_ITA}}
\PTBR{\input{patterns/patterns_opt_dbg_PTBR}}
\RU{\input{patterns/patterns_opt_dbg_RU}}
\THA{\input{patterns/patterns_opt_dbg_THA}}
\DE{\input{patterns/patterns_opt_dbg_DE}}
\FR{\input{patterns/patterns_opt_dbg_FR}}
\PL{\input{patterns/patterns_opt_dbg_PL}}

\RU{\section{Некоторые базовые понятия}}
\EN{\section{Some basics}}
\DE{\section{Einige Grundlagen}}
\FR{\section{Quelques bases}}
\ES{\section{\ESph{}}}
\ITA{\section{Alcune basi teoriche}}
\PTBR{\section{\PTBRph{}}}
\THA{\section{\THAph{}}}
\PL{\section{\PLph{}}}

% sections:
\EN{\input{patterns/intro_CPU_ISA_EN}}
\ES{\input{patterns/intro_CPU_ISA_ES}}
\ITA{\input{patterns/intro_CPU_ISA_ITA}}
\PTBR{\input{patterns/intro_CPU_ISA_PTBR}}
\RU{\input{patterns/intro_CPU_ISA_RU}}
\DE{\input{patterns/intro_CPU_ISA_DE}}
\FR{\input{patterns/intro_CPU_ISA_FR}}
\PL{\input{patterns/intro_CPU_ISA_PL}}

\EN{\input{patterns/numeral_EN}}
\RU{\input{patterns/numeral_RU}}
\ITA{\input{patterns/numeral_ITA}}
\DE{\input{patterns/numeral_DE}}
\FR{\input{patterns/numeral_FR}}
\PL{\input{patterns/numeral_PL}}

% chapters
\input{patterns/00_empty/main}
\input{patterns/011_ret/main}
\input{patterns/01_helloworld/main}
\input{patterns/015_prolog_epilogue/main}
\input{patterns/02_stack/main}
\input{patterns/03_printf/main}
\input{patterns/04_scanf/main}
\input{patterns/05_passing_arguments/main}
\input{patterns/06_return_results/main}
\input{patterns/061_pointers/main}
\input{patterns/065_GOTO/main}
\input{patterns/07_jcc/main}
\input{patterns/08_switch/main}
\input{patterns/09_loops/main}
\input{patterns/10_strings/main}
\input{patterns/11_arith_optimizations/main}
\input{patterns/12_FPU/main}
\input{patterns/13_arrays/main}
\input{patterns/14_bitfields/main}
\EN{\input{patterns/145_LCG/main_EN}}
\RU{\input{patterns/145_LCG/main_RU}}
\input{patterns/15_structs/main}
\input{patterns/17_unions/main}
\input{patterns/18_pointers_to_functions/main}
\input{patterns/185_64bit_in_32_env/main}

\EN{\input{patterns/19_SIMD/main_EN}}
\RU{\input{patterns/19_SIMD/main_RU}}
\DE{\input{patterns/19_SIMD/main_DE}}

\EN{\input{patterns/20_x64/main_EN}}
\RU{\input{patterns/20_x64/main_RU}}

\EN{\input{patterns/205_floating_SIMD/main_EN}}
\RU{\input{patterns/205_floating_SIMD/main_RU}}
\DE{\input{patterns/205_floating_SIMD/main_DE}}

\EN{\input{patterns/ARM/main_EN}}
\RU{\input{patterns/ARM/main_RU}}
\DE{\input{patterns/ARM/main_DE}}

\input{patterns/MIPS/main}

\ifdefined\SPANISH
\chapter{Patrones de código}
\fi % SPANISH

\ifdefined\GERMAN
\chapter{Code-Muster}
\fi % GERMAN

\ifdefined\ENGLISH
\chapter{Code Patterns}
\fi % ENGLISH

\ifdefined\ITALIAN
\chapter{Forme di codice}
\fi % ITALIAN

\ifdefined\RUSSIAN
\chapter{Образцы кода}
\fi % RUSSIAN

\ifdefined\BRAZILIAN
\chapter{Padrões de códigos}
\fi % BRAZILIAN

\ifdefined\THAI
\chapter{รูปแบบของโค้ด}
\fi % THAI

\ifdefined\FRENCH
\chapter{Modèle de code}
\fi % FRENCH

\ifdefined\POLISH
\chapter{\PLph{}}
\fi % POLISH

% sections
\EN{\input{patterns/patterns_opt_dbg_EN}}
\ES{\input{patterns/patterns_opt_dbg_ES}}
\ITA{\input{patterns/patterns_opt_dbg_ITA}}
\PTBR{\input{patterns/patterns_opt_dbg_PTBR}}
\RU{\input{patterns/patterns_opt_dbg_RU}}
\THA{\input{patterns/patterns_opt_dbg_THA}}
\DE{\input{patterns/patterns_opt_dbg_DE}}
\FR{\input{patterns/patterns_opt_dbg_FR}}
\PL{\input{patterns/patterns_opt_dbg_PL}}

\RU{\section{Некоторые базовые понятия}}
\EN{\section{Some basics}}
\DE{\section{Einige Grundlagen}}
\FR{\section{Quelques bases}}
\ES{\section{\ESph{}}}
\ITA{\section{Alcune basi teoriche}}
\PTBR{\section{\PTBRph{}}}
\THA{\section{\THAph{}}}
\PL{\section{\PLph{}}}

% sections:
\EN{\input{patterns/intro_CPU_ISA_EN}}
\ES{\input{patterns/intro_CPU_ISA_ES}}
\ITA{\input{patterns/intro_CPU_ISA_ITA}}
\PTBR{\input{patterns/intro_CPU_ISA_PTBR}}
\RU{\input{patterns/intro_CPU_ISA_RU}}
\DE{\input{patterns/intro_CPU_ISA_DE}}
\FR{\input{patterns/intro_CPU_ISA_FR}}
\PL{\input{patterns/intro_CPU_ISA_PL}}

\EN{\input{patterns/numeral_EN}}
\RU{\input{patterns/numeral_RU}}
\ITA{\input{patterns/numeral_ITA}}
\DE{\input{patterns/numeral_DE}}
\FR{\input{patterns/numeral_FR}}
\PL{\input{patterns/numeral_PL}}

% chapters
\input{patterns/00_empty/main}
\input{patterns/011_ret/main}
\input{patterns/01_helloworld/main}
\input{patterns/015_prolog_epilogue/main}
\input{patterns/02_stack/main}
\input{patterns/03_printf/main}
\input{patterns/04_scanf/main}
\input{patterns/05_passing_arguments/main}
\input{patterns/06_return_results/main}
\input{patterns/061_pointers/main}
\input{patterns/065_GOTO/main}
\input{patterns/07_jcc/main}
\input{patterns/08_switch/main}
\input{patterns/09_loops/main}
\input{patterns/10_strings/main}
\input{patterns/11_arith_optimizations/main}
\input{patterns/12_FPU/main}
\input{patterns/13_arrays/main}
\input{patterns/14_bitfields/main}
\EN{\input{patterns/145_LCG/main_EN}}
\RU{\input{patterns/145_LCG/main_RU}}
\input{patterns/15_structs/main}
\input{patterns/17_unions/main}
\input{patterns/18_pointers_to_functions/main}
\input{patterns/185_64bit_in_32_env/main}

\EN{\input{patterns/19_SIMD/main_EN}}
\RU{\input{patterns/19_SIMD/main_RU}}
\DE{\input{patterns/19_SIMD/main_DE}}

\EN{\input{patterns/20_x64/main_EN}}
\RU{\input{patterns/20_x64/main_RU}}

\EN{\input{patterns/205_floating_SIMD/main_EN}}
\RU{\input{patterns/205_floating_SIMD/main_RU}}
\DE{\input{patterns/205_floating_SIMD/main_DE}}

\EN{\input{patterns/ARM/main_EN}}
\RU{\input{patterns/ARM/main_RU}}
\DE{\input{patterns/ARM/main_DE}}

\input{patterns/MIPS/main}

\ifdefined\SPANISH
\chapter{Patrones de código}
\fi % SPANISH

\ifdefined\GERMAN
\chapter{Code-Muster}
\fi % GERMAN

\ifdefined\ENGLISH
\chapter{Code Patterns}
\fi % ENGLISH

\ifdefined\ITALIAN
\chapter{Forme di codice}
\fi % ITALIAN

\ifdefined\RUSSIAN
\chapter{Образцы кода}
\fi % RUSSIAN

\ifdefined\BRAZILIAN
\chapter{Padrões de códigos}
\fi % BRAZILIAN

\ifdefined\THAI
\chapter{รูปแบบของโค้ด}
\fi % THAI

\ifdefined\FRENCH
\chapter{Modèle de code}
\fi % FRENCH

\ifdefined\POLISH
\chapter{\PLph{}}
\fi % POLISH

% sections
\EN{\input{patterns/patterns_opt_dbg_EN}}
\ES{\input{patterns/patterns_opt_dbg_ES}}
\ITA{\input{patterns/patterns_opt_dbg_ITA}}
\PTBR{\input{patterns/patterns_opt_dbg_PTBR}}
\RU{\input{patterns/patterns_opt_dbg_RU}}
\THA{\input{patterns/patterns_opt_dbg_THA}}
\DE{\input{patterns/patterns_opt_dbg_DE}}
\FR{\input{patterns/patterns_opt_dbg_FR}}
\PL{\input{patterns/patterns_opt_dbg_PL}}

\RU{\section{Некоторые базовые понятия}}
\EN{\section{Some basics}}
\DE{\section{Einige Grundlagen}}
\FR{\section{Quelques bases}}
\ES{\section{\ESph{}}}
\ITA{\section{Alcune basi teoriche}}
\PTBR{\section{\PTBRph{}}}
\THA{\section{\THAph{}}}
\PL{\section{\PLph{}}}

% sections:
\EN{\input{patterns/intro_CPU_ISA_EN}}
\ES{\input{patterns/intro_CPU_ISA_ES}}
\ITA{\input{patterns/intro_CPU_ISA_ITA}}
\PTBR{\input{patterns/intro_CPU_ISA_PTBR}}
\RU{\input{patterns/intro_CPU_ISA_RU}}
\DE{\input{patterns/intro_CPU_ISA_DE}}
\FR{\input{patterns/intro_CPU_ISA_FR}}
\PL{\input{patterns/intro_CPU_ISA_PL}}

\EN{\input{patterns/numeral_EN}}
\RU{\input{patterns/numeral_RU}}
\ITA{\input{patterns/numeral_ITA}}
\DE{\input{patterns/numeral_DE}}
\FR{\input{patterns/numeral_FR}}
\PL{\input{patterns/numeral_PL}}

% chapters
\input{patterns/00_empty/main}
\input{patterns/011_ret/main}
\input{patterns/01_helloworld/main}
\input{patterns/015_prolog_epilogue/main}
\input{patterns/02_stack/main}
\input{patterns/03_printf/main}
\input{patterns/04_scanf/main}
\input{patterns/05_passing_arguments/main}
\input{patterns/06_return_results/main}
\input{patterns/061_pointers/main}
\input{patterns/065_GOTO/main}
\input{patterns/07_jcc/main}
\input{patterns/08_switch/main}
\input{patterns/09_loops/main}
\input{patterns/10_strings/main}
\input{patterns/11_arith_optimizations/main}
\input{patterns/12_FPU/main}
\input{patterns/13_arrays/main}
\input{patterns/14_bitfields/main}
\EN{\input{patterns/145_LCG/main_EN}}
\RU{\input{patterns/145_LCG/main_RU}}
\input{patterns/15_structs/main}
\input{patterns/17_unions/main}
\input{patterns/18_pointers_to_functions/main}
\input{patterns/185_64bit_in_32_env/main}

\EN{\input{patterns/19_SIMD/main_EN}}
\RU{\input{patterns/19_SIMD/main_RU}}
\DE{\input{patterns/19_SIMD/main_DE}}

\EN{\input{patterns/20_x64/main_EN}}
\RU{\input{patterns/20_x64/main_RU}}

\EN{\input{patterns/205_floating_SIMD/main_EN}}
\RU{\input{patterns/205_floating_SIMD/main_RU}}
\DE{\input{patterns/205_floating_SIMD/main_DE}}

\EN{\input{patterns/ARM/main_EN}}
\RU{\input{patterns/ARM/main_RU}}
\DE{\input{patterns/ARM/main_DE}}

\input{patterns/MIPS/main}

\ifdefined\SPANISH
\chapter{Patrones de código}
\fi % SPANISH

\ifdefined\GERMAN
\chapter{Code-Muster}
\fi % GERMAN

\ifdefined\ENGLISH
\chapter{Code Patterns}
\fi % ENGLISH

\ifdefined\ITALIAN
\chapter{Forme di codice}
\fi % ITALIAN

\ifdefined\RUSSIAN
\chapter{Образцы кода}
\fi % RUSSIAN

\ifdefined\BRAZILIAN
\chapter{Padrões de códigos}
\fi % BRAZILIAN

\ifdefined\THAI
\chapter{รูปแบบของโค้ด}
\fi % THAI

\ifdefined\FRENCH
\chapter{Modèle de code}
\fi % FRENCH

\ifdefined\POLISH
\chapter{\PLph{}}
\fi % POLISH

% sections
\EN{\input{patterns/patterns_opt_dbg_EN}}
\ES{\input{patterns/patterns_opt_dbg_ES}}
\ITA{\input{patterns/patterns_opt_dbg_ITA}}
\PTBR{\input{patterns/patterns_opt_dbg_PTBR}}
\RU{\input{patterns/patterns_opt_dbg_RU}}
\THA{\input{patterns/patterns_opt_dbg_THA}}
\DE{\input{patterns/patterns_opt_dbg_DE}}
\FR{\input{patterns/patterns_opt_dbg_FR}}
\PL{\input{patterns/patterns_opt_dbg_PL}}

\RU{\section{Некоторые базовые понятия}}
\EN{\section{Some basics}}
\DE{\section{Einige Grundlagen}}
\FR{\section{Quelques bases}}
\ES{\section{\ESph{}}}
\ITA{\section{Alcune basi teoriche}}
\PTBR{\section{\PTBRph{}}}
\THA{\section{\THAph{}}}
\PL{\section{\PLph{}}}

% sections:
\EN{\input{patterns/intro_CPU_ISA_EN}}
\ES{\input{patterns/intro_CPU_ISA_ES}}
\ITA{\input{patterns/intro_CPU_ISA_ITA}}
\PTBR{\input{patterns/intro_CPU_ISA_PTBR}}
\RU{\input{patterns/intro_CPU_ISA_RU}}
\DE{\input{patterns/intro_CPU_ISA_DE}}
\FR{\input{patterns/intro_CPU_ISA_FR}}
\PL{\input{patterns/intro_CPU_ISA_PL}}

\EN{\input{patterns/numeral_EN}}
\RU{\input{patterns/numeral_RU}}
\ITA{\input{patterns/numeral_ITA}}
\DE{\input{patterns/numeral_DE}}
\FR{\input{patterns/numeral_FR}}
\PL{\input{patterns/numeral_PL}}

% chapters
\input{patterns/00_empty/main}
\input{patterns/011_ret/main}
\input{patterns/01_helloworld/main}
\input{patterns/015_prolog_epilogue/main}
\input{patterns/02_stack/main}
\input{patterns/03_printf/main}
\input{patterns/04_scanf/main}
\input{patterns/05_passing_arguments/main}
\input{patterns/06_return_results/main}
\input{patterns/061_pointers/main}
\input{patterns/065_GOTO/main}
\input{patterns/07_jcc/main}
\input{patterns/08_switch/main}
\input{patterns/09_loops/main}
\input{patterns/10_strings/main}
\input{patterns/11_arith_optimizations/main}
\input{patterns/12_FPU/main}
\input{patterns/13_arrays/main}
\input{patterns/14_bitfields/main}
\EN{\input{patterns/145_LCG/main_EN}}
\RU{\input{patterns/145_LCG/main_RU}}
\input{patterns/15_structs/main}
\input{patterns/17_unions/main}
\input{patterns/18_pointers_to_functions/main}
\input{patterns/185_64bit_in_32_env/main}

\EN{\input{patterns/19_SIMD/main_EN}}
\RU{\input{patterns/19_SIMD/main_RU}}
\DE{\input{patterns/19_SIMD/main_DE}}

\EN{\input{patterns/20_x64/main_EN}}
\RU{\input{patterns/20_x64/main_RU}}

\EN{\input{patterns/205_floating_SIMD/main_EN}}
\RU{\input{patterns/205_floating_SIMD/main_RU}}
\DE{\input{patterns/205_floating_SIMD/main_DE}}

\EN{\input{patterns/ARM/main_EN}}
\RU{\input{patterns/ARM/main_RU}}
\DE{\input{patterns/ARM/main_DE}}

\input{patterns/MIPS/main}

\ifdefined\SPANISH
\chapter{Patrones de código}
\fi % SPANISH

\ifdefined\GERMAN
\chapter{Code-Muster}
\fi % GERMAN

\ifdefined\ENGLISH
\chapter{Code Patterns}
\fi % ENGLISH

\ifdefined\ITALIAN
\chapter{Forme di codice}
\fi % ITALIAN

\ifdefined\RUSSIAN
\chapter{Образцы кода}
\fi % RUSSIAN

\ifdefined\BRAZILIAN
\chapter{Padrões de códigos}
\fi % BRAZILIAN

\ifdefined\THAI
\chapter{รูปแบบของโค้ด}
\fi % THAI

\ifdefined\FRENCH
\chapter{Modèle de code}
\fi % FRENCH

\ifdefined\POLISH
\chapter{\PLph{}}
\fi % POLISH

% sections
\EN{\input{patterns/patterns_opt_dbg_EN}}
\ES{\input{patterns/patterns_opt_dbg_ES}}
\ITA{\input{patterns/patterns_opt_dbg_ITA}}
\PTBR{\input{patterns/patterns_opt_dbg_PTBR}}
\RU{\input{patterns/patterns_opt_dbg_RU}}
\THA{\input{patterns/patterns_opt_dbg_THA}}
\DE{\input{patterns/patterns_opt_dbg_DE}}
\FR{\input{patterns/patterns_opt_dbg_FR}}
\PL{\input{patterns/patterns_opt_dbg_PL}}

\RU{\section{Некоторые базовые понятия}}
\EN{\section{Some basics}}
\DE{\section{Einige Grundlagen}}
\FR{\section{Quelques bases}}
\ES{\section{\ESph{}}}
\ITA{\section{Alcune basi teoriche}}
\PTBR{\section{\PTBRph{}}}
\THA{\section{\THAph{}}}
\PL{\section{\PLph{}}}

% sections:
\EN{\input{patterns/intro_CPU_ISA_EN}}
\ES{\input{patterns/intro_CPU_ISA_ES}}
\ITA{\input{patterns/intro_CPU_ISA_ITA}}
\PTBR{\input{patterns/intro_CPU_ISA_PTBR}}
\RU{\input{patterns/intro_CPU_ISA_RU}}
\DE{\input{patterns/intro_CPU_ISA_DE}}
\FR{\input{patterns/intro_CPU_ISA_FR}}
\PL{\input{patterns/intro_CPU_ISA_PL}}

\EN{\input{patterns/numeral_EN}}
\RU{\input{patterns/numeral_RU}}
\ITA{\input{patterns/numeral_ITA}}
\DE{\input{patterns/numeral_DE}}
\FR{\input{patterns/numeral_FR}}
\PL{\input{patterns/numeral_PL}}

% chapters
\input{patterns/00_empty/main}
\input{patterns/011_ret/main}
\input{patterns/01_helloworld/main}
\input{patterns/015_prolog_epilogue/main}
\input{patterns/02_stack/main}
\input{patterns/03_printf/main}
\input{patterns/04_scanf/main}
\input{patterns/05_passing_arguments/main}
\input{patterns/06_return_results/main}
\input{patterns/061_pointers/main}
\input{patterns/065_GOTO/main}
\input{patterns/07_jcc/main}
\input{patterns/08_switch/main}
\input{patterns/09_loops/main}
\input{patterns/10_strings/main}
\input{patterns/11_arith_optimizations/main}
\input{patterns/12_FPU/main}
\input{patterns/13_arrays/main}
\input{patterns/14_bitfields/main}
\EN{\input{patterns/145_LCG/main_EN}}
\RU{\input{patterns/145_LCG/main_RU}}
\input{patterns/15_structs/main}
\input{patterns/17_unions/main}
\input{patterns/18_pointers_to_functions/main}
\input{patterns/185_64bit_in_32_env/main}

\EN{\input{patterns/19_SIMD/main_EN}}
\RU{\input{patterns/19_SIMD/main_RU}}
\DE{\input{patterns/19_SIMD/main_DE}}

\EN{\input{patterns/20_x64/main_EN}}
\RU{\input{patterns/20_x64/main_RU}}

\EN{\input{patterns/205_floating_SIMD/main_EN}}
\RU{\input{patterns/205_floating_SIMD/main_RU}}
\DE{\input{patterns/205_floating_SIMD/main_DE}}

\EN{\input{patterns/ARM/main_EN}}
\RU{\input{patterns/ARM/main_RU}}
\DE{\input{patterns/ARM/main_DE}}

\input{patterns/MIPS/main}

\EN{\input{patterns/12_FPU/main_EN}}
\RU{\input{patterns/12_FPU/main_RU}}
\DE{\input{patterns/12_FPU/main_DE}}
\FR{\input{patterns/12_FPU/main_FR}}


\ifdefined\SPANISH
\chapter{Patrones de código}
\fi % SPANISH

\ifdefined\GERMAN
\chapter{Code-Muster}
\fi % GERMAN

\ifdefined\ENGLISH
\chapter{Code Patterns}
\fi % ENGLISH

\ifdefined\ITALIAN
\chapter{Forme di codice}
\fi % ITALIAN

\ifdefined\RUSSIAN
\chapter{Образцы кода}
\fi % RUSSIAN

\ifdefined\BRAZILIAN
\chapter{Padrões de códigos}
\fi % BRAZILIAN

\ifdefined\THAI
\chapter{รูปแบบของโค้ด}
\fi % THAI

\ifdefined\FRENCH
\chapter{Modèle de code}
\fi % FRENCH

\ifdefined\POLISH
\chapter{\PLph{}}
\fi % POLISH

% sections
\EN{\input{patterns/patterns_opt_dbg_EN}}
\ES{\input{patterns/patterns_opt_dbg_ES}}
\ITA{\input{patterns/patterns_opt_dbg_ITA}}
\PTBR{\input{patterns/patterns_opt_dbg_PTBR}}
\RU{\input{patterns/patterns_opt_dbg_RU}}
\THA{\input{patterns/patterns_opt_dbg_THA}}
\DE{\input{patterns/patterns_opt_dbg_DE}}
\FR{\input{patterns/patterns_opt_dbg_FR}}
\PL{\input{patterns/patterns_opt_dbg_PL}}

\RU{\section{Некоторые базовые понятия}}
\EN{\section{Some basics}}
\DE{\section{Einige Grundlagen}}
\FR{\section{Quelques bases}}
\ES{\section{\ESph{}}}
\ITA{\section{Alcune basi teoriche}}
\PTBR{\section{\PTBRph{}}}
\THA{\section{\THAph{}}}
\PL{\section{\PLph{}}}

% sections:
\EN{\input{patterns/intro_CPU_ISA_EN}}
\ES{\input{patterns/intro_CPU_ISA_ES}}
\ITA{\input{patterns/intro_CPU_ISA_ITA}}
\PTBR{\input{patterns/intro_CPU_ISA_PTBR}}
\RU{\input{patterns/intro_CPU_ISA_RU}}
\DE{\input{patterns/intro_CPU_ISA_DE}}
\FR{\input{patterns/intro_CPU_ISA_FR}}
\PL{\input{patterns/intro_CPU_ISA_PL}}

\EN{\input{patterns/numeral_EN}}
\RU{\input{patterns/numeral_RU}}
\ITA{\input{patterns/numeral_ITA}}
\DE{\input{patterns/numeral_DE}}
\FR{\input{patterns/numeral_FR}}
\PL{\input{patterns/numeral_PL}}

% chapters
\input{patterns/00_empty/main}
\input{patterns/011_ret/main}
\input{patterns/01_helloworld/main}
\input{patterns/015_prolog_epilogue/main}
\input{patterns/02_stack/main}
\input{patterns/03_printf/main}
\input{patterns/04_scanf/main}
\input{patterns/05_passing_arguments/main}
\input{patterns/06_return_results/main}
\input{patterns/061_pointers/main}
\input{patterns/065_GOTO/main}
\input{patterns/07_jcc/main}
\input{patterns/08_switch/main}
\input{patterns/09_loops/main}
\input{patterns/10_strings/main}
\input{patterns/11_arith_optimizations/main}
\input{patterns/12_FPU/main}
\input{patterns/13_arrays/main}
\input{patterns/14_bitfields/main}
\EN{\input{patterns/145_LCG/main_EN}}
\RU{\input{patterns/145_LCG/main_RU}}
\input{patterns/15_structs/main}
\input{patterns/17_unions/main}
\input{patterns/18_pointers_to_functions/main}
\input{patterns/185_64bit_in_32_env/main}

\EN{\input{patterns/19_SIMD/main_EN}}
\RU{\input{patterns/19_SIMD/main_RU}}
\DE{\input{patterns/19_SIMD/main_DE}}

\EN{\input{patterns/20_x64/main_EN}}
\RU{\input{patterns/20_x64/main_RU}}

\EN{\input{patterns/205_floating_SIMD/main_EN}}
\RU{\input{patterns/205_floating_SIMD/main_RU}}
\DE{\input{patterns/205_floating_SIMD/main_DE}}

\EN{\input{patterns/ARM/main_EN}}
\RU{\input{patterns/ARM/main_RU}}
\DE{\input{patterns/ARM/main_DE}}

\input{patterns/MIPS/main}

\ifdefined\SPANISH
\chapter{Patrones de código}
\fi % SPANISH

\ifdefined\GERMAN
\chapter{Code-Muster}
\fi % GERMAN

\ifdefined\ENGLISH
\chapter{Code Patterns}
\fi % ENGLISH

\ifdefined\ITALIAN
\chapter{Forme di codice}
\fi % ITALIAN

\ifdefined\RUSSIAN
\chapter{Образцы кода}
\fi % RUSSIAN

\ifdefined\BRAZILIAN
\chapter{Padrões de códigos}
\fi % BRAZILIAN

\ifdefined\THAI
\chapter{รูปแบบของโค้ด}
\fi % THAI

\ifdefined\FRENCH
\chapter{Modèle de code}
\fi % FRENCH

\ifdefined\POLISH
\chapter{\PLph{}}
\fi % POLISH

% sections
\EN{\input{patterns/patterns_opt_dbg_EN}}
\ES{\input{patterns/patterns_opt_dbg_ES}}
\ITA{\input{patterns/patterns_opt_dbg_ITA}}
\PTBR{\input{patterns/patterns_opt_dbg_PTBR}}
\RU{\input{patterns/patterns_opt_dbg_RU}}
\THA{\input{patterns/patterns_opt_dbg_THA}}
\DE{\input{patterns/patterns_opt_dbg_DE}}
\FR{\input{patterns/patterns_opt_dbg_FR}}
\PL{\input{patterns/patterns_opt_dbg_PL}}

\RU{\section{Некоторые базовые понятия}}
\EN{\section{Some basics}}
\DE{\section{Einige Grundlagen}}
\FR{\section{Quelques bases}}
\ES{\section{\ESph{}}}
\ITA{\section{Alcune basi teoriche}}
\PTBR{\section{\PTBRph{}}}
\THA{\section{\THAph{}}}
\PL{\section{\PLph{}}}

% sections:
\EN{\input{patterns/intro_CPU_ISA_EN}}
\ES{\input{patterns/intro_CPU_ISA_ES}}
\ITA{\input{patterns/intro_CPU_ISA_ITA}}
\PTBR{\input{patterns/intro_CPU_ISA_PTBR}}
\RU{\input{patterns/intro_CPU_ISA_RU}}
\DE{\input{patterns/intro_CPU_ISA_DE}}
\FR{\input{patterns/intro_CPU_ISA_FR}}
\PL{\input{patterns/intro_CPU_ISA_PL}}

\EN{\input{patterns/numeral_EN}}
\RU{\input{patterns/numeral_RU}}
\ITA{\input{patterns/numeral_ITA}}
\DE{\input{patterns/numeral_DE}}
\FR{\input{patterns/numeral_FR}}
\PL{\input{patterns/numeral_PL}}

% chapters
\input{patterns/00_empty/main}
\input{patterns/011_ret/main}
\input{patterns/01_helloworld/main}
\input{patterns/015_prolog_epilogue/main}
\input{patterns/02_stack/main}
\input{patterns/03_printf/main}
\input{patterns/04_scanf/main}
\input{patterns/05_passing_arguments/main}
\input{patterns/06_return_results/main}
\input{patterns/061_pointers/main}
\input{patterns/065_GOTO/main}
\input{patterns/07_jcc/main}
\input{patterns/08_switch/main}
\input{patterns/09_loops/main}
\input{patterns/10_strings/main}
\input{patterns/11_arith_optimizations/main}
\input{patterns/12_FPU/main}
\input{patterns/13_arrays/main}
\input{patterns/14_bitfields/main}
\EN{\input{patterns/145_LCG/main_EN}}
\RU{\input{patterns/145_LCG/main_RU}}
\input{patterns/15_structs/main}
\input{patterns/17_unions/main}
\input{patterns/18_pointers_to_functions/main}
\input{patterns/185_64bit_in_32_env/main}

\EN{\input{patterns/19_SIMD/main_EN}}
\RU{\input{patterns/19_SIMD/main_RU}}
\DE{\input{patterns/19_SIMD/main_DE}}

\EN{\input{patterns/20_x64/main_EN}}
\RU{\input{patterns/20_x64/main_RU}}

\EN{\input{patterns/205_floating_SIMD/main_EN}}
\RU{\input{patterns/205_floating_SIMD/main_RU}}
\DE{\input{patterns/205_floating_SIMD/main_DE}}

\EN{\input{patterns/ARM/main_EN}}
\RU{\input{patterns/ARM/main_RU}}
\DE{\input{patterns/ARM/main_DE}}

\input{patterns/MIPS/main}

\EN{\section{Returning Values}
\label{ret_val_func}

Another simple function is the one that simply returns a constant value:

\lstinputlisting[caption=\EN{\CCpp Code},style=customc]{patterns/011_ret/1.c}

Let's compile it.

\subsection{x86}

Here's what both the GCC and MSVC compilers produce (with optimization) on the x86 platform:

\lstinputlisting[caption=\Optimizing GCC/MSVC (\assemblyOutput),style=customasmx86]{patterns/011_ret/1.s}

\myindex{x86!\Instructions!RET}
There are just two instructions: the first places the value 123 into the \EAX register,
which is used by convention for storing the return
value, and the second one is \RET, which returns execution to the \gls{caller}.

The caller will take the result from the \EAX register.

\subsection{ARM}

There are a few differences on the ARM platform:

\lstinputlisting[caption=\OptimizingKeilVI (\ARMMode) ASM Output,style=customasmARM]{patterns/011_ret/1_Keil_ARM_O3.s}

ARM uses the register \Reg{0} for returning the results of functions, so 123 is copied into \Reg{0}.

\myindex{ARM!\Instructions!MOV}
\myindex{x86!\Instructions!MOV}
It is worth noting that \MOV is a misleading name for the instruction in both the x86 and ARM \ac{ISA}s.

The data is not in fact \IT{moved}, but \IT{copied}.

\subsection{MIPS}

\label{MIPS_leaf_function_ex1}

The GCC assembly output below lists registers by number:

\lstinputlisting[caption=\Optimizing GCC 4.4.5 (\assemblyOutput),style=customasmMIPS]{patterns/011_ret/MIPS.s}

\dots while \IDA does it by their pseudo names:

\lstinputlisting[caption=\Optimizing GCC 4.4.5 (IDA),style=customasmMIPS]{patterns/011_ret/MIPS_IDA.lst}

The \$2 (or \$V0) register is used to store the function's return value.
\myindex{MIPS!\Pseudoinstructions!LI}
\INS{LI} stands for ``Load Immediate'' and is the MIPS equivalent to \MOV.

\myindex{MIPS!\Instructions!J}
The other instruction is the jump instruction (J or JR) which returns the execution flow to the \gls{caller}.

\myindex{MIPS!Branch delay slot}
You might be wondering why the positions of the load instruction (LI) and the jump instruction (J or JR) are swapped. This is due to a \ac{RISC} feature called ``branch delay slot''.

The reason this happens is a quirk in the architecture of some RISC \ac{ISA}s and isn't important for our
purposes---we must simply keep in mind that in MIPS, the instruction following a jump or branch instruction
is executed \IT{before} the jump/branch instruction itself.

As a consequence, branch instructions always swap places with the instruction executed immediately beforehand.


In practice, functions which merely return 1 (\IT{true}) or 0 (\IT{false}) are very frequent.

The smallest ever of the standard UNIX utilities, \IT{/bin/true} and \IT{/bin/false} return 0 and 1 respectively, as an exit code.
(Zero as an exit code usually means success, non-zero means error.)
}
\RU{\subsubsection{std::string}
\myindex{\Cpp!STL!std::string}
\label{std_string}

\myparagraph{Как устроена структура}

Многие строковые библиотеки \InSqBrackets{\CNotes 2.2} обеспечивают структуру содержащую ссылку 
на буфер собственно со строкой, переменная всегда содержащую длину строки 
(что очень удобно для массы функций \InSqBrackets{\CNotes 2.2.1}) и переменную содержащую текущий размер буфера.

Строка в буфере обыкновенно оканчивается нулем: это для того чтобы указатель на буфер можно было
передавать в функции требующие на вход обычную сишную \ac{ASCIIZ}-строку.

Стандарт \Cpp не описывает, как именно нужно реализовывать std::string,
но, как правило, они реализованы как описано выше, с небольшими дополнениями.

Строки в \Cpp это не класс (как, например, QString в Qt), а темплейт (basic\_string), 
это сделано для того чтобы поддерживать 
строки содержащие разного типа символы: как минимум \Tchar и \IT{wchar\_t}.

Так что, std::string это класс с базовым типом \Tchar.

А std::wstring это класс с базовым типом \IT{wchar\_t}.

\mysubparagraph{MSVC}

В реализации MSVC, вместо ссылки на буфер может содержаться сам буфер (если строка короче 16-и символов).

Это означает, что каждая короткая строка будет занимать в памяти по крайней мере $16 + 4 + 4 = 24$ 
байт для 32-битной среды либо $16 + 8 + 8 = 32$ 
байта в 64-битной, а если строка длиннее 16-и символов, то прибавьте еще длину самой строки.

\lstinputlisting[caption=пример для MSVC,style=customc]{\CURPATH/STL/string/MSVC_RU.cpp}

Собственно, из этого исходника почти всё ясно.

Несколько замечаний:

Если строка короче 16-и символов, 
то отдельный буфер для строки в \glslink{heap}{куче} выделяться не будет.

Это удобно потому что на практике, основная часть строк действительно короткие.
Вероятно, разработчики в Microsoft выбрали размер в 16 символов как разумный баланс.

Теперь очень важный момент в конце функции main(): мы не пользуемся методом c\_str(), тем не менее,
если это скомпилировать и запустить, то обе строки появятся в консоли!

Работает это вот почему.

В первом случае строка короче 16-и символов и в начале объекта std::string (его можно рассматривать
просто как структуру) расположен буфер с этой строкой.
\printf трактует указатель как указатель на массив символов оканчивающийся нулем и поэтому всё работает.

Вывод второй строки (длиннее 16-и символов) даже еще опаснее: это вообще типичная программистская ошибка 
(или опечатка), забыть дописать c\_str().
Это работает потому что в это время в начале структуры расположен указатель на буфер.
Это может надолго остаться незамеченным: до тех пока там не появится строка 
короче 16-и символов, тогда процесс упадет.

\mysubparagraph{GCC}

В реализации GCC в структуре есть еще одна переменная --- reference count.

Интересно, что указатель на экземпляр класса std::string в GCC указывает не на начало самой структуры, 
а на указатель на буфера.
В libstdc++-v3\textbackslash{}include\textbackslash{}bits\textbackslash{}basic\_string.h 
мы можем прочитать что это сделано для удобства отладки:

\begin{lstlisting}
   *  The reason you want _M_data pointing to the character %array and
   *  not the _Rep is so that the debugger can see the string
   *  contents. (Probably we should add a non-inline member to get
   *  the _Rep for the debugger to use, so users can check the actual
   *  string length.)
\end{lstlisting}

\href{http://go.yurichev.com/17085}{исходный код basic\_string.h}

В нашем примере мы учитываем это:

\lstinputlisting[caption=пример для GCC,style=customc]{\CURPATH/STL/string/GCC_RU.cpp}

Нужны еще небольшие хаки чтобы сымитировать типичную ошибку, которую мы уже видели выше, из-за
более ужесточенной проверки типов в GCC, тем не менее, printf() работает и здесь без c\_str().

\myparagraph{Чуть более сложный пример}

\lstinputlisting[style=customc]{\CURPATH/STL/string/3.cpp}

\lstinputlisting[caption=MSVC 2012,style=customasmx86]{\CURPATH/STL/string/3_MSVC_RU.asm}

Собственно, компилятор не конструирует строки статически: да в общем-то и как
это возможно, если буфер с ней нужно хранить в \glslink{heap}{куче}?

Вместо этого в сегменте данных хранятся обычные \ac{ASCIIZ}-строки, а позже, во время выполнения, 
при помощи метода \q{assign}, конструируются строки s1 и s2
.
При помощи \TT{operator+}, создается строка s3.

Обратите внимание на то что вызов метода c\_str() отсутствует,
потому что его код достаточно короткий и компилятор вставил его прямо здесь:
если строка короче 16-и байт, то в регистре EAX остается указатель на буфер,
а если длиннее, то из этого же места достается адрес на буфер расположенный в \glslink{heap}{куче}.

Далее следуют вызовы трех деструкторов, причем, они вызываются только если строка длиннее 16-и байт:
тогда нужно освободить буфера в \glslink{heap}{куче}.
В противном случае, так как все три объекта std::string хранятся в стеке,
они освобождаются автоматически после выхода из функции.

Следовательно, работа с короткими строками более быстрая из-за м\'{е}ньшего обращения к \glslink{heap}{куче}.

Код на GCC даже проще (из-за того, что в GCC, как мы уже видели, не реализована возможность хранить короткую
строку прямо в структуре):

% TODO1 comment each function meaning
\lstinputlisting[caption=GCC 4.8.1,style=customasmx86]{\CURPATH/STL/string/3_GCC_RU.s}

Можно заметить, что в деструкторы передается не указатель на объект,
а указатель на место за 12 байт (или 3 слова) перед ним, то есть, на настоящее начало структуры.

\myparagraph{std::string как глобальная переменная}
\label{sec:std_string_as_global_variable}

Опытные программисты на \Cpp знают, что глобальные переменные \ac{STL}-типов вполне можно объявлять.

Да, действительно:

\lstinputlisting[style=customc]{\CURPATH/STL/string/5.cpp}

Но как и где будет вызываться конструктор \TT{std::string}?

На самом деле, эта переменная будет инициализирована даже перед началом \main.

\lstinputlisting[caption=MSVC 2012: здесь конструируется глобальная переменная{,} а также регистрируется её деструктор,style=customasmx86]{\CURPATH/STL/string/5_MSVC_p2.asm}

\lstinputlisting[caption=MSVC 2012: здесь глобальная переменная используется в \main,style=customasmx86]{\CURPATH/STL/string/5_MSVC_p1.asm}

\lstinputlisting[caption=MSVC 2012: эта функция-деструктор вызывается перед выходом,style=customasmx86]{\CURPATH/STL/string/5_MSVC_p3.asm}

\myindex{\CStandardLibrary!atexit()}
В реальности, из \ac{CRT}, еще до вызова main(), вызывается специальная функция,
в которой перечислены все конструкторы подобных переменных.
Более того: при помощи atexit() регистрируется функция, которая будет вызвана в конце работы программы:
в этой функции компилятор собирает вызовы деструкторов всех подобных глобальных переменных.

GCC работает похожим образом:

\lstinputlisting[caption=GCC 4.8.1,style=customasmx86]{\CURPATH/STL/string/5_GCC.s}

Но он не выделяет отдельной функции в которой будут собраны деструкторы: 
каждый деструктор передается в atexit() по одному.

% TODO а если глобальная STL-переменная в другом модуле? надо проверить.

}
\ifdefined\SPANISH
\chapter{Patrones de código}
\fi % SPANISH

\ifdefined\GERMAN
\chapter{Code-Muster}
\fi % GERMAN

\ifdefined\ENGLISH
\chapter{Code Patterns}
\fi % ENGLISH

\ifdefined\ITALIAN
\chapter{Forme di codice}
\fi % ITALIAN

\ifdefined\RUSSIAN
\chapter{Образцы кода}
\fi % RUSSIAN

\ifdefined\BRAZILIAN
\chapter{Padrões de códigos}
\fi % BRAZILIAN

\ifdefined\THAI
\chapter{รูปแบบของโค้ด}
\fi % THAI

\ifdefined\FRENCH
\chapter{Modèle de code}
\fi % FRENCH

\ifdefined\POLISH
\chapter{\PLph{}}
\fi % POLISH

% sections
\EN{\input{patterns/patterns_opt_dbg_EN}}
\ES{\input{patterns/patterns_opt_dbg_ES}}
\ITA{\input{patterns/patterns_opt_dbg_ITA}}
\PTBR{\input{patterns/patterns_opt_dbg_PTBR}}
\RU{\input{patterns/patterns_opt_dbg_RU}}
\THA{\input{patterns/patterns_opt_dbg_THA}}
\DE{\input{patterns/patterns_opt_dbg_DE}}
\FR{\input{patterns/patterns_opt_dbg_FR}}
\PL{\input{patterns/patterns_opt_dbg_PL}}

\RU{\section{Некоторые базовые понятия}}
\EN{\section{Some basics}}
\DE{\section{Einige Grundlagen}}
\FR{\section{Quelques bases}}
\ES{\section{\ESph{}}}
\ITA{\section{Alcune basi teoriche}}
\PTBR{\section{\PTBRph{}}}
\THA{\section{\THAph{}}}
\PL{\section{\PLph{}}}

% sections:
\EN{\input{patterns/intro_CPU_ISA_EN}}
\ES{\input{patterns/intro_CPU_ISA_ES}}
\ITA{\input{patterns/intro_CPU_ISA_ITA}}
\PTBR{\input{patterns/intro_CPU_ISA_PTBR}}
\RU{\input{patterns/intro_CPU_ISA_RU}}
\DE{\input{patterns/intro_CPU_ISA_DE}}
\FR{\input{patterns/intro_CPU_ISA_FR}}
\PL{\input{patterns/intro_CPU_ISA_PL}}

\EN{\input{patterns/numeral_EN}}
\RU{\input{patterns/numeral_RU}}
\ITA{\input{patterns/numeral_ITA}}
\DE{\input{patterns/numeral_DE}}
\FR{\input{patterns/numeral_FR}}
\PL{\input{patterns/numeral_PL}}

% chapters
\input{patterns/00_empty/main}
\input{patterns/011_ret/main}
\input{patterns/01_helloworld/main}
\input{patterns/015_prolog_epilogue/main}
\input{patterns/02_stack/main}
\input{patterns/03_printf/main}
\input{patterns/04_scanf/main}
\input{patterns/05_passing_arguments/main}
\input{patterns/06_return_results/main}
\input{patterns/061_pointers/main}
\input{patterns/065_GOTO/main}
\input{patterns/07_jcc/main}
\input{patterns/08_switch/main}
\input{patterns/09_loops/main}
\input{patterns/10_strings/main}
\input{patterns/11_arith_optimizations/main}
\input{patterns/12_FPU/main}
\input{patterns/13_arrays/main}
\input{patterns/14_bitfields/main}
\EN{\input{patterns/145_LCG/main_EN}}
\RU{\input{patterns/145_LCG/main_RU}}
\input{patterns/15_structs/main}
\input{patterns/17_unions/main}
\input{patterns/18_pointers_to_functions/main}
\input{patterns/185_64bit_in_32_env/main}

\EN{\input{patterns/19_SIMD/main_EN}}
\RU{\input{patterns/19_SIMD/main_RU}}
\DE{\input{patterns/19_SIMD/main_DE}}

\EN{\input{patterns/20_x64/main_EN}}
\RU{\input{patterns/20_x64/main_RU}}

\EN{\input{patterns/205_floating_SIMD/main_EN}}
\RU{\input{patterns/205_floating_SIMD/main_RU}}
\DE{\input{patterns/205_floating_SIMD/main_DE}}

\EN{\input{patterns/ARM/main_EN}}
\RU{\input{patterns/ARM/main_RU}}
\DE{\input{patterns/ARM/main_DE}}

\input{patterns/MIPS/main}

\ifdefined\SPANISH
\chapter{Patrones de código}
\fi % SPANISH

\ifdefined\GERMAN
\chapter{Code-Muster}
\fi % GERMAN

\ifdefined\ENGLISH
\chapter{Code Patterns}
\fi % ENGLISH

\ifdefined\ITALIAN
\chapter{Forme di codice}
\fi % ITALIAN

\ifdefined\RUSSIAN
\chapter{Образцы кода}
\fi % RUSSIAN

\ifdefined\BRAZILIAN
\chapter{Padrões de códigos}
\fi % BRAZILIAN

\ifdefined\THAI
\chapter{รูปแบบของโค้ด}
\fi % THAI

\ifdefined\FRENCH
\chapter{Modèle de code}
\fi % FRENCH

\ifdefined\POLISH
\chapter{\PLph{}}
\fi % POLISH

% sections
\EN{\input{patterns/patterns_opt_dbg_EN}}
\ES{\input{patterns/patterns_opt_dbg_ES}}
\ITA{\input{patterns/patterns_opt_dbg_ITA}}
\PTBR{\input{patterns/patterns_opt_dbg_PTBR}}
\RU{\input{patterns/patterns_opt_dbg_RU}}
\THA{\input{patterns/patterns_opt_dbg_THA}}
\DE{\input{patterns/patterns_opt_dbg_DE}}
\FR{\input{patterns/patterns_opt_dbg_FR}}
\PL{\input{patterns/patterns_opt_dbg_PL}}

\RU{\section{Некоторые базовые понятия}}
\EN{\section{Some basics}}
\DE{\section{Einige Grundlagen}}
\FR{\section{Quelques bases}}
\ES{\section{\ESph{}}}
\ITA{\section{Alcune basi teoriche}}
\PTBR{\section{\PTBRph{}}}
\THA{\section{\THAph{}}}
\PL{\section{\PLph{}}}

% sections:
\EN{\input{patterns/intro_CPU_ISA_EN}}
\ES{\input{patterns/intro_CPU_ISA_ES}}
\ITA{\input{patterns/intro_CPU_ISA_ITA}}
\PTBR{\input{patterns/intro_CPU_ISA_PTBR}}
\RU{\input{patterns/intro_CPU_ISA_RU}}
\DE{\input{patterns/intro_CPU_ISA_DE}}
\FR{\input{patterns/intro_CPU_ISA_FR}}
\PL{\input{patterns/intro_CPU_ISA_PL}}

\EN{\input{patterns/numeral_EN}}
\RU{\input{patterns/numeral_RU}}
\ITA{\input{patterns/numeral_ITA}}
\DE{\input{patterns/numeral_DE}}
\FR{\input{patterns/numeral_FR}}
\PL{\input{patterns/numeral_PL}}

% chapters
\input{patterns/00_empty/main}
\input{patterns/011_ret/main}
\input{patterns/01_helloworld/main}
\input{patterns/015_prolog_epilogue/main}
\input{patterns/02_stack/main}
\input{patterns/03_printf/main}
\input{patterns/04_scanf/main}
\input{patterns/05_passing_arguments/main}
\input{patterns/06_return_results/main}
\input{patterns/061_pointers/main}
\input{patterns/065_GOTO/main}
\input{patterns/07_jcc/main}
\input{patterns/08_switch/main}
\input{patterns/09_loops/main}
\input{patterns/10_strings/main}
\input{patterns/11_arith_optimizations/main}
\input{patterns/12_FPU/main}
\input{patterns/13_arrays/main}
\input{patterns/14_bitfields/main}
\EN{\input{patterns/145_LCG/main_EN}}
\RU{\input{patterns/145_LCG/main_RU}}
\input{patterns/15_structs/main}
\input{patterns/17_unions/main}
\input{patterns/18_pointers_to_functions/main}
\input{patterns/185_64bit_in_32_env/main}

\EN{\input{patterns/19_SIMD/main_EN}}
\RU{\input{patterns/19_SIMD/main_RU}}
\DE{\input{patterns/19_SIMD/main_DE}}

\EN{\input{patterns/20_x64/main_EN}}
\RU{\input{patterns/20_x64/main_RU}}

\EN{\input{patterns/205_floating_SIMD/main_EN}}
\RU{\input{patterns/205_floating_SIMD/main_RU}}
\DE{\input{patterns/205_floating_SIMD/main_DE}}

\EN{\input{patterns/ARM/main_EN}}
\RU{\input{patterns/ARM/main_RU}}
\DE{\input{patterns/ARM/main_DE}}

\input{patterns/MIPS/main}

\ifdefined\SPANISH
\chapter{Patrones de código}
\fi % SPANISH

\ifdefined\GERMAN
\chapter{Code-Muster}
\fi % GERMAN

\ifdefined\ENGLISH
\chapter{Code Patterns}
\fi % ENGLISH

\ifdefined\ITALIAN
\chapter{Forme di codice}
\fi % ITALIAN

\ifdefined\RUSSIAN
\chapter{Образцы кода}
\fi % RUSSIAN

\ifdefined\BRAZILIAN
\chapter{Padrões de códigos}
\fi % BRAZILIAN

\ifdefined\THAI
\chapter{รูปแบบของโค้ด}
\fi % THAI

\ifdefined\FRENCH
\chapter{Modèle de code}
\fi % FRENCH

\ifdefined\POLISH
\chapter{\PLph{}}
\fi % POLISH

% sections
\EN{\input{patterns/patterns_opt_dbg_EN}}
\ES{\input{patterns/patterns_opt_dbg_ES}}
\ITA{\input{patterns/patterns_opt_dbg_ITA}}
\PTBR{\input{patterns/patterns_opt_dbg_PTBR}}
\RU{\input{patterns/patterns_opt_dbg_RU}}
\THA{\input{patterns/patterns_opt_dbg_THA}}
\DE{\input{patterns/patterns_opt_dbg_DE}}
\FR{\input{patterns/patterns_opt_dbg_FR}}
\PL{\input{patterns/patterns_opt_dbg_PL}}

\RU{\section{Некоторые базовые понятия}}
\EN{\section{Some basics}}
\DE{\section{Einige Grundlagen}}
\FR{\section{Quelques bases}}
\ES{\section{\ESph{}}}
\ITA{\section{Alcune basi teoriche}}
\PTBR{\section{\PTBRph{}}}
\THA{\section{\THAph{}}}
\PL{\section{\PLph{}}}

% sections:
\EN{\input{patterns/intro_CPU_ISA_EN}}
\ES{\input{patterns/intro_CPU_ISA_ES}}
\ITA{\input{patterns/intro_CPU_ISA_ITA}}
\PTBR{\input{patterns/intro_CPU_ISA_PTBR}}
\RU{\input{patterns/intro_CPU_ISA_RU}}
\DE{\input{patterns/intro_CPU_ISA_DE}}
\FR{\input{patterns/intro_CPU_ISA_FR}}
\PL{\input{patterns/intro_CPU_ISA_PL}}

\EN{\input{patterns/numeral_EN}}
\RU{\input{patterns/numeral_RU}}
\ITA{\input{patterns/numeral_ITA}}
\DE{\input{patterns/numeral_DE}}
\FR{\input{patterns/numeral_FR}}
\PL{\input{patterns/numeral_PL}}

% chapters
\input{patterns/00_empty/main}
\input{patterns/011_ret/main}
\input{patterns/01_helloworld/main}
\input{patterns/015_prolog_epilogue/main}
\input{patterns/02_stack/main}
\input{patterns/03_printf/main}
\input{patterns/04_scanf/main}
\input{patterns/05_passing_arguments/main}
\input{patterns/06_return_results/main}
\input{patterns/061_pointers/main}
\input{patterns/065_GOTO/main}
\input{patterns/07_jcc/main}
\input{patterns/08_switch/main}
\input{patterns/09_loops/main}
\input{patterns/10_strings/main}
\input{patterns/11_arith_optimizations/main}
\input{patterns/12_FPU/main}
\input{patterns/13_arrays/main}
\input{patterns/14_bitfields/main}
\EN{\input{patterns/145_LCG/main_EN}}
\RU{\input{patterns/145_LCG/main_RU}}
\input{patterns/15_structs/main}
\input{patterns/17_unions/main}
\input{patterns/18_pointers_to_functions/main}
\input{patterns/185_64bit_in_32_env/main}

\EN{\input{patterns/19_SIMD/main_EN}}
\RU{\input{patterns/19_SIMD/main_RU}}
\DE{\input{patterns/19_SIMD/main_DE}}

\EN{\input{patterns/20_x64/main_EN}}
\RU{\input{patterns/20_x64/main_RU}}

\EN{\input{patterns/205_floating_SIMD/main_EN}}
\RU{\input{patterns/205_floating_SIMD/main_RU}}
\DE{\input{patterns/205_floating_SIMD/main_DE}}

\EN{\input{patterns/ARM/main_EN}}
\RU{\input{patterns/ARM/main_RU}}
\DE{\input{patterns/ARM/main_DE}}

\input{patterns/MIPS/main}

\ifdefined\SPANISH
\chapter{Patrones de código}
\fi % SPANISH

\ifdefined\GERMAN
\chapter{Code-Muster}
\fi % GERMAN

\ifdefined\ENGLISH
\chapter{Code Patterns}
\fi % ENGLISH

\ifdefined\ITALIAN
\chapter{Forme di codice}
\fi % ITALIAN

\ifdefined\RUSSIAN
\chapter{Образцы кода}
\fi % RUSSIAN

\ifdefined\BRAZILIAN
\chapter{Padrões de códigos}
\fi % BRAZILIAN

\ifdefined\THAI
\chapter{รูปแบบของโค้ด}
\fi % THAI

\ifdefined\FRENCH
\chapter{Modèle de code}
\fi % FRENCH

\ifdefined\POLISH
\chapter{\PLph{}}
\fi % POLISH

% sections
\EN{\input{patterns/patterns_opt_dbg_EN}}
\ES{\input{patterns/patterns_opt_dbg_ES}}
\ITA{\input{patterns/patterns_opt_dbg_ITA}}
\PTBR{\input{patterns/patterns_opt_dbg_PTBR}}
\RU{\input{patterns/patterns_opt_dbg_RU}}
\THA{\input{patterns/patterns_opt_dbg_THA}}
\DE{\input{patterns/patterns_opt_dbg_DE}}
\FR{\input{patterns/patterns_opt_dbg_FR}}
\PL{\input{patterns/patterns_opt_dbg_PL}}

\RU{\section{Некоторые базовые понятия}}
\EN{\section{Some basics}}
\DE{\section{Einige Grundlagen}}
\FR{\section{Quelques bases}}
\ES{\section{\ESph{}}}
\ITA{\section{Alcune basi teoriche}}
\PTBR{\section{\PTBRph{}}}
\THA{\section{\THAph{}}}
\PL{\section{\PLph{}}}

% sections:
\EN{\input{patterns/intro_CPU_ISA_EN}}
\ES{\input{patterns/intro_CPU_ISA_ES}}
\ITA{\input{patterns/intro_CPU_ISA_ITA}}
\PTBR{\input{patterns/intro_CPU_ISA_PTBR}}
\RU{\input{patterns/intro_CPU_ISA_RU}}
\DE{\input{patterns/intro_CPU_ISA_DE}}
\FR{\input{patterns/intro_CPU_ISA_FR}}
\PL{\input{patterns/intro_CPU_ISA_PL}}

\EN{\input{patterns/numeral_EN}}
\RU{\input{patterns/numeral_RU}}
\ITA{\input{patterns/numeral_ITA}}
\DE{\input{patterns/numeral_DE}}
\FR{\input{patterns/numeral_FR}}
\PL{\input{patterns/numeral_PL}}

% chapters
\input{patterns/00_empty/main}
\input{patterns/011_ret/main}
\input{patterns/01_helloworld/main}
\input{patterns/015_prolog_epilogue/main}
\input{patterns/02_stack/main}
\input{patterns/03_printf/main}
\input{patterns/04_scanf/main}
\input{patterns/05_passing_arguments/main}
\input{patterns/06_return_results/main}
\input{patterns/061_pointers/main}
\input{patterns/065_GOTO/main}
\input{patterns/07_jcc/main}
\input{patterns/08_switch/main}
\input{patterns/09_loops/main}
\input{patterns/10_strings/main}
\input{patterns/11_arith_optimizations/main}
\input{patterns/12_FPU/main}
\input{patterns/13_arrays/main}
\input{patterns/14_bitfields/main}
\EN{\input{patterns/145_LCG/main_EN}}
\RU{\input{patterns/145_LCG/main_RU}}
\input{patterns/15_structs/main}
\input{patterns/17_unions/main}
\input{patterns/18_pointers_to_functions/main}
\input{patterns/185_64bit_in_32_env/main}

\EN{\input{patterns/19_SIMD/main_EN}}
\RU{\input{patterns/19_SIMD/main_RU}}
\DE{\input{patterns/19_SIMD/main_DE}}

\EN{\input{patterns/20_x64/main_EN}}
\RU{\input{patterns/20_x64/main_RU}}

\EN{\input{patterns/205_floating_SIMD/main_EN}}
\RU{\input{patterns/205_floating_SIMD/main_RU}}
\DE{\input{patterns/205_floating_SIMD/main_DE}}

\EN{\input{patterns/ARM/main_EN}}
\RU{\input{patterns/ARM/main_RU}}
\DE{\input{patterns/ARM/main_DE}}

\input{patterns/MIPS/main}


\EN{\section{Returning Values}
\label{ret_val_func}

Another simple function is the one that simply returns a constant value:

\lstinputlisting[caption=\EN{\CCpp Code},style=customc]{patterns/011_ret/1.c}

Let's compile it.

\subsection{x86}

Here's what both the GCC and MSVC compilers produce (with optimization) on the x86 platform:

\lstinputlisting[caption=\Optimizing GCC/MSVC (\assemblyOutput),style=customasmx86]{patterns/011_ret/1.s}

\myindex{x86!\Instructions!RET}
There are just two instructions: the first places the value 123 into the \EAX register,
which is used by convention for storing the return
value, and the second one is \RET, which returns execution to the \gls{caller}.

The caller will take the result from the \EAX register.

\subsection{ARM}

There are a few differences on the ARM platform:

\lstinputlisting[caption=\OptimizingKeilVI (\ARMMode) ASM Output,style=customasmARM]{patterns/011_ret/1_Keil_ARM_O3.s}

ARM uses the register \Reg{0} for returning the results of functions, so 123 is copied into \Reg{0}.

\myindex{ARM!\Instructions!MOV}
\myindex{x86!\Instructions!MOV}
It is worth noting that \MOV is a misleading name for the instruction in both the x86 and ARM \ac{ISA}s.

The data is not in fact \IT{moved}, but \IT{copied}.

\subsection{MIPS}

\label{MIPS_leaf_function_ex1}

The GCC assembly output below lists registers by number:

\lstinputlisting[caption=\Optimizing GCC 4.4.5 (\assemblyOutput),style=customasmMIPS]{patterns/011_ret/MIPS.s}

\dots while \IDA does it by their pseudo names:

\lstinputlisting[caption=\Optimizing GCC 4.4.5 (IDA),style=customasmMIPS]{patterns/011_ret/MIPS_IDA.lst}

The \$2 (or \$V0) register is used to store the function's return value.
\myindex{MIPS!\Pseudoinstructions!LI}
\INS{LI} stands for ``Load Immediate'' and is the MIPS equivalent to \MOV.

\myindex{MIPS!\Instructions!J}
The other instruction is the jump instruction (J or JR) which returns the execution flow to the \gls{caller}.

\myindex{MIPS!Branch delay slot}
You might be wondering why the positions of the load instruction (LI) and the jump instruction (J or JR) are swapped. This is due to a \ac{RISC} feature called ``branch delay slot''.

The reason this happens is a quirk in the architecture of some RISC \ac{ISA}s and isn't important for our
purposes---we must simply keep in mind that in MIPS, the instruction following a jump or branch instruction
is executed \IT{before} the jump/branch instruction itself.

As a consequence, branch instructions always swap places with the instruction executed immediately beforehand.


In practice, functions which merely return 1 (\IT{true}) or 0 (\IT{false}) are very frequent.

The smallest ever of the standard UNIX utilities, \IT{/bin/true} and \IT{/bin/false} return 0 and 1 respectively, as an exit code.
(Zero as an exit code usually means success, non-zero means error.)
}
\RU{\subsubsection{std::string}
\myindex{\Cpp!STL!std::string}
\label{std_string}

\myparagraph{Как устроена структура}

Многие строковые библиотеки \InSqBrackets{\CNotes 2.2} обеспечивают структуру содержащую ссылку 
на буфер собственно со строкой, переменная всегда содержащую длину строки 
(что очень удобно для массы функций \InSqBrackets{\CNotes 2.2.1}) и переменную содержащую текущий размер буфера.

Строка в буфере обыкновенно оканчивается нулем: это для того чтобы указатель на буфер можно было
передавать в функции требующие на вход обычную сишную \ac{ASCIIZ}-строку.

Стандарт \Cpp не описывает, как именно нужно реализовывать std::string,
но, как правило, они реализованы как описано выше, с небольшими дополнениями.

Строки в \Cpp это не класс (как, например, QString в Qt), а темплейт (basic\_string), 
это сделано для того чтобы поддерживать 
строки содержащие разного типа символы: как минимум \Tchar и \IT{wchar\_t}.

Так что, std::string это класс с базовым типом \Tchar.

А std::wstring это класс с базовым типом \IT{wchar\_t}.

\mysubparagraph{MSVC}

В реализации MSVC, вместо ссылки на буфер может содержаться сам буфер (если строка короче 16-и символов).

Это означает, что каждая короткая строка будет занимать в памяти по крайней мере $16 + 4 + 4 = 24$ 
байт для 32-битной среды либо $16 + 8 + 8 = 32$ 
байта в 64-битной, а если строка длиннее 16-и символов, то прибавьте еще длину самой строки.

\lstinputlisting[caption=пример для MSVC,style=customc]{\CURPATH/STL/string/MSVC_RU.cpp}

Собственно, из этого исходника почти всё ясно.

Несколько замечаний:

Если строка короче 16-и символов, 
то отдельный буфер для строки в \glslink{heap}{куче} выделяться не будет.

Это удобно потому что на практике, основная часть строк действительно короткие.
Вероятно, разработчики в Microsoft выбрали размер в 16 символов как разумный баланс.

Теперь очень важный момент в конце функции main(): мы не пользуемся методом c\_str(), тем не менее,
если это скомпилировать и запустить, то обе строки появятся в консоли!

Работает это вот почему.

В первом случае строка короче 16-и символов и в начале объекта std::string (его можно рассматривать
просто как структуру) расположен буфер с этой строкой.
\printf трактует указатель как указатель на массив символов оканчивающийся нулем и поэтому всё работает.

Вывод второй строки (длиннее 16-и символов) даже еще опаснее: это вообще типичная программистская ошибка 
(или опечатка), забыть дописать c\_str().
Это работает потому что в это время в начале структуры расположен указатель на буфер.
Это может надолго остаться незамеченным: до тех пока там не появится строка 
короче 16-и символов, тогда процесс упадет.

\mysubparagraph{GCC}

В реализации GCC в структуре есть еще одна переменная --- reference count.

Интересно, что указатель на экземпляр класса std::string в GCC указывает не на начало самой структуры, 
а на указатель на буфера.
В libstdc++-v3\textbackslash{}include\textbackslash{}bits\textbackslash{}basic\_string.h 
мы можем прочитать что это сделано для удобства отладки:

\begin{lstlisting}
   *  The reason you want _M_data pointing to the character %array and
   *  not the _Rep is so that the debugger can see the string
   *  contents. (Probably we should add a non-inline member to get
   *  the _Rep for the debugger to use, so users can check the actual
   *  string length.)
\end{lstlisting}

\href{http://go.yurichev.com/17085}{исходный код basic\_string.h}

В нашем примере мы учитываем это:

\lstinputlisting[caption=пример для GCC,style=customc]{\CURPATH/STL/string/GCC_RU.cpp}

Нужны еще небольшие хаки чтобы сымитировать типичную ошибку, которую мы уже видели выше, из-за
более ужесточенной проверки типов в GCC, тем не менее, printf() работает и здесь без c\_str().

\myparagraph{Чуть более сложный пример}

\lstinputlisting[style=customc]{\CURPATH/STL/string/3.cpp}

\lstinputlisting[caption=MSVC 2012,style=customasmx86]{\CURPATH/STL/string/3_MSVC_RU.asm}

Собственно, компилятор не конструирует строки статически: да в общем-то и как
это возможно, если буфер с ней нужно хранить в \glslink{heap}{куче}?

Вместо этого в сегменте данных хранятся обычные \ac{ASCIIZ}-строки, а позже, во время выполнения, 
при помощи метода \q{assign}, конструируются строки s1 и s2
.
При помощи \TT{operator+}, создается строка s3.

Обратите внимание на то что вызов метода c\_str() отсутствует,
потому что его код достаточно короткий и компилятор вставил его прямо здесь:
если строка короче 16-и байт, то в регистре EAX остается указатель на буфер,
а если длиннее, то из этого же места достается адрес на буфер расположенный в \glslink{heap}{куче}.

Далее следуют вызовы трех деструкторов, причем, они вызываются только если строка длиннее 16-и байт:
тогда нужно освободить буфера в \glslink{heap}{куче}.
В противном случае, так как все три объекта std::string хранятся в стеке,
они освобождаются автоматически после выхода из функции.

Следовательно, работа с короткими строками более быстрая из-за м\'{е}ньшего обращения к \glslink{heap}{куче}.

Код на GCC даже проще (из-за того, что в GCC, как мы уже видели, не реализована возможность хранить короткую
строку прямо в структуре):

% TODO1 comment each function meaning
\lstinputlisting[caption=GCC 4.8.1,style=customasmx86]{\CURPATH/STL/string/3_GCC_RU.s}

Можно заметить, что в деструкторы передается не указатель на объект,
а указатель на место за 12 байт (или 3 слова) перед ним, то есть, на настоящее начало структуры.

\myparagraph{std::string как глобальная переменная}
\label{sec:std_string_as_global_variable}

Опытные программисты на \Cpp знают, что глобальные переменные \ac{STL}-типов вполне можно объявлять.

Да, действительно:

\lstinputlisting[style=customc]{\CURPATH/STL/string/5.cpp}

Но как и где будет вызываться конструктор \TT{std::string}?

На самом деле, эта переменная будет инициализирована даже перед началом \main.

\lstinputlisting[caption=MSVC 2012: здесь конструируется глобальная переменная{,} а также регистрируется её деструктор,style=customasmx86]{\CURPATH/STL/string/5_MSVC_p2.asm}

\lstinputlisting[caption=MSVC 2012: здесь глобальная переменная используется в \main,style=customasmx86]{\CURPATH/STL/string/5_MSVC_p1.asm}

\lstinputlisting[caption=MSVC 2012: эта функция-деструктор вызывается перед выходом,style=customasmx86]{\CURPATH/STL/string/5_MSVC_p3.asm}

\myindex{\CStandardLibrary!atexit()}
В реальности, из \ac{CRT}, еще до вызова main(), вызывается специальная функция,
в которой перечислены все конструкторы подобных переменных.
Более того: при помощи atexit() регистрируется функция, которая будет вызвана в конце работы программы:
в этой функции компилятор собирает вызовы деструкторов всех подобных глобальных переменных.

GCC работает похожим образом:

\lstinputlisting[caption=GCC 4.8.1,style=customasmx86]{\CURPATH/STL/string/5_GCC.s}

Но он не выделяет отдельной функции в которой будут собраны деструкторы: 
каждый деструктор передается в atexit() по одному.

% TODO а если глобальная STL-переменная в другом модуле? надо проверить.

}
\DE{\subsection{Einfachste XOR-Verschlüsselung überhaupt}

Ich habe einmal eine Software gesehen, bei der alle Debugging-Ausgaben mit XOR mit dem Wert 3
verschlüsselt wurden. Mit anderen Worten, die beiden niedrigsten Bits aller Buchstaben wurden invertiert.

``Hello, world'' wurde zu ``Kfool/\#tlqog'':

\begin{lstlisting}
#!/usr/bin/python

msg="Hello, world!"

print "".join(map(lambda x: chr(ord(x)^3), msg))
\end{lstlisting}

Das ist eine ziemlich interessante Verschlüsselung (oder besser eine Verschleierung),
weil sie zwei wichtige Eigenschaften hat:
1) es ist eine einzige Funktion zum Verschlüsseln und entschlüsseln, sie muss nur wiederholt angewendet werden
2) die entstehenden Buchstaben befinden sich im druckbaren Bereich, also die ganze Zeichenkette kann ohne
Escape-Symbole im Code verwendet werden.

Die zweite Eigenschaft nutzt die Tatsache, dass alle druckbaren Zeichen in Reihen organisiert sind: 0x2x-0x7x,
und wenn die beiden niederwertigsten Bits invertiert werden, wird der Buchstabe um eine oder drei Stellen nach
links oder rechts \IT{verschoben}, aber niemals in eine andere Reihe:

\begin{figure}[H]
\centering
\includegraphics[width=0.7\textwidth]{ascii_clean.png}
\caption{7-Bit \ac{ASCII} Tabelle in Emacs}
\end{figure}

\dots mit dem Zeichen 0x7F als einziger Ausnahme.

Im Folgenden werden also beispielsweise die Zeichen A-Z \IT{verschlüsselt}:

\begin{lstlisting}
#!/usr/bin/python

msg="@ABCDEFGHIJKLMNO"

print "".join(map(lambda x: chr(ord(x)^3), msg))
\end{lstlisting}

Ergebnis:
% FIXME \verb  --  relevant comment for German?
\begin{lstlisting}
CBA@GFEDKJIHONML
\end{lstlisting}

Es sieht so aus als würden die Zeichen ``@'' und ``C'' sowie ``B'' und ``A'' vertauscht werden.

Hier ist noch ein interessantes Beispiel, in dem gezeigt wird, wie die Eigenschaften von XOR
ausgenutzt werden können: Exakt den gleichen Effekt, dass druckbare Zeichen auch druckbar bleiben,
kann man dadurch erzielen, dass irgendeine Kombination der niedrigsten vier Bits invertiert wird.
}

\EN{\section{Returning Values}
\label{ret_val_func}

Another simple function is the one that simply returns a constant value:

\lstinputlisting[caption=\EN{\CCpp Code},style=customc]{patterns/011_ret/1.c}

Let's compile it.

\subsection{x86}

Here's what both the GCC and MSVC compilers produce (with optimization) on the x86 platform:

\lstinputlisting[caption=\Optimizing GCC/MSVC (\assemblyOutput),style=customasmx86]{patterns/011_ret/1.s}

\myindex{x86!\Instructions!RET}
There are just two instructions: the first places the value 123 into the \EAX register,
which is used by convention for storing the return
value, and the second one is \RET, which returns execution to the \gls{caller}.

The caller will take the result from the \EAX register.

\subsection{ARM}

There are a few differences on the ARM platform:

\lstinputlisting[caption=\OptimizingKeilVI (\ARMMode) ASM Output,style=customasmARM]{patterns/011_ret/1_Keil_ARM_O3.s}

ARM uses the register \Reg{0} for returning the results of functions, so 123 is copied into \Reg{0}.

\myindex{ARM!\Instructions!MOV}
\myindex{x86!\Instructions!MOV}
It is worth noting that \MOV is a misleading name for the instruction in both the x86 and ARM \ac{ISA}s.

The data is not in fact \IT{moved}, but \IT{copied}.

\subsection{MIPS}

\label{MIPS_leaf_function_ex1}

The GCC assembly output below lists registers by number:

\lstinputlisting[caption=\Optimizing GCC 4.4.5 (\assemblyOutput),style=customasmMIPS]{patterns/011_ret/MIPS.s}

\dots while \IDA does it by their pseudo names:

\lstinputlisting[caption=\Optimizing GCC 4.4.5 (IDA),style=customasmMIPS]{patterns/011_ret/MIPS_IDA.lst}

The \$2 (or \$V0) register is used to store the function's return value.
\myindex{MIPS!\Pseudoinstructions!LI}
\INS{LI} stands for ``Load Immediate'' and is the MIPS equivalent to \MOV.

\myindex{MIPS!\Instructions!J}
The other instruction is the jump instruction (J or JR) which returns the execution flow to the \gls{caller}.

\myindex{MIPS!Branch delay slot}
You might be wondering why the positions of the load instruction (LI) and the jump instruction (J or JR) are swapped. This is due to a \ac{RISC} feature called ``branch delay slot''.

The reason this happens is a quirk in the architecture of some RISC \ac{ISA}s and isn't important for our
purposes---we must simply keep in mind that in MIPS, the instruction following a jump or branch instruction
is executed \IT{before} the jump/branch instruction itself.

As a consequence, branch instructions always swap places with the instruction executed immediately beforehand.


In practice, functions which merely return 1 (\IT{true}) or 0 (\IT{false}) are very frequent.

The smallest ever of the standard UNIX utilities, \IT{/bin/true} and \IT{/bin/false} return 0 and 1 respectively, as an exit code.
(Zero as an exit code usually means success, non-zero means error.)
}
\RU{\subsubsection{std::string}
\myindex{\Cpp!STL!std::string}
\label{std_string}

\myparagraph{Как устроена структура}

Многие строковые библиотеки \InSqBrackets{\CNotes 2.2} обеспечивают структуру содержащую ссылку 
на буфер собственно со строкой, переменная всегда содержащую длину строки 
(что очень удобно для массы функций \InSqBrackets{\CNotes 2.2.1}) и переменную содержащую текущий размер буфера.

Строка в буфере обыкновенно оканчивается нулем: это для того чтобы указатель на буфер можно было
передавать в функции требующие на вход обычную сишную \ac{ASCIIZ}-строку.

Стандарт \Cpp не описывает, как именно нужно реализовывать std::string,
но, как правило, они реализованы как описано выше, с небольшими дополнениями.

Строки в \Cpp это не класс (как, например, QString в Qt), а темплейт (basic\_string), 
это сделано для того чтобы поддерживать 
строки содержащие разного типа символы: как минимум \Tchar и \IT{wchar\_t}.

Так что, std::string это класс с базовым типом \Tchar.

А std::wstring это класс с базовым типом \IT{wchar\_t}.

\mysubparagraph{MSVC}

В реализации MSVC, вместо ссылки на буфер может содержаться сам буфер (если строка короче 16-и символов).

Это означает, что каждая короткая строка будет занимать в памяти по крайней мере $16 + 4 + 4 = 24$ 
байт для 32-битной среды либо $16 + 8 + 8 = 32$ 
байта в 64-битной, а если строка длиннее 16-и символов, то прибавьте еще длину самой строки.

\lstinputlisting[caption=пример для MSVC,style=customc]{\CURPATH/STL/string/MSVC_RU.cpp}

Собственно, из этого исходника почти всё ясно.

Несколько замечаний:

Если строка короче 16-и символов, 
то отдельный буфер для строки в \glslink{heap}{куче} выделяться не будет.

Это удобно потому что на практике, основная часть строк действительно короткие.
Вероятно, разработчики в Microsoft выбрали размер в 16 символов как разумный баланс.

Теперь очень важный момент в конце функции main(): мы не пользуемся методом c\_str(), тем не менее,
если это скомпилировать и запустить, то обе строки появятся в консоли!

Работает это вот почему.

В первом случае строка короче 16-и символов и в начале объекта std::string (его можно рассматривать
просто как структуру) расположен буфер с этой строкой.
\printf трактует указатель как указатель на массив символов оканчивающийся нулем и поэтому всё работает.

Вывод второй строки (длиннее 16-и символов) даже еще опаснее: это вообще типичная программистская ошибка 
(или опечатка), забыть дописать c\_str().
Это работает потому что в это время в начале структуры расположен указатель на буфер.
Это может надолго остаться незамеченным: до тех пока там не появится строка 
короче 16-и символов, тогда процесс упадет.

\mysubparagraph{GCC}

В реализации GCC в структуре есть еще одна переменная --- reference count.

Интересно, что указатель на экземпляр класса std::string в GCC указывает не на начало самой структуры, 
а на указатель на буфера.
В libstdc++-v3\textbackslash{}include\textbackslash{}bits\textbackslash{}basic\_string.h 
мы можем прочитать что это сделано для удобства отладки:

\begin{lstlisting}
   *  The reason you want _M_data pointing to the character %array and
   *  not the _Rep is so that the debugger can see the string
   *  contents. (Probably we should add a non-inline member to get
   *  the _Rep for the debugger to use, so users can check the actual
   *  string length.)
\end{lstlisting}

\href{http://go.yurichev.com/17085}{исходный код basic\_string.h}

В нашем примере мы учитываем это:

\lstinputlisting[caption=пример для GCC,style=customc]{\CURPATH/STL/string/GCC_RU.cpp}

Нужны еще небольшие хаки чтобы сымитировать типичную ошибку, которую мы уже видели выше, из-за
более ужесточенной проверки типов в GCC, тем не менее, printf() работает и здесь без c\_str().

\myparagraph{Чуть более сложный пример}

\lstinputlisting[style=customc]{\CURPATH/STL/string/3.cpp}

\lstinputlisting[caption=MSVC 2012,style=customasmx86]{\CURPATH/STL/string/3_MSVC_RU.asm}

Собственно, компилятор не конструирует строки статически: да в общем-то и как
это возможно, если буфер с ней нужно хранить в \glslink{heap}{куче}?

Вместо этого в сегменте данных хранятся обычные \ac{ASCIIZ}-строки, а позже, во время выполнения, 
при помощи метода \q{assign}, конструируются строки s1 и s2
.
При помощи \TT{operator+}, создается строка s3.

Обратите внимание на то что вызов метода c\_str() отсутствует,
потому что его код достаточно короткий и компилятор вставил его прямо здесь:
если строка короче 16-и байт, то в регистре EAX остается указатель на буфер,
а если длиннее, то из этого же места достается адрес на буфер расположенный в \glslink{heap}{куче}.

Далее следуют вызовы трех деструкторов, причем, они вызываются только если строка длиннее 16-и байт:
тогда нужно освободить буфера в \glslink{heap}{куче}.
В противном случае, так как все три объекта std::string хранятся в стеке,
они освобождаются автоматически после выхода из функции.

Следовательно, работа с короткими строками более быстрая из-за м\'{е}ньшего обращения к \glslink{heap}{куче}.

Код на GCC даже проще (из-за того, что в GCC, как мы уже видели, не реализована возможность хранить короткую
строку прямо в структуре):

% TODO1 comment each function meaning
\lstinputlisting[caption=GCC 4.8.1,style=customasmx86]{\CURPATH/STL/string/3_GCC_RU.s}

Можно заметить, что в деструкторы передается не указатель на объект,
а указатель на место за 12 байт (или 3 слова) перед ним, то есть, на настоящее начало структуры.

\myparagraph{std::string как глобальная переменная}
\label{sec:std_string_as_global_variable}

Опытные программисты на \Cpp знают, что глобальные переменные \ac{STL}-типов вполне можно объявлять.

Да, действительно:

\lstinputlisting[style=customc]{\CURPATH/STL/string/5.cpp}

Но как и где будет вызываться конструктор \TT{std::string}?

На самом деле, эта переменная будет инициализирована даже перед началом \main.

\lstinputlisting[caption=MSVC 2012: здесь конструируется глобальная переменная{,} а также регистрируется её деструктор,style=customasmx86]{\CURPATH/STL/string/5_MSVC_p2.asm}

\lstinputlisting[caption=MSVC 2012: здесь глобальная переменная используется в \main,style=customasmx86]{\CURPATH/STL/string/5_MSVC_p1.asm}

\lstinputlisting[caption=MSVC 2012: эта функция-деструктор вызывается перед выходом,style=customasmx86]{\CURPATH/STL/string/5_MSVC_p3.asm}

\myindex{\CStandardLibrary!atexit()}
В реальности, из \ac{CRT}, еще до вызова main(), вызывается специальная функция,
в которой перечислены все конструкторы подобных переменных.
Более того: при помощи atexit() регистрируется функция, которая будет вызвана в конце работы программы:
в этой функции компилятор собирает вызовы деструкторов всех подобных глобальных переменных.

GCC работает похожим образом:

\lstinputlisting[caption=GCC 4.8.1,style=customasmx86]{\CURPATH/STL/string/5_GCC.s}

Но он не выделяет отдельной функции в которой будут собраны деструкторы: 
каждый деструктор передается в atexit() по одному.

% TODO а если глобальная STL-переменная в другом модуле? надо проверить.

}

\EN{\section{Returning Values}
\label{ret_val_func}

Another simple function is the one that simply returns a constant value:

\lstinputlisting[caption=\EN{\CCpp Code},style=customc]{patterns/011_ret/1.c}

Let's compile it.

\subsection{x86}

Here's what both the GCC and MSVC compilers produce (with optimization) on the x86 platform:

\lstinputlisting[caption=\Optimizing GCC/MSVC (\assemblyOutput),style=customasmx86]{patterns/011_ret/1.s}

\myindex{x86!\Instructions!RET}
There are just two instructions: the first places the value 123 into the \EAX register,
which is used by convention for storing the return
value, and the second one is \RET, which returns execution to the \gls{caller}.

The caller will take the result from the \EAX register.

\subsection{ARM}

There are a few differences on the ARM platform:

\lstinputlisting[caption=\OptimizingKeilVI (\ARMMode) ASM Output,style=customasmARM]{patterns/011_ret/1_Keil_ARM_O3.s}

ARM uses the register \Reg{0} for returning the results of functions, so 123 is copied into \Reg{0}.

\myindex{ARM!\Instructions!MOV}
\myindex{x86!\Instructions!MOV}
It is worth noting that \MOV is a misleading name for the instruction in both the x86 and ARM \ac{ISA}s.

The data is not in fact \IT{moved}, but \IT{copied}.

\subsection{MIPS}

\label{MIPS_leaf_function_ex1}

The GCC assembly output below lists registers by number:

\lstinputlisting[caption=\Optimizing GCC 4.4.5 (\assemblyOutput),style=customasmMIPS]{patterns/011_ret/MIPS.s}

\dots while \IDA does it by their pseudo names:

\lstinputlisting[caption=\Optimizing GCC 4.4.5 (IDA),style=customasmMIPS]{patterns/011_ret/MIPS_IDA.lst}

The \$2 (or \$V0) register is used to store the function's return value.
\myindex{MIPS!\Pseudoinstructions!LI}
\INS{LI} stands for ``Load Immediate'' and is the MIPS equivalent to \MOV.

\myindex{MIPS!\Instructions!J}
The other instruction is the jump instruction (J or JR) which returns the execution flow to the \gls{caller}.

\myindex{MIPS!Branch delay slot}
You might be wondering why the positions of the load instruction (LI) and the jump instruction (J or JR) are swapped. This is due to a \ac{RISC} feature called ``branch delay slot''.

The reason this happens is a quirk in the architecture of some RISC \ac{ISA}s and isn't important for our
purposes---we must simply keep in mind that in MIPS, the instruction following a jump or branch instruction
is executed \IT{before} the jump/branch instruction itself.

As a consequence, branch instructions always swap places with the instruction executed immediately beforehand.


In practice, functions which merely return 1 (\IT{true}) or 0 (\IT{false}) are very frequent.

The smallest ever of the standard UNIX utilities, \IT{/bin/true} and \IT{/bin/false} return 0 and 1 respectively, as an exit code.
(Zero as an exit code usually means success, non-zero means error.)
}
\RU{\subsubsection{std::string}
\myindex{\Cpp!STL!std::string}
\label{std_string}

\myparagraph{Как устроена структура}

Многие строковые библиотеки \InSqBrackets{\CNotes 2.2} обеспечивают структуру содержащую ссылку 
на буфер собственно со строкой, переменная всегда содержащую длину строки 
(что очень удобно для массы функций \InSqBrackets{\CNotes 2.2.1}) и переменную содержащую текущий размер буфера.

Строка в буфере обыкновенно оканчивается нулем: это для того чтобы указатель на буфер можно было
передавать в функции требующие на вход обычную сишную \ac{ASCIIZ}-строку.

Стандарт \Cpp не описывает, как именно нужно реализовывать std::string,
но, как правило, они реализованы как описано выше, с небольшими дополнениями.

Строки в \Cpp это не класс (как, например, QString в Qt), а темплейт (basic\_string), 
это сделано для того чтобы поддерживать 
строки содержащие разного типа символы: как минимум \Tchar и \IT{wchar\_t}.

Так что, std::string это класс с базовым типом \Tchar.

А std::wstring это класс с базовым типом \IT{wchar\_t}.

\mysubparagraph{MSVC}

В реализации MSVC, вместо ссылки на буфер может содержаться сам буфер (если строка короче 16-и символов).

Это означает, что каждая короткая строка будет занимать в памяти по крайней мере $16 + 4 + 4 = 24$ 
байт для 32-битной среды либо $16 + 8 + 8 = 32$ 
байта в 64-битной, а если строка длиннее 16-и символов, то прибавьте еще длину самой строки.

\lstinputlisting[caption=пример для MSVC,style=customc]{\CURPATH/STL/string/MSVC_RU.cpp}

Собственно, из этого исходника почти всё ясно.

Несколько замечаний:

Если строка короче 16-и символов, 
то отдельный буфер для строки в \glslink{heap}{куче} выделяться не будет.

Это удобно потому что на практике, основная часть строк действительно короткие.
Вероятно, разработчики в Microsoft выбрали размер в 16 символов как разумный баланс.

Теперь очень важный момент в конце функции main(): мы не пользуемся методом c\_str(), тем не менее,
если это скомпилировать и запустить, то обе строки появятся в консоли!

Работает это вот почему.

В первом случае строка короче 16-и символов и в начале объекта std::string (его можно рассматривать
просто как структуру) расположен буфер с этой строкой.
\printf трактует указатель как указатель на массив символов оканчивающийся нулем и поэтому всё работает.

Вывод второй строки (длиннее 16-и символов) даже еще опаснее: это вообще типичная программистская ошибка 
(или опечатка), забыть дописать c\_str().
Это работает потому что в это время в начале структуры расположен указатель на буфер.
Это может надолго остаться незамеченным: до тех пока там не появится строка 
короче 16-и символов, тогда процесс упадет.

\mysubparagraph{GCC}

В реализации GCC в структуре есть еще одна переменная --- reference count.

Интересно, что указатель на экземпляр класса std::string в GCC указывает не на начало самой структуры, 
а на указатель на буфера.
В libstdc++-v3\textbackslash{}include\textbackslash{}bits\textbackslash{}basic\_string.h 
мы можем прочитать что это сделано для удобства отладки:

\begin{lstlisting}
   *  The reason you want _M_data pointing to the character %array and
   *  not the _Rep is so that the debugger can see the string
   *  contents. (Probably we should add a non-inline member to get
   *  the _Rep for the debugger to use, so users can check the actual
   *  string length.)
\end{lstlisting}

\href{http://go.yurichev.com/17085}{исходный код basic\_string.h}

В нашем примере мы учитываем это:

\lstinputlisting[caption=пример для GCC,style=customc]{\CURPATH/STL/string/GCC_RU.cpp}

Нужны еще небольшие хаки чтобы сымитировать типичную ошибку, которую мы уже видели выше, из-за
более ужесточенной проверки типов в GCC, тем не менее, printf() работает и здесь без c\_str().

\myparagraph{Чуть более сложный пример}

\lstinputlisting[style=customc]{\CURPATH/STL/string/3.cpp}

\lstinputlisting[caption=MSVC 2012,style=customasmx86]{\CURPATH/STL/string/3_MSVC_RU.asm}

Собственно, компилятор не конструирует строки статически: да в общем-то и как
это возможно, если буфер с ней нужно хранить в \glslink{heap}{куче}?

Вместо этого в сегменте данных хранятся обычные \ac{ASCIIZ}-строки, а позже, во время выполнения, 
при помощи метода \q{assign}, конструируются строки s1 и s2
.
При помощи \TT{operator+}, создается строка s3.

Обратите внимание на то что вызов метода c\_str() отсутствует,
потому что его код достаточно короткий и компилятор вставил его прямо здесь:
если строка короче 16-и байт, то в регистре EAX остается указатель на буфер,
а если длиннее, то из этого же места достается адрес на буфер расположенный в \glslink{heap}{куче}.

Далее следуют вызовы трех деструкторов, причем, они вызываются только если строка длиннее 16-и байт:
тогда нужно освободить буфера в \glslink{heap}{куче}.
В противном случае, так как все три объекта std::string хранятся в стеке,
они освобождаются автоматически после выхода из функции.

Следовательно, работа с короткими строками более быстрая из-за м\'{е}ньшего обращения к \glslink{heap}{куче}.

Код на GCC даже проще (из-за того, что в GCC, как мы уже видели, не реализована возможность хранить короткую
строку прямо в структуре):

% TODO1 comment each function meaning
\lstinputlisting[caption=GCC 4.8.1,style=customasmx86]{\CURPATH/STL/string/3_GCC_RU.s}

Можно заметить, что в деструкторы передается не указатель на объект,
а указатель на место за 12 байт (или 3 слова) перед ним, то есть, на настоящее начало структуры.

\myparagraph{std::string как глобальная переменная}
\label{sec:std_string_as_global_variable}

Опытные программисты на \Cpp знают, что глобальные переменные \ac{STL}-типов вполне можно объявлять.

Да, действительно:

\lstinputlisting[style=customc]{\CURPATH/STL/string/5.cpp}

Но как и где будет вызываться конструктор \TT{std::string}?

На самом деле, эта переменная будет инициализирована даже перед началом \main.

\lstinputlisting[caption=MSVC 2012: здесь конструируется глобальная переменная{,} а также регистрируется её деструктор,style=customasmx86]{\CURPATH/STL/string/5_MSVC_p2.asm}

\lstinputlisting[caption=MSVC 2012: здесь глобальная переменная используется в \main,style=customasmx86]{\CURPATH/STL/string/5_MSVC_p1.asm}

\lstinputlisting[caption=MSVC 2012: эта функция-деструктор вызывается перед выходом,style=customasmx86]{\CURPATH/STL/string/5_MSVC_p3.asm}

\myindex{\CStandardLibrary!atexit()}
В реальности, из \ac{CRT}, еще до вызова main(), вызывается специальная функция,
в которой перечислены все конструкторы подобных переменных.
Более того: при помощи atexit() регистрируется функция, которая будет вызвана в конце работы программы:
в этой функции компилятор собирает вызовы деструкторов всех подобных глобальных переменных.

GCC работает похожим образом:

\lstinputlisting[caption=GCC 4.8.1,style=customasmx86]{\CURPATH/STL/string/5_GCC.s}

Но он не выделяет отдельной функции в которой будут собраны деструкторы: 
каждый деструктор передается в atexit() по одному.

% TODO а если глобальная STL-переменная в другом модуле? надо проверить.

}
\DE{\subsection{Einfachste XOR-Verschlüsselung überhaupt}

Ich habe einmal eine Software gesehen, bei der alle Debugging-Ausgaben mit XOR mit dem Wert 3
verschlüsselt wurden. Mit anderen Worten, die beiden niedrigsten Bits aller Buchstaben wurden invertiert.

``Hello, world'' wurde zu ``Kfool/\#tlqog'':

\begin{lstlisting}
#!/usr/bin/python

msg="Hello, world!"

print "".join(map(lambda x: chr(ord(x)^3), msg))
\end{lstlisting}

Das ist eine ziemlich interessante Verschlüsselung (oder besser eine Verschleierung),
weil sie zwei wichtige Eigenschaften hat:
1) es ist eine einzige Funktion zum Verschlüsseln und entschlüsseln, sie muss nur wiederholt angewendet werden
2) die entstehenden Buchstaben befinden sich im druckbaren Bereich, also die ganze Zeichenkette kann ohne
Escape-Symbole im Code verwendet werden.

Die zweite Eigenschaft nutzt die Tatsache, dass alle druckbaren Zeichen in Reihen organisiert sind: 0x2x-0x7x,
und wenn die beiden niederwertigsten Bits invertiert werden, wird der Buchstabe um eine oder drei Stellen nach
links oder rechts \IT{verschoben}, aber niemals in eine andere Reihe:

\begin{figure}[H]
\centering
\includegraphics[width=0.7\textwidth]{ascii_clean.png}
\caption{7-Bit \ac{ASCII} Tabelle in Emacs}
\end{figure}

\dots mit dem Zeichen 0x7F als einziger Ausnahme.

Im Folgenden werden also beispielsweise die Zeichen A-Z \IT{verschlüsselt}:

\begin{lstlisting}
#!/usr/bin/python

msg="@ABCDEFGHIJKLMNO"

print "".join(map(lambda x: chr(ord(x)^3), msg))
\end{lstlisting}

Ergebnis:
% FIXME \verb  --  relevant comment for German?
\begin{lstlisting}
CBA@GFEDKJIHONML
\end{lstlisting}

Es sieht so aus als würden die Zeichen ``@'' und ``C'' sowie ``B'' und ``A'' vertauscht werden.

Hier ist noch ein interessantes Beispiel, in dem gezeigt wird, wie die Eigenschaften von XOR
ausgenutzt werden können: Exakt den gleichen Effekt, dass druckbare Zeichen auch druckbar bleiben,
kann man dadurch erzielen, dass irgendeine Kombination der niedrigsten vier Bits invertiert wird.
}

\EN{\section{Returning Values}
\label{ret_val_func}

Another simple function is the one that simply returns a constant value:

\lstinputlisting[caption=\EN{\CCpp Code},style=customc]{patterns/011_ret/1.c}

Let's compile it.

\subsection{x86}

Here's what both the GCC and MSVC compilers produce (with optimization) on the x86 platform:

\lstinputlisting[caption=\Optimizing GCC/MSVC (\assemblyOutput),style=customasmx86]{patterns/011_ret/1.s}

\myindex{x86!\Instructions!RET}
There are just two instructions: the first places the value 123 into the \EAX register,
which is used by convention for storing the return
value, and the second one is \RET, which returns execution to the \gls{caller}.

The caller will take the result from the \EAX register.

\subsection{ARM}

There are a few differences on the ARM platform:

\lstinputlisting[caption=\OptimizingKeilVI (\ARMMode) ASM Output,style=customasmARM]{patterns/011_ret/1_Keil_ARM_O3.s}

ARM uses the register \Reg{0} for returning the results of functions, so 123 is copied into \Reg{0}.

\myindex{ARM!\Instructions!MOV}
\myindex{x86!\Instructions!MOV}
It is worth noting that \MOV is a misleading name for the instruction in both the x86 and ARM \ac{ISA}s.

The data is not in fact \IT{moved}, but \IT{copied}.

\subsection{MIPS}

\label{MIPS_leaf_function_ex1}

The GCC assembly output below lists registers by number:

\lstinputlisting[caption=\Optimizing GCC 4.4.5 (\assemblyOutput),style=customasmMIPS]{patterns/011_ret/MIPS.s}

\dots while \IDA does it by their pseudo names:

\lstinputlisting[caption=\Optimizing GCC 4.4.5 (IDA),style=customasmMIPS]{patterns/011_ret/MIPS_IDA.lst}

The \$2 (or \$V0) register is used to store the function's return value.
\myindex{MIPS!\Pseudoinstructions!LI}
\INS{LI} stands for ``Load Immediate'' and is the MIPS equivalent to \MOV.

\myindex{MIPS!\Instructions!J}
The other instruction is the jump instruction (J or JR) which returns the execution flow to the \gls{caller}.

\myindex{MIPS!Branch delay slot}
You might be wondering why the positions of the load instruction (LI) and the jump instruction (J or JR) are swapped. This is due to a \ac{RISC} feature called ``branch delay slot''.

The reason this happens is a quirk in the architecture of some RISC \ac{ISA}s and isn't important for our
purposes---we must simply keep in mind that in MIPS, the instruction following a jump or branch instruction
is executed \IT{before} the jump/branch instruction itself.

As a consequence, branch instructions always swap places with the instruction executed immediately beforehand.


In practice, functions which merely return 1 (\IT{true}) or 0 (\IT{false}) are very frequent.

The smallest ever of the standard UNIX utilities, \IT{/bin/true} and \IT{/bin/false} return 0 and 1 respectively, as an exit code.
(Zero as an exit code usually means success, non-zero means error.)
}
\RU{\subsubsection{std::string}
\myindex{\Cpp!STL!std::string}
\label{std_string}

\myparagraph{Как устроена структура}

Многие строковые библиотеки \InSqBrackets{\CNotes 2.2} обеспечивают структуру содержащую ссылку 
на буфер собственно со строкой, переменная всегда содержащую длину строки 
(что очень удобно для массы функций \InSqBrackets{\CNotes 2.2.1}) и переменную содержащую текущий размер буфера.

Строка в буфере обыкновенно оканчивается нулем: это для того чтобы указатель на буфер можно было
передавать в функции требующие на вход обычную сишную \ac{ASCIIZ}-строку.

Стандарт \Cpp не описывает, как именно нужно реализовывать std::string,
но, как правило, они реализованы как описано выше, с небольшими дополнениями.

Строки в \Cpp это не класс (как, например, QString в Qt), а темплейт (basic\_string), 
это сделано для того чтобы поддерживать 
строки содержащие разного типа символы: как минимум \Tchar и \IT{wchar\_t}.

Так что, std::string это класс с базовым типом \Tchar.

А std::wstring это класс с базовым типом \IT{wchar\_t}.

\mysubparagraph{MSVC}

В реализации MSVC, вместо ссылки на буфер может содержаться сам буфер (если строка короче 16-и символов).

Это означает, что каждая короткая строка будет занимать в памяти по крайней мере $16 + 4 + 4 = 24$ 
байт для 32-битной среды либо $16 + 8 + 8 = 32$ 
байта в 64-битной, а если строка длиннее 16-и символов, то прибавьте еще длину самой строки.

\lstinputlisting[caption=пример для MSVC,style=customc]{\CURPATH/STL/string/MSVC_RU.cpp}

Собственно, из этого исходника почти всё ясно.

Несколько замечаний:

Если строка короче 16-и символов, 
то отдельный буфер для строки в \glslink{heap}{куче} выделяться не будет.

Это удобно потому что на практике, основная часть строк действительно короткие.
Вероятно, разработчики в Microsoft выбрали размер в 16 символов как разумный баланс.

Теперь очень важный момент в конце функции main(): мы не пользуемся методом c\_str(), тем не менее,
если это скомпилировать и запустить, то обе строки появятся в консоли!

Работает это вот почему.

В первом случае строка короче 16-и символов и в начале объекта std::string (его можно рассматривать
просто как структуру) расположен буфер с этой строкой.
\printf трактует указатель как указатель на массив символов оканчивающийся нулем и поэтому всё работает.

Вывод второй строки (длиннее 16-и символов) даже еще опаснее: это вообще типичная программистская ошибка 
(или опечатка), забыть дописать c\_str().
Это работает потому что в это время в начале структуры расположен указатель на буфер.
Это может надолго остаться незамеченным: до тех пока там не появится строка 
короче 16-и символов, тогда процесс упадет.

\mysubparagraph{GCC}

В реализации GCC в структуре есть еще одна переменная --- reference count.

Интересно, что указатель на экземпляр класса std::string в GCC указывает не на начало самой структуры, 
а на указатель на буфера.
В libstdc++-v3\textbackslash{}include\textbackslash{}bits\textbackslash{}basic\_string.h 
мы можем прочитать что это сделано для удобства отладки:

\begin{lstlisting}
   *  The reason you want _M_data pointing to the character %array and
   *  not the _Rep is so that the debugger can see the string
   *  contents. (Probably we should add a non-inline member to get
   *  the _Rep for the debugger to use, so users can check the actual
   *  string length.)
\end{lstlisting}

\href{http://go.yurichev.com/17085}{исходный код basic\_string.h}

В нашем примере мы учитываем это:

\lstinputlisting[caption=пример для GCC,style=customc]{\CURPATH/STL/string/GCC_RU.cpp}

Нужны еще небольшие хаки чтобы сымитировать типичную ошибку, которую мы уже видели выше, из-за
более ужесточенной проверки типов в GCC, тем не менее, printf() работает и здесь без c\_str().

\myparagraph{Чуть более сложный пример}

\lstinputlisting[style=customc]{\CURPATH/STL/string/3.cpp}

\lstinputlisting[caption=MSVC 2012,style=customasmx86]{\CURPATH/STL/string/3_MSVC_RU.asm}

Собственно, компилятор не конструирует строки статически: да в общем-то и как
это возможно, если буфер с ней нужно хранить в \glslink{heap}{куче}?

Вместо этого в сегменте данных хранятся обычные \ac{ASCIIZ}-строки, а позже, во время выполнения, 
при помощи метода \q{assign}, конструируются строки s1 и s2
.
При помощи \TT{operator+}, создается строка s3.

Обратите внимание на то что вызов метода c\_str() отсутствует,
потому что его код достаточно короткий и компилятор вставил его прямо здесь:
если строка короче 16-и байт, то в регистре EAX остается указатель на буфер,
а если длиннее, то из этого же места достается адрес на буфер расположенный в \glslink{heap}{куче}.

Далее следуют вызовы трех деструкторов, причем, они вызываются только если строка длиннее 16-и байт:
тогда нужно освободить буфера в \glslink{heap}{куче}.
В противном случае, так как все три объекта std::string хранятся в стеке,
они освобождаются автоматически после выхода из функции.

Следовательно, работа с короткими строками более быстрая из-за м\'{е}ньшего обращения к \glslink{heap}{куче}.

Код на GCC даже проще (из-за того, что в GCC, как мы уже видели, не реализована возможность хранить короткую
строку прямо в структуре):

% TODO1 comment each function meaning
\lstinputlisting[caption=GCC 4.8.1,style=customasmx86]{\CURPATH/STL/string/3_GCC_RU.s}

Можно заметить, что в деструкторы передается не указатель на объект,
а указатель на место за 12 байт (или 3 слова) перед ним, то есть, на настоящее начало структуры.

\myparagraph{std::string как глобальная переменная}
\label{sec:std_string_as_global_variable}

Опытные программисты на \Cpp знают, что глобальные переменные \ac{STL}-типов вполне можно объявлять.

Да, действительно:

\lstinputlisting[style=customc]{\CURPATH/STL/string/5.cpp}

Но как и где будет вызываться конструктор \TT{std::string}?

На самом деле, эта переменная будет инициализирована даже перед началом \main.

\lstinputlisting[caption=MSVC 2012: здесь конструируется глобальная переменная{,} а также регистрируется её деструктор,style=customasmx86]{\CURPATH/STL/string/5_MSVC_p2.asm}

\lstinputlisting[caption=MSVC 2012: здесь глобальная переменная используется в \main,style=customasmx86]{\CURPATH/STL/string/5_MSVC_p1.asm}

\lstinputlisting[caption=MSVC 2012: эта функция-деструктор вызывается перед выходом,style=customasmx86]{\CURPATH/STL/string/5_MSVC_p3.asm}

\myindex{\CStandardLibrary!atexit()}
В реальности, из \ac{CRT}, еще до вызова main(), вызывается специальная функция,
в которой перечислены все конструкторы подобных переменных.
Более того: при помощи atexit() регистрируется функция, которая будет вызвана в конце работы программы:
в этой функции компилятор собирает вызовы деструкторов всех подобных глобальных переменных.

GCC работает похожим образом:

\lstinputlisting[caption=GCC 4.8.1,style=customasmx86]{\CURPATH/STL/string/5_GCC.s}

Но он не выделяет отдельной функции в которой будут собраны деструкторы: 
каждый деструктор передается в atexit() по одному.

% TODO а если глобальная STL-переменная в другом модуле? надо проверить.

}
\DE{\subsection{Einfachste XOR-Verschlüsselung überhaupt}

Ich habe einmal eine Software gesehen, bei der alle Debugging-Ausgaben mit XOR mit dem Wert 3
verschlüsselt wurden. Mit anderen Worten, die beiden niedrigsten Bits aller Buchstaben wurden invertiert.

``Hello, world'' wurde zu ``Kfool/\#tlqog'':

\begin{lstlisting}
#!/usr/bin/python

msg="Hello, world!"

print "".join(map(lambda x: chr(ord(x)^3), msg))
\end{lstlisting}

Das ist eine ziemlich interessante Verschlüsselung (oder besser eine Verschleierung),
weil sie zwei wichtige Eigenschaften hat:
1) es ist eine einzige Funktion zum Verschlüsseln und entschlüsseln, sie muss nur wiederholt angewendet werden
2) die entstehenden Buchstaben befinden sich im druckbaren Bereich, also die ganze Zeichenkette kann ohne
Escape-Symbole im Code verwendet werden.

Die zweite Eigenschaft nutzt die Tatsache, dass alle druckbaren Zeichen in Reihen organisiert sind: 0x2x-0x7x,
und wenn die beiden niederwertigsten Bits invertiert werden, wird der Buchstabe um eine oder drei Stellen nach
links oder rechts \IT{verschoben}, aber niemals in eine andere Reihe:

\begin{figure}[H]
\centering
\includegraphics[width=0.7\textwidth]{ascii_clean.png}
\caption{7-Bit \ac{ASCII} Tabelle in Emacs}
\end{figure}

\dots mit dem Zeichen 0x7F als einziger Ausnahme.

Im Folgenden werden also beispielsweise die Zeichen A-Z \IT{verschlüsselt}:

\begin{lstlisting}
#!/usr/bin/python

msg="@ABCDEFGHIJKLMNO"

print "".join(map(lambda x: chr(ord(x)^3), msg))
\end{lstlisting}

Ergebnis:
% FIXME \verb  --  relevant comment for German?
\begin{lstlisting}
CBA@GFEDKJIHONML
\end{lstlisting}

Es sieht so aus als würden die Zeichen ``@'' und ``C'' sowie ``B'' und ``A'' vertauscht werden.

Hier ist noch ein interessantes Beispiel, in dem gezeigt wird, wie die Eigenschaften von XOR
ausgenutzt werden können: Exakt den gleichen Effekt, dass druckbare Zeichen auch druckbar bleiben,
kann man dadurch erzielen, dass irgendeine Kombination der niedrigsten vier Bits invertiert wird.
}

\ifdefined\SPANISH
\chapter{Patrones de código}
\fi % SPANISH

\ifdefined\GERMAN
\chapter{Code-Muster}
\fi % GERMAN

\ifdefined\ENGLISH
\chapter{Code Patterns}
\fi % ENGLISH

\ifdefined\ITALIAN
\chapter{Forme di codice}
\fi % ITALIAN

\ifdefined\RUSSIAN
\chapter{Образцы кода}
\fi % RUSSIAN

\ifdefined\BRAZILIAN
\chapter{Padrões de códigos}
\fi % BRAZILIAN

\ifdefined\THAI
\chapter{รูปแบบของโค้ด}
\fi % THAI

\ifdefined\FRENCH
\chapter{Modèle de code}
\fi % FRENCH

\ifdefined\POLISH
\chapter{\PLph{}}
\fi % POLISH

% sections
\EN{\input{patterns/patterns_opt_dbg_EN}}
\ES{\input{patterns/patterns_opt_dbg_ES}}
\ITA{\input{patterns/patterns_opt_dbg_ITA}}
\PTBR{\input{patterns/patterns_opt_dbg_PTBR}}
\RU{\input{patterns/patterns_opt_dbg_RU}}
\THA{\input{patterns/patterns_opt_dbg_THA}}
\DE{\input{patterns/patterns_opt_dbg_DE}}
\FR{\input{patterns/patterns_opt_dbg_FR}}
\PL{\input{patterns/patterns_opt_dbg_PL}}

\RU{\section{Некоторые базовые понятия}}
\EN{\section{Some basics}}
\DE{\section{Einige Grundlagen}}
\FR{\section{Quelques bases}}
\ES{\section{\ESph{}}}
\ITA{\section{Alcune basi teoriche}}
\PTBR{\section{\PTBRph{}}}
\THA{\section{\THAph{}}}
\PL{\section{\PLph{}}}

% sections:
\EN{\input{patterns/intro_CPU_ISA_EN}}
\ES{\input{patterns/intro_CPU_ISA_ES}}
\ITA{\input{patterns/intro_CPU_ISA_ITA}}
\PTBR{\input{patterns/intro_CPU_ISA_PTBR}}
\RU{\input{patterns/intro_CPU_ISA_RU}}
\DE{\input{patterns/intro_CPU_ISA_DE}}
\FR{\input{patterns/intro_CPU_ISA_FR}}
\PL{\input{patterns/intro_CPU_ISA_PL}}

\EN{\input{patterns/numeral_EN}}
\RU{\input{patterns/numeral_RU}}
\ITA{\input{patterns/numeral_ITA}}
\DE{\input{patterns/numeral_DE}}
\FR{\input{patterns/numeral_FR}}
\PL{\input{patterns/numeral_PL}}

% chapters
\input{patterns/00_empty/main}
\input{patterns/011_ret/main}
\input{patterns/01_helloworld/main}
\input{patterns/015_prolog_epilogue/main}
\input{patterns/02_stack/main}
\input{patterns/03_printf/main}
\input{patterns/04_scanf/main}
\input{patterns/05_passing_arguments/main}
\input{patterns/06_return_results/main}
\input{patterns/061_pointers/main}
\input{patterns/065_GOTO/main}
\input{patterns/07_jcc/main}
\input{patterns/08_switch/main}
\input{patterns/09_loops/main}
\input{patterns/10_strings/main}
\input{patterns/11_arith_optimizations/main}
\input{patterns/12_FPU/main}
\input{patterns/13_arrays/main}
\input{patterns/14_bitfields/main}
\EN{\input{patterns/145_LCG/main_EN}}
\RU{\input{patterns/145_LCG/main_RU}}
\input{patterns/15_structs/main}
\input{patterns/17_unions/main}
\input{patterns/18_pointers_to_functions/main}
\input{patterns/185_64bit_in_32_env/main}

\EN{\input{patterns/19_SIMD/main_EN}}
\RU{\input{patterns/19_SIMD/main_RU}}
\DE{\input{patterns/19_SIMD/main_DE}}

\EN{\input{patterns/20_x64/main_EN}}
\RU{\input{patterns/20_x64/main_RU}}

\EN{\input{patterns/205_floating_SIMD/main_EN}}
\RU{\input{patterns/205_floating_SIMD/main_RU}}
\DE{\input{patterns/205_floating_SIMD/main_DE}}

\EN{\input{patterns/ARM/main_EN}}
\RU{\input{patterns/ARM/main_RU}}
\DE{\input{patterns/ARM/main_DE}}

\input{patterns/MIPS/main}


\ifdefined\SPANISH
\chapter{Patrones de código}
\fi % SPANISH

\ifdefined\GERMAN
\chapter{Code-Muster}
\fi % GERMAN

\ifdefined\ENGLISH
\chapter{Code Patterns}
\fi % ENGLISH

\ifdefined\ITALIAN
\chapter{Forme di codice}
\fi % ITALIAN

\ifdefined\RUSSIAN
\chapter{Образцы кода}
\fi % RUSSIAN

\ifdefined\BRAZILIAN
\chapter{Padrões de códigos}
\fi % BRAZILIAN

\ifdefined\THAI
\chapter{รูปแบบของโค้ด}
\fi % THAI

\ifdefined\FRENCH
\chapter{Modèle de code}
\fi % FRENCH

\ifdefined\POLISH
\chapter{\PLph{}}
\fi % POLISH

% sections
\EN{\section{The method}

When the author of this book first started learning C and, later, \Cpp, he used to write small pieces of code, compile them,
and then look at the assembly language output. This made it very easy for him to understand what was going on in the code that he had written.
\footnote{In fact, he still does this when he can't understand what a particular bit of code does.}.
He did this so many times that the relationship between the \CCpp code and what the compiler produced was imprinted deeply in his mind.
It's now easy for him to imagine instantly a rough outline of a C code's appearance and function.
Perhaps this technique could be helpful for others.

%There are a lot of examples for both x86/x64 and ARM.
%Those who already familiar with one of architectures, may freely skim over pages.

By the way, there is a great website where you can do the same, with various compilers, instead of installing them on your box.
You can use it as well: \url{https://gcc.godbolt.org/}.

\section*{\Exercises}

When the author of this book studied assembly language, he also often compiled small C functions and then rewrote
them gradually to assembly, trying to make their code as short as possible.
This probably is not worth doing in real-world scenarios today,
because it's hard to compete with the latest compilers in terms of efficiency. It is, however, a very good way to gain a better understanding of assembly.
Feel free, therefore, to take any assembly code from this book and try to make it shorter.
However, don't forget to test what you have written.

% rewrote to show that debug\release and optimisations levels are orthogonal concepts.
\section*{Optimization levels and debug information}

Source code can be compiled by different compilers with various optimization levels.
A typical compiler has about three such levels, where level zero means that optimization is completely disabled.
Optimization can also be targeted towards code size or code speed.
A non-optimizing compiler is faster and produces more understandable (albeit verbose) code,
whereas an optimizing compiler is slower and tries to produce code that runs faster (but is not necessarily more compact).
In addition to optimization levels, a compiler can include some debug information in the resulting file,
producing code that is easy to debug.
One of the important features of the ´debug' code is that it might contain links
between each line of the source code and its respective machine code address.
Optimizing compilers, on the other hand, tend to produce output where entire lines of source code
can be optimized away and thus not even be present in the resulting machine code.
Reverse engineers can encounter either version, simply because some developers turn on the compiler's optimization flags and others do not.
Because of this, we'll try to work on examples of both debug and release versions of the code featured in this book, wherever possible.

Sometimes some pretty ancient compilers are used in this book, in order to get the shortest (or simplest) possible code snippet.
}
\ES{\input{patterns/patterns_opt_dbg_ES}}
\ITA{\input{patterns/patterns_opt_dbg_ITA}}
\PTBR{\input{patterns/patterns_opt_dbg_PTBR}}
\RU{\input{patterns/patterns_opt_dbg_RU}}
\THA{\input{patterns/patterns_opt_dbg_THA}}
\DE{\input{patterns/patterns_opt_dbg_DE}}
\FR{\input{patterns/patterns_opt_dbg_FR}}
\PL{\input{patterns/patterns_opt_dbg_PL}}

\RU{\section{Некоторые базовые понятия}}
\EN{\section{Some basics}}
\DE{\section{Einige Grundlagen}}
\FR{\section{Quelques bases}}
\ES{\section{\ESph{}}}
\ITA{\section{Alcune basi teoriche}}
\PTBR{\section{\PTBRph{}}}
\THA{\section{\THAph{}}}
\PL{\section{\PLph{}}}

% sections:
\EN{\input{patterns/intro_CPU_ISA_EN}}
\ES{\input{patterns/intro_CPU_ISA_ES}}
\ITA{\input{patterns/intro_CPU_ISA_ITA}}
\PTBR{\input{patterns/intro_CPU_ISA_PTBR}}
\RU{\input{patterns/intro_CPU_ISA_RU}}
\DE{\input{patterns/intro_CPU_ISA_DE}}
\FR{\input{patterns/intro_CPU_ISA_FR}}
\PL{\input{patterns/intro_CPU_ISA_PL}}

\EN{\subsection{Numeral Systems}

Humans have become accustomed to a decimal numeral system, probably because almost everyone has 10 fingers.
Nevertheless, the number \q{10} has no significant meaning in science and mathematics.
The natural numeral system in digital electronics is binary: 0 is for an absence of current in the wire, and 1 for presence.
10 in binary is 2 in decimal, 100 in binary is 4 in decimal, and so on.

% This sentence is a bit unweildy - maybe try 'Our ten-digit system would be described as having a radix...' - Renaissance
If the numeral system has 10 digits, it has a \IT{radix} (or \IT{base}) of 10.
The binary numeral system has a \IT{radix} of 2.

Important things to recall:

1) A \IT{number} is a number, while a \IT{digit} is a term from writing systems, and is usually one character

% The original is 'number' is not changed; I think the intent is value, and changed it - Renaissance
2) The value of a number does not change when converted to another radix; only the writing notation for that value has changed (and therefore the way of representing it in \ac{RAM}).

\subsection{Converting From One Radix To Another}

Positional notation is used almost every numerical system. This means that a digit has weight relative to where it is placed inside of the larger number.
If 2 is placed at the rightmost place, it's 2, but if it's placed one digit before rightmost, it's 20.

What does $1234$ stand for?

$10^3 \cdot 1 + 10^2 \cdot 2 + 10^1 \cdot 3 + 1 \cdot 4 = 1234$ or
$1000 \cdot 1 + 100 \cdot 2 + 10 \cdot 3 + 4 = 1234$

It's the same story for binary numbers, but the base is 2 instead of 10.
What does 0b101011 stand for?

$2^5 \cdot 1 + 2^4 \cdot 0 + 2^3 \cdot 1 + 2^2 \cdot 0 + 2^1 \cdot 1 + 2^0 \cdot 1 = 43$ or
$32 \cdot 1 + 16 \cdot 0 + 8 \cdot 1 + 4 \cdot 0 + 2 \cdot 1 + 1 = 43$

There is such a thing as non-positional notation, such as the Roman numeral system.
\footnote{About numeric system evolution, see \InSqBrackets{\TAOCPvolII{}, 195--213.}}.
% Maybe add a sentence to fill in that X is always 10, and is therefore non-positional, even though putting an I before subtracts and after adds, and is in that sense positional
Perhaps, humankind switched to positional notation because it's easier to do basic operations (addition, multiplication, etc.) on paper by hand.

Binary numbers can be added, subtracted and so on in the very same as taught in schools, but only 2 digits are available.

Binary numbers are bulky when represented in source code and dumps, so that is where the hexadecimal numeral system can be useful.
A hexadecimal radix uses the digits 0..9, and also 6 Latin characters: A..F.
Each hexadecimal digit takes 4 bits or 4 binary digits, so it's very easy to convert from binary number to hexadecimal and back, even manually, in one's mind.

\begin{center}
\begin{longtable}{ | l | l | l | }
\hline
\HeaderColor hexadecimal & \HeaderColor binary & \HeaderColor decimal \\
\hline
0	&0000	&0 \\
1	&0001	&1 \\
2	&0010	&2 \\
3	&0011	&3 \\
4	&0100	&4 \\
5	&0101	&5 \\
6	&0110	&6 \\
7	&0111	&7 \\
8	&1000	&8 \\
9	&1001	&9 \\
A	&1010	&10 \\
B	&1011	&11 \\
C	&1100	&12 \\
D	&1101	&13 \\
E	&1110	&14 \\
F	&1111	&15 \\
\hline
\end{longtable}
\end{center}

How can one tell which radix is being used in a specific instance?

Decimal numbers are usually written as is, i.e., 1234. Some assemblers allow an identifier on decimal radix numbers, in which the number would be written with a "d" suffix: 1234d.

Binary numbers are sometimes prepended with the "0b" prefix: 0b100110111 (\ac{GCC} has a non-standard language extension for this\footnote{\url{https://gcc.gnu.org/onlinedocs/gcc/Binary-constants.html}}).
There is also another way: using a "b" suffix, for example: 100110111b.
This book tries to use the "0b" prefix consistently throughout the book for binary numbers.

Hexadecimal numbers are prepended with "0x" prefix in \CCpp and other \ac{PL}s: 0x1234ABCD.
Alternatively, they are given a "h" suffix: 1234ABCDh. This is common way of representing them in assemblers and debuggers.
In this convention, if the number is started with a Latin (A..F) digit, a 0 is added at the beginning: 0ABCDEFh.
There was also convention that was popular in 8-bit home computers era, using \$ prefix, like \$ABCD.
The book will try to stick to "0x" prefix throughout the book for hexadecimal numbers.

Should one learn to convert numbers mentally? A table of 1-digit hexadecimal numbers can easily be memorized.
As for larger numbers, it's probably not worth tormenting yourself.

Perhaps the most visible hexadecimal numbers are in \ac{URL}s.
This is the way that non-Latin characters are encoded.
For example:
\url{https://en.wiktionary.org/wiki/na\%C3\%AFvet\%C3\%A9} is the \ac{URL} of Wiktionary article about \q{naïveté} word.

\subsubsection{Octal Radix}

Another numeral system heavily used in the past of computer programming is octal. In octal there are 8 digits (0..7), and each is mapped to 3 bits, so it's easy to convert numbers back and forth.
It has been superseded by the hexadecimal system almost everywhere, but, surprisingly, there is a *NIX utility, used often by many people, which takes octal numbers as argument: \TT{chmod}.

\myindex{UNIX!chmod}
As many *NIX users know, \TT{chmod} argument can be a number of 3 digits. The first digit represents the rights of the owner of the file (read, write and/or execute), the second is the rights for the group to which the file belongs, and the third is for everyone else.
Each digit that \TT{chmod} takes can be represented in binary form:

\begin{center}
\begin{longtable}{ | l | l | l | }
\hline
\HeaderColor decimal & \HeaderColor binary & \HeaderColor meaning \\
\hline
7	&111	&\textbf{rwx} \\
6	&110	&\textbf{rw-} \\
5	&101	&\textbf{r-x} \\
4	&100	&\textbf{r-{}-} \\
3	&011	&\textbf{-wx} \\
2	&010	&\textbf{-w-} \\
1	&001	&\textbf{-{}-x} \\
0	&000	&\textbf{-{}-{}-} \\
\hline
\end{longtable}
\end{center}

So each bit is mapped to a flag: read/write/execute.

The importance of \TT{chmod} here is that the whole number in argument can be represented as octal number.
Let's take, for example, 644.
When you run \TT{chmod 644 file}, you set read/write permissions for owner, read permissions for group and again, read permissions for everyone else.
If we convert the octal number 644 to binary, it would be \TT{110100100}, or, in groups of 3 bits, \TT{110 100 100}.

Now we see that each triplet describe permissions for owner/group/others: first is \TT{rw-}, second is \TT{r--} and third is \TT{r--}.

The octal numeral system was also popular on old computers like PDP-8, because word there could be 12, 24 or 36 bits, and these numbers are all divisible by 3, so the octal system was natural in that environment.
Nowadays, all popular computers employ word/address sizes of 16, 32 or 64 bits, and these numbers are all divisible by 4, so the hexadecimal system is more natural there.

The octal numeral system is supported by all standard \CCpp compilers.
This is a source of confusion sometimes, because octal numbers are encoded with a zero prepended, for example, 0377 is 255.
Sometimes, you might make a typo and write "09" instead of 9, and the compiler would report an error.
GCC might report something like this:\\
\TT{error: invalid digit "9" in octal constant}.

Also, the octal system is somewhat popular in Java. When the IDA shows Java strings with non-printable characters,
they are encoded in the octal system instead of hexadecimal.
\myindex{JAD}
The JAD Java decompiler behaves the same way.

\subsubsection{Divisibility}

When you see a decimal number like 120, you can quickly deduce that it's divisible by 10, because the last digit is zero.
In the same way, 123400 is divisible by 100, because the two last digits are zeros.

Likewise, the hexadecimal number 0x1230 is divisible by 0x10 (or 16), 0x123000 is divisible by 0x1000 (or 4096), etc.

The binary number 0b1000101000 is divisible by 0b1000 (8), etc.

This property can often be used to quickly realize if the size of some block in memory is padded to some boundary.
For example, sections in \ac{PE} files are almost always started at addresses ending with 3 hexadecimal zeros: 0x41000, 0x10001000, etc.
The reason behind this is the fact that almost all \ac{PE} sections are padded to a boundary of 0x1000 (4096) bytes.

\subsubsection{Multi-Precision Arithmetic and Radix}

\index{RSA}
Multi-precision arithmetic can use huge numbers, and each one may be stored in several bytes.
For example, RSA keys, both public and private, span up to 4096 bits, and maybe even more.

% I'm not sure how to change this, but the normal format for quoting would be just to mention the author or book, and footnote to the full reference
In \InSqBrackets{\TAOCPvolII, 265} we find the following idea: when you store a multi-precision number in several bytes,
the whole number can be represented as having a radix of $2^8=256$, and each digit goes to the corresponding byte.
Likewise, if you store a multi-precision number in several 32-bit integer values, each digit goes to each 32-bit slot,
and you may think about this number as stored in radix of $2^{32}$.

\subsubsection{How to Pronounce Non-Decimal Numbers}

Numbers in a non-decimal base are usually pronounced by digit by digit: ``one-zero-zero-one-one-...''.
Words like ``ten'' and ``thousand'' are usually not pronounced, to prevent confusion with the decimal base system.

\subsubsection{Floating point numbers}

To distinguish floating point numbers from integers, they are usually written with ``.0'' at the end,
like $0.0$, $123.0$, etc.
}
\RU{\subsection{Представление чисел}

Люди привыкли к десятичной системе счисления вероятно потому что почти у каждого есть по 10 пальцев.
Тем не менее, число 10 не имеет особого значения в науке и математике.
Двоичная система естествена для цифровой электроники: 0 означает отсутствие тока в проводе и 1 --- его присутствие.
10 в двоичной системе это 2 в десятичной; 100 в двоичной это 4 в десятичной, итд.

Если в системе счисления есть 10 цифр, её \IT{основание} или \IT{radix} это 10.
Двоичная система имеет \IT{основание} 2.

Важные вещи, которые полезно вспомнить:
1) \IT{число} это число, в то время как \IT{цифра} это термин из системы письменности, и это обычно один символ;
2) само число не меняется, когда конвертируется из одного основания в другое: меняется способ его записи (или представления
в памяти).

Как сконвертировать число из одного основания в другое?

Позиционная нотация используется почти везде, это означает, что всякая цифра имеет свой вес, в зависимости от её расположения
внутри числа.
Если 2 расположена в самом последнем месте справа, это 2.
Если она расположена в месте перед последним, это 20.

Что означает $1234$?

$10^3 \cdot 1 + 10^2 \cdot 2 + 10^1 \cdot 3 + 1 \cdot 4$ = 1234 или
$1000 \cdot 1 + 100 \cdot 2 + 10 \cdot 3 + 4 = 1234$

Та же история и для двоичных чисел, только основание там 2 вместо 10.
Что означает 0b101011?

$2^5 \cdot 1 + 2^4 \cdot 0 + 2^3 \cdot 1 + 2^2 \cdot 0 + 2^1 \cdot 1 + 2^0 \cdot 1 = 43$ или
$32 \cdot 1 + 16 \cdot 0 + 8 \cdot 1 + 4 \cdot 0 + 2 \cdot 1 + 1 = 43$

Позиционную нотацию можно противопоставить непозиционной нотации, такой как римская система записи чисел
\footnote{Об эволюции способов записи чисел, см.также: \InSqBrackets{\TAOCPvolII{}, 195--213.}}.
Вероятно, человечество перешло на позиционную нотацию, потому что так проще работать с числами (сложение, умножение, итд)
на бумаге, в ручную.

Действительно, двоичные числа можно складывать, вычитать, итд, точно также, как этому обычно обучают в школах,
только доступны лишь 2 цифры.

Двоичные числа громоздки, когда их используют в исходных кодах и дампах, так что в этих случаях применяется шестнадцатеричная
система.
Используются цифры 0..9 и еще 6 латинских букв: A..F.
Каждая шестнадцатеричная цифра занимает 4 бита или 4 двоичных цифры, так что конвертировать из двоичной системы в
шестнадцатеричную и назад, можно легко вручную, или даже в уме.

\begin{center}
\begin{longtable}{ | l | l | l | }
\hline
\HeaderColor шестнадцатеричная & \HeaderColor двоичная & \HeaderColor десятичная \\
\hline
0	&0000	&0 \\
1	&0001	&1 \\
2	&0010	&2 \\
3	&0011	&3 \\
4	&0100	&4 \\
5	&0101	&5 \\
6	&0110	&6 \\
7	&0111	&7 \\
8	&1000	&8 \\
9	&1001	&9 \\
A	&1010	&10 \\
B	&1011	&11 \\
C	&1100	&12 \\
D	&1101	&13 \\
E	&1110	&14 \\
F	&1111	&15 \\
\hline
\end{longtable}
\end{center}

Как понять, какое основание используется в конкретном месте?

Десятичные числа обычно записываются как есть, т.е., 1234. Но некоторые ассемблеры позволяют подчеркивать
этот факт для ясности, и это число может быть дополнено суффиксом "d": 1234d.

К двоичным числам иногда спереди добавляют префикс "0b": 0b100110111
(В \ac{GCC} для этого есть нестандартное расширение языка
\footnote{\url{https://gcc.gnu.org/onlinedocs/gcc/Binary-constants.html}}).
Есть также еще один способ: суффикс "b", например: 100110111b.
В этой книге я буду пытаться придерживаться префикса "0b" для двоичных чисел.

Шестнадцатеричные числа имеют префикс "0x" в \CCpp и некоторых других \ac{PL}: 0x1234ABCD.
Либо они имеют суффикс "h": 1234ABCDh --- обычно так они представляются в ассемблерах и отладчиках.
Если число начинается с цифры A..F, перед ним добавляется 0: 0ABCDEFh.
Во времена 8-битных домашних компьютеров, был также способ записи чисел используя префикс \$, например, \$ABCD.
В книге я попытаюсь придерживаться префикса "0x" для шестнадцатеричных чисел.

Нужно ли учиться конвертировать числа в уме? Таблицу шестнадцатеричных чисел из одной цифры легко запомнить.
А запоминать б\'{о}льшие числа, наверное, не стоит.

Наверное, чаще всего шестнадцатеричные числа можно увидеть в \ac{URL}-ах.
Так кодируются буквы не из числа латинских.
Например:
\url{https://en.wiktionary.org/wiki/na\%C3\%AFvet\%C3\%A9} это \ac{URL} страницы в Wiktionary о слове \q{naïveté}.

\subsubsection{Восьмеричная система}

Еще одна система, которая в прошлом много использовалась в программировании это восьмеричная: есть 8 цифр (0..7) и каждая
описывает 3 бита, так что легко конвертировать числа туда и назад.
Она почти везде была заменена шестнадцатеричной, но удивительно, в *NIX имеется утилита использующаяся многими людьми,
которая принимает на вход восьмеричное число: \TT{chmod}.

\myindex{UNIX!chmod}
Как знают многие пользователи *NIX, аргумент \TT{chmod} это число из трех цифр. Первая цифра это права владельца файла,
вторая это права группы (которой файл принадлежит), третья для всех остальных.
И каждая цифра может быть представлена в двоичном виде:

\begin{center}
\begin{longtable}{ | l | l | l | }
\hline
\HeaderColor десятичная & \HeaderColor двоичная & \HeaderColor значение \\
\hline
7	&111	&\textbf{rwx} \\
6	&110	&\textbf{rw-} \\
5	&101	&\textbf{r-x} \\
4	&100	&\textbf{r-{}-} \\
3	&011	&\textbf{-wx} \\
2	&010	&\textbf{-w-} \\
1	&001	&\textbf{-{}-x} \\
0	&000	&\textbf{-{}-{}-} \\
\hline
\end{longtable}
\end{center}

Так что каждый бит привязан к флагу: read/write/execute (чтение/запись/исполнение).

И вот почему я вспомнил здесь о \TT{chmod}, это потому что всё число может быть представлено как число в восьмеричной системе.
Для примера возьмем 644.
Когда вы запускаете \TT{chmod 644 file}, вы выставляете права read/write для владельца, права read для группы, и снова,
read для всех остальных.
Сконвертируем число 644 из восьмеричной системы в двоичную, это будет \TT{110100100}, или (в группах по 3 бита) \TT{110 100 100}.

Теперь мы видим, что каждая тройка описывает права для владельца/группы/остальных:
первая это \TT{rw-}, вторая это \TT{r--} и третья это \TT{r--}.

Восьмеричная система была также популярная на старых компьютерах вроде PDP-8, потому что слово там могло содержать 12, 24 или
36 бит, и эти числа делятся на 3, так что выбор восьмеричной системы в той среде был логичен.
Сейчас, все популярные компьютеры имеют размер слова/адреса 16, 32 или 64 бита, и эти числа делятся на 4,
так что шестнадцатеричная система здесь удобнее.

Восьмеричная система поддерживается всеми стандартными компиляторами \CCpp{}.
Это иногда источник недоумения, потому что восьмеричные числа кодируются с нулем вперед, например, 0377 это 255.
И иногда, вы можете сделать опечатку, и написать "09" вместо 9, и компилятор выдаст ошибку.
GCC может выдать что-то вроде:\\
\TT{error: invalid digit "9" in octal constant}.

Также, восьмеричная система популярна в Java: когда IDA показывает строку с непечатаемыми символами,
они кодируются в восьмеричной системе вместо шестнадцатеричной.
\myindex{JAD}
Точно также себя ведет декомпилятор с Java JAD.

\subsubsection{Делимость}

Когда вы видите десятичное число вроде 120, вы можете быстро понять что оно делится на 10, потому что последняя цифра это 0.
Точно также, 123400 делится на 100, потому что две последних цифры это нули.

Точно также, шестнадцатеричное число 0x1230 делится на 0x10 (или 16), 0x123000 делится на 0x1000 (или 4096), итд.

Двоичное число 0b1000101000 делится на 0b1000 (8), итд.

Это свойство можно часто использовать, чтобы быстро понять,
что длина какого-либо блока в памяти выровнена по некоторой границе.
Например, секции в \ac{PE}-файлах почти всегда начинаются с адресов заканчивающихся 3 шестнадцатеричными нулями:
0x41000, 0x10001000, итд.
Причина в том, что почти все секции в \ac{PE} выровнены по границе 0x1000 (4096) байт.

\subsubsection{Арифметика произвольной точности и основание}

\index{RSA}
Арифметика произвольной точности (multi-precision arithmetic) может использовать огромные числа,
которые могут храниться в нескольких байтах.
Например, ключи RSA, и открытые и закрытые, могут занимать до 4096 бит и даже больше.

В \InSqBrackets{\TAOCPvolII, 265} можно найти такую идею: когда вы сохраняете число произвольной точности в нескольких байтах,
всё число может быть представлено как имеющую систему счисления по основанию $2^8=256$, и каждая цифра находится
в соответствующем байте.
Точно также, если вы сохраняете число произвольной точности в нескольких 32-битных целочисленных значениях,
каждая цифра отправляется в каждый 32-битный слот, и вы можете считать что это число записано в системе с основанием $2^{32}$.

\subsubsection{Произношение}

Числа в недесятичных системах счислениях обычно произносятся по одной цифре: ``один-ноль-ноль-один-один-...''.
Слова вроде ``десять'', ``тысяча'', итд, обычно не произносятся, потому что тогда можно спутать с десятичной системой.

\subsubsection{Числа с плавающей запятой}

Чтобы отличать числа с плавающей запятой от целочисленных, часто, в конце добавляют ``.0'',
например $0.0$, $123.0$, итд.

}
\ITA{\input{patterns/numeral_ITA}}
\DE{\input{patterns/numeral_DE}}
\FR{\input{patterns/numeral_FR}}
\PL{\input{patterns/numeral_PL}}

% chapters
\ifdefined\SPANISH
\chapter{Patrones de código}
\fi % SPANISH

\ifdefined\GERMAN
\chapter{Code-Muster}
\fi % GERMAN

\ifdefined\ENGLISH
\chapter{Code Patterns}
\fi % ENGLISH

\ifdefined\ITALIAN
\chapter{Forme di codice}
\fi % ITALIAN

\ifdefined\RUSSIAN
\chapter{Образцы кода}
\fi % RUSSIAN

\ifdefined\BRAZILIAN
\chapter{Padrões de códigos}
\fi % BRAZILIAN

\ifdefined\THAI
\chapter{รูปแบบของโค้ด}
\fi % THAI

\ifdefined\FRENCH
\chapter{Modèle de code}
\fi % FRENCH

\ifdefined\POLISH
\chapter{\PLph{}}
\fi % POLISH

% sections
\EN{\input{patterns/patterns_opt_dbg_EN}}
\ES{\input{patterns/patterns_opt_dbg_ES}}
\ITA{\input{patterns/patterns_opt_dbg_ITA}}
\PTBR{\input{patterns/patterns_opt_dbg_PTBR}}
\RU{\input{patterns/patterns_opt_dbg_RU}}
\THA{\input{patterns/patterns_opt_dbg_THA}}
\DE{\input{patterns/patterns_opt_dbg_DE}}
\FR{\input{patterns/patterns_opt_dbg_FR}}
\PL{\input{patterns/patterns_opt_dbg_PL}}

\RU{\section{Некоторые базовые понятия}}
\EN{\section{Some basics}}
\DE{\section{Einige Grundlagen}}
\FR{\section{Quelques bases}}
\ES{\section{\ESph{}}}
\ITA{\section{Alcune basi teoriche}}
\PTBR{\section{\PTBRph{}}}
\THA{\section{\THAph{}}}
\PL{\section{\PLph{}}}

% sections:
\EN{\input{patterns/intro_CPU_ISA_EN}}
\ES{\input{patterns/intro_CPU_ISA_ES}}
\ITA{\input{patterns/intro_CPU_ISA_ITA}}
\PTBR{\input{patterns/intro_CPU_ISA_PTBR}}
\RU{\input{patterns/intro_CPU_ISA_RU}}
\DE{\input{patterns/intro_CPU_ISA_DE}}
\FR{\input{patterns/intro_CPU_ISA_FR}}
\PL{\input{patterns/intro_CPU_ISA_PL}}

\EN{\input{patterns/numeral_EN}}
\RU{\input{patterns/numeral_RU}}
\ITA{\input{patterns/numeral_ITA}}
\DE{\input{patterns/numeral_DE}}
\FR{\input{patterns/numeral_FR}}
\PL{\input{patterns/numeral_PL}}

% chapters
\input{patterns/00_empty/main}
\input{patterns/011_ret/main}
\input{patterns/01_helloworld/main}
\input{patterns/015_prolog_epilogue/main}
\input{patterns/02_stack/main}
\input{patterns/03_printf/main}
\input{patterns/04_scanf/main}
\input{patterns/05_passing_arguments/main}
\input{patterns/06_return_results/main}
\input{patterns/061_pointers/main}
\input{patterns/065_GOTO/main}
\input{patterns/07_jcc/main}
\input{patterns/08_switch/main}
\input{patterns/09_loops/main}
\input{patterns/10_strings/main}
\input{patterns/11_arith_optimizations/main}
\input{patterns/12_FPU/main}
\input{patterns/13_arrays/main}
\input{patterns/14_bitfields/main}
\EN{\input{patterns/145_LCG/main_EN}}
\RU{\input{patterns/145_LCG/main_RU}}
\input{patterns/15_structs/main}
\input{patterns/17_unions/main}
\input{patterns/18_pointers_to_functions/main}
\input{patterns/185_64bit_in_32_env/main}

\EN{\input{patterns/19_SIMD/main_EN}}
\RU{\input{patterns/19_SIMD/main_RU}}
\DE{\input{patterns/19_SIMD/main_DE}}

\EN{\input{patterns/20_x64/main_EN}}
\RU{\input{patterns/20_x64/main_RU}}

\EN{\input{patterns/205_floating_SIMD/main_EN}}
\RU{\input{patterns/205_floating_SIMD/main_RU}}
\DE{\input{patterns/205_floating_SIMD/main_DE}}

\EN{\input{patterns/ARM/main_EN}}
\RU{\input{patterns/ARM/main_RU}}
\DE{\input{patterns/ARM/main_DE}}

\input{patterns/MIPS/main}

\ifdefined\SPANISH
\chapter{Patrones de código}
\fi % SPANISH

\ifdefined\GERMAN
\chapter{Code-Muster}
\fi % GERMAN

\ifdefined\ENGLISH
\chapter{Code Patterns}
\fi % ENGLISH

\ifdefined\ITALIAN
\chapter{Forme di codice}
\fi % ITALIAN

\ifdefined\RUSSIAN
\chapter{Образцы кода}
\fi % RUSSIAN

\ifdefined\BRAZILIAN
\chapter{Padrões de códigos}
\fi % BRAZILIAN

\ifdefined\THAI
\chapter{รูปแบบของโค้ด}
\fi % THAI

\ifdefined\FRENCH
\chapter{Modèle de code}
\fi % FRENCH

\ifdefined\POLISH
\chapter{\PLph{}}
\fi % POLISH

% sections
\EN{\input{patterns/patterns_opt_dbg_EN}}
\ES{\input{patterns/patterns_opt_dbg_ES}}
\ITA{\input{patterns/patterns_opt_dbg_ITA}}
\PTBR{\input{patterns/patterns_opt_dbg_PTBR}}
\RU{\input{patterns/patterns_opt_dbg_RU}}
\THA{\input{patterns/patterns_opt_dbg_THA}}
\DE{\input{patterns/patterns_opt_dbg_DE}}
\FR{\input{patterns/patterns_opt_dbg_FR}}
\PL{\input{patterns/patterns_opt_dbg_PL}}

\RU{\section{Некоторые базовые понятия}}
\EN{\section{Some basics}}
\DE{\section{Einige Grundlagen}}
\FR{\section{Quelques bases}}
\ES{\section{\ESph{}}}
\ITA{\section{Alcune basi teoriche}}
\PTBR{\section{\PTBRph{}}}
\THA{\section{\THAph{}}}
\PL{\section{\PLph{}}}

% sections:
\EN{\input{patterns/intro_CPU_ISA_EN}}
\ES{\input{patterns/intro_CPU_ISA_ES}}
\ITA{\input{patterns/intro_CPU_ISA_ITA}}
\PTBR{\input{patterns/intro_CPU_ISA_PTBR}}
\RU{\input{patterns/intro_CPU_ISA_RU}}
\DE{\input{patterns/intro_CPU_ISA_DE}}
\FR{\input{patterns/intro_CPU_ISA_FR}}
\PL{\input{patterns/intro_CPU_ISA_PL}}

\EN{\input{patterns/numeral_EN}}
\RU{\input{patterns/numeral_RU}}
\ITA{\input{patterns/numeral_ITA}}
\DE{\input{patterns/numeral_DE}}
\FR{\input{patterns/numeral_FR}}
\PL{\input{patterns/numeral_PL}}

% chapters
\input{patterns/00_empty/main}
\input{patterns/011_ret/main}
\input{patterns/01_helloworld/main}
\input{patterns/015_prolog_epilogue/main}
\input{patterns/02_stack/main}
\input{patterns/03_printf/main}
\input{patterns/04_scanf/main}
\input{patterns/05_passing_arguments/main}
\input{patterns/06_return_results/main}
\input{patterns/061_pointers/main}
\input{patterns/065_GOTO/main}
\input{patterns/07_jcc/main}
\input{patterns/08_switch/main}
\input{patterns/09_loops/main}
\input{patterns/10_strings/main}
\input{patterns/11_arith_optimizations/main}
\input{patterns/12_FPU/main}
\input{patterns/13_arrays/main}
\input{patterns/14_bitfields/main}
\EN{\input{patterns/145_LCG/main_EN}}
\RU{\input{patterns/145_LCG/main_RU}}
\input{patterns/15_structs/main}
\input{patterns/17_unions/main}
\input{patterns/18_pointers_to_functions/main}
\input{patterns/185_64bit_in_32_env/main}

\EN{\input{patterns/19_SIMD/main_EN}}
\RU{\input{patterns/19_SIMD/main_RU}}
\DE{\input{patterns/19_SIMD/main_DE}}

\EN{\input{patterns/20_x64/main_EN}}
\RU{\input{patterns/20_x64/main_RU}}

\EN{\input{patterns/205_floating_SIMD/main_EN}}
\RU{\input{patterns/205_floating_SIMD/main_RU}}
\DE{\input{patterns/205_floating_SIMD/main_DE}}

\EN{\input{patterns/ARM/main_EN}}
\RU{\input{patterns/ARM/main_RU}}
\DE{\input{patterns/ARM/main_DE}}

\input{patterns/MIPS/main}

\ifdefined\SPANISH
\chapter{Patrones de código}
\fi % SPANISH

\ifdefined\GERMAN
\chapter{Code-Muster}
\fi % GERMAN

\ifdefined\ENGLISH
\chapter{Code Patterns}
\fi % ENGLISH

\ifdefined\ITALIAN
\chapter{Forme di codice}
\fi % ITALIAN

\ifdefined\RUSSIAN
\chapter{Образцы кода}
\fi % RUSSIAN

\ifdefined\BRAZILIAN
\chapter{Padrões de códigos}
\fi % BRAZILIAN

\ifdefined\THAI
\chapter{รูปแบบของโค้ด}
\fi % THAI

\ifdefined\FRENCH
\chapter{Modèle de code}
\fi % FRENCH

\ifdefined\POLISH
\chapter{\PLph{}}
\fi % POLISH

% sections
\EN{\input{patterns/patterns_opt_dbg_EN}}
\ES{\input{patterns/patterns_opt_dbg_ES}}
\ITA{\input{patterns/patterns_opt_dbg_ITA}}
\PTBR{\input{patterns/patterns_opt_dbg_PTBR}}
\RU{\input{patterns/patterns_opt_dbg_RU}}
\THA{\input{patterns/patterns_opt_dbg_THA}}
\DE{\input{patterns/patterns_opt_dbg_DE}}
\FR{\input{patterns/patterns_opt_dbg_FR}}
\PL{\input{patterns/patterns_opt_dbg_PL}}

\RU{\section{Некоторые базовые понятия}}
\EN{\section{Some basics}}
\DE{\section{Einige Grundlagen}}
\FR{\section{Quelques bases}}
\ES{\section{\ESph{}}}
\ITA{\section{Alcune basi teoriche}}
\PTBR{\section{\PTBRph{}}}
\THA{\section{\THAph{}}}
\PL{\section{\PLph{}}}

% sections:
\EN{\input{patterns/intro_CPU_ISA_EN}}
\ES{\input{patterns/intro_CPU_ISA_ES}}
\ITA{\input{patterns/intro_CPU_ISA_ITA}}
\PTBR{\input{patterns/intro_CPU_ISA_PTBR}}
\RU{\input{patterns/intro_CPU_ISA_RU}}
\DE{\input{patterns/intro_CPU_ISA_DE}}
\FR{\input{patterns/intro_CPU_ISA_FR}}
\PL{\input{patterns/intro_CPU_ISA_PL}}

\EN{\input{patterns/numeral_EN}}
\RU{\input{patterns/numeral_RU}}
\ITA{\input{patterns/numeral_ITA}}
\DE{\input{patterns/numeral_DE}}
\FR{\input{patterns/numeral_FR}}
\PL{\input{patterns/numeral_PL}}

% chapters
\input{patterns/00_empty/main}
\input{patterns/011_ret/main}
\input{patterns/01_helloworld/main}
\input{patterns/015_prolog_epilogue/main}
\input{patterns/02_stack/main}
\input{patterns/03_printf/main}
\input{patterns/04_scanf/main}
\input{patterns/05_passing_arguments/main}
\input{patterns/06_return_results/main}
\input{patterns/061_pointers/main}
\input{patterns/065_GOTO/main}
\input{patterns/07_jcc/main}
\input{patterns/08_switch/main}
\input{patterns/09_loops/main}
\input{patterns/10_strings/main}
\input{patterns/11_arith_optimizations/main}
\input{patterns/12_FPU/main}
\input{patterns/13_arrays/main}
\input{patterns/14_bitfields/main}
\EN{\input{patterns/145_LCG/main_EN}}
\RU{\input{patterns/145_LCG/main_RU}}
\input{patterns/15_structs/main}
\input{patterns/17_unions/main}
\input{patterns/18_pointers_to_functions/main}
\input{patterns/185_64bit_in_32_env/main}

\EN{\input{patterns/19_SIMD/main_EN}}
\RU{\input{patterns/19_SIMD/main_RU}}
\DE{\input{patterns/19_SIMD/main_DE}}

\EN{\input{patterns/20_x64/main_EN}}
\RU{\input{patterns/20_x64/main_RU}}

\EN{\input{patterns/205_floating_SIMD/main_EN}}
\RU{\input{patterns/205_floating_SIMD/main_RU}}
\DE{\input{patterns/205_floating_SIMD/main_DE}}

\EN{\input{patterns/ARM/main_EN}}
\RU{\input{patterns/ARM/main_RU}}
\DE{\input{patterns/ARM/main_DE}}

\input{patterns/MIPS/main}

\ifdefined\SPANISH
\chapter{Patrones de código}
\fi % SPANISH

\ifdefined\GERMAN
\chapter{Code-Muster}
\fi % GERMAN

\ifdefined\ENGLISH
\chapter{Code Patterns}
\fi % ENGLISH

\ifdefined\ITALIAN
\chapter{Forme di codice}
\fi % ITALIAN

\ifdefined\RUSSIAN
\chapter{Образцы кода}
\fi % RUSSIAN

\ifdefined\BRAZILIAN
\chapter{Padrões de códigos}
\fi % BRAZILIAN

\ifdefined\THAI
\chapter{รูปแบบของโค้ด}
\fi % THAI

\ifdefined\FRENCH
\chapter{Modèle de code}
\fi % FRENCH

\ifdefined\POLISH
\chapter{\PLph{}}
\fi % POLISH

% sections
\EN{\input{patterns/patterns_opt_dbg_EN}}
\ES{\input{patterns/patterns_opt_dbg_ES}}
\ITA{\input{patterns/patterns_opt_dbg_ITA}}
\PTBR{\input{patterns/patterns_opt_dbg_PTBR}}
\RU{\input{patterns/patterns_opt_dbg_RU}}
\THA{\input{patterns/patterns_opt_dbg_THA}}
\DE{\input{patterns/patterns_opt_dbg_DE}}
\FR{\input{patterns/patterns_opt_dbg_FR}}
\PL{\input{patterns/patterns_opt_dbg_PL}}

\RU{\section{Некоторые базовые понятия}}
\EN{\section{Some basics}}
\DE{\section{Einige Grundlagen}}
\FR{\section{Quelques bases}}
\ES{\section{\ESph{}}}
\ITA{\section{Alcune basi teoriche}}
\PTBR{\section{\PTBRph{}}}
\THA{\section{\THAph{}}}
\PL{\section{\PLph{}}}

% sections:
\EN{\input{patterns/intro_CPU_ISA_EN}}
\ES{\input{patterns/intro_CPU_ISA_ES}}
\ITA{\input{patterns/intro_CPU_ISA_ITA}}
\PTBR{\input{patterns/intro_CPU_ISA_PTBR}}
\RU{\input{patterns/intro_CPU_ISA_RU}}
\DE{\input{patterns/intro_CPU_ISA_DE}}
\FR{\input{patterns/intro_CPU_ISA_FR}}
\PL{\input{patterns/intro_CPU_ISA_PL}}

\EN{\input{patterns/numeral_EN}}
\RU{\input{patterns/numeral_RU}}
\ITA{\input{patterns/numeral_ITA}}
\DE{\input{patterns/numeral_DE}}
\FR{\input{patterns/numeral_FR}}
\PL{\input{patterns/numeral_PL}}

% chapters
\input{patterns/00_empty/main}
\input{patterns/011_ret/main}
\input{patterns/01_helloworld/main}
\input{patterns/015_prolog_epilogue/main}
\input{patterns/02_stack/main}
\input{patterns/03_printf/main}
\input{patterns/04_scanf/main}
\input{patterns/05_passing_arguments/main}
\input{patterns/06_return_results/main}
\input{patterns/061_pointers/main}
\input{patterns/065_GOTO/main}
\input{patterns/07_jcc/main}
\input{patterns/08_switch/main}
\input{patterns/09_loops/main}
\input{patterns/10_strings/main}
\input{patterns/11_arith_optimizations/main}
\input{patterns/12_FPU/main}
\input{patterns/13_arrays/main}
\input{patterns/14_bitfields/main}
\EN{\input{patterns/145_LCG/main_EN}}
\RU{\input{patterns/145_LCG/main_RU}}
\input{patterns/15_structs/main}
\input{patterns/17_unions/main}
\input{patterns/18_pointers_to_functions/main}
\input{patterns/185_64bit_in_32_env/main}

\EN{\input{patterns/19_SIMD/main_EN}}
\RU{\input{patterns/19_SIMD/main_RU}}
\DE{\input{patterns/19_SIMD/main_DE}}

\EN{\input{patterns/20_x64/main_EN}}
\RU{\input{patterns/20_x64/main_RU}}

\EN{\input{patterns/205_floating_SIMD/main_EN}}
\RU{\input{patterns/205_floating_SIMD/main_RU}}
\DE{\input{patterns/205_floating_SIMD/main_DE}}

\EN{\input{patterns/ARM/main_EN}}
\RU{\input{patterns/ARM/main_RU}}
\DE{\input{patterns/ARM/main_DE}}

\input{patterns/MIPS/main}

\ifdefined\SPANISH
\chapter{Patrones de código}
\fi % SPANISH

\ifdefined\GERMAN
\chapter{Code-Muster}
\fi % GERMAN

\ifdefined\ENGLISH
\chapter{Code Patterns}
\fi % ENGLISH

\ifdefined\ITALIAN
\chapter{Forme di codice}
\fi % ITALIAN

\ifdefined\RUSSIAN
\chapter{Образцы кода}
\fi % RUSSIAN

\ifdefined\BRAZILIAN
\chapter{Padrões de códigos}
\fi % BRAZILIAN

\ifdefined\THAI
\chapter{รูปแบบของโค้ด}
\fi % THAI

\ifdefined\FRENCH
\chapter{Modèle de code}
\fi % FRENCH

\ifdefined\POLISH
\chapter{\PLph{}}
\fi % POLISH

% sections
\EN{\input{patterns/patterns_opt_dbg_EN}}
\ES{\input{patterns/patterns_opt_dbg_ES}}
\ITA{\input{patterns/patterns_opt_dbg_ITA}}
\PTBR{\input{patterns/patterns_opt_dbg_PTBR}}
\RU{\input{patterns/patterns_opt_dbg_RU}}
\THA{\input{patterns/patterns_opt_dbg_THA}}
\DE{\input{patterns/patterns_opt_dbg_DE}}
\FR{\input{patterns/patterns_opt_dbg_FR}}
\PL{\input{patterns/patterns_opt_dbg_PL}}

\RU{\section{Некоторые базовые понятия}}
\EN{\section{Some basics}}
\DE{\section{Einige Grundlagen}}
\FR{\section{Quelques bases}}
\ES{\section{\ESph{}}}
\ITA{\section{Alcune basi teoriche}}
\PTBR{\section{\PTBRph{}}}
\THA{\section{\THAph{}}}
\PL{\section{\PLph{}}}

% sections:
\EN{\input{patterns/intro_CPU_ISA_EN}}
\ES{\input{patterns/intro_CPU_ISA_ES}}
\ITA{\input{patterns/intro_CPU_ISA_ITA}}
\PTBR{\input{patterns/intro_CPU_ISA_PTBR}}
\RU{\input{patterns/intro_CPU_ISA_RU}}
\DE{\input{patterns/intro_CPU_ISA_DE}}
\FR{\input{patterns/intro_CPU_ISA_FR}}
\PL{\input{patterns/intro_CPU_ISA_PL}}

\EN{\input{patterns/numeral_EN}}
\RU{\input{patterns/numeral_RU}}
\ITA{\input{patterns/numeral_ITA}}
\DE{\input{patterns/numeral_DE}}
\FR{\input{patterns/numeral_FR}}
\PL{\input{patterns/numeral_PL}}

% chapters
\input{patterns/00_empty/main}
\input{patterns/011_ret/main}
\input{patterns/01_helloworld/main}
\input{patterns/015_prolog_epilogue/main}
\input{patterns/02_stack/main}
\input{patterns/03_printf/main}
\input{patterns/04_scanf/main}
\input{patterns/05_passing_arguments/main}
\input{patterns/06_return_results/main}
\input{patterns/061_pointers/main}
\input{patterns/065_GOTO/main}
\input{patterns/07_jcc/main}
\input{patterns/08_switch/main}
\input{patterns/09_loops/main}
\input{patterns/10_strings/main}
\input{patterns/11_arith_optimizations/main}
\input{patterns/12_FPU/main}
\input{patterns/13_arrays/main}
\input{patterns/14_bitfields/main}
\EN{\input{patterns/145_LCG/main_EN}}
\RU{\input{patterns/145_LCG/main_RU}}
\input{patterns/15_structs/main}
\input{patterns/17_unions/main}
\input{patterns/18_pointers_to_functions/main}
\input{patterns/185_64bit_in_32_env/main}

\EN{\input{patterns/19_SIMD/main_EN}}
\RU{\input{patterns/19_SIMD/main_RU}}
\DE{\input{patterns/19_SIMD/main_DE}}

\EN{\input{patterns/20_x64/main_EN}}
\RU{\input{patterns/20_x64/main_RU}}

\EN{\input{patterns/205_floating_SIMD/main_EN}}
\RU{\input{patterns/205_floating_SIMD/main_RU}}
\DE{\input{patterns/205_floating_SIMD/main_DE}}

\EN{\input{patterns/ARM/main_EN}}
\RU{\input{patterns/ARM/main_RU}}
\DE{\input{patterns/ARM/main_DE}}

\input{patterns/MIPS/main}

\ifdefined\SPANISH
\chapter{Patrones de código}
\fi % SPANISH

\ifdefined\GERMAN
\chapter{Code-Muster}
\fi % GERMAN

\ifdefined\ENGLISH
\chapter{Code Patterns}
\fi % ENGLISH

\ifdefined\ITALIAN
\chapter{Forme di codice}
\fi % ITALIAN

\ifdefined\RUSSIAN
\chapter{Образцы кода}
\fi % RUSSIAN

\ifdefined\BRAZILIAN
\chapter{Padrões de códigos}
\fi % BRAZILIAN

\ifdefined\THAI
\chapter{รูปแบบของโค้ด}
\fi % THAI

\ifdefined\FRENCH
\chapter{Modèle de code}
\fi % FRENCH

\ifdefined\POLISH
\chapter{\PLph{}}
\fi % POLISH

% sections
\EN{\input{patterns/patterns_opt_dbg_EN}}
\ES{\input{patterns/patterns_opt_dbg_ES}}
\ITA{\input{patterns/patterns_opt_dbg_ITA}}
\PTBR{\input{patterns/patterns_opt_dbg_PTBR}}
\RU{\input{patterns/patterns_opt_dbg_RU}}
\THA{\input{patterns/patterns_opt_dbg_THA}}
\DE{\input{patterns/patterns_opt_dbg_DE}}
\FR{\input{patterns/patterns_opt_dbg_FR}}
\PL{\input{patterns/patterns_opt_dbg_PL}}

\RU{\section{Некоторые базовые понятия}}
\EN{\section{Some basics}}
\DE{\section{Einige Grundlagen}}
\FR{\section{Quelques bases}}
\ES{\section{\ESph{}}}
\ITA{\section{Alcune basi teoriche}}
\PTBR{\section{\PTBRph{}}}
\THA{\section{\THAph{}}}
\PL{\section{\PLph{}}}

% sections:
\EN{\input{patterns/intro_CPU_ISA_EN}}
\ES{\input{patterns/intro_CPU_ISA_ES}}
\ITA{\input{patterns/intro_CPU_ISA_ITA}}
\PTBR{\input{patterns/intro_CPU_ISA_PTBR}}
\RU{\input{patterns/intro_CPU_ISA_RU}}
\DE{\input{patterns/intro_CPU_ISA_DE}}
\FR{\input{patterns/intro_CPU_ISA_FR}}
\PL{\input{patterns/intro_CPU_ISA_PL}}

\EN{\input{patterns/numeral_EN}}
\RU{\input{patterns/numeral_RU}}
\ITA{\input{patterns/numeral_ITA}}
\DE{\input{patterns/numeral_DE}}
\FR{\input{patterns/numeral_FR}}
\PL{\input{patterns/numeral_PL}}

% chapters
\input{patterns/00_empty/main}
\input{patterns/011_ret/main}
\input{patterns/01_helloworld/main}
\input{patterns/015_prolog_epilogue/main}
\input{patterns/02_stack/main}
\input{patterns/03_printf/main}
\input{patterns/04_scanf/main}
\input{patterns/05_passing_arguments/main}
\input{patterns/06_return_results/main}
\input{patterns/061_pointers/main}
\input{patterns/065_GOTO/main}
\input{patterns/07_jcc/main}
\input{patterns/08_switch/main}
\input{patterns/09_loops/main}
\input{patterns/10_strings/main}
\input{patterns/11_arith_optimizations/main}
\input{patterns/12_FPU/main}
\input{patterns/13_arrays/main}
\input{patterns/14_bitfields/main}
\EN{\input{patterns/145_LCG/main_EN}}
\RU{\input{patterns/145_LCG/main_RU}}
\input{patterns/15_structs/main}
\input{patterns/17_unions/main}
\input{patterns/18_pointers_to_functions/main}
\input{patterns/185_64bit_in_32_env/main}

\EN{\input{patterns/19_SIMD/main_EN}}
\RU{\input{patterns/19_SIMD/main_RU}}
\DE{\input{patterns/19_SIMD/main_DE}}

\EN{\input{patterns/20_x64/main_EN}}
\RU{\input{patterns/20_x64/main_RU}}

\EN{\input{patterns/205_floating_SIMD/main_EN}}
\RU{\input{patterns/205_floating_SIMD/main_RU}}
\DE{\input{patterns/205_floating_SIMD/main_DE}}

\EN{\input{patterns/ARM/main_EN}}
\RU{\input{patterns/ARM/main_RU}}
\DE{\input{patterns/ARM/main_DE}}

\input{patterns/MIPS/main}

\ifdefined\SPANISH
\chapter{Patrones de código}
\fi % SPANISH

\ifdefined\GERMAN
\chapter{Code-Muster}
\fi % GERMAN

\ifdefined\ENGLISH
\chapter{Code Patterns}
\fi % ENGLISH

\ifdefined\ITALIAN
\chapter{Forme di codice}
\fi % ITALIAN

\ifdefined\RUSSIAN
\chapter{Образцы кода}
\fi % RUSSIAN

\ifdefined\BRAZILIAN
\chapter{Padrões de códigos}
\fi % BRAZILIAN

\ifdefined\THAI
\chapter{รูปแบบของโค้ด}
\fi % THAI

\ifdefined\FRENCH
\chapter{Modèle de code}
\fi % FRENCH

\ifdefined\POLISH
\chapter{\PLph{}}
\fi % POLISH

% sections
\EN{\input{patterns/patterns_opt_dbg_EN}}
\ES{\input{patterns/patterns_opt_dbg_ES}}
\ITA{\input{patterns/patterns_opt_dbg_ITA}}
\PTBR{\input{patterns/patterns_opt_dbg_PTBR}}
\RU{\input{patterns/patterns_opt_dbg_RU}}
\THA{\input{patterns/patterns_opt_dbg_THA}}
\DE{\input{patterns/patterns_opt_dbg_DE}}
\FR{\input{patterns/patterns_opt_dbg_FR}}
\PL{\input{patterns/patterns_opt_dbg_PL}}

\RU{\section{Некоторые базовые понятия}}
\EN{\section{Some basics}}
\DE{\section{Einige Grundlagen}}
\FR{\section{Quelques bases}}
\ES{\section{\ESph{}}}
\ITA{\section{Alcune basi teoriche}}
\PTBR{\section{\PTBRph{}}}
\THA{\section{\THAph{}}}
\PL{\section{\PLph{}}}

% sections:
\EN{\input{patterns/intro_CPU_ISA_EN}}
\ES{\input{patterns/intro_CPU_ISA_ES}}
\ITA{\input{patterns/intro_CPU_ISA_ITA}}
\PTBR{\input{patterns/intro_CPU_ISA_PTBR}}
\RU{\input{patterns/intro_CPU_ISA_RU}}
\DE{\input{patterns/intro_CPU_ISA_DE}}
\FR{\input{patterns/intro_CPU_ISA_FR}}
\PL{\input{patterns/intro_CPU_ISA_PL}}

\EN{\input{patterns/numeral_EN}}
\RU{\input{patterns/numeral_RU}}
\ITA{\input{patterns/numeral_ITA}}
\DE{\input{patterns/numeral_DE}}
\FR{\input{patterns/numeral_FR}}
\PL{\input{patterns/numeral_PL}}

% chapters
\input{patterns/00_empty/main}
\input{patterns/011_ret/main}
\input{patterns/01_helloworld/main}
\input{patterns/015_prolog_epilogue/main}
\input{patterns/02_stack/main}
\input{patterns/03_printf/main}
\input{patterns/04_scanf/main}
\input{patterns/05_passing_arguments/main}
\input{patterns/06_return_results/main}
\input{patterns/061_pointers/main}
\input{patterns/065_GOTO/main}
\input{patterns/07_jcc/main}
\input{patterns/08_switch/main}
\input{patterns/09_loops/main}
\input{patterns/10_strings/main}
\input{patterns/11_arith_optimizations/main}
\input{patterns/12_FPU/main}
\input{patterns/13_arrays/main}
\input{patterns/14_bitfields/main}
\EN{\input{patterns/145_LCG/main_EN}}
\RU{\input{patterns/145_LCG/main_RU}}
\input{patterns/15_structs/main}
\input{patterns/17_unions/main}
\input{patterns/18_pointers_to_functions/main}
\input{patterns/185_64bit_in_32_env/main}

\EN{\input{patterns/19_SIMD/main_EN}}
\RU{\input{patterns/19_SIMD/main_RU}}
\DE{\input{patterns/19_SIMD/main_DE}}

\EN{\input{patterns/20_x64/main_EN}}
\RU{\input{patterns/20_x64/main_RU}}

\EN{\input{patterns/205_floating_SIMD/main_EN}}
\RU{\input{patterns/205_floating_SIMD/main_RU}}
\DE{\input{patterns/205_floating_SIMD/main_DE}}

\EN{\input{patterns/ARM/main_EN}}
\RU{\input{patterns/ARM/main_RU}}
\DE{\input{patterns/ARM/main_DE}}

\input{patterns/MIPS/main}

\ifdefined\SPANISH
\chapter{Patrones de código}
\fi % SPANISH

\ifdefined\GERMAN
\chapter{Code-Muster}
\fi % GERMAN

\ifdefined\ENGLISH
\chapter{Code Patterns}
\fi % ENGLISH

\ifdefined\ITALIAN
\chapter{Forme di codice}
\fi % ITALIAN

\ifdefined\RUSSIAN
\chapter{Образцы кода}
\fi % RUSSIAN

\ifdefined\BRAZILIAN
\chapter{Padrões de códigos}
\fi % BRAZILIAN

\ifdefined\THAI
\chapter{รูปแบบของโค้ด}
\fi % THAI

\ifdefined\FRENCH
\chapter{Modèle de code}
\fi % FRENCH

\ifdefined\POLISH
\chapter{\PLph{}}
\fi % POLISH

% sections
\EN{\input{patterns/patterns_opt_dbg_EN}}
\ES{\input{patterns/patterns_opt_dbg_ES}}
\ITA{\input{patterns/patterns_opt_dbg_ITA}}
\PTBR{\input{patterns/patterns_opt_dbg_PTBR}}
\RU{\input{patterns/patterns_opt_dbg_RU}}
\THA{\input{patterns/patterns_opt_dbg_THA}}
\DE{\input{patterns/patterns_opt_dbg_DE}}
\FR{\input{patterns/patterns_opt_dbg_FR}}
\PL{\input{patterns/patterns_opt_dbg_PL}}

\RU{\section{Некоторые базовые понятия}}
\EN{\section{Some basics}}
\DE{\section{Einige Grundlagen}}
\FR{\section{Quelques bases}}
\ES{\section{\ESph{}}}
\ITA{\section{Alcune basi teoriche}}
\PTBR{\section{\PTBRph{}}}
\THA{\section{\THAph{}}}
\PL{\section{\PLph{}}}

% sections:
\EN{\input{patterns/intro_CPU_ISA_EN}}
\ES{\input{patterns/intro_CPU_ISA_ES}}
\ITA{\input{patterns/intro_CPU_ISA_ITA}}
\PTBR{\input{patterns/intro_CPU_ISA_PTBR}}
\RU{\input{patterns/intro_CPU_ISA_RU}}
\DE{\input{patterns/intro_CPU_ISA_DE}}
\FR{\input{patterns/intro_CPU_ISA_FR}}
\PL{\input{patterns/intro_CPU_ISA_PL}}

\EN{\input{patterns/numeral_EN}}
\RU{\input{patterns/numeral_RU}}
\ITA{\input{patterns/numeral_ITA}}
\DE{\input{patterns/numeral_DE}}
\FR{\input{patterns/numeral_FR}}
\PL{\input{patterns/numeral_PL}}

% chapters
\input{patterns/00_empty/main}
\input{patterns/011_ret/main}
\input{patterns/01_helloworld/main}
\input{patterns/015_prolog_epilogue/main}
\input{patterns/02_stack/main}
\input{patterns/03_printf/main}
\input{patterns/04_scanf/main}
\input{patterns/05_passing_arguments/main}
\input{patterns/06_return_results/main}
\input{patterns/061_pointers/main}
\input{patterns/065_GOTO/main}
\input{patterns/07_jcc/main}
\input{patterns/08_switch/main}
\input{patterns/09_loops/main}
\input{patterns/10_strings/main}
\input{patterns/11_arith_optimizations/main}
\input{patterns/12_FPU/main}
\input{patterns/13_arrays/main}
\input{patterns/14_bitfields/main}
\EN{\input{patterns/145_LCG/main_EN}}
\RU{\input{patterns/145_LCG/main_RU}}
\input{patterns/15_structs/main}
\input{patterns/17_unions/main}
\input{patterns/18_pointers_to_functions/main}
\input{patterns/185_64bit_in_32_env/main}

\EN{\input{patterns/19_SIMD/main_EN}}
\RU{\input{patterns/19_SIMD/main_RU}}
\DE{\input{patterns/19_SIMD/main_DE}}

\EN{\input{patterns/20_x64/main_EN}}
\RU{\input{patterns/20_x64/main_RU}}

\EN{\input{patterns/205_floating_SIMD/main_EN}}
\RU{\input{patterns/205_floating_SIMD/main_RU}}
\DE{\input{patterns/205_floating_SIMD/main_DE}}

\EN{\input{patterns/ARM/main_EN}}
\RU{\input{patterns/ARM/main_RU}}
\DE{\input{patterns/ARM/main_DE}}

\input{patterns/MIPS/main}

\ifdefined\SPANISH
\chapter{Patrones de código}
\fi % SPANISH

\ifdefined\GERMAN
\chapter{Code-Muster}
\fi % GERMAN

\ifdefined\ENGLISH
\chapter{Code Patterns}
\fi % ENGLISH

\ifdefined\ITALIAN
\chapter{Forme di codice}
\fi % ITALIAN

\ifdefined\RUSSIAN
\chapter{Образцы кода}
\fi % RUSSIAN

\ifdefined\BRAZILIAN
\chapter{Padrões de códigos}
\fi % BRAZILIAN

\ifdefined\THAI
\chapter{รูปแบบของโค้ด}
\fi % THAI

\ifdefined\FRENCH
\chapter{Modèle de code}
\fi % FRENCH

\ifdefined\POLISH
\chapter{\PLph{}}
\fi % POLISH

% sections
\EN{\input{patterns/patterns_opt_dbg_EN}}
\ES{\input{patterns/patterns_opt_dbg_ES}}
\ITA{\input{patterns/patterns_opt_dbg_ITA}}
\PTBR{\input{patterns/patterns_opt_dbg_PTBR}}
\RU{\input{patterns/patterns_opt_dbg_RU}}
\THA{\input{patterns/patterns_opt_dbg_THA}}
\DE{\input{patterns/patterns_opt_dbg_DE}}
\FR{\input{patterns/patterns_opt_dbg_FR}}
\PL{\input{patterns/patterns_opt_dbg_PL}}

\RU{\section{Некоторые базовые понятия}}
\EN{\section{Some basics}}
\DE{\section{Einige Grundlagen}}
\FR{\section{Quelques bases}}
\ES{\section{\ESph{}}}
\ITA{\section{Alcune basi teoriche}}
\PTBR{\section{\PTBRph{}}}
\THA{\section{\THAph{}}}
\PL{\section{\PLph{}}}

% sections:
\EN{\input{patterns/intro_CPU_ISA_EN}}
\ES{\input{patterns/intro_CPU_ISA_ES}}
\ITA{\input{patterns/intro_CPU_ISA_ITA}}
\PTBR{\input{patterns/intro_CPU_ISA_PTBR}}
\RU{\input{patterns/intro_CPU_ISA_RU}}
\DE{\input{patterns/intro_CPU_ISA_DE}}
\FR{\input{patterns/intro_CPU_ISA_FR}}
\PL{\input{patterns/intro_CPU_ISA_PL}}

\EN{\input{patterns/numeral_EN}}
\RU{\input{patterns/numeral_RU}}
\ITA{\input{patterns/numeral_ITA}}
\DE{\input{patterns/numeral_DE}}
\FR{\input{patterns/numeral_FR}}
\PL{\input{patterns/numeral_PL}}

% chapters
\input{patterns/00_empty/main}
\input{patterns/011_ret/main}
\input{patterns/01_helloworld/main}
\input{patterns/015_prolog_epilogue/main}
\input{patterns/02_stack/main}
\input{patterns/03_printf/main}
\input{patterns/04_scanf/main}
\input{patterns/05_passing_arguments/main}
\input{patterns/06_return_results/main}
\input{patterns/061_pointers/main}
\input{patterns/065_GOTO/main}
\input{patterns/07_jcc/main}
\input{patterns/08_switch/main}
\input{patterns/09_loops/main}
\input{patterns/10_strings/main}
\input{patterns/11_arith_optimizations/main}
\input{patterns/12_FPU/main}
\input{patterns/13_arrays/main}
\input{patterns/14_bitfields/main}
\EN{\input{patterns/145_LCG/main_EN}}
\RU{\input{patterns/145_LCG/main_RU}}
\input{patterns/15_structs/main}
\input{patterns/17_unions/main}
\input{patterns/18_pointers_to_functions/main}
\input{patterns/185_64bit_in_32_env/main}

\EN{\input{patterns/19_SIMD/main_EN}}
\RU{\input{patterns/19_SIMD/main_RU}}
\DE{\input{patterns/19_SIMD/main_DE}}

\EN{\input{patterns/20_x64/main_EN}}
\RU{\input{patterns/20_x64/main_RU}}

\EN{\input{patterns/205_floating_SIMD/main_EN}}
\RU{\input{patterns/205_floating_SIMD/main_RU}}
\DE{\input{patterns/205_floating_SIMD/main_DE}}

\EN{\input{patterns/ARM/main_EN}}
\RU{\input{patterns/ARM/main_RU}}
\DE{\input{patterns/ARM/main_DE}}

\input{patterns/MIPS/main}

\ifdefined\SPANISH
\chapter{Patrones de código}
\fi % SPANISH

\ifdefined\GERMAN
\chapter{Code-Muster}
\fi % GERMAN

\ifdefined\ENGLISH
\chapter{Code Patterns}
\fi % ENGLISH

\ifdefined\ITALIAN
\chapter{Forme di codice}
\fi % ITALIAN

\ifdefined\RUSSIAN
\chapter{Образцы кода}
\fi % RUSSIAN

\ifdefined\BRAZILIAN
\chapter{Padrões de códigos}
\fi % BRAZILIAN

\ifdefined\THAI
\chapter{รูปแบบของโค้ด}
\fi % THAI

\ifdefined\FRENCH
\chapter{Modèle de code}
\fi % FRENCH

\ifdefined\POLISH
\chapter{\PLph{}}
\fi % POLISH

% sections
\EN{\input{patterns/patterns_opt_dbg_EN}}
\ES{\input{patterns/patterns_opt_dbg_ES}}
\ITA{\input{patterns/patterns_opt_dbg_ITA}}
\PTBR{\input{patterns/patterns_opt_dbg_PTBR}}
\RU{\input{patterns/patterns_opt_dbg_RU}}
\THA{\input{patterns/patterns_opt_dbg_THA}}
\DE{\input{patterns/patterns_opt_dbg_DE}}
\FR{\input{patterns/patterns_opt_dbg_FR}}
\PL{\input{patterns/patterns_opt_dbg_PL}}

\RU{\section{Некоторые базовые понятия}}
\EN{\section{Some basics}}
\DE{\section{Einige Grundlagen}}
\FR{\section{Quelques bases}}
\ES{\section{\ESph{}}}
\ITA{\section{Alcune basi teoriche}}
\PTBR{\section{\PTBRph{}}}
\THA{\section{\THAph{}}}
\PL{\section{\PLph{}}}

% sections:
\EN{\input{patterns/intro_CPU_ISA_EN}}
\ES{\input{patterns/intro_CPU_ISA_ES}}
\ITA{\input{patterns/intro_CPU_ISA_ITA}}
\PTBR{\input{patterns/intro_CPU_ISA_PTBR}}
\RU{\input{patterns/intro_CPU_ISA_RU}}
\DE{\input{patterns/intro_CPU_ISA_DE}}
\FR{\input{patterns/intro_CPU_ISA_FR}}
\PL{\input{patterns/intro_CPU_ISA_PL}}

\EN{\input{patterns/numeral_EN}}
\RU{\input{patterns/numeral_RU}}
\ITA{\input{patterns/numeral_ITA}}
\DE{\input{patterns/numeral_DE}}
\FR{\input{patterns/numeral_FR}}
\PL{\input{patterns/numeral_PL}}

% chapters
\input{patterns/00_empty/main}
\input{patterns/011_ret/main}
\input{patterns/01_helloworld/main}
\input{patterns/015_prolog_epilogue/main}
\input{patterns/02_stack/main}
\input{patterns/03_printf/main}
\input{patterns/04_scanf/main}
\input{patterns/05_passing_arguments/main}
\input{patterns/06_return_results/main}
\input{patterns/061_pointers/main}
\input{patterns/065_GOTO/main}
\input{patterns/07_jcc/main}
\input{patterns/08_switch/main}
\input{patterns/09_loops/main}
\input{patterns/10_strings/main}
\input{patterns/11_arith_optimizations/main}
\input{patterns/12_FPU/main}
\input{patterns/13_arrays/main}
\input{patterns/14_bitfields/main}
\EN{\input{patterns/145_LCG/main_EN}}
\RU{\input{patterns/145_LCG/main_RU}}
\input{patterns/15_structs/main}
\input{patterns/17_unions/main}
\input{patterns/18_pointers_to_functions/main}
\input{patterns/185_64bit_in_32_env/main}

\EN{\input{patterns/19_SIMD/main_EN}}
\RU{\input{patterns/19_SIMD/main_RU}}
\DE{\input{patterns/19_SIMD/main_DE}}

\EN{\input{patterns/20_x64/main_EN}}
\RU{\input{patterns/20_x64/main_RU}}

\EN{\input{patterns/205_floating_SIMD/main_EN}}
\RU{\input{patterns/205_floating_SIMD/main_RU}}
\DE{\input{patterns/205_floating_SIMD/main_DE}}

\EN{\input{patterns/ARM/main_EN}}
\RU{\input{patterns/ARM/main_RU}}
\DE{\input{patterns/ARM/main_DE}}

\input{patterns/MIPS/main}

\ifdefined\SPANISH
\chapter{Patrones de código}
\fi % SPANISH

\ifdefined\GERMAN
\chapter{Code-Muster}
\fi % GERMAN

\ifdefined\ENGLISH
\chapter{Code Patterns}
\fi % ENGLISH

\ifdefined\ITALIAN
\chapter{Forme di codice}
\fi % ITALIAN

\ifdefined\RUSSIAN
\chapter{Образцы кода}
\fi % RUSSIAN

\ifdefined\BRAZILIAN
\chapter{Padrões de códigos}
\fi % BRAZILIAN

\ifdefined\THAI
\chapter{รูปแบบของโค้ด}
\fi % THAI

\ifdefined\FRENCH
\chapter{Modèle de code}
\fi % FRENCH

\ifdefined\POLISH
\chapter{\PLph{}}
\fi % POLISH

% sections
\EN{\input{patterns/patterns_opt_dbg_EN}}
\ES{\input{patterns/patterns_opt_dbg_ES}}
\ITA{\input{patterns/patterns_opt_dbg_ITA}}
\PTBR{\input{patterns/patterns_opt_dbg_PTBR}}
\RU{\input{patterns/patterns_opt_dbg_RU}}
\THA{\input{patterns/patterns_opt_dbg_THA}}
\DE{\input{patterns/patterns_opt_dbg_DE}}
\FR{\input{patterns/patterns_opt_dbg_FR}}
\PL{\input{patterns/patterns_opt_dbg_PL}}

\RU{\section{Некоторые базовые понятия}}
\EN{\section{Some basics}}
\DE{\section{Einige Grundlagen}}
\FR{\section{Quelques bases}}
\ES{\section{\ESph{}}}
\ITA{\section{Alcune basi teoriche}}
\PTBR{\section{\PTBRph{}}}
\THA{\section{\THAph{}}}
\PL{\section{\PLph{}}}

% sections:
\EN{\input{patterns/intro_CPU_ISA_EN}}
\ES{\input{patterns/intro_CPU_ISA_ES}}
\ITA{\input{patterns/intro_CPU_ISA_ITA}}
\PTBR{\input{patterns/intro_CPU_ISA_PTBR}}
\RU{\input{patterns/intro_CPU_ISA_RU}}
\DE{\input{patterns/intro_CPU_ISA_DE}}
\FR{\input{patterns/intro_CPU_ISA_FR}}
\PL{\input{patterns/intro_CPU_ISA_PL}}

\EN{\input{patterns/numeral_EN}}
\RU{\input{patterns/numeral_RU}}
\ITA{\input{patterns/numeral_ITA}}
\DE{\input{patterns/numeral_DE}}
\FR{\input{patterns/numeral_FR}}
\PL{\input{patterns/numeral_PL}}

% chapters
\input{patterns/00_empty/main}
\input{patterns/011_ret/main}
\input{patterns/01_helloworld/main}
\input{patterns/015_prolog_epilogue/main}
\input{patterns/02_stack/main}
\input{patterns/03_printf/main}
\input{patterns/04_scanf/main}
\input{patterns/05_passing_arguments/main}
\input{patterns/06_return_results/main}
\input{patterns/061_pointers/main}
\input{patterns/065_GOTO/main}
\input{patterns/07_jcc/main}
\input{patterns/08_switch/main}
\input{patterns/09_loops/main}
\input{patterns/10_strings/main}
\input{patterns/11_arith_optimizations/main}
\input{patterns/12_FPU/main}
\input{patterns/13_arrays/main}
\input{patterns/14_bitfields/main}
\EN{\input{patterns/145_LCG/main_EN}}
\RU{\input{patterns/145_LCG/main_RU}}
\input{patterns/15_structs/main}
\input{patterns/17_unions/main}
\input{patterns/18_pointers_to_functions/main}
\input{patterns/185_64bit_in_32_env/main}

\EN{\input{patterns/19_SIMD/main_EN}}
\RU{\input{patterns/19_SIMD/main_RU}}
\DE{\input{patterns/19_SIMD/main_DE}}

\EN{\input{patterns/20_x64/main_EN}}
\RU{\input{patterns/20_x64/main_RU}}

\EN{\input{patterns/205_floating_SIMD/main_EN}}
\RU{\input{patterns/205_floating_SIMD/main_RU}}
\DE{\input{patterns/205_floating_SIMD/main_DE}}

\EN{\input{patterns/ARM/main_EN}}
\RU{\input{patterns/ARM/main_RU}}
\DE{\input{patterns/ARM/main_DE}}

\input{patterns/MIPS/main}

\ifdefined\SPANISH
\chapter{Patrones de código}
\fi % SPANISH

\ifdefined\GERMAN
\chapter{Code-Muster}
\fi % GERMAN

\ifdefined\ENGLISH
\chapter{Code Patterns}
\fi % ENGLISH

\ifdefined\ITALIAN
\chapter{Forme di codice}
\fi % ITALIAN

\ifdefined\RUSSIAN
\chapter{Образцы кода}
\fi % RUSSIAN

\ifdefined\BRAZILIAN
\chapter{Padrões de códigos}
\fi % BRAZILIAN

\ifdefined\THAI
\chapter{รูปแบบของโค้ด}
\fi % THAI

\ifdefined\FRENCH
\chapter{Modèle de code}
\fi % FRENCH

\ifdefined\POLISH
\chapter{\PLph{}}
\fi % POLISH

% sections
\EN{\input{patterns/patterns_opt_dbg_EN}}
\ES{\input{patterns/patterns_opt_dbg_ES}}
\ITA{\input{patterns/patterns_opt_dbg_ITA}}
\PTBR{\input{patterns/patterns_opt_dbg_PTBR}}
\RU{\input{patterns/patterns_opt_dbg_RU}}
\THA{\input{patterns/patterns_opt_dbg_THA}}
\DE{\input{patterns/patterns_opt_dbg_DE}}
\FR{\input{patterns/patterns_opt_dbg_FR}}
\PL{\input{patterns/patterns_opt_dbg_PL}}

\RU{\section{Некоторые базовые понятия}}
\EN{\section{Some basics}}
\DE{\section{Einige Grundlagen}}
\FR{\section{Quelques bases}}
\ES{\section{\ESph{}}}
\ITA{\section{Alcune basi teoriche}}
\PTBR{\section{\PTBRph{}}}
\THA{\section{\THAph{}}}
\PL{\section{\PLph{}}}

% sections:
\EN{\input{patterns/intro_CPU_ISA_EN}}
\ES{\input{patterns/intro_CPU_ISA_ES}}
\ITA{\input{patterns/intro_CPU_ISA_ITA}}
\PTBR{\input{patterns/intro_CPU_ISA_PTBR}}
\RU{\input{patterns/intro_CPU_ISA_RU}}
\DE{\input{patterns/intro_CPU_ISA_DE}}
\FR{\input{patterns/intro_CPU_ISA_FR}}
\PL{\input{patterns/intro_CPU_ISA_PL}}

\EN{\input{patterns/numeral_EN}}
\RU{\input{patterns/numeral_RU}}
\ITA{\input{patterns/numeral_ITA}}
\DE{\input{patterns/numeral_DE}}
\FR{\input{patterns/numeral_FR}}
\PL{\input{patterns/numeral_PL}}

% chapters
\input{patterns/00_empty/main}
\input{patterns/011_ret/main}
\input{patterns/01_helloworld/main}
\input{patterns/015_prolog_epilogue/main}
\input{patterns/02_stack/main}
\input{patterns/03_printf/main}
\input{patterns/04_scanf/main}
\input{patterns/05_passing_arguments/main}
\input{patterns/06_return_results/main}
\input{patterns/061_pointers/main}
\input{patterns/065_GOTO/main}
\input{patterns/07_jcc/main}
\input{patterns/08_switch/main}
\input{patterns/09_loops/main}
\input{patterns/10_strings/main}
\input{patterns/11_arith_optimizations/main}
\input{patterns/12_FPU/main}
\input{patterns/13_arrays/main}
\input{patterns/14_bitfields/main}
\EN{\input{patterns/145_LCG/main_EN}}
\RU{\input{patterns/145_LCG/main_RU}}
\input{patterns/15_structs/main}
\input{patterns/17_unions/main}
\input{patterns/18_pointers_to_functions/main}
\input{patterns/185_64bit_in_32_env/main}

\EN{\input{patterns/19_SIMD/main_EN}}
\RU{\input{patterns/19_SIMD/main_RU}}
\DE{\input{patterns/19_SIMD/main_DE}}

\EN{\input{patterns/20_x64/main_EN}}
\RU{\input{patterns/20_x64/main_RU}}

\EN{\input{patterns/205_floating_SIMD/main_EN}}
\RU{\input{patterns/205_floating_SIMD/main_RU}}
\DE{\input{patterns/205_floating_SIMD/main_DE}}

\EN{\input{patterns/ARM/main_EN}}
\RU{\input{patterns/ARM/main_RU}}
\DE{\input{patterns/ARM/main_DE}}

\input{patterns/MIPS/main}

\ifdefined\SPANISH
\chapter{Patrones de código}
\fi % SPANISH

\ifdefined\GERMAN
\chapter{Code-Muster}
\fi % GERMAN

\ifdefined\ENGLISH
\chapter{Code Patterns}
\fi % ENGLISH

\ifdefined\ITALIAN
\chapter{Forme di codice}
\fi % ITALIAN

\ifdefined\RUSSIAN
\chapter{Образцы кода}
\fi % RUSSIAN

\ifdefined\BRAZILIAN
\chapter{Padrões de códigos}
\fi % BRAZILIAN

\ifdefined\THAI
\chapter{รูปแบบของโค้ด}
\fi % THAI

\ifdefined\FRENCH
\chapter{Modèle de code}
\fi % FRENCH

\ifdefined\POLISH
\chapter{\PLph{}}
\fi % POLISH

% sections
\EN{\input{patterns/patterns_opt_dbg_EN}}
\ES{\input{patterns/patterns_opt_dbg_ES}}
\ITA{\input{patterns/patterns_opt_dbg_ITA}}
\PTBR{\input{patterns/patterns_opt_dbg_PTBR}}
\RU{\input{patterns/patterns_opt_dbg_RU}}
\THA{\input{patterns/patterns_opt_dbg_THA}}
\DE{\input{patterns/patterns_opt_dbg_DE}}
\FR{\input{patterns/patterns_opt_dbg_FR}}
\PL{\input{patterns/patterns_opt_dbg_PL}}

\RU{\section{Некоторые базовые понятия}}
\EN{\section{Some basics}}
\DE{\section{Einige Grundlagen}}
\FR{\section{Quelques bases}}
\ES{\section{\ESph{}}}
\ITA{\section{Alcune basi teoriche}}
\PTBR{\section{\PTBRph{}}}
\THA{\section{\THAph{}}}
\PL{\section{\PLph{}}}

% sections:
\EN{\input{patterns/intro_CPU_ISA_EN}}
\ES{\input{patterns/intro_CPU_ISA_ES}}
\ITA{\input{patterns/intro_CPU_ISA_ITA}}
\PTBR{\input{patterns/intro_CPU_ISA_PTBR}}
\RU{\input{patterns/intro_CPU_ISA_RU}}
\DE{\input{patterns/intro_CPU_ISA_DE}}
\FR{\input{patterns/intro_CPU_ISA_FR}}
\PL{\input{patterns/intro_CPU_ISA_PL}}

\EN{\input{patterns/numeral_EN}}
\RU{\input{patterns/numeral_RU}}
\ITA{\input{patterns/numeral_ITA}}
\DE{\input{patterns/numeral_DE}}
\FR{\input{patterns/numeral_FR}}
\PL{\input{patterns/numeral_PL}}

% chapters
\input{patterns/00_empty/main}
\input{patterns/011_ret/main}
\input{patterns/01_helloworld/main}
\input{patterns/015_prolog_epilogue/main}
\input{patterns/02_stack/main}
\input{patterns/03_printf/main}
\input{patterns/04_scanf/main}
\input{patterns/05_passing_arguments/main}
\input{patterns/06_return_results/main}
\input{patterns/061_pointers/main}
\input{patterns/065_GOTO/main}
\input{patterns/07_jcc/main}
\input{patterns/08_switch/main}
\input{patterns/09_loops/main}
\input{patterns/10_strings/main}
\input{patterns/11_arith_optimizations/main}
\input{patterns/12_FPU/main}
\input{patterns/13_arrays/main}
\input{patterns/14_bitfields/main}
\EN{\input{patterns/145_LCG/main_EN}}
\RU{\input{patterns/145_LCG/main_RU}}
\input{patterns/15_structs/main}
\input{patterns/17_unions/main}
\input{patterns/18_pointers_to_functions/main}
\input{patterns/185_64bit_in_32_env/main}

\EN{\input{patterns/19_SIMD/main_EN}}
\RU{\input{patterns/19_SIMD/main_RU}}
\DE{\input{patterns/19_SIMD/main_DE}}

\EN{\input{patterns/20_x64/main_EN}}
\RU{\input{patterns/20_x64/main_RU}}

\EN{\input{patterns/205_floating_SIMD/main_EN}}
\RU{\input{patterns/205_floating_SIMD/main_RU}}
\DE{\input{patterns/205_floating_SIMD/main_DE}}

\EN{\input{patterns/ARM/main_EN}}
\RU{\input{patterns/ARM/main_RU}}
\DE{\input{patterns/ARM/main_DE}}

\input{patterns/MIPS/main}

\ifdefined\SPANISH
\chapter{Patrones de código}
\fi % SPANISH

\ifdefined\GERMAN
\chapter{Code-Muster}
\fi % GERMAN

\ifdefined\ENGLISH
\chapter{Code Patterns}
\fi % ENGLISH

\ifdefined\ITALIAN
\chapter{Forme di codice}
\fi % ITALIAN

\ifdefined\RUSSIAN
\chapter{Образцы кода}
\fi % RUSSIAN

\ifdefined\BRAZILIAN
\chapter{Padrões de códigos}
\fi % BRAZILIAN

\ifdefined\THAI
\chapter{รูปแบบของโค้ด}
\fi % THAI

\ifdefined\FRENCH
\chapter{Modèle de code}
\fi % FRENCH

\ifdefined\POLISH
\chapter{\PLph{}}
\fi % POLISH

% sections
\EN{\input{patterns/patterns_opt_dbg_EN}}
\ES{\input{patterns/patterns_opt_dbg_ES}}
\ITA{\input{patterns/patterns_opt_dbg_ITA}}
\PTBR{\input{patterns/patterns_opt_dbg_PTBR}}
\RU{\input{patterns/patterns_opt_dbg_RU}}
\THA{\input{patterns/patterns_opt_dbg_THA}}
\DE{\input{patterns/patterns_opt_dbg_DE}}
\FR{\input{patterns/patterns_opt_dbg_FR}}
\PL{\input{patterns/patterns_opt_dbg_PL}}

\RU{\section{Некоторые базовые понятия}}
\EN{\section{Some basics}}
\DE{\section{Einige Grundlagen}}
\FR{\section{Quelques bases}}
\ES{\section{\ESph{}}}
\ITA{\section{Alcune basi teoriche}}
\PTBR{\section{\PTBRph{}}}
\THA{\section{\THAph{}}}
\PL{\section{\PLph{}}}

% sections:
\EN{\input{patterns/intro_CPU_ISA_EN}}
\ES{\input{patterns/intro_CPU_ISA_ES}}
\ITA{\input{patterns/intro_CPU_ISA_ITA}}
\PTBR{\input{patterns/intro_CPU_ISA_PTBR}}
\RU{\input{patterns/intro_CPU_ISA_RU}}
\DE{\input{patterns/intro_CPU_ISA_DE}}
\FR{\input{patterns/intro_CPU_ISA_FR}}
\PL{\input{patterns/intro_CPU_ISA_PL}}

\EN{\input{patterns/numeral_EN}}
\RU{\input{patterns/numeral_RU}}
\ITA{\input{patterns/numeral_ITA}}
\DE{\input{patterns/numeral_DE}}
\FR{\input{patterns/numeral_FR}}
\PL{\input{patterns/numeral_PL}}

% chapters
\input{patterns/00_empty/main}
\input{patterns/011_ret/main}
\input{patterns/01_helloworld/main}
\input{patterns/015_prolog_epilogue/main}
\input{patterns/02_stack/main}
\input{patterns/03_printf/main}
\input{patterns/04_scanf/main}
\input{patterns/05_passing_arguments/main}
\input{patterns/06_return_results/main}
\input{patterns/061_pointers/main}
\input{patterns/065_GOTO/main}
\input{patterns/07_jcc/main}
\input{patterns/08_switch/main}
\input{patterns/09_loops/main}
\input{patterns/10_strings/main}
\input{patterns/11_arith_optimizations/main}
\input{patterns/12_FPU/main}
\input{patterns/13_arrays/main}
\input{patterns/14_bitfields/main}
\EN{\input{patterns/145_LCG/main_EN}}
\RU{\input{patterns/145_LCG/main_RU}}
\input{patterns/15_structs/main}
\input{patterns/17_unions/main}
\input{patterns/18_pointers_to_functions/main}
\input{patterns/185_64bit_in_32_env/main}

\EN{\input{patterns/19_SIMD/main_EN}}
\RU{\input{patterns/19_SIMD/main_RU}}
\DE{\input{patterns/19_SIMD/main_DE}}

\EN{\input{patterns/20_x64/main_EN}}
\RU{\input{patterns/20_x64/main_RU}}

\EN{\input{patterns/205_floating_SIMD/main_EN}}
\RU{\input{patterns/205_floating_SIMD/main_RU}}
\DE{\input{patterns/205_floating_SIMD/main_DE}}

\EN{\input{patterns/ARM/main_EN}}
\RU{\input{patterns/ARM/main_RU}}
\DE{\input{patterns/ARM/main_DE}}

\input{patterns/MIPS/main}

\ifdefined\SPANISH
\chapter{Patrones de código}
\fi % SPANISH

\ifdefined\GERMAN
\chapter{Code-Muster}
\fi % GERMAN

\ifdefined\ENGLISH
\chapter{Code Patterns}
\fi % ENGLISH

\ifdefined\ITALIAN
\chapter{Forme di codice}
\fi % ITALIAN

\ifdefined\RUSSIAN
\chapter{Образцы кода}
\fi % RUSSIAN

\ifdefined\BRAZILIAN
\chapter{Padrões de códigos}
\fi % BRAZILIAN

\ifdefined\THAI
\chapter{รูปแบบของโค้ด}
\fi % THAI

\ifdefined\FRENCH
\chapter{Modèle de code}
\fi % FRENCH

\ifdefined\POLISH
\chapter{\PLph{}}
\fi % POLISH

% sections
\EN{\input{patterns/patterns_opt_dbg_EN}}
\ES{\input{patterns/patterns_opt_dbg_ES}}
\ITA{\input{patterns/patterns_opt_dbg_ITA}}
\PTBR{\input{patterns/patterns_opt_dbg_PTBR}}
\RU{\input{patterns/patterns_opt_dbg_RU}}
\THA{\input{patterns/patterns_opt_dbg_THA}}
\DE{\input{patterns/patterns_opt_dbg_DE}}
\FR{\input{patterns/patterns_opt_dbg_FR}}
\PL{\input{patterns/patterns_opt_dbg_PL}}

\RU{\section{Некоторые базовые понятия}}
\EN{\section{Some basics}}
\DE{\section{Einige Grundlagen}}
\FR{\section{Quelques bases}}
\ES{\section{\ESph{}}}
\ITA{\section{Alcune basi teoriche}}
\PTBR{\section{\PTBRph{}}}
\THA{\section{\THAph{}}}
\PL{\section{\PLph{}}}

% sections:
\EN{\input{patterns/intro_CPU_ISA_EN}}
\ES{\input{patterns/intro_CPU_ISA_ES}}
\ITA{\input{patterns/intro_CPU_ISA_ITA}}
\PTBR{\input{patterns/intro_CPU_ISA_PTBR}}
\RU{\input{patterns/intro_CPU_ISA_RU}}
\DE{\input{patterns/intro_CPU_ISA_DE}}
\FR{\input{patterns/intro_CPU_ISA_FR}}
\PL{\input{patterns/intro_CPU_ISA_PL}}

\EN{\input{patterns/numeral_EN}}
\RU{\input{patterns/numeral_RU}}
\ITA{\input{patterns/numeral_ITA}}
\DE{\input{patterns/numeral_DE}}
\FR{\input{patterns/numeral_FR}}
\PL{\input{patterns/numeral_PL}}

% chapters
\input{patterns/00_empty/main}
\input{patterns/011_ret/main}
\input{patterns/01_helloworld/main}
\input{patterns/015_prolog_epilogue/main}
\input{patterns/02_stack/main}
\input{patterns/03_printf/main}
\input{patterns/04_scanf/main}
\input{patterns/05_passing_arguments/main}
\input{patterns/06_return_results/main}
\input{patterns/061_pointers/main}
\input{patterns/065_GOTO/main}
\input{patterns/07_jcc/main}
\input{patterns/08_switch/main}
\input{patterns/09_loops/main}
\input{patterns/10_strings/main}
\input{patterns/11_arith_optimizations/main}
\input{patterns/12_FPU/main}
\input{patterns/13_arrays/main}
\input{patterns/14_bitfields/main}
\EN{\input{patterns/145_LCG/main_EN}}
\RU{\input{patterns/145_LCG/main_RU}}
\input{patterns/15_structs/main}
\input{patterns/17_unions/main}
\input{patterns/18_pointers_to_functions/main}
\input{patterns/185_64bit_in_32_env/main}

\EN{\input{patterns/19_SIMD/main_EN}}
\RU{\input{patterns/19_SIMD/main_RU}}
\DE{\input{patterns/19_SIMD/main_DE}}

\EN{\input{patterns/20_x64/main_EN}}
\RU{\input{patterns/20_x64/main_RU}}

\EN{\input{patterns/205_floating_SIMD/main_EN}}
\RU{\input{patterns/205_floating_SIMD/main_RU}}
\DE{\input{patterns/205_floating_SIMD/main_DE}}

\EN{\input{patterns/ARM/main_EN}}
\RU{\input{patterns/ARM/main_RU}}
\DE{\input{patterns/ARM/main_DE}}

\input{patterns/MIPS/main}

\ifdefined\SPANISH
\chapter{Patrones de código}
\fi % SPANISH

\ifdefined\GERMAN
\chapter{Code-Muster}
\fi % GERMAN

\ifdefined\ENGLISH
\chapter{Code Patterns}
\fi % ENGLISH

\ifdefined\ITALIAN
\chapter{Forme di codice}
\fi % ITALIAN

\ifdefined\RUSSIAN
\chapter{Образцы кода}
\fi % RUSSIAN

\ifdefined\BRAZILIAN
\chapter{Padrões de códigos}
\fi % BRAZILIAN

\ifdefined\THAI
\chapter{รูปแบบของโค้ด}
\fi % THAI

\ifdefined\FRENCH
\chapter{Modèle de code}
\fi % FRENCH

\ifdefined\POLISH
\chapter{\PLph{}}
\fi % POLISH

% sections
\EN{\input{patterns/patterns_opt_dbg_EN}}
\ES{\input{patterns/patterns_opt_dbg_ES}}
\ITA{\input{patterns/patterns_opt_dbg_ITA}}
\PTBR{\input{patterns/patterns_opt_dbg_PTBR}}
\RU{\input{patterns/patterns_opt_dbg_RU}}
\THA{\input{patterns/patterns_opt_dbg_THA}}
\DE{\input{patterns/patterns_opt_dbg_DE}}
\FR{\input{patterns/patterns_opt_dbg_FR}}
\PL{\input{patterns/patterns_opt_dbg_PL}}

\RU{\section{Некоторые базовые понятия}}
\EN{\section{Some basics}}
\DE{\section{Einige Grundlagen}}
\FR{\section{Quelques bases}}
\ES{\section{\ESph{}}}
\ITA{\section{Alcune basi teoriche}}
\PTBR{\section{\PTBRph{}}}
\THA{\section{\THAph{}}}
\PL{\section{\PLph{}}}

% sections:
\EN{\input{patterns/intro_CPU_ISA_EN}}
\ES{\input{patterns/intro_CPU_ISA_ES}}
\ITA{\input{patterns/intro_CPU_ISA_ITA}}
\PTBR{\input{patterns/intro_CPU_ISA_PTBR}}
\RU{\input{patterns/intro_CPU_ISA_RU}}
\DE{\input{patterns/intro_CPU_ISA_DE}}
\FR{\input{patterns/intro_CPU_ISA_FR}}
\PL{\input{patterns/intro_CPU_ISA_PL}}

\EN{\input{patterns/numeral_EN}}
\RU{\input{patterns/numeral_RU}}
\ITA{\input{patterns/numeral_ITA}}
\DE{\input{patterns/numeral_DE}}
\FR{\input{patterns/numeral_FR}}
\PL{\input{patterns/numeral_PL}}

% chapters
\input{patterns/00_empty/main}
\input{patterns/011_ret/main}
\input{patterns/01_helloworld/main}
\input{patterns/015_prolog_epilogue/main}
\input{patterns/02_stack/main}
\input{patterns/03_printf/main}
\input{patterns/04_scanf/main}
\input{patterns/05_passing_arguments/main}
\input{patterns/06_return_results/main}
\input{patterns/061_pointers/main}
\input{patterns/065_GOTO/main}
\input{patterns/07_jcc/main}
\input{patterns/08_switch/main}
\input{patterns/09_loops/main}
\input{patterns/10_strings/main}
\input{patterns/11_arith_optimizations/main}
\input{patterns/12_FPU/main}
\input{patterns/13_arrays/main}
\input{patterns/14_bitfields/main}
\EN{\input{patterns/145_LCG/main_EN}}
\RU{\input{patterns/145_LCG/main_RU}}
\input{patterns/15_structs/main}
\input{patterns/17_unions/main}
\input{patterns/18_pointers_to_functions/main}
\input{patterns/185_64bit_in_32_env/main}

\EN{\input{patterns/19_SIMD/main_EN}}
\RU{\input{patterns/19_SIMD/main_RU}}
\DE{\input{patterns/19_SIMD/main_DE}}

\EN{\input{patterns/20_x64/main_EN}}
\RU{\input{patterns/20_x64/main_RU}}

\EN{\input{patterns/205_floating_SIMD/main_EN}}
\RU{\input{patterns/205_floating_SIMD/main_RU}}
\DE{\input{patterns/205_floating_SIMD/main_DE}}

\EN{\input{patterns/ARM/main_EN}}
\RU{\input{patterns/ARM/main_RU}}
\DE{\input{patterns/ARM/main_DE}}

\input{patterns/MIPS/main}

\EN{\input{patterns/12_FPU/main_EN}}
\RU{\input{patterns/12_FPU/main_RU}}
\DE{\input{patterns/12_FPU/main_DE}}
\FR{\input{patterns/12_FPU/main_FR}}


\ifdefined\SPANISH
\chapter{Patrones de código}
\fi % SPANISH

\ifdefined\GERMAN
\chapter{Code-Muster}
\fi % GERMAN

\ifdefined\ENGLISH
\chapter{Code Patterns}
\fi % ENGLISH

\ifdefined\ITALIAN
\chapter{Forme di codice}
\fi % ITALIAN

\ifdefined\RUSSIAN
\chapter{Образцы кода}
\fi % RUSSIAN

\ifdefined\BRAZILIAN
\chapter{Padrões de códigos}
\fi % BRAZILIAN

\ifdefined\THAI
\chapter{รูปแบบของโค้ด}
\fi % THAI

\ifdefined\FRENCH
\chapter{Modèle de code}
\fi % FRENCH

\ifdefined\POLISH
\chapter{\PLph{}}
\fi % POLISH

% sections
\EN{\input{patterns/patterns_opt_dbg_EN}}
\ES{\input{patterns/patterns_opt_dbg_ES}}
\ITA{\input{patterns/patterns_opt_dbg_ITA}}
\PTBR{\input{patterns/patterns_opt_dbg_PTBR}}
\RU{\input{patterns/patterns_opt_dbg_RU}}
\THA{\input{patterns/patterns_opt_dbg_THA}}
\DE{\input{patterns/patterns_opt_dbg_DE}}
\FR{\input{patterns/patterns_opt_dbg_FR}}
\PL{\input{patterns/patterns_opt_dbg_PL}}

\RU{\section{Некоторые базовые понятия}}
\EN{\section{Some basics}}
\DE{\section{Einige Grundlagen}}
\FR{\section{Quelques bases}}
\ES{\section{\ESph{}}}
\ITA{\section{Alcune basi teoriche}}
\PTBR{\section{\PTBRph{}}}
\THA{\section{\THAph{}}}
\PL{\section{\PLph{}}}

% sections:
\EN{\input{patterns/intro_CPU_ISA_EN}}
\ES{\input{patterns/intro_CPU_ISA_ES}}
\ITA{\input{patterns/intro_CPU_ISA_ITA}}
\PTBR{\input{patterns/intro_CPU_ISA_PTBR}}
\RU{\input{patterns/intro_CPU_ISA_RU}}
\DE{\input{patterns/intro_CPU_ISA_DE}}
\FR{\input{patterns/intro_CPU_ISA_FR}}
\PL{\input{patterns/intro_CPU_ISA_PL}}

\EN{\input{patterns/numeral_EN}}
\RU{\input{patterns/numeral_RU}}
\ITA{\input{patterns/numeral_ITA}}
\DE{\input{patterns/numeral_DE}}
\FR{\input{patterns/numeral_FR}}
\PL{\input{patterns/numeral_PL}}

% chapters
\input{patterns/00_empty/main}
\input{patterns/011_ret/main}
\input{patterns/01_helloworld/main}
\input{patterns/015_prolog_epilogue/main}
\input{patterns/02_stack/main}
\input{patterns/03_printf/main}
\input{patterns/04_scanf/main}
\input{patterns/05_passing_arguments/main}
\input{patterns/06_return_results/main}
\input{patterns/061_pointers/main}
\input{patterns/065_GOTO/main}
\input{patterns/07_jcc/main}
\input{patterns/08_switch/main}
\input{patterns/09_loops/main}
\input{patterns/10_strings/main}
\input{patterns/11_arith_optimizations/main}
\input{patterns/12_FPU/main}
\input{patterns/13_arrays/main}
\input{patterns/14_bitfields/main}
\EN{\input{patterns/145_LCG/main_EN}}
\RU{\input{patterns/145_LCG/main_RU}}
\input{patterns/15_structs/main}
\input{patterns/17_unions/main}
\input{patterns/18_pointers_to_functions/main}
\input{patterns/185_64bit_in_32_env/main}

\EN{\input{patterns/19_SIMD/main_EN}}
\RU{\input{patterns/19_SIMD/main_RU}}
\DE{\input{patterns/19_SIMD/main_DE}}

\EN{\input{patterns/20_x64/main_EN}}
\RU{\input{patterns/20_x64/main_RU}}

\EN{\input{patterns/205_floating_SIMD/main_EN}}
\RU{\input{patterns/205_floating_SIMD/main_RU}}
\DE{\input{patterns/205_floating_SIMD/main_DE}}

\EN{\input{patterns/ARM/main_EN}}
\RU{\input{patterns/ARM/main_RU}}
\DE{\input{patterns/ARM/main_DE}}

\input{patterns/MIPS/main}

\ifdefined\SPANISH
\chapter{Patrones de código}
\fi % SPANISH

\ifdefined\GERMAN
\chapter{Code-Muster}
\fi % GERMAN

\ifdefined\ENGLISH
\chapter{Code Patterns}
\fi % ENGLISH

\ifdefined\ITALIAN
\chapter{Forme di codice}
\fi % ITALIAN

\ifdefined\RUSSIAN
\chapter{Образцы кода}
\fi % RUSSIAN

\ifdefined\BRAZILIAN
\chapter{Padrões de códigos}
\fi % BRAZILIAN

\ifdefined\THAI
\chapter{รูปแบบของโค้ด}
\fi % THAI

\ifdefined\FRENCH
\chapter{Modèle de code}
\fi % FRENCH

\ifdefined\POLISH
\chapter{\PLph{}}
\fi % POLISH

% sections
\EN{\input{patterns/patterns_opt_dbg_EN}}
\ES{\input{patterns/patterns_opt_dbg_ES}}
\ITA{\input{patterns/patterns_opt_dbg_ITA}}
\PTBR{\input{patterns/patterns_opt_dbg_PTBR}}
\RU{\input{patterns/patterns_opt_dbg_RU}}
\THA{\input{patterns/patterns_opt_dbg_THA}}
\DE{\input{patterns/patterns_opt_dbg_DE}}
\FR{\input{patterns/patterns_opt_dbg_FR}}
\PL{\input{patterns/patterns_opt_dbg_PL}}

\RU{\section{Некоторые базовые понятия}}
\EN{\section{Some basics}}
\DE{\section{Einige Grundlagen}}
\FR{\section{Quelques bases}}
\ES{\section{\ESph{}}}
\ITA{\section{Alcune basi teoriche}}
\PTBR{\section{\PTBRph{}}}
\THA{\section{\THAph{}}}
\PL{\section{\PLph{}}}

% sections:
\EN{\input{patterns/intro_CPU_ISA_EN}}
\ES{\input{patterns/intro_CPU_ISA_ES}}
\ITA{\input{patterns/intro_CPU_ISA_ITA}}
\PTBR{\input{patterns/intro_CPU_ISA_PTBR}}
\RU{\input{patterns/intro_CPU_ISA_RU}}
\DE{\input{patterns/intro_CPU_ISA_DE}}
\FR{\input{patterns/intro_CPU_ISA_FR}}
\PL{\input{patterns/intro_CPU_ISA_PL}}

\EN{\input{patterns/numeral_EN}}
\RU{\input{patterns/numeral_RU}}
\ITA{\input{patterns/numeral_ITA}}
\DE{\input{patterns/numeral_DE}}
\FR{\input{patterns/numeral_FR}}
\PL{\input{patterns/numeral_PL}}

% chapters
\input{patterns/00_empty/main}
\input{patterns/011_ret/main}
\input{patterns/01_helloworld/main}
\input{patterns/015_prolog_epilogue/main}
\input{patterns/02_stack/main}
\input{patterns/03_printf/main}
\input{patterns/04_scanf/main}
\input{patterns/05_passing_arguments/main}
\input{patterns/06_return_results/main}
\input{patterns/061_pointers/main}
\input{patterns/065_GOTO/main}
\input{patterns/07_jcc/main}
\input{patterns/08_switch/main}
\input{patterns/09_loops/main}
\input{patterns/10_strings/main}
\input{patterns/11_arith_optimizations/main}
\input{patterns/12_FPU/main}
\input{patterns/13_arrays/main}
\input{patterns/14_bitfields/main}
\EN{\input{patterns/145_LCG/main_EN}}
\RU{\input{patterns/145_LCG/main_RU}}
\input{patterns/15_structs/main}
\input{patterns/17_unions/main}
\input{patterns/18_pointers_to_functions/main}
\input{patterns/185_64bit_in_32_env/main}

\EN{\input{patterns/19_SIMD/main_EN}}
\RU{\input{patterns/19_SIMD/main_RU}}
\DE{\input{patterns/19_SIMD/main_DE}}

\EN{\input{patterns/20_x64/main_EN}}
\RU{\input{patterns/20_x64/main_RU}}

\EN{\input{patterns/205_floating_SIMD/main_EN}}
\RU{\input{patterns/205_floating_SIMD/main_RU}}
\DE{\input{patterns/205_floating_SIMD/main_DE}}

\EN{\input{patterns/ARM/main_EN}}
\RU{\input{patterns/ARM/main_RU}}
\DE{\input{patterns/ARM/main_DE}}

\input{patterns/MIPS/main}

\EN{\section{Returning Values}
\label{ret_val_func}

Another simple function is the one that simply returns a constant value:

\lstinputlisting[caption=\EN{\CCpp Code},style=customc]{patterns/011_ret/1.c}

Let's compile it.

\subsection{x86}

Here's what both the GCC and MSVC compilers produce (with optimization) on the x86 platform:

\lstinputlisting[caption=\Optimizing GCC/MSVC (\assemblyOutput),style=customasmx86]{patterns/011_ret/1.s}

\myindex{x86!\Instructions!RET}
There are just two instructions: the first places the value 123 into the \EAX register,
which is used by convention for storing the return
value, and the second one is \RET, which returns execution to the \gls{caller}.

The caller will take the result from the \EAX register.

\subsection{ARM}

There are a few differences on the ARM platform:

\lstinputlisting[caption=\OptimizingKeilVI (\ARMMode) ASM Output,style=customasmARM]{patterns/011_ret/1_Keil_ARM_O3.s}

ARM uses the register \Reg{0} for returning the results of functions, so 123 is copied into \Reg{0}.

\myindex{ARM!\Instructions!MOV}
\myindex{x86!\Instructions!MOV}
It is worth noting that \MOV is a misleading name for the instruction in both the x86 and ARM \ac{ISA}s.

The data is not in fact \IT{moved}, but \IT{copied}.

\subsection{MIPS}

\label{MIPS_leaf_function_ex1}

The GCC assembly output below lists registers by number:

\lstinputlisting[caption=\Optimizing GCC 4.4.5 (\assemblyOutput),style=customasmMIPS]{patterns/011_ret/MIPS.s}

\dots while \IDA does it by their pseudo names:

\lstinputlisting[caption=\Optimizing GCC 4.4.5 (IDA),style=customasmMIPS]{patterns/011_ret/MIPS_IDA.lst}

The \$2 (or \$V0) register is used to store the function's return value.
\myindex{MIPS!\Pseudoinstructions!LI}
\INS{LI} stands for ``Load Immediate'' and is the MIPS equivalent to \MOV.

\myindex{MIPS!\Instructions!J}
The other instruction is the jump instruction (J or JR) which returns the execution flow to the \gls{caller}.

\myindex{MIPS!Branch delay slot}
You might be wondering why the positions of the load instruction (LI) and the jump instruction (J or JR) are swapped. This is due to a \ac{RISC} feature called ``branch delay slot''.

The reason this happens is a quirk in the architecture of some RISC \ac{ISA}s and isn't important for our
purposes---we must simply keep in mind that in MIPS, the instruction following a jump or branch instruction
is executed \IT{before} the jump/branch instruction itself.

As a consequence, branch instructions always swap places with the instruction executed immediately beforehand.


In practice, functions which merely return 1 (\IT{true}) or 0 (\IT{false}) are very frequent.

The smallest ever of the standard UNIX utilities, \IT{/bin/true} and \IT{/bin/false} return 0 and 1 respectively, as an exit code.
(Zero as an exit code usually means success, non-zero means error.)
}
\RU{\subsubsection{std::string}
\myindex{\Cpp!STL!std::string}
\label{std_string}

\myparagraph{Как устроена структура}

Многие строковые библиотеки \InSqBrackets{\CNotes 2.2} обеспечивают структуру содержащую ссылку 
на буфер собственно со строкой, переменная всегда содержащую длину строки 
(что очень удобно для массы функций \InSqBrackets{\CNotes 2.2.1}) и переменную содержащую текущий размер буфера.

Строка в буфере обыкновенно оканчивается нулем: это для того чтобы указатель на буфер можно было
передавать в функции требующие на вход обычную сишную \ac{ASCIIZ}-строку.

Стандарт \Cpp не описывает, как именно нужно реализовывать std::string,
но, как правило, они реализованы как описано выше, с небольшими дополнениями.

Строки в \Cpp это не класс (как, например, QString в Qt), а темплейт (basic\_string), 
это сделано для того чтобы поддерживать 
строки содержащие разного типа символы: как минимум \Tchar и \IT{wchar\_t}.

Так что, std::string это класс с базовым типом \Tchar.

А std::wstring это класс с базовым типом \IT{wchar\_t}.

\mysubparagraph{MSVC}

В реализации MSVC, вместо ссылки на буфер может содержаться сам буфер (если строка короче 16-и символов).

Это означает, что каждая короткая строка будет занимать в памяти по крайней мере $16 + 4 + 4 = 24$ 
байт для 32-битной среды либо $16 + 8 + 8 = 32$ 
байта в 64-битной, а если строка длиннее 16-и символов, то прибавьте еще длину самой строки.

\lstinputlisting[caption=пример для MSVC,style=customc]{\CURPATH/STL/string/MSVC_RU.cpp}

Собственно, из этого исходника почти всё ясно.

Несколько замечаний:

Если строка короче 16-и символов, 
то отдельный буфер для строки в \glslink{heap}{куче} выделяться не будет.

Это удобно потому что на практике, основная часть строк действительно короткие.
Вероятно, разработчики в Microsoft выбрали размер в 16 символов как разумный баланс.

Теперь очень важный момент в конце функции main(): мы не пользуемся методом c\_str(), тем не менее,
если это скомпилировать и запустить, то обе строки появятся в консоли!

Работает это вот почему.

В первом случае строка короче 16-и символов и в начале объекта std::string (его можно рассматривать
просто как структуру) расположен буфер с этой строкой.
\printf трактует указатель как указатель на массив символов оканчивающийся нулем и поэтому всё работает.

Вывод второй строки (длиннее 16-и символов) даже еще опаснее: это вообще типичная программистская ошибка 
(или опечатка), забыть дописать c\_str().
Это работает потому что в это время в начале структуры расположен указатель на буфер.
Это может надолго остаться незамеченным: до тех пока там не появится строка 
короче 16-и символов, тогда процесс упадет.

\mysubparagraph{GCC}

В реализации GCC в структуре есть еще одна переменная --- reference count.

Интересно, что указатель на экземпляр класса std::string в GCC указывает не на начало самой структуры, 
а на указатель на буфера.
В libstdc++-v3\textbackslash{}include\textbackslash{}bits\textbackslash{}basic\_string.h 
мы можем прочитать что это сделано для удобства отладки:

\begin{lstlisting}
   *  The reason you want _M_data pointing to the character %array and
   *  not the _Rep is so that the debugger can see the string
   *  contents. (Probably we should add a non-inline member to get
   *  the _Rep for the debugger to use, so users can check the actual
   *  string length.)
\end{lstlisting}

\href{http://go.yurichev.com/17085}{исходный код basic\_string.h}

В нашем примере мы учитываем это:

\lstinputlisting[caption=пример для GCC,style=customc]{\CURPATH/STL/string/GCC_RU.cpp}

Нужны еще небольшие хаки чтобы сымитировать типичную ошибку, которую мы уже видели выше, из-за
более ужесточенной проверки типов в GCC, тем не менее, printf() работает и здесь без c\_str().

\myparagraph{Чуть более сложный пример}

\lstinputlisting[style=customc]{\CURPATH/STL/string/3.cpp}

\lstinputlisting[caption=MSVC 2012,style=customasmx86]{\CURPATH/STL/string/3_MSVC_RU.asm}

Собственно, компилятор не конструирует строки статически: да в общем-то и как
это возможно, если буфер с ней нужно хранить в \glslink{heap}{куче}?

Вместо этого в сегменте данных хранятся обычные \ac{ASCIIZ}-строки, а позже, во время выполнения, 
при помощи метода \q{assign}, конструируются строки s1 и s2
.
При помощи \TT{operator+}, создается строка s3.

Обратите внимание на то что вызов метода c\_str() отсутствует,
потому что его код достаточно короткий и компилятор вставил его прямо здесь:
если строка короче 16-и байт, то в регистре EAX остается указатель на буфер,
а если длиннее, то из этого же места достается адрес на буфер расположенный в \glslink{heap}{куче}.

Далее следуют вызовы трех деструкторов, причем, они вызываются только если строка длиннее 16-и байт:
тогда нужно освободить буфера в \glslink{heap}{куче}.
В противном случае, так как все три объекта std::string хранятся в стеке,
они освобождаются автоматически после выхода из функции.

Следовательно, работа с короткими строками более быстрая из-за м\'{е}ньшего обращения к \glslink{heap}{куче}.

Код на GCC даже проще (из-за того, что в GCC, как мы уже видели, не реализована возможность хранить короткую
строку прямо в структуре):

% TODO1 comment each function meaning
\lstinputlisting[caption=GCC 4.8.1,style=customasmx86]{\CURPATH/STL/string/3_GCC_RU.s}

Можно заметить, что в деструкторы передается не указатель на объект,
а указатель на место за 12 байт (или 3 слова) перед ним, то есть, на настоящее начало структуры.

\myparagraph{std::string как глобальная переменная}
\label{sec:std_string_as_global_variable}

Опытные программисты на \Cpp знают, что глобальные переменные \ac{STL}-типов вполне можно объявлять.

Да, действительно:

\lstinputlisting[style=customc]{\CURPATH/STL/string/5.cpp}

Но как и где будет вызываться конструктор \TT{std::string}?

На самом деле, эта переменная будет инициализирована даже перед началом \main.

\lstinputlisting[caption=MSVC 2012: здесь конструируется глобальная переменная{,} а также регистрируется её деструктор,style=customasmx86]{\CURPATH/STL/string/5_MSVC_p2.asm}

\lstinputlisting[caption=MSVC 2012: здесь глобальная переменная используется в \main,style=customasmx86]{\CURPATH/STL/string/5_MSVC_p1.asm}

\lstinputlisting[caption=MSVC 2012: эта функция-деструктор вызывается перед выходом,style=customasmx86]{\CURPATH/STL/string/5_MSVC_p3.asm}

\myindex{\CStandardLibrary!atexit()}
В реальности, из \ac{CRT}, еще до вызова main(), вызывается специальная функция,
в которой перечислены все конструкторы подобных переменных.
Более того: при помощи atexit() регистрируется функция, которая будет вызвана в конце работы программы:
в этой функции компилятор собирает вызовы деструкторов всех подобных глобальных переменных.

GCC работает похожим образом:

\lstinputlisting[caption=GCC 4.8.1,style=customasmx86]{\CURPATH/STL/string/5_GCC.s}

Но он не выделяет отдельной функции в которой будут собраны деструкторы: 
каждый деструктор передается в atexit() по одному.

% TODO а если глобальная STL-переменная в другом модуле? надо проверить.

}
\ifdefined\SPANISH
\chapter{Patrones de código}
\fi % SPANISH

\ifdefined\GERMAN
\chapter{Code-Muster}
\fi % GERMAN

\ifdefined\ENGLISH
\chapter{Code Patterns}
\fi % ENGLISH

\ifdefined\ITALIAN
\chapter{Forme di codice}
\fi % ITALIAN

\ifdefined\RUSSIAN
\chapter{Образцы кода}
\fi % RUSSIAN

\ifdefined\BRAZILIAN
\chapter{Padrões de códigos}
\fi % BRAZILIAN

\ifdefined\THAI
\chapter{รูปแบบของโค้ด}
\fi % THAI

\ifdefined\FRENCH
\chapter{Modèle de code}
\fi % FRENCH

\ifdefined\POLISH
\chapter{\PLph{}}
\fi % POLISH

% sections
\EN{\input{patterns/patterns_opt_dbg_EN}}
\ES{\input{patterns/patterns_opt_dbg_ES}}
\ITA{\input{patterns/patterns_opt_dbg_ITA}}
\PTBR{\input{patterns/patterns_opt_dbg_PTBR}}
\RU{\input{patterns/patterns_opt_dbg_RU}}
\THA{\input{patterns/patterns_opt_dbg_THA}}
\DE{\input{patterns/patterns_opt_dbg_DE}}
\FR{\input{patterns/patterns_opt_dbg_FR}}
\PL{\input{patterns/patterns_opt_dbg_PL}}

\RU{\section{Некоторые базовые понятия}}
\EN{\section{Some basics}}
\DE{\section{Einige Grundlagen}}
\FR{\section{Quelques bases}}
\ES{\section{\ESph{}}}
\ITA{\section{Alcune basi teoriche}}
\PTBR{\section{\PTBRph{}}}
\THA{\section{\THAph{}}}
\PL{\section{\PLph{}}}

% sections:
\EN{\input{patterns/intro_CPU_ISA_EN}}
\ES{\input{patterns/intro_CPU_ISA_ES}}
\ITA{\input{patterns/intro_CPU_ISA_ITA}}
\PTBR{\input{patterns/intro_CPU_ISA_PTBR}}
\RU{\input{patterns/intro_CPU_ISA_RU}}
\DE{\input{patterns/intro_CPU_ISA_DE}}
\FR{\input{patterns/intro_CPU_ISA_FR}}
\PL{\input{patterns/intro_CPU_ISA_PL}}

\EN{\input{patterns/numeral_EN}}
\RU{\input{patterns/numeral_RU}}
\ITA{\input{patterns/numeral_ITA}}
\DE{\input{patterns/numeral_DE}}
\FR{\input{patterns/numeral_FR}}
\PL{\input{patterns/numeral_PL}}

% chapters
\input{patterns/00_empty/main}
\input{patterns/011_ret/main}
\input{patterns/01_helloworld/main}
\input{patterns/015_prolog_epilogue/main}
\input{patterns/02_stack/main}
\input{patterns/03_printf/main}
\input{patterns/04_scanf/main}
\input{patterns/05_passing_arguments/main}
\input{patterns/06_return_results/main}
\input{patterns/061_pointers/main}
\input{patterns/065_GOTO/main}
\input{patterns/07_jcc/main}
\input{patterns/08_switch/main}
\input{patterns/09_loops/main}
\input{patterns/10_strings/main}
\input{patterns/11_arith_optimizations/main}
\input{patterns/12_FPU/main}
\input{patterns/13_arrays/main}
\input{patterns/14_bitfields/main}
\EN{\input{patterns/145_LCG/main_EN}}
\RU{\input{patterns/145_LCG/main_RU}}
\input{patterns/15_structs/main}
\input{patterns/17_unions/main}
\input{patterns/18_pointers_to_functions/main}
\input{patterns/185_64bit_in_32_env/main}

\EN{\input{patterns/19_SIMD/main_EN}}
\RU{\input{patterns/19_SIMD/main_RU}}
\DE{\input{patterns/19_SIMD/main_DE}}

\EN{\input{patterns/20_x64/main_EN}}
\RU{\input{patterns/20_x64/main_RU}}

\EN{\input{patterns/205_floating_SIMD/main_EN}}
\RU{\input{patterns/205_floating_SIMD/main_RU}}
\DE{\input{patterns/205_floating_SIMD/main_DE}}

\EN{\input{patterns/ARM/main_EN}}
\RU{\input{patterns/ARM/main_RU}}
\DE{\input{patterns/ARM/main_DE}}

\input{patterns/MIPS/main}

\ifdefined\SPANISH
\chapter{Patrones de código}
\fi % SPANISH

\ifdefined\GERMAN
\chapter{Code-Muster}
\fi % GERMAN

\ifdefined\ENGLISH
\chapter{Code Patterns}
\fi % ENGLISH

\ifdefined\ITALIAN
\chapter{Forme di codice}
\fi % ITALIAN

\ifdefined\RUSSIAN
\chapter{Образцы кода}
\fi % RUSSIAN

\ifdefined\BRAZILIAN
\chapter{Padrões de códigos}
\fi % BRAZILIAN

\ifdefined\THAI
\chapter{รูปแบบของโค้ด}
\fi % THAI

\ifdefined\FRENCH
\chapter{Modèle de code}
\fi % FRENCH

\ifdefined\POLISH
\chapter{\PLph{}}
\fi % POLISH

% sections
\EN{\input{patterns/patterns_opt_dbg_EN}}
\ES{\input{patterns/patterns_opt_dbg_ES}}
\ITA{\input{patterns/patterns_opt_dbg_ITA}}
\PTBR{\input{patterns/patterns_opt_dbg_PTBR}}
\RU{\input{patterns/patterns_opt_dbg_RU}}
\THA{\input{patterns/patterns_opt_dbg_THA}}
\DE{\input{patterns/patterns_opt_dbg_DE}}
\FR{\input{patterns/patterns_opt_dbg_FR}}
\PL{\input{patterns/patterns_opt_dbg_PL}}

\RU{\section{Некоторые базовые понятия}}
\EN{\section{Some basics}}
\DE{\section{Einige Grundlagen}}
\FR{\section{Quelques bases}}
\ES{\section{\ESph{}}}
\ITA{\section{Alcune basi teoriche}}
\PTBR{\section{\PTBRph{}}}
\THA{\section{\THAph{}}}
\PL{\section{\PLph{}}}

% sections:
\EN{\input{patterns/intro_CPU_ISA_EN}}
\ES{\input{patterns/intro_CPU_ISA_ES}}
\ITA{\input{patterns/intro_CPU_ISA_ITA}}
\PTBR{\input{patterns/intro_CPU_ISA_PTBR}}
\RU{\input{patterns/intro_CPU_ISA_RU}}
\DE{\input{patterns/intro_CPU_ISA_DE}}
\FR{\input{patterns/intro_CPU_ISA_FR}}
\PL{\input{patterns/intro_CPU_ISA_PL}}

\EN{\input{patterns/numeral_EN}}
\RU{\input{patterns/numeral_RU}}
\ITA{\input{patterns/numeral_ITA}}
\DE{\input{patterns/numeral_DE}}
\FR{\input{patterns/numeral_FR}}
\PL{\input{patterns/numeral_PL}}

% chapters
\input{patterns/00_empty/main}
\input{patterns/011_ret/main}
\input{patterns/01_helloworld/main}
\input{patterns/015_prolog_epilogue/main}
\input{patterns/02_stack/main}
\input{patterns/03_printf/main}
\input{patterns/04_scanf/main}
\input{patterns/05_passing_arguments/main}
\input{patterns/06_return_results/main}
\input{patterns/061_pointers/main}
\input{patterns/065_GOTO/main}
\input{patterns/07_jcc/main}
\input{patterns/08_switch/main}
\input{patterns/09_loops/main}
\input{patterns/10_strings/main}
\input{patterns/11_arith_optimizations/main}
\input{patterns/12_FPU/main}
\input{patterns/13_arrays/main}
\input{patterns/14_bitfields/main}
\EN{\input{patterns/145_LCG/main_EN}}
\RU{\input{patterns/145_LCG/main_RU}}
\input{patterns/15_structs/main}
\input{patterns/17_unions/main}
\input{patterns/18_pointers_to_functions/main}
\input{patterns/185_64bit_in_32_env/main}

\EN{\input{patterns/19_SIMD/main_EN}}
\RU{\input{patterns/19_SIMD/main_RU}}
\DE{\input{patterns/19_SIMD/main_DE}}

\EN{\input{patterns/20_x64/main_EN}}
\RU{\input{patterns/20_x64/main_RU}}

\EN{\input{patterns/205_floating_SIMD/main_EN}}
\RU{\input{patterns/205_floating_SIMD/main_RU}}
\DE{\input{patterns/205_floating_SIMD/main_DE}}

\EN{\input{patterns/ARM/main_EN}}
\RU{\input{patterns/ARM/main_RU}}
\DE{\input{patterns/ARM/main_DE}}

\input{patterns/MIPS/main}

\ifdefined\SPANISH
\chapter{Patrones de código}
\fi % SPANISH

\ifdefined\GERMAN
\chapter{Code-Muster}
\fi % GERMAN

\ifdefined\ENGLISH
\chapter{Code Patterns}
\fi % ENGLISH

\ifdefined\ITALIAN
\chapter{Forme di codice}
\fi % ITALIAN

\ifdefined\RUSSIAN
\chapter{Образцы кода}
\fi % RUSSIAN

\ifdefined\BRAZILIAN
\chapter{Padrões de códigos}
\fi % BRAZILIAN

\ifdefined\THAI
\chapter{รูปแบบของโค้ด}
\fi % THAI

\ifdefined\FRENCH
\chapter{Modèle de code}
\fi % FRENCH

\ifdefined\POLISH
\chapter{\PLph{}}
\fi % POLISH

% sections
\EN{\input{patterns/patterns_opt_dbg_EN}}
\ES{\input{patterns/patterns_opt_dbg_ES}}
\ITA{\input{patterns/patterns_opt_dbg_ITA}}
\PTBR{\input{patterns/patterns_opt_dbg_PTBR}}
\RU{\input{patterns/patterns_opt_dbg_RU}}
\THA{\input{patterns/patterns_opt_dbg_THA}}
\DE{\input{patterns/patterns_opt_dbg_DE}}
\FR{\input{patterns/patterns_opt_dbg_FR}}
\PL{\input{patterns/patterns_opt_dbg_PL}}

\RU{\section{Некоторые базовые понятия}}
\EN{\section{Some basics}}
\DE{\section{Einige Grundlagen}}
\FR{\section{Quelques bases}}
\ES{\section{\ESph{}}}
\ITA{\section{Alcune basi teoriche}}
\PTBR{\section{\PTBRph{}}}
\THA{\section{\THAph{}}}
\PL{\section{\PLph{}}}

% sections:
\EN{\input{patterns/intro_CPU_ISA_EN}}
\ES{\input{patterns/intro_CPU_ISA_ES}}
\ITA{\input{patterns/intro_CPU_ISA_ITA}}
\PTBR{\input{patterns/intro_CPU_ISA_PTBR}}
\RU{\input{patterns/intro_CPU_ISA_RU}}
\DE{\input{patterns/intro_CPU_ISA_DE}}
\FR{\input{patterns/intro_CPU_ISA_FR}}
\PL{\input{patterns/intro_CPU_ISA_PL}}

\EN{\input{patterns/numeral_EN}}
\RU{\input{patterns/numeral_RU}}
\ITA{\input{patterns/numeral_ITA}}
\DE{\input{patterns/numeral_DE}}
\FR{\input{patterns/numeral_FR}}
\PL{\input{patterns/numeral_PL}}

% chapters
\input{patterns/00_empty/main}
\input{patterns/011_ret/main}
\input{patterns/01_helloworld/main}
\input{patterns/015_prolog_epilogue/main}
\input{patterns/02_stack/main}
\input{patterns/03_printf/main}
\input{patterns/04_scanf/main}
\input{patterns/05_passing_arguments/main}
\input{patterns/06_return_results/main}
\input{patterns/061_pointers/main}
\input{patterns/065_GOTO/main}
\input{patterns/07_jcc/main}
\input{patterns/08_switch/main}
\input{patterns/09_loops/main}
\input{patterns/10_strings/main}
\input{patterns/11_arith_optimizations/main}
\input{patterns/12_FPU/main}
\input{patterns/13_arrays/main}
\input{patterns/14_bitfields/main}
\EN{\input{patterns/145_LCG/main_EN}}
\RU{\input{patterns/145_LCG/main_RU}}
\input{patterns/15_structs/main}
\input{patterns/17_unions/main}
\input{patterns/18_pointers_to_functions/main}
\input{patterns/185_64bit_in_32_env/main}

\EN{\input{patterns/19_SIMD/main_EN}}
\RU{\input{patterns/19_SIMD/main_RU}}
\DE{\input{patterns/19_SIMD/main_DE}}

\EN{\input{patterns/20_x64/main_EN}}
\RU{\input{patterns/20_x64/main_RU}}

\EN{\input{patterns/205_floating_SIMD/main_EN}}
\RU{\input{patterns/205_floating_SIMD/main_RU}}
\DE{\input{patterns/205_floating_SIMD/main_DE}}

\EN{\input{patterns/ARM/main_EN}}
\RU{\input{patterns/ARM/main_RU}}
\DE{\input{patterns/ARM/main_DE}}

\input{patterns/MIPS/main}

\ifdefined\SPANISH
\chapter{Patrones de código}
\fi % SPANISH

\ifdefined\GERMAN
\chapter{Code-Muster}
\fi % GERMAN

\ifdefined\ENGLISH
\chapter{Code Patterns}
\fi % ENGLISH

\ifdefined\ITALIAN
\chapter{Forme di codice}
\fi % ITALIAN

\ifdefined\RUSSIAN
\chapter{Образцы кода}
\fi % RUSSIAN

\ifdefined\BRAZILIAN
\chapter{Padrões de códigos}
\fi % BRAZILIAN

\ifdefined\THAI
\chapter{รูปแบบของโค้ด}
\fi % THAI

\ifdefined\FRENCH
\chapter{Modèle de code}
\fi % FRENCH

\ifdefined\POLISH
\chapter{\PLph{}}
\fi % POLISH

% sections
\EN{\input{patterns/patterns_opt_dbg_EN}}
\ES{\input{patterns/patterns_opt_dbg_ES}}
\ITA{\input{patterns/patterns_opt_dbg_ITA}}
\PTBR{\input{patterns/patterns_opt_dbg_PTBR}}
\RU{\input{patterns/patterns_opt_dbg_RU}}
\THA{\input{patterns/patterns_opt_dbg_THA}}
\DE{\input{patterns/patterns_opt_dbg_DE}}
\FR{\input{patterns/patterns_opt_dbg_FR}}
\PL{\input{patterns/patterns_opt_dbg_PL}}

\RU{\section{Некоторые базовые понятия}}
\EN{\section{Some basics}}
\DE{\section{Einige Grundlagen}}
\FR{\section{Quelques bases}}
\ES{\section{\ESph{}}}
\ITA{\section{Alcune basi teoriche}}
\PTBR{\section{\PTBRph{}}}
\THA{\section{\THAph{}}}
\PL{\section{\PLph{}}}

% sections:
\EN{\input{patterns/intro_CPU_ISA_EN}}
\ES{\input{patterns/intro_CPU_ISA_ES}}
\ITA{\input{patterns/intro_CPU_ISA_ITA}}
\PTBR{\input{patterns/intro_CPU_ISA_PTBR}}
\RU{\input{patterns/intro_CPU_ISA_RU}}
\DE{\input{patterns/intro_CPU_ISA_DE}}
\FR{\input{patterns/intro_CPU_ISA_FR}}
\PL{\input{patterns/intro_CPU_ISA_PL}}

\EN{\input{patterns/numeral_EN}}
\RU{\input{patterns/numeral_RU}}
\ITA{\input{patterns/numeral_ITA}}
\DE{\input{patterns/numeral_DE}}
\FR{\input{patterns/numeral_FR}}
\PL{\input{patterns/numeral_PL}}

% chapters
\input{patterns/00_empty/main}
\input{patterns/011_ret/main}
\input{patterns/01_helloworld/main}
\input{patterns/015_prolog_epilogue/main}
\input{patterns/02_stack/main}
\input{patterns/03_printf/main}
\input{patterns/04_scanf/main}
\input{patterns/05_passing_arguments/main}
\input{patterns/06_return_results/main}
\input{patterns/061_pointers/main}
\input{patterns/065_GOTO/main}
\input{patterns/07_jcc/main}
\input{patterns/08_switch/main}
\input{patterns/09_loops/main}
\input{patterns/10_strings/main}
\input{patterns/11_arith_optimizations/main}
\input{patterns/12_FPU/main}
\input{patterns/13_arrays/main}
\input{patterns/14_bitfields/main}
\EN{\input{patterns/145_LCG/main_EN}}
\RU{\input{patterns/145_LCG/main_RU}}
\input{patterns/15_structs/main}
\input{patterns/17_unions/main}
\input{patterns/18_pointers_to_functions/main}
\input{patterns/185_64bit_in_32_env/main}

\EN{\input{patterns/19_SIMD/main_EN}}
\RU{\input{patterns/19_SIMD/main_RU}}
\DE{\input{patterns/19_SIMD/main_DE}}

\EN{\input{patterns/20_x64/main_EN}}
\RU{\input{patterns/20_x64/main_RU}}

\EN{\input{patterns/205_floating_SIMD/main_EN}}
\RU{\input{patterns/205_floating_SIMD/main_RU}}
\DE{\input{patterns/205_floating_SIMD/main_DE}}

\EN{\input{patterns/ARM/main_EN}}
\RU{\input{patterns/ARM/main_RU}}
\DE{\input{patterns/ARM/main_DE}}

\input{patterns/MIPS/main}


\EN{\section{Returning Values}
\label{ret_val_func}

Another simple function is the one that simply returns a constant value:

\lstinputlisting[caption=\EN{\CCpp Code},style=customc]{patterns/011_ret/1.c}

Let's compile it.

\subsection{x86}

Here's what both the GCC and MSVC compilers produce (with optimization) on the x86 platform:

\lstinputlisting[caption=\Optimizing GCC/MSVC (\assemblyOutput),style=customasmx86]{patterns/011_ret/1.s}

\myindex{x86!\Instructions!RET}
There are just two instructions: the first places the value 123 into the \EAX register,
which is used by convention for storing the return
value, and the second one is \RET, which returns execution to the \gls{caller}.

The caller will take the result from the \EAX register.

\subsection{ARM}

There are a few differences on the ARM platform:

\lstinputlisting[caption=\OptimizingKeilVI (\ARMMode) ASM Output,style=customasmARM]{patterns/011_ret/1_Keil_ARM_O3.s}

ARM uses the register \Reg{0} for returning the results of functions, so 123 is copied into \Reg{0}.

\myindex{ARM!\Instructions!MOV}
\myindex{x86!\Instructions!MOV}
It is worth noting that \MOV is a misleading name for the instruction in both the x86 and ARM \ac{ISA}s.

The data is not in fact \IT{moved}, but \IT{copied}.

\subsection{MIPS}

\label{MIPS_leaf_function_ex1}

The GCC assembly output below lists registers by number:

\lstinputlisting[caption=\Optimizing GCC 4.4.5 (\assemblyOutput),style=customasmMIPS]{patterns/011_ret/MIPS.s}

\dots while \IDA does it by their pseudo names:

\lstinputlisting[caption=\Optimizing GCC 4.4.5 (IDA),style=customasmMIPS]{patterns/011_ret/MIPS_IDA.lst}

The \$2 (or \$V0) register is used to store the function's return value.
\myindex{MIPS!\Pseudoinstructions!LI}
\INS{LI} stands for ``Load Immediate'' and is the MIPS equivalent to \MOV.

\myindex{MIPS!\Instructions!J}
The other instruction is the jump instruction (J or JR) which returns the execution flow to the \gls{caller}.

\myindex{MIPS!Branch delay slot}
You might be wondering why the positions of the load instruction (LI) and the jump instruction (J or JR) are swapped. This is due to a \ac{RISC} feature called ``branch delay slot''.

The reason this happens is a quirk in the architecture of some RISC \ac{ISA}s and isn't important for our
purposes---we must simply keep in mind that in MIPS, the instruction following a jump or branch instruction
is executed \IT{before} the jump/branch instruction itself.

As a consequence, branch instructions always swap places with the instruction executed immediately beforehand.


In practice, functions which merely return 1 (\IT{true}) or 0 (\IT{false}) are very frequent.

The smallest ever of the standard UNIX utilities, \IT{/bin/true} and \IT{/bin/false} return 0 and 1 respectively, as an exit code.
(Zero as an exit code usually means success, non-zero means error.)
}
\RU{\subsubsection{std::string}
\myindex{\Cpp!STL!std::string}
\label{std_string}

\myparagraph{Как устроена структура}

Многие строковые библиотеки \InSqBrackets{\CNotes 2.2} обеспечивают структуру содержащую ссылку 
на буфер собственно со строкой, переменная всегда содержащую длину строки 
(что очень удобно для массы функций \InSqBrackets{\CNotes 2.2.1}) и переменную содержащую текущий размер буфера.

Строка в буфере обыкновенно оканчивается нулем: это для того чтобы указатель на буфер можно было
передавать в функции требующие на вход обычную сишную \ac{ASCIIZ}-строку.

Стандарт \Cpp не описывает, как именно нужно реализовывать std::string,
но, как правило, они реализованы как описано выше, с небольшими дополнениями.

Строки в \Cpp это не класс (как, например, QString в Qt), а темплейт (basic\_string), 
это сделано для того чтобы поддерживать 
строки содержащие разного типа символы: как минимум \Tchar и \IT{wchar\_t}.

Так что, std::string это класс с базовым типом \Tchar.

А std::wstring это класс с базовым типом \IT{wchar\_t}.

\mysubparagraph{MSVC}

В реализации MSVC, вместо ссылки на буфер может содержаться сам буфер (если строка короче 16-и символов).

Это означает, что каждая короткая строка будет занимать в памяти по крайней мере $16 + 4 + 4 = 24$ 
байт для 32-битной среды либо $16 + 8 + 8 = 32$ 
байта в 64-битной, а если строка длиннее 16-и символов, то прибавьте еще длину самой строки.

\lstinputlisting[caption=пример для MSVC,style=customc]{\CURPATH/STL/string/MSVC_RU.cpp}

Собственно, из этого исходника почти всё ясно.

Несколько замечаний:

Если строка короче 16-и символов, 
то отдельный буфер для строки в \glslink{heap}{куче} выделяться не будет.

Это удобно потому что на практике, основная часть строк действительно короткие.
Вероятно, разработчики в Microsoft выбрали размер в 16 символов как разумный баланс.

Теперь очень важный момент в конце функции main(): мы не пользуемся методом c\_str(), тем не менее,
если это скомпилировать и запустить, то обе строки появятся в консоли!

Работает это вот почему.

В первом случае строка короче 16-и символов и в начале объекта std::string (его можно рассматривать
просто как структуру) расположен буфер с этой строкой.
\printf трактует указатель как указатель на массив символов оканчивающийся нулем и поэтому всё работает.

Вывод второй строки (длиннее 16-и символов) даже еще опаснее: это вообще типичная программистская ошибка 
(или опечатка), забыть дописать c\_str().
Это работает потому что в это время в начале структуры расположен указатель на буфер.
Это может надолго остаться незамеченным: до тех пока там не появится строка 
короче 16-и символов, тогда процесс упадет.

\mysubparagraph{GCC}

В реализации GCC в структуре есть еще одна переменная --- reference count.

Интересно, что указатель на экземпляр класса std::string в GCC указывает не на начало самой структуры, 
а на указатель на буфера.
В libstdc++-v3\textbackslash{}include\textbackslash{}bits\textbackslash{}basic\_string.h 
мы можем прочитать что это сделано для удобства отладки:

\begin{lstlisting}
   *  The reason you want _M_data pointing to the character %array and
   *  not the _Rep is so that the debugger can see the string
   *  contents. (Probably we should add a non-inline member to get
   *  the _Rep for the debugger to use, so users can check the actual
   *  string length.)
\end{lstlisting}

\href{http://go.yurichev.com/17085}{исходный код basic\_string.h}

В нашем примере мы учитываем это:

\lstinputlisting[caption=пример для GCC,style=customc]{\CURPATH/STL/string/GCC_RU.cpp}

Нужны еще небольшие хаки чтобы сымитировать типичную ошибку, которую мы уже видели выше, из-за
более ужесточенной проверки типов в GCC, тем не менее, printf() работает и здесь без c\_str().

\myparagraph{Чуть более сложный пример}

\lstinputlisting[style=customc]{\CURPATH/STL/string/3.cpp}

\lstinputlisting[caption=MSVC 2012,style=customasmx86]{\CURPATH/STL/string/3_MSVC_RU.asm}

Собственно, компилятор не конструирует строки статически: да в общем-то и как
это возможно, если буфер с ней нужно хранить в \glslink{heap}{куче}?

Вместо этого в сегменте данных хранятся обычные \ac{ASCIIZ}-строки, а позже, во время выполнения, 
при помощи метода \q{assign}, конструируются строки s1 и s2
.
При помощи \TT{operator+}, создается строка s3.

Обратите внимание на то что вызов метода c\_str() отсутствует,
потому что его код достаточно короткий и компилятор вставил его прямо здесь:
если строка короче 16-и байт, то в регистре EAX остается указатель на буфер,
а если длиннее, то из этого же места достается адрес на буфер расположенный в \glslink{heap}{куче}.

Далее следуют вызовы трех деструкторов, причем, они вызываются только если строка длиннее 16-и байт:
тогда нужно освободить буфера в \glslink{heap}{куче}.
В противном случае, так как все три объекта std::string хранятся в стеке,
они освобождаются автоматически после выхода из функции.

Следовательно, работа с короткими строками более быстрая из-за м\'{е}ньшего обращения к \glslink{heap}{куче}.

Код на GCC даже проще (из-за того, что в GCC, как мы уже видели, не реализована возможность хранить короткую
строку прямо в структуре):

% TODO1 comment each function meaning
\lstinputlisting[caption=GCC 4.8.1,style=customasmx86]{\CURPATH/STL/string/3_GCC_RU.s}

Можно заметить, что в деструкторы передается не указатель на объект,
а указатель на место за 12 байт (или 3 слова) перед ним, то есть, на настоящее начало структуры.

\myparagraph{std::string как глобальная переменная}
\label{sec:std_string_as_global_variable}

Опытные программисты на \Cpp знают, что глобальные переменные \ac{STL}-типов вполне можно объявлять.

Да, действительно:

\lstinputlisting[style=customc]{\CURPATH/STL/string/5.cpp}

Но как и где будет вызываться конструктор \TT{std::string}?

На самом деле, эта переменная будет инициализирована даже перед началом \main.

\lstinputlisting[caption=MSVC 2012: здесь конструируется глобальная переменная{,} а также регистрируется её деструктор,style=customasmx86]{\CURPATH/STL/string/5_MSVC_p2.asm}

\lstinputlisting[caption=MSVC 2012: здесь глобальная переменная используется в \main,style=customasmx86]{\CURPATH/STL/string/5_MSVC_p1.asm}

\lstinputlisting[caption=MSVC 2012: эта функция-деструктор вызывается перед выходом,style=customasmx86]{\CURPATH/STL/string/5_MSVC_p3.asm}

\myindex{\CStandardLibrary!atexit()}
В реальности, из \ac{CRT}, еще до вызова main(), вызывается специальная функция,
в которой перечислены все конструкторы подобных переменных.
Более того: при помощи atexit() регистрируется функция, которая будет вызвана в конце работы программы:
в этой функции компилятор собирает вызовы деструкторов всех подобных глобальных переменных.

GCC работает похожим образом:

\lstinputlisting[caption=GCC 4.8.1,style=customasmx86]{\CURPATH/STL/string/5_GCC.s}

Но он не выделяет отдельной функции в которой будут собраны деструкторы: 
каждый деструктор передается в atexit() по одному.

% TODO а если глобальная STL-переменная в другом модуле? надо проверить.

}
\DE{\subsection{Einfachste XOR-Verschlüsselung überhaupt}

Ich habe einmal eine Software gesehen, bei der alle Debugging-Ausgaben mit XOR mit dem Wert 3
verschlüsselt wurden. Mit anderen Worten, die beiden niedrigsten Bits aller Buchstaben wurden invertiert.

``Hello, world'' wurde zu ``Kfool/\#tlqog'':

\begin{lstlisting}
#!/usr/bin/python

msg="Hello, world!"

print "".join(map(lambda x: chr(ord(x)^3), msg))
\end{lstlisting}

Das ist eine ziemlich interessante Verschlüsselung (oder besser eine Verschleierung),
weil sie zwei wichtige Eigenschaften hat:
1) es ist eine einzige Funktion zum Verschlüsseln und entschlüsseln, sie muss nur wiederholt angewendet werden
2) die entstehenden Buchstaben befinden sich im druckbaren Bereich, also die ganze Zeichenkette kann ohne
Escape-Symbole im Code verwendet werden.

Die zweite Eigenschaft nutzt die Tatsache, dass alle druckbaren Zeichen in Reihen organisiert sind: 0x2x-0x7x,
und wenn die beiden niederwertigsten Bits invertiert werden, wird der Buchstabe um eine oder drei Stellen nach
links oder rechts \IT{verschoben}, aber niemals in eine andere Reihe:

\begin{figure}[H]
\centering
\includegraphics[width=0.7\textwidth]{ascii_clean.png}
\caption{7-Bit \ac{ASCII} Tabelle in Emacs}
\end{figure}

\dots mit dem Zeichen 0x7F als einziger Ausnahme.

Im Folgenden werden also beispielsweise die Zeichen A-Z \IT{verschlüsselt}:

\begin{lstlisting}
#!/usr/bin/python

msg="@ABCDEFGHIJKLMNO"

print "".join(map(lambda x: chr(ord(x)^3), msg))
\end{lstlisting}

Ergebnis:
% FIXME \verb  --  relevant comment for German?
\begin{lstlisting}
CBA@GFEDKJIHONML
\end{lstlisting}

Es sieht so aus als würden die Zeichen ``@'' und ``C'' sowie ``B'' und ``A'' vertauscht werden.

Hier ist noch ein interessantes Beispiel, in dem gezeigt wird, wie die Eigenschaften von XOR
ausgenutzt werden können: Exakt den gleichen Effekt, dass druckbare Zeichen auch druckbar bleiben,
kann man dadurch erzielen, dass irgendeine Kombination der niedrigsten vier Bits invertiert wird.
}

\EN{\section{Returning Values}
\label{ret_val_func}

Another simple function is the one that simply returns a constant value:

\lstinputlisting[caption=\EN{\CCpp Code},style=customc]{patterns/011_ret/1.c}

Let's compile it.

\subsection{x86}

Here's what both the GCC and MSVC compilers produce (with optimization) on the x86 platform:

\lstinputlisting[caption=\Optimizing GCC/MSVC (\assemblyOutput),style=customasmx86]{patterns/011_ret/1.s}

\myindex{x86!\Instructions!RET}
There are just two instructions: the first places the value 123 into the \EAX register,
which is used by convention for storing the return
value, and the second one is \RET, which returns execution to the \gls{caller}.

The caller will take the result from the \EAX register.

\subsection{ARM}

There are a few differences on the ARM platform:

\lstinputlisting[caption=\OptimizingKeilVI (\ARMMode) ASM Output,style=customasmARM]{patterns/011_ret/1_Keil_ARM_O3.s}

ARM uses the register \Reg{0} for returning the results of functions, so 123 is copied into \Reg{0}.

\myindex{ARM!\Instructions!MOV}
\myindex{x86!\Instructions!MOV}
It is worth noting that \MOV is a misleading name for the instruction in both the x86 and ARM \ac{ISA}s.

The data is not in fact \IT{moved}, but \IT{copied}.

\subsection{MIPS}

\label{MIPS_leaf_function_ex1}

The GCC assembly output below lists registers by number:

\lstinputlisting[caption=\Optimizing GCC 4.4.5 (\assemblyOutput),style=customasmMIPS]{patterns/011_ret/MIPS.s}

\dots while \IDA does it by their pseudo names:

\lstinputlisting[caption=\Optimizing GCC 4.4.5 (IDA),style=customasmMIPS]{patterns/011_ret/MIPS_IDA.lst}

The \$2 (or \$V0) register is used to store the function's return value.
\myindex{MIPS!\Pseudoinstructions!LI}
\INS{LI} stands for ``Load Immediate'' and is the MIPS equivalent to \MOV.

\myindex{MIPS!\Instructions!J}
The other instruction is the jump instruction (J or JR) which returns the execution flow to the \gls{caller}.

\myindex{MIPS!Branch delay slot}
You might be wondering why the positions of the load instruction (LI) and the jump instruction (J or JR) are swapped. This is due to a \ac{RISC} feature called ``branch delay slot''.

The reason this happens is a quirk in the architecture of some RISC \ac{ISA}s and isn't important for our
purposes---we must simply keep in mind that in MIPS, the instruction following a jump or branch instruction
is executed \IT{before} the jump/branch instruction itself.

As a consequence, branch instructions always swap places with the instruction executed immediately beforehand.


In practice, functions which merely return 1 (\IT{true}) or 0 (\IT{false}) are very frequent.

The smallest ever of the standard UNIX utilities, \IT{/bin/true} and \IT{/bin/false} return 0 and 1 respectively, as an exit code.
(Zero as an exit code usually means success, non-zero means error.)
}
\RU{\subsubsection{std::string}
\myindex{\Cpp!STL!std::string}
\label{std_string}

\myparagraph{Как устроена структура}

Многие строковые библиотеки \InSqBrackets{\CNotes 2.2} обеспечивают структуру содержащую ссылку 
на буфер собственно со строкой, переменная всегда содержащую длину строки 
(что очень удобно для массы функций \InSqBrackets{\CNotes 2.2.1}) и переменную содержащую текущий размер буфера.

Строка в буфере обыкновенно оканчивается нулем: это для того чтобы указатель на буфер можно было
передавать в функции требующие на вход обычную сишную \ac{ASCIIZ}-строку.

Стандарт \Cpp не описывает, как именно нужно реализовывать std::string,
но, как правило, они реализованы как описано выше, с небольшими дополнениями.

Строки в \Cpp это не класс (как, например, QString в Qt), а темплейт (basic\_string), 
это сделано для того чтобы поддерживать 
строки содержащие разного типа символы: как минимум \Tchar и \IT{wchar\_t}.

Так что, std::string это класс с базовым типом \Tchar.

А std::wstring это класс с базовым типом \IT{wchar\_t}.

\mysubparagraph{MSVC}

В реализации MSVC, вместо ссылки на буфер может содержаться сам буфер (если строка короче 16-и символов).

Это означает, что каждая короткая строка будет занимать в памяти по крайней мере $16 + 4 + 4 = 24$ 
байт для 32-битной среды либо $16 + 8 + 8 = 32$ 
байта в 64-битной, а если строка длиннее 16-и символов, то прибавьте еще длину самой строки.

\lstinputlisting[caption=пример для MSVC,style=customc]{\CURPATH/STL/string/MSVC_RU.cpp}

Собственно, из этого исходника почти всё ясно.

Несколько замечаний:

Если строка короче 16-и символов, 
то отдельный буфер для строки в \glslink{heap}{куче} выделяться не будет.

Это удобно потому что на практике, основная часть строк действительно короткие.
Вероятно, разработчики в Microsoft выбрали размер в 16 символов как разумный баланс.

Теперь очень важный момент в конце функции main(): мы не пользуемся методом c\_str(), тем не менее,
если это скомпилировать и запустить, то обе строки появятся в консоли!

Работает это вот почему.

В первом случае строка короче 16-и символов и в начале объекта std::string (его можно рассматривать
просто как структуру) расположен буфер с этой строкой.
\printf трактует указатель как указатель на массив символов оканчивающийся нулем и поэтому всё работает.

Вывод второй строки (длиннее 16-и символов) даже еще опаснее: это вообще типичная программистская ошибка 
(или опечатка), забыть дописать c\_str().
Это работает потому что в это время в начале структуры расположен указатель на буфер.
Это может надолго остаться незамеченным: до тех пока там не появится строка 
короче 16-и символов, тогда процесс упадет.

\mysubparagraph{GCC}

В реализации GCC в структуре есть еще одна переменная --- reference count.

Интересно, что указатель на экземпляр класса std::string в GCC указывает не на начало самой структуры, 
а на указатель на буфера.
В libstdc++-v3\textbackslash{}include\textbackslash{}bits\textbackslash{}basic\_string.h 
мы можем прочитать что это сделано для удобства отладки:

\begin{lstlisting}
   *  The reason you want _M_data pointing to the character %array and
   *  not the _Rep is so that the debugger can see the string
   *  contents. (Probably we should add a non-inline member to get
   *  the _Rep for the debugger to use, so users can check the actual
   *  string length.)
\end{lstlisting}

\href{http://go.yurichev.com/17085}{исходный код basic\_string.h}

В нашем примере мы учитываем это:

\lstinputlisting[caption=пример для GCC,style=customc]{\CURPATH/STL/string/GCC_RU.cpp}

Нужны еще небольшие хаки чтобы сымитировать типичную ошибку, которую мы уже видели выше, из-за
более ужесточенной проверки типов в GCC, тем не менее, printf() работает и здесь без c\_str().

\myparagraph{Чуть более сложный пример}

\lstinputlisting[style=customc]{\CURPATH/STL/string/3.cpp}

\lstinputlisting[caption=MSVC 2012,style=customasmx86]{\CURPATH/STL/string/3_MSVC_RU.asm}

Собственно, компилятор не конструирует строки статически: да в общем-то и как
это возможно, если буфер с ней нужно хранить в \glslink{heap}{куче}?

Вместо этого в сегменте данных хранятся обычные \ac{ASCIIZ}-строки, а позже, во время выполнения, 
при помощи метода \q{assign}, конструируются строки s1 и s2
.
При помощи \TT{operator+}, создается строка s3.

Обратите внимание на то что вызов метода c\_str() отсутствует,
потому что его код достаточно короткий и компилятор вставил его прямо здесь:
если строка короче 16-и байт, то в регистре EAX остается указатель на буфер,
а если длиннее, то из этого же места достается адрес на буфер расположенный в \glslink{heap}{куче}.

Далее следуют вызовы трех деструкторов, причем, они вызываются только если строка длиннее 16-и байт:
тогда нужно освободить буфера в \glslink{heap}{куче}.
В противном случае, так как все три объекта std::string хранятся в стеке,
они освобождаются автоматически после выхода из функции.

Следовательно, работа с короткими строками более быстрая из-за м\'{е}ньшего обращения к \glslink{heap}{куче}.

Код на GCC даже проще (из-за того, что в GCC, как мы уже видели, не реализована возможность хранить короткую
строку прямо в структуре):

% TODO1 comment each function meaning
\lstinputlisting[caption=GCC 4.8.1,style=customasmx86]{\CURPATH/STL/string/3_GCC_RU.s}

Можно заметить, что в деструкторы передается не указатель на объект,
а указатель на место за 12 байт (или 3 слова) перед ним, то есть, на настоящее начало структуры.

\myparagraph{std::string как глобальная переменная}
\label{sec:std_string_as_global_variable}

Опытные программисты на \Cpp знают, что глобальные переменные \ac{STL}-типов вполне можно объявлять.

Да, действительно:

\lstinputlisting[style=customc]{\CURPATH/STL/string/5.cpp}

Но как и где будет вызываться конструктор \TT{std::string}?

На самом деле, эта переменная будет инициализирована даже перед началом \main.

\lstinputlisting[caption=MSVC 2012: здесь конструируется глобальная переменная{,} а также регистрируется её деструктор,style=customasmx86]{\CURPATH/STL/string/5_MSVC_p2.asm}

\lstinputlisting[caption=MSVC 2012: здесь глобальная переменная используется в \main,style=customasmx86]{\CURPATH/STL/string/5_MSVC_p1.asm}

\lstinputlisting[caption=MSVC 2012: эта функция-деструктор вызывается перед выходом,style=customasmx86]{\CURPATH/STL/string/5_MSVC_p3.asm}

\myindex{\CStandardLibrary!atexit()}
В реальности, из \ac{CRT}, еще до вызова main(), вызывается специальная функция,
в которой перечислены все конструкторы подобных переменных.
Более того: при помощи atexit() регистрируется функция, которая будет вызвана в конце работы программы:
в этой функции компилятор собирает вызовы деструкторов всех подобных глобальных переменных.

GCC работает похожим образом:

\lstinputlisting[caption=GCC 4.8.1,style=customasmx86]{\CURPATH/STL/string/5_GCC.s}

Но он не выделяет отдельной функции в которой будут собраны деструкторы: 
каждый деструктор передается в atexit() по одному.

% TODO а если глобальная STL-переменная в другом модуле? надо проверить.

}

\EN{\section{Returning Values}
\label{ret_val_func}

Another simple function is the one that simply returns a constant value:

\lstinputlisting[caption=\EN{\CCpp Code},style=customc]{patterns/011_ret/1.c}

Let's compile it.

\subsection{x86}

Here's what both the GCC and MSVC compilers produce (with optimization) on the x86 platform:

\lstinputlisting[caption=\Optimizing GCC/MSVC (\assemblyOutput),style=customasmx86]{patterns/011_ret/1.s}

\myindex{x86!\Instructions!RET}
There are just two instructions: the first places the value 123 into the \EAX register,
which is used by convention for storing the return
value, and the second one is \RET, which returns execution to the \gls{caller}.

The caller will take the result from the \EAX register.

\subsection{ARM}

There are a few differences on the ARM platform:

\lstinputlisting[caption=\OptimizingKeilVI (\ARMMode) ASM Output,style=customasmARM]{patterns/011_ret/1_Keil_ARM_O3.s}

ARM uses the register \Reg{0} for returning the results of functions, so 123 is copied into \Reg{0}.

\myindex{ARM!\Instructions!MOV}
\myindex{x86!\Instructions!MOV}
It is worth noting that \MOV is a misleading name for the instruction in both the x86 and ARM \ac{ISA}s.

The data is not in fact \IT{moved}, but \IT{copied}.

\subsection{MIPS}

\label{MIPS_leaf_function_ex1}

The GCC assembly output below lists registers by number:

\lstinputlisting[caption=\Optimizing GCC 4.4.5 (\assemblyOutput),style=customasmMIPS]{patterns/011_ret/MIPS.s}

\dots while \IDA does it by their pseudo names:

\lstinputlisting[caption=\Optimizing GCC 4.4.5 (IDA),style=customasmMIPS]{patterns/011_ret/MIPS_IDA.lst}

The \$2 (or \$V0) register is used to store the function's return value.
\myindex{MIPS!\Pseudoinstructions!LI}
\INS{LI} stands for ``Load Immediate'' and is the MIPS equivalent to \MOV.

\myindex{MIPS!\Instructions!J}
The other instruction is the jump instruction (J or JR) which returns the execution flow to the \gls{caller}.

\myindex{MIPS!Branch delay slot}
You might be wondering why the positions of the load instruction (LI) and the jump instruction (J or JR) are swapped. This is due to a \ac{RISC} feature called ``branch delay slot''.

The reason this happens is a quirk in the architecture of some RISC \ac{ISA}s and isn't important for our
purposes---we must simply keep in mind that in MIPS, the instruction following a jump or branch instruction
is executed \IT{before} the jump/branch instruction itself.

As a consequence, branch instructions always swap places with the instruction executed immediately beforehand.


In practice, functions which merely return 1 (\IT{true}) or 0 (\IT{false}) are very frequent.

The smallest ever of the standard UNIX utilities, \IT{/bin/true} and \IT{/bin/false} return 0 and 1 respectively, as an exit code.
(Zero as an exit code usually means success, non-zero means error.)
}
\RU{\subsubsection{std::string}
\myindex{\Cpp!STL!std::string}
\label{std_string}

\myparagraph{Как устроена структура}

Многие строковые библиотеки \InSqBrackets{\CNotes 2.2} обеспечивают структуру содержащую ссылку 
на буфер собственно со строкой, переменная всегда содержащую длину строки 
(что очень удобно для массы функций \InSqBrackets{\CNotes 2.2.1}) и переменную содержащую текущий размер буфера.

Строка в буфере обыкновенно оканчивается нулем: это для того чтобы указатель на буфер можно было
передавать в функции требующие на вход обычную сишную \ac{ASCIIZ}-строку.

Стандарт \Cpp не описывает, как именно нужно реализовывать std::string,
но, как правило, они реализованы как описано выше, с небольшими дополнениями.

Строки в \Cpp это не класс (как, например, QString в Qt), а темплейт (basic\_string), 
это сделано для того чтобы поддерживать 
строки содержащие разного типа символы: как минимум \Tchar и \IT{wchar\_t}.

Так что, std::string это класс с базовым типом \Tchar.

А std::wstring это класс с базовым типом \IT{wchar\_t}.

\mysubparagraph{MSVC}

В реализации MSVC, вместо ссылки на буфер может содержаться сам буфер (если строка короче 16-и символов).

Это означает, что каждая короткая строка будет занимать в памяти по крайней мере $16 + 4 + 4 = 24$ 
байт для 32-битной среды либо $16 + 8 + 8 = 32$ 
байта в 64-битной, а если строка длиннее 16-и символов, то прибавьте еще длину самой строки.

\lstinputlisting[caption=пример для MSVC,style=customc]{\CURPATH/STL/string/MSVC_RU.cpp}

Собственно, из этого исходника почти всё ясно.

Несколько замечаний:

Если строка короче 16-и символов, 
то отдельный буфер для строки в \glslink{heap}{куче} выделяться не будет.

Это удобно потому что на практике, основная часть строк действительно короткие.
Вероятно, разработчики в Microsoft выбрали размер в 16 символов как разумный баланс.

Теперь очень важный момент в конце функции main(): мы не пользуемся методом c\_str(), тем не менее,
если это скомпилировать и запустить, то обе строки появятся в консоли!

Работает это вот почему.

В первом случае строка короче 16-и символов и в начале объекта std::string (его можно рассматривать
просто как структуру) расположен буфер с этой строкой.
\printf трактует указатель как указатель на массив символов оканчивающийся нулем и поэтому всё работает.

Вывод второй строки (длиннее 16-и символов) даже еще опаснее: это вообще типичная программистская ошибка 
(или опечатка), забыть дописать c\_str().
Это работает потому что в это время в начале структуры расположен указатель на буфер.
Это может надолго остаться незамеченным: до тех пока там не появится строка 
короче 16-и символов, тогда процесс упадет.

\mysubparagraph{GCC}

В реализации GCC в структуре есть еще одна переменная --- reference count.

Интересно, что указатель на экземпляр класса std::string в GCC указывает не на начало самой структуры, 
а на указатель на буфера.
В libstdc++-v3\textbackslash{}include\textbackslash{}bits\textbackslash{}basic\_string.h 
мы можем прочитать что это сделано для удобства отладки:

\begin{lstlisting}
   *  The reason you want _M_data pointing to the character %array and
   *  not the _Rep is so that the debugger can see the string
   *  contents. (Probably we should add a non-inline member to get
   *  the _Rep for the debugger to use, so users can check the actual
   *  string length.)
\end{lstlisting}

\href{http://go.yurichev.com/17085}{исходный код basic\_string.h}

В нашем примере мы учитываем это:

\lstinputlisting[caption=пример для GCC,style=customc]{\CURPATH/STL/string/GCC_RU.cpp}

Нужны еще небольшие хаки чтобы сымитировать типичную ошибку, которую мы уже видели выше, из-за
более ужесточенной проверки типов в GCC, тем не менее, printf() работает и здесь без c\_str().

\myparagraph{Чуть более сложный пример}

\lstinputlisting[style=customc]{\CURPATH/STL/string/3.cpp}

\lstinputlisting[caption=MSVC 2012,style=customasmx86]{\CURPATH/STL/string/3_MSVC_RU.asm}

Собственно, компилятор не конструирует строки статически: да в общем-то и как
это возможно, если буфер с ней нужно хранить в \glslink{heap}{куче}?

Вместо этого в сегменте данных хранятся обычные \ac{ASCIIZ}-строки, а позже, во время выполнения, 
при помощи метода \q{assign}, конструируются строки s1 и s2
.
При помощи \TT{operator+}, создается строка s3.

Обратите внимание на то что вызов метода c\_str() отсутствует,
потому что его код достаточно короткий и компилятор вставил его прямо здесь:
если строка короче 16-и байт, то в регистре EAX остается указатель на буфер,
а если длиннее, то из этого же места достается адрес на буфер расположенный в \glslink{heap}{куче}.

Далее следуют вызовы трех деструкторов, причем, они вызываются только если строка длиннее 16-и байт:
тогда нужно освободить буфера в \glslink{heap}{куче}.
В противном случае, так как все три объекта std::string хранятся в стеке,
они освобождаются автоматически после выхода из функции.

Следовательно, работа с короткими строками более быстрая из-за м\'{е}ньшего обращения к \glslink{heap}{куче}.

Код на GCC даже проще (из-за того, что в GCC, как мы уже видели, не реализована возможность хранить короткую
строку прямо в структуре):

% TODO1 comment each function meaning
\lstinputlisting[caption=GCC 4.8.1,style=customasmx86]{\CURPATH/STL/string/3_GCC_RU.s}

Можно заметить, что в деструкторы передается не указатель на объект,
а указатель на место за 12 байт (или 3 слова) перед ним, то есть, на настоящее начало структуры.

\myparagraph{std::string как глобальная переменная}
\label{sec:std_string_as_global_variable}

Опытные программисты на \Cpp знают, что глобальные переменные \ac{STL}-типов вполне можно объявлять.

Да, действительно:

\lstinputlisting[style=customc]{\CURPATH/STL/string/5.cpp}

Но как и где будет вызываться конструктор \TT{std::string}?

На самом деле, эта переменная будет инициализирована даже перед началом \main.

\lstinputlisting[caption=MSVC 2012: здесь конструируется глобальная переменная{,} а также регистрируется её деструктор,style=customasmx86]{\CURPATH/STL/string/5_MSVC_p2.asm}

\lstinputlisting[caption=MSVC 2012: здесь глобальная переменная используется в \main,style=customasmx86]{\CURPATH/STL/string/5_MSVC_p1.asm}

\lstinputlisting[caption=MSVC 2012: эта функция-деструктор вызывается перед выходом,style=customasmx86]{\CURPATH/STL/string/5_MSVC_p3.asm}

\myindex{\CStandardLibrary!atexit()}
В реальности, из \ac{CRT}, еще до вызова main(), вызывается специальная функция,
в которой перечислены все конструкторы подобных переменных.
Более того: при помощи atexit() регистрируется функция, которая будет вызвана в конце работы программы:
в этой функции компилятор собирает вызовы деструкторов всех подобных глобальных переменных.

GCC работает похожим образом:

\lstinputlisting[caption=GCC 4.8.1,style=customasmx86]{\CURPATH/STL/string/5_GCC.s}

Но он не выделяет отдельной функции в которой будут собраны деструкторы: 
каждый деструктор передается в atexit() по одному.

% TODO а если глобальная STL-переменная в другом модуле? надо проверить.

}
\DE{\subsection{Einfachste XOR-Verschlüsselung überhaupt}

Ich habe einmal eine Software gesehen, bei der alle Debugging-Ausgaben mit XOR mit dem Wert 3
verschlüsselt wurden. Mit anderen Worten, die beiden niedrigsten Bits aller Buchstaben wurden invertiert.

``Hello, world'' wurde zu ``Kfool/\#tlqog'':

\begin{lstlisting}
#!/usr/bin/python

msg="Hello, world!"

print "".join(map(lambda x: chr(ord(x)^3), msg))
\end{lstlisting}

Das ist eine ziemlich interessante Verschlüsselung (oder besser eine Verschleierung),
weil sie zwei wichtige Eigenschaften hat:
1) es ist eine einzige Funktion zum Verschlüsseln und entschlüsseln, sie muss nur wiederholt angewendet werden
2) die entstehenden Buchstaben befinden sich im druckbaren Bereich, also die ganze Zeichenkette kann ohne
Escape-Symbole im Code verwendet werden.

Die zweite Eigenschaft nutzt die Tatsache, dass alle druckbaren Zeichen in Reihen organisiert sind: 0x2x-0x7x,
und wenn die beiden niederwertigsten Bits invertiert werden, wird der Buchstabe um eine oder drei Stellen nach
links oder rechts \IT{verschoben}, aber niemals in eine andere Reihe:

\begin{figure}[H]
\centering
\includegraphics[width=0.7\textwidth]{ascii_clean.png}
\caption{7-Bit \ac{ASCII} Tabelle in Emacs}
\end{figure}

\dots mit dem Zeichen 0x7F als einziger Ausnahme.

Im Folgenden werden also beispielsweise die Zeichen A-Z \IT{verschlüsselt}:

\begin{lstlisting}
#!/usr/bin/python

msg="@ABCDEFGHIJKLMNO"

print "".join(map(lambda x: chr(ord(x)^3), msg))
\end{lstlisting}

Ergebnis:
% FIXME \verb  --  relevant comment for German?
\begin{lstlisting}
CBA@GFEDKJIHONML
\end{lstlisting}

Es sieht so aus als würden die Zeichen ``@'' und ``C'' sowie ``B'' und ``A'' vertauscht werden.

Hier ist noch ein interessantes Beispiel, in dem gezeigt wird, wie die Eigenschaften von XOR
ausgenutzt werden können: Exakt den gleichen Effekt, dass druckbare Zeichen auch druckbar bleiben,
kann man dadurch erzielen, dass irgendeine Kombination der niedrigsten vier Bits invertiert wird.
}

\EN{\section{Returning Values}
\label{ret_val_func}

Another simple function is the one that simply returns a constant value:

\lstinputlisting[caption=\EN{\CCpp Code},style=customc]{patterns/011_ret/1.c}

Let's compile it.

\subsection{x86}

Here's what both the GCC and MSVC compilers produce (with optimization) on the x86 platform:

\lstinputlisting[caption=\Optimizing GCC/MSVC (\assemblyOutput),style=customasmx86]{patterns/011_ret/1.s}

\myindex{x86!\Instructions!RET}
There are just two instructions: the first places the value 123 into the \EAX register,
which is used by convention for storing the return
value, and the second one is \RET, which returns execution to the \gls{caller}.

The caller will take the result from the \EAX register.

\subsection{ARM}

There are a few differences on the ARM platform:

\lstinputlisting[caption=\OptimizingKeilVI (\ARMMode) ASM Output,style=customasmARM]{patterns/011_ret/1_Keil_ARM_O3.s}

ARM uses the register \Reg{0} for returning the results of functions, so 123 is copied into \Reg{0}.

\myindex{ARM!\Instructions!MOV}
\myindex{x86!\Instructions!MOV}
It is worth noting that \MOV is a misleading name for the instruction in both the x86 and ARM \ac{ISA}s.

The data is not in fact \IT{moved}, but \IT{copied}.

\subsection{MIPS}

\label{MIPS_leaf_function_ex1}

The GCC assembly output below lists registers by number:

\lstinputlisting[caption=\Optimizing GCC 4.4.5 (\assemblyOutput),style=customasmMIPS]{patterns/011_ret/MIPS.s}

\dots while \IDA does it by their pseudo names:

\lstinputlisting[caption=\Optimizing GCC 4.4.5 (IDA),style=customasmMIPS]{patterns/011_ret/MIPS_IDA.lst}

The \$2 (or \$V0) register is used to store the function's return value.
\myindex{MIPS!\Pseudoinstructions!LI}
\INS{LI} stands for ``Load Immediate'' and is the MIPS equivalent to \MOV.

\myindex{MIPS!\Instructions!J}
The other instruction is the jump instruction (J or JR) which returns the execution flow to the \gls{caller}.

\myindex{MIPS!Branch delay slot}
You might be wondering why the positions of the load instruction (LI) and the jump instruction (J or JR) are swapped. This is due to a \ac{RISC} feature called ``branch delay slot''.

The reason this happens is a quirk in the architecture of some RISC \ac{ISA}s and isn't important for our
purposes---we must simply keep in mind that in MIPS, the instruction following a jump or branch instruction
is executed \IT{before} the jump/branch instruction itself.

As a consequence, branch instructions always swap places with the instruction executed immediately beforehand.


In practice, functions which merely return 1 (\IT{true}) or 0 (\IT{false}) are very frequent.

The smallest ever of the standard UNIX utilities, \IT{/bin/true} and \IT{/bin/false} return 0 and 1 respectively, as an exit code.
(Zero as an exit code usually means success, non-zero means error.)
}
\RU{\subsubsection{std::string}
\myindex{\Cpp!STL!std::string}
\label{std_string}

\myparagraph{Как устроена структура}

Многие строковые библиотеки \InSqBrackets{\CNotes 2.2} обеспечивают структуру содержащую ссылку 
на буфер собственно со строкой, переменная всегда содержащую длину строки 
(что очень удобно для массы функций \InSqBrackets{\CNotes 2.2.1}) и переменную содержащую текущий размер буфера.

Строка в буфере обыкновенно оканчивается нулем: это для того чтобы указатель на буфер можно было
передавать в функции требующие на вход обычную сишную \ac{ASCIIZ}-строку.

Стандарт \Cpp не описывает, как именно нужно реализовывать std::string,
но, как правило, они реализованы как описано выше, с небольшими дополнениями.

Строки в \Cpp это не класс (как, например, QString в Qt), а темплейт (basic\_string), 
это сделано для того чтобы поддерживать 
строки содержащие разного типа символы: как минимум \Tchar и \IT{wchar\_t}.

Так что, std::string это класс с базовым типом \Tchar.

А std::wstring это класс с базовым типом \IT{wchar\_t}.

\mysubparagraph{MSVC}

В реализации MSVC, вместо ссылки на буфер может содержаться сам буфер (если строка короче 16-и символов).

Это означает, что каждая короткая строка будет занимать в памяти по крайней мере $16 + 4 + 4 = 24$ 
байт для 32-битной среды либо $16 + 8 + 8 = 32$ 
байта в 64-битной, а если строка длиннее 16-и символов, то прибавьте еще длину самой строки.

\lstinputlisting[caption=пример для MSVC,style=customc]{\CURPATH/STL/string/MSVC_RU.cpp}

Собственно, из этого исходника почти всё ясно.

Несколько замечаний:

Если строка короче 16-и символов, 
то отдельный буфер для строки в \glslink{heap}{куче} выделяться не будет.

Это удобно потому что на практике, основная часть строк действительно короткие.
Вероятно, разработчики в Microsoft выбрали размер в 16 символов как разумный баланс.

Теперь очень важный момент в конце функции main(): мы не пользуемся методом c\_str(), тем не менее,
если это скомпилировать и запустить, то обе строки появятся в консоли!

Работает это вот почему.

В первом случае строка короче 16-и символов и в начале объекта std::string (его можно рассматривать
просто как структуру) расположен буфер с этой строкой.
\printf трактует указатель как указатель на массив символов оканчивающийся нулем и поэтому всё работает.

Вывод второй строки (длиннее 16-и символов) даже еще опаснее: это вообще типичная программистская ошибка 
(или опечатка), забыть дописать c\_str().
Это работает потому что в это время в начале структуры расположен указатель на буфер.
Это может надолго остаться незамеченным: до тех пока там не появится строка 
короче 16-и символов, тогда процесс упадет.

\mysubparagraph{GCC}

В реализации GCC в структуре есть еще одна переменная --- reference count.

Интересно, что указатель на экземпляр класса std::string в GCC указывает не на начало самой структуры, 
а на указатель на буфера.
В libstdc++-v3\textbackslash{}include\textbackslash{}bits\textbackslash{}basic\_string.h 
мы можем прочитать что это сделано для удобства отладки:

\begin{lstlisting}
   *  The reason you want _M_data pointing to the character %array and
   *  not the _Rep is so that the debugger can see the string
   *  contents. (Probably we should add a non-inline member to get
   *  the _Rep for the debugger to use, so users can check the actual
   *  string length.)
\end{lstlisting}

\href{http://go.yurichev.com/17085}{исходный код basic\_string.h}

В нашем примере мы учитываем это:

\lstinputlisting[caption=пример для GCC,style=customc]{\CURPATH/STL/string/GCC_RU.cpp}

Нужны еще небольшие хаки чтобы сымитировать типичную ошибку, которую мы уже видели выше, из-за
более ужесточенной проверки типов в GCC, тем не менее, printf() работает и здесь без c\_str().

\myparagraph{Чуть более сложный пример}

\lstinputlisting[style=customc]{\CURPATH/STL/string/3.cpp}

\lstinputlisting[caption=MSVC 2012,style=customasmx86]{\CURPATH/STL/string/3_MSVC_RU.asm}

Собственно, компилятор не конструирует строки статически: да в общем-то и как
это возможно, если буфер с ней нужно хранить в \glslink{heap}{куче}?

Вместо этого в сегменте данных хранятся обычные \ac{ASCIIZ}-строки, а позже, во время выполнения, 
при помощи метода \q{assign}, конструируются строки s1 и s2
.
При помощи \TT{operator+}, создается строка s3.

Обратите внимание на то что вызов метода c\_str() отсутствует,
потому что его код достаточно короткий и компилятор вставил его прямо здесь:
если строка короче 16-и байт, то в регистре EAX остается указатель на буфер,
а если длиннее, то из этого же места достается адрес на буфер расположенный в \glslink{heap}{куче}.

Далее следуют вызовы трех деструкторов, причем, они вызываются только если строка длиннее 16-и байт:
тогда нужно освободить буфера в \glslink{heap}{куче}.
В противном случае, так как все три объекта std::string хранятся в стеке,
они освобождаются автоматически после выхода из функции.

Следовательно, работа с короткими строками более быстрая из-за м\'{е}ньшего обращения к \glslink{heap}{куче}.

Код на GCC даже проще (из-за того, что в GCC, как мы уже видели, не реализована возможность хранить короткую
строку прямо в структуре):

% TODO1 comment each function meaning
\lstinputlisting[caption=GCC 4.8.1,style=customasmx86]{\CURPATH/STL/string/3_GCC_RU.s}

Можно заметить, что в деструкторы передается не указатель на объект,
а указатель на место за 12 байт (или 3 слова) перед ним, то есть, на настоящее начало структуры.

\myparagraph{std::string как глобальная переменная}
\label{sec:std_string_as_global_variable}

Опытные программисты на \Cpp знают, что глобальные переменные \ac{STL}-типов вполне можно объявлять.

Да, действительно:

\lstinputlisting[style=customc]{\CURPATH/STL/string/5.cpp}

Но как и где будет вызываться конструктор \TT{std::string}?

На самом деле, эта переменная будет инициализирована даже перед началом \main.

\lstinputlisting[caption=MSVC 2012: здесь конструируется глобальная переменная{,} а также регистрируется её деструктор,style=customasmx86]{\CURPATH/STL/string/5_MSVC_p2.asm}

\lstinputlisting[caption=MSVC 2012: здесь глобальная переменная используется в \main,style=customasmx86]{\CURPATH/STL/string/5_MSVC_p1.asm}

\lstinputlisting[caption=MSVC 2012: эта функция-деструктор вызывается перед выходом,style=customasmx86]{\CURPATH/STL/string/5_MSVC_p3.asm}

\myindex{\CStandardLibrary!atexit()}
В реальности, из \ac{CRT}, еще до вызова main(), вызывается специальная функция,
в которой перечислены все конструкторы подобных переменных.
Более того: при помощи atexit() регистрируется функция, которая будет вызвана в конце работы программы:
в этой функции компилятор собирает вызовы деструкторов всех подобных глобальных переменных.

GCC работает похожим образом:

\lstinputlisting[caption=GCC 4.8.1,style=customasmx86]{\CURPATH/STL/string/5_GCC.s}

Но он не выделяет отдельной функции в которой будут собраны деструкторы: 
каждый деструктор передается в atexit() по одному.

% TODO а если глобальная STL-переменная в другом модуле? надо проверить.

}
\DE{\subsection{Einfachste XOR-Verschlüsselung überhaupt}

Ich habe einmal eine Software gesehen, bei der alle Debugging-Ausgaben mit XOR mit dem Wert 3
verschlüsselt wurden. Mit anderen Worten, die beiden niedrigsten Bits aller Buchstaben wurden invertiert.

``Hello, world'' wurde zu ``Kfool/\#tlqog'':

\begin{lstlisting}
#!/usr/bin/python

msg="Hello, world!"

print "".join(map(lambda x: chr(ord(x)^3), msg))
\end{lstlisting}

Das ist eine ziemlich interessante Verschlüsselung (oder besser eine Verschleierung),
weil sie zwei wichtige Eigenschaften hat:
1) es ist eine einzige Funktion zum Verschlüsseln und entschlüsseln, sie muss nur wiederholt angewendet werden
2) die entstehenden Buchstaben befinden sich im druckbaren Bereich, also die ganze Zeichenkette kann ohne
Escape-Symbole im Code verwendet werden.

Die zweite Eigenschaft nutzt die Tatsache, dass alle druckbaren Zeichen in Reihen organisiert sind: 0x2x-0x7x,
und wenn die beiden niederwertigsten Bits invertiert werden, wird der Buchstabe um eine oder drei Stellen nach
links oder rechts \IT{verschoben}, aber niemals in eine andere Reihe:

\begin{figure}[H]
\centering
\includegraphics[width=0.7\textwidth]{ascii_clean.png}
\caption{7-Bit \ac{ASCII} Tabelle in Emacs}
\end{figure}

\dots mit dem Zeichen 0x7F als einziger Ausnahme.

Im Folgenden werden also beispielsweise die Zeichen A-Z \IT{verschlüsselt}:

\begin{lstlisting}
#!/usr/bin/python

msg="@ABCDEFGHIJKLMNO"

print "".join(map(lambda x: chr(ord(x)^3), msg))
\end{lstlisting}

Ergebnis:
% FIXME \verb  --  relevant comment for German?
\begin{lstlisting}
CBA@GFEDKJIHONML
\end{lstlisting}

Es sieht so aus als würden die Zeichen ``@'' und ``C'' sowie ``B'' und ``A'' vertauscht werden.

Hier ist noch ein interessantes Beispiel, in dem gezeigt wird, wie die Eigenschaften von XOR
ausgenutzt werden können: Exakt den gleichen Effekt, dass druckbare Zeichen auch druckbar bleiben,
kann man dadurch erzielen, dass irgendeine Kombination der niedrigsten vier Bits invertiert wird.
}

\ifdefined\SPANISH
\chapter{Patrones de código}
\fi % SPANISH

\ifdefined\GERMAN
\chapter{Code-Muster}
\fi % GERMAN

\ifdefined\ENGLISH
\chapter{Code Patterns}
\fi % ENGLISH

\ifdefined\ITALIAN
\chapter{Forme di codice}
\fi % ITALIAN

\ifdefined\RUSSIAN
\chapter{Образцы кода}
\fi % RUSSIAN

\ifdefined\BRAZILIAN
\chapter{Padrões de códigos}
\fi % BRAZILIAN

\ifdefined\THAI
\chapter{รูปแบบของโค้ด}
\fi % THAI

\ifdefined\FRENCH
\chapter{Modèle de code}
\fi % FRENCH

\ifdefined\POLISH
\chapter{\PLph{}}
\fi % POLISH

% sections
\EN{\input{patterns/patterns_opt_dbg_EN}}
\ES{\input{patterns/patterns_opt_dbg_ES}}
\ITA{\input{patterns/patterns_opt_dbg_ITA}}
\PTBR{\input{patterns/patterns_opt_dbg_PTBR}}
\RU{\input{patterns/patterns_opt_dbg_RU}}
\THA{\input{patterns/patterns_opt_dbg_THA}}
\DE{\input{patterns/patterns_opt_dbg_DE}}
\FR{\input{patterns/patterns_opt_dbg_FR}}
\PL{\input{patterns/patterns_opt_dbg_PL}}

\RU{\section{Некоторые базовые понятия}}
\EN{\section{Some basics}}
\DE{\section{Einige Grundlagen}}
\FR{\section{Quelques bases}}
\ES{\section{\ESph{}}}
\ITA{\section{Alcune basi teoriche}}
\PTBR{\section{\PTBRph{}}}
\THA{\section{\THAph{}}}
\PL{\section{\PLph{}}}

% sections:
\EN{\input{patterns/intro_CPU_ISA_EN}}
\ES{\input{patterns/intro_CPU_ISA_ES}}
\ITA{\input{patterns/intro_CPU_ISA_ITA}}
\PTBR{\input{patterns/intro_CPU_ISA_PTBR}}
\RU{\input{patterns/intro_CPU_ISA_RU}}
\DE{\input{patterns/intro_CPU_ISA_DE}}
\FR{\input{patterns/intro_CPU_ISA_FR}}
\PL{\input{patterns/intro_CPU_ISA_PL}}

\EN{\input{patterns/numeral_EN}}
\RU{\input{patterns/numeral_RU}}
\ITA{\input{patterns/numeral_ITA}}
\DE{\input{patterns/numeral_DE}}
\FR{\input{patterns/numeral_FR}}
\PL{\input{patterns/numeral_PL}}

% chapters
\input{patterns/00_empty/main}
\input{patterns/011_ret/main}
\input{patterns/01_helloworld/main}
\input{patterns/015_prolog_epilogue/main}
\input{patterns/02_stack/main}
\input{patterns/03_printf/main}
\input{patterns/04_scanf/main}
\input{patterns/05_passing_arguments/main}
\input{patterns/06_return_results/main}
\input{patterns/061_pointers/main}
\input{patterns/065_GOTO/main}
\input{patterns/07_jcc/main}
\input{patterns/08_switch/main}
\input{patterns/09_loops/main}
\input{patterns/10_strings/main}
\input{patterns/11_arith_optimizations/main}
\input{patterns/12_FPU/main}
\input{patterns/13_arrays/main}
\input{patterns/14_bitfields/main}
\EN{\input{patterns/145_LCG/main_EN}}
\RU{\input{patterns/145_LCG/main_RU}}
\input{patterns/15_structs/main}
\input{patterns/17_unions/main}
\input{patterns/18_pointers_to_functions/main}
\input{patterns/185_64bit_in_32_env/main}

\EN{\input{patterns/19_SIMD/main_EN}}
\RU{\input{patterns/19_SIMD/main_RU}}
\DE{\input{patterns/19_SIMD/main_DE}}

\EN{\input{patterns/20_x64/main_EN}}
\RU{\input{patterns/20_x64/main_RU}}

\EN{\input{patterns/205_floating_SIMD/main_EN}}
\RU{\input{patterns/205_floating_SIMD/main_RU}}
\DE{\input{patterns/205_floating_SIMD/main_DE}}

\EN{\input{patterns/ARM/main_EN}}
\RU{\input{patterns/ARM/main_RU}}
\DE{\input{patterns/ARM/main_DE}}

\input{patterns/MIPS/main}


\EN{\section{Returning Values}
\label{ret_val_func}

Another simple function is the one that simply returns a constant value:

\lstinputlisting[caption=\EN{\CCpp Code},style=customc]{patterns/011_ret/1.c}

Let's compile it.

\subsection{x86}

Here's what both the GCC and MSVC compilers produce (with optimization) on the x86 platform:

\lstinputlisting[caption=\Optimizing GCC/MSVC (\assemblyOutput),style=customasmx86]{patterns/011_ret/1.s}

\myindex{x86!\Instructions!RET}
There are just two instructions: the first places the value 123 into the \EAX register,
which is used by convention for storing the return
value, and the second one is \RET, which returns execution to the \gls{caller}.

The caller will take the result from the \EAX register.

\subsection{ARM}

There are a few differences on the ARM platform:

\lstinputlisting[caption=\OptimizingKeilVI (\ARMMode) ASM Output,style=customasmARM]{patterns/011_ret/1_Keil_ARM_O3.s}

ARM uses the register \Reg{0} for returning the results of functions, so 123 is copied into \Reg{0}.

\myindex{ARM!\Instructions!MOV}
\myindex{x86!\Instructions!MOV}
It is worth noting that \MOV is a misleading name for the instruction in both the x86 and ARM \ac{ISA}s.

The data is not in fact \IT{moved}, but \IT{copied}.

\subsection{MIPS}

\label{MIPS_leaf_function_ex1}

The GCC assembly output below lists registers by number:

\lstinputlisting[caption=\Optimizing GCC 4.4.5 (\assemblyOutput),style=customasmMIPS]{patterns/011_ret/MIPS.s}

\dots while \IDA does it by their pseudo names:

\lstinputlisting[caption=\Optimizing GCC 4.4.5 (IDA),style=customasmMIPS]{patterns/011_ret/MIPS_IDA.lst}

The \$2 (or \$V0) register is used to store the function's return value.
\myindex{MIPS!\Pseudoinstructions!LI}
\INS{LI} stands for ``Load Immediate'' and is the MIPS equivalent to \MOV.

\myindex{MIPS!\Instructions!J}
The other instruction is the jump instruction (J or JR) which returns the execution flow to the \gls{caller}.

\myindex{MIPS!Branch delay slot}
You might be wondering why the positions of the load instruction (LI) and the jump instruction (J or JR) are swapped. This is due to a \ac{RISC} feature called ``branch delay slot''.

The reason this happens is a quirk in the architecture of some RISC \ac{ISA}s and isn't important for our
purposes---we must simply keep in mind that in MIPS, the instruction following a jump or branch instruction
is executed \IT{before} the jump/branch instruction itself.

As a consequence, branch instructions always swap places with the instruction executed immediately beforehand.


In practice, functions which merely return 1 (\IT{true}) or 0 (\IT{false}) are very frequent.

The smallest ever of the standard UNIX utilities, \IT{/bin/true} and \IT{/bin/false} return 0 and 1 respectively, as an exit code.
(Zero as an exit code usually means success, non-zero means error.)
}
\RU{\subsubsection{std::string}
\myindex{\Cpp!STL!std::string}
\label{std_string}

\myparagraph{Как устроена структура}

Многие строковые библиотеки \InSqBrackets{\CNotes 2.2} обеспечивают структуру содержащую ссылку 
на буфер собственно со строкой, переменная всегда содержащую длину строки 
(что очень удобно для массы функций \InSqBrackets{\CNotes 2.2.1}) и переменную содержащую текущий размер буфера.

Строка в буфере обыкновенно оканчивается нулем: это для того чтобы указатель на буфер можно было
передавать в функции требующие на вход обычную сишную \ac{ASCIIZ}-строку.

Стандарт \Cpp не описывает, как именно нужно реализовывать std::string,
но, как правило, они реализованы как описано выше, с небольшими дополнениями.

Строки в \Cpp это не класс (как, например, QString в Qt), а темплейт (basic\_string), 
это сделано для того чтобы поддерживать 
строки содержащие разного типа символы: как минимум \Tchar и \IT{wchar\_t}.

Так что, std::string это класс с базовым типом \Tchar.

А std::wstring это класс с базовым типом \IT{wchar\_t}.

\mysubparagraph{MSVC}

В реализации MSVC, вместо ссылки на буфер может содержаться сам буфер (если строка короче 16-и символов).

Это означает, что каждая короткая строка будет занимать в памяти по крайней мере $16 + 4 + 4 = 24$ 
байт для 32-битной среды либо $16 + 8 + 8 = 32$ 
байта в 64-битной, а если строка длиннее 16-и символов, то прибавьте еще длину самой строки.

\lstinputlisting[caption=пример для MSVC,style=customc]{\CURPATH/STL/string/MSVC_RU.cpp}

Собственно, из этого исходника почти всё ясно.

Несколько замечаний:

Если строка короче 16-и символов, 
то отдельный буфер для строки в \glslink{heap}{куче} выделяться не будет.

Это удобно потому что на практике, основная часть строк действительно короткие.
Вероятно, разработчики в Microsoft выбрали размер в 16 символов как разумный баланс.

Теперь очень важный момент в конце функции main(): мы не пользуемся методом c\_str(), тем не менее,
если это скомпилировать и запустить, то обе строки появятся в консоли!

Работает это вот почему.

В первом случае строка короче 16-и символов и в начале объекта std::string (его можно рассматривать
просто как структуру) расположен буфер с этой строкой.
\printf трактует указатель как указатель на массив символов оканчивающийся нулем и поэтому всё работает.

Вывод второй строки (длиннее 16-и символов) даже еще опаснее: это вообще типичная программистская ошибка 
(или опечатка), забыть дописать c\_str().
Это работает потому что в это время в начале структуры расположен указатель на буфер.
Это может надолго остаться незамеченным: до тех пока там не появится строка 
короче 16-и символов, тогда процесс упадет.

\mysubparagraph{GCC}

В реализации GCC в структуре есть еще одна переменная --- reference count.

Интересно, что указатель на экземпляр класса std::string в GCC указывает не на начало самой структуры, 
а на указатель на буфера.
В libstdc++-v3\textbackslash{}include\textbackslash{}bits\textbackslash{}basic\_string.h 
мы можем прочитать что это сделано для удобства отладки:

\begin{lstlisting}
   *  The reason you want _M_data pointing to the character %array and
   *  not the _Rep is so that the debugger can see the string
   *  contents. (Probably we should add a non-inline member to get
   *  the _Rep for the debugger to use, so users can check the actual
   *  string length.)
\end{lstlisting}

\href{http://go.yurichev.com/17085}{исходный код basic\_string.h}

В нашем примере мы учитываем это:

\lstinputlisting[caption=пример для GCC,style=customc]{\CURPATH/STL/string/GCC_RU.cpp}

Нужны еще небольшие хаки чтобы сымитировать типичную ошибку, которую мы уже видели выше, из-за
более ужесточенной проверки типов в GCC, тем не менее, printf() работает и здесь без c\_str().

\myparagraph{Чуть более сложный пример}

\lstinputlisting[style=customc]{\CURPATH/STL/string/3.cpp}

\lstinputlisting[caption=MSVC 2012,style=customasmx86]{\CURPATH/STL/string/3_MSVC_RU.asm}

Собственно, компилятор не конструирует строки статически: да в общем-то и как
это возможно, если буфер с ней нужно хранить в \glslink{heap}{куче}?

Вместо этого в сегменте данных хранятся обычные \ac{ASCIIZ}-строки, а позже, во время выполнения, 
при помощи метода \q{assign}, конструируются строки s1 и s2
.
При помощи \TT{operator+}, создается строка s3.

Обратите внимание на то что вызов метода c\_str() отсутствует,
потому что его код достаточно короткий и компилятор вставил его прямо здесь:
если строка короче 16-и байт, то в регистре EAX остается указатель на буфер,
а если длиннее, то из этого же места достается адрес на буфер расположенный в \glslink{heap}{куче}.

Далее следуют вызовы трех деструкторов, причем, они вызываются только если строка длиннее 16-и байт:
тогда нужно освободить буфера в \glslink{heap}{куче}.
В противном случае, так как все три объекта std::string хранятся в стеке,
они освобождаются автоматически после выхода из функции.

Следовательно, работа с короткими строками более быстрая из-за м\'{е}ньшего обращения к \glslink{heap}{куче}.

Код на GCC даже проще (из-за того, что в GCC, как мы уже видели, не реализована возможность хранить короткую
строку прямо в структуре):

% TODO1 comment each function meaning
\lstinputlisting[caption=GCC 4.8.1,style=customasmx86]{\CURPATH/STL/string/3_GCC_RU.s}

Можно заметить, что в деструкторы передается не указатель на объект,
а указатель на место за 12 байт (или 3 слова) перед ним, то есть, на настоящее начало структуры.

\myparagraph{std::string как глобальная переменная}
\label{sec:std_string_as_global_variable}

Опытные программисты на \Cpp знают, что глобальные переменные \ac{STL}-типов вполне можно объявлять.

Да, действительно:

\lstinputlisting[style=customc]{\CURPATH/STL/string/5.cpp}

Но как и где будет вызываться конструктор \TT{std::string}?

На самом деле, эта переменная будет инициализирована даже перед началом \main.

\lstinputlisting[caption=MSVC 2012: здесь конструируется глобальная переменная{,} а также регистрируется её деструктор,style=customasmx86]{\CURPATH/STL/string/5_MSVC_p2.asm}

\lstinputlisting[caption=MSVC 2012: здесь глобальная переменная используется в \main,style=customasmx86]{\CURPATH/STL/string/5_MSVC_p1.asm}

\lstinputlisting[caption=MSVC 2012: эта функция-деструктор вызывается перед выходом,style=customasmx86]{\CURPATH/STL/string/5_MSVC_p3.asm}

\myindex{\CStandardLibrary!atexit()}
В реальности, из \ac{CRT}, еще до вызова main(), вызывается специальная функция,
в которой перечислены все конструкторы подобных переменных.
Более того: при помощи atexit() регистрируется функция, которая будет вызвана в конце работы программы:
в этой функции компилятор собирает вызовы деструкторов всех подобных глобальных переменных.

GCC работает похожим образом:

\lstinputlisting[caption=GCC 4.8.1,style=customasmx86]{\CURPATH/STL/string/5_GCC.s}

Но он не выделяет отдельной функции в которой будут собраны деструкторы: 
каждый деструктор передается в atexit() по одному.

% TODO а если глобальная STL-переменная в другом модуле? надо проверить.

}
\ifdefined\SPANISH
\chapter{Patrones de código}
\fi % SPANISH

\ifdefined\GERMAN
\chapter{Code-Muster}
\fi % GERMAN

\ifdefined\ENGLISH
\chapter{Code Patterns}
\fi % ENGLISH

\ifdefined\ITALIAN
\chapter{Forme di codice}
\fi % ITALIAN

\ifdefined\RUSSIAN
\chapter{Образцы кода}
\fi % RUSSIAN

\ifdefined\BRAZILIAN
\chapter{Padrões de códigos}
\fi % BRAZILIAN

\ifdefined\THAI
\chapter{รูปแบบของโค้ด}
\fi % THAI

\ifdefined\FRENCH
\chapter{Modèle de code}
\fi % FRENCH

\ifdefined\POLISH
\chapter{\PLph{}}
\fi % POLISH

% sections
\EN{\section{The method}

When the author of this book first started learning C and, later, \Cpp, he used to write small pieces of code, compile them,
and then look at the assembly language output. This made it very easy for him to understand what was going on in the code that he had written.
\footnote{In fact, he still does this when he can't understand what a particular bit of code does.}.
He did this so many times that the relationship between the \CCpp code and what the compiler produced was imprinted deeply in his mind.
It's now easy for him to imagine instantly a rough outline of a C code's appearance and function.
Perhaps this technique could be helpful for others.

%There are a lot of examples for both x86/x64 and ARM.
%Those who already familiar with one of architectures, may freely skim over pages.

By the way, there is a great website where you can do the same, with various compilers, instead of installing them on your box.
You can use it as well: \url{https://gcc.godbolt.org/}.

\section*{\Exercises}

When the author of this book studied assembly language, he also often compiled small C functions and then rewrote
them gradually to assembly, trying to make their code as short as possible.
This probably is not worth doing in real-world scenarios today,
because it's hard to compete with the latest compilers in terms of efficiency. It is, however, a very good way to gain a better understanding of assembly.
Feel free, therefore, to take any assembly code from this book and try to make it shorter.
However, don't forget to test what you have written.

% rewrote to show that debug\release and optimisations levels are orthogonal concepts.
\section*{Optimization levels and debug information}

Source code can be compiled by different compilers with various optimization levels.
A typical compiler has about three such levels, where level zero means that optimization is completely disabled.
Optimization can also be targeted towards code size or code speed.
A non-optimizing compiler is faster and produces more understandable (albeit verbose) code,
whereas an optimizing compiler is slower and tries to produce code that runs faster (but is not necessarily more compact).
In addition to optimization levels, a compiler can include some debug information in the resulting file,
producing code that is easy to debug.
One of the important features of the ´debug' code is that it might contain links
between each line of the source code and its respective machine code address.
Optimizing compilers, on the other hand, tend to produce output where entire lines of source code
can be optimized away and thus not even be present in the resulting machine code.
Reverse engineers can encounter either version, simply because some developers turn on the compiler's optimization flags and others do not.
Because of this, we'll try to work on examples of both debug and release versions of the code featured in this book, wherever possible.

Sometimes some pretty ancient compilers are used in this book, in order to get the shortest (or simplest) possible code snippet.
}
\ES{\input{patterns/patterns_opt_dbg_ES}}
\ITA{\input{patterns/patterns_opt_dbg_ITA}}
\PTBR{\input{patterns/patterns_opt_dbg_PTBR}}
\RU{\input{patterns/patterns_opt_dbg_RU}}
\THA{\input{patterns/patterns_opt_dbg_THA}}
\DE{\input{patterns/patterns_opt_dbg_DE}}
\FR{\input{patterns/patterns_opt_dbg_FR}}
\PL{\input{patterns/patterns_opt_dbg_PL}}

\RU{\section{Некоторые базовые понятия}}
\EN{\section{Some basics}}
\DE{\section{Einige Grundlagen}}
\FR{\section{Quelques bases}}
\ES{\section{\ESph{}}}
\ITA{\section{Alcune basi teoriche}}
\PTBR{\section{\PTBRph{}}}
\THA{\section{\THAph{}}}
\PL{\section{\PLph{}}}

% sections:
\EN{\input{patterns/intro_CPU_ISA_EN}}
\ES{\input{patterns/intro_CPU_ISA_ES}}
\ITA{\input{patterns/intro_CPU_ISA_ITA}}
\PTBR{\input{patterns/intro_CPU_ISA_PTBR}}
\RU{\input{patterns/intro_CPU_ISA_RU}}
\DE{\input{patterns/intro_CPU_ISA_DE}}
\FR{\input{patterns/intro_CPU_ISA_FR}}
\PL{\input{patterns/intro_CPU_ISA_PL}}

\EN{\subsection{Numeral Systems}

Humans have become accustomed to a decimal numeral system, probably because almost everyone has 10 fingers.
Nevertheless, the number \q{10} has no significant meaning in science and mathematics.
The natural numeral system in digital electronics is binary: 0 is for an absence of current in the wire, and 1 for presence.
10 in binary is 2 in decimal, 100 in binary is 4 in decimal, and so on.

% This sentence is a bit unweildy - maybe try 'Our ten-digit system would be described as having a radix...' - Renaissance
If the numeral system has 10 digits, it has a \IT{radix} (or \IT{base}) of 10.
The binary numeral system has a \IT{radix} of 2.

Important things to recall:

1) A \IT{number} is a number, while a \IT{digit} is a term from writing systems, and is usually one character

% The original is 'number' is not changed; I think the intent is value, and changed it - Renaissance
2) The value of a number does not change when converted to another radix; only the writing notation for that value has changed (and therefore the way of representing it in \ac{RAM}).

\subsection{Converting From One Radix To Another}

Positional notation is used almost every numerical system. This means that a digit has weight relative to where it is placed inside of the larger number.
If 2 is placed at the rightmost place, it's 2, but if it's placed one digit before rightmost, it's 20.

What does $1234$ stand for?

$10^3 \cdot 1 + 10^2 \cdot 2 + 10^1 \cdot 3 + 1 \cdot 4 = 1234$ or
$1000 \cdot 1 + 100 \cdot 2 + 10 \cdot 3 + 4 = 1234$

It's the same story for binary numbers, but the base is 2 instead of 10.
What does 0b101011 stand for?

$2^5 \cdot 1 + 2^4 \cdot 0 + 2^3 \cdot 1 + 2^2 \cdot 0 + 2^1 \cdot 1 + 2^0 \cdot 1 = 43$ or
$32 \cdot 1 + 16 \cdot 0 + 8 \cdot 1 + 4 \cdot 0 + 2 \cdot 1 + 1 = 43$

There is such a thing as non-positional notation, such as the Roman numeral system.
\footnote{About numeric system evolution, see \InSqBrackets{\TAOCPvolII{}, 195--213.}}.
% Maybe add a sentence to fill in that X is always 10, and is therefore non-positional, even though putting an I before subtracts and after adds, and is in that sense positional
Perhaps, humankind switched to positional notation because it's easier to do basic operations (addition, multiplication, etc.) on paper by hand.

Binary numbers can be added, subtracted and so on in the very same as taught in schools, but only 2 digits are available.

Binary numbers are bulky when represented in source code and dumps, so that is where the hexadecimal numeral system can be useful.
A hexadecimal radix uses the digits 0..9, and also 6 Latin characters: A..F.
Each hexadecimal digit takes 4 bits or 4 binary digits, so it's very easy to convert from binary number to hexadecimal and back, even manually, in one's mind.

\begin{center}
\begin{longtable}{ | l | l | l | }
\hline
\HeaderColor hexadecimal & \HeaderColor binary & \HeaderColor decimal \\
\hline
0	&0000	&0 \\
1	&0001	&1 \\
2	&0010	&2 \\
3	&0011	&3 \\
4	&0100	&4 \\
5	&0101	&5 \\
6	&0110	&6 \\
7	&0111	&7 \\
8	&1000	&8 \\
9	&1001	&9 \\
A	&1010	&10 \\
B	&1011	&11 \\
C	&1100	&12 \\
D	&1101	&13 \\
E	&1110	&14 \\
F	&1111	&15 \\
\hline
\end{longtable}
\end{center}

How can one tell which radix is being used in a specific instance?

Decimal numbers are usually written as is, i.e., 1234. Some assemblers allow an identifier on decimal radix numbers, in which the number would be written with a "d" suffix: 1234d.

Binary numbers are sometimes prepended with the "0b" prefix: 0b100110111 (\ac{GCC} has a non-standard language extension for this\footnote{\url{https://gcc.gnu.org/onlinedocs/gcc/Binary-constants.html}}).
There is also another way: using a "b" suffix, for example: 100110111b.
This book tries to use the "0b" prefix consistently throughout the book for binary numbers.

Hexadecimal numbers are prepended with "0x" prefix in \CCpp and other \ac{PL}s: 0x1234ABCD.
Alternatively, they are given a "h" suffix: 1234ABCDh. This is common way of representing them in assemblers and debuggers.
In this convention, if the number is started with a Latin (A..F) digit, a 0 is added at the beginning: 0ABCDEFh.
There was also convention that was popular in 8-bit home computers era, using \$ prefix, like \$ABCD.
The book will try to stick to "0x" prefix throughout the book for hexadecimal numbers.

Should one learn to convert numbers mentally? A table of 1-digit hexadecimal numbers can easily be memorized.
As for larger numbers, it's probably not worth tormenting yourself.

Perhaps the most visible hexadecimal numbers are in \ac{URL}s.
This is the way that non-Latin characters are encoded.
For example:
\url{https://en.wiktionary.org/wiki/na\%C3\%AFvet\%C3\%A9} is the \ac{URL} of Wiktionary article about \q{naïveté} word.

\subsubsection{Octal Radix}

Another numeral system heavily used in the past of computer programming is octal. In octal there are 8 digits (0..7), and each is mapped to 3 bits, so it's easy to convert numbers back and forth.
It has been superseded by the hexadecimal system almost everywhere, but, surprisingly, there is a *NIX utility, used often by many people, which takes octal numbers as argument: \TT{chmod}.

\myindex{UNIX!chmod}
As many *NIX users know, \TT{chmod} argument can be a number of 3 digits. The first digit represents the rights of the owner of the file (read, write and/or execute), the second is the rights for the group to which the file belongs, and the third is for everyone else.
Each digit that \TT{chmod} takes can be represented in binary form:

\begin{center}
\begin{longtable}{ | l | l | l | }
\hline
\HeaderColor decimal & \HeaderColor binary & \HeaderColor meaning \\
\hline
7	&111	&\textbf{rwx} \\
6	&110	&\textbf{rw-} \\
5	&101	&\textbf{r-x} \\
4	&100	&\textbf{r-{}-} \\
3	&011	&\textbf{-wx} \\
2	&010	&\textbf{-w-} \\
1	&001	&\textbf{-{}-x} \\
0	&000	&\textbf{-{}-{}-} \\
\hline
\end{longtable}
\end{center}

So each bit is mapped to a flag: read/write/execute.

The importance of \TT{chmod} here is that the whole number in argument can be represented as octal number.
Let's take, for example, 644.
When you run \TT{chmod 644 file}, you set read/write permissions for owner, read permissions for group and again, read permissions for everyone else.
If we convert the octal number 644 to binary, it would be \TT{110100100}, or, in groups of 3 bits, \TT{110 100 100}.

Now we see that each triplet describe permissions for owner/group/others: first is \TT{rw-}, second is \TT{r--} and third is \TT{r--}.

The octal numeral system was also popular on old computers like PDP-8, because word there could be 12, 24 or 36 bits, and these numbers are all divisible by 3, so the octal system was natural in that environment.
Nowadays, all popular computers employ word/address sizes of 16, 32 or 64 bits, and these numbers are all divisible by 4, so the hexadecimal system is more natural there.

The octal numeral system is supported by all standard \CCpp compilers.
This is a source of confusion sometimes, because octal numbers are encoded with a zero prepended, for example, 0377 is 255.
Sometimes, you might make a typo and write "09" instead of 9, and the compiler would report an error.
GCC might report something like this:\\
\TT{error: invalid digit "9" in octal constant}.

Also, the octal system is somewhat popular in Java. When the IDA shows Java strings with non-printable characters,
they are encoded in the octal system instead of hexadecimal.
\myindex{JAD}
The JAD Java decompiler behaves the same way.

\subsubsection{Divisibility}

When you see a decimal number like 120, you can quickly deduce that it's divisible by 10, because the last digit is zero.
In the same way, 123400 is divisible by 100, because the two last digits are zeros.

Likewise, the hexadecimal number 0x1230 is divisible by 0x10 (or 16), 0x123000 is divisible by 0x1000 (or 4096), etc.

The binary number 0b1000101000 is divisible by 0b1000 (8), etc.

This property can often be used to quickly realize if the size of some block in memory is padded to some boundary.
For example, sections in \ac{PE} files are almost always started at addresses ending with 3 hexadecimal zeros: 0x41000, 0x10001000, etc.
The reason behind this is the fact that almost all \ac{PE} sections are padded to a boundary of 0x1000 (4096) bytes.

\subsubsection{Multi-Precision Arithmetic and Radix}

\index{RSA}
Multi-precision arithmetic can use huge numbers, and each one may be stored in several bytes.
For example, RSA keys, both public and private, span up to 4096 bits, and maybe even more.

% I'm not sure how to change this, but the normal format for quoting would be just to mention the author or book, and footnote to the full reference
In \InSqBrackets{\TAOCPvolII, 265} we find the following idea: when you store a multi-precision number in several bytes,
the whole number can be represented as having a radix of $2^8=256$, and each digit goes to the corresponding byte.
Likewise, if you store a multi-precision number in several 32-bit integer values, each digit goes to each 32-bit slot,
and you may think about this number as stored in radix of $2^{32}$.

\subsubsection{How to Pronounce Non-Decimal Numbers}

Numbers in a non-decimal base are usually pronounced by digit by digit: ``one-zero-zero-one-one-...''.
Words like ``ten'' and ``thousand'' are usually not pronounced, to prevent confusion with the decimal base system.

\subsubsection{Floating point numbers}

To distinguish floating point numbers from integers, they are usually written with ``.0'' at the end,
like $0.0$, $123.0$, etc.
}
\RU{\subsection{Представление чисел}

Люди привыкли к десятичной системе счисления вероятно потому что почти у каждого есть по 10 пальцев.
Тем не менее, число 10 не имеет особого значения в науке и математике.
Двоичная система естествена для цифровой электроники: 0 означает отсутствие тока в проводе и 1 --- его присутствие.
10 в двоичной системе это 2 в десятичной; 100 в двоичной это 4 в десятичной, итд.

Если в системе счисления есть 10 цифр, её \IT{основание} или \IT{radix} это 10.
Двоичная система имеет \IT{основание} 2.

Важные вещи, которые полезно вспомнить:
1) \IT{число} это число, в то время как \IT{цифра} это термин из системы письменности, и это обычно один символ;
2) само число не меняется, когда конвертируется из одного основания в другое: меняется способ его записи (или представления
в памяти).

Как сконвертировать число из одного основания в другое?

Позиционная нотация используется почти везде, это означает, что всякая цифра имеет свой вес, в зависимости от её расположения
внутри числа.
Если 2 расположена в самом последнем месте справа, это 2.
Если она расположена в месте перед последним, это 20.

Что означает $1234$?

$10^3 \cdot 1 + 10^2 \cdot 2 + 10^1 \cdot 3 + 1 \cdot 4$ = 1234 или
$1000 \cdot 1 + 100 \cdot 2 + 10 \cdot 3 + 4 = 1234$

Та же история и для двоичных чисел, только основание там 2 вместо 10.
Что означает 0b101011?

$2^5 \cdot 1 + 2^4 \cdot 0 + 2^3 \cdot 1 + 2^2 \cdot 0 + 2^1 \cdot 1 + 2^0 \cdot 1 = 43$ или
$32 \cdot 1 + 16 \cdot 0 + 8 \cdot 1 + 4 \cdot 0 + 2 \cdot 1 + 1 = 43$

Позиционную нотацию можно противопоставить непозиционной нотации, такой как римская система записи чисел
\footnote{Об эволюции способов записи чисел, см.также: \InSqBrackets{\TAOCPvolII{}, 195--213.}}.
Вероятно, человечество перешло на позиционную нотацию, потому что так проще работать с числами (сложение, умножение, итд)
на бумаге, в ручную.

Действительно, двоичные числа можно складывать, вычитать, итд, точно также, как этому обычно обучают в школах,
только доступны лишь 2 цифры.

Двоичные числа громоздки, когда их используют в исходных кодах и дампах, так что в этих случаях применяется шестнадцатеричная
система.
Используются цифры 0..9 и еще 6 латинских букв: A..F.
Каждая шестнадцатеричная цифра занимает 4 бита или 4 двоичных цифры, так что конвертировать из двоичной системы в
шестнадцатеричную и назад, можно легко вручную, или даже в уме.

\begin{center}
\begin{longtable}{ | l | l | l | }
\hline
\HeaderColor шестнадцатеричная & \HeaderColor двоичная & \HeaderColor десятичная \\
\hline
0	&0000	&0 \\
1	&0001	&1 \\
2	&0010	&2 \\
3	&0011	&3 \\
4	&0100	&4 \\
5	&0101	&5 \\
6	&0110	&6 \\
7	&0111	&7 \\
8	&1000	&8 \\
9	&1001	&9 \\
A	&1010	&10 \\
B	&1011	&11 \\
C	&1100	&12 \\
D	&1101	&13 \\
E	&1110	&14 \\
F	&1111	&15 \\
\hline
\end{longtable}
\end{center}

Как понять, какое основание используется в конкретном месте?

Десятичные числа обычно записываются как есть, т.е., 1234. Но некоторые ассемблеры позволяют подчеркивать
этот факт для ясности, и это число может быть дополнено суффиксом "d": 1234d.

К двоичным числам иногда спереди добавляют префикс "0b": 0b100110111
(В \ac{GCC} для этого есть нестандартное расширение языка
\footnote{\url{https://gcc.gnu.org/onlinedocs/gcc/Binary-constants.html}}).
Есть также еще один способ: суффикс "b", например: 100110111b.
В этой книге я буду пытаться придерживаться префикса "0b" для двоичных чисел.

Шестнадцатеричные числа имеют префикс "0x" в \CCpp и некоторых других \ac{PL}: 0x1234ABCD.
Либо они имеют суффикс "h": 1234ABCDh --- обычно так они представляются в ассемблерах и отладчиках.
Если число начинается с цифры A..F, перед ним добавляется 0: 0ABCDEFh.
Во времена 8-битных домашних компьютеров, был также способ записи чисел используя префикс \$, например, \$ABCD.
В книге я попытаюсь придерживаться префикса "0x" для шестнадцатеричных чисел.

Нужно ли учиться конвертировать числа в уме? Таблицу шестнадцатеричных чисел из одной цифры легко запомнить.
А запоминать б\'{о}льшие числа, наверное, не стоит.

Наверное, чаще всего шестнадцатеричные числа можно увидеть в \ac{URL}-ах.
Так кодируются буквы не из числа латинских.
Например:
\url{https://en.wiktionary.org/wiki/na\%C3\%AFvet\%C3\%A9} это \ac{URL} страницы в Wiktionary о слове \q{naïveté}.

\subsubsection{Восьмеричная система}

Еще одна система, которая в прошлом много использовалась в программировании это восьмеричная: есть 8 цифр (0..7) и каждая
описывает 3 бита, так что легко конвертировать числа туда и назад.
Она почти везде была заменена шестнадцатеричной, но удивительно, в *NIX имеется утилита использующаяся многими людьми,
которая принимает на вход восьмеричное число: \TT{chmod}.

\myindex{UNIX!chmod}
Как знают многие пользователи *NIX, аргумент \TT{chmod} это число из трех цифр. Первая цифра это права владельца файла,
вторая это права группы (которой файл принадлежит), третья для всех остальных.
И каждая цифра может быть представлена в двоичном виде:

\begin{center}
\begin{longtable}{ | l | l | l | }
\hline
\HeaderColor десятичная & \HeaderColor двоичная & \HeaderColor значение \\
\hline
7	&111	&\textbf{rwx} \\
6	&110	&\textbf{rw-} \\
5	&101	&\textbf{r-x} \\
4	&100	&\textbf{r-{}-} \\
3	&011	&\textbf{-wx} \\
2	&010	&\textbf{-w-} \\
1	&001	&\textbf{-{}-x} \\
0	&000	&\textbf{-{}-{}-} \\
\hline
\end{longtable}
\end{center}

Так что каждый бит привязан к флагу: read/write/execute (чтение/запись/исполнение).

И вот почему я вспомнил здесь о \TT{chmod}, это потому что всё число может быть представлено как число в восьмеричной системе.
Для примера возьмем 644.
Когда вы запускаете \TT{chmod 644 file}, вы выставляете права read/write для владельца, права read для группы, и снова,
read для всех остальных.
Сконвертируем число 644 из восьмеричной системы в двоичную, это будет \TT{110100100}, или (в группах по 3 бита) \TT{110 100 100}.

Теперь мы видим, что каждая тройка описывает права для владельца/группы/остальных:
первая это \TT{rw-}, вторая это \TT{r--} и третья это \TT{r--}.

Восьмеричная система была также популярная на старых компьютерах вроде PDP-8, потому что слово там могло содержать 12, 24 или
36 бит, и эти числа делятся на 3, так что выбор восьмеричной системы в той среде был логичен.
Сейчас, все популярные компьютеры имеют размер слова/адреса 16, 32 или 64 бита, и эти числа делятся на 4,
так что шестнадцатеричная система здесь удобнее.

Восьмеричная система поддерживается всеми стандартными компиляторами \CCpp{}.
Это иногда источник недоумения, потому что восьмеричные числа кодируются с нулем вперед, например, 0377 это 255.
И иногда, вы можете сделать опечатку, и написать "09" вместо 9, и компилятор выдаст ошибку.
GCC может выдать что-то вроде:\\
\TT{error: invalid digit "9" in octal constant}.

Также, восьмеричная система популярна в Java: когда IDA показывает строку с непечатаемыми символами,
они кодируются в восьмеричной системе вместо шестнадцатеричной.
\myindex{JAD}
Точно также себя ведет декомпилятор с Java JAD.

\subsubsection{Делимость}

Когда вы видите десятичное число вроде 120, вы можете быстро понять что оно делится на 10, потому что последняя цифра это 0.
Точно также, 123400 делится на 100, потому что две последних цифры это нули.

Точно также, шестнадцатеричное число 0x1230 делится на 0x10 (или 16), 0x123000 делится на 0x1000 (или 4096), итд.

Двоичное число 0b1000101000 делится на 0b1000 (8), итд.

Это свойство можно часто использовать, чтобы быстро понять,
что длина какого-либо блока в памяти выровнена по некоторой границе.
Например, секции в \ac{PE}-файлах почти всегда начинаются с адресов заканчивающихся 3 шестнадцатеричными нулями:
0x41000, 0x10001000, итд.
Причина в том, что почти все секции в \ac{PE} выровнены по границе 0x1000 (4096) байт.

\subsubsection{Арифметика произвольной точности и основание}

\index{RSA}
Арифметика произвольной точности (multi-precision arithmetic) может использовать огромные числа,
которые могут храниться в нескольких байтах.
Например, ключи RSA, и открытые и закрытые, могут занимать до 4096 бит и даже больше.

В \InSqBrackets{\TAOCPvolII, 265} можно найти такую идею: когда вы сохраняете число произвольной точности в нескольких байтах,
всё число может быть представлено как имеющую систему счисления по основанию $2^8=256$, и каждая цифра находится
в соответствующем байте.
Точно также, если вы сохраняете число произвольной точности в нескольких 32-битных целочисленных значениях,
каждая цифра отправляется в каждый 32-битный слот, и вы можете считать что это число записано в системе с основанием $2^{32}$.

\subsubsection{Произношение}

Числа в недесятичных системах счислениях обычно произносятся по одной цифре: ``один-ноль-ноль-один-один-...''.
Слова вроде ``десять'', ``тысяча'', итд, обычно не произносятся, потому что тогда можно спутать с десятичной системой.

\subsubsection{Числа с плавающей запятой}

Чтобы отличать числа с плавающей запятой от целочисленных, часто, в конце добавляют ``.0'',
например $0.0$, $123.0$, итд.

}
\ITA{\input{patterns/numeral_ITA}}
\DE{\input{patterns/numeral_DE}}
\FR{\input{patterns/numeral_FR}}
\PL{\input{patterns/numeral_PL}}

% chapters
\ifdefined\SPANISH
\chapter{Patrones de código}
\fi % SPANISH

\ifdefined\GERMAN
\chapter{Code-Muster}
\fi % GERMAN

\ifdefined\ENGLISH
\chapter{Code Patterns}
\fi % ENGLISH

\ifdefined\ITALIAN
\chapter{Forme di codice}
\fi % ITALIAN

\ifdefined\RUSSIAN
\chapter{Образцы кода}
\fi % RUSSIAN

\ifdefined\BRAZILIAN
\chapter{Padrões de códigos}
\fi % BRAZILIAN

\ifdefined\THAI
\chapter{รูปแบบของโค้ด}
\fi % THAI

\ifdefined\FRENCH
\chapter{Modèle de code}
\fi % FRENCH

\ifdefined\POLISH
\chapter{\PLph{}}
\fi % POLISH

% sections
\EN{\input{patterns/patterns_opt_dbg_EN}}
\ES{\input{patterns/patterns_opt_dbg_ES}}
\ITA{\input{patterns/patterns_opt_dbg_ITA}}
\PTBR{\input{patterns/patterns_opt_dbg_PTBR}}
\RU{\input{patterns/patterns_opt_dbg_RU}}
\THA{\input{patterns/patterns_opt_dbg_THA}}
\DE{\input{patterns/patterns_opt_dbg_DE}}
\FR{\input{patterns/patterns_opt_dbg_FR}}
\PL{\input{patterns/patterns_opt_dbg_PL}}

\RU{\section{Некоторые базовые понятия}}
\EN{\section{Some basics}}
\DE{\section{Einige Grundlagen}}
\FR{\section{Quelques bases}}
\ES{\section{\ESph{}}}
\ITA{\section{Alcune basi teoriche}}
\PTBR{\section{\PTBRph{}}}
\THA{\section{\THAph{}}}
\PL{\section{\PLph{}}}

% sections:
\EN{\input{patterns/intro_CPU_ISA_EN}}
\ES{\input{patterns/intro_CPU_ISA_ES}}
\ITA{\input{patterns/intro_CPU_ISA_ITA}}
\PTBR{\input{patterns/intro_CPU_ISA_PTBR}}
\RU{\input{patterns/intro_CPU_ISA_RU}}
\DE{\input{patterns/intro_CPU_ISA_DE}}
\FR{\input{patterns/intro_CPU_ISA_FR}}
\PL{\input{patterns/intro_CPU_ISA_PL}}

\EN{\input{patterns/numeral_EN}}
\RU{\input{patterns/numeral_RU}}
\ITA{\input{patterns/numeral_ITA}}
\DE{\input{patterns/numeral_DE}}
\FR{\input{patterns/numeral_FR}}
\PL{\input{patterns/numeral_PL}}

% chapters
\input{patterns/00_empty/main}
\input{patterns/011_ret/main}
\input{patterns/01_helloworld/main}
\input{patterns/015_prolog_epilogue/main}
\input{patterns/02_stack/main}
\input{patterns/03_printf/main}
\input{patterns/04_scanf/main}
\input{patterns/05_passing_arguments/main}
\input{patterns/06_return_results/main}
\input{patterns/061_pointers/main}
\input{patterns/065_GOTO/main}
\input{patterns/07_jcc/main}
\input{patterns/08_switch/main}
\input{patterns/09_loops/main}
\input{patterns/10_strings/main}
\input{patterns/11_arith_optimizations/main}
\input{patterns/12_FPU/main}
\input{patterns/13_arrays/main}
\input{patterns/14_bitfields/main}
\EN{\input{patterns/145_LCG/main_EN}}
\RU{\input{patterns/145_LCG/main_RU}}
\input{patterns/15_structs/main}
\input{patterns/17_unions/main}
\input{patterns/18_pointers_to_functions/main}
\input{patterns/185_64bit_in_32_env/main}

\EN{\input{patterns/19_SIMD/main_EN}}
\RU{\input{patterns/19_SIMD/main_RU}}
\DE{\input{patterns/19_SIMD/main_DE}}

\EN{\input{patterns/20_x64/main_EN}}
\RU{\input{patterns/20_x64/main_RU}}

\EN{\input{patterns/205_floating_SIMD/main_EN}}
\RU{\input{patterns/205_floating_SIMD/main_RU}}
\DE{\input{patterns/205_floating_SIMD/main_DE}}

\EN{\input{patterns/ARM/main_EN}}
\RU{\input{patterns/ARM/main_RU}}
\DE{\input{patterns/ARM/main_DE}}

\input{patterns/MIPS/main}

\ifdefined\SPANISH
\chapter{Patrones de código}
\fi % SPANISH

\ifdefined\GERMAN
\chapter{Code-Muster}
\fi % GERMAN

\ifdefined\ENGLISH
\chapter{Code Patterns}
\fi % ENGLISH

\ifdefined\ITALIAN
\chapter{Forme di codice}
\fi % ITALIAN

\ifdefined\RUSSIAN
\chapter{Образцы кода}
\fi % RUSSIAN

\ifdefined\BRAZILIAN
\chapter{Padrões de códigos}
\fi % BRAZILIAN

\ifdefined\THAI
\chapter{รูปแบบของโค้ด}
\fi % THAI

\ifdefined\FRENCH
\chapter{Modèle de code}
\fi % FRENCH

\ifdefined\POLISH
\chapter{\PLph{}}
\fi % POLISH

% sections
\EN{\input{patterns/patterns_opt_dbg_EN}}
\ES{\input{patterns/patterns_opt_dbg_ES}}
\ITA{\input{patterns/patterns_opt_dbg_ITA}}
\PTBR{\input{patterns/patterns_opt_dbg_PTBR}}
\RU{\input{patterns/patterns_opt_dbg_RU}}
\THA{\input{patterns/patterns_opt_dbg_THA}}
\DE{\input{patterns/patterns_opt_dbg_DE}}
\FR{\input{patterns/patterns_opt_dbg_FR}}
\PL{\input{patterns/patterns_opt_dbg_PL}}

\RU{\section{Некоторые базовые понятия}}
\EN{\section{Some basics}}
\DE{\section{Einige Grundlagen}}
\FR{\section{Quelques bases}}
\ES{\section{\ESph{}}}
\ITA{\section{Alcune basi teoriche}}
\PTBR{\section{\PTBRph{}}}
\THA{\section{\THAph{}}}
\PL{\section{\PLph{}}}

% sections:
\EN{\input{patterns/intro_CPU_ISA_EN}}
\ES{\input{patterns/intro_CPU_ISA_ES}}
\ITA{\input{patterns/intro_CPU_ISA_ITA}}
\PTBR{\input{patterns/intro_CPU_ISA_PTBR}}
\RU{\input{patterns/intro_CPU_ISA_RU}}
\DE{\input{patterns/intro_CPU_ISA_DE}}
\FR{\input{patterns/intro_CPU_ISA_FR}}
\PL{\input{patterns/intro_CPU_ISA_PL}}

\EN{\input{patterns/numeral_EN}}
\RU{\input{patterns/numeral_RU}}
\ITA{\input{patterns/numeral_ITA}}
\DE{\input{patterns/numeral_DE}}
\FR{\input{patterns/numeral_FR}}
\PL{\input{patterns/numeral_PL}}

% chapters
\input{patterns/00_empty/main}
\input{patterns/011_ret/main}
\input{patterns/01_helloworld/main}
\input{patterns/015_prolog_epilogue/main}
\input{patterns/02_stack/main}
\input{patterns/03_printf/main}
\input{patterns/04_scanf/main}
\input{patterns/05_passing_arguments/main}
\input{patterns/06_return_results/main}
\input{patterns/061_pointers/main}
\input{patterns/065_GOTO/main}
\input{patterns/07_jcc/main}
\input{patterns/08_switch/main}
\input{patterns/09_loops/main}
\input{patterns/10_strings/main}
\input{patterns/11_arith_optimizations/main}
\input{patterns/12_FPU/main}
\input{patterns/13_arrays/main}
\input{patterns/14_bitfields/main}
\EN{\input{patterns/145_LCG/main_EN}}
\RU{\input{patterns/145_LCG/main_RU}}
\input{patterns/15_structs/main}
\input{patterns/17_unions/main}
\input{patterns/18_pointers_to_functions/main}
\input{patterns/185_64bit_in_32_env/main}

\EN{\input{patterns/19_SIMD/main_EN}}
\RU{\input{patterns/19_SIMD/main_RU}}
\DE{\input{patterns/19_SIMD/main_DE}}

\EN{\input{patterns/20_x64/main_EN}}
\RU{\input{patterns/20_x64/main_RU}}

\EN{\input{patterns/205_floating_SIMD/main_EN}}
\RU{\input{patterns/205_floating_SIMD/main_RU}}
\DE{\input{patterns/205_floating_SIMD/main_DE}}

\EN{\input{patterns/ARM/main_EN}}
\RU{\input{patterns/ARM/main_RU}}
\DE{\input{patterns/ARM/main_DE}}

\input{patterns/MIPS/main}

\ifdefined\SPANISH
\chapter{Patrones de código}
\fi % SPANISH

\ifdefined\GERMAN
\chapter{Code-Muster}
\fi % GERMAN

\ifdefined\ENGLISH
\chapter{Code Patterns}
\fi % ENGLISH

\ifdefined\ITALIAN
\chapter{Forme di codice}
\fi % ITALIAN

\ifdefined\RUSSIAN
\chapter{Образцы кода}
\fi % RUSSIAN

\ifdefined\BRAZILIAN
\chapter{Padrões de códigos}
\fi % BRAZILIAN

\ifdefined\THAI
\chapter{รูปแบบของโค้ด}
\fi % THAI

\ifdefined\FRENCH
\chapter{Modèle de code}
\fi % FRENCH

\ifdefined\POLISH
\chapter{\PLph{}}
\fi % POLISH

% sections
\EN{\input{patterns/patterns_opt_dbg_EN}}
\ES{\input{patterns/patterns_opt_dbg_ES}}
\ITA{\input{patterns/patterns_opt_dbg_ITA}}
\PTBR{\input{patterns/patterns_opt_dbg_PTBR}}
\RU{\input{patterns/patterns_opt_dbg_RU}}
\THA{\input{patterns/patterns_opt_dbg_THA}}
\DE{\input{patterns/patterns_opt_dbg_DE}}
\FR{\input{patterns/patterns_opt_dbg_FR}}
\PL{\input{patterns/patterns_opt_dbg_PL}}

\RU{\section{Некоторые базовые понятия}}
\EN{\section{Some basics}}
\DE{\section{Einige Grundlagen}}
\FR{\section{Quelques bases}}
\ES{\section{\ESph{}}}
\ITA{\section{Alcune basi teoriche}}
\PTBR{\section{\PTBRph{}}}
\THA{\section{\THAph{}}}
\PL{\section{\PLph{}}}

% sections:
\EN{\input{patterns/intro_CPU_ISA_EN}}
\ES{\input{patterns/intro_CPU_ISA_ES}}
\ITA{\input{patterns/intro_CPU_ISA_ITA}}
\PTBR{\input{patterns/intro_CPU_ISA_PTBR}}
\RU{\input{patterns/intro_CPU_ISA_RU}}
\DE{\input{patterns/intro_CPU_ISA_DE}}
\FR{\input{patterns/intro_CPU_ISA_FR}}
\PL{\input{patterns/intro_CPU_ISA_PL}}

\EN{\input{patterns/numeral_EN}}
\RU{\input{patterns/numeral_RU}}
\ITA{\input{patterns/numeral_ITA}}
\DE{\input{patterns/numeral_DE}}
\FR{\input{patterns/numeral_FR}}
\PL{\input{patterns/numeral_PL}}

% chapters
\input{patterns/00_empty/main}
\input{patterns/011_ret/main}
\input{patterns/01_helloworld/main}
\input{patterns/015_prolog_epilogue/main}
\input{patterns/02_stack/main}
\input{patterns/03_printf/main}
\input{patterns/04_scanf/main}
\input{patterns/05_passing_arguments/main}
\input{patterns/06_return_results/main}
\input{patterns/061_pointers/main}
\input{patterns/065_GOTO/main}
\input{patterns/07_jcc/main}
\input{patterns/08_switch/main}
\input{patterns/09_loops/main}
\input{patterns/10_strings/main}
\input{patterns/11_arith_optimizations/main}
\input{patterns/12_FPU/main}
\input{patterns/13_arrays/main}
\input{patterns/14_bitfields/main}
\EN{\input{patterns/145_LCG/main_EN}}
\RU{\input{patterns/145_LCG/main_RU}}
\input{patterns/15_structs/main}
\input{patterns/17_unions/main}
\input{patterns/18_pointers_to_functions/main}
\input{patterns/185_64bit_in_32_env/main}

\EN{\input{patterns/19_SIMD/main_EN}}
\RU{\input{patterns/19_SIMD/main_RU}}
\DE{\input{patterns/19_SIMD/main_DE}}

\EN{\input{patterns/20_x64/main_EN}}
\RU{\input{patterns/20_x64/main_RU}}

\EN{\input{patterns/205_floating_SIMD/main_EN}}
\RU{\input{patterns/205_floating_SIMD/main_RU}}
\DE{\input{patterns/205_floating_SIMD/main_DE}}

\EN{\input{patterns/ARM/main_EN}}
\RU{\input{patterns/ARM/main_RU}}
\DE{\input{patterns/ARM/main_DE}}

\input{patterns/MIPS/main}

\ifdefined\SPANISH
\chapter{Patrones de código}
\fi % SPANISH

\ifdefined\GERMAN
\chapter{Code-Muster}
\fi % GERMAN

\ifdefined\ENGLISH
\chapter{Code Patterns}
\fi % ENGLISH

\ifdefined\ITALIAN
\chapter{Forme di codice}
\fi % ITALIAN

\ifdefined\RUSSIAN
\chapter{Образцы кода}
\fi % RUSSIAN

\ifdefined\BRAZILIAN
\chapter{Padrões de códigos}
\fi % BRAZILIAN

\ifdefined\THAI
\chapter{รูปแบบของโค้ด}
\fi % THAI

\ifdefined\FRENCH
\chapter{Modèle de code}
\fi % FRENCH

\ifdefined\POLISH
\chapter{\PLph{}}
\fi % POLISH

% sections
\EN{\input{patterns/patterns_opt_dbg_EN}}
\ES{\input{patterns/patterns_opt_dbg_ES}}
\ITA{\input{patterns/patterns_opt_dbg_ITA}}
\PTBR{\input{patterns/patterns_opt_dbg_PTBR}}
\RU{\input{patterns/patterns_opt_dbg_RU}}
\THA{\input{patterns/patterns_opt_dbg_THA}}
\DE{\input{patterns/patterns_opt_dbg_DE}}
\FR{\input{patterns/patterns_opt_dbg_FR}}
\PL{\input{patterns/patterns_opt_dbg_PL}}

\RU{\section{Некоторые базовые понятия}}
\EN{\section{Some basics}}
\DE{\section{Einige Grundlagen}}
\FR{\section{Quelques bases}}
\ES{\section{\ESph{}}}
\ITA{\section{Alcune basi teoriche}}
\PTBR{\section{\PTBRph{}}}
\THA{\section{\THAph{}}}
\PL{\section{\PLph{}}}

% sections:
\EN{\input{patterns/intro_CPU_ISA_EN}}
\ES{\input{patterns/intro_CPU_ISA_ES}}
\ITA{\input{patterns/intro_CPU_ISA_ITA}}
\PTBR{\input{patterns/intro_CPU_ISA_PTBR}}
\RU{\input{patterns/intro_CPU_ISA_RU}}
\DE{\input{patterns/intro_CPU_ISA_DE}}
\FR{\input{patterns/intro_CPU_ISA_FR}}
\PL{\input{patterns/intro_CPU_ISA_PL}}

\EN{\input{patterns/numeral_EN}}
\RU{\input{patterns/numeral_RU}}
\ITA{\input{patterns/numeral_ITA}}
\DE{\input{patterns/numeral_DE}}
\FR{\input{patterns/numeral_FR}}
\PL{\input{patterns/numeral_PL}}

% chapters
\input{patterns/00_empty/main}
\input{patterns/011_ret/main}
\input{patterns/01_helloworld/main}
\input{patterns/015_prolog_epilogue/main}
\input{patterns/02_stack/main}
\input{patterns/03_printf/main}
\input{patterns/04_scanf/main}
\input{patterns/05_passing_arguments/main}
\input{patterns/06_return_results/main}
\input{patterns/061_pointers/main}
\input{patterns/065_GOTO/main}
\input{patterns/07_jcc/main}
\input{patterns/08_switch/main}
\input{patterns/09_loops/main}
\input{patterns/10_strings/main}
\input{patterns/11_arith_optimizations/main}
\input{patterns/12_FPU/main}
\input{patterns/13_arrays/main}
\input{patterns/14_bitfields/main}
\EN{\input{patterns/145_LCG/main_EN}}
\RU{\input{patterns/145_LCG/main_RU}}
\input{patterns/15_structs/main}
\input{patterns/17_unions/main}
\input{patterns/18_pointers_to_functions/main}
\input{patterns/185_64bit_in_32_env/main}

\EN{\input{patterns/19_SIMD/main_EN}}
\RU{\input{patterns/19_SIMD/main_RU}}
\DE{\input{patterns/19_SIMD/main_DE}}

\EN{\input{patterns/20_x64/main_EN}}
\RU{\input{patterns/20_x64/main_RU}}

\EN{\input{patterns/205_floating_SIMD/main_EN}}
\RU{\input{patterns/205_floating_SIMD/main_RU}}
\DE{\input{patterns/205_floating_SIMD/main_DE}}

\EN{\input{patterns/ARM/main_EN}}
\RU{\input{patterns/ARM/main_RU}}
\DE{\input{patterns/ARM/main_DE}}

\input{patterns/MIPS/main}

\ifdefined\SPANISH
\chapter{Patrones de código}
\fi % SPANISH

\ifdefined\GERMAN
\chapter{Code-Muster}
\fi % GERMAN

\ifdefined\ENGLISH
\chapter{Code Patterns}
\fi % ENGLISH

\ifdefined\ITALIAN
\chapter{Forme di codice}
\fi % ITALIAN

\ifdefined\RUSSIAN
\chapter{Образцы кода}
\fi % RUSSIAN

\ifdefined\BRAZILIAN
\chapter{Padrões de códigos}
\fi % BRAZILIAN

\ifdefined\THAI
\chapter{รูปแบบของโค้ด}
\fi % THAI

\ifdefined\FRENCH
\chapter{Modèle de code}
\fi % FRENCH

\ifdefined\POLISH
\chapter{\PLph{}}
\fi % POLISH

% sections
\EN{\input{patterns/patterns_opt_dbg_EN}}
\ES{\input{patterns/patterns_opt_dbg_ES}}
\ITA{\input{patterns/patterns_opt_dbg_ITA}}
\PTBR{\input{patterns/patterns_opt_dbg_PTBR}}
\RU{\input{patterns/patterns_opt_dbg_RU}}
\THA{\input{patterns/patterns_opt_dbg_THA}}
\DE{\input{patterns/patterns_opt_dbg_DE}}
\FR{\input{patterns/patterns_opt_dbg_FR}}
\PL{\input{patterns/patterns_opt_dbg_PL}}

\RU{\section{Некоторые базовые понятия}}
\EN{\section{Some basics}}
\DE{\section{Einige Grundlagen}}
\FR{\section{Quelques bases}}
\ES{\section{\ESph{}}}
\ITA{\section{Alcune basi teoriche}}
\PTBR{\section{\PTBRph{}}}
\THA{\section{\THAph{}}}
\PL{\section{\PLph{}}}

% sections:
\EN{\input{patterns/intro_CPU_ISA_EN}}
\ES{\input{patterns/intro_CPU_ISA_ES}}
\ITA{\input{patterns/intro_CPU_ISA_ITA}}
\PTBR{\input{patterns/intro_CPU_ISA_PTBR}}
\RU{\input{patterns/intro_CPU_ISA_RU}}
\DE{\input{patterns/intro_CPU_ISA_DE}}
\FR{\input{patterns/intro_CPU_ISA_FR}}
\PL{\input{patterns/intro_CPU_ISA_PL}}

\EN{\input{patterns/numeral_EN}}
\RU{\input{patterns/numeral_RU}}
\ITA{\input{patterns/numeral_ITA}}
\DE{\input{patterns/numeral_DE}}
\FR{\input{patterns/numeral_FR}}
\PL{\input{patterns/numeral_PL}}

% chapters
\input{patterns/00_empty/main}
\input{patterns/011_ret/main}
\input{patterns/01_helloworld/main}
\input{patterns/015_prolog_epilogue/main}
\input{patterns/02_stack/main}
\input{patterns/03_printf/main}
\input{patterns/04_scanf/main}
\input{patterns/05_passing_arguments/main}
\input{patterns/06_return_results/main}
\input{patterns/061_pointers/main}
\input{patterns/065_GOTO/main}
\input{patterns/07_jcc/main}
\input{patterns/08_switch/main}
\input{patterns/09_loops/main}
\input{patterns/10_strings/main}
\input{patterns/11_arith_optimizations/main}
\input{patterns/12_FPU/main}
\input{patterns/13_arrays/main}
\input{patterns/14_bitfields/main}
\EN{\input{patterns/145_LCG/main_EN}}
\RU{\input{patterns/145_LCG/main_RU}}
\input{patterns/15_structs/main}
\input{patterns/17_unions/main}
\input{patterns/18_pointers_to_functions/main}
\input{patterns/185_64bit_in_32_env/main}

\EN{\input{patterns/19_SIMD/main_EN}}
\RU{\input{patterns/19_SIMD/main_RU}}
\DE{\input{patterns/19_SIMD/main_DE}}

\EN{\input{patterns/20_x64/main_EN}}
\RU{\input{patterns/20_x64/main_RU}}

\EN{\input{patterns/205_floating_SIMD/main_EN}}
\RU{\input{patterns/205_floating_SIMD/main_RU}}
\DE{\input{patterns/205_floating_SIMD/main_DE}}

\EN{\input{patterns/ARM/main_EN}}
\RU{\input{patterns/ARM/main_RU}}
\DE{\input{patterns/ARM/main_DE}}

\input{patterns/MIPS/main}

\ifdefined\SPANISH
\chapter{Patrones de código}
\fi % SPANISH

\ifdefined\GERMAN
\chapter{Code-Muster}
\fi % GERMAN

\ifdefined\ENGLISH
\chapter{Code Patterns}
\fi % ENGLISH

\ifdefined\ITALIAN
\chapter{Forme di codice}
\fi % ITALIAN

\ifdefined\RUSSIAN
\chapter{Образцы кода}
\fi % RUSSIAN

\ifdefined\BRAZILIAN
\chapter{Padrões de códigos}
\fi % BRAZILIAN

\ifdefined\THAI
\chapter{รูปแบบของโค้ด}
\fi % THAI

\ifdefined\FRENCH
\chapter{Modèle de code}
\fi % FRENCH

\ifdefined\POLISH
\chapter{\PLph{}}
\fi % POLISH

% sections
\EN{\input{patterns/patterns_opt_dbg_EN}}
\ES{\input{patterns/patterns_opt_dbg_ES}}
\ITA{\input{patterns/patterns_opt_dbg_ITA}}
\PTBR{\input{patterns/patterns_opt_dbg_PTBR}}
\RU{\input{patterns/patterns_opt_dbg_RU}}
\THA{\input{patterns/patterns_opt_dbg_THA}}
\DE{\input{patterns/patterns_opt_dbg_DE}}
\FR{\input{patterns/patterns_opt_dbg_FR}}
\PL{\input{patterns/patterns_opt_dbg_PL}}

\RU{\section{Некоторые базовые понятия}}
\EN{\section{Some basics}}
\DE{\section{Einige Grundlagen}}
\FR{\section{Quelques bases}}
\ES{\section{\ESph{}}}
\ITA{\section{Alcune basi teoriche}}
\PTBR{\section{\PTBRph{}}}
\THA{\section{\THAph{}}}
\PL{\section{\PLph{}}}

% sections:
\EN{\input{patterns/intro_CPU_ISA_EN}}
\ES{\input{patterns/intro_CPU_ISA_ES}}
\ITA{\input{patterns/intro_CPU_ISA_ITA}}
\PTBR{\input{patterns/intro_CPU_ISA_PTBR}}
\RU{\input{patterns/intro_CPU_ISA_RU}}
\DE{\input{patterns/intro_CPU_ISA_DE}}
\FR{\input{patterns/intro_CPU_ISA_FR}}
\PL{\input{patterns/intro_CPU_ISA_PL}}

\EN{\input{patterns/numeral_EN}}
\RU{\input{patterns/numeral_RU}}
\ITA{\input{patterns/numeral_ITA}}
\DE{\input{patterns/numeral_DE}}
\FR{\input{patterns/numeral_FR}}
\PL{\input{patterns/numeral_PL}}

% chapters
\input{patterns/00_empty/main}
\input{patterns/011_ret/main}
\input{patterns/01_helloworld/main}
\input{patterns/015_prolog_epilogue/main}
\input{patterns/02_stack/main}
\input{patterns/03_printf/main}
\input{patterns/04_scanf/main}
\input{patterns/05_passing_arguments/main}
\input{patterns/06_return_results/main}
\input{patterns/061_pointers/main}
\input{patterns/065_GOTO/main}
\input{patterns/07_jcc/main}
\input{patterns/08_switch/main}
\input{patterns/09_loops/main}
\input{patterns/10_strings/main}
\input{patterns/11_arith_optimizations/main}
\input{patterns/12_FPU/main}
\input{patterns/13_arrays/main}
\input{patterns/14_bitfields/main}
\EN{\input{patterns/145_LCG/main_EN}}
\RU{\input{patterns/145_LCG/main_RU}}
\input{patterns/15_structs/main}
\input{patterns/17_unions/main}
\input{patterns/18_pointers_to_functions/main}
\input{patterns/185_64bit_in_32_env/main}

\EN{\input{patterns/19_SIMD/main_EN}}
\RU{\input{patterns/19_SIMD/main_RU}}
\DE{\input{patterns/19_SIMD/main_DE}}

\EN{\input{patterns/20_x64/main_EN}}
\RU{\input{patterns/20_x64/main_RU}}

\EN{\input{patterns/205_floating_SIMD/main_EN}}
\RU{\input{patterns/205_floating_SIMD/main_RU}}
\DE{\input{patterns/205_floating_SIMD/main_DE}}

\EN{\input{patterns/ARM/main_EN}}
\RU{\input{patterns/ARM/main_RU}}
\DE{\input{patterns/ARM/main_DE}}

\input{patterns/MIPS/main}

\ifdefined\SPANISH
\chapter{Patrones de código}
\fi % SPANISH

\ifdefined\GERMAN
\chapter{Code-Muster}
\fi % GERMAN

\ifdefined\ENGLISH
\chapter{Code Patterns}
\fi % ENGLISH

\ifdefined\ITALIAN
\chapter{Forme di codice}
\fi % ITALIAN

\ifdefined\RUSSIAN
\chapter{Образцы кода}
\fi % RUSSIAN

\ifdefined\BRAZILIAN
\chapter{Padrões de códigos}
\fi % BRAZILIAN

\ifdefined\THAI
\chapter{รูปแบบของโค้ด}
\fi % THAI

\ifdefined\FRENCH
\chapter{Modèle de code}
\fi % FRENCH

\ifdefined\POLISH
\chapter{\PLph{}}
\fi % POLISH

% sections
\EN{\input{patterns/patterns_opt_dbg_EN}}
\ES{\input{patterns/patterns_opt_dbg_ES}}
\ITA{\input{patterns/patterns_opt_dbg_ITA}}
\PTBR{\input{patterns/patterns_opt_dbg_PTBR}}
\RU{\input{patterns/patterns_opt_dbg_RU}}
\THA{\input{patterns/patterns_opt_dbg_THA}}
\DE{\input{patterns/patterns_opt_dbg_DE}}
\FR{\input{patterns/patterns_opt_dbg_FR}}
\PL{\input{patterns/patterns_opt_dbg_PL}}

\RU{\section{Некоторые базовые понятия}}
\EN{\section{Some basics}}
\DE{\section{Einige Grundlagen}}
\FR{\section{Quelques bases}}
\ES{\section{\ESph{}}}
\ITA{\section{Alcune basi teoriche}}
\PTBR{\section{\PTBRph{}}}
\THA{\section{\THAph{}}}
\PL{\section{\PLph{}}}

% sections:
\EN{\input{patterns/intro_CPU_ISA_EN}}
\ES{\input{patterns/intro_CPU_ISA_ES}}
\ITA{\input{patterns/intro_CPU_ISA_ITA}}
\PTBR{\input{patterns/intro_CPU_ISA_PTBR}}
\RU{\input{patterns/intro_CPU_ISA_RU}}
\DE{\input{patterns/intro_CPU_ISA_DE}}
\FR{\input{patterns/intro_CPU_ISA_FR}}
\PL{\input{patterns/intro_CPU_ISA_PL}}

\EN{\input{patterns/numeral_EN}}
\RU{\input{patterns/numeral_RU}}
\ITA{\input{patterns/numeral_ITA}}
\DE{\input{patterns/numeral_DE}}
\FR{\input{patterns/numeral_FR}}
\PL{\input{patterns/numeral_PL}}

% chapters
\input{patterns/00_empty/main}
\input{patterns/011_ret/main}
\input{patterns/01_helloworld/main}
\input{patterns/015_prolog_epilogue/main}
\input{patterns/02_stack/main}
\input{patterns/03_printf/main}
\input{patterns/04_scanf/main}
\input{patterns/05_passing_arguments/main}
\input{patterns/06_return_results/main}
\input{patterns/061_pointers/main}
\input{patterns/065_GOTO/main}
\input{patterns/07_jcc/main}
\input{patterns/08_switch/main}
\input{patterns/09_loops/main}
\input{patterns/10_strings/main}
\input{patterns/11_arith_optimizations/main}
\input{patterns/12_FPU/main}
\input{patterns/13_arrays/main}
\input{patterns/14_bitfields/main}
\EN{\input{patterns/145_LCG/main_EN}}
\RU{\input{patterns/145_LCG/main_RU}}
\input{patterns/15_structs/main}
\input{patterns/17_unions/main}
\input{patterns/18_pointers_to_functions/main}
\input{patterns/185_64bit_in_32_env/main}

\EN{\input{patterns/19_SIMD/main_EN}}
\RU{\input{patterns/19_SIMD/main_RU}}
\DE{\input{patterns/19_SIMD/main_DE}}

\EN{\input{patterns/20_x64/main_EN}}
\RU{\input{patterns/20_x64/main_RU}}

\EN{\input{patterns/205_floating_SIMD/main_EN}}
\RU{\input{patterns/205_floating_SIMD/main_RU}}
\DE{\input{patterns/205_floating_SIMD/main_DE}}

\EN{\input{patterns/ARM/main_EN}}
\RU{\input{patterns/ARM/main_RU}}
\DE{\input{patterns/ARM/main_DE}}

\input{patterns/MIPS/main}

\ifdefined\SPANISH
\chapter{Patrones de código}
\fi % SPANISH

\ifdefined\GERMAN
\chapter{Code-Muster}
\fi % GERMAN

\ifdefined\ENGLISH
\chapter{Code Patterns}
\fi % ENGLISH

\ifdefined\ITALIAN
\chapter{Forme di codice}
\fi % ITALIAN

\ifdefined\RUSSIAN
\chapter{Образцы кода}
\fi % RUSSIAN

\ifdefined\BRAZILIAN
\chapter{Padrões de códigos}
\fi % BRAZILIAN

\ifdefined\THAI
\chapter{รูปแบบของโค้ด}
\fi % THAI

\ifdefined\FRENCH
\chapter{Modèle de code}
\fi % FRENCH

\ifdefined\POLISH
\chapter{\PLph{}}
\fi % POLISH

% sections
\EN{\input{patterns/patterns_opt_dbg_EN}}
\ES{\input{patterns/patterns_opt_dbg_ES}}
\ITA{\input{patterns/patterns_opt_dbg_ITA}}
\PTBR{\input{patterns/patterns_opt_dbg_PTBR}}
\RU{\input{patterns/patterns_opt_dbg_RU}}
\THA{\input{patterns/patterns_opt_dbg_THA}}
\DE{\input{patterns/patterns_opt_dbg_DE}}
\FR{\input{patterns/patterns_opt_dbg_FR}}
\PL{\input{patterns/patterns_opt_dbg_PL}}

\RU{\section{Некоторые базовые понятия}}
\EN{\section{Some basics}}
\DE{\section{Einige Grundlagen}}
\FR{\section{Quelques bases}}
\ES{\section{\ESph{}}}
\ITA{\section{Alcune basi teoriche}}
\PTBR{\section{\PTBRph{}}}
\THA{\section{\THAph{}}}
\PL{\section{\PLph{}}}

% sections:
\EN{\input{patterns/intro_CPU_ISA_EN}}
\ES{\input{patterns/intro_CPU_ISA_ES}}
\ITA{\input{patterns/intro_CPU_ISA_ITA}}
\PTBR{\input{patterns/intro_CPU_ISA_PTBR}}
\RU{\input{patterns/intro_CPU_ISA_RU}}
\DE{\input{patterns/intro_CPU_ISA_DE}}
\FR{\input{patterns/intro_CPU_ISA_FR}}
\PL{\input{patterns/intro_CPU_ISA_PL}}

\EN{\input{patterns/numeral_EN}}
\RU{\input{patterns/numeral_RU}}
\ITA{\input{patterns/numeral_ITA}}
\DE{\input{patterns/numeral_DE}}
\FR{\input{patterns/numeral_FR}}
\PL{\input{patterns/numeral_PL}}

% chapters
\input{patterns/00_empty/main}
\input{patterns/011_ret/main}
\input{patterns/01_helloworld/main}
\input{patterns/015_prolog_epilogue/main}
\input{patterns/02_stack/main}
\input{patterns/03_printf/main}
\input{patterns/04_scanf/main}
\input{patterns/05_passing_arguments/main}
\input{patterns/06_return_results/main}
\input{patterns/061_pointers/main}
\input{patterns/065_GOTO/main}
\input{patterns/07_jcc/main}
\input{patterns/08_switch/main}
\input{patterns/09_loops/main}
\input{patterns/10_strings/main}
\input{patterns/11_arith_optimizations/main}
\input{patterns/12_FPU/main}
\input{patterns/13_arrays/main}
\input{patterns/14_bitfields/main}
\EN{\input{patterns/145_LCG/main_EN}}
\RU{\input{patterns/145_LCG/main_RU}}
\input{patterns/15_structs/main}
\input{patterns/17_unions/main}
\input{patterns/18_pointers_to_functions/main}
\input{patterns/185_64bit_in_32_env/main}

\EN{\input{patterns/19_SIMD/main_EN}}
\RU{\input{patterns/19_SIMD/main_RU}}
\DE{\input{patterns/19_SIMD/main_DE}}

\EN{\input{patterns/20_x64/main_EN}}
\RU{\input{patterns/20_x64/main_RU}}

\EN{\input{patterns/205_floating_SIMD/main_EN}}
\RU{\input{patterns/205_floating_SIMD/main_RU}}
\DE{\input{patterns/205_floating_SIMD/main_DE}}

\EN{\input{patterns/ARM/main_EN}}
\RU{\input{patterns/ARM/main_RU}}
\DE{\input{patterns/ARM/main_DE}}

\input{patterns/MIPS/main}

\ifdefined\SPANISH
\chapter{Patrones de código}
\fi % SPANISH

\ifdefined\GERMAN
\chapter{Code-Muster}
\fi % GERMAN

\ifdefined\ENGLISH
\chapter{Code Patterns}
\fi % ENGLISH

\ifdefined\ITALIAN
\chapter{Forme di codice}
\fi % ITALIAN

\ifdefined\RUSSIAN
\chapter{Образцы кода}
\fi % RUSSIAN

\ifdefined\BRAZILIAN
\chapter{Padrões de códigos}
\fi % BRAZILIAN

\ifdefined\THAI
\chapter{รูปแบบของโค้ด}
\fi % THAI

\ifdefined\FRENCH
\chapter{Modèle de code}
\fi % FRENCH

\ifdefined\POLISH
\chapter{\PLph{}}
\fi % POLISH

% sections
\EN{\input{patterns/patterns_opt_dbg_EN}}
\ES{\input{patterns/patterns_opt_dbg_ES}}
\ITA{\input{patterns/patterns_opt_dbg_ITA}}
\PTBR{\input{patterns/patterns_opt_dbg_PTBR}}
\RU{\input{patterns/patterns_opt_dbg_RU}}
\THA{\input{patterns/patterns_opt_dbg_THA}}
\DE{\input{patterns/patterns_opt_dbg_DE}}
\FR{\input{patterns/patterns_opt_dbg_FR}}
\PL{\input{patterns/patterns_opt_dbg_PL}}

\RU{\section{Некоторые базовые понятия}}
\EN{\section{Some basics}}
\DE{\section{Einige Grundlagen}}
\FR{\section{Quelques bases}}
\ES{\section{\ESph{}}}
\ITA{\section{Alcune basi teoriche}}
\PTBR{\section{\PTBRph{}}}
\THA{\section{\THAph{}}}
\PL{\section{\PLph{}}}

% sections:
\EN{\input{patterns/intro_CPU_ISA_EN}}
\ES{\input{patterns/intro_CPU_ISA_ES}}
\ITA{\input{patterns/intro_CPU_ISA_ITA}}
\PTBR{\input{patterns/intro_CPU_ISA_PTBR}}
\RU{\input{patterns/intro_CPU_ISA_RU}}
\DE{\input{patterns/intro_CPU_ISA_DE}}
\FR{\input{patterns/intro_CPU_ISA_FR}}
\PL{\input{patterns/intro_CPU_ISA_PL}}

\EN{\input{patterns/numeral_EN}}
\RU{\input{patterns/numeral_RU}}
\ITA{\input{patterns/numeral_ITA}}
\DE{\input{patterns/numeral_DE}}
\FR{\input{patterns/numeral_FR}}
\PL{\input{patterns/numeral_PL}}

% chapters
\input{patterns/00_empty/main}
\input{patterns/011_ret/main}
\input{patterns/01_helloworld/main}
\input{patterns/015_prolog_epilogue/main}
\input{patterns/02_stack/main}
\input{patterns/03_printf/main}
\input{patterns/04_scanf/main}
\input{patterns/05_passing_arguments/main}
\input{patterns/06_return_results/main}
\input{patterns/061_pointers/main}
\input{patterns/065_GOTO/main}
\input{patterns/07_jcc/main}
\input{patterns/08_switch/main}
\input{patterns/09_loops/main}
\input{patterns/10_strings/main}
\input{patterns/11_arith_optimizations/main}
\input{patterns/12_FPU/main}
\input{patterns/13_arrays/main}
\input{patterns/14_bitfields/main}
\EN{\input{patterns/145_LCG/main_EN}}
\RU{\input{patterns/145_LCG/main_RU}}
\input{patterns/15_structs/main}
\input{patterns/17_unions/main}
\input{patterns/18_pointers_to_functions/main}
\input{patterns/185_64bit_in_32_env/main}

\EN{\input{patterns/19_SIMD/main_EN}}
\RU{\input{patterns/19_SIMD/main_RU}}
\DE{\input{patterns/19_SIMD/main_DE}}

\EN{\input{patterns/20_x64/main_EN}}
\RU{\input{patterns/20_x64/main_RU}}

\EN{\input{patterns/205_floating_SIMD/main_EN}}
\RU{\input{patterns/205_floating_SIMD/main_RU}}
\DE{\input{patterns/205_floating_SIMD/main_DE}}

\EN{\input{patterns/ARM/main_EN}}
\RU{\input{patterns/ARM/main_RU}}
\DE{\input{patterns/ARM/main_DE}}

\input{patterns/MIPS/main}

\ifdefined\SPANISH
\chapter{Patrones de código}
\fi % SPANISH

\ifdefined\GERMAN
\chapter{Code-Muster}
\fi % GERMAN

\ifdefined\ENGLISH
\chapter{Code Patterns}
\fi % ENGLISH

\ifdefined\ITALIAN
\chapter{Forme di codice}
\fi % ITALIAN

\ifdefined\RUSSIAN
\chapter{Образцы кода}
\fi % RUSSIAN

\ifdefined\BRAZILIAN
\chapter{Padrões de códigos}
\fi % BRAZILIAN

\ifdefined\THAI
\chapter{รูปแบบของโค้ด}
\fi % THAI

\ifdefined\FRENCH
\chapter{Modèle de code}
\fi % FRENCH

\ifdefined\POLISH
\chapter{\PLph{}}
\fi % POLISH

% sections
\EN{\input{patterns/patterns_opt_dbg_EN}}
\ES{\input{patterns/patterns_opt_dbg_ES}}
\ITA{\input{patterns/patterns_opt_dbg_ITA}}
\PTBR{\input{patterns/patterns_opt_dbg_PTBR}}
\RU{\input{patterns/patterns_opt_dbg_RU}}
\THA{\input{patterns/patterns_opt_dbg_THA}}
\DE{\input{patterns/patterns_opt_dbg_DE}}
\FR{\input{patterns/patterns_opt_dbg_FR}}
\PL{\input{patterns/patterns_opt_dbg_PL}}

\RU{\section{Некоторые базовые понятия}}
\EN{\section{Some basics}}
\DE{\section{Einige Grundlagen}}
\FR{\section{Quelques bases}}
\ES{\section{\ESph{}}}
\ITA{\section{Alcune basi teoriche}}
\PTBR{\section{\PTBRph{}}}
\THA{\section{\THAph{}}}
\PL{\section{\PLph{}}}

% sections:
\EN{\input{patterns/intro_CPU_ISA_EN}}
\ES{\input{patterns/intro_CPU_ISA_ES}}
\ITA{\input{patterns/intro_CPU_ISA_ITA}}
\PTBR{\input{patterns/intro_CPU_ISA_PTBR}}
\RU{\input{patterns/intro_CPU_ISA_RU}}
\DE{\input{patterns/intro_CPU_ISA_DE}}
\FR{\input{patterns/intro_CPU_ISA_FR}}
\PL{\input{patterns/intro_CPU_ISA_PL}}

\EN{\input{patterns/numeral_EN}}
\RU{\input{patterns/numeral_RU}}
\ITA{\input{patterns/numeral_ITA}}
\DE{\input{patterns/numeral_DE}}
\FR{\input{patterns/numeral_FR}}
\PL{\input{patterns/numeral_PL}}

% chapters
\input{patterns/00_empty/main}
\input{patterns/011_ret/main}
\input{patterns/01_helloworld/main}
\input{patterns/015_prolog_epilogue/main}
\input{patterns/02_stack/main}
\input{patterns/03_printf/main}
\input{patterns/04_scanf/main}
\input{patterns/05_passing_arguments/main}
\input{patterns/06_return_results/main}
\input{patterns/061_pointers/main}
\input{patterns/065_GOTO/main}
\input{patterns/07_jcc/main}
\input{patterns/08_switch/main}
\input{patterns/09_loops/main}
\input{patterns/10_strings/main}
\input{patterns/11_arith_optimizations/main}
\input{patterns/12_FPU/main}
\input{patterns/13_arrays/main}
\input{patterns/14_bitfields/main}
\EN{\input{patterns/145_LCG/main_EN}}
\RU{\input{patterns/145_LCG/main_RU}}
\input{patterns/15_structs/main}
\input{patterns/17_unions/main}
\input{patterns/18_pointers_to_functions/main}
\input{patterns/185_64bit_in_32_env/main}

\EN{\input{patterns/19_SIMD/main_EN}}
\RU{\input{patterns/19_SIMD/main_RU}}
\DE{\input{patterns/19_SIMD/main_DE}}

\EN{\input{patterns/20_x64/main_EN}}
\RU{\input{patterns/20_x64/main_RU}}

\EN{\input{patterns/205_floating_SIMD/main_EN}}
\RU{\input{patterns/205_floating_SIMD/main_RU}}
\DE{\input{patterns/205_floating_SIMD/main_DE}}

\EN{\input{patterns/ARM/main_EN}}
\RU{\input{patterns/ARM/main_RU}}
\DE{\input{patterns/ARM/main_DE}}

\input{patterns/MIPS/main}

\ifdefined\SPANISH
\chapter{Patrones de código}
\fi % SPANISH

\ifdefined\GERMAN
\chapter{Code-Muster}
\fi % GERMAN

\ifdefined\ENGLISH
\chapter{Code Patterns}
\fi % ENGLISH

\ifdefined\ITALIAN
\chapter{Forme di codice}
\fi % ITALIAN

\ifdefined\RUSSIAN
\chapter{Образцы кода}
\fi % RUSSIAN

\ifdefined\BRAZILIAN
\chapter{Padrões de códigos}
\fi % BRAZILIAN

\ifdefined\THAI
\chapter{รูปแบบของโค้ด}
\fi % THAI

\ifdefined\FRENCH
\chapter{Modèle de code}
\fi % FRENCH

\ifdefined\POLISH
\chapter{\PLph{}}
\fi % POLISH

% sections
\EN{\input{patterns/patterns_opt_dbg_EN}}
\ES{\input{patterns/patterns_opt_dbg_ES}}
\ITA{\input{patterns/patterns_opt_dbg_ITA}}
\PTBR{\input{patterns/patterns_opt_dbg_PTBR}}
\RU{\input{patterns/patterns_opt_dbg_RU}}
\THA{\input{patterns/patterns_opt_dbg_THA}}
\DE{\input{patterns/patterns_opt_dbg_DE}}
\FR{\input{patterns/patterns_opt_dbg_FR}}
\PL{\input{patterns/patterns_opt_dbg_PL}}

\RU{\section{Некоторые базовые понятия}}
\EN{\section{Some basics}}
\DE{\section{Einige Grundlagen}}
\FR{\section{Quelques bases}}
\ES{\section{\ESph{}}}
\ITA{\section{Alcune basi teoriche}}
\PTBR{\section{\PTBRph{}}}
\THA{\section{\THAph{}}}
\PL{\section{\PLph{}}}

% sections:
\EN{\input{patterns/intro_CPU_ISA_EN}}
\ES{\input{patterns/intro_CPU_ISA_ES}}
\ITA{\input{patterns/intro_CPU_ISA_ITA}}
\PTBR{\input{patterns/intro_CPU_ISA_PTBR}}
\RU{\input{patterns/intro_CPU_ISA_RU}}
\DE{\input{patterns/intro_CPU_ISA_DE}}
\FR{\input{patterns/intro_CPU_ISA_FR}}
\PL{\input{patterns/intro_CPU_ISA_PL}}

\EN{\input{patterns/numeral_EN}}
\RU{\input{patterns/numeral_RU}}
\ITA{\input{patterns/numeral_ITA}}
\DE{\input{patterns/numeral_DE}}
\FR{\input{patterns/numeral_FR}}
\PL{\input{patterns/numeral_PL}}

% chapters
\input{patterns/00_empty/main}
\input{patterns/011_ret/main}
\input{patterns/01_helloworld/main}
\input{patterns/015_prolog_epilogue/main}
\input{patterns/02_stack/main}
\input{patterns/03_printf/main}
\input{patterns/04_scanf/main}
\input{patterns/05_passing_arguments/main}
\input{patterns/06_return_results/main}
\input{patterns/061_pointers/main}
\input{patterns/065_GOTO/main}
\input{patterns/07_jcc/main}
\input{patterns/08_switch/main}
\input{patterns/09_loops/main}
\input{patterns/10_strings/main}
\input{patterns/11_arith_optimizations/main}
\input{patterns/12_FPU/main}
\input{patterns/13_arrays/main}
\input{patterns/14_bitfields/main}
\EN{\input{patterns/145_LCG/main_EN}}
\RU{\input{patterns/145_LCG/main_RU}}
\input{patterns/15_structs/main}
\input{patterns/17_unions/main}
\input{patterns/18_pointers_to_functions/main}
\input{patterns/185_64bit_in_32_env/main}

\EN{\input{patterns/19_SIMD/main_EN}}
\RU{\input{patterns/19_SIMD/main_RU}}
\DE{\input{patterns/19_SIMD/main_DE}}

\EN{\input{patterns/20_x64/main_EN}}
\RU{\input{patterns/20_x64/main_RU}}

\EN{\input{patterns/205_floating_SIMD/main_EN}}
\RU{\input{patterns/205_floating_SIMD/main_RU}}
\DE{\input{patterns/205_floating_SIMD/main_DE}}

\EN{\input{patterns/ARM/main_EN}}
\RU{\input{patterns/ARM/main_RU}}
\DE{\input{patterns/ARM/main_DE}}

\input{patterns/MIPS/main}

\ifdefined\SPANISH
\chapter{Patrones de código}
\fi % SPANISH

\ifdefined\GERMAN
\chapter{Code-Muster}
\fi % GERMAN

\ifdefined\ENGLISH
\chapter{Code Patterns}
\fi % ENGLISH

\ifdefined\ITALIAN
\chapter{Forme di codice}
\fi % ITALIAN

\ifdefined\RUSSIAN
\chapter{Образцы кода}
\fi % RUSSIAN

\ifdefined\BRAZILIAN
\chapter{Padrões de códigos}
\fi % BRAZILIAN

\ifdefined\THAI
\chapter{รูปแบบของโค้ด}
\fi % THAI

\ifdefined\FRENCH
\chapter{Modèle de code}
\fi % FRENCH

\ifdefined\POLISH
\chapter{\PLph{}}
\fi % POLISH

% sections
\EN{\input{patterns/patterns_opt_dbg_EN}}
\ES{\input{patterns/patterns_opt_dbg_ES}}
\ITA{\input{patterns/patterns_opt_dbg_ITA}}
\PTBR{\input{patterns/patterns_opt_dbg_PTBR}}
\RU{\input{patterns/patterns_opt_dbg_RU}}
\THA{\input{patterns/patterns_opt_dbg_THA}}
\DE{\input{patterns/patterns_opt_dbg_DE}}
\FR{\input{patterns/patterns_opt_dbg_FR}}
\PL{\input{patterns/patterns_opt_dbg_PL}}

\RU{\section{Некоторые базовые понятия}}
\EN{\section{Some basics}}
\DE{\section{Einige Grundlagen}}
\FR{\section{Quelques bases}}
\ES{\section{\ESph{}}}
\ITA{\section{Alcune basi teoriche}}
\PTBR{\section{\PTBRph{}}}
\THA{\section{\THAph{}}}
\PL{\section{\PLph{}}}

% sections:
\EN{\input{patterns/intro_CPU_ISA_EN}}
\ES{\input{patterns/intro_CPU_ISA_ES}}
\ITA{\input{patterns/intro_CPU_ISA_ITA}}
\PTBR{\input{patterns/intro_CPU_ISA_PTBR}}
\RU{\input{patterns/intro_CPU_ISA_RU}}
\DE{\input{patterns/intro_CPU_ISA_DE}}
\FR{\input{patterns/intro_CPU_ISA_FR}}
\PL{\input{patterns/intro_CPU_ISA_PL}}

\EN{\input{patterns/numeral_EN}}
\RU{\input{patterns/numeral_RU}}
\ITA{\input{patterns/numeral_ITA}}
\DE{\input{patterns/numeral_DE}}
\FR{\input{patterns/numeral_FR}}
\PL{\input{patterns/numeral_PL}}

% chapters
\input{patterns/00_empty/main}
\input{patterns/011_ret/main}
\input{patterns/01_helloworld/main}
\input{patterns/015_prolog_epilogue/main}
\input{patterns/02_stack/main}
\input{patterns/03_printf/main}
\input{patterns/04_scanf/main}
\input{patterns/05_passing_arguments/main}
\input{patterns/06_return_results/main}
\input{patterns/061_pointers/main}
\input{patterns/065_GOTO/main}
\input{patterns/07_jcc/main}
\input{patterns/08_switch/main}
\input{patterns/09_loops/main}
\input{patterns/10_strings/main}
\input{patterns/11_arith_optimizations/main}
\input{patterns/12_FPU/main}
\input{patterns/13_arrays/main}
\input{patterns/14_bitfields/main}
\EN{\input{patterns/145_LCG/main_EN}}
\RU{\input{patterns/145_LCG/main_RU}}
\input{patterns/15_structs/main}
\input{patterns/17_unions/main}
\input{patterns/18_pointers_to_functions/main}
\input{patterns/185_64bit_in_32_env/main}

\EN{\input{patterns/19_SIMD/main_EN}}
\RU{\input{patterns/19_SIMD/main_RU}}
\DE{\input{patterns/19_SIMD/main_DE}}

\EN{\input{patterns/20_x64/main_EN}}
\RU{\input{patterns/20_x64/main_RU}}

\EN{\input{patterns/205_floating_SIMD/main_EN}}
\RU{\input{patterns/205_floating_SIMD/main_RU}}
\DE{\input{patterns/205_floating_SIMD/main_DE}}

\EN{\input{patterns/ARM/main_EN}}
\RU{\input{patterns/ARM/main_RU}}
\DE{\input{patterns/ARM/main_DE}}

\input{patterns/MIPS/main}

\ifdefined\SPANISH
\chapter{Patrones de código}
\fi % SPANISH

\ifdefined\GERMAN
\chapter{Code-Muster}
\fi % GERMAN

\ifdefined\ENGLISH
\chapter{Code Patterns}
\fi % ENGLISH

\ifdefined\ITALIAN
\chapter{Forme di codice}
\fi % ITALIAN

\ifdefined\RUSSIAN
\chapter{Образцы кода}
\fi % RUSSIAN

\ifdefined\BRAZILIAN
\chapter{Padrões de códigos}
\fi % BRAZILIAN

\ifdefined\THAI
\chapter{รูปแบบของโค้ด}
\fi % THAI

\ifdefined\FRENCH
\chapter{Modèle de code}
\fi % FRENCH

\ifdefined\POLISH
\chapter{\PLph{}}
\fi % POLISH

% sections
\EN{\input{patterns/patterns_opt_dbg_EN}}
\ES{\input{patterns/patterns_opt_dbg_ES}}
\ITA{\input{patterns/patterns_opt_dbg_ITA}}
\PTBR{\input{patterns/patterns_opt_dbg_PTBR}}
\RU{\input{patterns/patterns_opt_dbg_RU}}
\THA{\input{patterns/patterns_opt_dbg_THA}}
\DE{\input{patterns/patterns_opt_dbg_DE}}
\FR{\input{patterns/patterns_opt_dbg_FR}}
\PL{\input{patterns/patterns_opt_dbg_PL}}

\RU{\section{Некоторые базовые понятия}}
\EN{\section{Some basics}}
\DE{\section{Einige Grundlagen}}
\FR{\section{Quelques bases}}
\ES{\section{\ESph{}}}
\ITA{\section{Alcune basi teoriche}}
\PTBR{\section{\PTBRph{}}}
\THA{\section{\THAph{}}}
\PL{\section{\PLph{}}}

% sections:
\EN{\input{patterns/intro_CPU_ISA_EN}}
\ES{\input{patterns/intro_CPU_ISA_ES}}
\ITA{\input{patterns/intro_CPU_ISA_ITA}}
\PTBR{\input{patterns/intro_CPU_ISA_PTBR}}
\RU{\input{patterns/intro_CPU_ISA_RU}}
\DE{\input{patterns/intro_CPU_ISA_DE}}
\FR{\input{patterns/intro_CPU_ISA_FR}}
\PL{\input{patterns/intro_CPU_ISA_PL}}

\EN{\input{patterns/numeral_EN}}
\RU{\input{patterns/numeral_RU}}
\ITA{\input{patterns/numeral_ITA}}
\DE{\input{patterns/numeral_DE}}
\FR{\input{patterns/numeral_FR}}
\PL{\input{patterns/numeral_PL}}

% chapters
\input{patterns/00_empty/main}
\input{patterns/011_ret/main}
\input{patterns/01_helloworld/main}
\input{patterns/015_prolog_epilogue/main}
\input{patterns/02_stack/main}
\input{patterns/03_printf/main}
\input{patterns/04_scanf/main}
\input{patterns/05_passing_arguments/main}
\input{patterns/06_return_results/main}
\input{patterns/061_pointers/main}
\input{patterns/065_GOTO/main}
\input{patterns/07_jcc/main}
\input{patterns/08_switch/main}
\input{patterns/09_loops/main}
\input{patterns/10_strings/main}
\input{patterns/11_arith_optimizations/main}
\input{patterns/12_FPU/main}
\input{patterns/13_arrays/main}
\input{patterns/14_bitfields/main}
\EN{\input{patterns/145_LCG/main_EN}}
\RU{\input{patterns/145_LCG/main_RU}}
\input{patterns/15_structs/main}
\input{patterns/17_unions/main}
\input{patterns/18_pointers_to_functions/main}
\input{patterns/185_64bit_in_32_env/main}

\EN{\input{patterns/19_SIMD/main_EN}}
\RU{\input{patterns/19_SIMD/main_RU}}
\DE{\input{patterns/19_SIMD/main_DE}}

\EN{\input{patterns/20_x64/main_EN}}
\RU{\input{patterns/20_x64/main_RU}}

\EN{\input{patterns/205_floating_SIMD/main_EN}}
\RU{\input{patterns/205_floating_SIMD/main_RU}}
\DE{\input{patterns/205_floating_SIMD/main_DE}}

\EN{\input{patterns/ARM/main_EN}}
\RU{\input{patterns/ARM/main_RU}}
\DE{\input{patterns/ARM/main_DE}}

\input{patterns/MIPS/main}

\ifdefined\SPANISH
\chapter{Patrones de código}
\fi % SPANISH

\ifdefined\GERMAN
\chapter{Code-Muster}
\fi % GERMAN

\ifdefined\ENGLISH
\chapter{Code Patterns}
\fi % ENGLISH

\ifdefined\ITALIAN
\chapter{Forme di codice}
\fi % ITALIAN

\ifdefined\RUSSIAN
\chapter{Образцы кода}
\fi % RUSSIAN

\ifdefined\BRAZILIAN
\chapter{Padrões de códigos}
\fi % BRAZILIAN

\ifdefined\THAI
\chapter{รูปแบบของโค้ด}
\fi % THAI

\ifdefined\FRENCH
\chapter{Modèle de code}
\fi % FRENCH

\ifdefined\POLISH
\chapter{\PLph{}}
\fi % POLISH

% sections
\EN{\input{patterns/patterns_opt_dbg_EN}}
\ES{\input{patterns/patterns_opt_dbg_ES}}
\ITA{\input{patterns/patterns_opt_dbg_ITA}}
\PTBR{\input{patterns/patterns_opt_dbg_PTBR}}
\RU{\input{patterns/patterns_opt_dbg_RU}}
\THA{\input{patterns/patterns_opt_dbg_THA}}
\DE{\input{patterns/patterns_opt_dbg_DE}}
\FR{\input{patterns/patterns_opt_dbg_FR}}
\PL{\input{patterns/patterns_opt_dbg_PL}}

\RU{\section{Некоторые базовые понятия}}
\EN{\section{Some basics}}
\DE{\section{Einige Grundlagen}}
\FR{\section{Quelques bases}}
\ES{\section{\ESph{}}}
\ITA{\section{Alcune basi teoriche}}
\PTBR{\section{\PTBRph{}}}
\THA{\section{\THAph{}}}
\PL{\section{\PLph{}}}

% sections:
\EN{\input{patterns/intro_CPU_ISA_EN}}
\ES{\input{patterns/intro_CPU_ISA_ES}}
\ITA{\input{patterns/intro_CPU_ISA_ITA}}
\PTBR{\input{patterns/intro_CPU_ISA_PTBR}}
\RU{\input{patterns/intro_CPU_ISA_RU}}
\DE{\input{patterns/intro_CPU_ISA_DE}}
\FR{\input{patterns/intro_CPU_ISA_FR}}
\PL{\input{patterns/intro_CPU_ISA_PL}}

\EN{\input{patterns/numeral_EN}}
\RU{\input{patterns/numeral_RU}}
\ITA{\input{patterns/numeral_ITA}}
\DE{\input{patterns/numeral_DE}}
\FR{\input{patterns/numeral_FR}}
\PL{\input{patterns/numeral_PL}}

% chapters
\input{patterns/00_empty/main}
\input{patterns/011_ret/main}
\input{patterns/01_helloworld/main}
\input{patterns/015_prolog_epilogue/main}
\input{patterns/02_stack/main}
\input{patterns/03_printf/main}
\input{patterns/04_scanf/main}
\input{patterns/05_passing_arguments/main}
\input{patterns/06_return_results/main}
\input{patterns/061_pointers/main}
\input{patterns/065_GOTO/main}
\input{patterns/07_jcc/main}
\input{patterns/08_switch/main}
\input{patterns/09_loops/main}
\input{patterns/10_strings/main}
\input{patterns/11_arith_optimizations/main}
\input{patterns/12_FPU/main}
\input{patterns/13_arrays/main}
\input{patterns/14_bitfields/main}
\EN{\input{patterns/145_LCG/main_EN}}
\RU{\input{patterns/145_LCG/main_RU}}
\input{patterns/15_structs/main}
\input{patterns/17_unions/main}
\input{patterns/18_pointers_to_functions/main}
\input{patterns/185_64bit_in_32_env/main}

\EN{\input{patterns/19_SIMD/main_EN}}
\RU{\input{patterns/19_SIMD/main_RU}}
\DE{\input{patterns/19_SIMD/main_DE}}

\EN{\input{patterns/20_x64/main_EN}}
\RU{\input{patterns/20_x64/main_RU}}

\EN{\input{patterns/205_floating_SIMD/main_EN}}
\RU{\input{patterns/205_floating_SIMD/main_RU}}
\DE{\input{patterns/205_floating_SIMD/main_DE}}

\EN{\input{patterns/ARM/main_EN}}
\RU{\input{patterns/ARM/main_RU}}
\DE{\input{patterns/ARM/main_DE}}

\input{patterns/MIPS/main}

\ifdefined\SPANISH
\chapter{Patrones de código}
\fi % SPANISH

\ifdefined\GERMAN
\chapter{Code-Muster}
\fi % GERMAN

\ifdefined\ENGLISH
\chapter{Code Patterns}
\fi % ENGLISH

\ifdefined\ITALIAN
\chapter{Forme di codice}
\fi % ITALIAN

\ifdefined\RUSSIAN
\chapter{Образцы кода}
\fi % RUSSIAN

\ifdefined\BRAZILIAN
\chapter{Padrões de códigos}
\fi % BRAZILIAN

\ifdefined\THAI
\chapter{รูปแบบของโค้ด}
\fi % THAI

\ifdefined\FRENCH
\chapter{Modèle de code}
\fi % FRENCH

\ifdefined\POLISH
\chapter{\PLph{}}
\fi % POLISH

% sections
\EN{\input{patterns/patterns_opt_dbg_EN}}
\ES{\input{patterns/patterns_opt_dbg_ES}}
\ITA{\input{patterns/patterns_opt_dbg_ITA}}
\PTBR{\input{patterns/patterns_opt_dbg_PTBR}}
\RU{\input{patterns/patterns_opt_dbg_RU}}
\THA{\input{patterns/patterns_opt_dbg_THA}}
\DE{\input{patterns/patterns_opt_dbg_DE}}
\FR{\input{patterns/patterns_opt_dbg_FR}}
\PL{\input{patterns/patterns_opt_dbg_PL}}

\RU{\section{Некоторые базовые понятия}}
\EN{\section{Some basics}}
\DE{\section{Einige Grundlagen}}
\FR{\section{Quelques bases}}
\ES{\section{\ESph{}}}
\ITA{\section{Alcune basi teoriche}}
\PTBR{\section{\PTBRph{}}}
\THA{\section{\THAph{}}}
\PL{\section{\PLph{}}}

% sections:
\EN{\input{patterns/intro_CPU_ISA_EN}}
\ES{\input{patterns/intro_CPU_ISA_ES}}
\ITA{\input{patterns/intro_CPU_ISA_ITA}}
\PTBR{\input{patterns/intro_CPU_ISA_PTBR}}
\RU{\input{patterns/intro_CPU_ISA_RU}}
\DE{\input{patterns/intro_CPU_ISA_DE}}
\FR{\input{patterns/intro_CPU_ISA_FR}}
\PL{\input{patterns/intro_CPU_ISA_PL}}

\EN{\input{patterns/numeral_EN}}
\RU{\input{patterns/numeral_RU}}
\ITA{\input{patterns/numeral_ITA}}
\DE{\input{patterns/numeral_DE}}
\FR{\input{patterns/numeral_FR}}
\PL{\input{patterns/numeral_PL}}

% chapters
\input{patterns/00_empty/main}
\input{patterns/011_ret/main}
\input{patterns/01_helloworld/main}
\input{patterns/015_prolog_epilogue/main}
\input{patterns/02_stack/main}
\input{patterns/03_printf/main}
\input{patterns/04_scanf/main}
\input{patterns/05_passing_arguments/main}
\input{patterns/06_return_results/main}
\input{patterns/061_pointers/main}
\input{patterns/065_GOTO/main}
\input{patterns/07_jcc/main}
\input{patterns/08_switch/main}
\input{patterns/09_loops/main}
\input{patterns/10_strings/main}
\input{patterns/11_arith_optimizations/main}
\input{patterns/12_FPU/main}
\input{patterns/13_arrays/main}
\input{patterns/14_bitfields/main}
\EN{\input{patterns/145_LCG/main_EN}}
\RU{\input{patterns/145_LCG/main_RU}}
\input{patterns/15_structs/main}
\input{patterns/17_unions/main}
\input{patterns/18_pointers_to_functions/main}
\input{patterns/185_64bit_in_32_env/main}

\EN{\input{patterns/19_SIMD/main_EN}}
\RU{\input{patterns/19_SIMD/main_RU}}
\DE{\input{patterns/19_SIMD/main_DE}}

\EN{\input{patterns/20_x64/main_EN}}
\RU{\input{patterns/20_x64/main_RU}}

\EN{\input{patterns/205_floating_SIMD/main_EN}}
\RU{\input{patterns/205_floating_SIMD/main_RU}}
\DE{\input{patterns/205_floating_SIMD/main_DE}}

\EN{\input{patterns/ARM/main_EN}}
\RU{\input{patterns/ARM/main_RU}}
\DE{\input{patterns/ARM/main_DE}}

\input{patterns/MIPS/main}

\ifdefined\SPANISH
\chapter{Patrones de código}
\fi % SPANISH

\ifdefined\GERMAN
\chapter{Code-Muster}
\fi % GERMAN

\ifdefined\ENGLISH
\chapter{Code Patterns}
\fi % ENGLISH

\ifdefined\ITALIAN
\chapter{Forme di codice}
\fi % ITALIAN

\ifdefined\RUSSIAN
\chapter{Образцы кода}
\fi % RUSSIAN

\ifdefined\BRAZILIAN
\chapter{Padrões de códigos}
\fi % BRAZILIAN

\ifdefined\THAI
\chapter{รูปแบบของโค้ด}
\fi % THAI

\ifdefined\FRENCH
\chapter{Modèle de code}
\fi % FRENCH

\ifdefined\POLISH
\chapter{\PLph{}}
\fi % POLISH

% sections
\EN{\input{patterns/patterns_opt_dbg_EN}}
\ES{\input{patterns/patterns_opt_dbg_ES}}
\ITA{\input{patterns/patterns_opt_dbg_ITA}}
\PTBR{\input{patterns/patterns_opt_dbg_PTBR}}
\RU{\input{patterns/patterns_opt_dbg_RU}}
\THA{\input{patterns/patterns_opt_dbg_THA}}
\DE{\input{patterns/patterns_opt_dbg_DE}}
\FR{\input{patterns/patterns_opt_dbg_FR}}
\PL{\input{patterns/patterns_opt_dbg_PL}}

\RU{\section{Некоторые базовые понятия}}
\EN{\section{Some basics}}
\DE{\section{Einige Grundlagen}}
\FR{\section{Quelques bases}}
\ES{\section{\ESph{}}}
\ITA{\section{Alcune basi teoriche}}
\PTBR{\section{\PTBRph{}}}
\THA{\section{\THAph{}}}
\PL{\section{\PLph{}}}

% sections:
\EN{\input{patterns/intro_CPU_ISA_EN}}
\ES{\input{patterns/intro_CPU_ISA_ES}}
\ITA{\input{patterns/intro_CPU_ISA_ITA}}
\PTBR{\input{patterns/intro_CPU_ISA_PTBR}}
\RU{\input{patterns/intro_CPU_ISA_RU}}
\DE{\input{patterns/intro_CPU_ISA_DE}}
\FR{\input{patterns/intro_CPU_ISA_FR}}
\PL{\input{patterns/intro_CPU_ISA_PL}}

\EN{\input{patterns/numeral_EN}}
\RU{\input{patterns/numeral_RU}}
\ITA{\input{patterns/numeral_ITA}}
\DE{\input{patterns/numeral_DE}}
\FR{\input{patterns/numeral_FR}}
\PL{\input{patterns/numeral_PL}}

% chapters
\input{patterns/00_empty/main}
\input{patterns/011_ret/main}
\input{patterns/01_helloworld/main}
\input{patterns/015_prolog_epilogue/main}
\input{patterns/02_stack/main}
\input{patterns/03_printf/main}
\input{patterns/04_scanf/main}
\input{patterns/05_passing_arguments/main}
\input{patterns/06_return_results/main}
\input{patterns/061_pointers/main}
\input{patterns/065_GOTO/main}
\input{patterns/07_jcc/main}
\input{patterns/08_switch/main}
\input{patterns/09_loops/main}
\input{patterns/10_strings/main}
\input{patterns/11_arith_optimizations/main}
\input{patterns/12_FPU/main}
\input{patterns/13_arrays/main}
\input{patterns/14_bitfields/main}
\EN{\input{patterns/145_LCG/main_EN}}
\RU{\input{patterns/145_LCG/main_RU}}
\input{patterns/15_structs/main}
\input{patterns/17_unions/main}
\input{patterns/18_pointers_to_functions/main}
\input{patterns/185_64bit_in_32_env/main}

\EN{\input{patterns/19_SIMD/main_EN}}
\RU{\input{patterns/19_SIMD/main_RU}}
\DE{\input{patterns/19_SIMD/main_DE}}

\EN{\input{patterns/20_x64/main_EN}}
\RU{\input{patterns/20_x64/main_RU}}

\EN{\input{patterns/205_floating_SIMD/main_EN}}
\RU{\input{patterns/205_floating_SIMD/main_RU}}
\DE{\input{patterns/205_floating_SIMD/main_DE}}

\EN{\input{patterns/ARM/main_EN}}
\RU{\input{patterns/ARM/main_RU}}
\DE{\input{patterns/ARM/main_DE}}

\input{patterns/MIPS/main}

\EN{\input{patterns/12_FPU/main_EN}}
\RU{\input{patterns/12_FPU/main_RU}}
\DE{\input{patterns/12_FPU/main_DE}}
\FR{\input{patterns/12_FPU/main_FR}}


\ifdefined\SPANISH
\chapter{Patrones de código}
\fi % SPANISH

\ifdefined\GERMAN
\chapter{Code-Muster}
\fi % GERMAN

\ifdefined\ENGLISH
\chapter{Code Patterns}
\fi % ENGLISH

\ifdefined\ITALIAN
\chapter{Forme di codice}
\fi % ITALIAN

\ifdefined\RUSSIAN
\chapter{Образцы кода}
\fi % RUSSIAN

\ifdefined\BRAZILIAN
\chapter{Padrões de códigos}
\fi % BRAZILIAN

\ifdefined\THAI
\chapter{รูปแบบของโค้ด}
\fi % THAI

\ifdefined\FRENCH
\chapter{Modèle de code}
\fi % FRENCH

\ifdefined\POLISH
\chapter{\PLph{}}
\fi % POLISH

% sections
\EN{\input{patterns/patterns_opt_dbg_EN}}
\ES{\input{patterns/patterns_opt_dbg_ES}}
\ITA{\input{patterns/patterns_opt_dbg_ITA}}
\PTBR{\input{patterns/patterns_opt_dbg_PTBR}}
\RU{\input{patterns/patterns_opt_dbg_RU}}
\THA{\input{patterns/patterns_opt_dbg_THA}}
\DE{\input{patterns/patterns_opt_dbg_DE}}
\FR{\input{patterns/patterns_opt_dbg_FR}}
\PL{\input{patterns/patterns_opt_dbg_PL}}

\RU{\section{Некоторые базовые понятия}}
\EN{\section{Some basics}}
\DE{\section{Einige Grundlagen}}
\FR{\section{Quelques bases}}
\ES{\section{\ESph{}}}
\ITA{\section{Alcune basi teoriche}}
\PTBR{\section{\PTBRph{}}}
\THA{\section{\THAph{}}}
\PL{\section{\PLph{}}}

% sections:
\EN{\input{patterns/intro_CPU_ISA_EN}}
\ES{\input{patterns/intro_CPU_ISA_ES}}
\ITA{\input{patterns/intro_CPU_ISA_ITA}}
\PTBR{\input{patterns/intro_CPU_ISA_PTBR}}
\RU{\input{patterns/intro_CPU_ISA_RU}}
\DE{\input{patterns/intro_CPU_ISA_DE}}
\FR{\input{patterns/intro_CPU_ISA_FR}}
\PL{\input{patterns/intro_CPU_ISA_PL}}

\EN{\input{patterns/numeral_EN}}
\RU{\input{patterns/numeral_RU}}
\ITA{\input{patterns/numeral_ITA}}
\DE{\input{patterns/numeral_DE}}
\FR{\input{patterns/numeral_FR}}
\PL{\input{patterns/numeral_PL}}

% chapters
\input{patterns/00_empty/main}
\input{patterns/011_ret/main}
\input{patterns/01_helloworld/main}
\input{patterns/015_prolog_epilogue/main}
\input{patterns/02_stack/main}
\input{patterns/03_printf/main}
\input{patterns/04_scanf/main}
\input{patterns/05_passing_arguments/main}
\input{patterns/06_return_results/main}
\input{patterns/061_pointers/main}
\input{patterns/065_GOTO/main}
\input{patterns/07_jcc/main}
\input{patterns/08_switch/main}
\input{patterns/09_loops/main}
\input{patterns/10_strings/main}
\input{patterns/11_arith_optimizations/main}
\input{patterns/12_FPU/main}
\input{patterns/13_arrays/main}
\input{patterns/14_bitfields/main}
\EN{\input{patterns/145_LCG/main_EN}}
\RU{\input{patterns/145_LCG/main_RU}}
\input{patterns/15_structs/main}
\input{patterns/17_unions/main}
\input{patterns/18_pointers_to_functions/main}
\input{patterns/185_64bit_in_32_env/main}

\EN{\input{patterns/19_SIMD/main_EN}}
\RU{\input{patterns/19_SIMD/main_RU}}
\DE{\input{patterns/19_SIMD/main_DE}}

\EN{\input{patterns/20_x64/main_EN}}
\RU{\input{patterns/20_x64/main_RU}}

\EN{\input{patterns/205_floating_SIMD/main_EN}}
\RU{\input{patterns/205_floating_SIMD/main_RU}}
\DE{\input{patterns/205_floating_SIMD/main_DE}}

\EN{\input{patterns/ARM/main_EN}}
\RU{\input{patterns/ARM/main_RU}}
\DE{\input{patterns/ARM/main_DE}}

\input{patterns/MIPS/main}

\ifdefined\SPANISH
\chapter{Patrones de código}
\fi % SPANISH

\ifdefined\GERMAN
\chapter{Code-Muster}
\fi % GERMAN

\ifdefined\ENGLISH
\chapter{Code Patterns}
\fi % ENGLISH

\ifdefined\ITALIAN
\chapter{Forme di codice}
\fi % ITALIAN

\ifdefined\RUSSIAN
\chapter{Образцы кода}
\fi % RUSSIAN

\ifdefined\BRAZILIAN
\chapter{Padrões de códigos}
\fi % BRAZILIAN

\ifdefined\THAI
\chapter{รูปแบบของโค้ด}
\fi % THAI

\ifdefined\FRENCH
\chapter{Modèle de code}
\fi % FRENCH

\ifdefined\POLISH
\chapter{\PLph{}}
\fi % POLISH

% sections
\EN{\input{patterns/patterns_opt_dbg_EN}}
\ES{\input{patterns/patterns_opt_dbg_ES}}
\ITA{\input{patterns/patterns_opt_dbg_ITA}}
\PTBR{\input{patterns/patterns_opt_dbg_PTBR}}
\RU{\input{patterns/patterns_opt_dbg_RU}}
\THA{\input{patterns/patterns_opt_dbg_THA}}
\DE{\input{patterns/patterns_opt_dbg_DE}}
\FR{\input{patterns/patterns_opt_dbg_FR}}
\PL{\input{patterns/patterns_opt_dbg_PL}}

\RU{\section{Некоторые базовые понятия}}
\EN{\section{Some basics}}
\DE{\section{Einige Grundlagen}}
\FR{\section{Quelques bases}}
\ES{\section{\ESph{}}}
\ITA{\section{Alcune basi teoriche}}
\PTBR{\section{\PTBRph{}}}
\THA{\section{\THAph{}}}
\PL{\section{\PLph{}}}

% sections:
\EN{\input{patterns/intro_CPU_ISA_EN}}
\ES{\input{patterns/intro_CPU_ISA_ES}}
\ITA{\input{patterns/intro_CPU_ISA_ITA}}
\PTBR{\input{patterns/intro_CPU_ISA_PTBR}}
\RU{\input{patterns/intro_CPU_ISA_RU}}
\DE{\input{patterns/intro_CPU_ISA_DE}}
\FR{\input{patterns/intro_CPU_ISA_FR}}
\PL{\input{patterns/intro_CPU_ISA_PL}}

\EN{\input{patterns/numeral_EN}}
\RU{\input{patterns/numeral_RU}}
\ITA{\input{patterns/numeral_ITA}}
\DE{\input{patterns/numeral_DE}}
\FR{\input{patterns/numeral_FR}}
\PL{\input{patterns/numeral_PL}}

% chapters
\input{patterns/00_empty/main}
\input{patterns/011_ret/main}
\input{patterns/01_helloworld/main}
\input{patterns/015_prolog_epilogue/main}
\input{patterns/02_stack/main}
\input{patterns/03_printf/main}
\input{patterns/04_scanf/main}
\input{patterns/05_passing_arguments/main}
\input{patterns/06_return_results/main}
\input{patterns/061_pointers/main}
\input{patterns/065_GOTO/main}
\input{patterns/07_jcc/main}
\input{patterns/08_switch/main}
\input{patterns/09_loops/main}
\input{patterns/10_strings/main}
\input{patterns/11_arith_optimizations/main}
\input{patterns/12_FPU/main}
\input{patterns/13_arrays/main}
\input{patterns/14_bitfields/main}
\EN{\input{patterns/145_LCG/main_EN}}
\RU{\input{patterns/145_LCG/main_RU}}
\input{patterns/15_structs/main}
\input{patterns/17_unions/main}
\input{patterns/18_pointers_to_functions/main}
\input{patterns/185_64bit_in_32_env/main}

\EN{\input{patterns/19_SIMD/main_EN}}
\RU{\input{patterns/19_SIMD/main_RU}}
\DE{\input{patterns/19_SIMD/main_DE}}

\EN{\input{patterns/20_x64/main_EN}}
\RU{\input{patterns/20_x64/main_RU}}

\EN{\input{patterns/205_floating_SIMD/main_EN}}
\RU{\input{patterns/205_floating_SIMD/main_RU}}
\DE{\input{patterns/205_floating_SIMD/main_DE}}

\EN{\input{patterns/ARM/main_EN}}
\RU{\input{patterns/ARM/main_RU}}
\DE{\input{patterns/ARM/main_DE}}

\input{patterns/MIPS/main}

\EN{\section{Returning Values}
\label{ret_val_func}

Another simple function is the one that simply returns a constant value:

\lstinputlisting[caption=\EN{\CCpp Code},style=customc]{patterns/011_ret/1.c}

Let's compile it.

\subsection{x86}

Here's what both the GCC and MSVC compilers produce (with optimization) on the x86 platform:

\lstinputlisting[caption=\Optimizing GCC/MSVC (\assemblyOutput),style=customasmx86]{patterns/011_ret/1.s}

\myindex{x86!\Instructions!RET}
There are just two instructions: the first places the value 123 into the \EAX register,
which is used by convention for storing the return
value, and the second one is \RET, which returns execution to the \gls{caller}.

The caller will take the result from the \EAX register.

\subsection{ARM}

There are a few differences on the ARM platform:

\lstinputlisting[caption=\OptimizingKeilVI (\ARMMode) ASM Output,style=customasmARM]{patterns/011_ret/1_Keil_ARM_O3.s}

ARM uses the register \Reg{0} for returning the results of functions, so 123 is copied into \Reg{0}.

\myindex{ARM!\Instructions!MOV}
\myindex{x86!\Instructions!MOV}
It is worth noting that \MOV is a misleading name for the instruction in both the x86 and ARM \ac{ISA}s.

The data is not in fact \IT{moved}, but \IT{copied}.

\subsection{MIPS}

\label{MIPS_leaf_function_ex1}

The GCC assembly output below lists registers by number:

\lstinputlisting[caption=\Optimizing GCC 4.4.5 (\assemblyOutput),style=customasmMIPS]{patterns/011_ret/MIPS.s}

\dots while \IDA does it by their pseudo names:

\lstinputlisting[caption=\Optimizing GCC 4.4.5 (IDA),style=customasmMIPS]{patterns/011_ret/MIPS_IDA.lst}

The \$2 (or \$V0) register is used to store the function's return value.
\myindex{MIPS!\Pseudoinstructions!LI}
\INS{LI} stands for ``Load Immediate'' and is the MIPS equivalent to \MOV.

\myindex{MIPS!\Instructions!J}
The other instruction is the jump instruction (J or JR) which returns the execution flow to the \gls{caller}.

\myindex{MIPS!Branch delay slot}
You might be wondering why the positions of the load instruction (LI) and the jump instruction (J or JR) are swapped. This is due to a \ac{RISC} feature called ``branch delay slot''.

The reason this happens is a quirk in the architecture of some RISC \ac{ISA}s and isn't important for our
purposes---we must simply keep in mind that in MIPS, the instruction following a jump or branch instruction
is executed \IT{before} the jump/branch instruction itself.

As a consequence, branch instructions always swap places with the instruction executed immediately beforehand.


In practice, functions which merely return 1 (\IT{true}) or 0 (\IT{false}) are very frequent.

The smallest ever of the standard UNIX utilities, \IT{/bin/true} and \IT{/bin/false} return 0 and 1 respectively, as an exit code.
(Zero as an exit code usually means success, non-zero means error.)
}
\RU{\subsubsection{std::string}
\myindex{\Cpp!STL!std::string}
\label{std_string}

\myparagraph{Как устроена структура}

Многие строковые библиотеки \InSqBrackets{\CNotes 2.2} обеспечивают структуру содержащую ссылку 
на буфер собственно со строкой, переменная всегда содержащую длину строки 
(что очень удобно для массы функций \InSqBrackets{\CNotes 2.2.1}) и переменную содержащую текущий размер буфера.

Строка в буфере обыкновенно оканчивается нулем: это для того чтобы указатель на буфер можно было
передавать в функции требующие на вход обычную сишную \ac{ASCIIZ}-строку.

Стандарт \Cpp не описывает, как именно нужно реализовывать std::string,
но, как правило, они реализованы как описано выше, с небольшими дополнениями.

Строки в \Cpp это не класс (как, например, QString в Qt), а темплейт (basic\_string), 
это сделано для того чтобы поддерживать 
строки содержащие разного типа символы: как минимум \Tchar и \IT{wchar\_t}.

Так что, std::string это класс с базовым типом \Tchar.

А std::wstring это класс с базовым типом \IT{wchar\_t}.

\mysubparagraph{MSVC}

В реализации MSVC, вместо ссылки на буфер может содержаться сам буфер (если строка короче 16-и символов).

Это означает, что каждая короткая строка будет занимать в памяти по крайней мере $16 + 4 + 4 = 24$ 
байт для 32-битной среды либо $16 + 8 + 8 = 32$ 
байта в 64-битной, а если строка длиннее 16-и символов, то прибавьте еще длину самой строки.

\lstinputlisting[caption=пример для MSVC,style=customc]{\CURPATH/STL/string/MSVC_RU.cpp}

Собственно, из этого исходника почти всё ясно.

Несколько замечаний:

Если строка короче 16-и символов, 
то отдельный буфер для строки в \glslink{heap}{куче} выделяться не будет.

Это удобно потому что на практике, основная часть строк действительно короткие.
Вероятно, разработчики в Microsoft выбрали размер в 16 символов как разумный баланс.

Теперь очень важный момент в конце функции main(): мы не пользуемся методом c\_str(), тем не менее,
если это скомпилировать и запустить, то обе строки появятся в консоли!

Работает это вот почему.

В первом случае строка короче 16-и символов и в начале объекта std::string (его можно рассматривать
просто как структуру) расположен буфер с этой строкой.
\printf трактует указатель как указатель на массив символов оканчивающийся нулем и поэтому всё работает.

Вывод второй строки (длиннее 16-и символов) даже еще опаснее: это вообще типичная программистская ошибка 
(или опечатка), забыть дописать c\_str().
Это работает потому что в это время в начале структуры расположен указатель на буфер.
Это может надолго остаться незамеченным: до тех пока там не появится строка 
короче 16-и символов, тогда процесс упадет.

\mysubparagraph{GCC}

В реализации GCC в структуре есть еще одна переменная --- reference count.

Интересно, что указатель на экземпляр класса std::string в GCC указывает не на начало самой структуры, 
а на указатель на буфера.
В libstdc++-v3\textbackslash{}include\textbackslash{}bits\textbackslash{}basic\_string.h 
мы можем прочитать что это сделано для удобства отладки:

\begin{lstlisting}
   *  The reason you want _M_data pointing to the character %array and
   *  not the _Rep is so that the debugger can see the string
   *  contents. (Probably we should add a non-inline member to get
   *  the _Rep for the debugger to use, so users can check the actual
   *  string length.)
\end{lstlisting}

\href{http://go.yurichev.com/17085}{исходный код basic\_string.h}

В нашем примере мы учитываем это:

\lstinputlisting[caption=пример для GCC,style=customc]{\CURPATH/STL/string/GCC_RU.cpp}

Нужны еще небольшие хаки чтобы сымитировать типичную ошибку, которую мы уже видели выше, из-за
более ужесточенной проверки типов в GCC, тем не менее, printf() работает и здесь без c\_str().

\myparagraph{Чуть более сложный пример}

\lstinputlisting[style=customc]{\CURPATH/STL/string/3.cpp}

\lstinputlisting[caption=MSVC 2012,style=customasmx86]{\CURPATH/STL/string/3_MSVC_RU.asm}

Собственно, компилятор не конструирует строки статически: да в общем-то и как
это возможно, если буфер с ней нужно хранить в \glslink{heap}{куче}?

Вместо этого в сегменте данных хранятся обычные \ac{ASCIIZ}-строки, а позже, во время выполнения, 
при помощи метода \q{assign}, конструируются строки s1 и s2
.
При помощи \TT{operator+}, создается строка s3.

Обратите внимание на то что вызов метода c\_str() отсутствует,
потому что его код достаточно короткий и компилятор вставил его прямо здесь:
если строка короче 16-и байт, то в регистре EAX остается указатель на буфер,
а если длиннее, то из этого же места достается адрес на буфер расположенный в \glslink{heap}{куче}.

Далее следуют вызовы трех деструкторов, причем, они вызываются только если строка длиннее 16-и байт:
тогда нужно освободить буфера в \glslink{heap}{куче}.
В противном случае, так как все три объекта std::string хранятся в стеке,
они освобождаются автоматически после выхода из функции.

Следовательно, работа с короткими строками более быстрая из-за м\'{е}ньшего обращения к \glslink{heap}{куче}.

Код на GCC даже проще (из-за того, что в GCC, как мы уже видели, не реализована возможность хранить короткую
строку прямо в структуре):

% TODO1 comment each function meaning
\lstinputlisting[caption=GCC 4.8.1,style=customasmx86]{\CURPATH/STL/string/3_GCC_RU.s}

Можно заметить, что в деструкторы передается не указатель на объект,
а указатель на место за 12 байт (или 3 слова) перед ним, то есть, на настоящее начало структуры.

\myparagraph{std::string как глобальная переменная}
\label{sec:std_string_as_global_variable}

Опытные программисты на \Cpp знают, что глобальные переменные \ac{STL}-типов вполне можно объявлять.

Да, действительно:

\lstinputlisting[style=customc]{\CURPATH/STL/string/5.cpp}

Но как и где будет вызываться конструктор \TT{std::string}?

На самом деле, эта переменная будет инициализирована даже перед началом \main.

\lstinputlisting[caption=MSVC 2012: здесь конструируется глобальная переменная{,} а также регистрируется её деструктор,style=customasmx86]{\CURPATH/STL/string/5_MSVC_p2.asm}

\lstinputlisting[caption=MSVC 2012: здесь глобальная переменная используется в \main,style=customasmx86]{\CURPATH/STL/string/5_MSVC_p1.asm}

\lstinputlisting[caption=MSVC 2012: эта функция-деструктор вызывается перед выходом,style=customasmx86]{\CURPATH/STL/string/5_MSVC_p3.asm}

\myindex{\CStandardLibrary!atexit()}
В реальности, из \ac{CRT}, еще до вызова main(), вызывается специальная функция,
в которой перечислены все конструкторы подобных переменных.
Более того: при помощи atexit() регистрируется функция, которая будет вызвана в конце работы программы:
в этой функции компилятор собирает вызовы деструкторов всех подобных глобальных переменных.

GCC работает похожим образом:

\lstinputlisting[caption=GCC 4.8.1,style=customasmx86]{\CURPATH/STL/string/5_GCC.s}

Но он не выделяет отдельной функции в которой будут собраны деструкторы: 
каждый деструктор передается в atexit() по одному.

% TODO а если глобальная STL-переменная в другом модуле? надо проверить.

}
\ifdefined\SPANISH
\chapter{Patrones de código}
\fi % SPANISH

\ifdefined\GERMAN
\chapter{Code-Muster}
\fi % GERMAN

\ifdefined\ENGLISH
\chapter{Code Patterns}
\fi % ENGLISH

\ifdefined\ITALIAN
\chapter{Forme di codice}
\fi % ITALIAN

\ifdefined\RUSSIAN
\chapter{Образцы кода}
\fi % RUSSIAN

\ifdefined\BRAZILIAN
\chapter{Padrões de códigos}
\fi % BRAZILIAN

\ifdefined\THAI
\chapter{รูปแบบของโค้ด}
\fi % THAI

\ifdefined\FRENCH
\chapter{Modèle de code}
\fi % FRENCH

\ifdefined\POLISH
\chapter{\PLph{}}
\fi % POLISH

% sections
\EN{\input{patterns/patterns_opt_dbg_EN}}
\ES{\input{patterns/patterns_opt_dbg_ES}}
\ITA{\input{patterns/patterns_opt_dbg_ITA}}
\PTBR{\input{patterns/patterns_opt_dbg_PTBR}}
\RU{\input{patterns/patterns_opt_dbg_RU}}
\THA{\input{patterns/patterns_opt_dbg_THA}}
\DE{\input{patterns/patterns_opt_dbg_DE}}
\FR{\input{patterns/patterns_opt_dbg_FR}}
\PL{\input{patterns/patterns_opt_dbg_PL}}

\RU{\section{Некоторые базовые понятия}}
\EN{\section{Some basics}}
\DE{\section{Einige Grundlagen}}
\FR{\section{Quelques bases}}
\ES{\section{\ESph{}}}
\ITA{\section{Alcune basi teoriche}}
\PTBR{\section{\PTBRph{}}}
\THA{\section{\THAph{}}}
\PL{\section{\PLph{}}}

% sections:
\EN{\input{patterns/intro_CPU_ISA_EN}}
\ES{\input{patterns/intro_CPU_ISA_ES}}
\ITA{\input{patterns/intro_CPU_ISA_ITA}}
\PTBR{\input{patterns/intro_CPU_ISA_PTBR}}
\RU{\input{patterns/intro_CPU_ISA_RU}}
\DE{\input{patterns/intro_CPU_ISA_DE}}
\FR{\input{patterns/intro_CPU_ISA_FR}}
\PL{\input{patterns/intro_CPU_ISA_PL}}

\EN{\input{patterns/numeral_EN}}
\RU{\input{patterns/numeral_RU}}
\ITA{\input{patterns/numeral_ITA}}
\DE{\input{patterns/numeral_DE}}
\FR{\input{patterns/numeral_FR}}
\PL{\input{patterns/numeral_PL}}

% chapters
\input{patterns/00_empty/main}
\input{patterns/011_ret/main}
\input{patterns/01_helloworld/main}
\input{patterns/015_prolog_epilogue/main}
\input{patterns/02_stack/main}
\input{patterns/03_printf/main}
\input{patterns/04_scanf/main}
\input{patterns/05_passing_arguments/main}
\input{patterns/06_return_results/main}
\input{patterns/061_pointers/main}
\input{patterns/065_GOTO/main}
\input{patterns/07_jcc/main}
\input{patterns/08_switch/main}
\input{patterns/09_loops/main}
\input{patterns/10_strings/main}
\input{patterns/11_arith_optimizations/main}
\input{patterns/12_FPU/main}
\input{patterns/13_arrays/main}
\input{patterns/14_bitfields/main}
\EN{\input{patterns/145_LCG/main_EN}}
\RU{\input{patterns/145_LCG/main_RU}}
\input{patterns/15_structs/main}
\input{patterns/17_unions/main}
\input{patterns/18_pointers_to_functions/main}
\input{patterns/185_64bit_in_32_env/main}

\EN{\input{patterns/19_SIMD/main_EN}}
\RU{\input{patterns/19_SIMD/main_RU}}
\DE{\input{patterns/19_SIMD/main_DE}}

\EN{\input{patterns/20_x64/main_EN}}
\RU{\input{patterns/20_x64/main_RU}}

\EN{\input{patterns/205_floating_SIMD/main_EN}}
\RU{\input{patterns/205_floating_SIMD/main_RU}}
\DE{\input{patterns/205_floating_SIMD/main_DE}}

\EN{\input{patterns/ARM/main_EN}}
\RU{\input{patterns/ARM/main_RU}}
\DE{\input{patterns/ARM/main_DE}}

\input{patterns/MIPS/main}

\ifdefined\SPANISH
\chapter{Patrones de código}
\fi % SPANISH

\ifdefined\GERMAN
\chapter{Code-Muster}
\fi % GERMAN

\ifdefined\ENGLISH
\chapter{Code Patterns}
\fi % ENGLISH

\ifdefined\ITALIAN
\chapter{Forme di codice}
\fi % ITALIAN

\ifdefined\RUSSIAN
\chapter{Образцы кода}
\fi % RUSSIAN

\ifdefined\BRAZILIAN
\chapter{Padrões de códigos}
\fi % BRAZILIAN

\ifdefined\THAI
\chapter{รูปแบบของโค้ด}
\fi % THAI

\ifdefined\FRENCH
\chapter{Modèle de code}
\fi % FRENCH

\ifdefined\POLISH
\chapter{\PLph{}}
\fi % POLISH

% sections
\EN{\input{patterns/patterns_opt_dbg_EN}}
\ES{\input{patterns/patterns_opt_dbg_ES}}
\ITA{\input{patterns/patterns_opt_dbg_ITA}}
\PTBR{\input{patterns/patterns_opt_dbg_PTBR}}
\RU{\input{patterns/patterns_opt_dbg_RU}}
\THA{\input{patterns/patterns_opt_dbg_THA}}
\DE{\input{patterns/patterns_opt_dbg_DE}}
\FR{\input{patterns/patterns_opt_dbg_FR}}
\PL{\input{patterns/patterns_opt_dbg_PL}}

\RU{\section{Некоторые базовые понятия}}
\EN{\section{Some basics}}
\DE{\section{Einige Grundlagen}}
\FR{\section{Quelques bases}}
\ES{\section{\ESph{}}}
\ITA{\section{Alcune basi teoriche}}
\PTBR{\section{\PTBRph{}}}
\THA{\section{\THAph{}}}
\PL{\section{\PLph{}}}

% sections:
\EN{\input{patterns/intro_CPU_ISA_EN}}
\ES{\input{patterns/intro_CPU_ISA_ES}}
\ITA{\input{patterns/intro_CPU_ISA_ITA}}
\PTBR{\input{patterns/intro_CPU_ISA_PTBR}}
\RU{\input{patterns/intro_CPU_ISA_RU}}
\DE{\input{patterns/intro_CPU_ISA_DE}}
\FR{\input{patterns/intro_CPU_ISA_FR}}
\PL{\input{patterns/intro_CPU_ISA_PL}}

\EN{\input{patterns/numeral_EN}}
\RU{\input{patterns/numeral_RU}}
\ITA{\input{patterns/numeral_ITA}}
\DE{\input{patterns/numeral_DE}}
\FR{\input{patterns/numeral_FR}}
\PL{\input{patterns/numeral_PL}}

% chapters
\input{patterns/00_empty/main}
\input{patterns/011_ret/main}
\input{patterns/01_helloworld/main}
\input{patterns/015_prolog_epilogue/main}
\input{patterns/02_stack/main}
\input{patterns/03_printf/main}
\input{patterns/04_scanf/main}
\input{patterns/05_passing_arguments/main}
\input{patterns/06_return_results/main}
\input{patterns/061_pointers/main}
\input{patterns/065_GOTO/main}
\input{patterns/07_jcc/main}
\input{patterns/08_switch/main}
\input{patterns/09_loops/main}
\input{patterns/10_strings/main}
\input{patterns/11_arith_optimizations/main}
\input{patterns/12_FPU/main}
\input{patterns/13_arrays/main}
\input{patterns/14_bitfields/main}
\EN{\input{patterns/145_LCG/main_EN}}
\RU{\input{patterns/145_LCG/main_RU}}
\input{patterns/15_structs/main}
\input{patterns/17_unions/main}
\input{patterns/18_pointers_to_functions/main}
\input{patterns/185_64bit_in_32_env/main}

\EN{\input{patterns/19_SIMD/main_EN}}
\RU{\input{patterns/19_SIMD/main_RU}}
\DE{\input{patterns/19_SIMD/main_DE}}

\EN{\input{patterns/20_x64/main_EN}}
\RU{\input{patterns/20_x64/main_RU}}

\EN{\input{patterns/205_floating_SIMD/main_EN}}
\RU{\input{patterns/205_floating_SIMD/main_RU}}
\DE{\input{patterns/205_floating_SIMD/main_DE}}

\EN{\input{patterns/ARM/main_EN}}
\RU{\input{patterns/ARM/main_RU}}
\DE{\input{patterns/ARM/main_DE}}

\input{patterns/MIPS/main}

\ifdefined\SPANISH
\chapter{Patrones de código}
\fi % SPANISH

\ifdefined\GERMAN
\chapter{Code-Muster}
\fi % GERMAN

\ifdefined\ENGLISH
\chapter{Code Patterns}
\fi % ENGLISH

\ifdefined\ITALIAN
\chapter{Forme di codice}
\fi % ITALIAN

\ifdefined\RUSSIAN
\chapter{Образцы кода}
\fi % RUSSIAN

\ifdefined\BRAZILIAN
\chapter{Padrões de códigos}
\fi % BRAZILIAN

\ifdefined\THAI
\chapter{รูปแบบของโค้ด}
\fi % THAI

\ifdefined\FRENCH
\chapter{Modèle de code}
\fi % FRENCH

\ifdefined\POLISH
\chapter{\PLph{}}
\fi % POLISH

% sections
\EN{\input{patterns/patterns_opt_dbg_EN}}
\ES{\input{patterns/patterns_opt_dbg_ES}}
\ITA{\input{patterns/patterns_opt_dbg_ITA}}
\PTBR{\input{patterns/patterns_opt_dbg_PTBR}}
\RU{\input{patterns/patterns_opt_dbg_RU}}
\THA{\input{patterns/patterns_opt_dbg_THA}}
\DE{\input{patterns/patterns_opt_dbg_DE}}
\FR{\input{patterns/patterns_opt_dbg_FR}}
\PL{\input{patterns/patterns_opt_dbg_PL}}

\RU{\section{Некоторые базовые понятия}}
\EN{\section{Some basics}}
\DE{\section{Einige Grundlagen}}
\FR{\section{Quelques bases}}
\ES{\section{\ESph{}}}
\ITA{\section{Alcune basi teoriche}}
\PTBR{\section{\PTBRph{}}}
\THA{\section{\THAph{}}}
\PL{\section{\PLph{}}}

% sections:
\EN{\input{patterns/intro_CPU_ISA_EN}}
\ES{\input{patterns/intro_CPU_ISA_ES}}
\ITA{\input{patterns/intro_CPU_ISA_ITA}}
\PTBR{\input{patterns/intro_CPU_ISA_PTBR}}
\RU{\input{patterns/intro_CPU_ISA_RU}}
\DE{\input{patterns/intro_CPU_ISA_DE}}
\FR{\input{patterns/intro_CPU_ISA_FR}}
\PL{\input{patterns/intro_CPU_ISA_PL}}

\EN{\input{patterns/numeral_EN}}
\RU{\input{patterns/numeral_RU}}
\ITA{\input{patterns/numeral_ITA}}
\DE{\input{patterns/numeral_DE}}
\FR{\input{patterns/numeral_FR}}
\PL{\input{patterns/numeral_PL}}

% chapters
\input{patterns/00_empty/main}
\input{patterns/011_ret/main}
\input{patterns/01_helloworld/main}
\input{patterns/015_prolog_epilogue/main}
\input{patterns/02_stack/main}
\input{patterns/03_printf/main}
\input{patterns/04_scanf/main}
\input{patterns/05_passing_arguments/main}
\input{patterns/06_return_results/main}
\input{patterns/061_pointers/main}
\input{patterns/065_GOTO/main}
\input{patterns/07_jcc/main}
\input{patterns/08_switch/main}
\input{patterns/09_loops/main}
\input{patterns/10_strings/main}
\input{patterns/11_arith_optimizations/main}
\input{patterns/12_FPU/main}
\input{patterns/13_arrays/main}
\input{patterns/14_bitfields/main}
\EN{\input{patterns/145_LCG/main_EN}}
\RU{\input{patterns/145_LCG/main_RU}}
\input{patterns/15_structs/main}
\input{patterns/17_unions/main}
\input{patterns/18_pointers_to_functions/main}
\input{patterns/185_64bit_in_32_env/main}

\EN{\input{patterns/19_SIMD/main_EN}}
\RU{\input{patterns/19_SIMD/main_RU}}
\DE{\input{patterns/19_SIMD/main_DE}}

\EN{\input{patterns/20_x64/main_EN}}
\RU{\input{patterns/20_x64/main_RU}}

\EN{\input{patterns/205_floating_SIMD/main_EN}}
\RU{\input{patterns/205_floating_SIMD/main_RU}}
\DE{\input{patterns/205_floating_SIMD/main_DE}}

\EN{\input{patterns/ARM/main_EN}}
\RU{\input{patterns/ARM/main_RU}}
\DE{\input{patterns/ARM/main_DE}}

\input{patterns/MIPS/main}

\ifdefined\SPANISH
\chapter{Patrones de código}
\fi % SPANISH

\ifdefined\GERMAN
\chapter{Code-Muster}
\fi % GERMAN

\ifdefined\ENGLISH
\chapter{Code Patterns}
\fi % ENGLISH

\ifdefined\ITALIAN
\chapter{Forme di codice}
\fi % ITALIAN

\ifdefined\RUSSIAN
\chapter{Образцы кода}
\fi % RUSSIAN

\ifdefined\BRAZILIAN
\chapter{Padrões de códigos}
\fi % BRAZILIAN

\ifdefined\THAI
\chapter{รูปแบบของโค้ด}
\fi % THAI

\ifdefined\FRENCH
\chapter{Modèle de code}
\fi % FRENCH

\ifdefined\POLISH
\chapter{\PLph{}}
\fi % POLISH

% sections
\EN{\input{patterns/patterns_opt_dbg_EN}}
\ES{\input{patterns/patterns_opt_dbg_ES}}
\ITA{\input{patterns/patterns_opt_dbg_ITA}}
\PTBR{\input{patterns/patterns_opt_dbg_PTBR}}
\RU{\input{patterns/patterns_opt_dbg_RU}}
\THA{\input{patterns/patterns_opt_dbg_THA}}
\DE{\input{patterns/patterns_opt_dbg_DE}}
\FR{\input{patterns/patterns_opt_dbg_FR}}
\PL{\input{patterns/patterns_opt_dbg_PL}}

\RU{\section{Некоторые базовые понятия}}
\EN{\section{Some basics}}
\DE{\section{Einige Grundlagen}}
\FR{\section{Quelques bases}}
\ES{\section{\ESph{}}}
\ITA{\section{Alcune basi teoriche}}
\PTBR{\section{\PTBRph{}}}
\THA{\section{\THAph{}}}
\PL{\section{\PLph{}}}

% sections:
\EN{\input{patterns/intro_CPU_ISA_EN}}
\ES{\input{patterns/intro_CPU_ISA_ES}}
\ITA{\input{patterns/intro_CPU_ISA_ITA}}
\PTBR{\input{patterns/intro_CPU_ISA_PTBR}}
\RU{\input{patterns/intro_CPU_ISA_RU}}
\DE{\input{patterns/intro_CPU_ISA_DE}}
\FR{\input{patterns/intro_CPU_ISA_FR}}
\PL{\input{patterns/intro_CPU_ISA_PL}}

\EN{\input{patterns/numeral_EN}}
\RU{\input{patterns/numeral_RU}}
\ITA{\input{patterns/numeral_ITA}}
\DE{\input{patterns/numeral_DE}}
\FR{\input{patterns/numeral_FR}}
\PL{\input{patterns/numeral_PL}}

% chapters
\input{patterns/00_empty/main}
\input{patterns/011_ret/main}
\input{patterns/01_helloworld/main}
\input{patterns/015_prolog_epilogue/main}
\input{patterns/02_stack/main}
\input{patterns/03_printf/main}
\input{patterns/04_scanf/main}
\input{patterns/05_passing_arguments/main}
\input{patterns/06_return_results/main}
\input{patterns/061_pointers/main}
\input{patterns/065_GOTO/main}
\input{patterns/07_jcc/main}
\input{patterns/08_switch/main}
\input{patterns/09_loops/main}
\input{patterns/10_strings/main}
\input{patterns/11_arith_optimizations/main}
\input{patterns/12_FPU/main}
\input{patterns/13_arrays/main}
\input{patterns/14_bitfields/main}
\EN{\input{patterns/145_LCG/main_EN}}
\RU{\input{patterns/145_LCG/main_RU}}
\input{patterns/15_structs/main}
\input{patterns/17_unions/main}
\input{patterns/18_pointers_to_functions/main}
\input{patterns/185_64bit_in_32_env/main}

\EN{\input{patterns/19_SIMD/main_EN}}
\RU{\input{patterns/19_SIMD/main_RU}}
\DE{\input{patterns/19_SIMD/main_DE}}

\EN{\input{patterns/20_x64/main_EN}}
\RU{\input{patterns/20_x64/main_RU}}

\EN{\input{patterns/205_floating_SIMD/main_EN}}
\RU{\input{patterns/205_floating_SIMD/main_RU}}
\DE{\input{patterns/205_floating_SIMD/main_DE}}

\EN{\input{patterns/ARM/main_EN}}
\RU{\input{patterns/ARM/main_RU}}
\DE{\input{patterns/ARM/main_DE}}

\input{patterns/MIPS/main}


\EN{\section{Returning Values}
\label{ret_val_func}

Another simple function is the one that simply returns a constant value:

\lstinputlisting[caption=\EN{\CCpp Code},style=customc]{patterns/011_ret/1.c}

Let's compile it.

\subsection{x86}

Here's what both the GCC and MSVC compilers produce (with optimization) on the x86 platform:

\lstinputlisting[caption=\Optimizing GCC/MSVC (\assemblyOutput),style=customasmx86]{patterns/011_ret/1.s}

\myindex{x86!\Instructions!RET}
There are just two instructions: the first places the value 123 into the \EAX register,
which is used by convention for storing the return
value, and the second one is \RET, which returns execution to the \gls{caller}.

The caller will take the result from the \EAX register.

\subsection{ARM}

There are a few differences on the ARM platform:

\lstinputlisting[caption=\OptimizingKeilVI (\ARMMode) ASM Output,style=customasmARM]{patterns/011_ret/1_Keil_ARM_O3.s}

ARM uses the register \Reg{0} for returning the results of functions, so 123 is copied into \Reg{0}.

\myindex{ARM!\Instructions!MOV}
\myindex{x86!\Instructions!MOV}
It is worth noting that \MOV is a misleading name for the instruction in both the x86 and ARM \ac{ISA}s.

The data is not in fact \IT{moved}, but \IT{copied}.

\subsection{MIPS}

\label{MIPS_leaf_function_ex1}

The GCC assembly output below lists registers by number:

\lstinputlisting[caption=\Optimizing GCC 4.4.5 (\assemblyOutput),style=customasmMIPS]{patterns/011_ret/MIPS.s}

\dots while \IDA does it by their pseudo names:

\lstinputlisting[caption=\Optimizing GCC 4.4.5 (IDA),style=customasmMIPS]{patterns/011_ret/MIPS_IDA.lst}

The \$2 (or \$V0) register is used to store the function's return value.
\myindex{MIPS!\Pseudoinstructions!LI}
\INS{LI} stands for ``Load Immediate'' and is the MIPS equivalent to \MOV.

\myindex{MIPS!\Instructions!J}
The other instruction is the jump instruction (J or JR) which returns the execution flow to the \gls{caller}.

\myindex{MIPS!Branch delay slot}
You might be wondering why the positions of the load instruction (LI) and the jump instruction (J or JR) are swapped. This is due to a \ac{RISC} feature called ``branch delay slot''.

The reason this happens is a quirk in the architecture of some RISC \ac{ISA}s and isn't important for our
purposes---we must simply keep in mind that in MIPS, the instruction following a jump or branch instruction
is executed \IT{before} the jump/branch instruction itself.

As a consequence, branch instructions always swap places with the instruction executed immediately beforehand.


In practice, functions which merely return 1 (\IT{true}) or 0 (\IT{false}) are very frequent.

The smallest ever of the standard UNIX utilities, \IT{/bin/true} and \IT{/bin/false} return 0 and 1 respectively, as an exit code.
(Zero as an exit code usually means success, non-zero means error.)
}
\RU{\subsubsection{std::string}
\myindex{\Cpp!STL!std::string}
\label{std_string}

\myparagraph{Как устроена структура}

Многие строковые библиотеки \InSqBrackets{\CNotes 2.2} обеспечивают структуру содержащую ссылку 
на буфер собственно со строкой, переменная всегда содержащую длину строки 
(что очень удобно для массы функций \InSqBrackets{\CNotes 2.2.1}) и переменную содержащую текущий размер буфера.

Строка в буфере обыкновенно оканчивается нулем: это для того чтобы указатель на буфер можно было
передавать в функции требующие на вход обычную сишную \ac{ASCIIZ}-строку.

Стандарт \Cpp не описывает, как именно нужно реализовывать std::string,
но, как правило, они реализованы как описано выше, с небольшими дополнениями.

Строки в \Cpp это не класс (как, например, QString в Qt), а темплейт (basic\_string), 
это сделано для того чтобы поддерживать 
строки содержащие разного типа символы: как минимум \Tchar и \IT{wchar\_t}.

Так что, std::string это класс с базовым типом \Tchar.

А std::wstring это класс с базовым типом \IT{wchar\_t}.

\mysubparagraph{MSVC}

В реализации MSVC, вместо ссылки на буфер может содержаться сам буфер (если строка короче 16-и символов).

Это означает, что каждая короткая строка будет занимать в памяти по крайней мере $16 + 4 + 4 = 24$ 
байт для 32-битной среды либо $16 + 8 + 8 = 32$ 
байта в 64-битной, а если строка длиннее 16-и символов, то прибавьте еще длину самой строки.

\lstinputlisting[caption=пример для MSVC,style=customc]{\CURPATH/STL/string/MSVC_RU.cpp}

Собственно, из этого исходника почти всё ясно.

Несколько замечаний:

Если строка короче 16-и символов, 
то отдельный буфер для строки в \glslink{heap}{куче} выделяться не будет.

Это удобно потому что на практике, основная часть строк действительно короткие.
Вероятно, разработчики в Microsoft выбрали размер в 16 символов как разумный баланс.

Теперь очень важный момент в конце функции main(): мы не пользуемся методом c\_str(), тем не менее,
если это скомпилировать и запустить, то обе строки появятся в консоли!

Работает это вот почему.

В первом случае строка короче 16-и символов и в начале объекта std::string (его можно рассматривать
просто как структуру) расположен буфер с этой строкой.
\printf трактует указатель как указатель на массив символов оканчивающийся нулем и поэтому всё работает.

Вывод второй строки (длиннее 16-и символов) даже еще опаснее: это вообще типичная программистская ошибка 
(или опечатка), забыть дописать c\_str().
Это работает потому что в это время в начале структуры расположен указатель на буфер.
Это может надолго остаться незамеченным: до тех пока там не появится строка 
короче 16-и символов, тогда процесс упадет.

\mysubparagraph{GCC}

В реализации GCC в структуре есть еще одна переменная --- reference count.

Интересно, что указатель на экземпляр класса std::string в GCC указывает не на начало самой структуры, 
а на указатель на буфера.
В libstdc++-v3\textbackslash{}include\textbackslash{}bits\textbackslash{}basic\_string.h 
мы можем прочитать что это сделано для удобства отладки:

\begin{lstlisting}
   *  The reason you want _M_data pointing to the character %array and
   *  not the _Rep is so that the debugger can see the string
   *  contents. (Probably we should add a non-inline member to get
   *  the _Rep for the debugger to use, so users can check the actual
   *  string length.)
\end{lstlisting}

\href{http://go.yurichev.com/17085}{исходный код basic\_string.h}

В нашем примере мы учитываем это:

\lstinputlisting[caption=пример для GCC,style=customc]{\CURPATH/STL/string/GCC_RU.cpp}

Нужны еще небольшие хаки чтобы сымитировать типичную ошибку, которую мы уже видели выше, из-за
более ужесточенной проверки типов в GCC, тем не менее, printf() работает и здесь без c\_str().

\myparagraph{Чуть более сложный пример}

\lstinputlisting[style=customc]{\CURPATH/STL/string/3.cpp}

\lstinputlisting[caption=MSVC 2012,style=customasmx86]{\CURPATH/STL/string/3_MSVC_RU.asm}

Собственно, компилятор не конструирует строки статически: да в общем-то и как
это возможно, если буфер с ней нужно хранить в \glslink{heap}{куче}?

Вместо этого в сегменте данных хранятся обычные \ac{ASCIIZ}-строки, а позже, во время выполнения, 
при помощи метода \q{assign}, конструируются строки s1 и s2
.
При помощи \TT{operator+}, создается строка s3.

Обратите внимание на то что вызов метода c\_str() отсутствует,
потому что его код достаточно короткий и компилятор вставил его прямо здесь:
если строка короче 16-и байт, то в регистре EAX остается указатель на буфер,
а если длиннее, то из этого же места достается адрес на буфер расположенный в \glslink{heap}{куче}.

Далее следуют вызовы трех деструкторов, причем, они вызываются только если строка длиннее 16-и байт:
тогда нужно освободить буфера в \glslink{heap}{куче}.
В противном случае, так как все три объекта std::string хранятся в стеке,
они освобождаются автоматически после выхода из функции.

Следовательно, работа с короткими строками более быстрая из-за м\'{е}ньшего обращения к \glslink{heap}{куче}.

Код на GCC даже проще (из-за того, что в GCC, как мы уже видели, не реализована возможность хранить короткую
строку прямо в структуре):

% TODO1 comment each function meaning
\lstinputlisting[caption=GCC 4.8.1,style=customasmx86]{\CURPATH/STL/string/3_GCC_RU.s}

Можно заметить, что в деструкторы передается не указатель на объект,
а указатель на место за 12 байт (или 3 слова) перед ним, то есть, на настоящее начало структуры.

\myparagraph{std::string как глобальная переменная}
\label{sec:std_string_as_global_variable}

Опытные программисты на \Cpp знают, что глобальные переменные \ac{STL}-типов вполне можно объявлять.

Да, действительно:

\lstinputlisting[style=customc]{\CURPATH/STL/string/5.cpp}

Но как и где будет вызываться конструктор \TT{std::string}?

На самом деле, эта переменная будет инициализирована даже перед началом \main.

\lstinputlisting[caption=MSVC 2012: здесь конструируется глобальная переменная{,} а также регистрируется её деструктор,style=customasmx86]{\CURPATH/STL/string/5_MSVC_p2.asm}

\lstinputlisting[caption=MSVC 2012: здесь глобальная переменная используется в \main,style=customasmx86]{\CURPATH/STL/string/5_MSVC_p1.asm}

\lstinputlisting[caption=MSVC 2012: эта функция-деструктор вызывается перед выходом,style=customasmx86]{\CURPATH/STL/string/5_MSVC_p3.asm}

\myindex{\CStandardLibrary!atexit()}
В реальности, из \ac{CRT}, еще до вызова main(), вызывается специальная функция,
в которой перечислены все конструкторы подобных переменных.
Более того: при помощи atexit() регистрируется функция, которая будет вызвана в конце работы программы:
в этой функции компилятор собирает вызовы деструкторов всех подобных глобальных переменных.

GCC работает похожим образом:

\lstinputlisting[caption=GCC 4.8.1,style=customasmx86]{\CURPATH/STL/string/5_GCC.s}

Но он не выделяет отдельной функции в которой будут собраны деструкторы: 
каждый деструктор передается в atexit() по одному.

% TODO а если глобальная STL-переменная в другом модуле? надо проверить.

}
\DE{\subsection{Einfachste XOR-Verschlüsselung überhaupt}

Ich habe einmal eine Software gesehen, bei der alle Debugging-Ausgaben mit XOR mit dem Wert 3
verschlüsselt wurden. Mit anderen Worten, die beiden niedrigsten Bits aller Buchstaben wurden invertiert.

``Hello, world'' wurde zu ``Kfool/\#tlqog'':

\begin{lstlisting}
#!/usr/bin/python

msg="Hello, world!"

print "".join(map(lambda x: chr(ord(x)^3), msg))
\end{lstlisting}

Das ist eine ziemlich interessante Verschlüsselung (oder besser eine Verschleierung),
weil sie zwei wichtige Eigenschaften hat:
1) es ist eine einzige Funktion zum Verschlüsseln und entschlüsseln, sie muss nur wiederholt angewendet werden
2) die entstehenden Buchstaben befinden sich im druckbaren Bereich, also die ganze Zeichenkette kann ohne
Escape-Symbole im Code verwendet werden.

Die zweite Eigenschaft nutzt die Tatsache, dass alle druckbaren Zeichen in Reihen organisiert sind: 0x2x-0x7x,
und wenn die beiden niederwertigsten Bits invertiert werden, wird der Buchstabe um eine oder drei Stellen nach
links oder rechts \IT{verschoben}, aber niemals in eine andere Reihe:

\begin{figure}[H]
\centering
\includegraphics[width=0.7\textwidth]{ascii_clean.png}
\caption{7-Bit \ac{ASCII} Tabelle in Emacs}
\end{figure}

\dots mit dem Zeichen 0x7F als einziger Ausnahme.

Im Folgenden werden also beispielsweise die Zeichen A-Z \IT{verschlüsselt}:

\begin{lstlisting}
#!/usr/bin/python

msg="@ABCDEFGHIJKLMNO"

print "".join(map(lambda x: chr(ord(x)^3), msg))
\end{lstlisting}

Ergebnis:
% FIXME \verb  --  relevant comment for German?
\begin{lstlisting}
CBA@GFEDKJIHONML
\end{lstlisting}

Es sieht so aus als würden die Zeichen ``@'' und ``C'' sowie ``B'' und ``A'' vertauscht werden.

Hier ist noch ein interessantes Beispiel, in dem gezeigt wird, wie die Eigenschaften von XOR
ausgenutzt werden können: Exakt den gleichen Effekt, dass druckbare Zeichen auch druckbar bleiben,
kann man dadurch erzielen, dass irgendeine Kombination der niedrigsten vier Bits invertiert wird.
}

\EN{\section{Returning Values}
\label{ret_val_func}

Another simple function is the one that simply returns a constant value:

\lstinputlisting[caption=\EN{\CCpp Code},style=customc]{patterns/011_ret/1.c}

Let's compile it.

\subsection{x86}

Here's what both the GCC and MSVC compilers produce (with optimization) on the x86 platform:

\lstinputlisting[caption=\Optimizing GCC/MSVC (\assemblyOutput),style=customasmx86]{patterns/011_ret/1.s}

\myindex{x86!\Instructions!RET}
There are just two instructions: the first places the value 123 into the \EAX register,
which is used by convention for storing the return
value, and the second one is \RET, which returns execution to the \gls{caller}.

The caller will take the result from the \EAX register.

\subsection{ARM}

There are a few differences on the ARM platform:

\lstinputlisting[caption=\OptimizingKeilVI (\ARMMode) ASM Output,style=customasmARM]{patterns/011_ret/1_Keil_ARM_O3.s}

ARM uses the register \Reg{0} for returning the results of functions, so 123 is copied into \Reg{0}.

\myindex{ARM!\Instructions!MOV}
\myindex{x86!\Instructions!MOV}
It is worth noting that \MOV is a misleading name for the instruction in both the x86 and ARM \ac{ISA}s.

The data is not in fact \IT{moved}, but \IT{copied}.

\subsection{MIPS}

\label{MIPS_leaf_function_ex1}

The GCC assembly output below lists registers by number:

\lstinputlisting[caption=\Optimizing GCC 4.4.5 (\assemblyOutput),style=customasmMIPS]{patterns/011_ret/MIPS.s}

\dots while \IDA does it by their pseudo names:

\lstinputlisting[caption=\Optimizing GCC 4.4.5 (IDA),style=customasmMIPS]{patterns/011_ret/MIPS_IDA.lst}

The \$2 (or \$V0) register is used to store the function's return value.
\myindex{MIPS!\Pseudoinstructions!LI}
\INS{LI} stands for ``Load Immediate'' and is the MIPS equivalent to \MOV.

\myindex{MIPS!\Instructions!J}
The other instruction is the jump instruction (J or JR) which returns the execution flow to the \gls{caller}.

\myindex{MIPS!Branch delay slot}
You might be wondering why the positions of the load instruction (LI) and the jump instruction (J or JR) are swapped. This is due to a \ac{RISC} feature called ``branch delay slot''.

The reason this happens is a quirk in the architecture of some RISC \ac{ISA}s and isn't important for our
purposes---we must simply keep in mind that in MIPS, the instruction following a jump or branch instruction
is executed \IT{before} the jump/branch instruction itself.

As a consequence, branch instructions always swap places with the instruction executed immediately beforehand.


In practice, functions which merely return 1 (\IT{true}) or 0 (\IT{false}) are very frequent.

The smallest ever of the standard UNIX utilities, \IT{/bin/true} and \IT{/bin/false} return 0 and 1 respectively, as an exit code.
(Zero as an exit code usually means success, non-zero means error.)
}
\RU{\subsubsection{std::string}
\myindex{\Cpp!STL!std::string}
\label{std_string}

\myparagraph{Как устроена структура}

Многие строковые библиотеки \InSqBrackets{\CNotes 2.2} обеспечивают структуру содержащую ссылку 
на буфер собственно со строкой, переменная всегда содержащую длину строки 
(что очень удобно для массы функций \InSqBrackets{\CNotes 2.2.1}) и переменную содержащую текущий размер буфера.

Строка в буфере обыкновенно оканчивается нулем: это для того чтобы указатель на буфер можно было
передавать в функции требующие на вход обычную сишную \ac{ASCIIZ}-строку.

Стандарт \Cpp не описывает, как именно нужно реализовывать std::string,
но, как правило, они реализованы как описано выше, с небольшими дополнениями.

Строки в \Cpp это не класс (как, например, QString в Qt), а темплейт (basic\_string), 
это сделано для того чтобы поддерживать 
строки содержащие разного типа символы: как минимум \Tchar и \IT{wchar\_t}.

Так что, std::string это класс с базовым типом \Tchar.

А std::wstring это класс с базовым типом \IT{wchar\_t}.

\mysubparagraph{MSVC}

В реализации MSVC, вместо ссылки на буфер может содержаться сам буфер (если строка короче 16-и символов).

Это означает, что каждая короткая строка будет занимать в памяти по крайней мере $16 + 4 + 4 = 24$ 
байт для 32-битной среды либо $16 + 8 + 8 = 32$ 
байта в 64-битной, а если строка длиннее 16-и символов, то прибавьте еще длину самой строки.

\lstinputlisting[caption=пример для MSVC,style=customc]{\CURPATH/STL/string/MSVC_RU.cpp}

Собственно, из этого исходника почти всё ясно.

Несколько замечаний:

Если строка короче 16-и символов, 
то отдельный буфер для строки в \glslink{heap}{куче} выделяться не будет.

Это удобно потому что на практике, основная часть строк действительно короткие.
Вероятно, разработчики в Microsoft выбрали размер в 16 символов как разумный баланс.

Теперь очень важный момент в конце функции main(): мы не пользуемся методом c\_str(), тем не менее,
если это скомпилировать и запустить, то обе строки появятся в консоли!

Работает это вот почему.

В первом случае строка короче 16-и символов и в начале объекта std::string (его можно рассматривать
просто как структуру) расположен буфер с этой строкой.
\printf трактует указатель как указатель на массив символов оканчивающийся нулем и поэтому всё работает.

Вывод второй строки (длиннее 16-и символов) даже еще опаснее: это вообще типичная программистская ошибка 
(или опечатка), забыть дописать c\_str().
Это работает потому что в это время в начале структуры расположен указатель на буфер.
Это может надолго остаться незамеченным: до тех пока там не появится строка 
короче 16-и символов, тогда процесс упадет.

\mysubparagraph{GCC}

В реализации GCC в структуре есть еще одна переменная --- reference count.

Интересно, что указатель на экземпляр класса std::string в GCC указывает не на начало самой структуры, 
а на указатель на буфера.
В libstdc++-v3\textbackslash{}include\textbackslash{}bits\textbackslash{}basic\_string.h 
мы можем прочитать что это сделано для удобства отладки:

\begin{lstlisting}
   *  The reason you want _M_data pointing to the character %array and
   *  not the _Rep is so that the debugger can see the string
   *  contents. (Probably we should add a non-inline member to get
   *  the _Rep for the debugger to use, so users can check the actual
   *  string length.)
\end{lstlisting}

\href{http://go.yurichev.com/17085}{исходный код basic\_string.h}

В нашем примере мы учитываем это:

\lstinputlisting[caption=пример для GCC,style=customc]{\CURPATH/STL/string/GCC_RU.cpp}

Нужны еще небольшие хаки чтобы сымитировать типичную ошибку, которую мы уже видели выше, из-за
более ужесточенной проверки типов в GCC, тем не менее, printf() работает и здесь без c\_str().

\myparagraph{Чуть более сложный пример}

\lstinputlisting[style=customc]{\CURPATH/STL/string/3.cpp}

\lstinputlisting[caption=MSVC 2012,style=customasmx86]{\CURPATH/STL/string/3_MSVC_RU.asm}

Собственно, компилятор не конструирует строки статически: да в общем-то и как
это возможно, если буфер с ней нужно хранить в \glslink{heap}{куче}?

Вместо этого в сегменте данных хранятся обычные \ac{ASCIIZ}-строки, а позже, во время выполнения, 
при помощи метода \q{assign}, конструируются строки s1 и s2
.
При помощи \TT{operator+}, создается строка s3.

Обратите внимание на то что вызов метода c\_str() отсутствует,
потому что его код достаточно короткий и компилятор вставил его прямо здесь:
если строка короче 16-и байт, то в регистре EAX остается указатель на буфер,
а если длиннее, то из этого же места достается адрес на буфер расположенный в \glslink{heap}{куче}.

Далее следуют вызовы трех деструкторов, причем, они вызываются только если строка длиннее 16-и байт:
тогда нужно освободить буфера в \glslink{heap}{куче}.
В противном случае, так как все три объекта std::string хранятся в стеке,
они освобождаются автоматически после выхода из функции.

Следовательно, работа с короткими строками более быстрая из-за м\'{е}ньшего обращения к \glslink{heap}{куче}.

Код на GCC даже проще (из-за того, что в GCC, как мы уже видели, не реализована возможность хранить короткую
строку прямо в структуре):

% TODO1 comment each function meaning
\lstinputlisting[caption=GCC 4.8.1,style=customasmx86]{\CURPATH/STL/string/3_GCC_RU.s}

Можно заметить, что в деструкторы передается не указатель на объект,
а указатель на место за 12 байт (или 3 слова) перед ним, то есть, на настоящее начало структуры.

\myparagraph{std::string как глобальная переменная}
\label{sec:std_string_as_global_variable}

Опытные программисты на \Cpp знают, что глобальные переменные \ac{STL}-типов вполне можно объявлять.

Да, действительно:

\lstinputlisting[style=customc]{\CURPATH/STL/string/5.cpp}

Но как и где будет вызываться конструктор \TT{std::string}?

На самом деле, эта переменная будет инициализирована даже перед началом \main.

\lstinputlisting[caption=MSVC 2012: здесь конструируется глобальная переменная{,} а также регистрируется её деструктор,style=customasmx86]{\CURPATH/STL/string/5_MSVC_p2.asm}

\lstinputlisting[caption=MSVC 2012: здесь глобальная переменная используется в \main,style=customasmx86]{\CURPATH/STL/string/5_MSVC_p1.asm}

\lstinputlisting[caption=MSVC 2012: эта функция-деструктор вызывается перед выходом,style=customasmx86]{\CURPATH/STL/string/5_MSVC_p3.asm}

\myindex{\CStandardLibrary!atexit()}
В реальности, из \ac{CRT}, еще до вызова main(), вызывается специальная функция,
в которой перечислены все конструкторы подобных переменных.
Более того: при помощи atexit() регистрируется функция, которая будет вызвана в конце работы программы:
в этой функции компилятор собирает вызовы деструкторов всех подобных глобальных переменных.

GCC работает похожим образом:

\lstinputlisting[caption=GCC 4.8.1,style=customasmx86]{\CURPATH/STL/string/5_GCC.s}

Но он не выделяет отдельной функции в которой будут собраны деструкторы: 
каждый деструктор передается в atexit() по одному.

% TODO а если глобальная STL-переменная в другом модуле? надо проверить.

}

\EN{\section{Returning Values}
\label{ret_val_func}

Another simple function is the one that simply returns a constant value:

\lstinputlisting[caption=\EN{\CCpp Code},style=customc]{patterns/011_ret/1.c}

Let's compile it.

\subsection{x86}

Here's what both the GCC and MSVC compilers produce (with optimization) on the x86 platform:

\lstinputlisting[caption=\Optimizing GCC/MSVC (\assemblyOutput),style=customasmx86]{patterns/011_ret/1.s}

\myindex{x86!\Instructions!RET}
There are just two instructions: the first places the value 123 into the \EAX register,
which is used by convention for storing the return
value, and the second one is \RET, which returns execution to the \gls{caller}.

The caller will take the result from the \EAX register.

\subsection{ARM}

There are a few differences on the ARM platform:

\lstinputlisting[caption=\OptimizingKeilVI (\ARMMode) ASM Output,style=customasmARM]{patterns/011_ret/1_Keil_ARM_O3.s}

ARM uses the register \Reg{0} for returning the results of functions, so 123 is copied into \Reg{0}.

\myindex{ARM!\Instructions!MOV}
\myindex{x86!\Instructions!MOV}
It is worth noting that \MOV is a misleading name for the instruction in both the x86 and ARM \ac{ISA}s.

The data is not in fact \IT{moved}, but \IT{copied}.

\subsection{MIPS}

\label{MIPS_leaf_function_ex1}

The GCC assembly output below lists registers by number:

\lstinputlisting[caption=\Optimizing GCC 4.4.5 (\assemblyOutput),style=customasmMIPS]{patterns/011_ret/MIPS.s}

\dots while \IDA does it by their pseudo names:

\lstinputlisting[caption=\Optimizing GCC 4.4.5 (IDA),style=customasmMIPS]{patterns/011_ret/MIPS_IDA.lst}

The \$2 (or \$V0) register is used to store the function's return value.
\myindex{MIPS!\Pseudoinstructions!LI}
\INS{LI} stands for ``Load Immediate'' and is the MIPS equivalent to \MOV.

\myindex{MIPS!\Instructions!J}
The other instruction is the jump instruction (J or JR) which returns the execution flow to the \gls{caller}.

\myindex{MIPS!Branch delay slot}
You might be wondering why the positions of the load instruction (LI) and the jump instruction (J or JR) are swapped. This is due to a \ac{RISC} feature called ``branch delay slot''.

The reason this happens is a quirk in the architecture of some RISC \ac{ISA}s and isn't important for our
purposes---we must simply keep in mind that in MIPS, the instruction following a jump or branch instruction
is executed \IT{before} the jump/branch instruction itself.

As a consequence, branch instructions always swap places with the instruction executed immediately beforehand.


In practice, functions which merely return 1 (\IT{true}) or 0 (\IT{false}) are very frequent.

The smallest ever of the standard UNIX utilities, \IT{/bin/true} and \IT{/bin/false} return 0 and 1 respectively, as an exit code.
(Zero as an exit code usually means success, non-zero means error.)
}
\RU{\subsubsection{std::string}
\myindex{\Cpp!STL!std::string}
\label{std_string}

\myparagraph{Как устроена структура}

Многие строковые библиотеки \InSqBrackets{\CNotes 2.2} обеспечивают структуру содержащую ссылку 
на буфер собственно со строкой, переменная всегда содержащую длину строки 
(что очень удобно для массы функций \InSqBrackets{\CNotes 2.2.1}) и переменную содержащую текущий размер буфера.

Строка в буфере обыкновенно оканчивается нулем: это для того чтобы указатель на буфер можно было
передавать в функции требующие на вход обычную сишную \ac{ASCIIZ}-строку.

Стандарт \Cpp не описывает, как именно нужно реализовывать std::string,
но, как правило, они реализованы как описано выше, с небольшими дополнениями.

Строки в \Cpp это не класс (как, например, QString в Qt), а темплейт (basic\_string), 
это сделано для того чтобы поддерживать 
строки содержащие разного типа символы: как минимум \Tchar и \IT{wchar\_t}.

Так что, std::string это класс с базовым типом \Tchar.

А std::wstring это класс с базовым типом \IT{wchar\_t}.

\mysubparagraph{MSVC}

В реализации MSVC, вместо ссылки на буфер может содержаться сам буфер (если строка короче 16-и символов).

Это означает, что каждая короткая строка будет занимать в памяти по крайней мере $16 + 4 + 4 = 24$ 
байт для 32-битной среды либо $16 + 8 + 8 = 32$ 
байта в 64-битной, а если строка длиннее 16-и символов, то прибавьте еще длину самой строки.

\lstinputlisting[caption=пример для MSVC,style=customc]{\CURPATH/STL/string/MSVC_RU.cpp}

Собственно, из этого исходника почти всё ясно.

Несколько замечаний:

Если строка короче 16-и символов, 
то отдельный буфер для строки в \glslink{heap}{куче} выделяться не будет.

Это удобно потому что на практике, основная часть строк действительно короткие.
Вероятно, разработчики в Microsoft выбрали размер в 16 символов как разумный баланс.

Теперь очень важный момент в конце функции main(): мы не пользуемся методом c\_str(), тем не менее,
если это скомпилировать и запустить, то обе строки появятся в консоли!

Работает это вот почему.

В первом случае строка короче 16-и символов и в начале объекта std::string (его можно рассматривать
просто как структуру) расположен буфер с этой строкой.
\printf трактует указатель как указатель на массив символов оканчивающийся нулем и поэтому всё работает.

Вывод второй строки (длиннее 16-и символов) даже еще опаснее: это вообще типичная программистская ошибка 
(или опечатка), забыть дописать c\_str().
Это работает потому что в это время в начале структуры расположен указатель на буфер.
Это может надолго остаться незамеченным: до тех пока там не появится строка 
короче 16-и символов, тогда процесс упадет.

\mysubparagraph{GCC}

В реализации GCC в структуре есть еще одна переменная --- reference count.

Интересно, что указатель на экземпляр класса std::string в GCC указывает не на начало самой структуры, 
а на указатель на буфера.
В libstdc++-v3\textbackslash{}include\textbackslash{}bits\textbackslash{}basic\_string.h 
мы можем прочитать что это сделано для удобства отладки:

\begin{lstlisting}
   *  The reason you want _M_data pointing to the character %array and
   *  not the _Rep is so that the debugger can see the string
   *  contents. (Probably we should add a non-inline member to get
   *  the _Rep for the debugger to use, so users can check the actual
   *  string length.)
\end{lstlisting}

\href{http://go.yurichev.com/17085}{исходный код basic\_string.h}

В нашем примере мы учитываем это:

\lstinputlisting[caption=пример для GCC,style=customc]{\CURPATH/STL/string/GCC_RU.cpp}

Нужны еще небольшие хаки чтобы сымитировать типичную ошибку, которую мы уже видели выше, из-за
более ужесточенной проверки типов в GCC, тем не менее, printf() работает и здесь без c\_str().

\myparagraph{Чуть более сложный пример}

\lstinputlisting[style=customc]{\CURPATH/STL/string/3.cpp}

\lstinputlisting[caption=MSVC 2012,style=customasmx86]{\CURPATH/STL/string/3_MSVC_RU.asm}

Собственно, компилятор не конструирует строки статически: да в общем-то и как
это возможно, если буфер с ней нужно хранить в \glslink{heap}{куче}?

Вместо этого в сегменте данных хранятся обычные \ac{ASCIIZ}-строки, а позже, во время выполнения, 
при помощи метода \q{assign}, конструируются строки s1 и s2
.
При помощи \TT{operator+}, создается строка s3.

Обратите внимание на то что вызов метода c\_str() отсутствует,
потому что его код достаточно короткий и компилятор вставил его прямо здесь:
если строка короче 16-и байт, то в регистре EAX остается указатель на буфер,
а если длиннее, то из этого же места достается адрес на буфер расположенный в \glslink{heap}{куче}.

Далее следуют вызовы трех деструкторов, причем, они вызываются только если строка длиннее 16-и байт:
тогда нужно освободить буфера в \glslink{heap}{куче}.
В противном случае, так как все три объекта std::string хранятся в стеке,
они освобождаются автоматически после выхода из функции.

Следовательно, работа с короткими строками более быстрая из-за м\'{е}ньшего обращения к \glslink{heap}{куче}.

Код на GCC даже проще (из-за того, что в GCC, как мы уже видели, не реализована возможность хранить короткую
строку прямо в структуре):

% TODO1 comment each function meaning
\lstinputlisting[caption=GCC 4.8.1,style=customasmx86]{\CURPATH/STL/string/3_GCC_RU.s}

Можно заметить, что в деструкторы передается не указатель на объект,
а указатель на место за 12 байт (или 3 слова) перед ним, то есть, на настоящее начало структуры.

\myparagraph{std::string как глобальная переменная}
\label{sec:std_string_as_global_variable}

Опытные программисты на \Cpp знают, что глобальные переменные \ac{STL}-типов вполне можно объявлять.

Да, действительно:

\lstinputlisting[style=customc]{\CURPATH/STL/string/5.cpp}

Но как и где будет вызываться конструктор \TT{std::string}?

На самом деле, эта переменная будет инициализирована даже перед началом \main.

\lstinputlisting[caption=MSVC 2012: здесь конструируется глобальная переменная{,} а также регистрируется её деструктор,style=customasmx86]{\CURPATH/STL/string/5_MSVC_p2.asm}

\lstinputlisting[caption=MSVC 2012: здесь глобальная переменная используется в \main,style=customasmx86]{\CURPATH/STL/string/5_MSVC_p1.asm}

\lstinputlisting[caption=MSVC 2012: эта функция-деструктор вызывается перед выходом,style=customasmx86]{\CURPATH/STL/string/5_MSVC_p3.asm}

\myindex{\CStandardLibrary!atexit()}
В реальности, из \ac{CRT}, еще до вызова main(), вызывается специальная функция,
в которой перечислены все конструкторы подобных переменных.
Более того: при помощи atexit() регистрируется функция, которая будет вызвана в конце работы программы:
в этой функции компилятор собирает вызовы деструкторов всех подобных глобальных переменных.

GCC работает похожим образом:

\lstinputlisting[caption=GCC 4.8.1,style=customasmx86]{\CURPATH/STL/string/5_GCC.s}

Но он не выделяет отдельной функции в которой будут собраны деструкторы: 
каждый деструктор передается в atexit() по одному.

% TODO а если глобальная STL-переменная в другом модуле? надо проверить.

}
\DE{\subsection{Einfachste XOR-Verschlüsselung überhaupt}

Ich habe einmal eine Software gesehen, bei der alle Debugging-Ausgaben mit XOR mit dem Wert 3
verschlüsselt wurden. Mit anderen Worten, die beiden niedrigsten Bits aller Buchstaben wurden invertiert.

``Hello, world'' wurde zu ``Kfool/\#tlqog'':

\begin{lstlisting}
#!/usr/bin/python

msg="Hello, world!"

print "".join(map(lambda x: chr(ord(x)^3), msg))
\end{lstlisting}

Das ist eine ziemlich interessante Verschlüsselung (oder besser eine Verschleierung),
weil sie zwei wichtige Eigenschaften hat:
1) es ist eine einzige Funktion zum Verschlüsseln und entschlüsseln, sie muss nur wiederholt angewendet werden
2) die entstehenden Buchstaben befinden sich im druckbaren Bereich, also die ganze Zeichenkette kann ohne
Escape-Symbole im Code verwendet werden.

Die zweite Eigenschaft nutzt die Tatsache, dass alle druckbaren Zeichen in Reihen organisiert sind: 0x2x-0x7x,
und wenn die beiden niederwertigsten Bits invertiert werden, wird der Buchstabe um eine oder drei Stellen nach
links oder rechts \IT{verschoben}, aber niemals in eine andere Reihe:

\begin{figure}[H]
\centering
\includegraphics[width=0.7\textwidth]{ascii_clean.png}
\caption{7-Bit \ac{ASCII} Tabelle in Emacs}
\end{figure}

\dots mit dem Zeichen 0x7F als einziger Ausnahme.

Im Folgenden werden also beispielsweise die Zeichen A-Z \IT{verschlüsselt}:

\begin{lstlisting}
#!/usr/bin/python

msg="@ABCDEFGHIJKLMNO"

print "".join(map(lambda x: chr(ord(x)^3), msg))
\end{lstlisting}

Ergebnis:
% FIXME \verb  --  relevant comment for German?
\begin{lstlisting}
CBA@GFEDKJIHONML
\end{lstlisting}

Es sieht so aus als würden die Zeichen ``@'' und ``C'' sowie ``B'' und ``A'' vertauscht werden.

Hier ist noch ein interessantes Beispiel, in dem gezeigt wird, wie die Eigenschaften von XOR
ausgenutzt werden können: Exakt den gleichen Effekt, dass druckbare Zeichen auch druckbar bleiben,
kann man dadurch erzielen, dass irgendeine Kombination der niedrigsten vier Bits invertiert wird.
}

\EN{\section{Returning Values}
\label{ret_val_func}

Another simple function is the one that simply returns a constant value:

\lstinputlisting[caption=\EN{\CCpp Code},style=customc]{patterns/011_ret/1.c}

Let's compile it.

\subsection{x86}

Here's what both the GCC and MSVC compilers produce (with optimization) on the x86 platform:

\lstinputlisting[caption=\Optimizing GCC/MSVC (\assemblyOutput),style=customasmx86]{patterns/011_ret/1.s}

\myindex{x86!\Instructions!RET}
There are just two instructions: the first places the value 123 into the \EAX register,
which is used by convention for storing the return
value, and the second one is \RET, which returns execution to the \gls{caller}.

The caller will take the result from the \EAX register.

\subsection{ARM}

There are a few differences on the ARM platform:

\lstinputlisting[caption=\OptimizingKeilVI (\ARMMode) ASM Output,style=customasmARM]{patterns/011_ret/1_Keil_ARM_O3.s}

ARM uses the register \Reg{0} for returning the results of functions, so 123 is copied into \Reg{0}.

\myindex{ARM!\Instructions!MOV}
\myindex{x86!\Instructions!MOV}
It is worth noting that \MOV is a misleading name for the instruction in both the x86 and ARM \ac{ISA}s.

The data is not in fact \IT{moved}, but \IT{copied}.

\subsection{MIPS}

\label{MIPS_leaf_function_ex1}

The GCC assembly output below lists registers by number:

\lstinputlisting[caption=\Optimizing GCC 4.4.5 (\assemblyOutput),style=customasmMIPS]{patterns/011_ret/MIPS.s}

\dots while \IDA does it by their pseudo names:

\lstinputlisting[caption=\Optimizing GCC 4.4.5 (IDA),style=customasmMIPS]{patterns/011_ret/MIPS_IDA.lst}

The \$2 (or \$V0) register is used to store the function's return value.
\myindex{MIPS!\Pseudoinstructions!LI}
\INS{LI} stands for ``Load Immediate'' and is the MIPS equivalent to \MOV.

\myindex{MIPS!\Instructions!J}
The other instruction is the jump instruction (J or JR) which returns the execution flow to the \gls{caller}.

\myindex{MIPS!Branch delay slot}
You might be wondering why the positions of the load instruction (LI) and the jump instruction (J or JR) are swapped. This is due to a \ac{RISC} feature called ``branch delay slot''.

The reason this happens is a quirk in the architecture of some RISC \ac{ISA}s and isn't important for our
purposes---we must simply keep in mind that in MIPS, the instruction following a jump or branch instruction
is executed \IT{before} the jump/branch instruction itself.

As a consequence, branch instructions always swap places with the instruction executed immediately beforehand.


In practice, functions which merely return 1 (\IT{true}) or 0 (\IT{false}) are very frequent.

The smallest ever of the standard UNIX utilities, \IT{/bin/true} and \IT{/bin/false} return 0 and 1 respectively, as an exit code.
(Zero as an exit code usually means success, non-zero means error.)
}
\RU{\subsubsection{std::string}
\myindex{\Cpp!STL!std::string}
\label{std_string}

\myparagraph{Как устроена структура}

Многие строковые библиотеки \InSqBrackets{\CNotes 2.2} обеспечивают структуру содержащую ссылку 
на буфер собственно со строкой, переменная всегда содержащую длину строки 
(что очень удобно для массы функций \InSqBrackets{\CNotes 2.2.1}) и переменную содержащую текущий размер буфера.

Строка в буфере обыкновенно оканчивается нулем: это для того чтобы указатель на буфер можно было
передавать в функции требующие на вход обычную сишную \ac{ASCIIZ}-строку.

Стандарт \Cpp не описывает, как именно нужно реализовывать std::string,
но, как правило, они реализованы как описано выше, с небольшими дополнениями.

Строки в \Cpp это не класс (как, например, QString в Qt), а темплейт (basic\_string), 
это сделано для того чтобы поддерживать 
строки содержащие разного типа символы: как минимум \Tchar и \IT{wchar\_t}.

Так что, std::string это класс с базовым типом \Tchar.

А std::wstring это класс с базовым типом \IT{wchar\_t}.

\mysubparagraph{MSVC}

В реализации MSVC, вместо ссылки на буфер может содержаться сам буфер (если строка короче 16-и символов).

Это означает, что каждая короткая строка будет занимать в памяти по крайней мере $16 + 4 + 4 = 24$ 
байт для 32-битной среды либо $16 + 8 + 8 = 32$ 
байта в 64-битной, а если строка длиннее 16-и символов, то прибавьте еще длину самой строки.

\lstinputlisting[caption=пример для MSVC,style=customc]{\CURPATH/STL/string/MSVC_RU.cpp}

Собственно, из этого исходника почти всё ясно.

Несколько замечаний:

Если строка короче 16-и символов, 
то отдельный буфер для строки в \glslink{heap}{куче} выделяться не будет.

Это удобно потому что на практике, основная часть строк действительно короткие.
Вероятно, разработчики в Microsoft выбрали размер в 16 символов как разумный баланс.

Теперь очень важный момент в конце функции main(): мы не пользуемся методом c\_str(), тем не менее,
если это скомпилировать и запустить, то обе строки появятся в консоли!

Работает это вот почему.

В первом случае строка короче 16-и символов и в начале объекта std::string (его можно рассматривать
просто как структуру) расположен буфер с этой строкой.
\printf трактует указатель как указатель на массив символов оканчивающийся нулем и поэтому всё работает.

Вывод второй строки (длиннее 16-и символов) даже еще опаснее: это вообще типичная программистская ошибка 
(или опечатка), забыть дописать c\_str().
Это работает потому что в это время в начале структуры расположен указатель на буфер.
Это может надолго остаться незамеченным: до тех пока там не появится строка 
короче 16-и символов, тогда процесс упадет.

\mysubparagraph{GCC}

В реализации GCC в структуре есть еще одна переменная --- reference count.

Интересно, что указатель на экземпляр класса std::string в GCC указывает не на начало самой структуры, 
а на указатель на буфера.
В libstdc++-v3\textbackslash{}include\textbackslash{}bits\textbackslash{}basic\_string.h 
мы можем прочитать что это сделано для удобства отладки:

\begin{lstlisting}
   *  The reason you want _M_data pointing to the character %array and
   *  not the _Rep is so that the debugger can see the string
   *  contents. (Probably we should add a non-inline member to get
   *  the _Rep for the debugger to use, so users can check the actual
   *  string length.)
\end{lstlisting}

\href{http://go.yurichev.com/17085}{исходный код basic\_string.h}

В нашем примере мы учитываем это:

\lstinputlisting[caption=пример для GCC,style=customc]{\CURPATH/STL/string/GCC_RU.cpp}

Нужны еще небольшие хаки чтобы сымитировать типичную ошибку, которую мы уже видели выше, из-за
более ужесточенной проверки типов в GCC, тем не менее, printf() работает и здесь без c\_str().

\myparagraph{Чуть более сложный пример}

\lstinputlisting[style=customc]{\CURPATH/STL/string/3.cpp}

\lstinputlisting[caption=MSVC 2012,style=customasmx86]{\CURPATH/STL/string/3_MSVC_RU.asm}

Собственно, компилятор не конструирует строки статически: да в общем-то и как
это возможно, если буфер с ней нужно хранить в \glslink{heap}{куче}?

Вместо этого в сегменте данных хранятся обычные \ac{ASCIIZ}-строки, а позже, во время выполнения, 
при помощи метода \q{assign}, конструируются строки s1 и s2
.
При помощи \TT{operator+}, создается строка s3.

Обратите внимание на то что вызов метода c\_str() отсутствует,
потому что его код достаточно короткий и компилятор вставил его прямо здесь:
если строка короче 16-и байт, то в регистре EAX остается указатель на буфер,
а если длиннее, то из этого же места достается адрес на буфер расположенный в \glslink{heap}{куче}.

Далее следуют вызовы трех деструкторов, причем, они вызываются только если строка длиннее 16-и байт:
тогда нужно освободить буфера в \glslink{heap}{куче}.
В противном случае, так как все три объекта std::string хранятся в стеке,
они освобождаются автоматически после выхода из функции.

Следовательно, работа с короткими строками более быстрая из-за м\'{е}ньшего обращения к \glslink{heap}{куче}.

Код на GCC даже проще (из-за того, что в GCC, как мы уже видели, не реализована возможность хранить короткую
строку прямо в структуре):

% TODO1 comment each function meaning
\lstinputlisting[caption=GCC 4.8.1,style=customasmx86]{\CURPATH/STL/string/3_GCC_RU.s}

Можно заметить, что в деструкторы передается не указатель на объект,
а указатель на место за 12 байт (или 3 слова) перед ним, то есть, на настоящее начало структуры.

\myparagraph{std::string как глобальная переменная}
\label{sec:std_string_as_global_variable}

Опытные программисты на \Cpp знают, что глобальные переменные \ac{STL}-типов вполне можно объявлять.

Да, действительно:

\lstinputlisting[style=customc]{\CURPATH/STL/string/5.cpp}

Но как и где будет вызываться конструктор \TT{std::string}?

На самом деле, эта переменная будет инициализирована даже перед началом \main.

\lstinputlisting[caption=MSVC 2012: здесь конструируется глобальная переменная{,} а также регистрируется её деструктор,style=customasmx86]{\CURPATH/STL/string/5_MSVC_p2.asm}

\lstinputlisting[caption=MSVC 2012: здесь глобальная переменная используется в \main,style=customasmx86]{\CURPATH/STL/string/5_MSVC_p1.asm}

\lstinputlisting[caption=MSVC 2012: эта функция-деструктор вызывается перед выходом,style=customasmx86]{\CURPATH/STL/string/5_MSVC_p3.asm}

\myindex{\CStandardLibrary!atexit()}
В реальности, из \ac{CRT}, еще до вызова main(), вызывается специальная функция,
в которой перечислены все конструкторы подобных переменных.
Более того: при помощи atexit() регистрируется функция, которая будет вызвана в конце работы программы:
в этой функции компилятор собирает вызовы деструкторов всех подобных глобальных переменных.

GCC работает похожим образом:

\lstinputlisting[caption=GCC 4.8.1,style=customasmx86]{\CURPATH/STL/string/5_GCC.s}

Но он не выделяет отдельной функции в которой будут собраны деструкторы: 
каждый деструктор передается в atexit() по одному.

% TODO а если глобальная STL-переменная в другом модуле? надо проверить.

}
\DE{\subsection{Einfachste XOR-Verschlüsselung überhaupt}

Ich habe einmal eine Software gesehen, bei der alle Debugging-Ausgaben mit XOR mit dem Wert 3
verschlüsselt wurden. Mit anderen Worten, die beiden niedrigsten Bits aller Buchstaben wurden invertiert.

``Hello, world'' wurde zu ``Kfool/\#tlqog'':

\begin{lstlisting}
#!/usr/bin/python

msg="Hello, world!"

print "".join(map(lambda x: chr(ord(x)^3), msg))
\end{lstlisting}

Das ist eine ziemlich interessante Verschlüsselung (oder besser eine Verschleierung),
weil sie zwei wichtige Eigenschaften hat:
1) es ist eine einzige Funktion zum Verschlüsseln und entschlüsseln, sie muss nur wiederholt angewendet werden
2) die entstehenden Buchstaben befinden sich im druckbaren Bereich, also die ganze Zeichenkette kann ohne
Escape-Symbole im Code verwendet werden.

Die zweite Eigenschaft nutzt die Tatsache, dass alle druckbaren Zeichen in Reihen organisiert sind: 0x2x-0x7x,
und wenn die beiden niederwertigsten Bits invertiert werden, wird der Buchstabe um eine oder drei Stellen nach
links oder rechts \IT{verschoben}, aber niemals in eine andere Reihe:

\begin{figure}[H]
\centering
\includegraphics[width=0.7\textwidth]{ascii_clean.png}
\caption{7-Bit \ac{ASCII} Tabelle in Emacs}
\end{figure}

\dots mit dem Zeichen 0x7F als einziger Ausnahme.

Im Folgenden werden also beispielsweise die Zeichen A-Z \IT{verschlüsselt}:

\begin{lstlisting}
#!/usr/bin/python

msg="@ABCDEFGHIJKLMNO"

print "".join(map(lambda x: chr(ord(x)^3), msg))
\end{lstlisting}

Ergebnis:
% FIXME \verb  --  relevant comment for German?
\begin{lstlisting}
CBA@GFEDKJIHONML
\end{lstlisting}

Es sieht so aus als würden die Zeichen ``@'' und ``C'' sowie ``B'' und ``A'' vertauscht werden.

Hier ist noch ein interessantes Beispiel, in dem gezeigt wird, wie die Eigenschaften von XOR
ausgenutzt werden können: Exakt den gleichen Effekt, dass druckbare Zeichen auch druckbar bleiben,
kann man dadurch erzielen, dass irgendeine Kombination der niedrigsten vier Bits invertiert wird.
}

\ifdefined\SPANISH
\chapter{Patrones de código}
\fi % SPANISH

\ifdefined\GERMAN
\chapter{Code-Muster}
\fi % GERMAN

\ifdefined\ENGLISH
\chapter{Code Patterns}
\fi % ENGLISH

\ifdefined\ITALIAN
\chapter{Forme di codice}
\fi % ITALIAN

\ifdefined\RUSSIAN
\chapter{Образцы кода}
\fi % RUSSIAN

\ifdefined\BRAZILIAN
\chapter{Padrões de códigos}
\fi % BRAZILIAN

\ifdefined\THAI
\chapter{รูปแบบของโค้ด}
\fi % THAI

\ifdefined\FRENCH
\chapter{Modèle de code}
\fi % FRENCH

\ifdefined\POLISH
\chapter{\PLph{}}
\fi % POLISH

% sections
\EN{\input{patterns/patterns_opt_dbg_EN}}
\ES{\input{patterns/patterns_opt_dbg_ES}}
\ITA{\input{patterns/patterns_opt_dbg_ITA}}
\PTBR{\input{patterns/patterns_opt_dbg_PTBR}}
\RU{\input{patterns/patterns_opt_dbg_RU}}
\THA{\input{patterns/patterns_opt_dbg_THA}}
\DE{\input{patterns/patterns_opt_dbg_DE}}
\FR{\input{patterns/patterns_opt_dbg_FR}}
\PL{\input{patterns/patterns_opt_dbg_PL}}

\RU{\section{Некоторые базовые понятия}}
\EN{\section{Some basics}}
\DE{\section{Einige Grundlagen}}
\FR{\section{Quelques bases}}
\ES{\section{\ESph{}}}
\ITA{\section{Alcune basi teoriche}}
\PTBR{\section{\PTBRph{}}}
\THA{\section{\THAph{}}}
\PL{\section{\PLph{}}}

% sections:
\EN{\input{patterns/intro_CPU_ISA_EN}}
\ES{\input{patterns/intro_CPU_ISA_ES}}
\ITA{\input{patterns/intro_CPU_ISA_ITA}}
\PTBR{\input{patterns/intro_CPU_ISA_PTBR}}
\RU{\input{patterns/intro_CPU_ISA_RU}}
\DE{\input{patterns/intro_CPU_ISA_DE}}
\FR{\input{patterns/intro_CPU_ISA_FR}}
\PL{\input{patterns/intro_CPU_ISA_PL}}

\EN{\input{patterns/numeral_EN}}
\RU{\input{patterns/numeral_RU}}
\ITA{\input{patterns/numeral_ITA}}
\DE{\input{patterns/numeral_DE}}
\FR{\input{patterns/numeral_FR}}
\PL{\input{patterns/numeral_PL}}

% chapters
\input{patterns/00_empty/main}
\input{patterns/011_ret/main}
\input{patterns/01_helloworld/main}
\input{patterns/015_prolog_epilogue/main}
\input{patterns/02_stack/main}
\input{patterns/03_printf/main}
\input{patterns/04_scanf/main}
\input{patterns/05_passing_arguments/main}
\input{patterns/06_return_results/main}
\input{patterns/061_pointers/main}
\input{patterns/065_GOTO/main}
\input{patterns/07_jcc/main}
\input{patterns/08_switch/main}
\input{patterns/09_loops/main}
\input{patterns/10_strings/main}
\input{patterns/11_arith_optimizations/main}
\input{patterns/12_FPU/main}
\input{patterns/13_arrays/main}
\input{patterns/14_bitfields/main}
\EN{\input{patterns/145_LCG/main_EN}}
\RU{\input{patterns/145_LCG/main_RU}}
\input{patterns/15_structs/main}
\input{patterns/17_unions/main}
\input{patterns/18_pointers_to_functions/main}
\input{patterns/185_64bit_in_32_env/main}

\EN{\input{patterns/19_SIMD/main_EN}}
\RU{\input{patterns/19_SIMD/main_RU}}
\DE{\input{patterns/19_SIMD/main_DE}}

\EN{\input{patterns/20_x64/main_EN}}
\RU{\input{patterns/20_x64/main_RU}}

\EN{\input{patterns/205_floating_SIMD/main_EN}}
\RU{\input{patterns/205_floating_SIMD/main_RU}}
\DE{\input{patterns/205_floating_SIMD/main_DE}}

\EN{\input{patterns/ARM/main_EN}}
\RU{\input{patterns/ARM/main_RU}}
\DE{\input{patterns/ARM/main_DE}}

\input{patterns/MIPS/main}


\ifdefined\SPANISH
\chapter{Patrones de código}
\fi % SPANISH

\ifdefined\GERMAN
\chapter{Code-Muster}
\fi % GERMAN

\ifdefined\ENGLISH
\chapter{Code Patterns}
\fi % ENGLISH

\ifdefined\ITALIAN
\chapter{Forme di codice}
\fi % ITALIAN

\ifdefined\RUSSIAN
\chapter{Образцы кода}
\fi % RUSSIAN

\ifdefined\BRAZILIAN
\chapter{Padrões de códigos}
\fi % BRAZILIAN

\ifdefined\THAI
\chapter{รูปแบบของโค้ด}
\fi % THAI

\ifdefined\FRENCH
\chapter{Modèle de code}
\fi % FRENCH

\ifdefined\POLISH
\chapter{\PLph{}}
\fi % POLISH

% sections
\EN{\section{The method}

When the author of this book first started learning C and, later, \Cpp, he used to write small pieces of code, compile them,
and then look at the assembly language output. This made it very easy for him to understand what was going on in the code that he had written.
\footnote{In fact, he still does this when he can't understand what a particular bit of code does.}.
He did this so many times that the relationship between the \CCpp code and what the compiler produced was imprinted deeply in his mind.
It's now easy for him to imagine instantly a rough outline of a C code's appearance and function.
Perhaps this technique could be helpful for others.

%There are a lot of examples for both x86/x64 and ARM.
%Those who already familiar with one of architectures, may freely skim over pages.

By the way, there is a great website where you can do the same, with various compilers, instead of installing them on your box.
You can use it as well: \url{https://gcc.godbolt.org/}.

\section*{\Exercises}

When the author of this book studied assembly language, he also often compiled small C functions and then rewrote
them gradually to assembly, trying to make their code as short as possible.
This probably is not worth doing in real-world scenarios today,
because it's hard to compete with the latest compilers in terms of efficiency. It is, however, a very good way to gain a better understanding of assembly.
Feel free, therefore, to take any assembly code from this book and try to make it shorter.
However, don't forget to test what you have written.

% rewrote to show that debug\release and optimisations levels are orthogonal concepts.
\section*{Optimization levels and debug information}

Source code can be compiled by different compilers with various optimization levels.
A typical compiler has about three such levels, where level zero means that optimization is completely disabled.
Optimization can also be targeted towards code size or code speed.
A non-optimizing compiler is faster and produces more understandable (albeit verbose) code,
whereas an optimizing compiler is slower and tries to produce code that runs faster (but is not necessarily more compact).
In addition to optimization levels, a compiler can include some debug information in the resulting file,
producing code that is easy to debug.
One of the important features of the ´debug' code is that it might contain links
between each line of the source code and its respective machine code address.
Optimizing compilers, on the other hand, tend to produce output where entire lines of source code
can be optimized away and thus not even be present in the resulting machine code.
Reverse engineers can encounter either version, simply because some developers turn on the compiler's optimization flags and others do not.
Because of this, we'll try to work on examples of both debug and release versions of the code featured in this book, wherever possible.

Sometimes some pretty ancient compilers are used in this book, in order to get the shortest (or simplest) possible code snippet.
}
\ES{\input{patterns/patterns_opt_dbg_ES}}
\ITA{\input{patterns/patterns_opt_dbg_ITA}}
\PTBR{\input{patterns/patterns_opt_dbg_PTBR}}
\RU{\input{patterns/patterns_opt_dbg_RU}}
\THA{\input{patterns/patterns_opt_dbg_THA}}
\DE{\input{patterns/patterns_opt_dbg_DE}}
\FR{\input{patterns/patterns_opt_dbg_FR}}
\PL{\input{patterns/patterns_opt_dbg_PL}}

\RU{\section{Некоторые базовые понятия}}
\EN{\section{Some basics}}
\DE{\section{Einige Grundlagen}}
\FR{\section{Quelques bases}}
\ES{\section{\ESph{}}}
\ITA{\section{Alcune basi teoriche}}
\PTBR{\section{\PTBRph{}}}
\THA{\section{\THAph{}}}
\PL{\section{\PLph{}}}

% sections:
\EN{\input{patterns/intro_CPU_ISA_EN}}
\ES{\input{patterns/intro_CPU_ISA_ES}}
\ITA{\input{patterns/intro_CPU_ISA_ITA}}
\PTBR{\input{patterns/intro_CPU_ISA_PTBR}}
\RU{\input{patterns/intro_CPU_ISA_RU}}
\DE{\input{patterns/intro_CPU_ISA_DE}}
\FR{\input{patterns/intro_CPU_ISA_FR}}
\PL{\input{patterns/intro_CPU_ISA_PL}}

\EN{\subsection{Numeral Systems}

Humans have become accustomed to a decimal numeral system, probably because almost everyone has 10 fingers.
Nevertheless, the number \q{10} has no significant meaning in science and mathematics.
The natural numeral system in digital electronics is binary: 0 is for an absence of current in the wire, and 1 for presence.
10 in binary is 2 in decimal, 100 in binary is 4 in decimal, and so on.

% This sentence is a bit unweildy - maybe try 'Our ten-digit system would be described as having a radix...' - Renaissance
If the numeral system has 10 digits, it has a \IT{radix} (or \IT{base}) of 10.
The binary numeral system has a \IT{radix} of 2.

Important things to recall:

1) A \IT{number} is a number, while a \IT{digit} is a term from writing systems, and is usually one character

% The original is 'number' is not changed; I think the intent is value, and changed it - Renaissance
2) The value of a number does not change when converted to another radix; only the writing notation for that value has changed (and therefore the way of representing it in \ac{RAM}).

\subsection{Converting From One Radix To Another}

Positional notation is used almost every numerical system. This means that a digit has weight relative to where it is placed inside of the larger number.
If 2 is placed at the rightmost place, it's 2, but if it's placed one digit before rightmost, it's 20.

What does $1234$ stand for?

$10^3 \cdot 1 + 10^2 \cdot 2 + 10^1 \cdot 3 + 1 \cdot 4 = 1234$ or
$1000 \cdot 1 + 100 \cdot 2 + 10 \cdot 3 + 4 = 1234$

It's the same story for binary numbers, but the base is 2 instead of 10.
What does 0b101011 stand for?

$2^5 \cdot 1 + 2^4 \cdot 0 + 2^3 \cdot 1 + 2^2 \cdot 0 + 2^1 \cdot 1 + 2^0 \cdot 1 = 43$ or
$32 \cdot 1 + 16 \cdot 0 + 8 \cdot 1 + 4 \cdot 0 + 2 \cdot 1 + 1 = 43$

There is such a thing as non-positional notation, such as the Roman numeral system.
\footnote{About numeric system evolution, see \InSqBrackets{\TAOCPvolII{}, 195--213.}}.
% Maybe add a sentence to fill in that X is always 10, and is therefore non-positional, even though putting an I before subtracts and after adds, and is in that sense positional
Perhaps, humankind switched to positional notation because it's easier to do basic operations (addition, multiplication, etc.) on paper by hand.

Binary numbers can be added, subtracted and so on in the very same as taught in schools, but only 2 digits are available.

Binary numbers are bulky when represented in source code and dumps, so that is where the hexadecimal numeral system can be useful.
A hexadecimal radix uses the digits 0..9, and also 6 Latin characters: A..F.
Each hexadecimal digit takes 4 bits or 4 binary digits, so it's very easy to convert from binary number to hexadecimal and back, even manually, in one's mind.

\begin{center}
\begin{longtable}{ | l | l | l | }
\hline
\HeaderColor hexadecimal & \HeaderColor binary & \HeaderColor decimal \\
\hline
0	&0000	&0 \\
1	&0001	&1 \\
2	&0010	&2 \\
3	&0011	&3 \\
4	&0100	&4 \\
5	&0101	&5 \\
6	&0110	&6 \\
7	&0111	&7 \\
8	&1000	&8 \\
9	&1001	&9 \\
A	&1010	&10 \\
B	&1011	&11 \\
C	&1100	&12 \\
D	&1101	&13 \\
E	&1110	&14 \\
F	&1111	&15 \\
\hline
\end{longtable}
\end{center}

How can one tell which radix is being used in a specific instance?

Decimal numbers are usually written as is, i.e., 1234. Some assemblers allow an identifier on decimal radix numbers, in which the number would be written with a "d" suffix: 1234d.

Binary numbers are sometimes prepended with the "0b" prefix: 0b100110111 (\ac{GCC} has a non-standard language extension for this\footnote{\url{https://gcc.gnu.org/onlinedocs/gcc/Binary-constants.html}}).
There is also another way: using a "b" suffix, for example: 100110111b.
This book tries to use the "0b" prefix consistently throughout the book for binary numbers.

Hexadecimal numbers are prepended with "0x" prefix in \CCpp and other \ac{PL}s: 0x1234ABCD.
Alternatively, they are given a "h" suffix: 1234ABCDh. This is common way of representing them in assemblers and debuggers.
In this convention, if the number is started with a Latin (A..F) digit, a 0 is added at the beginning: 0ABCDEFh.
There was also convention that was popular in 8-bit home computers era, using \$ prefix, like \$ABCD.
The book will try to stick to "0x" prefix throughout the book for hexadecimal numbers.

Should one learn to convert numbers mentally? A table of 1-digit hexadecimal numbers can easily be memorized.
As for larger numbers, it's probably not worth tormenting yourself.

Perhaps the most visible hexadecimal numbers are in \ac{URL}s.
This is the way that non-Latin characters are encoded.
For example:
\url{https://en.wiktionary.org/wiki/na\%C3\%AFvet\%C3\%A9} is the \ac{URL} of Wiktionary article about \q{naïveté} word.

\subsubsection{Octal Radix}

Another numeral system heavily used in the past of computer programming is octal. In octal there are 8 digits (0..7), and each is mapped to 3 bits, so it's easy to convert numbers back and forth.
It has been superseded by the hexadecimal system almost everywhere, but, surprisingly, there is a *NIX utility, used often by many people, which takes octal numbers as argument: \TT{chmod}.

\myindex{UNIX!chmod}
As many *NIX users know, \TT{chmod} argument can be a number of 3 digits. The first digit represents the rights of the owner of the file (read, write and/or execute), the second is the rights for the group to which the file belongs, and the third is for everyone else.
Each digit that \TT{chmod} takes can be represented in binary form:

\begin{center}
\begin{longtable}{ | l | l | l | }
\hline
\HeaderColor decimal & \HeaderColor binary & \HeaderColor meaning \\
\hline
7	&111	&\textbf{rwx} \\
6	&110	&\textbf{rw-} \\
5	&101	&\textbf{r-x} \\
4	&100	&\textbf{r-{}-} \\
3	&011	&\textbf{-wx} \\
2	&010	&\textbf{-w-} \\
1	&001	&\textbf{-{}-x} \\
0	&000	&\textbf{-{}-{}-} \\
\hline
\end{longtable}
\end{center}

So each bit is mapped to a flag: read/write/execute.

The importance of \TT{chmod} here is that the whole number in argument can be represented as octal number.
Let's take, for example, 644.
When you run \TT{chmod 644 file}, you set read/write permissions for owner, read permissions for group and again, read permissions for everyone else.
If we convert the octal number 644 to binary, it would be \TT{110100100}, or, in groups of 3 bits, \TT{110 100 100}.

Now we see that each triplet describe permissions for owner/group/others: first is \TT{rw-}, second is \TT{r--} and third is \TT{r--}.

The octal numeral system was also popular on old computers like PDP-8, because word there could be 12, 24 or 36 bits, and these numbers are all divisible by 3, so the octal system was natural in that environment.
Nowadays, all popular computers employ word/address sizes of 16, 32 or 64 bits, and these numbers are all divisible by 4, so the hexadecimal system is more natural there.

The octal numeral system is supported by all standard \CCpp compilers.
This is a source of confusion sometimes, because octal numbers are encoded with a zero prepended, for example, 0377 is 255.
Sometimes, you might make a typo and write "09" instead of 9, and the compiler would report an error.
GCC might report something like this:\\
\TT{error: invalid digit "9" in octal constant}.

Also, the octal system is somewhat popular in Java. When the IDA shows Java strings with non-printable characters,
they are encoded in the octal system instead of hexadecimal.
\myindex{JAD}
The JAD Java decompiler behaves the same way.

\subsubsection{Divisibility}

When you see a decimal number like 120, you can quickly deduce that it's divisible by 10, because the last digit is zero.
In the same way, 123400 is divisible by 100, because the two last digits are zeros.

Likewise, the hexadecimal number 0x1230 is divisible by 0x10 (or 16), 0x123000 is divisible by 0x1000 (or 4096), etc.

The binary number 0b1000101000 is divisible by 0b1000 (8), etc.

This property can often be used to quickly realize if the size of some block in memory is padded to some boundary.
For example, sections in \ac{PE} files are almost always started at addresses ending with 3 hexadecimal zeros: 0x41000, 0x10001000, etc.
The reason behind this is the fact that almost all \ac{PE} sections are padded to a boundary of 0x1000 (4096) bytes.

\subsubsection{Multi-Precision Arithmetic and Radix}

\index{RSA}
Multi-precision arithmetic can use huge numbers, and each one may be stored in several bytes.
For example, RSA keys, both public and private, span up to 4096 bits, and maybe even more.

% I'm not sure how to change this, but the normal format for quoting would be just to mention the author or book, and footnote to the full reference
In \InSqBrackets{\TAOCPvolII, 265} we find the following idea: when you store a multi-precision number in several bytes,
the whole number can be represented as having a radix of $2^8=256$, and each digit goes to the corresponding byte.
Likewise, if you store a multi-precision number in several 32-bit integer values, each digit goes to each 32-bit slot,
and you may think about this number as stored in radix of $2^{32}$.

\subsubsection{How to Pronounce Non-Decimal Numbers}

Numbers in a non-decimal base are usually pronounced by digit by digit: ``one-zero-zero-one-one-...''.
Words like ``ten'' and ``thousand'' are usually not pronounced, to prevent confusion with the decimal base system.

\subsubsection{Floating point numbers}

To distinguish floating point numbers from integers, they are usually written with ``.0'' at the end,
like $0.0$, $123.0$, etc.
}
\RU{\subsection{Представление чисел}

Люди привыкли к десятичной системе счисления вероятно потому что почти у каждого есть по 10 пальцев.
Тем не менее, число 10 не имеет особого значения в науке и математике.
Двоичная система естествена для цифровой электроники: 0 означает отсутствие тока в проводе и 1 --- его присутствие.
10 в двоичной системе это 2 в десятичной; 100 в двоичной это 4 в десятичной, итд.

Если в системе счисления есть 10 цифр, её \IT{основание} или \IT{radix} это 10.
Двоичная система имеет \IT{основание} 2.

Важные вещи, которые полезно вспомнить:
1) \IT{число} это число, в то время как \IT{цифра} это термин из системы письменности, и это обычно один символ;
2) само число не меняется, когда конвертируется из одного основания в другое: меняется способ его записи (или представления
в памяти).

Как сконвертировать число из одного основания в другое?

Позиционная нотация используется почти везде, это означает, что всякая цифра имеет свой вес, в зависимости от её расположения
внутри числа.
Если 2 расположена в самом последнем месте справа, это 2.
Если она расположена в месте перед последним, это 20.

Что означает $1234$?

$10^3 \cdot 1 + 10^2 \cdot 2 + 10^1 \cdot 3 + 1 \cdot 4$ = 1234 или
$1000 \cdot 1 + 100 \cdot 2 + 10 \cdot 3 + 4 = 1234$

Та же история и для двоичных чисел, только основание там 2 вместо 10.
Что означает 0b101011?

$2^5 \cdot 1 + 2^4 \cdot 0 + 2^3 \cdot 1 + 2^2 \cdot 0 + 2^1 \cdot 1 + 2^0 \cdot 1 = 43$ или
$32 \cdot 1 + 16 \cdot 0 + 8 \cdot 1 + 4 \cdot 0 + 2 \cdot 1 + 1 = 43$

Позиционную нотацию можно противопоставить непозиционной нотации, такой как римская система записи чисел
\footnote{Об эволюции способов записи чисел, см.также: \InSqBrackets{\TAOCPvolII{}, 195--213.}}.
Вероятно, человечество перешло на позиционную нотацию, потому что так проще работать с числами (сложение, умножение, итд)
на бумаге, в ручную.

Действительно, двоичные числа можно складывать, вычитать, итд, точно также, как этому обычно обучают в школах,
только доступны лишь 2 цифры.

Двоичные числа громоздки, когда их используют в исходных кодах и дампах, так что в этих случаях применяется шестнадцатеричная
система.
Используются цифры 0..9 и еще 6 латинских букв: A..F.
Каждая шестнадцатеричная цифра занимает 4 бита или 4 двоичных цифры, так что конвертировать из двоичной системы в
шестнадцатеричную и назад, можно легко вручную, или даже в уме.

\begin{center}
\begin{longtable}{ | l | l | l | }
\hline
\HeaderColor шестнадцатеричная & \HeaderColor двоичная & \HeaderColor десятичная \\
\hline
0	&0000	&0 \\
1	&0001	&1 \\
2	&0010	&2 \\
3	&0011	&3 \\
4	&0100	&4 \\
5	&0101	&5 \\
6	&0110	&6 \\
7	&0111	&7 \\
8	&1000	&8 \\
9	&1001	&9 \\
A	&1010	&10 \\
B	&1011	&11 \\
C	&1100	&12 \\
D	&1101	&13 \\
E	&1110	&14 \\
F	&1111	&15 \\
\hline
\end{longtable}
\end{center}

Как понять, какое основание используется в конкретном месте?

Десятичные числа обычно записываются как есть, т.е., 1234. Но некоторые ассемблеры позволяют подчеркивать
этот факт для ясности, и это число может быть дополнено суффиксом "d": 1234d.

К двоичным числам иногда спереди добавляют префикс "0b": 0b100110111
(В \ac{GCC} для этого есть нестандартное расширение языка
\footnote{\url{https://gcc.gnu.org/onlinedocs/gcc/Binary-constants.html}}).
Есть также еще один способ: суффикс "b", например: 100110111b.
В этой книге я буду пытаться придерживаться префикса "0b" для двоичных чисел.

Шестнадцатеричные числа имеют префикс "0x" в \CCpp и некоторых других \ac{PL}: 0x1234ABCD.
Либо они имеют суффикс "h": 1234ABCDh --- обычно так они представляются в ассемблерах и отладчиках.
Если число начинается с цифры A..F, перед ним добавляется 0: 0ABCDEFh.
Во времена 8-битных домашних компьютеров, был также способ записи чисел используя префикс \$, например, \$ABCD.
В книге я попытаюсь придерживаться префикса "0x" для шестнадцатеричных чисел.

Нужно ли учиться конвертировать числа в уме? Таблицу шестнадцатеричных чисел из одной цифры легко запомнить.
А запоминать б\'{о}льшие числа, наверное, не стоит.

Наверное, чаще всего шестнадцатеричные числа можно увидеть в \ac{URL}-ах.
Так кодируются буквы не из числа латинских.
Например:
\url{https://en.wiktionary.org/wiki/na\%C3\%AFvet\%C3\%A9} это \ac{URL} страницы в Wiktionary о слове \q{naïveté}.

\subsubsection{Восьмеричная система}

Еще одна система, которая в прошлом много использовалась в программировании это восьмеричная: есть 8 цифр (0..7) и каждая
описывает 3 бита, так что легко конвертировать числа туда и назад.
Она почти везде была заменена шестнадцатеричной, но удивительно, в *NIX имеется утилита использующаяся многими людьми,
которая принимает на вход восьмеричное число: \TT{chmod}.

\myindex{UNIX!chmod}
Как знают многие пользователи *NIX, аргумент \TT{chmod} это число из трех цифр. Первая цифра это права владельца файла,
вторая это права группы (которой файл принадлежит), третья для всех остальных.
И каждая цифра может быть представлена в двоичном виде:

\begin{center}
\begin{longtable}{ | l | l | l | }
\hline
\HeaderColor десятичная & \HeaderColor двоичная & \HeaderColor значение \\
\hline
7	&111	&\textbf{rwx} \\
6	&110	&\textbf{rw-} \\
5	&101	&\textbf{r-x} \\
4	&100	&\textbf{r-{}-} \\
3	&011	&\textbf{-wx} \\
2	&010	&\textbf{-w-} \\
1	&001	&\textbf{-{}-x} \\
0	&000	&\textbf{-{}-{}-} \\
\hline
\end{longtable}
\end{center}

Так что каждый бит привязан к флагу: read/write/execute (чтение/запись/исполнение).

И вот почему я вспомнил здесь о \TT{chmod}, это потому что всё число может быть представлено как число в восьмеричной системе.
Для примера возьмем 644.
Когда вы запускаете \TT{chmod 644 file}, вы выставляете права read/write для владельца, права read для группы, и снова,
read для всех остальных.
Сконвертируем число 644 из восьмеричной системы в двоичную, это будет \TT{110100100}, или (в группах по 3 бита) \TT{110 100 100}.

Теперь мы видим, что каждая тройка описывает права для владельца/группы/остальных:
первая это \TT{rw-}, вторая это \TT{r--} и третья это \TT{r--}.

Восьмеричная система была также популярная на старых компьютерах вроде PDP-8, потому что слово там могло содержать 12, 24 или
36 бит, и эти числа делятся на 3, так что выбор восьмеричной системы в той среде был логичен.
Сейчас, все популярные компьютеры имеют размер слова/адреса 16, 32 или 64 бита, и эти числа делятся на 4,
так что шестнадцатеричная система здесь удобнее.

Восьмеричная система поддерживается всеми стандартными компиляторами \CCpp{}.
Это иногда источник недоумения, потому что восьмеричные числа кодируются с нулем вперед, например, 0377 это 255.
И иногда, вы можете сделать опечатку, и написать "09" вместо 9, и компилятор выдаст ошибку.
GCC может выдать что-то вроде:\\
\TT{error: invalid digit "9" in octal constant}.

Также, восьмеричная система популярна в Java: когда IDA показывает строку с непечатаемыми символами,
они кодируются в восьмеричной системе вместо шестнадцатеричной.
\myindex{JAD}
Точно также себя ведет декомпилятор с Java JAD.

\subsubsection{Делимость}

Когда вы видите десятичное число вроде 120, вы можете быстро понять что оно делится на 10, потому что последняя цифра это 0.
Точно также, 123400 делится на 100, потому что две последних цифры это нули.

Точно также, шестнадцатеричное число 0x1230 делится на 0x10 (или 16), 0x123000 делится на 0x1000 (или 4096), итд.

Двоичное число 0b1000101000 делится на 0b1000 (8), итд.

Это свойство можно часто использовать, чтобы быстро понять,
что длина какого-либо блока в памяти выровнена по некоторой границе.
Например, секции в \ac{PE}-файлах почти всегда начинаются с адресов заканчивающихся 3 шестнадцатеричными нулями:
0x41000, 0x10001000, итд.
Причина в том, что почти все секции в \ac{PE} выровнены по границе 0x1000 (4096) байт.

\subsubsection{Арифметика произвольной точности и основание}

\index{RSA}
Арифметика произвольной точности (multi-precision arithmetic) может использовать огромные числа,
которые могут храниться в нескольких байтах.
Например, ключи RSA, и открытые и закрытые, могут занимать до 4096 бит и даже больше.

В \InSqBrackets{\TAOCPvolII, 265} можно найти такую идею: когда вы сохраняете число произвольной точности в нескольких байтах,
всё число может быть представлено как имеющую систему счисления по основанию $2^8=256$, и каждая цифра находится
в соответствующем байте.
Точно также, если вы сохраняете число произвольной точности в нескольких 32-битных целочисленных значениях,
каждая цифра отправляется в каждый 32-битный слот, и вы можете считать что это число записано в системе с основанием $2^{32}$.

\subsubsection{Произношение}

Числа в недесятичных системах счислениях обычно произносятся по одной цифре: ``один-ноль-ноль-один-один-...''.
Слова вроде ``десять'', ``тысяча'', итд, обычно не произносятся, потому что тогда можно спутать с десятичной системой.

\subsubsection{Числа с плавающей запятой}

Чтобы отличать числа с плавающей запятой от целочисленных, часто, в конце добавляют ``.0'',
например $0.0$, $123.0$, итд.

}
\ITA{\input{patterns/numeral_ITA}}
\DE{\input{patterns/numeral_DE}}
\FR{\input{patterns/numeral_FR}}
\PL{\input{patterns/numeral_PL}}

% chapters
\ifdefined\SPANISH
\chapter{Patrones de código}
\fi % SPANISH

\ifdefined\GERMAN
\chapter{Code-Muster}
\fi % GERMAN

\ifdefined\ENGLISH
\chapter{Code Patterns}
\fi % ENGLISH

\ifdefined\ITALIAN
\chapter{Forme di codice}
\fi % ITALIAN

\ifdefined\RUSSIAN
\chapter{Образцы кода}
\fi % RUSSIAN

\ifdefined\BRAZILIAN
\chapter{Padrões de códigos}
\fi % BRAZILIAN

\ifdefined\THAI
\chapter{รูปแบบของโค้ด}
\fi % THAI

\ifdefined\FRENCH
\chapter{Modèle de code}
\fi % FRENCH

\ifdefined\POLISH
\chapter{\PLph{}}
\fi % POLISH

% sections
\EN{\input{patterns/patterns_opt_dbg_EN}}
\ES{\input{patterns/patterns_opt_dbg_ES}}
\ITA{\input{patterns/patterns_opt_dbg_ITA}}
\PTBR{\input{patterns/patterns_opt_dbg_PTBR}}
\RU{\input{patterns/patterns_opt_dbg_RU}}
\THA{\input{patterns/patterns_opt_dbg_THA}}
\DE{\input{patterns/patterns_opt_dbg_DE}}
\FR{\input{patterns/patterns_opt_dbg_FR}}
\PL{\input{patterns/patterns_opt_dbg_PL}}

\RU{\section{Некоторые базовые понятия}}
\EN{\section{Some basics}}
\DE{\section{Einige Grundlagen}}
\FR{\section{Quelques bases}}
\ES{\section{\ESph{}}}
\ITA{\section{Alcune basi teoriche}}
\PTBR{\section{\PTBRph{}}}
\THA{\section{\THAph{}}}
\PL{\section{\PLph{}}}

% sections:
\EN{\input{patterns/intro_CPU_ISA_EN}}
\ES{\input{patterns/intro_CPU_ISA_ES}}
\ITA{\input{patterns/intro_CPU_ISA_ITA}}
\PTBR{\input{patterns/intro_CPU_ISA_PTBR}}
\RU{\input{patterns/intro_CPU_ISA_RU}}
\DE{\input{patterns/intro_CPU_ISA_DE}}
\FR{\input{patterns/intro_CPU_ISA_FR}}
\PL{\input{patterns/intro_CPU_ISA_PL}}

\EN{\input{patterns/numeral_EN}}
\RU{\input{patterns/numeral_RU}}
\ITA{\input{patterns/numeral_ITA}}
\DE{\input{patterns/numeral_DE}}
\FR{\input{patterns/numeral_FR}}
\PL{\input{patterns/numeral_PL}}

% chapters
\input{patterns/00_empty/main}
\input{patterns/011_ret/main}
\input{patterns/01_helloworld/main}
\input{patterns/015_prolog_epilogue/main}
\input{patterns/02_stack/main}
\input{patterns/03_printf/main}
\input{patterns/04_scanf/main}
\input{patterns/05_passing_arguments/main}
\input{patterns/06_return_results/main}
\input{patterns/061_pointers/main}
\input{patterns/065_GOTO/main}
\input{patterns/07_jcc/main}
\input{patterns/08_switch/main}
\input{patterns/09_loops/main}
\input{patterns/10_strings/main}
\input{patterns/11_arith_optimizations/main}
\input{patterns/12_FPU/main}
\input{patterns/13_arrays/main}
\input{patterns/14_bitfields/main}
\EN{\input{patterns/145_LCG/main_EN}}
\RU{\input{patterns/145_LCG/main_RU}}
\input{patterns/15_structs/main}
\input{patterns/17_unions/main}
\input{patterns/18_pointers_to_functions/main}
\input{patterns/185_64bit_in_32_env/main}

\EN{\input{patterns/19_SIMD/main_EN}}
\RU{\input{patterns/19_SIMD/main_RU}}
\DE{\input{patterns/19_SIMD/main_DE}}

\EN{\input{patterns/20_x64/main_EN}}
\RU{\input{patterns/20_x64/main_RU}}

\EN{\input{patterns/205_floating_SIMD/main_EN}}
\RU{\input{patterns/205_floating_SIMD/main_RU}}
\DE{\input{patterns/205_floating_SIMD/main_DE}}

\EN{\input{patterns/ARM/main_EN}}
\RU{\input{patterns/ARM/main_RU}}
\DE{\input{patterns/ARM/main_DE}}

\input{patterns/MIPS/main}

\ifdefined\SPANISH
\chapter{Patrones de código}
\fi % SPANISH

\ifdefined\GERMAN
\chapter{Code-Muster}
\fi % GERMAN

\ifdefined\ENGLISH
\chapter{Code Patterns}
\fi % ENGLISH

\ifdefined\ITALIAN
\chapter{Forme di codice}
\fi % ITALIAN

\ifdefined\RUSSIAN
\chapter{Образцы кода}
\fi % RUSSIAN

\ifdefined\BRAZILIAN
\chapter{Padrões de códigos}
\fi % BRAZILIAN

\ifdefined\THAI
\chapter{รูปแบบของโค้ด}
\fi % THAI

\ifdefined\FRENCH
\chapter{Modèle de code}
\fi % FRENCH

\ifdefined\POLISH
\chapter{\PLph{}}
\fi % POLISH

% sections
\EN{\input{patterns/patterns_opt_dbg_EN}}
\ES{\input{patterns/patterns_opt_dbg_ES}}
\ITA{\input{patterns/patterns_opt_dbg_ITA}}
\PTBR{\input{patterns/patterns_opt_dbg_PTBR}}
\RU{\input{patterns/patterns_opt_dbg_RU}}
\THA{\input{patterns/patterns_opt_dbg_THA}}
\DE{\input{patterns/patterns_opt_dbg_DE}}
\FR{\input{patterns/patterns_opt_dbg_FR}}
\PL{\input{patterns/patterns_opt_dbg_PL}}

\RU{\section{Некоторые базовые понятия}}
\EN{\section{Some basics}}
\DE{\section{Einige Grundlagen}}
\FR{\section{Quelques bases}}
\ES{\section{\ESph{}}}
\ITA{\section{Alcune basi teoriche}}
\PTBR{\section{\PTBRph{}}}
\THA{\section{\THAph{}}}
\PL{\section{\PLph{}}}

% sections:
\EN{\input{patterns/intro_CPU_ISA_EN}}
\ES{\input{patterns/intro_CPU_ISA_ES}}
\ITA{\input{patterns/intro_CPU_ISA_ITA}}
\PTBR{\input{patterns/intro_CPU_ISA_PTBR}}
\RU{\input{patterns/intro_CPU_ISA_RU}}
\DE{\input{patterns/intro_CPU_ISA_DE}}
\FR{\input{patterns/intro_CPU_ISA_FR}}
\PL{\input{patterns/intro_CPU_ISA_PL}}

\EN{\input{patterns/numeral_EN}}
\RU{\input{patterns/numeral_RU}}
\ITA{\input{patterns/numeral_ITA}}
\DE{\input{patterns/numeral_DE}}
\FR{\input{patterns/numeral_FR}}
\PL{\input{patterns/numeral_PL}}

% chapters
\input{patterns/00_empty/main}
\input{patterns/011_ret/main}
\input{patterns/01_helloworld/main}
\input{patterns/015_prolog_epilogue/main}
\input{patterns/02_stack/main}
\input{patterns/03_printf/main}
\input{patterns/04_scanf/main}
\input{patterns/05_passing_arguments/main}
\input{patterns/06_return_results/main}
\input{patterns/061_pointers/main}
\input{patterns/065_GOTO/main}
\input{patterns/07_jcc/main}
\input{patterns/08_switch/main}
\input{patterns/09_loops/main}
\input{patterns/10_strings/main}
\input{patterns/11_arith_optimizations/main}
\input{patterns/12_FPU/main}
\input{patterns/13_arrays/main}
\input{patterns/14_bitfields/main}
\EN{\input{patterns/145_LCG/main_EN}}
\RU{\input{patterns/145_LCG/main_RU}}
\input{patterns/15_structs/main}
\input{patterns/17_unions/main}
\input{patterns/18_pointers_to_functions/main}
\input{patterns/185_64bit_in_32_env/main}

\EN{\input{patterns/19_SIMD/main_EN}}
\RU{\input{patterns/19_SIMD/main_RU}}
\DE{\input{patterns/19_SIMD/main_DE}}

\EN{\input{patterns/20_x64/main_EN}}
\RU{\input{patterns/20_x64/main_RU}}

\EN{\input{patterns/205_floating_SIMD/main_EN}}
\RU{\input{patterns/205_floating_SIMD/main_RU}}
\DE{\input{patterns/205_floating_SIMD/main_DE}}

\EN{\input{patterns/ARM/main_EN}}
\RU{\input{patterns/ARM/main_RU}}
\DE{\input{patterns/ARM/main_DE}}

\input{patterns/MIPS/main}

\ifdefined\SPANISH
\chapter{Patrones de código}
\fi % SPANISH

\ifdefined\GERMAN
\chapter{Code-Muster}
\fi % GERMAN

\ifdefined\ENGLISH
\chapter{Code Patterns}
\fi % ENGLISH

\ifdefined\ITALIAN
\chapter{Forme di codice}
\fi % ITALIAN

\ifdefined\RUSSIAN
\chapter{Образцы кода}
\fi % RUSSIAN

\ifdefined\BRAZILIAN
\chapter{Padrões de códigos}
\fi % BRAZILIAN

\ifdefined\THAI
\chapter{รูปแบบของโค้ด}
\fi % THAI

\ifdefined\FRENCH
\chapter{Modèle de code}
\fi % FRENCH

\ifdefined\POLISH
\chapter{\PLph{}}
\fi % POLISH

% sections
\EN{\input{patterns/patterns_opt_dbg_EN}}
\ES{\input{patterns/patterns_opt_dbg_ES}}
\ITA{\input{patterns/patterns_opt_dbg_ITA}}
\PTBR{\input{patterns/patterns_opt_dbg_PTBR}}
\RU{\input{patterns/patterns_opt_dbg_RU}}
\THA{\input{patterns/patterns_opt_dbg_THA}}
\DE{\input{patterns/patterns_opt_dbg_DE}}
\FR{\input{patterns/patterns_opt_dbg_FR}}
\PL{\input{patterns/patterns_opt_dbg_PL}}

\RU{\section{Некоторые базовые понятия}}
\EN{\section{Some basics}}
\DE{\section{Einige Grundlagen}}
\FR{\section{Quelques bases}}
\ES{\section{\ESph{}}}
\ITA{\section{Alcune basi teoriche}}
\PTBR{\section{\PTBRph{}}}
\THA{\section{\THAph{}}}
\PL{\section{\PLph{}}}

% sections:
\EN{\input{patterns/intro_CPU_ISA_EN}}
\ES{\input{patterns/intro_CPU_ISA_ES}}
\ITA{\input{patterns/intro_CPU_ISA_ITA}}
\PTBR{\input{patterns/intro_CPU_ISA_PTBR}}
\RU{\input{patterns/intro_CPU_ISA_RU}}
\DE{\input{patterns/intro_CPU_ISA_DE}}
\FR{\input{patterns/intro_CPU_ISA_FR}}
\PL{\input{patterns/intro_CPU_ISA_PL}}

\EN{\input{patterns/numeral_EN}}
\RU{\input{patterns/numeral_RU}}
\ITA{\input{patterns/numeral_ITA}}
\DE{\input{patterns/numeral_DE}}
\FR{\input{patterns/numeral_FR}}
\PL{\input{patterns/numeral_PL}}

% chapters
\input{patterns/00_empty/main}
\input{patterns/011_ret/main}
\input{patterns/01_helloworld/main}
\input{patterns/015_prolog_epilogue/main}
\input{patterns/02_stack/main}
\input{patterns/03_printf/main}
\input{patterns/04_scanf/main}
\input{patterns/05_passing_arguments/main}
\input{patterns/06_return_results/main}
\input{patterns/061_pointers/main}
\input{patterns/065_GOTO/main}
\input{patterns/07_jcc/main}
\input{patterns/08_switch/main}
\input{patterns/09_loops/main}
\input{patterns/10_strings/main}
\input{patterns/11_arith_optimizations/main}
\input{patterns/12_FPU/main}
\input{patterns/13_arrays/main}
\input{patterns/14_bitfields/main}
\EN{\input{patterns/145_LCG/main_EN}}
\RU{\input{patterns/145_LCG/main_RU}}
\input{patterns/15_structs/main}
\input{patterns/17_unions/main}
\input{patterns/18_pointers_to_functions/main}
\input{patterns/185_64bit_in_32_env/main}

\EN{\input{patterns/19_SIMD/main_EN}}
\RU{\input{patterns/19_SIMD/main_RU}}
\DE{\input{patterns/19_SIMD/main_DE}}

\EN{\input{patterns/20_x64/main_EN}}
\RU{\input{patterns/20_x64/main_RU}}

\EN{\input{patterns/205_floating_SIMD/main_EN}}
\RU{\input{patterns/205_floating_SIMD/main_RU}}
\DE{\input{patterns/205_floating_SIMD/main_DE}}

\EN{\input{patterns/ARM/main_EN}}
\RU{\input{patterns/ARM/main_RU}}
\DE{\input{patterns/ARM/main_DE}}

\input{patterns/MIPS/main}

\ifdefined\SPANISH
\chapter{Patrones de código}
\fi % SPANISH

\ifdefined\GERMAN
\chapter{Code-Muster}
\fi % GERMAN

\ifdefined\ENGLISH
\chapter{Code Patterns}
\fi % ENGLISH

\ifdefined\ITALIAN
\chapter{Forme di codice}
\fi % ITALIAN

\ifdefined\RUSSIAN
\chapter{Образцы кода}
\fi % RUSSIAN

\ifdefined\BRAZILIAN
\chapter{Padrões de códigos}
\fi % BRAZILIAN

\ifdefined\THAI
\chapter{รูปแบบของโค้ด}
\fi % THAI

\ifdefined\FRENCH
\chapter{Modèle de code}
\fi % FRENCH

\ifdefined\POLISH
\chapter{\PLph{}}
\fi % POLISH

% sections
\EN{\input{patterns/patterns_opt_dbg_EN}}
\ES{\input{patterns/patterns_opt_dbg_ES}}
\ITA{\input{patterns/patterns_opt_dbg_ITA}}
\PTBR{\input{patterns/patterns_opt_dbg_PTBR}}
\RU{\input{patterns/patterns_opt_dbg_RU}}
\THA{\input{patterns/patterns_opt_dbg_THA}}
\DE{\input{patterns/patterns_opt_dbg_DE}}
\FR{\input{patterns/patterns_opt_dbg_FR}}
\PL{\input{patterns/patterns_opt_dbg_PL}}

\RU{\section{Некоторые базовые понятия}}
\EN{\section{Some basics}}
\DE{\section{Einige Grundlagen}}
\FR{\section{Quelques bases}}
\ES{\section{\ESph{}}}
\ITA{\section{Alcune basi teoriche}}
\PTBR{\section{\PTBRph{}}}
\THA{\section{\THAph{}}}
\PL{\section{\PLph{}}}

% sections:
\EN{\input{patterns/intro_CPU_ISA_EN}}
\ES{\input{patterns/intro_CPU_ISA_ES}}
\ITA{\input{patterns/intro_CPU_ISA_ITA}}
\PTBR{\input{patterns/intro_CPU_ISA_PTBR}}
\RU{\input{patterns/intro_CPU_ISA_RU}}
\DE{\input{patterns/intro_CPU_ISA_DE}}
\FR{\input{patterns/intro_CPU_ISA_FR}}
\PL{\input{patterns/intro_CPU_ISA_PL}}

\EN{\input{patterns/numeral_EN}}
\RU{\input{patterns/numeral_RU}}
\ITA{\input{patterns/numeral_ITA}}
\DE{\input{patterns/numeral_DE}}
\FR{\input{patterns/numeral_FR}}
\PL{\input{patterns/numeral_PL}}

% chapters
\input{patterns/00_empty/main}
\input{patterns/011_ret/main}
\input{patterns/01_helloworld/main}
\input{patterns/015_prolog_epilogue/main}
\input{patterns/02_stack/main}
\input{patterns/03_printf/main}
\input{patterns/04_scanf/main}
\input{patterns/05_passing_arguments/main}
\input{patterns/06_return_results/main}
\input{patterns/061_pointers/main}
\input{patterns/065_GOTO/main}
\input{patterns/07_jcc/main}
\input{patterns/08_switch/main}
\input{patterns/09_loops/main}
\input{patterns/10_strings/main}
\input{patterns/11_arith_optimizations/main}
\input{patterns/12_FPU/main}
\input{patterns/13_arrays/main}
\input{patterns/14_bitfields/main}
\EN{\input{patterns/145_LCG/main_EN}}
\RU{\input{patterns/145_LCG/main_RU}}
\input{patterns/15_structs/main}
\input{patterns/17_unions/main}
\input{patterns/18_pointers_to_functions/main}
\input{patterns/185_64bit_in_32_env/main}

\EN{\input{patterns/19_SIMD/main_EN}}
\RU{\input{patterns/19_SIMD/main_RU}}
\DE{\input{patterns/19_SIMD/main_DE}}

\EN{\input{patterns/20_x64/main_EN}}
\RU{\input{patterns/20_x64/main_RU}}

\EN{\input{patterns/205_floating_SIMD/main_EN}}
\RU{\input{patterns/205_floating_SIMD/main_RU}}
\DE{\input{patterns/205_floating_SIMD/main_DE}}

\EN{\input{patterns/ARM/main_EN}}
\RU{\input{patterns/ARM/main_RU}}
\DE{\input{patterns/ARM/main_DE}}

\input{patterns/MIPS/main}

\ifdefined\SPANISH
\chapter{Patrones de código}
\fi % SPANISH

\ifdefined\GERMAN
\chapter{Code-Muster}
\fi % GERMAN

\ifdefined\ENGLISH
\chapter{Code Patterns}
\fi % ENGLISH

\ifdefined\ITALIAN
\chapter{Forme di codice}
\fi % ITALIAN

\ifdefined\RUSSIAN
\chapter{Образцы кода}
\fi % RUSSIAN

\ifdefined\BRAZILIAN
\chapter{Padrões de códigos}
\fi % BRAZILIAN

\ifdefined\THAI
\chapter{รูปแบบของโค้ด}
\fi % THAI

\ifdefined\FRENCH
\chapter{Modèle de code}
\fi % FRENCH

\ifdefined\POLISH
\chapter{\PLph{}}
\fi % POLISH

% sections
\EN{\input{patterns/patterns_opt_dbg_EN}}
\ES{\input{patterns/patterns_opt_dbg_ES}}
\ITA{\input{patterns/patterns_opt_dbg_ITA}}
\PTBR{\input{patterns/patterns_opt_dbg_PTBR}}
\RU{\input{patterns/patterns_opt_dbg_RU}}
\THA{\input{patterns/patterns_opt_dbg_THA}}
\DE{\input{patterns/patterns_opt_dbg_DE}}
\FR{\input{patterns/patterns_opt_dbg_FR}}
\PL{\input{patterns/patterns_opt_dbg_PL}}

\RU{\section{Некоторые базовые понятия}}
\EN{\section{Some basics}}
\DE{\section{Einige Grundlagen}}
\FR{\section{Quelques bases}}
\ES{\section{\ESph{}}}
\ITA{\section{Alcune basi teoriche}}
\PTBR{\section{\PTBRph{}}}
\THA{\section{\THAph{}}}
\PL{\section{\PLph{}}}

% sections:
\EN{\input{patterns/intro_CPU_ISA_EN}}
\ES{\input{patterns/intro_CPU_ISA_ES}}
\ITA{\input{patterns/intro_CPU_ISA_ITA}}
\PTBR{\input{patterns/intro_CPU_ISA_PTBR}}
\RU{\input{patterns/intro_CPU_ISA_RU}}
\DE{\input{patterns/intro_CPU_ISA_DE}}
\FR{\input{patterns/intro_CPU_ISA_FR}}
\PL{\input{patterns/intro_CPU_ISA_PL}}

\EN{\input{patterns/numeral_EN}}
\RU{\input{patterns/numeral_RU}}
\ITA{\input{patterns/numeral_ITA}}
\DE{\input{patterns/numeral_DE}}
\FR{\input{patterns/numeral_FR}}
\PL{\input{patterns/numeral_PL}}

% chapters
\input{patterns/00_empty/main}
\input{patterns/011_ret/main}
\input{patterns/01_helloworld/main}
\input{patterns/015_prolog_epilogue/main}
\input{patterns/02_stack/main}
\input{patterns/03_printf/main}
\input{patterns/04_scanf/main}
\input{patterns/05_passing_arguments/main}
\input{patterns/06_return_results/main}
\input{patterns/061_pointers/main}
\input{patterns/065_GOTO/main}
\input{patterns/07_jcc/main}
\input{patterns/08_switch/main}
\input{patterns/09_loops/main}
\input{patterns/10_strings/main}
\input{patterns/11_arith_optimizations/main}
\input{patterns/12_FPU/main}
\input{patterns/13_arrays/main}
\input{patterns/14_bitfields/main}
\EN{\input{patterns/145_LCG/main_EN}}
\RU{\input{patterns/145_LCG/main_RU}}
\input{patterns/15_structs/main}
\input{patterns/17_unions/main}
\input{patterns/18_pointers_to_functions/main}
\input{patterns/185_64bit_in_32_env/main}

\EN{\input{patterns/19_SIMD/main_EN}}
\RU{\input{patterns/19_SIMD/main_RU}}
\DE{\input{patterns/19_SIMD/main_DE}}

\EN{\input{patterns/20_x64/main_EN}}
\RU{\input{patterns/20_x64/main_RU}}

\EN{\input{patterns/205_floating_SIMD/main_EN}}
\RU{\input{patterns/205_floating_SIMD/main_RU}}
\DE{\input{patterns/205_floating_SIMD/main_DE}}

\EN{\input{patterns/ARM/main_EN}}
\RU{\input{patterns/ARM/main_RU}}
\DE{\input{patterns/ARM/main_DE}}

\input{patterns/MIPS/main}

\ifdefined\SPANISH
\chapter{Patrones de código}
\fi % SPANISH

\ifdefined\GERMAN
\chapter{Code-Muster}
\fi % GERMAN

\ifdefined\ENGLISH
\chapter{Code Patterns}
\fi % ENGLISH

\ifdefined\ITALIAN
\chapter{Forme di codice}
\fi % ITALIAN

\ifdefined\RUSSIAN
\chapter{Образцы кода}
\fi % RUSSIAN

\ifdefined\BRAZILIAN
\chapter{Padrões de códigos}
\fi % BRAZILIAN

\ifdefined\THAI
\chapter{รูปแบบของโค้ด}
\fi % THAI

\ifdefined\FRENCH
\chapter{Modèle de code}
\fi % FRENCH

\ifdefined\POLISH
\chapter{\PLph{}}
\fi % POLISH

% sections
\EN{\input{patterns/patterns_opt_dbg_EN}}
\ES{\input{patterns/patterns_opt_dbg_ES}}
\ITA{\input{patterns/patterns_opt_dbg_ITA}}
\PTBR{\input{patterns/patterns_opt_dbg_PTBR}}
\RU{\input{patterns/patterns_opt_dbg_RU}}
\THA{\input{patterns/patterns_opt_dbg_THA}}
\DE{\input{patterns/patterns_opt_dbg_DE}}
\FR{\input{patterns/patterns_opt_dbg_FR}}
\PL{\input{patterns/patterns_opt_dbg_PL}}

\RU{\section{Некоторые базовые понятия}}
\EN{\section{Some basics}}
\DE{\section{Einige Grundlagen}}
\FR{\section{Quelques bases}}
\ES{\section{\ESph{}}}
\ITA{\section{Alcune basi teoriche}}
\PTBR{\section{\PTBRph{}}}
\THA{\section{\THAph{}}}
\PL{\section{\PLph{}}}

% sections:
\EN{\input{patterns/intro_CPU_ISA_EN}}
\ES{\input{patterns/intro_CPU_ISA_ES}}
\ITA{\input{patterns/intro_CPU_ISA_ITA}}
\PTBR{\input{patterns/intro_CPU_ISA_PTBR}}
\RU{\input{patterns/intro_CPU_ISA_RU}}
\DE{\input{patterns/intro_CPU_ISA_DE}}
\FR{\input{patterns/intro_CPU_ISA_FR}}
\PL{\input{patterns/intro_CPU_ISA_PL}}

\EN{\input{patterns/numeral_EN}}
\RU{\input{patterns/numeral_RU}}
\ITA{\input{patterns/numeral_ITA}}
\DE{\input{patterns/numeral_DE}}
\FR{\input{patterns/numeral_FR}}
\PL{\input{patterns/numeral_PL}}

% chapters
\input{patterns/00_empty/main}
\input{patterns/011_ret/main}
\input{patterns/01_helloworld/main}
\input{patterns/015_prolog_epilogue/main}
\input{patterns/02_stack/main}
\input{patterns/03_printf/main}
\input{patterns/04_scanf/main}
\input{patterns/05_passing_arguments/main}
\input{patterns/06_return_results/main}
\input{patterns/061_pointers/main}
\input{patterns/065_GOTO/main}
\input{patterns/07_jcc/main}
\input{patterns/08_switch/main}
\input{patterns/09_loops/main}
\input{patterns/10_strings/main}
\input{patterns/11_arith_optimizations/main}
\input{patterns/12_FPU/main}
\input{patterns/13_arrays/main}
\input{patterns/14_bitfields/main}
\EN{\input{patterns/145_LCG/main_EN}}
\RU{\input{patterns/145_LCG/main_RU}}
\input{patterns/15_structs/main}
\input{patterns/17_unions/main}
\input{patterns/18_pointers_to_functions/main}
\input{patterns/185_64bit_in_32_env/main}

\EN{\input{patterns/19_SIMD/main_EN}}
\RU{\input{patterns/19_SIMD/main_RU}}
\DE{\input{patterns/19_SIMD/main_DE}}

\EN{\input{patterns/20_x64/main_EN}}
\RU{\input{patterns/20_x64/main_RU}}

\EN{\input{patterns/205_floating_SIMD/main_EN}}
\RU{\input{patterns/205_floating_SIMD/main_RU}}
\DE{\input{patterns/205_floating_SIMD/main_DE}}

\EN{\input{patterns/ARM/main_EN}}
\RU{\input{patterns/ARM/main_RU}}
\DE{\input{patterns/ARM/main_DE}}

\input{patterns/MIPS/main}

\ifdefined\SPANISH
\chapter{Patrones de código}
\fi % SPANISH

\ifdefined\GERMAN
\chapter{Code-Muster}
\fi % GERMAN

\ifdefined\ENGLISH
\chapter{Code Patterns}
\fi % ENGLISH

\ifdefined\ITALIAN
\chapter{Forme di codice}
\fi % ITALIAN

\ifdefined\RUSSIAN
\chapter{Образцы кода}
\fi % RUSSIAN

\ifdefined\BRAZILIAN
\chapter{Padrões de códigos}
\fi % BRAZILIAN

\ifdefined\THAI
\chapter{รูปแบบของโค้ด}
\fi % THAI

\ifdefined\FRENCH
\chapter{Modèle de code}
\fi % FRENCH

\ifdefined\POLISH
\chapter{\PLph{}}
\fi % POLISH

% sections
\EN{\input{patterns/patterns_opt_dbg_EN}}
\ES{\input{patterns/patterns_opt_dbg_ES}}
\ITA{\input{patterns/patterns_opt_dbg_ITA}}
\PTBR{\input{patterns/patterns_opt_dbg_PTBR}}
\RU{\input{patterns/patterns_opt_dbg_RU}}
\THA{\input{patterns/patterns_opt_dbg_THA}}
\DE{\input{patterns/patterns_opt_dbg_DE}}
\FR{\input{patterns/patterns_opt_dbg_FR}}
\PL{\input{patterns/patterns_opt_dbg_PL}}

\RU{\section{Некоторые базовые понятия}}
\EN{\section{Some basics}}
\DE{\section{Einige Grundlagen}}
\FR{\section{Quelques bases}}
\ES{\section{\ESph{}}}
\ITA{\section{Alcune basi teoriche}}
\PTBR{\section{\PTBRph{}}}
\THA{\section{\THAph{}}}
\PL{\section{\PLph{}}}

% sections:
\EN{\input{patterns/intro_CPU_ISA_EN}}
\ES{\input{patterns/intro_CPU_ISA_ES}}
\ITA{\input{patterns/intro_CPU_ISA_ITA}}
\PTBR{\input{patterns/intro_CPU_ISA_PTBR}}
\RU{\input{patterns/intro_CPU_ISA_RU}}
\DE{\input{patterns/intro_CPU_ISA_DE}}
\FR{\input{patterns/intro_CPU_ISA_FR}}
\PL{\input{patterns/intro_CPU_ISA_PL}}

\EN{\input{patterns/numeral_EN}}
\RU{\input{patterns/numeral_RU}}
\ITA{\input{patterns/numeral_ITA}}
\DE{\input{patterns/numeral_DE}}
\FR{\input{patterns/numeral_FR}}
\PL{\input{patterns/numeral_PL}}

% chapters
\input{patterns/00_empty/main}
\input{patterns/011_ret/main}
\input{patterns/01_helloworld/main}
\input{patterns/015_prolog_epilogue/main}
\input{patterns/02_stack/main}
\input{patterns/03_printf/main}
\input{patterns/04_scanf/main}
\input{patterns/05_passing_arguments/main}
\input{patterns/06_return_results/main}
\input{patterns/061_pointers/main}
\input{patterns/065_GOTO/main}
\input{patterns/07_jcc/main}
\input{patterns/08_switch/main}
\input{patterns/09_loops/main}
\input{patterns/10_strings/main}
\input{patterns/11_arith_optimizations/main}
\input{patterns/12_FPU/main}
\input{patterns/13_arrays/main}
\input{patterns/14_bitfields/main}
\EN{\input{patterns/145_LCG/main_EN}}
\RU{\input{patterns/145_LCG/main_RU}}
\input{patterns/15_structs/main}
\input{patterns/17_unions/main}
\input{patterns/18_pointers_to_functions/main}
\input{patterns/185_64bit_in_32_env/main}

\EN{\input{patterns/19_SIMD/main_EN}}
\RU{\input{patterns/19_SIMD/main_RU}}
\DE{\input{patterns/19_SIMD/main_DE}}

\EN{\input{patterns/20_x64/main_EN}}
\RU{\input{patterns/20_x64/main_RU}}

\EN{\input{patterns/205_floating_SIMD/main_EN}}
\RU{\input{patterns/205_floating_SIMD/main_RU}}
\DE{\input{patterns/205_floating_SIMD/main_DE}}

\EN{\input{patterns/ARM/main_EN}}
\RU{\input{patterns/ARM/main_RU}}
\DE{\input{patterns/ARM/main_DE}}

\input{patterns/MIPS/main}

\ifdefined\SPANISH
\chapter{Patrones de código}
\fi % SPANISH

\ifdefined\GERMAN
\chapter{Code-Muster}
\fi % GERMAN

\ifdefined\ENGLISH
\chapter{Code Patterns}
\fi % ENGLISH

\ifdefined\ITALIAN
\chapter{Forme di codice}
\fi % ITALIAN

\ifdefined\RUSSIAN
\chapter{Образцы кода}
\fi % RUSSIAN

\ifdefined\BRAZILIAN
\chapter{Padrões de códigos}
\fi % BRAZILIAN

\ifdefined\THAI
\chapter{รูปแบบของโค้ด}
\fi % THAI

\ifdefined\FRENCH
\chapter{Modèle de code}
\fi % FRENCH

\ifdefined\POLISH
\chapter{\PLph{}}
\fi % POLISH

% sections
\EN{\input{patterns/patterns_opt_dbg_EN}}
\ES{\input{patterns/patterns_opt_dbg_ES}}
\ITA{\input{patterns/patterns_opt_dbg_ITA}}
\PTBR{\input{patterns/patterns_opt_dbg_PTBR}}
\RU{\input{patterns/patterns_opt_dbg_RU}}
\THA{\input{patterns/patterns_opt_dbg_THA}}
\DE{\input{patterns/patterns_opt_dbg_DE}}
\FR{\input{patterns/patterns_opt_dbg_FR}}
\PL{\input{patterns/patterns_opt_dbg_PL}}

\RU{\section{Некоторые базовые понятия}}
\EN{\section{Some basics}}
\DE{\section{Einige Grundlagen}}
\FR{\section{Quelques bases}}
\ES{\section{\ESph{}}}
\ITA{\section{Alcune basi teoriche}}
\PTBR{\section{\PTBRph{}}}
\THA{\section{\THAph{}}}
\PL{\section{\PLph{}}}

% sections:
\EN{\input{patterns/intro_CPU_ISA_EN}}
\ES{\input{patterns/intro_CPU_ISA_ES}}
\ITA{\input{patterns/intro_CPU_ISA_ITA}}
\PTBR{\input{patterns/intro_CPU_ISA_PTBR}}
\RU{\input{patterns/intro_CPU_ISA_RU}}
\DE{\input{patterns/intro_CPU_ISA_DE}}
\FR{\input{patterns/intro_CPU_ISA_FR}}
\PL{\input{patterns/intro_CPU_ISA_PL}}

\EN{\input{patterns/numeral_EN}}
\RU{\input{patterns/numeral_RU}}
\ITA{\input{patterns/numeral_ITA}}
\DE{\input{patterns/numeral_DE}}
\FR{\input{patterns/numeral_FR}}
\PL{\input{patterns/numeral_PL}}

% chapters
\input{patterns/00_empty/main}
\input{patterns/011_ret/main}
\input{patterns/01_helloworld/main}
\input{patterns/015_prolog_epilogue/main}
\input{patterns/02_stack/main}
\input{patterns/03_printf/main}
\input{patterns/04_scanf/main}
\input{patterns/05_passing_arguments/main}
\input{patterns/06_return_results/main}
\input{patterns/061_pointers/main}
\input{patterns/065_GOTO/main}
\input{patterns/07_jcc/main}
\input{patterns/08_switch/main}
\input{patterns/09_loops/main}
\input{patterns/10_strings/main}
\input{patterns/11_arith_optimizations/main}
\input{patterns/12_FPU/main}
\input{patterns/13_arrays/main}
\input{patterns/14_bitfields/main}
\EN{\input{patterns/145_LCG/main_EN}}
\RU{\input{patterns/145_LCG/main_RU}}
\input{patterns/15_structs/main}
\input{patterns/17_unions/main}
\input{patterns/18_pointers_to_functions/main}
\input{patterns/185_64bit_in_32_env/main}

\EN{\input{patterns/19_SIMD/main_EN}}
\RU{\input{patterns/19_SIMD/main_RU}}
\DE{\input{patterns/19_SIMD/main_DE}}

\EN{\input{patterns/20_x64/main_EN}}
\RU{\input{patterns/20_x64/main_RU}}

\EN{\input{patterns/205_floating_SIMD/main_EN}}
\RU{\input{patterns/205_floating_SIMD/main_RU}}
\DE{\input{patterns/205_floating_SIMD/main_DE}}

\EN{\input{patterns/ARM/main_EN}}
\RU{\input{patterns/ARM/main_RU}}
\DE{\input{patterns/ARM/main_DE}}

\input{patterns/MIPS/main}

\ifdefined\SPANISH
\chapter{Patrones de código}
\fi % SPANISH

\ifdefined\GERMAN
\chapter{Code-Muster}
\fi % GERMAN

\ifdefined\ENGLISH
\chapter{Code Patterns}
\fi % ENGLISH

\ifdefined\ITALIAN
\chapter{Forme di codice}
\fi % ITALIAN

\ifdefined\RUSSIAN
\chapter{Образцы кода}
\fi % RUSSIAN

\ifdefined\BRAZILIAN
\chapter{Padrões de códigos}
\fi % BRAZILIAN

\ifdefined\THAI
\chapter{รูปแบบของโค้ด}
\fi % THAI

\ifdefined\FRENCH
\chapter{Modèle de code}
\fi % FRENCH

\ifdefined\POLISH
\chapter{\PLph{}}
\fi % POLISH

% sections
\EN{\input{patterns/patterns_opt_dbg_EN}}
\ES{\input{patterns/patterns_opt_dbg_ES}}
\ITA{\input{patterns/patterns_opt_dbg_ITA}}
\PTBR{\input{patterns/patterns_opt_dbg_PTBR}}
\RU{\input{patterns/patterns_opt_dbg_RU}}
\THA{\input{patterns/patterns_opt_dbg_THA}}
\DE{\input{patterns/patterns_opt_dbg_DE}}
\FR{\input{patterns/patterns_opt_dbg_FR}}
\PL{\input{patterns/patterns_opt_dbg_PL}}

\RU{\section{Некоторые базовые понятия}}
\EN{\section{Some basics}}
\DE{\section{Einige Grundlagen}}
\FR{\section{Quelques bases}}
\ES{\section{\ESph{}}}
\ITA{\section{Alcune basi teoriche}}
\PTBR{\section{\PTBRph{}}}
\THA{\section{\THAph{}}}
\PL{\section{\PLph{}}}

% sections:
\EN{\input{patterns/intro_CPU_ISA_EN}}
\ES{\input{patterns/intro_CPU_ISA_ES}}
\ITA{\input{patterns/intro_CPU_ISA_ITA}}
\PTBR{\input{patterns/intro_CPU_ISA_PTBR}}
\RU{\input{patterns/intro_CPU_ISA_RU}}
\DE{\input{patterns/intro_CPU_ISA_DE}}
\FR{\input{patterns/intro_CPU_ISA_FR}}
\PL{\input{patterns/intro_CPU_ISA_PL}}

\EN{\input{patterns/numeral_EN}}
\RU{\input{patterns/numeral_RU}}
\ITA{\input{patterns/numeral_ITA}}
\DE{\input{patterns/numeral_DE}}
\FR{\input{patterns/numeral_FR}}
\PL{\input{patterns/numeral_PL}}

% chapters
\input{patterns/00_empty/main}
\input{patterns/011_ret/main}
\input{patterns/01_helloworld/main}
\input{patterns/015_prolog_epilogue/main}
\input{patterns/02_stack/main}
\input{patterns/03_printf/main}
\input{patterns/04_scanf/main}
\input{patterns/05_passing_arguments/main}
\input{patterns/06_return_results/main}
\input{patterns/061_pointers/main}
\input{patterns/065_GOTO/main}
\input{patterns/07_jcc/main}
\input{patterns/08_switch/main}
\input{patterns/09_loops/main}
\input{patterns/10_strings/main}
\input{patterns/11_arith_optimizations/main}
\input{patterns/12_FPU/main}
\input{patterns/13_arrays/main}
\input{patterns/14_bitfields/main}
\EN{\input{patterns/145_LCG/main_EN}}
\RU{\input{patterns/145_LCG/main_RU}}
\input{patterns/15_structs/main}
\input{patterns/17_unions/main}
\input{patterns/18_pointers_to_functions/main}
\input{patterns/185_64bit_in_32_env/main}

\EN{\input{patterns/19_SIMD/main_EN}}
\RU{\input{patterns/19_SIMD/main_RU}}
\DE{\input{patterns/19_SIMD/main_DE}}

\EN{\input{patterns/20_x64/main_EN}}
\RU{\input{patterns/20_x64/main_RU}}

\EN{\input{patterns/205_floating_SIMD/main_EN}}
\RU{\input{patterns/205_floating_SIMD/main_RU}}
\DE{\input{patterns/205_floating_SIMD/main_DE}}

\EN{\input{patterns/ARM/main_EN}}
\RU{\input{patterns/ARM/main_RU}}
\DE{\input{patterns/ARM/main_DE}}

\input{patterns/MIPS/main}

\ifdefined\SPANISH
\chapter{Patrones de código}
\fi % SPANISH

\ifdefined\GERMAN
\chapter{Code-Muster}
\fi % GERMAN

\ifdefined\ENGLISH
\chapter{Code Patterns}
\fi % ENGLISH

\ifdefined\ITALIAN
\chapter{Forme di codice}
\fi % ITALIAN

\ifdefined\RUSSIAN
\chapter{Образцы кода}
\fi % RUSSIAN

\ifdefined\BRAZILIAN
\chapter{Padrões de códigos}
\fi % BRAZILIAN

\ifdefined\THAI
\chapter{รูปแบบของโค้ด}
\fi % THAI

\ifdefined\FRENCH
\chapter{Modèle de code}
\fi % FRENCH

\ifdefined\POLISH
\chapter{\PLph{}}
\fi % POLISH

% sections
\EN{\input{patterns/patterns_opt_dbg_EN}}
\ES{\input{patterns/patterns_opt_dbg_ES}}
\ITA{\input{patterns/patterns_opt_dbg_ITA}}
\PTBR{\input{patterns/patterns_opt_dbg_PTBR}}
\RU{\input{patterns/patterns_opt_dbg_RU}}
\THA{\input{patterns/patterns_opt_dbg_THA}}
\DE{\input{patterns/patterns_opt_dbg_DE}}
\FR{\input{patterns/patterns_opt_dbg_FR}}
\PL{\input{patterns/patterns_opt_dbg_PL}}

\RU{\section{Некоторые базовые понятия}}
\EN{\section{Some basics}}
\DE{\section{Einige Grundlagen}}
\FR{\section{Quelques bases}}
\ES{\section{\ESph{}}}
\ITA{\section{Alcune basi teoriche}}
\PTBR{\section{\PTBRph{}}}
\THA{\section{\THAph{}}}
\PL{\section{\PLph{}}}

% sections:
\EN{\input{patterns/intro_CPU_ISA_EN}}
\ES{\input{patterns/intro_CPU_ISA_ES}}
\ITA{\input{patterns/intro_CPU_ISA_ITA}}
\PTBR{\input{patterns/intro_CPU_ISA_PTBR}}
\RU{\input{patterns/intro_CPU_ISA_RU}}
\DE{\input{patterns/intro_CPU_ISA_DE}}
\FR{\input{patterns/intro_CPU_ISA_FR}}
\PL{\input{patterns/intro_CPU_ISA_PL}}

\EN{\input{patterns/numeral_EN}}
\RU{\input{patterns/numeral_RU}}
\ITA{\input{patterns/numeral_ITA}}
\DE{\input{patterns/numeral_DE}}
\FR{\input{patterns/numeral_FR}}
\PL{\input{patterns/numeral_PL}}

% chapters
\input{patterns/00_empty/main}
\input{patterns/011_ret/main}
\input{patterns/01_helloworld/main}
\input{patterns/015_prolog_epilogue/main}
\input{patterns/02_stack/main}
\input{patterns/03_printf/main}
\input{patterns/04_scanf/main}
\input{patterns/05_passing_arguments/main}
\input{patterns/06_return_results/main}
\input{patterns/061_pointers/main}
\input{patterns/065_GOTO/main}
\input{patterns/07_jcc/main}
\input{patterns/08_switch/main}
\input{patterns/09_loops/main}
\input{patterns/10_strings/main}
\input{patterns/11_arith_optimizations/main}
\input{patterns/12_FPU/main}
\input{patterns/13_arrays/main}
\input{patterns/14_bitfields/main}
\EN{\input{patterns/145_LCG/main_EN}}
\RU{\input{patterns/145_LCG/main_RU}}
\input{patterns/15_structs/main}
\input{patterns/17_unions/main}
\input{patterns/18_pointers_to_functions/main}
\input{patterns/185_64bit_in_32_env/main}

\EN{\input{patterns/19_SIMD/main_EN}}
\RU{\input{patterns/19_SIMD/main_RU}}
\DE{\input{patterns/19_SIMD/main_DE}}

\EN{\input{patterns/20_x64/main_EN}}
\RU{\input{patterns/20_x64/main_RU}}

\EN{\input{patterns/205_floating_SIMD/main_EN}}
\RU{\input{patterns/205_floating_SIMD/main_RU}}
\DE{\input{patterns/205_floating_SIMD/main_DE}}

\EN{\input{patterns/ARM/main_EN}}
\RU{\input{patterns/ARM/main_RU}}
\DE{\input{patterns/ARM/main_DE}}

\input{patterns/MIPS/main}

\ifdefined\SPANISH
\chapter{Patrones de código}
\fi % SPANISH

\ifdefined\GERMAN
\chapter{Code-Muster}
\fi % GERMAN

\ifdefined\ENGLISH
\chapter{Code Patterns}
\fi % ENGLISH

\ifdefined\ITALIAN
\chapter{Forme di codice}
\fi % ITALIAN

\ifdefined\RUSSIAN
\chapter{Образцы кода}
\fi % RUSSIAN

\ifdefined\BRAZILIAN
\chapter{Padrões de códigos}
\fi % BRAZILIAN

\ifdefined\THAI
\chapter{รูปแบบของโค้ด}
\fi % THAI

\ifdefined\FRENCH
\chapter{Modèle de code}
\fi % FRENCH

\ifdefined\POLISH
\chapter{\PLph{}}
\fi % POLISH

% sections
\EN{\input{patterns/patterns_opt_dbg_EN}}
\ES{\input{patterns/patterns_opt_dbg_ES}}
\ITA{\input{patterns/patterns_opt_dbg_ITA}}
\PTBR{\input{patterns/patterns_opt_dbg_PTBR}}
\RU{\input{patterns/patterns_opt_dbg_RU}}
\THA{\input{patterns/patterns_opt_dbg_THA}}
\DE{\input{patterns/patterns_opt_dbg_DE}}
\FR{\input{patterns/patterns_opt_dbg_FR}}
\PL{\input{patterns/patterns_opt_dbg_PL}}

\RU{\section{Некоторые базовые понятия}}
\EN{\section{Some basics}}
\DE{\section{Einige Grundlagen}}
\FR{\section{Quelques bases}}
\ES{\section{\ESph{}}}
\ITA{\section{Alcune basi teoriche}}
\PTBR{\section{\PTBRph{}}}
\THA{\section{\THAph{}}}
\PL{\section{\PLph{}}}

% sections:
\EN{\input{patterns/intro_CPU_ISA_EN}}
\ES{\input{patterns/intro_CPU_ISA_ES}}
\ITA{\input{patterns/intro_CPU_ISA_ITA}}
\PTBR{\input{patterns/intro_CPU_ISA_PTBR}}
\RU{\input{patterns/intro_CPU_ISA_RU}}
\DE{\input{patterns/intro_CPU_ISA_DE}}
\FR{\input{patterns/intro_CPU_ISA_FR}}
\PL{\input{patterns/intro_CPU_ISA_PL}}

\EN{\input{patterns/numeral_EN}}
\RU{\input{patterns/numeral_RU}}
\ITA{\input{patterns/numeral_ITA}}
\DE{\input{patterns/numeral_DE}}
\FR{\input{patterns/numeral_FR}}
\PL{\input{patterns/numeral_PL}}

% chapters
\input{patterns/00_empty/main}
\input{patterns/011_ret/main}
\input{patterns/01_helloworld/main}
\input{patterns/015_prolog_epilogue/main}
\input{patterns/02_stack/main}
\input{patterns/03_printf/main}
\input{patterns/04_scanf/main}
\input{patterns/05_passing_arguments/main}
\input{patterns/06_return_results/main}
\input{patterns/061_pointers/main}
\input{patterns/065_GOTO/main}
\input{patterns/07_jcc/main}
\input{patterns/08_switch/main}
\input{patterns/09_loops/main}
\input{patterns/10_strings/main}
\input{patterns/11_arith_optimizations/main}
\input{patterns/12_FPU/main}
\input{patterns/13_arrays/main}
\input{patterns/14_bitfields/main}
\EN{\input{patterns/145_LCG/main_EN}}
\RU{\input{patterns/145_LCG/main_RU}}
\input{patterns/15_structs/main}
\input{patterns/17_unions/main}
\input{patterns/18_pointers_to_functions/main}
\input{patterns/185_64bit_in_32_env/main}

\EN{\input{patterns/19_SIMD/main_EN}}
\RU{\input{patterns/19_SIMD/main_RU}}
\DE{\input{patterns/19_SIMD/main_DE}}

\EN{\input{patterns/20_x64/main_EN}}
\RU{\input{patterns/20_x64/main_RU}}

\EN{\input{patterns/205_floating_SIMD/main_EN}}
\RU{\input{patterns/205_floating_SIMD/main_RU}}
\DE{\input{patterns/205_floating_SIMD/main_DE}}

\EN{\input{patterns/ARM/main_EN}}
\RU{\input{patterns/ARM/main_RU}}
\DE{\input{patterns/ARM/main_DE}}

\input{patterns/MIPS/main}

\ifdefined\SPANISH
\chapter{Patrones de código}
\fi % SPANISH

\ifdefined\GERMAN
\chapter{Code-Muster}
\fi % GERMAN

\ifdefined\ENGLISH
\chapter{Code Patterns}
\fi % ENGLISH

\ifdefined\ITALIAN
\chapter{Forme di codice}
\fi % ITALIAN

\ifdefined\RUSSIAN
\chapter{Образцы кода}
\fi % RUSSIAN

\ifdefined\BRAZILIAN
\chapter{Padrões de códigos}
\fi % BRAZILIAN

\ifdefined\THAI
\chapter{รูปแบบของโค้ด}
\fi % THAI

\ifdefined\FRENCH
\chapter{Modèle de code}
\fi % FRENCH

\ifdefined\POLISH
\chapter{\PLph{}}
\fi % POLISH

% sections
\EN{\input{patterns/patterns_opt_dbg_EN}}
\ES{\input{patterns/patterns_opt_dbg_ES}}
\ITA{\input{patterns/patterns_opt_dbg_ITA}}
\PTBR{\input{patterns/patterns_opt_dbg_PTBR}}
\RU{\input{patterns/patterns_opt_dbg_RU}}
\THA{\input{patterns/patterns_opt_dbg_THA}}
\DE{\input{patterns/patterns_opt_dbg_DE}}
\FR{\input{patterns/patterns_opt_dbg_FR}}
\PL{\input{patterns/patterns_opt_dbg_PL}}

\RU{\section{Некоторые базовые понятия}}
\EN{\section{Some basics}}
\DE{\section{Einige Grundlagen}}
\FR{\section{Quelques bases}}
\ES{\section{\ESph{}}}
\ITA{\section{Alcune basi teoriche}}
\PTBR{\section{\PTBRph{}}}
\THA{\section{\THAph{}}}
\PL{\section{\PLph{}}}

% sections:
\EN{\input{patterns/intro_CPU_ISA_EN}}
\ES{\input{patterns/intro_CPU_ISA_ES}}
\ITA{\input{patterns/intro_CPU_ISA_ITA}}
\PTBR{\input{patterns/intro_CPU_ISA_PTBR}}
\RU{\input{patterns/intro_CPU_ISA_RU}}
\DE{\input{patterns/intro_CPU_ISA_DE}}
\FR{\input{patterns/intro_CPU_ISA_FR}}
\PL{\input{patterns/intro_CPU_ISA_PL}}

\EN{\input{patterns/numeral_EN}}
\RU{\input{patterns/numeral_RU}}
\ITA{\input{patterns/numeral_ITA}}
\DE{\input{patterns/numeral_DE}}
\FR{\input{patterns/numeral_FR}}
\PL{\input{patterns/numeral_PL}}

% chapters
\input{patterns/00_empty/main}
\input{patterns/011_ret/main}
\input{patterns/01_helloworld/main}
\input{patterns/015_prolog_epilogue/main}
\input{patterns/02_stack/main}
\input{patterns/03_printf/main}
\input{patterns/04_scanf/main}
\input{patterns/05_passing_arguments/main}
\input{patterns/06_return_results/main}
\input{patterns/061_pointers/main}
\input{patterns/065_GOTO/main}
\input{patterns/07_jcc/main}
\input{patterns/08_switch/main}
\input{patterns/09_loops/main}
\input{patterns/10_strings/main}
\input{patterns/11_arith_optimizations/main}
\input{patterns/12_FPU/main}
\input{patterns/13_arrays/main}
\input{patterns/14_bitfields/main}
\EN{\input{patterns/145_LCG/main_EN}}
\RU{\input{patterns/145_LCG/main_RU}}
\input{patterns/15_structs/main}
\input{patterns/17_unions/main}
\input{patterns/18_pointers_to_functions/main}
\input{patterns/185_64bit_in_32_env/main}

\EN{\input{patterns/19_SIMD/main_EN}}
\RU{\input{patterns/19_SIMD/main_RU}}
\DE{\input{patterns/19_SIMD/main_DE}}

\EN{\input{patterns/20_x64/main_EN}}
\RU{\input{patterns/20_x64/main_RU}}

\EN{\input{patterns/205_floating_SIMD/main_EN}}
\RU{\input{patterns/205_floating_SIMD/main_RU}}
\DE{\input{patterns/205_floating_SIMD/main_DE}}

\EN{\input{patterns/ARM/main_EN}}
\RU{\input{patterns/ARM/main_RU}}
\DE{\input{patterns/ARM/main_DE}}

\input{patterns/MIPS/main}

\ifdefined\SPANISH
\chapter{Patrones de código}
\fi % SPANISH

\ifdefined\GERMAN
\chapter{Code-Muster}
\fi % GERMAN

\ifdefined\ENGLISH
\chapter{Code Patterns}
\fi % ENGLISH

\ifdefined\ITALIAN
\chapter{Forme di codice}
\fi % ITALIAN

\ifdefined\RUSSIAN
\chapter{Образцы кода}
\fi % RUSSIAN

\ifdefined\BRAZILIAN
\chapter{Padrões de códigos}
\fi % BRAZILIAN

\ifdefined\THAI
\chapter{รูปแบบของโค้ด}
\fi % THAI

\ifdefined\FRENCH
\chapter{Modèle de code}
\fi % FRENCH

\ifdefined\POLISH
\chapter{\PLph{}}
\fi % POLISH

% sections
\EN{\input{patterns/patterns_opt_dbg_EN}}
\ES{\input{patterns/patterns_opt_dbg_ES}}
\ITA{\input{patterns/patterns_opt_dbg_ITA}}
\PTBR{\input{patterns/patterns_opt_dbg_PTBR}}
\RU{\input{patterns/patterns_opt_dbg_RU}}
\THA{\input{patterns/patterns_opt_dbg_THA}}
\DE{\input{patterns/patterns_opt_dbg_DE}}
\FR{\input{patterns/patterns_opt_dbg_FR}}
\PL{\input{patterns/patterns_opt_dbg_PL}}

\RU{\section{Некоторые базовые понятия}}
\EN{\section{Some basics}}
\DE{\section{Einige Grundlagen}}
\FR{\section{Quelques bases}}
\ES{\section{\ESph{}}}
\ITA{\section{Alcune basi teoriche}}
\PTBR{\section{\PTBRph{}}}
\THA{\section{\THAph{}}}
\PL{\section{\PLph{}}}

% sections:
\EN{\input{patterns/intro_CPU_ISA_EN}}
\ES{\input{patterns/intro_CPU_ISA_ES}}
\ITA{\input{patterns/intro_CPU_ISA_ITA}}
\PTBR{\input{patterns/intro_CPU_ISA_PTBR}}
\RU{\input{patterns/intro_CPU_ISA_RU}}
\DE{\input{patterns/intro_CPU_ISA_DE}}
\FR{\input{patterns/intro_CPU_ISA_FR}}
\PL{\input{patterns/intro_CPU_ISA_PL}}

\EN{\input{patterns/numeral_EN}}
\RU{\input{patterns/numeral_RU}}
\ITA{\input{patterns/numeral_ITA}}
\DE{\input{patterns/numeral_DE}}
\FR{\input{patterns/numeral_FR}}
\PL{\input{patterns/numeral_PL}}

% chapters
\input{patterns/00_empty/main}
\input{patterns/011_ret/main}
\input{patterns/01_helloworld/main}
\input{patterns/015_prolog_epilogue/main}
\input{patterns/02_stack/main}
\input{patterns/03_printf/main}
\input{patterns/04_scanf/main}
\input{patterns/05_passing_arguments/main}
\input{patterns/06_return_results/main}
\input{patterns/061_pointers/main}
\input{patterns/065_GOTO/main}
\input{patterns/07_jcc/main}
\input{patterns/08_switch/main}
\input{patterns/09_loops/main}
\input{patterns/10_strings/main}
\input{patterns/11_arith_optimizations/main}
\input{patterns/12_FPU/main}
\input{patterns/13_arrays/main}
\input{patterns/14_bitfields/main}
\EN{\input{patterns/145_LCG/main_EN}}
\RU{\input{patterns/145_LCG/main_RU}}
\input{patterns/15_structs/main}
\input{patterns/17_unions/main}
\input{patterns/18_pointers_to_functions/main}
\input{patterns/185_64bit_in_32_env/main}

\EN{\input{patterns/19_SIMD/main_EN}}
\RU{\input{patterns/19_SIMD/main_RU}}
\DE{\input{patterns/19_SIMD/main_DE}}

\EN{\input{patterns/20_x64/main_EN}}
\RU{\input{patterns/20_x64/main_RU}}

\EN{\input{patterns/205_floating_SIMD/main_EN}}
\RU{\input{patterns/205_floating_SIMD/main_RU}}
\DE{\input{patterns/205_floating_SIMD/main_DE}}

\EN{\input{patterns/ARM/main_EN}}
\RU{\input{patterns/ARM/main_RU}}
\DE{\input{patterns/ARM/main_DE}}

\input{patterns/MIPS/main}

\ifdefined\SPANISH
\chapter{Patrones de código}
\fi % SPANISH

\ifdefined\GERMAN
\chapter{Code-Muster}
\fi % GERMAN

\ifdefined\ENGLISH
\chapter{Code Patterns}
\fi % ENGLISH

\ifdefined\ITALIAN
\chapter{Forme di codice}
\fi % ITALIAN

\ifdefined\RUSSIAN
\chapter{Образцы кода}
\fi % RUSSIAN

\ifdefined\BRAZILIAN
\chapter{Padrões de códigos}
\fi % BRAZILIAN

\ifdefined\THAI
\chapter{รูปแบบของโค้ด}
\fi % THAI

\ifdefined\FRENCH
\chapter{Modèle de code}
\fi % FRENCH

\ifdefined\POLISH
\chapter{\PLph{}}
\fi % POLISH

% sections
\EN{\input{patterns/patterns_opt_dbg_EN}}
\ES{\input{patterns/patterns_opt_dbg_ES}}
\ITA{\input{patterns/patterns_opt_dbg_ITA}}
\PTBR{\input{patterns/patterns_opt_dbg_PTBR}}
\RU{\input{patterns/patterns_opt_dbg_RU}}
\THA{\input{patterns/patterns_opt_dbg_THA}}
\DE{\input{patterns/patterns_opt_dbg_DE}}
\FR{\input{patterns/patterns_opt_dbg_FR}}
\PL{\input{patterns/patterns_opt_dbg_PL}}

\RU{\section{Некоторые базовые понятия}}
\EN{\section{Some basics}}
\DE{\section{Einige Grundlagen}}
\FR{\section{Quelques bases}}
\ES{\section{\ESph{}}}
\ITA{\section{Alcune basi teoriche}}
\PTBR{\section{\PTBRph{}}}
\THA{\section{\THAph{}}}
\PL{\section{\PLph{}}}

% sections:
\EN{\input{patterns/intro_CPU_ISA_EN}}
\ES{\input{patterns/intro_CPU_ISA_ES}}
\ITA{\input{patterns/intro_CPU_ISA_ITA}}
\PTBR{\input{patterns/intro_CPU_ISA_PTBR}}
\RU{\input{patterns/intro_CPU_ISA_RU}}
\DE{\input{patterns/intro_CPU_ISA_DE}}
\FR{\input{patterns/intro_CPU_ISA_FR}}
\PL{\input{patterns/intro_CPU_ISA_PL}}

\EN{\input{patterns/numeral_EN}}
\RU{\input{patterns/numeral_RU}}
\ITA{\input{patterns/numeral_ITA}}
\DE{\input{patterns/numeral_DE}}
\FR{\input{patterns/numeral_FR}}
\PL{\input{patterns/numeral_PL}}

% chapters
\input{patterns/00_empty/main}
\input{patterns/011_ret/main}
\input{patterns/01_helloworld/main}
\input{patterns/015_prolog_epilogue/main}
\input{patterns/02_stack/main}
\input{patterns/03_printf/main}
\input{patterns/04_scanf/main}
\input{patterns/05_passing_arguments/main}
\input{patterns/06_return_results/main}
\input{patterns/061_pointers/main}
\input{patterns/065_GOTO/main}
\input{patterns/07_jcc/main}
\input{patterns/08_switch/main}
\input{patterns/09_loops/main}
\input{patterns/10_strings/main}
\input{patterns/11_arith_optimizations/main}
\input{patterns/12_FPU/main}
\input{patterns/13_arrays/main}
\input{patterns/14_bitfields/main}
\EN{\input{patterns/145_LCG/main_EN}}
\RU{\input{patterns/145_LCG/main_RU}}
\input{patterns/15_structs/main}
\input{patterns/17_unions/main}
\input{patterns/18_pointers_to_functions/main}
\input{patterns/185_64bit_in_32_env/main}

\EN{\input{patterns/19_SIMD/main_EN}}
\RU{\input{patterns/19_SIMD/main_RU}}
\DE{\input{patterns/19_SIMD/main_DE}}

\EN{\input{patterns/20_x64/main_EN}}
\RU{\input{patterns/20_x64/main_RU}}

\EN{\input{patterns/205_floating_SIMD/main_EN}}
\RU{\input{patterns/205_floating_SIMD/main_RU}}
\DE{\input{patterns/205_floating_SIMD/main_DE}}

\EN{\input{patterns/ARM/main_EN}}
\RU{\input{patterns/ARM/main_RU}}
\DE{\input{patterns/ARM/main_DE}}

\input{patterns/MIPS/main}

\ifdefined\SPANISH
\chapter{Patrones de código}
\fi % SPANISH

\ifdefined\GERMAN
\chapter{Code-Muster}
\fi % GERMAN

\ifdefined\ENGLISH
\chapter{Code Patterns}
\fi % ENGLISH

\ifdefined\ITALIAN
\chapter{Forme di codice}
\fi % ITALIAN

\ifdefined\RUSSIAN
\chapter{Образцы кода}
\fi % RUSSIAN

\ifdefined\BRAZILIAN
\chapter{Padrões de códigos}
\fi % BRAZILIAN

\ifdefined\THAI
\chapter{รูปแบบของโค้ด}
\fi % THAI

\ifdefined\FRENCH
\chapter{Modèle de code}
\fi % FRENCH

\ifdefined\POLISH
\chapter{\PLph{}}
\fi % POLISH

% sections
\EN{\input{patterns/patterns_opt_dbg_EN}}
\ES{\input{patterns/patterns_opt_dbg_ES}}
\ITA{\input{patterns/patterns_opt_dbg_ITA}}
\PTBR{\input{patterns/patterns_opt_dbg_PTBR}}
\RU{\input{patterns/patterns_opt_dbg_RU}}
\THA{\input{patterns/patterns_opt_dbg_THA}}
\DE{\input{patterns/patterns_opt_dbg_DE}}
\FR{\input{patterns/patterns_opt_dbg_FR}}
\PL{\input{patterns/patterns_opt_dbg_PL}}

\RU{\section{Некоторые базовые понятия}}
\EN{\section{Some basics}}
\DE{\section{Einige Grundlagen}}
\FR{\section{Quelques bases}}
\ES{\section{\ESph{}}}
\ITA{\section{Alcune basi teoriche}}
\PTBR{\section{\PTBRph{}}}
\THA{\section{\THAph{}}}
\PL{\section{\PLph{}}}

% sections:
\EN{\input{patterns/intro_CPU_ISA_EN}}
\ES{\input{patterns/intro_CPU_ISA_ES}}
\ITA{\input{patterns/intro_CPU_ISA_ITA}}
\PTBR{\input{patterns/intro_CPU_ISA_PTBR}}
\RU{\input{patterns/intro_CPU_ISA_RU}}
\DE{\input{patterns/intro_CPU_ISA_DE}}
\FR{\input{patterns/intro_CPU_ISA_FR}}
\PL{\input{patterns/intro_CPU_ISA_PL}}

\EN{\input{patterns/numeral_EN}}
\RU{\input{patterns/numeral_RU}}
\ITA{\input{patterns/numeral_ITA}}
\DE{\input{patterns/numeral_DE}}
\FR{\input{patterns/numeral_FR}}
\PL{\input{patterns/numeral_PL}}

% chapters
\input{patterns/00_empty/main}
\input{patterns/011_ret/main}
\input{patterns/01_helloworld/main}
\input{patterns/015_prolog_epilogue/main}
\input{patterns/02_stack/main}
\input{patterns/03_printf/main}
\input{patterns/04_scanf/main}
\input{patterns/05_passing_arguments/main}
\input{patterns/06_return_results/main}
\input{patterns/061_pointers/main}
\input{patterns/065_GOTO/main}
\input{patterns/07_jcc/main}
\input{patterns/08_switch/main}
\input{patterns/09_loops/main}
\input{patterns/10_strings/main}
\input{patterns/11_arith_optimizations/main}
\input{patterns/12_FPU/main}
\input{patterns/13_arrays/main}
\input{patterns/14_bitfields/main}
\EN{\input{patterns/145_LCG/main_EN}}
\RU{\input{patterns/145_LCG/main_RU}}
\input{patterns/15_structs/main}
\input{patterns/17_unions/main}
\input{patterns/18_pointers_to_functions/main}
\input{patterns/185_64bit_in_32_env/main}

\EN{\input{patterns/19_SIMD/main_EN}}
\RU{\input{patterns/19_SIMD/main_RU}}
\DE{\input{patterns/19_SIMD/main_DE}}

\EN{\input{patterns/20_x64/main_EN}}
\RU{\input{patterns/20_x64/main_RU}}

\EN{\input{patterns/205_floating_SIMD/main_EN}}
\RU{\input{patterns/205_floating_SIMD/main_RU}}
\DE{\input{patterns/205_floating_SIMD/main_DE}}

\EN{\input{patterns/ARM/main_EN}}
\RU{\input{patterns/ARM/main_RU}}
\DE{\input{patterns/ARM/main_DE}}

\input{patterns/MIPS/main}

\ifdefined\SPANISH
\chapter{Patrones de código}
\fi % SPANISH

\ifdefined\GERMAN
\chapter{Code-Muster}
\fi % GERMAN

\ifdefined\ENGLISH
\chapter{Code Patterns}
\fi % ENGLISH

\ifdefined\ITALIAN
\chapter{Forme di codice}
\fi % ITALIAN

\ifdefined\RUSSIAN
\chapter{Образцы кода}
\fi % RUSSIAN

\ifdefined\BRAZILIAN
\chapter{Padrões de códigos}
\fi % BRAZILIAN

\ifdefined\THAI
\chapter{รูปแบบของโค้ด}
\fi % THAI

\ifdefined\FRENCH
\chapter{Modèle de code}
\fi % FRENCH

\ifdefined\POLISH
\chapter{\PLph{}}
\fi % POLISH

% sections
\EN{\input{patterns/patterns_opt_dbg_EN}}
\ES{\input{patterns/patterns_opt_dbg_ES}}
\ITA{\input{patterns/patterns_opt_dbg_ITA}}
\PTBR{\input{patterns/patterns_opt_dbg_PTBR}}
\RU{\input{patterns/patterns_opt_dbg_RU}}
\THA{\input{patterns/patterns_opt_dbg_THA}}
\DE{\input{patterns/patterns_opt_dbg_DE}}
\FR{\input{patterns/patterns_opt_dbg_FR}}
\PL{\input{patterns/patterns_opt_dbg_PL}}

\RU{\section{Некоторые базовые понятия}}
\EN{\section{Some basics}}
\DE{\section{Einige Grundlagen}}
\FR{\section{Quelques bases}}
\ES{\section{\ESph{}}}
\ITA{\section{Alcune basi teoriche}}
\PTBR{\section{\PTBRph{}}}
\THA{\section{\THAph{}}}
\PL{\section{\PLph{}}}

% sections:
\EN{\input{patterns/intro_CPU_ISA_EN}}
\ES{\input{patterns/intro_CPU_ISA_ES}}
\ITA{\input{patterns/intro_CPU_ISA_ITA}}
\PTBR{\input{patterns/intro_CPU_ISA_PTBR}}
\RU{\input{patterns/intro_CPU_ISA_RU}}
\DE{\input{patterns/intro_CPU_ISA_DE}}
\FR{\input{patterns/intro_CPU_ISA_FR}}
\PL{\input{patterns/intro_CPU_ISA_PL}}

\EN{\input{patterns/numeral_EN}}
\RU{\input{patterns/numeral_RU}}
\ITA{\input{patterns/numeral_ITA}}
\DE{\input{patterns/numeral_DE}}
\FR{\input{patterns/numeral_FR}}
\PL{\input{patterns/numeral_PL}}

% chapters
\input{patterns/00_empty/main}
\input{patterns/011_ret/main}
\input{patterns/01_helloworld/main}
\input{patterns/015_prolog_epilogue/main}
\input{patterns/02_stack/main}
\input{patterns/03_printf/main}
\input{patterns/04_scanf/main}
\input{patterns/05_passing_arguments/main}
\input{patterns/06_return_results/main}
\input{patterns/061_pointers/main}
\input{patterns/065_GOTO/main}
\input{patterns/07_jcc/main}
\input{patterns/08_switch/main}
\input{patterns/09_loops/main}
\input{patterns/10_strings/main}
\input{patterns/11_arith_optimizations/main}
\input{patterns/12_FPU/main}
\input{patterns/13_arrays/main}
\input{patterns/14_bitfields/main}
\EN{\input{patterns/145_LCG/main_EN}}
\RU{\input{patterns/145_LCG/main_RU}}
\input{patterns/15_structs/main}
\input{patterns/17_unions/main}
\input{patterns/18_pointers_to_functions/main}
\input{patterns/185_64bit_in_32_env/main}

\EN{\input{patterns/19_SIMD/main_EN}}
\RU{\input{patterns/19_SIMD/main_RU}}
\DE{\input{patterns/19_SIMD/main_DE}}

\EN{\input{patterns/20_x64/main_EN}}
\RU{\input{patterns/20_x64/main_RU}}

\EN{\input{patterns/205_floating_SIMD/main_EN}}
\RU{\input{patterns/205_floating_SIMD/main_RU}}
\DE{\input{patterns/205_floating_SIMD/main_DE}}

\EN{\input{patterns/ARM/main_EN}}
\RU{\input{patterns/ARM/main_RU}}
\DE{\input{patterns/ARM/main_DE}}

\input{patterns/MIPS/main}

\EN{\input{patterns/12_FPU/main_EN}}
\RU{\input{patterns/12_FPU/main_RU}}
\DE{\input{patterns/12_FPU/main_DE}}
\FR{\input{patterns/12_FPU/main_FR}}


\ifdefined\SPANISH
\chapter{Patrones de código}
\fi % SPANISH

\ifdefined\GERMAN
\chapter{Code-Muster}
\fi % GERMAN

\ifdefined\ENGLISH
\chapter{Code Patterns}
\fi % ENGLISH

\ifdefined\ITALIAN
\chapter{Forme di codice}
\fi % ITALIAN

\ifdefined\RUSSIAN
\chapter{Образцы кода}
\fi % RUSSIAN

\ifdefined\BRAZILIAN
\chapter{Padrões de códigos}
\fi % BRAZILIAN

\ifdefined\THAI
\chapter{รูปแบบของโค้ด}
\fi % THAI

\ifdefined\FRENCH
\chapter{Modèle de code}
\fi % FRENCH

\ifdefined\POLISH
\chapter{\PLph{}}
\fi % POLISH

% sections
\EN{\input{patterns/patterns_opt_dbg_EN}}
\ES{\input{patterns/patterns_opt_dbg_ES}}
\ITA{\input{patterns/patterns_opt_dbg_ITA}}
\PTBR{\input{patterns/patterns_opt_dbg_PTBR}}
\RU{\input{patterns/patterns_opt_dbg_RU}}
\THA{\input{patterns/patterns_opt_dbg_THA}}
\DE{\input{patterns/patterns_opt_dbg_DE}}
\FR{\input{patterns/patterns_opt_dbg_FR}}
\PL{\input{patterns/patterns_opt_dbg_PL}}

\RU{\section{Некоторые базовые понятия}}
\EN{\section{Some basics}}
\DE{\section{Einige Grundlagen}}
\FR{\section{Quelques bases}}
\ES{\section{\ESph{}}}
\ITA{\section{Alcune basi teoriche}}
\PTBR{\section{\PTBRph{}}}
\THA{\section{\THAph{}}}
\PL{\section{\PLph{}}}

% sections:
\EN{\input{patterns/intro_CPU_ISA_EN}}
\ES{\input{patterns/intro_CPU_ISA_ES}}
\ITA{\input{patterns/intro_CPU_ISA_ITA}}
\PTBR{\input{patterns/intro_CPU_ISA_PTBR}}
\RU{\input{patterns/intro_CPU_ISA_RU}}
\DE{\input{patterns/intro_CPU_ISA_DE}}
\FR{\input{patterns/intro_CPU_ISA_FR}}
\PL{\input{patterns/intro_CPU_ISA_PL}}

\EN{\input{patterns/numeral_EN}}
\RU{\input{patterns/numeral_RU}}
\ITA{\input{patterns/numeral_ITA}}
\DE{\input{patterns/numeral_DE}}
\FR{\input{patterns/numeral_FR}}
\PL{\input{patterns/numeral_PL}}

% chapters
\input{patterns/00_empty/main}
\input{patterns/011_ret/main}
\input{patterns/01_helloworld/main}
\input{patterns/015_prolog_epilogue/main}
\input{patterns/02_stack/main}
\input{patterns/03_printf/main}
\input{patterns/04_scanf/main}
\input{patterns/05_passing_arguments/main}
\input{patterns/06_return_results/main}
\input{patterns/061_pointers/main}
\input{patterns/065_GOTO/main}
\input{patterns/07_jcc/main}
\input{patterns/08_switch/main}
\input{patterns/09_loops/main}
\input{patterns/10_strings/main}
\input{patterns/11_arith_optimizations/main}
\input{patterns/12_FPU/main}
\input{patterns/13_arrays/main}
\input{patterns/14_bitfields/main}
\EN{\input{patterns/145_LCG/main_EN}}
\RU{\input{patterns/145_LCG/main_RU}}
\input{patterns/15_structs/main}
\input{patterns/17_unions/main}
\input{patterns/18_pointers_to_functions/main}
\input{patterns/185_64bit_in_32_env/main}

\EN{\input{patterns/19_SIMD/main_EN}}
\RU{\input{patterns/19_SIMD/main_RU}}
\DE{\input{patterns/19_SIMD/main_DE}}

\EN{\input{patterns/20_x64/main_EN}}
\RU{\input{patterns/20_x64/main_RU}}

\EN{\input{patterns/205_floating_SIMD/main_EN}}
\RU{\input{patterns/205_floating_SIMD/main_RU}}
\DE{\input{patterns/205_floating_SIMD/main_DE}}

\EN{\input{patterns/ARM/main_EN}}
\RU{\input{patterns/ARM/main_RU}}
\DE{\input{patterns/ARM/main_DE}}

\input{patterns/MIPS/main}

\ifdefined\SPANISH
\chapter{Patrones de código}
\fi % SPANISH

\ifdefined\GERMAN
\chapter{Code-Muster}
\fi % GERMAN

\ifdefined\ENGLISH
\chapter{Code Patterns}
\fi % ENGLISH

\ifdefined\ITALIAN
\chapter{Forme di codice}
\fi % ITALIAN

\ifdefined\RUSSIAN
\chapter{Образцы кода}
\fi % RUSSIAN

\ifdefined\BRAZILIAN
\chapter{Padrões de códigos}
\fi % BRAZILIAN

\ifdefined\THAI
\chapter{รูปแบบของโค้ด}
\fi % THAI

\ifdefined\FRENCH
\chapter{Modèle de code}
\fi % FRENCH

\ifdefined\POLISH
\chapter{\PLph{}}
\fi % POLISH

% sections
\EN{\input{patterns/patterns_opt_dbg_EN}}
\ES{\input{patterns/patterns_opt_dbg_ES}}
\ITA{\input{patterns/patterns_opt_dbg_ITA}}
\PTBR{\input{patterns/patterns_opt_dbg_PTBR}}
\RU{\input{patterns/patterns_opt_dbg_RU}}
\THA{\input{patterns/patterns_opt_dbg_THA}}
\DE{\input{patterns/patterns_opt_dbg_DE}}
\FR{\input{patterns/patterns_opt_dbg_FR}}
\PL{\input{patterns/patterns_opt_dbg_PL}}

\RU{\section{Некоторые базовые понятия}}
\EN{\section{Some basics}}
\DE{\section{Einige Grundlagen}}
\FR{\section{Quelques bases}}
\ES{\section{\ESph{}}}
\ITA{\section{Alcune basi teoriche}}
\PTBR{\section{\PTBRph{}}}
\THA{\section{\THAph{}}}
\PL{\section{\PLph{}}}

% sections:
\EN{\input{patterns/intro_CPU_ISA_EN}}
\ES{\input{patterns/intro_CPU_ISA_ES}}
\ITA{\input{patterns/intro_CPU_ISA_ITA}}
\PTBR{\input{patterns/intro_CPU_ISA_PTBR}}
\RU{\input{patterns/intro_CPU_ISA_RU}}
\DE{\input{patterns/intro_CPU_ISA_DE}}
\FR{\input{patterns/intro_CPU_ISA_FR}}
\PL{\input{patterns/intro_CPU_ISA_PL}}

\EN{\input{patterns/numeral_EN}}
\RU{\input{patterns/numeral_RU}}
\ITA{\input{patterns/numeral_ITA}}
\DE{\input{patterns/numeral_DE}}
\FR{\input{patterns/numeral_FR}}
\PL{\input{patterns/numeral_PL}}

% chapters
\input{patterns/00_empty/main}
\input{patterns/011_ret/main}
\input{patterns/01_helloworld/main}
\input{patterns/015_prolog_epilogue/main}
\input{patterns/02_stack/main}
\input{patterns/03_printf/main}
\input{patterns/04_scanf/main}
\input{patterns/05_passing_arguments/main}
\input{patterns/06_return_results/main}
\input{patterns/061_pointers/main}
\input{patterns/065_GOTO/main}
\input{patterns/07_jcc/main}
\input{patterns/08_switch/main}
\input{patterns/09_loops/main}
\input{patterns/10_strings/main}
\input{patterns/11_arith_optimizations/main}
\input{patterns/12_FPU/main}
\input{patterns/13_arrays/main}
\input{patterns/14_bitfields/main}
\EN{\input{patterns/145_LCG/main_EN}}
\RU{\input{patterns/145_LCG/main_RU}}
\input{patterns/15_structs/main}
\input{patterns/17_unions/main}
\input{patterns/18_pointers_to_functions/main}
\input{patterns/185_64bit_in_32_env/main}

\EN{\input{patterns/19_SIMD/main_EN}}
\RU{\input{patterns/19_SIMD/main_RU}}
\DE{\input{patterns/19_SIMD/main_DE}}

\EN{\input{patterns/20_x64/main_EN}}
\RU{\input{patterns/20_x64/main_RU}}

\EN{\input{patterns/205_floating_SIMD/main_EN}}
\RU{\input{patterns/205_floating_SIMD/main_RU}}
\DE{\input{patterns/205_floating_SIMD/main_DE}}

\EN{\input{patterns/ARM/main_EN}}
\RU{\input{patterns/ARM/main_RU}}
\DE{\input{patterns/ARM/main_DE}}

\input{patterns/MIPS/main}

\EN{\section{Returning Values}
\label{ret_val_func}

Another simple function is the one that simply returns a constant value:

\lstinputlisting[caption=\EN{\CCpp Code},style=customc]{patterns/011_ret/1.c}

Let's compile it.

\subsection{x86}

Here's what both the GCC and MSVC compilers produce (with optimization) on the x86 platform:

\lstinputlisting[caption=\Optimizing GCC/MSVC (\assemblyOutput),style=customasmx86]{patterns/011_ret/1.s}

\myindex{x86!\Instructions!RET}
There are just two instructions: the first places the value 123 into the \EAX register,
which is used by convention for storing the return
value, and the second one is \RET, which returns execution to the \gls{caller}.

The caller will take the result from the \EAX register.

\subsection{ARM}

There are a few differences on the ARM platform:

\lstinputlisting[caption=\OptimizingKeilVI (\ARMMode) ASM Output,style=customasmARM]{patterns/011_ret/1_Keil_ARM_O3.s}

ARM uses the register \Reg{0} for returning the results of functions, so 123 is copied into \Reg{0}.

\myindex{ARM!\Instructions!MOV}
\myindex{x86!\Instructions!MOV}
It is worth noting that \MOV is a misleading name for the instruction in both the x86 and ARM \ac{ISA}s.

The data is not in fact \IT{moved}, but \IT{copied}.

\subsection{MIPS}

\label{MIPS_leaf_function_ex1}

The GCC assembly output below lists registers by number:

\lstinputlisting[caption=\Optimizing GCC 4.4.5 (\assemblyOutput),style=customasmMIPS]{patterns/011_ret/MIPS.s}

\dots while \IDA does it by their pseudo names:

\lstinputlisting[caption=\Optimizing GCC 4.4.5 (IDA),style=customasmMIPS]{patterns/011_ret/MIPS_IDA.lst}

The \$2 (or \$V0) register is used to store the function's return value.
\myindex{MIPS!\Pseudoinstructions!LI}
\INS{LI} stands for ``Load Immediate'' and is the MIPS equivalent to \MOV.

\myindex{MIPS!\Instructions!J}
The other instruction is the jump instruction (J or JR) which returns the execution flow to the \gls{caller}.

\myindex{MIPS!Branch delay slot}
You might be wondering why the positions of the load instruction (LI) and the jump instruction (J or JR) are swapped. This is due to a \ac{RISC} feature called ``branch delay slot''.

The reason this happens is a quirk in the architecture of some RISC \ac{ISA}s and isn't important for our
purposes---we must simply keep in mind that in MIPS, the instruction following a jump or branch instruction
is executed \IT{before} the jump/branch instruction itself.

As a consequence, branch instructions always swap places with the instruction executed immediately beforehand.


In practice, functions which merely return 1 (\IT{true}) or 0 (\IT{false}) are very frequent.

The smallest ever of the standard UNIX utilities, \IT{/bin/true} and \IT{/bin/false} return 0 and 1 respectively, as an exit code.
(Zero as an exit code usually means success, non-zero means error.)
}
\RU{\subsubsection{std::string}
\myindex{\Cpp!STL!std::string}
\label{std_string}

\myparagraph{Как устроена структура}

Многие строковые библиотеки \InSqBrackets{\CNotes 2.2} обеспечивают структуру содержащую ссылку 
на буфер собственно со строкой, переменная всегда содержащую длину строки 
(что очень удобно для массы функций \InSqBrackets{\CNotes 2.2.1}) и переменную содержащую текущий размер буфера.

Строка в буфере обыкновенно оканчивается нулем: это для того чтобы указатель на буфер можно было
передавать в функции требующие на вход обычную сишную \ac{ASCIIZ}-строку.

Стандарт \Cpp не описывает, как именно нужно реализовывать std::string,
но, как правило, они реализованы как описано выше, с небольшими дополнениями.

Строки в \Cpp это не класс (как, например, QString в Qt), а темплейт (basic\_string), 
это сделано для того чтобы поддерживать 
строки содержащие разного типа символы: как минимум \Tchar и \IT{wchar\_t}.

Так что, std::string это класс с базовым типом \Tchar.

А std::wstring это класс с базовым типом \IT{wchar\_t}.

\mysubparagraph{MSVC}

В реализации MSVC, вместо ссылки на буфер может содержаться сам буфер (если строка короче 16-и символов).

Это означает, что каждая короткая строка будет занимать в памяти по крайней мере $16 + 4 + 4 = 24$ 
байт для 32-битной среды либо $16 + 8 + 8 = 32$ 
байта в 64-битной, а если строка длиннее 16-и символов, то прибавьте еще длину самой строки.

\lstinputlisting[caption=пример для MSVC,style=customc]{\CURPATH/STL/string/MSVC_RU.cpp}

Собственно, из этого исходника почти всё ясно.

Несколько замечаний:

Если строка короче 16-и символов, 
то отдельный буфер для строки в \glslink{heap}{куче} выделяться не будет.

Это удобно потому что на практике, основная часть строк действительно короткие.
Вероятно, разработчики в Microsoft выбрали размер в 16 символов как разумный баланс.

Теперь очень важный момент в конце функции main(): мы не пользуемся методом c\_str(), тем не менее,
если это скомпилировать и запустить, то обе строки появятся в консоли!

Работает это вот почему.

В первом случае строка короче 16-и символов и в начале объекта std::string (его можно рассматривать
просто как структуру) расположен буфер с этой строкой.
\printf трактует указатель как указатель на массив символов оканчивающийся нулем и поэтому всё работает.

Вывод второй строки (длиннее 16-и символов) даже еще опаснее: это вообще типичная программистская ошибка 
(или опечатка), забыть дописать c\_str().
Это работает потому что в это время в начале структуры расположен указатель на буфер.
Это может надолго остаться незамеченным: до тех пока там не появится строка 
короче 16-и символов, тогда процесс упадет.

\mysubparagraph{GCC}

В реализации GCC в структуре есть еще одна переменная --- reference count.

Интересно, что указатель на экземпляр класса std::string в GCC указывает не на начало самой структуры, 
а на указатель на буфера.
В libstdc++-v3\textbackslash{}include\textbackslash{}bits\textbackslash{}basic\_string.h 
мы можем прочитать что это сделано для удобства отладки:

\begin{lstlisting}
   *  The reason you want _M_data pointing to the character %array and
   *  not the _Rep is so that the debugger can see the string
   *  contents. (Probably we should add a non-inline member to get
   *  the _Rep for the debugger to use, so users can check the actual
   *  string length.)
\end{lstlisting}

\href{http://go.yurichev.com/17085}{исходный код basic\_string.h}

В нашем примере мы учитываем это:

\lstinputlisting[caption=пример для GCC,style=customc]{\CURPATH/STL/string/GCC_RU.cpp}

Нужны еще небольшие хаки чтобы сымитировать типичную ошибку, которую мы уже видели выше, из-за
более ужесточенной проверки типов в GCC, тем не менее, printf() работает и здесь без c\_str().

\myparagraph{Чуть более сложный пример}

\lstinputlisting[style=customc]{\CURPATH/STL/string/3.cpp}

\lstinputlisting[caption=MSVC 2012,style=customasmx86]{\CURPATH/STL/string/3_MSVC_RU.asm}

Собственно, компилятор не конструирует строки статически: да в общем-то и как
это возможно, если буфер с ней нужно хранить в \glslink{heap}{куче}?

Вместо этого в сегменте данных хранятся обычные \ac{ASCIIZ}-строки, а позже, во время выполнения, 
при помощи метода \q{assign}, конструируются строки s1 и s2
.
При помощи \TT{operator+}, создается строка s3.

Обратите внимание на то что вызов метода c\_str() отсутствует,
потому что его код достаточно короткий и компилятор вставил его прямо здесь:
если строка короче 16-и байт, то в регистре EAX остается указатель на буфер,
а если длиннее, то из этого же места достается адрес на буфер расположенный в \glslink{heap}{куче}.

Далее следуют вызовы трех деструкторов, причем, они вызываются только если строка длиннее 16-и байт:
тогда нужно освободить буфера в \glslink{heap}{куче}.
В противном случае, так как все три объекта std::string хранятся в стеке,
они освобождаются автоматически после выхода из функции.

Следовательно, работа с короткими строками более быстрая из-за м\'{е}ньшего обращения к \glslink{heap}{куче}.

Код на GCC даже проще (из-за того, что в GCC, как мы уже видели, не реализована возможность хранить короткую
строку прямо в структуре):

% TODO1 comment each function meaning
\lstinputlisting[caption=GCC 4.8.1,style=customasmx86]{\CURPATH/STL/string/3_GCC_RU.s}

Можно заметить, что в деструкторы передается не указатель на объект,
а указатель на место за 12 байт (или 3 слова) перед ним, то есть, на настоящее начало структуры.

\myparagraph{std::string как глобальная переменная}
\label{sec:std_string_as_global_variable}

Опытные программисты на \Cpp знают, что глобальные переменные \ac{STL}-типов вполне можно объявлять.

Да, действительно:

\lstinputlisting[style=customc]{\CURPATH/STL/string/5.cpp}

Но как и где будет вызываться конструктор \TT{std::string}?

На самом деле, эта переменная будет инициализирована даже перед началом \main.

\lstinputlisting[caption=MSVC 2012: здесь конструируется глобальная переменная{,} а также регистрируется её деструктор,style=customasmx86]{\CURPATH/STL/string/5_MSVC_p2.asm}

\lstinputlisting[caption=MSVC 2012: здесь глобальная переменная используется в \main,style=customasmx86]{\CURPATH/STL/string/5_MSVC_p1.asm}

\lstinputlisting[caption=MSVC 2012: эта функция-деструктор вызывается перед выходом,style=customasmx86]{\CURPATH/STL/string/5_MSVC_p3.asm}

\myindex{\CStandardLibrary!atexit()}
В реальности, из \ac{CRT}, еще до вызова main(), вызывается специальная функция,
в которой перечислены все конструкторы подобных переменных.
Более того: при помощи atexit() регистрируется функция, которая будет вызвана в конце работы программы:
в этой функции компилятор собирает вызовы деструкторов всех подобных глобальных переменных.

GCC работает похожим образом:

\lstinputlisting[caption=GCC 4.8.1,style=customasmx86]{\CURPATH/STL/string/5_GCC.s}

Но он не выделяет отдельной функции в которой будут собраны деструкторы: 
каждый деструктор передается в atexit() по одному.

% TODO а если глобальная STL-переменная в другом модуле? надо проверить.

}
\ifdefined\SPANISH
\chapter{Patrones de código}
\fi % SPANISH

\ifdefined\GERMAN
\chapter{Code-Muster}
\fi % GERMAN

\ifdefined\ENGLISH
\chapter{Code Patterns}
\fi % ENGLISH

\ifdefined\ITALIAN
\chapter{Forme di codice}
\fi % ITALIAN

\ifdefined\RUSSIAN
\chapter{Образцы кода}
\fi % RUSSIAN

\ifdefined\BRAZILIAN
\chapter{Padrões de códigos}
\fi % BRAZILIAN

\ifdefined\THAI
\chapter{รูปแบบของโค้ด}
\fi % THAI

\ifdefined\FRENCH
\chapter{Modèle de code}
\fi % FRENCH

\ifdefined\POLISH
\chapter{\PLph{}}
\fi % POLISH

% sections
\EN{\input{patterns/patterns_opt_dbg_EN}}
\ES{\input{patterns/patterns_opt_dbg_ES}}
\ITA{\input{patterns/patterns_opt_dbg_ITA}}
\PTBR{\input{patterns/patterns_opt_dbg_PTBR}}
\RU{\input{patterns/patterns_opt_dbg_RU}}
\THA{\input{patterns/patterns_opt_dbg_THA}}
\DE{\input{patterns/patterns_opt_dbg_DE}}
\FR{\input{patterns/patterns_opt_dbg_FR}}
\PL{\input{patterns/patterns_opt_dbg_PL}}

\RU{\section{Некоторые базовые понятия}}
\EN{\section{Some basics}}
\DE{\section{Einige Grundlagen}}
\FR{\section{Quelques bases}}
\ES{\section{\ESph{}}}
\ITA{\section{Alcune basi teoriche}}
\PTBR{\section{\PTBRph{}}}
\THA{\section{\THAph{}}}
\PL{\section{\PLph{}}}

% sections:
\EN{\input{patterns/intro_CPU_ISA_EN}}
\ES{\input{patterns/intro_CPU_ISA_ES}}
\ITA{\input{patterns/intro_CPU_ISA_ITA}}
\PTBR{\input{patterns/intro_CPU_ISA_PTBR}}
\RU{\input{patterns/intro_CPU_ISA_RU}}
\DE{\input{patterns/intro_CPU_ISA_DE}}
\FR{\input{patterns/intro_CPU_ISA_FR}}
\PL{\input{patterns/intro_CPU_ISA_PL}}

\EN{\input{patterns/numeral_EN}}
\RU{\input{patterns/numeral_RU}}
\ITA{\input{patterns/numeral_ITA}}
\DE{\input{patterns/numeral_DE}}
\FR{\input{patterns/numeral_FR}}
\PL{\input{patterns/numeral_PL}}

% chapters
\input{patterns/00_empty/main}
\input{patterns/011_ret/main}
\input{patterns/01_helloworld/main}
\input{patterns/015_prolog_epilogue/main}
\input{patterns/02_stack/main}
\input{patterns/03_printf/main}
\input{patterns/04_scanf/main}
\input{patterns/05_passing_arguments/main}
\input{patterns/06_return_results/main}
\input{patterns/061_pointers/main}
\input{patterns/065_GOTO/main}
\input{patterns/07_jcc/main}
\input{patterns/08_switch/main}
\input{patterns/09_loops/main}
\input{patterns/10_strings/main}
\input{patterns/11_arith_optimizations/main}
\input{patterns/12_FPU/main}
\input{patterns/13_arrays/main}
\input{patterns/14_bitfields/main}
\EN{\input{patterns/145_LCG/main_EN}}
\RU{\input{patterns/145_LCG/main_RU}}
\input{patterns/15_structs/main}
\input{patterns/17_unions/main}
\input{patterns/18_pointers_to_functions/main}
\input{patterns/185_64bit_in_32_env/main}

\EN{\input{patterns/19_SIMD/main_EN}}
\RU{\input{patterns/19_SIMD/main_RU}}
\DE{\input{patterns/19_SIMD/main_DE}}

\EN{\input{patterns/20_x64/main_EN}}
\RU{\input{patterns/20_x64/main_RU}}

\EN{\input{patterns/205_floating_SIMD/main_EN}}
\RU{\input{patterns/205_floating_SIMD/main_RU}}
\DE{\input{patterns/205_floating_SIMD/main_DE}}

\EN{\input{patterns/ARM/main_EN}}
\RU{\input{patterns/ARM/main_RU}}
\DE{\input{patterns/ARM/main_DE}}

\input{patterns/MIPS/main}

\ifdefined\SPANISH
\chapter{Patrones de código}
\fi % SPANISH

\ifdefined\GERMAN
\chapter{Code-Muster}
\fi % GERMAN

\ifdefined\ENGLISH
\chapter{Code Patterns}
\fi % ENGLISH

\ifdefined\ITALIAN
\chapter{Forme di codice}
\fi % ITALIAN

\ifdefined\RUSSIAN
\chapter{Образцы кода}
\fi % RUSSIAN

\ifdefined\BRAZILIAN
\chapter{Padrões de códigos}
\fi % BRAZILIAN

\ifdefined\THAI
\chapter{รูปแบบของโค้ด}
\fi % THAI

\ifdefined\FRENCH
\chapter{Modèle de code}
\fi % FRENCH

\ifdefined\POLISH
\chapter{\PLph{}}
\fi % POLISH

% sections
\EN{\input{patterns/patterns_opt_dbg_EN}}
\ES{\input{patterns/patterns_opt_dbg_ES}}
\ITA{\input{patterns/patterns_opt_dbg_ITA}}
\PTBR{\input{patterns/patterns_opt_dbg_PTBR}}
\RU{\input{patterns/patterns_opt_dbg_RU}}
\THA{\input{patterns/patterns_opt_dbg_THA}}
\DE{\input{patterns/patterns_opt_dbg_DE}}
\FR{\input{patterns/patterns_opt_dbg_FR}}
\PL{\input{patterns/patterns_opt_dbg_PL}}

\RU{\section{Некоторые базовые понятия}}
\EN{\section{Some basics}}
\DE{\section{Einige Grundlagen}}
\FR{\section{Quelques bases}}
\ES{\section{\ESph{}}}
\ITA{\section{Alcune basi teoriche}}
\PTBR{\section{\PTBRph{}}}
\THA{\section{\THAph{}}}
\PL{\section{\PLph{}}}

% sections:
\EN{\input{patterns/intro_CPU_ISA_EN}}
\ES{\input{patterns/intro_CPU_ISA_ES}}
\ITA{\input{patterns/intro_CPU_ISA_ITA}}
\PTBR{\input{patterns/intro_CPU_ISA_PTBR}}
\RU{\input{patterns/intro_CPU_ISA_RU}}
\DE{\input{patterns/intro_CPU_ISA_DE}}
\FR{\input{patterns/intro_CPU_ISA_FR}}
\PL{\input{patterns/intro_CPU_ISA_PL}}

\EN{\input{patterns/numeral_EN}}
\RU{\input{patterns/numeral_RU}}
\ITA{\input{patterns/numeral_ITA}}
\DE{\input{patterns/numeral_DE}}
\FR{\input{patterns/numeral_FR}}
\PL{\input{patterns/numeral_PL}}

% chapters
\input{patterns/00_empty/main}
\input{patterns/011_ret/main}
\input{patterns/01_helloworld/main}
\input{patterns/015_prolog_epilogue/main}
\input{patterns/02_stack/main}
\input{patterns/03_printf/main}
\input{patterns/04_scanf/main}
\input{patterns/05_passing_arguments/main}
\input{patterns/06_return_results/main}
\input{patterns/061_pointers/main}
\input{patterns/065_GOTO/main}
\input{patterns/07_jcc/main}
\input{patterns/08_switch/main}
\input{patterns/09_loops/main}
\input{patterns/10_strings/main}
\input{patterns/11_arith_optimizations/main}
\input{patterns/12_FPU/main}
\input{patterns/13_arrays/main}
\input{patterns/14_bitfields/main}
\EN{\input{patterns/145_LCG/main_EN}}
\RU{\input{patterns/145_LCG/main_RU}}
\input{patterns/15_structs/main}
\input{patterns/17_unions/main}
\input{patterns/18_pointers_to_functions/main}
\input{patterns/185_64bit_in_32_env/main}

\EN{\input{patterns/19_SIMD/main_EN}}
\RU{\input{patterns/19_SIMD/main_RU}}
\DE{\input{patterns/19_SIMD/main_DE}}

\EN{\input{patterns/20_x64/main_EN}}
\RU{\input{patterns/20_x64/main_RU}}

\EN{\input{patterns/205_floating_SIMD/main_EN}}
\RU{\input{patterns/205_floating_SIMD/main_RU}}
\DE{\input{patterns/205_floating_SIMD/main_DE}}

\EN{\input{patterns/ARM/main_EN}}
\RU{\input{patterns/ARM/main_RU}}
\DE{\input{patterns/ARM/main_DE}}

\input{patterns/MIPS/main}

\ifdefined\SPANISH
\chapter{Patrones de código}
\fi % SPANISH

\ifdefined\GERMAN
\chapter{Code-Muster}
\fi % GERMAN

\ifdefined\ENGLISH
\chapter{Code Patterns}
\fi % ENGLISH

\ifdefined\ITALIAN
\chapter{Forme di codice}
\fi % ITALIAN

\ifdefined\RUSSIAN
\chapter{Образцы кода}
\fi % RUSSIAN

\ifdefined\BRAZILIAN
\chapter{Padrões de códigos}
\fi % BRAZILIAN

\ifdefined\THAI
\chapter{รูปแบบของโค้ด}
\fi % THAI

\ifdefined\FRENCH
\chapter{Modèle de code}
\fi % FRENCH

\ifdefined\POLISH
\chapter{\PLph{}}
\fi % POLISH

% sections
\EN{\input{patterns/patterns_opt_dbg_EN}}
\ES{\input{patterns/patterns_opt_dbg_ES}}
\ITA{\input{patterns/patterns_opt_dbg_ITA}}
\PTBR{\input{patterns/patterns_opt_dbg_PTBR}}
\RU{\input{patterns/patterns_opt_dbg_RU}}
\THA{\input{patterns/patterns_opt_dbg_THA}}
\DE{\input{patterns/patterns_opt_dbg_DE}}
\FR{\input{patterns/patterns_opt_dbg_FR}}
\PL{\input{patterns/patterns_opt_dbg_PL}}

\RU{\section{Некоторые базовые понятия}}
\EN{\section{Some basics}}
\DE{\section{Einige Grundlagen}}
\FR{\section{Quelques bases}}
\ES{\section{\ESph{}}}
\ITA{\section{Alcune basi teoriche}}
\PTBR{\section{\PTBRph{}}}
\THA{\section{\THAph{}}}
\PL{\section{\PLph{}}}

% sections:
\EN{\input{patterns/intro_CPU_ISA_EN}}
\ES{\input{patterns/intro_CPU_ISA_ES}}
\ITA{\input{patterns/intro_CPU_ISA_ITA}}
\PTBR{\input{patterns/intro_CPU_ISA_PTBR}}
\RU{\input{patterns/intro_CPU_ISA_RU}}
\DE{\input{patterns/intro_CPU_ISA_DE}}
\FR{\input{patterns/intro_CPU_ISA_FR}}
\PL{\input{patterns/intro_CPU_ISA_PL}}

\EN{\input{patterns/numeral_EN}}
\RU{\input{patterns/numeral_RU}}
\ITA{\input{patterns/numeral_ITA}}
\DE{\input{patterns/numeral_DE}}
\FR{\input{patterns/numeral_FR}}
\PL{\input{patterns/numeral_PL}}

% chapters
\input{patterns/00_empty/main}
\input{patterns/011_ret/main}
\input{patterns/01_helloworld/main}
\input{patterns/015_prolog_epilogue/main}
\input{patterns/02_stack/main}
\input{patterns/03_printf/main}
\input{patterns/04_scanf/main}
\input{patterns/05_passing_arguments/main}
\input{patterns/06_return_results/main}
\input{patterns/061_pointers/main}
\input{patterns/065_GOTO/main}
\input{patterns/07_jcc/main}
\input{patterns/08_switch/main}
\input{patterns/09_loops/main}
\input{patterns/10_strings/main}
\input{patterns/11_arith_optimizations/main}
\input{patterns/12_FPU/main}
\input{patterns/13_arrays/main}
\input{patterns/14_bitfields/main}
\EN{\input{patterns/145_LCG/main_EN}}
\RU{\input{patterns/145_LCG/main_RU}}
\input{patterns/15_structs/main}
\input{patterns/17_unions/main}
\input{patterns/18_pointers_to_functions/main}
\input{patterns/185_64bit_in_32_env/main}

\EN{\input{patterns/19_SIMD/main_EN}}
\RU{\input{patterns/19_SIMD/main_RU}}
\DE{\input{patterns/19_SIMD/main_DE}}

\EN{\input{patterns/20_x64/main_EN}}
\RU{\input{patterns/20_x64/main_RU}}

\EN{\input{patterns/205_floating_SIMD/main_EN}}
\RU{\input{patterns/205_floating_SIMD/main_RU}}
\DE{\input{patterns/205_floating_SIMD/main_DE}}

\EN{\input{patterns/ARM/main_EN}}
\RU{\input{patterns/ARM/main_RU}}
\DE{\input{patterns/ARM/main_DE}}

\input{patterns/MIPS/main}

\ifdefined\SPANISH
\chapter{Patrones de código}
\fi % SPANISH

\ifdefined\GERMAN
\chapter{Code-Muster}
\fi % GERMAN

\ifdefined\ENGLISH
\chapter{Code Patterns}
\fi % ENGLISH

\ifdefined\ITALIAN
\chapter{Forme di codice}
\fi % ITALIAN

\ifdefined\RUSSIAN
\chapter{Образцы кода}
\fi % RUSSIAN

\ifdefined\BRAZILIAN
\chapter{Padrões de códigos}
\fi % BRAZILIAN

\ifdefined\THAI
\chapter{รูปแบบของโค้ด}
\fi % THAI

\ifdefined\FRENCH
\chapter{Modèle de code}
\fi % FRENCH

\ifdefined\POLISH
\chapter{\PLph{}}
\fi % POLISH

% sections
\EN{\input{patterns/patterns_opt_dbg_EN}}
\ES{\input{patterns/patterns_opt_dbg_ES}}
\ITA{\input{patterns/patterns_opt_dbg_ITA}}
\PTBR{\input{patterns/patterns_opt_dbg_PTBR}}
\RU{\input{patterns/patterns_opt_dbg_RU}}
\THA{\input{patterns/patterns_opt_dbg_THA}}
\DE{\input{patterns/patterns_opt_dbg_DE}}
\FR{\input{patterns/patterns_opt_dbg_FR}}
\PL{\input{patterns/patterns_opt_dbg_PL}}

\RU{\section{Некоторые базовые понятия}}
\EN{\section{Some basics}}
\DE{\section{Einige Grundlagen}}
\FR{\section{Quelques bases}}
\ES{\section{\ESph{}}}
\ITA{\section{Alcune basi teoriche}}
\PTBR{\section{\PTBRph{}}}
\THA{\section{\THAph{}}}
\PL{\section{\PLph{}}}

% sections:
\EN{\input{patterns/intro_CPU_ISA_EN}}
\ES{\input{patterns/intro_CPU_ISA_ES}}
\ITA{\input{patterns/intro_CPU_ISA_ITA}}
\PTBR{\input{patterns/intro_CPU_ISA_PTBR}}
\RU{\input{patterns/intro_CPU_ISA_RU}}
\DE{\input{patterns/intro_CPU_ISA_DE}}
\FR{\input{patterns/intro_CPU_ISA_FR}}
\PL{\input{patterns/intro_CPU_ISA_PL}}

\EN{\input{patterns/numeral_EN}}
\RU{\input{patterns/numeral_RU}}
\ITA{\input{patterns/numeral_ITA}}
\DE{\input{patterns/numeral_DE}}
\FR{\input{patterns/numeral_FR}}
\PL{\input{patterns/numeral_PL}}

% chapters
\input{patterns/00_empty/main}
\input{patterns/011_ret/main}
\input{patterns/01_helloworld/main}
\input{patterns/015_prolog_epilogue/main}
\input{patterns/02_stack/main}
\input{patterns/03_printf/main}
\input{patterns/04_scanf/main}
\input{patterns/05_passing_arguments/main}
\input{patterns/06_return_results/main}
\input{patterns/061_pointers/main}
\input{patterns/065_GOTO/main}
\input{patterns/07_jcc/main}
\input{patterns/08_switch/main}
\input{patterns/09_loops/main}
\input{patterns/10_strings/main}
\input{patterns/11_arith_optimizations/main}
\input{patterns/12_FPU/main}
\input{patterns/13_arrays/main}
\input{patterns/14_bitfields/main}
\EN{\input{patterns/145_LCG/main_EN}}
\RU{\input{patterns/145_LCG/main_RU}}
\input{patterns/15_structs/main}
\input{patterns/17_unions/main}
\input{patterns/18_pointers_to_functions/main}
\input{patterns/185_64bit_in_32_env/main}

\EN{\input{patterns/19_SIMD/main_EN}}
\RU{\input{patterns/19_SIMD/main_RU}}
\DE{\input{patterns/19_SIMD/main_DE}}

\EN{\input{patterns/20_x64/main_EN}}
\RU{\input{patterns/20_x64/main_RU}}

\EN{\input{patterns/205_floating_SIMD/main_EN}}
\RU{\input{patterns/205_floating_SIMD/main_RU}}
\DE{\input{patterns/205_floating_SIMD/main_DE}}

\EN{\input{patterns/ARM/main_EN}}
\RU{\input{patterns/ARM/main_RU}}
\DE{\input{patterns/ARM/main_DE}}

\input{patterns/MIPS/main}


\EN{\section{Returning Values}
\label{ret_val_func}

Another simple function is the one that simply returns a constant value:

\lstinputlisting[caption=\EN{\CCpp Code},style=customc]{patterns/011_ret/1.c}

Let's compile it.

\subsection{x86}

Here's what both the GCC and MSVC compilers produce (with optimization) on the x86 platform:

\lstinputlisting[caption=\Optimizing GCC/MSVC (\assemblyOutput),style=customasmx86]{patterns/011_ret/1.s}

\myindex{x86!\Instructions!RET}
There are just two instructions: the first places the value 123 into the \EAX register,
which is used by convention for storing the return
value, and the second one is \RET, which returns execution to the \gls{caller}.

The caller will take the result from the \EAX register.

\subsection{ARM}

There are a few differences on the ARM platform:

\lstinputlisting[caption=\OptimizingKeilVI (\ARMMode) ASM Output,style=customasmARM]{patterns/011_ret/1_Keil_ARM_O3.s}

ARM uses the register \Reg{0} for returning the results of functions, so 123 is copied into \Reg{0}.

\myindex{ARM!\Instructions!MOV}
\myindex{x86!\Instructions!MOV}
It is worth noting that \MOV is a misleading name for the instruction in both the x86 and ARM \ac{ISA}s.

The data is not in fact \IT{moved}, but \IT{copied}.

\subsection{MIPS}

\label{MIPS_leaf_function_ex1}

The GCC assembly output below lists registers by number:

\lstinputlisting[caption=\Optimizing GCC 4.4.5 (\assemblyOutput),style=customasmMIPS]{patterns/011_ret/MIPS.s}

\dots while \IDA does it by their pseudo names:

\lstinputlisting[caption=\Optimizing GCC 4.4.5 (IDA),style=customasmMIPS]{patterns/011_ret/MIPS_IDA.lst}

The \$2 (or \$V0) register is used to store the function's return value.
\myindex{MIPS!\Pseudoinstructions!LI}
\INS{LI} stands for ``Load Immediate'' and is the MIPS equivalent to \MOV.

\myindex{MIPS!\Instructions!J}
The other instruction is the jump instruction (J or JR) which returns the execution flow to the \gls{caller}.

\myindex{MIPS!Branch delay slot}
You might be wondering why the positions of the load instruction (LI) and the jump instruction (J or JR) are swapped. This is due to a \ac{RISC} feature called ``branch delay slot''.

The reason this happens is a quirk in the architecture of some RISC \ac{ISA}s and isn't important for our
purposes---we must simply keep in mind that in MIPS, the instruction following a jump or branch instruction
is executed \IT{before} the jump/branch instruction itself.

As a consequence, branch instructions always swap places with the instruction executed immediately beforehand.


In practice, functions which merely return 1 (\IT{true}) or 0 (\IT{false}) are very frequent.

The smallest ever of the standard UNIX utilities, \IT{/bin/true} and \IT{/bin/false} return 0 and 1 respectively, as an exit code.
(Zero as an exit code usually means success, non-zero means error.)
}
\RU{\subsubsection{std::string}
\myindex{\Cpp!STL!std::string}
\label{std_string}

\myparagraph{Как устроена структура}

Многие строковые библиотеки \InSqBrackets{\CNotes 2.2} обеспечивают структуру содержащую ссылку 
на буфер собственно со строкой, переменная всегда содержащую длину строки 
(что очень удобно для массы функций \InSqBrackets{\CNotes 2.2.1}) и переменную содержащую текущий размер буфера.

Строка в буфере обыкновенно оканчивается нулем: это для того чтобы указатель на буфер можно было
передавать в функции требующие на вход обычную сишную \ac{ASCIIZ}-строку.

Стандарт \Cpp не описывает, как именно нужно реализовывать std::string,
но, как правило, они реализованы как описано выше, с небольшими дополнениями.

Строки в \Cpp это не класс (как, например, QString в Qt), а темплейт (basic\_string), 
это сделано для того чтобы поддерживать 
строки содержащие разного типа символы: как минимум \Tchar и \IT{wchar\_t}.

Так что, std::string это класс с базовым типом \Tchar.

А std::wstring это класс с базовым типом \IT{wchar\_t}.

\mysubparagraph{MSVC}

В реализации MSVC, вместо ссылки на буфер может содержаться сам буфер (если строка короче 16-и символов).

Это означает, что каждая короткая строка будет занимать в памяти по крайней мере $16 + 4 + 4 = 24$ 
байт для 32-битной среды либо $16 + 8 + 8 = 32$ 
байта в 64-битной, а если строка длиннее 16-и символов, то прибавьте еще длину самой строки.

\lstinputlisting[caption=пример для MSVC,style=customc]{\CURPATH/STL/string/MSVC_RU.cpp}

Собственно, из этого исходника почти всё ясно.

Несколько замечаний:

Если строка короче 16-и символов, 
то отдельный буфер для строки в \glslink{heap}{куче} выделяться не будет.

Это удобно потому что на практике, основная часть строк действительно короткие.
Вероятно, разработчики в Microsoft выбрали размер в 16 символов как разумный баланс.

Теперь очень важный момент в конце функции main(): мы не пользуемся методом c\_str(), тем не менее,
если это скомпилировать и запустить, то обе строки появятся в консоли!

Работает это вот почему.

В первом случае строка короче 16-и символов и в начале объекта std::string (его можно рассматривать
просто как структуру) расположен буфер с этой строкой.
\printf трактует указатель как указатель на массив символов оканчивающийся нулем и поэтому всё работает.

Вывод второй строки (длиннее 16-и символов) даже еще опаснее: это вообще типичная программистская ошибка 
(или опечатка), забыть дописать c\_str().
Это работает потому что в это время в начале структуры расположен указатель на буфер.
Это может надолго остаться незамеченным: до тех пока там не появится строка 
короче 16-и символов, тогда процесс упадет.

\mysubparagraph{GCC}

В реализации GCC в структуре есть еще одна переменная --- reference count.

Интересно, что указатель на экземпляр класса std::string в GCC указывает не на начало самой структуры, 
а на указатель на буфера.
В libstdc++-v3\textbackslash{}include\textbackslash{}bits\textbackslash{}basic\_string.h 
мы можем прочитать что это сделано для удобства отладки:

\begin{lstlisting}
   *  The reason you want _M_data pointing to the character %array and
   *  not the _Rep is so that the debugger can see the string
   *  contents. (Probably we should add a non-inline member to get
   *  the _Rep for the debugger to use, so users can check the actual
   *  string length.)
\end{lstlisting}

\href{http://go.yurichev.com/17085}{исходный код basic\_string.h}

В нашем примере мы учитываем это:

\lstinputlisting[caption=пример для GCC,style=customc]{\CURPATH/STL/string/GCC_RU.cpp}

Нужны еще небольшие хаки чтобы сымитировать типичную ошибку, которую мы уже видели выше, из-за
более ужесточенной проверки типов в GCC, тем не менее, printf() работает и здесь без c\_str().

\myparagraph{Чуть более сложный пример}

\lstinputlisting[style=customc]{\CURPATH/STL/string/3.cpp}

\lstinputlisting[caption=MSVC 2012,style=customasmx86]{\CURPATH/STL/string/3_MSVC_RU.asm}

Собственно, компилятор не конструирует строки статически: да в общем-то и как
это возможно, если буфер с ней нужно хранить в \glslink{heap}{куче}?

Вместо этого в сегменте данных хранятся обычные \ac{ASCIIZ}-строки, а позже, во время выполнения, 
при помощи метода \q{assign}, конструируются строки s1 и s2
.
При помощи \TT{operator+}, создается строка s3.

Обратите внимание на то что вызов метода c\_str() отсутствует,
потому что его код достаточно короткий и компилятор вставил его прямо здесь:
если строка короче 16-и байт, то в регистре EAX остается указатель на буфер,
а если длиннее, то из этого же места достается адрес на буфер расположенный в \glslink{heap}{куче}.

Далее следуют вызовы трех деструкторов, причем, они вызываются только если строка длиннее 16-и байт:
тогда нужно освободить буфера в \glslink{heap}{куче}.
В противном случае, так как все три объекта std::string хранятся в стеке,
они освобождаются автоматически после выхода из функции.

Следовательно, работа с короткими строками более быстрая из-за м\'{е}ньшего обращения к \glslink{heap}{куче}.

Код на GCC даже проще (из-за того, что в GCC, как мы уже видели, не реализована возможность хранить короткую
строку прямо в структуре):

% TODO1 comment each function meaning
\lstinputlisting[caption=GCC 4.8.1,style=customasmx86]{\CURPATH/STL/string/3_GCC_RU.s}

Можно заметить, что в деструкторы передается не указатель на объект,
а указатель на место за 12 байт (или 3 слова) перед ним, то есть, на настоящее начало структуры.

\myparagraph{std::string как глобальная переменная}
\label{sec:std_string_as_global_variable}

Опытные программисты на \Cpp знают, что глобальные переменные \ac{STL}-типов вполне можно объявлять.

Да, действительно:

\lstinputlisting[style=customc]{\CURPATH/STL/string/5.cpp}

Но как и где будет вызываться конструктор \TT{std::string}?

На самом деле, эта переменная будет инициализирована даже перед началом \main.

\lstinputlisting[caption=MSVC 2012: здесь конструируется глобальная переменная{,} а также регистрируется её деструктор,style=customasmx86]{\CURPATH/STL/string/5_MSVC_p2.asm}

\lstinputlisting[caption=MSVC 2012: здесь глобальная переменная используется в \main,style=customasmx86]{\CURPATH/STL/string/5_MSVC_p1.asm}

\lstinputlisting[caption=MSVC 2012: эта функция-деструктор вызывается перед выходом,style=customasmx86]{\CURPATH/STL/string/5_MSVC_p3.asm}

\myindex{\CStandardLibrary!atexit()}
В реальности, из \ac{CRT}, еще до вызова main(), вызывается специальная функция,
в которой перечислены все конструкторы подобных переменных.
Более того: при помощи atexit() регистрируется функция, которая будет вызвана в конце работы программы:
в этой функции компилятор собирает вызовы деструкторов всех подобных глобальных переменных.

GCC работает похожим образом:

\lstinputlisting[caption=GCC 4.8.1,style=customasmx86]{\CURPATH/STL/string/5_GCC.s}

Но он не выделяет отдельной функции в которой будут собраны деструкторы: 
каждый деструктор передается в atexit() по одному.

% TODO а если глобальная STL-переменная в другом модуле? надо проверить.

}
\DE{\subsection{Einfachste XOR-Verschlüsselung überhaupt}

Ich habe einmal eine Software gesehen, bei der alle Debugging-Ausgaben mit XOR mit dem Wert 3
verschlüsselt wurden. Mit anderen Worten, die beiden niedrigsten Bits aller Buchstaben wurden invertiert.

``Hello, world'' wurde zu ``Kfool/\#tlqog'':

\begin{lstlisting}
#!/usr/bin/python

msg="Hello, world!"

print "".join(map(lambda x: chr(ord(x)^3), msg))
\end{lstlisting}

Das ist eine ziemlich interessante Verschlüsselung (oder besser eine Verschleierung),
weil sie zwei wichtige Eigenschaften hat:
1) es ist eine einzige Funktion zum Verschlüsseln und entschlüsseln, sie muss nur wiederholt angewendet werden
2) die entstehenden Buchstaben befinden sich im druckbaren Bereich, also die ganze Zeichenkette kann ohne
Escape-Symbole im Code verwendet werden.

Die zweite Eigenschaft nutzt die Tatsache, dass alle druckbaren Zeichen in Reihen organisiert sind: 0x2x-0x7x,
und wenn die beiden niederwertigsten Bits invertiert werden, wird der Buchstabe um eine oder drei Stellen nach
links oder rechts \IT{verschoben}, aber niemals in eine andere Reihe:

\begin{figure}[H]
\centering
\includegraphics[width=0.7\textwidth]{ascii_clean.png}
\caption{7-Bit \ac{ASCII} Tabelle in Emacs}
\end{figure}

\dots mit dem Zeichen 0x7F als einziger Ausnahme.

Im Folgenden werden also beispielsweise die Zeichen A-Z \IT{verschlüsselt}:

\begin{lstlisting}
#!/usr/bin/python

msg="@ABCDEFGHIJKLMNO"

print "".join(map(lambda x: chr(ord(x)^3), msg))
\end{lstlisting}

Ergebnis:
% FIXME \verb  --  relevant comment for German?
\begin{lstlisting}
CBA@GFEDKJIHONML
\end{lstlisting}

Es sieht so aus als würden die Zeichen ``@'' und ``C'' sowie ``B'' und ``A'' vertauscht werden.

Hier ist noch ein interessantes Beispiel, in dem gezeigt wird, wie die Eigenschaften von XOR
ausgenutzt werden können: Exakt den gleichen Effekt, dass druckbare Zeichen auch druckbar bleiben,
kann man dadurch erzielen, dass irgendeine Kombination der niedrigsten vier Bits invertiert wird.
}

\EN{\section{Returning Values}
\label{ret_val_func}

Another simple function is the one that simply returns a constant value:

\lstinputlisting[caption=\EN{\CCpp Code},style=customc]{patterns/011_ret/1.c}

Let's compile it.

\subsection{x86}

Here's what both the GCC and MSVC compilers produce (with optimization) on the x86 platform:

\lstinputlisting[caption=\Optimizing GCC/MSVC (\assemblyOutput),style=customasmx86]{patterns/011_ret/1.s}

\myindex{x86!\Instructions!RET}
There are just two instructions: the first places the value 123 into the \EAX register,
which is used by convention for storing the return
value, and the second one is \RET, which returns execution to the \gls{caller}.

The caller will take the result from the \EAX register.

\subsection{ARM}

There are a few differences on the ARM platform:

\lstinputlisting[caption=\OptimizingKeilVI (\ARMMode) ASM Output,style=customasmARM]{patterns/011_ret/1_Keil_ARM_O3.s}

ARM uses the register \Reg{0} for returning the results of functions, so 123 is copied into \Reg{0}.

\myindex{ARM!\Instructions!MOV}
\myindex{x86!\Instructions!MOV}
It is worth noting that \MOV is a misleading name for the instruction in both the x86 and ARM \ac{ISA}s.

The data is not in fact \IT{moved}, but \IT{copied}.

\subsection{MIPS}

\label{MIPS_leaf_function_ex1}

The GCC assembly output below lists registers by number:

\lstinputlisting[caption=\Optimizing GCC 4.4.5 (\assemblyOutput),style=customasmMIPS]{patterns/011_ret/MIPS.s}

\dots while \IDA does it by their pseudo names:

\lstinputlisting[caption=\Optimizing GCC 4.4.5 (IDA),style=customasmMIPS]{patterns/011_ret/MIPS_IDA.lst}

The \$2 (or \$V0) register is used to store the function's return value.
\myindex{MIPS!\Pseudoinstructions!LI}
\INS{LI} stands for ``Load Immediate'' and is the MIPS equivalent to \MOV.

\myindex{MIPS!\Instructions!J}
The other instruction is the jump instruction (J or JR) which returns the execution flow to the \gls{caller}.

\myindex{MIPS!Branch delay slot}
You might be wondering why the positions of the load instruction (LI) and the jump instruction (J or JR) are swapped. This is due to a \ac{RISC} feature called ``branch delay slot''.

The reason this happens is a quirk in the architecture of some RISC \ac{ISA}s and isn't important for our
purposes---we must simply keep in mind that in MIPS, the instruction following a jump or branch instruction
is executed \IT{before} the jump/branch instruction itself.

As a consequence, branch instructions always swap places with the instruction executed immediately beforehand.


In practice, functions which merely return 1 (\IT{true}) or 0 (\IT{false}) are very frequent.

The smallest ever of the standard UNIX utilities, \IT{/bin/true} and \IT{/bin/false} return 0 and 1 respectively, as an exit code.
(Zero as an exit code usually means success, non-zero means error.)
}
\RU{\subsubsection{std::string}
\myindex{\Cpp!STL!std::string}
\label{std_string}

\myparagraph{Как устроена структура}

Многие строковые библиотеки \InSqBrackets{\CNotes 2.2} обеспечивают структуру содержащую ссылку 
на буфер собственно со строкой, переменная всегда содержащую длину строки 
(что очень удобно для массы функций \InSqBrackets{\CNotes 2.2.1}) и переменную содержащую текущий размер буфера.

Строка в буфере обыкновенно оканчивается нулем: это для того чтобы указатель на буфер можно было
передавать в функции требующие на вход обычную сишную \ac{ASCIIZ}-строку.

Стандарт \Cpp не описывает, как именно нужно реализовывать std::string,
но, как правило, они реализованы как описано выше, с небольшими дополнениями.

Строки в \Cpp это не класс (как, например, QString в Qt), а темплейт (basic\_string), 
это сделано для того чтобы поддерживать 
строки содержащие разного типа символы: как минимум \Tchar и \IT{wchar\_t}.

Так что, std::string это класс с базовым типом \Tchar.

А std::wstring это класс с базовым типом \IT{wchar\_t}.

\mysubparagraph{MSVC}

В реализации MSVC, вместо ссылки на буфер может содержаться сам буфер (если строка короче 16-и символов).

Это означает, что каждая короткая строка будет занимать в памяти по крайней мере $16 + 4 + 4 = 24$ 
байт для 32-битной среды либо $16 + 8 + 8 = 32$ 
байта в 64-битной, а если строка длиннее 16-и символов, то прибавьте еще длину самой строки.

\lstinputlisting[caption=пример для MSVC,style=customc]{\CURPATH/STL/string/MSVC_RU.cpp}

Собственно, из этого исходника почти всё ясно.

Несколько замечаний:

Если строка короче 16-и символов, 
то отдельный буфер для строки в \glslink{heap}{куче} выделяться не будет.

Это удобно потому что на практике, основная часть строк действительно короткие.
Вероятно, разработчики в Microsoft выбрали размер в 16 символов как разумный баланс.

Теперь очень важный момент в конце функции main(): мы не пользуемся методом c\_str(), тем не менее,
если это скомпилировать и запустить, то обе строки появятся в консоли!

Работает это вот почему.

В первом случае строка короче 16-и символов и в начале объекта std::string (его можно рассматривать
просто как структуру) расположен буфер с этой строкой.
\printf трактует указатель как указатель на массив символов оканчивающийся нулем и поэтому всё работает.

Вывод второй строки (длиннее 16-и символов) даже еще опаснее: это вообще типичная программистская ошибка 
(или опечатка), забыть дописать c\_str().
Это работает потому что в это время в начале структуры расположен указатель на буфер.
Это может надолго остаться незамеченным: до тех пока там не появится строка 
короче 16-и символов, тогда процесс упадет.

\mysubparagraph{GCC}

В реализации GCC в структуре есть еще одна переменная --- reference count.

Интересно, что указатель на экземпляр класса std::string в GCC указывает не на начало самой структуры, 
а на указатель на буфера.
В libstdc++-v3\textbackslash{}include\textbackslash{}bits\textbackslash{}basic\_string.h 
мы можем прочитать что это сделано для удобства отладки:

\begin{lstlisting}
   *  The reason you want _M_data pointing to the character %array and
   *  not the _Rep is so that the debugger can see the string
   *  contents. (Probably we should add a non-inline member to get
   *  the _Rep for the debugger to use, so users can check the actual
   *  string length.)
\end{lstlisting}

\href{http://go.yurichev.com/17085}{исходный код basic\_string.h}

В нашем примере мы учитываем это:

\lstinputlisting[caption=пример для GCC,style=customc]{\CURPATH/STL/string/GCC_RU.cpp}

Нужны еще небольшие хаки чтобы сымитировать типичную ошибку, которую мы уже видели выше, из-за
более ужесточенной проверки типов в GCC, тем не менее, printf() работает и здесь без c\_str().

\myparagraph{Чуть более сложный пример}

\lstinputlisting[style=customc]{\CURPATH/STL/string/3.cpp}

\lstinputlisting[caption=MSVC 2012,style=customasmx86]{\CURPATH/STL/string/3_MSVC_RU.asm}

Собственно, компилятор не конструирует строки статически: да в общем-то и как
это возможно, если буфер с ней нужно хранить в \glslink{heap}{куче}?

Вместо этого в сегменте данных хранятся обычные \ac{ASCIIZ}-строки, а позже, во время выполнения, 
при помощи метода \q{assign}, конструируются строки s1 и s2
.
При помощи \TT{operator+}, создается строка s3.

Обратите внимание на то что вызов метода c\_str() отсутствует,
потому что его код достаточно короткий и компилятор вставил его прямо здесь:
если строка короче 16-и байт, то в регистре EAX остается указатель на буфер,
а если длиннее, то из этого же места достается адрес на буфер расположенный в \glslink{heap}{куче}.

Далее следуют вызовы трех деструкторов, причем, они вызываются только если строка длиннее 16-и байт:
тогда нужно освободить буфера в \glslink{heap}{куче}.
В противном случае, так как все три объекта std::string хранятся в стеке,
они освобождаются автоматически после выхода из функции.

Следовательно, работа с короткими строками более быстрая из-за м\'{е}ньшего обращения к \glslink{heap}{куче}.

Код на GCC даже проще (из-за того, что в GCC, как мы уже видели, не реализована возможность хранить короткую
строку прямо в структуре):

% TODO1 comment each function meaning
\lstinputlisting[caption=GCC 4.8.1,style=customasmx86]{\CURPATH/STL/string/3_GCC_RU.s}

Можно заметить, что в деструкторы передается не указатель на объект,
а указатель на место за 12 байт (или 3 слова) перед ним, то есть, на настоящее начало структуры.

\myparagraph{std::string как глобальная переменная}
\label{sec:std_string_as_global_variable}

Опытные программисты на \Cpp знают, что глобальные переменные \ac{STL}-типов вполне можно объявлять.

Да, действительно:

\lstinputlisting[style=customc]{\CURPATH/STL/string/5.cpp}

Но как и где будет вызываться конструктор \TT{std::string}?

На самом деле, эта переменная будет инициализирована даже перед началом \main.

\lstinputlisting[caption=MSVC 2012: здесь конструируется глобальная переменная{,} а также регистрируется её деструктор,style=customasmx86]{\CURPATH/STL/string/5_MSVC_p2.asm}

\lstinputlisting[caption=MSVC 2012: здесь глобальная переменная используется в \main,style=customasmx86]{\CURPATH/STL/string/5_MSVC_p1.asm}

\lstinputlisting[caption=MSVC 2012: эта функция-деструктор вызывается перед выходом,style=customasmx86]{\CURPATH/STL/string/5_MSVC_p3.asm}

\myindex{\CStandardLibrary!atexit()}
В реальности, из \ac{CRT}, еще до вызова main(), вызывается специальная функция,
в которой перечислены все конструкторы подобных переменных.
Более того: при помощи atexit() регистрируется функция, которая будет вызвана в конце работы программы:
в этой функции компилятор собирает вызовы деструкторов всех подобных глобальных переменных.

GCC работает похожим образом:

\lstinputlisting[caption=GCC 4.8.1,style=customasmx86]{\CURPATH/STL/string/5_GCC.s}

Но он не выделяет отдельной функции в которой будут собраны деструкторы: 
каждый деструктор передается в atexit() по одному.

% TODO а если глобальная STL-переменная в другом модуле? надо проверить.

}

\EN{\section{Returning Values}
\label{ret_val_func}

Another simple function is the one that simply returns a constant value:

\lstinputlisting[caption=\EN{\CCpp Code},style=customc]{patterns/011_ret/1.c}

Let's compile it.

\subsection{x86}

Here's what both the GCC and MSVC compilers produce (with optimization) on the x86 platform:

\lstinputlisting[caption=\Optimizing GCC/MSVC (\assemblyOutput),style=customasmx86]{patterns/011_ret/1.s}

\myindex{x86!\Instructions!RET}
There are just two instructions: the first places the value 123 into the \EAX register,
which is used by convention for storing the return
value, and the second one is \RET, which returns execution to the \gls{caller}.

The caller will take the result from the \EAX register.

\subsection{ARM}

There are a few differences on the ARM platform:

\lstinputlisting[caption=\OptimizingKeilVI (\ARMMode) ASM Output,style=customasmARM]{patterns/011_ret/1_Keil_ARM_O3.s}

ARM uses the register \Reg{0} for returning the results of functions, so 123 is copied into \Reg{0}.

\myindex{ARM!\Instructions!MOV}
\myindex{x86!\Instructions!MOV}
It is worth noting that \MOV is a misleading name for the instruction in both the x86 and ARM \ac{ISA}s.

The data is not in fact \IT{moved}, but \IT{copied}.

\subsection{MIPS}

\label{MIPS_leaf_function_ex1}

The GCC assembly output below lists registers by number:

\lstinputlisting[caption=\Optimizing GCC 4.4.5 (\assemblyOutput),style=customasmMIPS]{patterns/011_ret/MIPS.s}

\dots while \IDA does it by their pseudo names:

\lstinputlisting[caption=\Optimizing GCC 4.4.5 (IDA),style=customasmMIPS]{patterns/011_ret/MIPS_IDA.lst}

The \$2 (or \$V0) register is used to store the function's return value.
\myindex{MIPS!\Pseudoinstructions!LI}
\INS{LI} stands for ``Load Immediate'' and is the MIPS equivalent to \MOV.

\myindex{MIPS!\Instructions!J}
The other instruction is the jump instruction (J or JR) which returns the execution flow to the \gls{caller}.

\myindex{MIPS!Branch delay slot}
You might be wondering why the positions of the load instruction (LI) and the jump instruction (J or JR) are swapped. This is due to a \ac{RISC} feature called ``branch delay slot''.

The reason this happens is a quirk in the architecture of some RISC \ac{ISA}s and isn't important for our
purposes---we must simply keep in mind that in MIPS, the instruction following a jump or branch instruction
is executed \IT{before} the jump/branch instruction itself.

As a consequence, branch instructions always swap places with the instruction executed immediately beforehand.


In practice, functions which merely return 1 (\IT{true}) or 0 (\IT{false}) are very frequent.

The smallest ever of the standard UNIX utilities, \IT{/bin/true} and \IT{/bin/false} return 0 and 1 respectively, as an exit code.
(Zero as an exit code usually means success, non-zero means error.)
}
\RU{\subsubsection{std::string}
\myindex{\Cpp!STL!std::string}
\label{std_string}

\myparagraph{Как устроена структура}

Многие строковые библиотеки \InSqBrackets{\CNotes 2.2} обеспечивают структуру содержащую ссылку 
на буфер собственно со строкой, переменная всегда содержащую длину строки 
(что очень удобно для массы функций \InSqBrackets{\CNotes 2.2.1}) и переменную содержащую текущий размер буфера.

Строка в буфере обыкновенно оканчивается нулем: это для того чтобы указатель на буфер можно было
передавать в функции требующие на вход обычную сишную \ac{ASCIIZ}-строку.

Стандарт \Cpp не описывает, как именно нужно реализовывать std::string,
но, как правило, они реализованы как описано выше, с небольшими дополнениями.

Строки в \Cpp это не класс (как, например, QString в Qt), а темплейт (basic\_string), 
это сделано для того чтобы поддерживать 
строки содержащие разного типа символы: как минимум \Tchar и \IT{wchar\_t}.

Так что, std::string это класс с базовым типом \Tchar.

А std::wstring это класс с базовым типом \IT{wchar\_t}.

\mysubparagraph{MSVC}

В реализации MSVC, вместо ссылки на буфер может содержаться сам буфер (если строка короче 16-и символов).

Это означает, что каждая короткая строка будет занимать в памяти по крайней мере $16 + 4 + 4 = 24$ 
байт для 32-битной среды либо $16 + 8 + 8 = 32$ 
байта в 64-битной, а если строка длиннее 16-и символов, то прибавьте еще длину самой строки.

\lstinputlisting[caption=пример для MSVC,style=customc]{\CURPATH/STL/string/MSVC_RU.cpp}

Собственно, из этого исходника почти всё ясно.

Несколько замечаний:

Если строка короче 16-и символов, 
то отдельный буфер для строки в \glslink{heap}{куче} выделяться не будет.

Это удобно потому что на практике, основная часть строк действительно короткие.
Вероятно, разработчики в Microsoft выбрали размер в 16 символов как разумный баланс.

Теперь очень важный момент в конце функции main(): мы не пользуемся методом c\_str(), тем не менее,
если это скомпилировать и запустить, то обе строки появятся в консоли!

Работает это вот почему.

В первом случае строка короче 16-и символов и в начале объекта std::string (его можно рассматривать
просто как структуру) расположен буфер с этой строкой.
\printf трактует указатель как указатель на массив символов оканчивающийся нулем и поэтому всё работает.

Вывод второй строки (длиннее 16-и символов) даже еще опаснее: это вообще типичная программистская ошибка 
(или опечатка), забыть дописать c\_str().
Это работает потому что в это время в начале структуры расположен указатель на буфер.
Это может надолго остаться незамеченным: до тех пока там не появится строка 
короче 16-и символов, тогда процесс упадет.

\mysubparagraph{GCC}

В реализации GCC в структуре есть еще одна переменная --- reference count.

Интересно, что указатель на экземпляр класса std::string в GCC указывает не на начало самой структуры, 
а на указатель на буфера.
В libstdc++-v3\textbackslash{}include\textbackslash{}bits\textbackslash{}basic\_string.h 
мы можем прочитать что это сделано для удобства отладки:

\begin{lstlisting}
   *  The reason you want _M_data pointing to the character %array and
   *  not the _Rep is so that the debugger can see the string
   *  contents. (Probably we should add a non-inline member to get
   *  the _Rep for the debugger to use, so users can check the actual
   *  string length.)
\end{lstlisting}

\href{http://go.yurichev.com/17085}{исходный код basic\_string.h}

В нашем примере мы учитываем это:

\lstinputlisting[caption=пример для GCC,style=customc]{\CURPATH/STL/string/GCC_RU.cpp}

Нужны еще небольшие хаки чтобы сымитировать типичную ошибку, которую мы уже видели выше, из-за
более ужесточенной проверки типов в GCC, тем не менее, printf() работает и здесь без c\_str().

\myparagraph{Чуть более сложный пример}

\lstinputlisting[style=customc]{\CURPATH/STL/string/3.cpp}

\lstinputlisting[caption=MSVC 2012,style=customasmx86]{\CURPATH/STL/string/3_MSVC_RU.asm}

Собственно, компилятор не конструирует строки статически: да в общем-то и как
это возможно, если буфер с ней нужно хранить в \glslink{heap}{куче}?

Вместо этого в сегменте данных хранятся обычные \ac{ASCIIZ}-строки, а позже, во время выполнения, 
при помощи метода \q{assign}, конструируются строки s1 и s2
.
При помощи \TT{operator+}, создается строка s3.

Обратите внимание на то что вызов метода c\_str() отсутствует,
потому что его код достаточно короткий и компилятор вставил его прямо здесь:
если строка короче 16-и байт, то в регистре EAX остается указатель на буфер,
а если длиннее, то из этого же места достается адрес на буфер расположенный в \glslink{heap}{куче}.

Далее следуют вызовы трех деструкторов, причем, они вызываются только если строка длиннее 16-и байт:
тогда нужно освободить буфера в \glslink{heap}{куче}.
В противном случае, так как все три объекта std::string хранятся в стеке,
они освобождаются автоматически после выхода из функции.

Следовательно, работа с короткими строками более быстрая из-за м\'{е}ньшего обращения к \glslink{heap}{куче}.

Код на GCC даже проще (из-за того, что в GCC, как мы уже видели, не реализована возможность хранить короткую
строку прямо в структуре):

% TODO1 comment each function meaning
\lstinputlisting[caption=GCC 4.8.1,style=customasmx86]{\CURPATH/STL/string/3_GCC_RU.s}

Можно заметить, что в деструкторы передается не указатель на объект,
а указатель на место за 12 байт (или 3 слова) перед ним, то есть, на настоящее начало структуры.

\myparagraph{std::string как глобальная переменная}
\label{sec:std_string_as_global_variable}

Опытные программисты на \Cpp знают, что глобальные переменные \ac{STL}-типов вполне можно объявлять.

Да, действительно:

\lstinputlisting[style=customc]{\CURPATH/STL/string/5.cpp}

Но как и где будет вызываться конструктор \TT{std::string}?

На самом деле, эта переменная будет инициализирована даже перед началом \main.

\lstinputlisting[caption=MSVC 2012: здесь конструируется глобальная переменная{,} а также регистрируется её деструктор,style=customasmx86]{\CURPATH/STL/string/5_MSVC_p2.asm}

\lstinputlisting[caption=MSVC 2012: здесь глобальная переменная используется в \main,style=customasmx86]{\CURPATH/STL/string/5_MSVC_p1.asm}

\lstinputlisting[caption=MSVC 2012: эта функция-деструктор вызывается перед выходом,style=customasmx86]{\CURPATH/STL/string/5_MSVC_p3.asm}

\myindex{\CStandardLibrary!atexit()}
В реальности, из \ac{CRT}, еще до вызова main(), вызывается специальная функция,
в которой перечислены все конструкторы подобных переменных.
Более того: при помощи atexit() регистрируется функция, которая будет вызвана в конце работы программы:
в этой функции компилятор собирает вызовы деструкторов всех подобных глобальных переменных.

GCC работает похожим образом:

\lstinputlisting[caption=GCC 4.8.1,style=customasmx86]{\CURPATH/STL/string/5_GCC.s}

Но он не выделяет отдельной функции в которой будут собраны деструкторы: 
каждый деструктор передается в atexit() по одному.

% TODO а если глобальная STL-переменная в другом модуле? надо проверить.

}
\DE{\subsection{Einfachste XOR-Verschlüsselung überhaupt}

Ich habe einmal eine Software gesehen, bei der alle Debugging-Ausgaben mit XOR mit dem Wert 3
verschlüsselt wurden. Mit anderen Worten, die beiden niedrigsten Bits aller Buchstaben wurden invertiert.

``Hello, world'' wurde zu ``Kfool/\#tlqog'':

\begin{lstlisting}
#!/usr/bin/python

msg="Hello, world!"

print "".join(map(lambda x: chr(ord(x)^3), msg))
\end{lstlisting}

Das ist eine ziemlich interessante Verschlüsselung (oder besser eine Verschleierung),
weil sie zwei wichtige Eigenschaften hat:
1) es ist eine einzige Funktion zum Verschlüsseln und entschlüsseln, sie muss nur wiederholt angewendet werden
2) die entstehenden Buchstaben befinden sich im druckbaren Bereich, also die ganze Zeichenkette kann ohne
Escape-Symbole im Code verwendet werden.

Die zweite Eigenschaft nutzt die Tatsache, dass alle druckbaren Zeichen in Reihen organisiert sind: 0x2x-0x7x,
und wenn die beiden niederwertigsten Bits invertiert werden, wird der Buchstabe um eine oder drei Stellen nach
links oder rechts \IT{verschoben}, aber niemals in eine andere Reihe:

\begin{figure}[H]
\centering
\includegraphics[width=0.7\textwidth]{ascii_clean.png}
\caption{7-Bit \ac{ASCII} Tabelle in Emacs}
\end{figure}

\dots mit dem Zeichen 0x7F als einziger Ausnahme.

Im Folgenden werden also beispielsweise die Zeichen A-Z \IT{verschlüsselt}:

\begin{lstlisting}
#!/usr/bin/python

msg="@ABCDEFGHIJKLMNO"

print "".join(map(lambda x: chr(ord(x)^3), msg))
\end{lstlisting}

Ergebnis:
% FIXME \verb  --  relevant comment for German?
\begin{lstlisting}
CBA@GFEDKJIHONML
\end{lstlisting}

Es sieht so aus als würden die Zeichen ``@'' und ``C'' sowie ``B'' und ``A'' vertauscht werden.

Hier ist noch ein interessantes Beispiel, in dem gezeigt wird, wie die Eigenschaften von XOR
ausgenutzt werden können: Exakt den gleichen Effekt, dass druckbare Zeichen auch druckbar bleiben,
kann man dadurch erzielen, dass irgendeine Kombination der niedrigsten vier Bits invertiert wird.
}

\EN{\section{Returning Values}
\label{ret_val_func}

Another simple function is the one that simply returns a constant value:

\lstinputlisting[caption=\EN{\CCpp Code},style=customc]{patterns/011_ret/1.c}

Let's compile it.

\subsection{x86}

Here's what both the GCC and MSVC compilers produce (with optimization) on the x86 platform:

\lstinputlisting[caption=\Optimizing GCC/MSVC (\assemblyOutput),style=customasmx86]{patterns/011_ret/1.s}

\myindex{x86!\Instructions!RET}
There are just two instructions: the first places the value 123 into the \EAX register,
which is used by convention for storing the return
value, and the second one is \RET, which returns execution to the \gls{caller}.

The caller will take the result from the \EAX register.

\subsection{ARM}

There are a few differences on the ARM platform:

\lstinputlisting[caption=\OptimizingKeilVI (\ARMMode) ASM Output,style=customasmARM]{patterns/011_ret/1_Keil_ARM_O3.s}

ARM uses the register \Reg{0} for returning the results of functions, so 123 is copied into \Reg{0}.

\myindex{ARM!\Instructions!MOV}
\myindex{x86!\Instructions!MOV}
It is worth noting that \MOV is a misleading name for the instruction in both the x86 and ARM \ac{ISA}s.

The data is not in fact \IT{moved}, but \IT{copied}.

\subsection{MIPS}

\label{MIPS_leaf_function_ex1}

The GCC assembly output below lists registers by number:

\lstinputlisting[caption=\Optimizing GCC 4.4.5 (\assemblyOutput),style=customasmMIPS]{patterns/011_ret/MIPS.s}

\dots while \IDA does it by their pseudo names:

\lstinputlisting[caption=\Optimizing GCC 4.4.5 (IDA),style=customasmMIPS]{patterns/011_ret/MIPS_IDA.lst}

The \$2 (or \$V0) register is used to store the function's return value.
\myindex{MIPS!\Pseudoinstructions!LI}
\INS{LI} stands for ``Load Immediate'' and is the MIPS equivalent to \MOV.

\myindex{MIPS!\Instructions!J}
The other instruction is the jump instruction (J or JR) which returns the execution flow to the \gls{caller}.

\myindex{MIPS!Branch delay slot}
You might be wondering why the positions of the load instruction (LI) and the jump instruction (J or JR) are swapped. This is due to a \ac{RISC} feature called ``branch delay slot''.

The reason this happens is a quirk in the architecture of some RISC \ac{ISA}s and isn't important for our
purposes---we must simply keep in mind that in MIPS, the instruction following a jump or branch instruction
is executed \IT{before} the jump/branch instruction itself.

As a consequence, branch instructions always swap places with the instruction executed immediately beforehand.


In practice, functions which merely return 1 (\IT{true}) or 0 (\IT{false}) are very frequent.

The smallest ever of the standard UNIX utilities, \IT{/bin/true} and \IT{/bin/false} return 0 and 1 respectively, as an exit code.
(Zero as an exit code usually means success, non-zero means error.)
}
\RU{\subsubsection{std::string}
\myindex{\Cpp!STL!std::string}
\label{std_string}

\myparagraph{Как устроена структура}

Многие строковые библиотеки \InSqBrackets{\CNotes 2.2} обеспечивают структуру содержащую ссылку 
на буфер собственно со строкой, переменная всегда содержащую длину строки 
(что очень удобно для массы функций \InSqBrackets{\CNotes 2.2.1}) и переменную содержащую текущий размер буфера.

Строка в буфере обыкновенно оканчивается нулем: это для того чтобы указатель на буфер можно было
передавать в функции требующие на вход обычную сишную \ac{ASCIIZ}-строку.

Стандарт \Cpp не описывает, как именно нужно реализовывать std::string,
но, как правило, они реализованы как описано выше, с небольшими дополнениями.

Строки в \Cpp это не класс (как, например, QString в Qt), а темплейт (basic\_string), 
это сделано для того чтобы поддерживать 
строки содержащие разного типа символы: как минимум \Tchar и \IT{wchar\_t}.

Так что, std::string это класс с базовым типом \Tchar.

А std::wstring это класс с базовым типом \IT{wchar\_t}.

\mysubparagraph{MSVC}

В реализации MSVC, вместо ссылки на буфер может содержаться сам буфер (если строка короче 16-и символов).

Это означает, что каждая короткая строка будет занимать в памяти по крайней мере $16 + 4 + 4 = 24$ 
байт для 32-битной среды либо $16 + 8 + 8 = 32$ 
байта в 64-битной, а если строка длиннее 16-и символов, то прибавьте еще длину самой строки.

\lstinputlisting[caption=пример для MSVC,style=customc]{\CURPATH/STL/string/MSVC_RU.cpp}

Собственно, из этого исходника почти всё ясно.

Несколько замечаний:

Если строка короче 16-и символов, 
то отдельный буфер для строки в \glslink{heap}{куче} выделяться не будет.

Это удобно потому что на практике, основная часть строк действительно короткие.
Вероятно, разработчики в Microsoft выбрали размер в 16 символов как разумный баланс.

Теперь очень важный момент в конце функции main(): мы не пользуемся методом c\_str(), тем не менее,
если это скомпилировать и запустить, то обе строки появятся в консоли!

Работает это вот почему.

В первом случае строка короче 16-и символов и в начале объекта std::string (его можно рассматривать
просто как структуру) расположен буфер с этой строкой.
\printf трактует указатель как указатель на массив символов оканчивающийся нулем и поэтому всё работает.

Вывод второй строки (длиннее 16-и символов) даже еще опаснее: это вообще типичная программистская ошибка 
(или опечатка), забыть дописать c\_str().
Это работает потому что в это время в начале структуры расположен указатель на буфер.
Это может надолго остаться незамеченным: до тех пока там не появится строка 
короче 16-и символов, тогда процесс упадет.

\mysubparagraph{GCC}

В реализации GCC в структуре есть еще одна переменная --- reference count.

Интересно, что указатель на экземпляр класса std::string в GCC указывает не на начало самой структуры, 
а на указатель на буфера.
В libstdc++-v3\textbackslash{}include\textbackslash{}bits\textbackslash{}basic\_string.h 
мы можем прочитать что это сделано для удобства отладки:

\begin{lstlisting}
   *  The reason you want _M_data pointing to the character %array and
   *  not the _Rep is so that the debugger can see the string
   *  contents. (Probably we should add a non-inline member to get
   *  the _Rep for the debugger to use, so users can check the actual
   *  string length.)
\end{lstlisting}

\href{http://go.yurichev.com/17085}{исходный код basic\_string.h}

В нашем примере мы учитываем это:

\lstinputlisting[caption=пример для GCC,style=customc]{\CURPATH/STL/string/GCC_RU.cpp}

Нужны еще небольшие хаки чтобы сымитировать типичную ошибку, которую мы уже видели выше, из-за
более ужесточенной проверки типов в GCC, тем не менее, printf() работает и здесь без c\_str().

\myparagraph{Чуть более сложный пример}

\lstinputlisting[style=customc]{\CURPATH/STL/string/3.cpp}

\lstinputlisting[caption=MSVC 2012,style=customasmx86]{\CURPATH/STL/string/3_MSVC_RU.asm}

Собственно, компилятор не конструирует строки статически: да в общем-то и как
это возможно, если буфер с ней нужно хранить в \glslink{heap}{куче}?

Вместо этого в сегменте данных хранятся обычные \ac{ASCIIZ}-строки, а позже, во время выполнения, 
при помощи метода \q{assign}, конструируются строки s1 и s2
.
При помощи \TT{operator+}, создается строка s3.

Обратите внимание на то что вызов метода c\_str() отсутствует,
потому что его код достаточно короткий и компилятор вставил его прямо здесь:
если строка короче 16-и байт, то в регистре EAX остается указатель на буфер,
а если длиннее, то из этого же места достается адрес на буфер расположенный в \glslink{heap}{куче}.

Далее следуют вызовы трех деструкторов, причем, они вызываются только если строка длиннее 16-и байт:
тогда нужно освободить буфера в \glslink{heap}{куче}.
В противном случае, так как все три объекта std::string хранятся в стеке,
они освобождаются автоматически после выхода из функции.

Следовательно, работа с короткими строками более быстрая из-за м\'{е}ньшего обращения к \glslink{heap}{куче}.

Код на GCC даже проще (из-за того, что в GCC, как мы уже видели, не реализована возможность хранить короткую
строку прямо в структуре):

% TODO1 comment each function meaning
\lstinputlisting[caption=GCC 4.8.1,style=customasmx86]{\CURPATH/STL/string/3_GCC_RU.s}

Можно заметить, что в деструкторы передается не указатель на объект,
а указатель на место за 12 байт (или 3 слова) перед ним, то есть, на настоящее начало структуры.

\myparagraph{std::string как глобальная переменная}
\label{sec:std_string_as_global_variable}

Опытные программисты на \Cpp знают, что глобальные переменные \ac{STL}-типов вполне можно объявлять.

Да, действительно:

\lstinputlisting[style=customc]{\CURPATH/STL/string/5.cpp}

Но как и где будет вызываться конструктор \TT{std::string}?

На самом деле, эта переменная будет инициализирована даже перед началом \main.

\lstinputlisting[caption=MSVC 2012: здесь конструируется глобальная переменная{,} а также регистрируется её деструктор,style=customasmx86]{\CURPATH/STL/string/5_MSVC_p2.asm}

\lstinputlisting[caption=MSVC 2012: здесь глобальная переменная используется в \main,style=customasmx86]{\CURPATH/STL/string/5_MSVC_p1.asm}

\lstinputlisting[caption=MSVC 2012: эта функция-деструктор вызывается перед выходом,style=customasmx86]{\CURPATH/STL/string/5_MSVC_p3.asm}

\myindex{\CStandardLibrary!atexit()}
В реальности, из \ac{CRT}, еще до вызова main(), вызывается специальная функция,
в которой перечислены все конструкторы подобных переменных.
Более того: при помощи atexit() регистрируется функция, которая будет вызвана в конце работы программы:
в этой функции компилятор собирает вызовы деструкторов всех подобных глобальных переменных.

GCC работает похожим образом:

\lstinputlisting[caption=GCC 4.8.1,style=customasmx86]{\CURPATH/STL/string/5_GCC.s}

Но он не выделяет отдельной функции в которой будут собраны деструкторы: 
каждый деструктор передается в atexit() по одному.

% TODO а если глобальная STL-переменная в другом модуле? надо проверить.

}
\DE{\subsection{Einfachste XOR-Verschlüsselung überhaupt}

Ich habe einmal eine Software gesehen, bei der alle Debugging-Ausgaben mit XOR mit dem Wert 3
verschlüsselt wurden. Mit anderen Worten, die beiden niedrigsten Bits aller Buchstaben wurden invertiert.

``Hello, world'' wurde zu ``Kfool/\#tlqog'':

\begin{lstlisting}
#!/usr/bin/python

msg="Hello, world!"

print "".join(map(lambda x: chr(ord(x)^3), msg))
\end{lstlisting}

Das ist eine ziemlich interessante Verschlüsselung (oder besser eine Verschleierung),
weil sie zwei wichtige Eigenschaften hat:
1) es ist eine einzige Funktion zum Verschlüsseln und entschlüsseln, sie muss nur wiederholt angewendet werden
2) die entstehenden Buchstaben befinden sich im druckbaren Bereich, also die ganze Zeichenkette kann ohne
Escape-Symbole im Code verwendet werden.

Die zweite Eigenschaft nutzt die Tatsache, dass alle druckbaren Zeichen in Reihen organisiert sind: 0x2x-0x7x,
und wenn die beiden niederwertigsten Bits invertiert werden, wird der Buchstabe um eine oder drei Stellen nach
links oder rechts \IT{verschoben}, aber niemals in eine andere Reihe:

\begin{figure}[H]
\centering
\includegraphics[width=0.7\textwidth]{ascii_clean.png}
\caption{7-Bit \ac{ASCII} Tabelle in Emacs}
\end{figure}

\dots mit dem Zeichen 0x7F als einziger Ausnahme.

Im Folgenden werden also beispielsweise die Zeichen A-Z \IT{verschlüsselt}:

\begin{lstlisting}
#!/usr/bin/python

msg="@ABCDEFGHIJKLMNO"

print "".join(map(lambda x: chr(ord(x)^3), msg))
\end{lstlisting}

Ergebnis:
% FIXME \verb  --  relevant comment for German?
\begin{lstlisting}
CBA@GFEDKJIHONML
\end{lstlisting}

Es sieht so aus als würden die Zeichen ``@'' und ``C'' sowie ``B'' und ``A'' vertauscht werden.

Hier ist noch ein interessantes Beispiel, in dem gezeigt wird, wie die Eigenschaften von XOR
ausgenutzt werden können: Exakt den gleichen Effekt, dass druckbare Zeichen auch druckbar bleiben,
kann man dadurch erzielen, dass irgendeine Kombination der niedrigsten vier Bits invertiert wird.
}

\ifdefined\SPANISH
\chapter{Patrones de código}
\fi % SPANISH

\ifdefined\GERMAN
\chapter{Code-Muster}
\fi % GERMAN

\ifdefined\ENGLISH
\chapter{Code Patterns}
\fi % ENGLISH

\ifdefined\ITALIAN
\chapter{Forme di codice}
\fi % ITALIAN

\ifdefined\RUSSIAN
\chapter{Образцы кода}
\fi % RUSSIAN

\ifdefined\BRAZILIAN
\chapter{Padrões de códigos}
\fi % BRAZILIAN

\ifdefined\THAI
\chapter{รูปแบบของโค้ด}
\fi % THAI

\ifdefined\FRENCH
\chapter{Modèle de code}
\fi % FRENCH

\ifdefined\POLISH
\chapter{\PLph{}}
\fi % POLISH

% sections
\EN{\input{patterns/patterns_opt_dbg_EN}}
\ES{\input{patterns/patterns_opt_dbg_ES}}
\ITA{\input{patterns/patterns_opt_dbg_ITA}}
\PTBR{\input{patterns/patterns_opt_dbg_PTBR}}
\RU{\input{patterns/patterns_opt_dbg_RU}}
\THA{\input{patterns/patterns_opt_dbg_THA}}
\DE{\input{patterns/patterns_opt_dbg_DE}}
\FR{\input{patterns/patterns_opt_dbg_FR}}
\PL{\input{patterns/patterns_opt_dbg_PL}}

\RU{\section{Некоторые базовые понятия}}
\EN{\section{Some basics}}
\DE{\section{Einige Grundlagen}}
\FR{\section{Quelques bases}}
\ES{\section{\ESph{}}}
\ITA{\section{Alcune basi teoriche}}
\PTBR{\section{\PTBRph{}}}
\THA{\section{\THAph{}}}
\PL{\section{\PLph{}}}

% sections:
\EN{\input{patterns/intro_CPU_ISA_EN}}
\ES{\input{patterns/intro_CPU_ISA_ES}}
\ITA{\input{patterns/intro_CPU_ISA_ITA}}
\PTBR{\input{patterns/intro_CPU_ISA_PTBR}}
\RU{\input{patterns/intro_CPU_ISA_RU}}
\DE{\input{patterns/intro_CPU_ISA_DE}}
\FR{\input{patterns/intro_CPU_ISA_FR}}
\PL{\input{patterns/intro_CPU_ISA_PL}}

\EN{\input{patterns/numeral_EN}}
\RU{\input{patterns/numeral_RU}}
\ITA{\input{patterns/numeral_ITA}}
\DE{\input{patterns/numeral_DE}}
\FR{\input{patterns/numeral_FR}}
\PL{\input{patterns/numeral_PL}}

% chapters
\input{patterns/00_empty/main}
\input{patterns/011_ret/main}
\input{patterns/01_helloworld/main}
\input{patterns/015_prolog_epilogue/main}
\input{patterns/02_stack/main}
\input{patterns/03_printf/main}
\input{patterns/04_scanf/main}
\input{patterns/05_passing_arguments/main}
\input{patterns/06_return_results/main}
\input{patterns/061_pointers/main}
\input{patterns/065_GOTO/main}
\input{patterns/07_jcc/main}
\input{patterns/08_switch/main}
\input{patterns/09_loops/main}
\input{patterns/10_strings/main}
\input{patterns/11_arith_optimizations/main}
\input{patterns/12_FPU/main}
\input{patterns/13_arrays/main}
\input{patterns/14_bitfields/main}
\EN{\input{patterns/145_LCG/main_EN}}
\RU{\input{patterns/145_LCG/main_RU}}
\input{patterns/15_structs/main}
\input{patterns/17_unions/main}
\input{patterns/18_pointers_to_functions/main}
\input{patterns/185_64bit_in_32_env/main}

\EN{\input{patterns/19_SIMD/main_EN}}
\RU{\input{patterns/19_SIMD/main_RU}}
\DE{\input{patterns/19_SIMD/main_DE}}

\EN{\input{patterns/20_x64/main_EN}}
\RU{\input{patterns/20_x64/main_RU}}

\EN{\input{patterns/205_floating_SIMD/main_EN}}
\RU{\input{patterns/205_floating_SIMD/main_RU}}
\DE{\input{patterns/205_floating_SIMD/main_DE}}

\EN{\input{patterns/ARM/main_EN}}
\RU{\input{patterns/ARM/main_RU}}
\DE{\input{patterns/ARM/main_DE}}

\input{patterns/MIPS/main}


\ifdefined\SPANISH
\chapter{Patrones de código}
\fi % SPANISH

\ifdefined\GERMAN
\chapter{Code-Muster}
\fi % GERMAN

\ifdefined\ENGLISH
\chapter{Code Patterns}
\fi % ENGLISH

\ifdefined\ITALIAN
\chapter{Forme di codice}
\fi % ITALIAN

\ifdefined\RUSSIAN
\chapter{Образцы кода}
\fi % RUSSIAN

\ifdefined\BRAZILIAN
\chapter{Padrões de códigos}
\fi % BRAZILIAN

\ifdefined\THAI
\chapter{รูปแบบของโค้ด}
\fi % THAI

\ifdefined\FRENCH
\chapter{Modèle de code}
\fi % FRENCH

\ifdefined\POLISH
\chapter{\PLph{}}
\fi % POLISH

% sections
\EN{\section{The method}

When the author of this book first started learning C and, later, \Cpp, he used to write small pieces of code, compile them,
and then look at the assembly language output. This made it very easy for him to understand what was going on in the code that he had written.
\footnote{In fact, he still does this when he can't understand what a particular bit of code does.}.
He did this so many times that the relationship between the \CCpp code and what the compiler produced was imprinted deeply in his mind.
It's now easy for him to imagine instantly a rough outline of a C code's appearance and function.
Perhaps this technique could be helpful for others.

%There are a lot of examples for both x86/x64 and ARM.
%Those who already familiar with one of architectures, may freely skim over pages.

By the way, there is a great website where you can do the same, with various compilers, instead of installing them on your box.
You can use it as well: \url{https://gcc.godbolt.org/}.

\section*{\Exercises}

When the author of this book studied assembly language, he also often compiled small C functions and then rewrote
them gradually to assembly, trying to make their code as short as possible.
This probably is not worth doing in real-world scenarios today,
because it's hard to compete with the latest compilers in terms of efficiency. It is, however, a very good way to gain a better understanding of assembly.
Feel free, therefore, to take any assembly code from this book and try to make it shorter.
However, don't forget to test what you have written.

% rewrote to show that debug\release and optimisations levels are orthogonal concepts.
\section*{Optimization levels and debug information}

Source code can be compiled by different compilers with various optimization levels.
A typical compiler has about three such levels, where level zero means that optimization is completely disabled.
Optimization can also be targeted towards code size or code speed.
A non-optimizing compiler is faster and produces more understandable (albeit verbose) code,
whereas an optimizing compiler is slower and tries to produce code that runs faster (but is not necessarily more compact).
In addition to optimization levels, a compiler can include some debug information in the resulting file,
producing code that is easy to debug.
One of the important features of the ´debug' code is that it might contain links
between each line of the source code and its respective machine code address.
Optimizing compilers, on the other hand, tend to produce output where entire lines of source code
can be optimized away and thus not even be present in the resulting machine code.
Reverse engineers can encounter either version, simply because some developers turn on the compiler's optimization flags and others do not.
Because of this, we'll try to work on examples of both debug and release versions of the code featured in this book, wherever possible.

Sometimes some pretty ancient compilers are used in this book, in order to get the shortest (or simplest) possible code snippet.
}
\ES{\input{patterns/patterns_opt_dbg_ES}}
\ITA{\input{patterns/patterns_opt_dbg_ITA}}
\PTBR{\input{patterns/patterns_opt_dbg_PTBR}}
\RU{\input{patterns/patterns_opt_dbg_RU}}
\THA{\input{patterns/patterns_opt_dbg_THA}}
\DE{\input{patterns/patterns_opt_dbg_DE}}
\FR{\input{patterns/patterns_opt_dbg_FR}}
\PL{\input{patterns/patterns_opt_dbg_PL}}

\RU{\section{Некоторые базовые понятия}}
\EN{\section{Some basics}}
\DE{\section{Einige Grundlagen}}
\FR{\section{Quelques bases}}
\ES{\section{\ESph{}}}
\ITA{\section{Alcune basi teoriche}}
\PTBR{\section{\PTBRph{}}}
\THA{\section{\THAph{}}}
\PL{\section{\PLph{}}}

% sections:
\EN{\input{patterns/intro_CPU_ISA_EN}}
\ES{\input{patterns/intro_CPU_ISA_ES}}
\ITA{\input{patterns/intro_CPU_ISA_ITA}}
\PTBR{\input{patterns/intro_CPU_ISA_PTBR}}
\RU{\input{patterns/intro_CPU_ISA_RU}}
\DE{\input{patterns/intro_CPU_ISA_DE}}
\FR{\input{patterns/intro_CPU_ISA_FR}}
\PL{\input{patterns/intro_CPU_ISA_PL}}

\EN{\subsection{Numeral Systems}

Humans have become accustomed to a decimal numeral system, probably because almost everyone has 10 fingers.
Nevertheless, the number \q{10} has no significant meaning in science and mathematics.
The natural numeral system in digital electronics is binary: 0 is for an absence of current in the wire, and 1 for presence.
10 in binary is 2 in decimal, 100 in binary is 4 in decimal, and so on.

% This sentence is a bit unweildy - maybe try 'Our ten-digit system would be described as having a radix...' - Renaissance
If the numeral system has 10 digits, it has a \IT{radix} (or \IT{base}) of 10.
The binary numeral system has a \IT{radix} of 2.

Important things to recall:

1) A \IT{number} is a number, while a \IT{digit} is a term from writing systems, and is usually one character

% The original is 'number' is not changed; I think the intent is value, and changed it - Renaissance
2) The value of a number does not change when converted to another radix; only the writing notation for that value has changed (and therefore the way of representing it in \ac{RAM}).

\subsection{Converting From One Radix To Another}

Positional notation is used almost every numerical system. This means that a digit has weight relative to where it is placed inside of the larger number.
If 2 is placed at the rightmost place, it's 2, but if it's placed one digit before rightmost, it's 20.

What does $1234$ stand for?

$10^3 \cdot 1 + 10^2 \cdot 2 + 10^1 \cdot 3 + 1 \cdot 4 = 1234$ or
$1000 \cdot 1 + 100 \cdot 2 + 10 \cdot 3 + 4 = 1234$

It's the same story for binary numbers, but the base is 2 instead of 10.
What does 0b101011 stand for?

$2^5 \cdot 1 + 2^4 \cdot 0 + 2^3 \cdot 1 + 2^2 \cdot 0 + 2^1 \cdot 1 + 2^0 \cdot 1 = 43$ or
$32 \cdot 1 + 16 \cdot 0 + 8 \cdot 1 + 4 \cdot 0 + 2 \cdot 1 + 1 = 43$

There is such a thing as non-positional notation, such as the Roman numeral system.
\footnote{About numeric system evolution, see \InSqBrackets{\TAOCPvolII{}, 195--213.}}.
% Maybe add a sentence to fill in that X is always 10, and is therefore non-positional, even though putting an I before subtracts and after adds, and is in that sense positional
Perhaps, humankind switched to positional notation because it's easier to do basic operations (addition, multiplication, etc.) on paper by hand.

Binary numbers can be added, subtracted and so on in the very same as taught in schools, but only 2 digits are available.

Binary numbers are bulky when represented in source code and dumps, so that is where the hexadecimal numeral system can be useful.
A hexadecimal radix uses the digits 0..9, and also 6 Latin characters: A..F.
Each hexadecimal digit takes 4 bits or 4 binary digits, so it's very easy to convert from binary number to hexadecimal and back, even manually, in one's mind.

\begin{center}
\begin{longtable}{ | l | l | l | }
\hline
\HeaderColor hexadecimal & \HeaderColor binary & \HeaderColor decimal \\
\hline
0	&0000	&0 \\
1	&0001	&1 \\
2	&0010	&2 \\
3	&0011	&3 \\
4	&0100	&4 \\
5	&0101	&5 \\
6	&0110	&6 \\
7	&0111	&7 \\
8	&1000	&8 \\
9	&1001	&9 \\
A	&1010	&10 \\
B	&1011	&11 \\
C	&1100	&12 \\
D	&1101	&13 \\
E	&1110	&14 \\
F	&1111	&15 \\
\hline
\end{longtable}
\end{center}

How can one tell which radix is being used in a specific instance?

Decimal numbers are usually written as is, i.e., 1234. Some assemblers allow an identifier on decimal radix numbers, in which the number would be written with a "d" suffix: 1234d.

Binary numbers are sometimes prepended with the "0b" prefix: 0b100110111 (\ac{GCC} has a non-standard language extension for this\footnote{\url{https://gcc.gnu.org/onlinedocs/gcc/Binary-constants.html}}).
There is also another way: using a "b" suffix, for example: 100110111b.
This book tries to use the "0b" prefix consistently throughout the book for binary numbers.

Hexadecimal numbers are prepended with "0x" prefix in \CCpp and other \ac{PL}s: 0x1234ABCD.
Alternatively, they are given a "h" suffix: 1234ABCDh. This is common way of representing them in assemblers and debuggers.
In this convention, if the number is started with a Latin (A..F) digit, a 0 is added at the beginning: 0ABCDEFh.
There was also convention that was popular in 8-bit home computers era, using \$ prefix, like \$ABCD.
The book will try to stick to "0x" prefix throughout the book for hexadecimal numbers.

Should one learn to convert numbers mentally? A table of 1-digit hexadecimal numbers can easily be memorized.
As for larger numbers, it's probably not worth tormenting yourself.

Perhaps the most visible hexadecimal numbers are in \ac{URL}s.
This is the way that non-Latin characters are encoded.
For example:
\url{https://en.wiktionary.org/wiki/na\%C3\%AFvet\%C3\%A9} is the \ac{URL} of Wiktionary article about \q{naïveté} word.

\subsubsection{Octal Radix}

Another numeral system heavily used in the past of computer programming is octal. In octal there are 8 digits (0..7), and each is mapped to 3 bits, so it's easy to convert numbers back and forth.
It has been superseded by the hexadecimal system almost everywhere, but, surprisingly, there is a *NIX utility, used often by many people, which takes octal numbers as argument: \TT{chmod}.

\myindex{UNIX!chmod}
As many *NIX users know, \TT{chmod} argument can be a number of 3 digits. The first digit represents the rights of the owner of the file (read, write and/or execute), the second is the rights for the group to which the file belongs, and the third is for everyone else.
Each digit that \TT{chmod} takes can be represented in binary form:

\begin{center}
\begin{longtable}{ | l | l | l | }
\hline
\HeaderColor decimal & \HeaderColor binary & \HeaderColor meaning \\
\hline
7	&111	&\textbf{rwx} \\
6	&110	&\textbf{rw-} \\
5	&101	&\textbf{r-x} \\
4	&100	&\textbf{r-{}-} \\
3	&011	&\textbf{-wx} \\
2	&010	&\textbf{-w-} \\
1	&001	&\textbf{-{}-x} \\
0	&000	&\textbf{-{}-{}-} \\
\hline
\end{longtable}
\end{center}

So each bit is mapped to a flag: read/write/execute.

The importance of \TT{chmod} here is that the whole number in argument can be represented as octal number.
Let's take, for example, 644.
When you run \TT{chmod 644 file}, you set read/write permissions for owner, read permissions for group and again, read permissions for everyone else.
If we convert the octal number 644 to binary, it would be \TT{110100100}, or, in groups of 3 bits, \TT{110 100 100}.

Now we see that each triplet describe permissions for owner/group/others: first is \TT{rw-}, second is \TT{r--} and third is \TT{r--}.

The octal numeral system was also popular on old computers like PDP-8, because word there could be 12, 24 or 36 bits, and these numbers are all divisible by 3, so the octal system was natural in that environment.
Nowadays, all popular computers employ word/address sizes of 16, 32 or 64 bits, and these numbers are all divisible by 4, so the hexadecimal system is more natural there.

The octal numeral system is supported by all standard \CCpp compilers.
This is a source of confusion sometimes, because octal numbers are encoded with a zero prepended, for example, 0377 is 255.
Sometimes, you might make a typo and write "09" instead of 9, and the compiler would report an error.
GCC might report something like this:\\
\TT{error: invalid digit "9" in octal constant}.

Also, the octal system is somewhat popular in Java. When the IDA shows Java strings with non-printable characters,
they are encoded in the octal system instead of hexadecimal.
\myindex{JAD}
The JAD Java decompiler behaves the same way.

\subsubsection{Divisibility}

When you see a decimal number like 120, you can quickly deduce that it's divisible by 10, because the last digit is zero.
In the same way, 123400 is divisible by 100, because the two last digits are zeros.

Likewise, the hexadecimal number 0x1230 is divisible by 0x10 (or 16), 0x123000 is divisible by 0x1000 (or 4096), etc.

The binary number 0b1000101000 is divisible by 0b1000 (8), etc.

This property can often be used to quickly realize if the size of some block in memory is padded to some boundary.
For example, sections in \ac{PE} files are almost always started at addresses ending with 3 hexadecimal zeros: 0x41000, 0x10001000, etc.
The reason behind this is the fact that almost all \ac{PE} sections are padded to a boundary of 0x1000 (4096) bytes.

\subsubsection{Multi-Precision Arithmetic and Radix}

\index{RSA}
Multi-precision arithmetic can use huge numbers, and each one may be stored in several bytes.
For example, RSA keys, both public and private, span up to 4096 bits, and maybe even more.

% I'm not sure how to change this, but the normal format for quoting would be just to mention the author or book, and footnote to the full reference
In \InSqBrackets{\TAOCPvolII, 265} we find the following idea: when you store a multi-precision number in several bytes,
the whole number can be represented as having a radix of $2^8=256$, and each digit goes to the corresponding byte.
Likewise, if you store a multi-precision number in several 32-bit integer values, each digit goes to each 32-bit slot,
and you may think about this number as stored in radix of $2^{32}$.

\subsubsection{How to Pronounce Non-Decimal Numbers}

Numbers in a non-decimal base are usually pronounced by digit by digit: ``one-zero-zero-one-one-...''.
Words like ``ten'' and ``thousand'' are usually not pronounced, to prevent confusion with the decimal base system.

\subsubsection{Floating point numbers}

To distinguish floating point numbers from integers, they are usually written with ``.0'' at the end,
like $0.0$, $123.0$, etc.
}
\RU{\subsection{Представление чисел}

Люди привыкли к десятичной системе счисления вероятно потому что почти у каждого есть по 10 пальцев.
Тем не менее, число 10 не имеет особого значения в науке и математике.
Двоичная система естествена для цифровой электроники: 0 означает отсутствие тока в проводе и 1 --- его присутствие.
10 в двоичной системе это 2 в десятичной; 100 в двоичной это 4 в десятичной, итд.

Если в системе счисления есть 10 цифр, её \IT{основание} или \IT{radix} это 10.
Двоичная система имеет \IT{основание} 2.

Важные вещи, которые полезно вспомнить:
1) \IT{число} это число, в то время как \IT{цифра} это термин из системы письменности, и это обычно один символ;
2) само число не меняется, когда конвертируется из одного основания в другое: меняется способ его записи (или представления
в памяти).

Как сконвертировать число из одного основания в другое?

Позиционная нотация используется почти везде, это означает, что всякая цифра имеет свой вес, в зависимости от её расположения
внутри числа.
Если 2 расположена в самом последнем месте справа, это 2.
Если она расположена в месте перед последним, это 20.

Что означает $1234$?

$10^3 \cdot 1 + 10^2 \cdot 2 + 10^1 \cdot 3 + 1 \cdot 4$ = 1234 или
$1000 \cdot 1 + 100 \cdot 2 + 10 \cdot 3 + 4 = 1234$

Та же история и для двоичных чисел, только основание там 2 вместо 10.
Что означает 0b101011?

$2^5 \cdot 1 + 2^4 \cdot 0 + 2^3 \cdot 1 + 2^2 \cdot 0 + 2^1 \cdot 1 + 2^0 \cdot 1 = 43$ или
$32 \cdot 1 + 16 \cdot 0 + 8 \cdot 1 + 4 \cdot 0 + 2 \cdot 1 + 1 = 43$

Позиционную нотацию можно противопоставить непозиционной нотации, такой как римская система записи чисел
\footnote{Об эволюции способов записи чисел, см.также: \InSqBrackets{\TAOCPvolII{}, 195--213.}}.
Вероятно, человечество перешло на позиционную нотацию, потому что так проще работать с числами (сложение, умножение, итд)
на бумаге, в ручную.

Действительно, двоичные числа можно складывать, вычитать, итд, точно также, как этому обычно обучают в школах,
только доступны лишь 2 цифры.

Двоичные числа громоздки, когда их используют в исходных кодах и дампах, так что в этих случаях применяется шестнадцатеричная
система.
Используются цифры 0..9 и еще 6 латинских букв: A..F.
Каждая шестнадцатеричная цифра занимает 4 бита или 4 двоичных цифры, так что конвертировать из двоичной системы в
шестнадцатеричную и назад, можно легко вручную, или даже в уме.

\begin{center}
\begin{longtable}{ | l | l | l | }
\hline
\HeaderColor шестнадцатеричная & \HeaderColor двоичная & \HeaderColor десятичная \\
\hline
0	&0000	&0 \\
1	&0001	&1 \\
2	&0010	&2 \\
3	&0011	&3 \\
4	&0100	&4 \\
5	&0101	&5 \\
6	&0110	&6 \\
7	&0111	&7 \\
8	&1000	&8 \\
9	&1001	&9 \\
A	&1010	&10 \\
B	&1011	&11 \\
C	&1100	&12 \\
D	&1101	&13 \\
E	&1110	&14 \\
F	&1111	&15 \\
\hline
\end{longtable}
\end{center}

Как понять, какое основание используется в конкретном месте?

Десятичные числа обычно записываются как есть, т.е., 1234. Но некоторые ассемблеры позволяют подчеркивать
этот факт для ясности, и это число может быть дополнено суффиксом "d": 1234d.

К двоичным числам иногда спереди добавляют префикс "0b": 0b100110111
(В \ac{GCC} для этого есть нестандартное расширение языка
\footnote{\url{https://gcc.gnu.org/onlinedocs/gcc/Binary-constants.html}}).
Есть также еще один способ: суффикс "b", например: 100110111b.
В этой книге я буду пытаться придерживаться префикса "0b" для двоичных чисел.

Шестнадцатеричные числа имеют префикс "0x" в \CCpp и некоторых других \ac{PL}: 0x1234ABCD.
Либо они имеют суффикс "h": 1234ABCDh --- обычно так они представляются в ассемблерах и отладчиках.
Если число начинается с цифры A..F, перед ним добавляется 0: 0ABCDEFh.
Во времена 8-битных домашних компьютеров, был также способ записи чисел используя префикс \$, например, \$ABCD.
В книге я попытаюсь придерживаться префикса "0x" для шестнадцатеричных чисел.

Нужно ли учиться конвертировать числа в уме? Таблицу шестнадцатеричных чисел из одной цифры легко запомнить.
А запоминать б\'{о}льшие числа, наверное, не стоит.

Наверное, чаще всего шестнадцатеричные числа можно увидеть в \ac{URL}-ах.
Так кодируются буквы не из числа латинских.
Например:
\url{https://en.wiktionary.org/wiki/na\%C3\%AFvet\%C3\%A9} это \ac{URL} страницы в Wiktionary о слове \q{naïveté}.

\subsubsection{Восьмеричная система}

Еще одна система, которая в прошлом много использовалась в программировании это восьмеричная: есть 8 цифр (0..7) и каждая
описывает 3 бита, так что легко конвертировать числа туда и назад.
Она почти везде была заменена шестнадцатеричной, но удивительно, в *NIX имеется утилита использующаяся многими людьми,
которая принимает на вход восьмеричное число: \TT{chmod}.

\myindex{UNIX!chmod}
Как знают многие пользователи *NIX, аргумент \TT{chmod} это число из трех цифр. Первая цифра это права владельца файла,
вторая это права группы (которой файл принадлежит), третья для всех остальных.
И каждая цифра может быть представлена в двоичном виде:

\begin{center}
\begin{longtable}{ | l | l | l | }
\hline
\HeaderColor десятичная & \HeaderColor двоичная & \HeaderColor значение \\
\hline
7	&111	&\textbf{rwx} \\
6	&110	&\textbf{rw-} \\
5	&101	&\textbf{r-x} \\
4	&100	&\textbf{r-{}-} \\
3	&011	&\textbf{-wx} \\
2	&010	&\textbf{-w-} \\
1	&001	&\textbf{-{}-x} \\
0	&000	&\textbf{-{}-{}-} \\
\hline
\end{longtable}
\end{center}

Так что каждый бит привязан к флагу: read/write/execute (чтение/запись/исполнение).

И вот почему я вспомнил здесь о \TT{chmod}, это потому что всё число может быть представлено как число в восьмеричной системе.
Для примера возьмем 644.
Когда вы запускаете \TT{chmod 644 file}, вы выставляете права read/write для владельца, права read для группы, и снова,
read для всех остальных.
Сконвертируем число 644 из восьмеричной системы в двоичную, это будет \TT{110100100}, или (в группах по 3 бита) \TT{110 100 100}.

Теперь мы видим, что каждая тройка описывает права для владельца/группы/остальных:
первая это \TT{rw-}, вторая это \TT{r--} и третья это \TT{r--}.

Восьмеричная система была также популярная на старых компьютерах вроде PDP-8, потому что слово там могло содержать 12, 24 или
36 бит, и эти числа делятся на 3, так что выбор восьмеричной системы в той среде был логичен.
Сейчас, все популярные компьютеры имеют размер слова/адреса 16, 32 или 64 бита, и эти числа делятся на 4,
так что шестнадцатеричная система здесь удобнее.

Восьмеричная система поддерживается всеми стандартными компиляторами \CCpp{}.
Это иногда источник недоумения, потому что восьмеричные числа кодируются с нулем вперед, например, 0377 это 255.
И иногда, вы можете сделать опечатку, и написать "09" вместо 9, и компилятор выдаст ошибку.
GCC может выдать что-то вроде:\\
\TT{error: invalid digit "9" in octal constant}.

Также, восьмеричная система популярна в Java: когда IDA показывает строку с непечатаемыми символами,
они кодируются в восьмеричной системе вместо шестнадцатеричной.
\myindex{JAD}
Точно также себя ведет декомпилятор с Java JAD.

\subsubsection{Делимость}

Когда вы видите десятичное число вроде 120, вы можете быстро понять что оно делится на 10, потому что последняя цифра это 0.
Точно также, 123400 делится на 100, потому что две последних цифры это нули.

Точно также, шестнадцатеричное число 0x1230 делится на 0x10 (или 16), 0x123000 делится на 0x1000 (или 4096), итд.

Двоичное число 0b1000101000 делится на 0b1000 (8), итд.

Это свойство можно часто использовать, чтобы быстро понять,
что длина какого-либо блока в памяти выровнена по некоторой границе.
Например, секции в \ac{PE}-файлах почти всегда начинаются с адресов заканчивающихся 3 шестнадцатеричными нулями:
0x41000, 0x10001000, итд.
Причина в том, что почти все секции в \ac{PE} выровнены по границе 0x1000 (4096) байт.

\subsubsection{Арифметика произвольной точности и основание}

\index{RSA}
Арифметика произвольной точности (multi-precision arithmetic) может использовать огромные числа,
которые могут храниться в нескольких байтах.
Например, ключи RSA, и открытые и закрытые, могут занимать до 4096 бит и даже больше.

В \InSqBrackets{\TAOCPvolII, 265} можно найти такую идею: когда вы сохраняете число произвольной точности в нескольких байтах,
всё число может быть представлено как имеющую систему счисления по основанию $2^8=256$, и каждая цифра находится
в соответствующем байте.
Точно также, если вы сохраняете число произвольной точности в нескольких 32-битных целочисленных значениях,
каждая цифра отправляется в каждый 32-битный слот, и вы можете считать что это число записано в системе с основанием $2^{32}$.

\subsubsection{Произношение}

Числа в недесятичных системах счислениях обычно произносятся по одной цифре: ``один-ноль-ноль-один-один-...''.
Слова вроде ``десять'', ``тысяча'', итд, обычно не произносятся, потому что тогда можно спутать с десятичной системой.

\subsubsection{Числа с плавающей запятой}

Чтобы отличать числа с плавающей запятой от целочисленных, часто, в конце добавляют ``.0'',
например $0.0$, $123.0$, итд.

}
\ITA{\input{patterns/numeral_ITA}}
\DE{\input{patterns/numeral_DE}}
\FR{\input{patterns/numeral_FR}}
\PL{\input{patterns/numeral_PL}}

% chapters
\ifdefined\SPANISH
\chapter{Patrones de código}
\fi % SPANISH

\ifdefined\GERMAN
\chapter{Code-Muster}
\fi % GERMAN

\ifdefined\ENGLISH
\chapter{Code Patterns}
\fi % ENGLISH

\ifdefined\ITALIAN
\chapter{Forme di codice}
\fi % ITALIAN

\ifdefined\RUSSIAN
\chapter{Образцы кода}
\fi % RUSSIAN

\ifdefined\BRAZILIAN
\chapter{Padrões de códigos}
\fi % BRAZILIAN

\ifdefined\THAI
\chapter{รูปแบบของโค้ด}
\fi % THAI

\ifdefined\FRENCH
\chapter{Modèle de code}
\fi % FRENCH

\ifdefined\POLISH
\chapter{\PLph{}}
\fi % POLISH

% sections
\EN{\input{patterns/patterns_opt_dbg_EN}}
\ES{\input{patterns/patterns_opt_dbg_ES}}
\ITA{\input{patterns/patterns_opt_dbg_ITA}}
\PTBR{\input{patterns/patterns_opt_dbg_PTBR}}
\RU{\input{patterns/patterns_opt_dbg_RU}}
\THA{\input{patterns/patterns_opt_dbg_THA}}
\DE{\input{patterns/patterns_opt_dbg_DE}}
\FR{\input{patterns/patterns_opt_dbg_FR}}
\PL{\input{patterns/patterns_opt_dbg_PL}}

\RU{\section{Некоторые базовые понятия}}
\EN{\section{Some basics}}
\DE{\section{Einige Grundlagen}}
\FR{\section{Quelques bases}}
\ES{\section{\ESph{}}}
\ITA{\section{Alcune basi teoriche}}
\PTBR{\section{\PTBRph{}}}
\THA{\section{\THAph{}}}
\PL{\section{\PLph{}}}

% sections:
\EN{\input{patterns/intro_CPU_ISA_EN}}
\ES{\input{patterns/intro_CPU_ISA_ES}}
\ITA{\input{patterns/intro_CPU_ISA_ITA}}
\PTBR{\input{patterns/intro_CPU_ISA_PTBR}}
\RU{\input{patterns/intro_CPU_ISA_RU}}
\DE{\input{patterns/intro_CPU_ISA_DE}}
\FR{\input{patterns/intro_CPU_ISA_FR}}
\PL{\input{patterns/intro_CPU_ISA_PL}}

\EN{\input{patterns/numeral_EN}}
\RU{\input{patterns/numeral_RU}}
\ITA{\input{patterns/numeral_ITA}}
\DE{\input{patterns/numeral_DE}}
\FR{\input{patterns/numeral_FR}}
\PL{\input{patterns/numeral_PL}}

% chapters
\input{patterns/00_empty/main}
\input{patterns/011_ret/main}
\input{patterns/01_helloworld/main}
\input{patterns/015_prolog_epilogue/main}
\input{patterns/02_stack/main}
\input{patterns/03_printf/main}
\input{patterns/04_scanf/main}
\input{patterns/05_passing_arguments/main}
\input{patterns/06_return_results/main}
\input{patterns/061_pointers/main}
\input{patterns/065_GOTO/main}
\input{patterns/07_jcc/main}
\input{patterns/08_switch/main}
\input{patterns/09_loops/main}
\input{patterns/10_strings/main}
\input{patterns/11_arith_optimizations/main}
\input{patterns/12_FPU/main}
\input{patterns/13_arrays/main}
\input{patterns/14_bitfields/main}
\EN{\input{patterns/145_LCG/main_EN}}
\RU{\input{patterns/145_LCG/main_RU}}
\input{patterns/15_structs/main}
\input{patterns/17_unions/main}
\input{patterns/18_pointers_to_functions/main}
\input{patterns/185_64bit_in_32_env/main}

\EN{\input{patterns/19_SIMD/main_EN}}
\RU{\input{patterns/19_SIMD/main_RU}}
\DE{\input{patterns/19_SIMD/main_DE}}

\EN{\input{patterns/20_x64/main_EN}}
\RU{\input{patterns/20_x64/main_RU}}

\EN{\input{patterns/205_floating_SIMD/main_EN}}
\RU{\input{patterns/205_floating_SIMD/main_RU}}
\DE{\input{patterns/205_floating_SIMD/main_DE}}

\EN{\input{patterns/ARM/main_EN}}
\RU{\input{patterns/ARM/main_RU}}
\DE{\input{patterns/ARM/main_DE}}

\input{patterns/MIPS/main}

\ifdefined\SPANISH
\chapter{Patrones de código}
\fi % SPANISH

\ifdefined\GERMAN
\chapter{Code-Muster}
\fi % GERMAN

\ifdefined\ENGLISH
\chapter{Code Patterns}
\fi % ENGLISH

\ifdefined\ITALIAN
\chapter{Forme di codice}
\fi % ITALIAN

\ifdefined\RUSSIAN
\chapter{Образцы кода}
\fi % RUSSIAN

\ifdefined\BRAZILIAN
\chapter{Padrões de códigos}
\fi % BRAZILIAN

\ifdefined\THAI
\chapter{รูปแบบของโค้ด}
\fi % THAI

\ifdefined\FRENCH
\chapter{Modèle de code}
\fi % FRENCH

\ifdefined\POLISH
\chapter{\PLph{}}
\fi % POLISH

% sections
\EN{\input{patterns/patterns_opt_dbg_EN}}
\ES{\input{patterns/patterns_opt_dbg_ES}}
\ITA{\input{patterns/patterns_opt_dbg_ITA}}
\PTBR{\input{patterns/patterns_opt_dbg_PTBR}}
\RU{\input{patterns/patterns_opt_dbg_RU}}
\THA{\input{patterns/patterns_opt_dbg_THA}}
\DE{\input{patterns/patterns_opt_dbg_DE}}
\FR{\input{patterns/patterns_opt_dbg_FR}}
\PL{\input{patterns/patterns_opt_dbg_PL}}

\RU{\section{Некоторые базовые понятия}}
\EN{\section{Some basics}}
\DE{\section{Einige Grundlagen}}
\FR{\section{Quelques bases}}
\ES{\section{\ESph{}}}
\ITA{\section{Alcune basi teoriche}}
\PTBR{\section{\PTBRph{}}}
\THA{\section{\THAph{}}}
\PL{\section{\PLph{}}}

% sections:
\EN{\input{patterns/intro_CPU_ISA_EN}}
\ES{\input{patterns/intro_CPU_ISA_ES}}
\ITA{\input{patterns/intro_CPU_ISA_ITA}}
\PTBR{\input{patterns/intro_CPU_ISA_PTBR}}
\RU{\input{patterns/intro_CPU_ISA_RU}}
\DE{\input{patterns/intro_CPU_ISA_DE}}
\FR{\input{patterns/intro_CPU_ISA_FR}}
\PL{\input{patterns/intro_CPU_ISA_PL}}

\EN{\input{patterns/numeral_EN}}
\RU{\input{patterns/numeral_RU}}
\ITA{\input{patterns/numeral_ITA}}
\DE{\input{patterns/numeral_DE}}
\FR{\input{patterns/numeral_FR}}
\PL{\input{patterns/numeral_PL}}

% chapters
\input{patterns/00_empty/main}
\input{patterns/011_ret/main}
\input{patterns/01_helloworld/main}
\input{patterns/015_prolog_epilogue/main}
\input{patterns/02_stack/main}
\input{patterns/03_printf/main}
\input{patterns/04_scanf/main}
\input{patterns/05_passing_arguments/main}
\input{patterns/06_return_results/main}
\input{patterns/061_pointers/main}
\input{patterns/065_GOTO/main}
\input{patterns/07_jcc/main}
\input{patterns/08_switch/main}
\input{patterns/09_loops/main}
\input{patterns/10_strings/main}
\input{patterns/11_arith_optimizations/main}
\input{patterns/12_FPU/main}
\input{patterns/13_arrays/main}
\input{patterns/14_bitfields/main}
\EN{\input{patterns/145_LCG/main_EN}}
\RU{\input{patterns/145_LCG/main_RU}}
\input{patterns/15_structs/main}
\input{patterns/17_unions/main}
\input{patterns/18_pointers_to_functions/main}
\input{patterns/185_64bit_in_32_env/main}

\EN{\input{patterns/19_SIMD/main_EN}}
\RU{\input{patterns/19_SIMD/main_RU}}
\DE{\input{patterns/19_SIMD/main_DE}}

\EN{\input{patterns/20_x64/main_EN}}
\RU{\input{patterns/20_x64/main_RU}}

\EN{\input{patterns/205_floating_SIMD/main_EN}}
\RU{\input{patterns/205_floating_SIMD/main_RU}}
\DE{\input{patterns/205_floating_SIMD/main_DE}}

\EN{\input{patterns/ARM/main_EN}}
\RU{\input{patterns/ARM/main_RU}}
\DE{\input{patterns/ARM/main_DE}}

\input{patterns/MIPS/main}

\ifdefined\SPANISH
\chapter{Patrones de código}
\fi % SPANISH

\ifdefined\GERMAN
\chapter{Code-Muster}
\fi % GERMAN

\ifdefined\ENGLISH
\chapter{Code Patterns}
\fi % ENGLISH

\ifdefined\ITALIAN
\chapter{Forme di codice}
\fi % ITALIAN

\ifdefined\RUSSIAN
\chapter{Образцы кода}
\fi % RUSSIAN

\ifdefined\BRAZILIAN
\chapter{Padrões de códigos}
\fi % BRAZILIAN

\ifdefined\THAI
\chapter{รูปแบบของโค้ด}
\fi % THAI

\ifdefined\FRENCH
\chapter{Modèle de code}
\fi % FRENCH

\ifdefined\POLISH
\chapter{\PLph{}}
\fi % POLISH

% sections
\EN{\input{patterns/patterns_opt_dbg_EN}}
\ES{\input{patterns/patterns_opt_dbg_ES}}
\ITA{\input{patterns/patterns_opt_dbg_ITA}}
\PTBR{\input{patterns/patterns_opt_dbg_PTBR}}
\RU{\input{patterns/patterns_opt_dbg_RU}}
\THA{\input{patterns/patterns_opt_dbg_THA}}
\DE{\input{patterns/patterns_opt_dbg_DE}}
\FR{\input{patterns/patterns_opt_dbg_FR}}
\PL{\input{patterns/patterns_opt_dbg_PL}}

\RU{\section{Некоторые базовые понятия}}
\EN{\section{Some basics}}
\DE{\section{Einige Grundlagen}}
\FR{\section{Quelques bases}}
\ES{\section{\ESph{}}}
\ITA{\section{Alcune basi teoriche}}
\PTBR{\section{\PTBRph{}}}
\THA{\section{\THAph{}}}
\PL{\section{\PLph{}}}

% sections:
\EN{\input{patterns/intro_CPU_ISA_EN}}
\ES{\input{patterns/intro_CPU_ISA_ES}}
\ITA{\input{patterns/intro_CPU_ISA_ITA}}
\PTBR{\input{patterns/intro_CPU_ISA_PTBR}}
\RU{\input{patterns/intro_CPU_ISA_RU}}
\DE{\input{patterns/intro_CPU_ISA_DE}}
\FR{\input{patterns/intro_CPU_ISA_FR}}
\PL{\input{patterns/intro_CPU_ISA_PL}}

\EN{\input{patterns/numeral_EN}}
\RU{\input{patterns/numeral_RU}}
\ITA{\input{patterns/numeral_ITA}}
\DE{\input{patterns/numeral_DE}}
\FR{\input{patterns/numeral_FR}}
\PL{\input{patterns/numeral_PL}}

% chapters
\input{patterns/00_empty/main}
\input{patterns/011_ret/main}
\input{patterns/01_helloworld/main}
\input{patterns/015_prolog_epilogue/main}
\input{patterns/02_stack/main}
\input{patterns/03_printf/main}
\input{patterns/04_scanf/main}
\input{patterns/05_passing_arguments/main}
\input{patterns/06_return_results/main}
\input{patterns/061_pointers/main}
\input{patterns/065_GOTO/main}
\input{patterns/07_jcc/main}
\input{patterns/08_switch/main}
\input{patterns/09_loops/main}
\input{patterns/10_strings/main}
\input{patterns/11_arith_optimizations/main}
\input{patterns/12_FPU/main}
\input{patterns/13_arrays/main}
\input{patterns/14_bitfields/main}
\EN{\input{patterns/145_LCG/main_EN}}
\RU{\input{patterns/145_LCG/main_RU}}
\input{patterns/15_structs/main}
\input{patterns/17_unions/main}
\input{patterns/18_pointers_to_functions/main}
\input{patterns/185_64bit_in_32_env/main}

\EN{\input{patterns/19_SIMD/main_EN}}
\RU{\input{patterns/19_SIMD/main_RU}}
\DE{\input{patterns/19_SIMD/main_DE}}

\EN{\input{patterns/20_x64/main_EN}}
\RU{\input{patterns/20_x64/main_RU}}

\EN{\input{patterns/205_floating_SIMD/main_EN}}
\RU{\input{patterns/205_floating_SIMD/main_RU}}
\DE{\input{patterns/205_floating_SIMD/main_DE}}

\EN{\input{patterns/ARM/main_EN}}
\RU{\input{patterns/ARM/main_RU}}
\DE{\input{patterns/ARM/main_DE}}

\input{patterns/MIPS/main}

\ifdefined\SPANISH
\chapter{Patrones de código}
\fi % SPANISH

\ifdefined\GERMAN
\chapter{Code-Muster}
\fi % GERMAN

\ifdefined\ENGLISH
\chapter{Code Patterns}
\fi % ENGLISH

\ifdefined\ITALIAN
\chapter{Forme di codice}
\fi % ITALIAN

\ifdefined\RUSSIAN
\chapter{Образцы кода}
\fi % RUSSIAN

\ifdefined\BRAZILIAN
\chapter{Padrões de códigos}
\fi % BRAZILIAN

\ifdefined\THAI
\chapter{รูปแบบของโค้ด}
\fi % THAI

\ifdefined\FRENCH
\chapter{Modèle de code}
\fi % FRENCH

\ifdefined\POLISH
\chapter{\PLph{}}
\fi % POLISH

% sections
\EN{\input{patterns/patterns_opt_dbg_EN}}
\ES{\input{patterns/patterns_opt_dbg_ES}}
\ITA{\input{patterns/patterns_opt_dbg_ITA}}
\PTBR{\input{patterns/patterns_opt_dbg_PTBR}}
\RU{\input{patterns/patterns_opt_dbg_RU}}
\THA{\input{patterns/patterns_opt_dbg_THA}}
\DE{\input{patterns/patterns_opt_dbg_DE}}
\FR{\input{patterns/patterns_opt_dbg_FR}}
\PL{\input{patterns/patterns_opt_dbg_PL}}

\RU{\section{Некоторые базовые понятия}}
\EN{\section{Some basics}}
\DE{\section{Einige Grundlagen}}
\FR{\section{Quelques bases}}
\ES{\section{\ESph{}}}
\ITA{\section{Alcune basi teoriche}}
\PTBR{\section{\PTBRph{}}}
\THA{\section{\THAph{}}}
\PL{\section{\PLph{}}}

% sections:
\EN{\input{patterns/intro_CPU_ISA_EN}}
\ES{\input{patterns/intro_CPU_ISA_ES}}
\ITA{\input{patterns/intro_CPU_ISA_ITA}}
\PTBR{\input{patterns/intro_CPU_ISA_PTBR}}
\RU{\input{patterns/intro_CPU_ISA_RU}}
\DE{\input{patterns/intro_CPU_ISA_DE}}
\FR{\input{patterns/intro_CPU_ISA_FR}}
\PL{\input{patterns/intro_CPU_ISA_PL}}

\EN{\input{patterns/numeral_EN}}
\RU{\input{patterns/numeral_RU}}
\ITA{\input{patterns/numeral_ITA}}
\DE{\input{patterns/numeral_DE}}
\FR{\input{patterns/numeral_FR}}
\PL{\input{patterns/numeral_PL}}

% chapters
\input{patterns/00_empty/main}
\input{patterns/011_ret/main}
\input{patterns/01_helloworld/main}
\input{patterns/015_prolog_epilogue/main}
\input{patterns/02_stack/main}
\input{patterns/03_printf/main}
\input{patterns/04_scanf/main}
\input{patterns/05_passing_arguments/main}
\input{patterns/06_return_results/main}
\input{patterns/061_pointers/main}
\input{patterns/065_GOTO/main}
\input{patterns/07_jcc/main}
\input{patterns/08_switch/main}
\input{patterns/09_loops/main}
\input{patterns/10_strings/main}
\input{patterns/11_arith_optimizations/main}
\input{patterns/12_FPU/main}
\input{patterns/13_arrays/main}
\input{patterns/14_bitfields/main}
\EN{\input{patterns/145_LCG/main_EN}}
\RU{\input{patterns/145_LCG/main_RU}}
\input{patterns/15_structs/main}
\input{patterns/17_unions/main}
\input{patterns/18_pointers_to_functions/main}
\input{patterns/185_64bit_in_32_env/main}

\EN{\input{patterns/19_SIMD/main_EN}}
\RU{\input{patterns/19_SIMD/main_RU}}
\DE{\input{patterns/19_SIMD/main_DE}}

\EN{\input{patterns/20_x64/main_EN}}
\RU{\input{patterns/20_x64/main_RU}}

\EN{\input{patterns/205_floating_SIMD/main_EN}}
\RU{\input{patterns/205_floating_SIMD/main_RU}}
\DE{\input{patterns/205_floating_SIMD/main_DE}}

\EN{\input{patterns/ARM/main_EN}}
\RU{\input{patterns/ARM/main_RU}}
\DE{\input{patterns/ARM/main_DE}}

\input{patterns/MIPS/main}

\ifdefined\SPANISH
\chapter{Patrones de código}
\fi % SPANISH

\ifdefined\GERMAN
\chapter{Code-Muster}
\fi % GERMAN

\ifdefined\ENGLISH
\chapter{Code Patterns}
\fi % ENGLISH

\ifdefined\ITALIAN
\chapter{Forme di codice}
\fi % ITALIAN

\ifdefined\RUSSIAN
\chapter{Образцы кода}
\fi % RUSSIAN

\ifdefined\BRAZILIAN
\chapter{Padrões de códigos}
\fi % BRAZILIAN

\ifdefined\THAI
\chapter{รูปแบบของโค้ด}
\fi % THAI

\ifdefined\FRENCH
\chapter{Modèle de code}
\fi % FRENCH

\ifdefined\POLISH
\chapter{\PLph{}}
\fi % POLISH

% sections
\EN{\input{patterns/patterns_opt_dbg_EN}}
\ES{\input{patterns/patterns_opt_dbg_ES}}
\ITA{\input{patterns/patterns_opt_dbg_ITA}}
\PTBR{\input{patterns/patterns_opt_dbg_PTBR}}
\RU{\input{patterns/patterns_opt_dbg_RU}}
\THA{\input{patterns/patterns_opt_dbg_THA}}
\DE{\input{patterns/patterns_opt_dbg_DE}}
\FR{\input{patterns/patterns_opt_dbg_FR}}
\PL{\input{patterns/patterns_opt_dbg_PL}}

\RU{\section{Некоторые базовые понятия}}
\EN{\section{Some basics}}
\DE{\section{Einige Grundlagen}}
\FR{\section{Quelques bases}}
\ES{\section{\ESph{}}}
\ITA{\section{Alcune basi teoriche}}
\PTBR{\section{\PTBRph{}}}
\THA{\section{\THAph{}}}
\PL{\section{\PLph{}}}

% sections:
\EN{\input{patterns/intro_CPU_ISA_EN}}
\ES{\input{patterns/intro_CPU_ISA_ES}}
\ITA{\input{patterns/intro_CPU_ISA_ITA}}
\PTBR{\input{patterns/intro_CPU_ISA_PTBR}}
\RU{\input{patterns/intro_CPU_ISA_RU}}
\DE{\input{patterns/intro_CPU_ISA_DE}}
\FR{\input{patterns/intro_CPU_ISA_FR}}
\PL{\input{patterns/intro_CPU_ISA_PL}}

\EN{\input{patterns/numeral_EN}}
\RU{\input{patterns/numeral_RU}}
\ITA{\input{patterns/numeral_ITA}}
\DE{\input{patterns/numeral_DE}}
\FR{\input{patterns/numeral_FR}}
\PL{\input{patterns/numeral_PL}}

% chapters
\input{patterns/00_empty/main}
\input{patterns/011_ret/main}
\input{patterns/01_helloworld/main}
\input{patterns/015_prolog_epilogue/main}
\input{patterns/02_stack/main}
\input{patterns/03_printf/main}
\input{patterns/04_scanf/main}
\input{patterns/05_passing_arguments/main}
\input{patterns/06_return_results/main}
\input{patterns/061_pointers/main}
\input{patterns/065_GOTO/main}
\input{patterns/07_jcc/main}
\input{patterns/08_switch/main}
\input{patterns/09_loops/main}
\input{patterns/10_strings/main}
\input{patterns/11_arith_optimizations/main}
\input{patterns/12_FPU/main}
\input{patterns/13_arrays/main}
\input{patterns/14_bitfields/main}
\EN{\input{patterns/145_LCG/main_EN}}
\RU{\input{patterns/145_LCG/main_RU}}
\input{patterns/15_structs/main}
\input{patterns/17_unions/main}
\input{patterns/18_pointers_to_functions/main}
\input{patterns/185_64bit_in_32_env/main}

\EN{\input{patterns/19_SIMD/main_EN}}
\RU{\input{patterns/19_SIMD/main_RU}}
\DE{\input{patterns/19_SIMD/main_DE}}

\EN{\input{patterns/20_x64/main_EN}}
\RU{\input{patterns/20_x64/main_RU}}

\EN{\input{patterns/205_floating_SIMD/main_EN}}
\RU{\input{patterns/205_floating_SIMD/main_RU}}
\DE{\input{patterns/205_floating_SIMD/main_DE}}

\EN{\input{patterns/ARM/main_EN}}
\RU{\input{patterns/ARM/main_RU}}
\DE{\input{patterns/ARM/main_DE}}

\input{patterns/MIPS/main}

\ifdefined\SPANISH
\chapter{Patrones de código}
\fi % SPANISH

\ifdefined\GERMAN
\chapter{Code-Muster}
\fi % GERMAN

\ifdefined\ENGLISH
\chapter{Code Patterns}
\fi % ENGLISH

\ifdefined\ITALIAN
\chapter{Forme di codice}
\fi % ITALIAN

\ifdefined\RUSSIAN
\chapter{Образцы кода}
\fi % RUSSIAN

\ifdefined\BRAZILIAN
\chapter{Padrões de códigos}
\fi % BRAZILIAN

\ifdefined\THAI
\chapter{รูปแบบของโค้ด}
\fi % THAI

\ifdefined\FRENCH
\chapter{Modèle de code}
\fi % FRENCH

\ifdefined\POLISH
\chapter{\PLph{}}
\fi % POLISH

% sections
\EN{\input{patterns/patterns_opt_dbg_EN}}
\ES{\input{patterns/patterns_opt_dbg_ES}}
\ITA{\input{patterns/patterns_opt_dbg_ITA}}
\PTBR{\input{patterns/patterns_opt_dbg_PTBR}}
\RU{\input{patterns/patterns_opt_dbg_RU}}
\THA{\input{patterns/patterns_opt_dbg_THA}}
\DE{\input{patterns/patterns_opt_dbg_DE}}
\FR{\input{patterns/patterns_opt_dbg_FR}}
\PL{\input{patterns/patterns_opt_dbg_PL}}

\RU{\section{Некоторые базовые понятия}}
\EN{\section{Some basics}}
\DE{\section{Einige Grundlagen}}
\FR{\section{Quelques bases}}
\ES{\section{\ESph{}}}
\ITA{\section{Alcune basi teoriche}}
\PTBR{\section{\PTBRph{}}}
\THA{\section{\THAph{}}}
\PL{\section{\PLph{}}}

% sections:
\EN{\input{patterns/intro_CPU_ISA_EN}}
\ES{\input{patterns/intro_CPU_ISA_ES}}
\ITA{\input{patterns/intro_CPU_ISA_ITA}}
\PTBR{\input{patterns/intro_CPU_ISA_PTBR}}
\RU{\input{patterns/intro_CPU_ISA_RU}}
\DE{\input{patterns/intro_CPU_ISA_DE}}
\FR{\input{patterns/intro_CPU_ISA_FR}}
\PL{\input{patterns/intro_CPU_ISA_PL}}

\EN{\input{patterns/numeral_EN}}
\RU{\input{patterns/numeral_RU}}
\ITA{\input{patterns/numeral_ITA}}
\DE{\input{patterns/numeral_DE}}
\FR{\input{patterns/numeral_FR}}
\PL{\input{patterns/numeral_PL}}

% chapters
\input{patterns/00_empty/main}
\input{patterns/011_ret/main}
\input{patterns/01_helloworld/main}
\input{patterns/015_prolog_epilogue/main}
\input{patterns/02_stack/main}
\input{patterns/03_printf/main}
\input{patterns/04_scanf/main}
\input{patterns/05_passing_arguments/main}
\input{patterns/06_return_results/main}
\input{patterns/061_pointers/main}
\input{patterns/065_GOTO/main}
\input{patterns/07_jcc/main}
\input{patterns/08_switch/main}
\input{patterns/09_loops/main}
\input{patterns/10_strings/main}
\input{patterns/11_arith_optimizations/main}
\input{patterns/12_FPU/main}
\input{patterns/13_arrays/main}
\input{patterns/14_bitfields/main}
\EN{\input{patterns/145_LCG/main_EN}}
\RU{\input{patterns/145_LCG/main_RU}}
\input{patterns/15_structs/main}
\input{patterns/17_unions/main}
\input{patterns/18_pointers_to_functions/main}
\input{patterns/185_64bit_in_32_env/main}

\EN{\input{patterns/19_SIMD/main_EN}}
\RU{\input{patterns/19_SIMD/main_RU}}
\DE{\input{patterns/19_SIMD/main_DE}}

\EN{\input{patterns/20_x64/main_EN}}
\RU{\input{patterns/20_x64/main_RU}}

\EN{\input{patterns/205_floating_SIMD/main_EN}}
\RU{\input{patterns/205_floating_SIMD/main_RU}}
\DE{\input{patterns/205_floating_SIMD/main_DE}}

\EN{\input{patterns/ARM/main_EN}}
\RU{\input{patterns/ARM/main_RU}}
\DE{\input{patterns/ARM/main_DE}}

\input{patterns/MIPS/main}

\ifdefined\SPANISH
\chapter{Patrones de código}
\fi % SPANISH

\ifdefined\GERMAN
\chapter{Code-Muster}
\fi % GERMAN

\ifdefined\ENGLISH
\chapter{Code Patterns}
\fi % ENGLISH

\ifdefined\ITALIAN
\chapter{Forme di codice}
\fi % ITALIAN

\ifdefined\RUSSIAN
\chapter{Образцы кода}
\fi % RUSSIAN

\ifdefined\BRAZILIAN
\chapter{Padrões de códigos}
\fi % BRAZILIAN

\ifdefined\THAI
\chapter{รูปแบบของโค้ด}
\fi % THAI

\ifdefined\FRENCH
\chapter{Modèle de code}
\fi % FRENCH

\ifdefined\POLISH
\chapter{\PLph{}}
\fi % POLISH

% sections
\EN{\input{patterns/patterns_opt_dbg_EN}}
\ES{\input{patterns/patterns_opt_dbg_ES}}
\ITA{\input{patterns/patterns_opt_dbg_ITA}}
\PTBR{\input{patterns/patterns_opt_dbg_PTBR}}
\RU{\input{patterns/patterns_opt_dbg_RU}}
\THA{\input{patterns/patterns_opt_dbg_THA}}
\DE{\input{patterns/patterns_opt_dbg_DE}}
\FR{\input{patterns/patterns_opt_dbg_FR}}
\PL{\input{patterns/patterns_opt_dbg_PL}}

\RU{\section{Некоторые базовые понятия}}
\EN{\section{Some basics}}
\DE{\section{Einige Grundlagen}}
\FR{\section{Quelques bases}}
\ES{\section{\ESph{}}}
\ITA{\section{Alcune basi teoriche}}
\PTBR{\section{\PTBRph{}}}
\THA{\section{\THAph{}}}
\PL{\section{\PLph{}}}

% sections:
\EN{\input{patterns/intro_CPU_ISA_EN}}
\ES{\input{patterns/intro_CPU_ISA_ES}}
\ITA{\input{patterns/intro_CPU_ISA_ITA}}
\PTBR{\input{patterns/intro_CPU_ISA_PTBR}}
\RU{\input{patterns/intro_CPU_ISA_RU}}
\DE{\input{patterns/intro_CPU_ISA_DE}}
\FR{\input{patterns/intro_CPU_ISA_FR}}
\PL{\input{patterns/intro_CPU_ISA_PL}}

\EN{\input{patterns/numeral_EN}}
\RU{\input{patterns/numeral_RU}}
\ITA{\input{patterns/numeral_ITA}}
\DE{\input{patterns/numeral_DE}}
\FR{\input{patterns/numeral_FR}}
\PL{\input{patterns/numeral_PL}}

% chapters
\input{patterns/00_empty/main}
\input{patterns/011_ret/main}
\input{patterns/01_helloworld/main}
\input{patterns/015_prolog_epilogue/main}
\input{patterns/02_stack/main}
\input{patterns/03_printf/main}
\input{patterns/04_scanf/main}
\input{patterns/05_passing_arguments/main}
\input{patterns/06_return_results/main}
\input{patterns/061_pointers/main}
\input{patterns/065_GOTO/main}
\input{patterns/07_jcc/main}
\input{patterns/08_switch/main}
\input{patterns/09_loops/main}
\input{patterns/10_strings/main}
\input{patterns/11_arith_optimizations/main}
\input{patterns/12_FPU/main}
\input{patterns/13_arrays/main}
\input{patterns/14_bitfields/main}
\EN{\input{patterns/145_LCG/main_EN}}
\RU{\input{patterns/145_LCG/main_RU}}
\input{patterns/15_structs/main}
\input{patterns/17_unions/main}
\input{patterns/18_pointers_to_functions/main}
\input{patterns/185_64bit_in_32_env/main}

\EN{\input{patterns/19_SIMD/main_EN}}
\RU{\input{patterns/19_SIMD/main_RU}}
\DE{\input{patterns/19_SIMD/main_DE}}

\EN{\input{patterns/20_x64/main_EN}}
\RU{\input{patterns/20_x64/main_RU}}

\EN{\input{patterns/205_floating_SIMD/main_EN}}
\RU{\input{patterns/205_floating_SIMD/main_RU}}
\DE{\input{patterns/205_floating_SIMD/main_DE}}

\EN{\input{patterns/ARM/main_EN}}
\RU{\input{patterns/ARM/main_RU}}
\DE{\input{patterns/ARM/main_DE}}

\input{patterns/MIPS/main}

\ifdefined\SPANISH
\chapter{Patrones de código}
\fi % SPANISH

\ifdefined\GERMAN
\chapter{Code-Muster}
\fi % GERMAN

\ifdefined\ENGLISH
\chapter{Code Patterns}
\fi % ENGLISH

\ifdefined\ITALIAN
\chapter{Forme di codice}
\fi % ITALIAN

\ifdefined\RUSSIAN
\chapter{Образцы кода}
\fi % RUSSIAN

\ifdefined\BRAZILIAN
\chapter{Padrões de códigos}
\fi % BRAZILIAN

\ifdefined\THAI
\chapter{รูปแบบของโค้ด}
\fi % THAI

\ifdefined\FRENCH
\chapter{Modèle de code}
\fi % FRENCH

\ifdefined\POLISH
\chapter{\PLph{}}
\fi % POLISH

% sections
\EN{\input{patterns/patterns_opt_dbg_EN}}
\ES{\input{patterns/patterns_opt_dbg_ES}}
\ITA{\input{patterns/patterns_opt_dbg_ITA}}
\PTBR{\input{patterns/patterns_opt_dbg_PTBR}}
\RU{\input{patterns/patterns_opt_dbg_RU}}
\THA{\input{patterns/patterns_opt_dbg_THA}}
\DE{\input{patterns/patterns_opt_dbg_DE}}
\FR{\input{patterns/patterns_opt_dbg_FR}}
\PL{\input{patterns/patterns_opt_dbg_PL}}

\RU{\section{Некоторые базовые понятия}}
\EN{\section{Some basics}}
\DE{\section{Einige Grundlagen}}
\FR{\section{Quelques bases}}
\ES{\section{\ESph{}}}
\ITA{\section{Alcune basi teoriche}}
\PTBR{\section{\PTBRph{}}}
\THA{\section{\THAph{}}}
\PL{\section{\PLph{}}}

% sections:
\EN{\input{patterns/intro_CPU_ISA_EN}}
\ES{\input{patterns/intro_CPU_ISA_ES}}
\ITA{\input{patterns/intro_CPU_ISA_ITA}}
\PTBR{\input{patterns/intro_CPU_ISA_PTBR}}
\RU{\input{patterns/intro_CPU_ISA_RU}}
\DE{\input{patterns/intro_CPU_ISA_DE}}
\FR{\input{patterns/intro_CPU_ISA_FR}}
\PL{\input{patterns/intro_CPU_ISA_PL}}

\EN{\input{patterns/numeral_EN}}
\RU{\input{patterns/numeral_RU}}
\ITA{\input{patterns/numeral_ITA}}
\DE{\input{patterns/numeral_DE}}
\FR{\input{patterns/numeral_FR}}
\PL{\input{patterns/numeral_PL}}

% chapters
\input{patterns/00_empty/main}
\input{patterns/011_ret/main}
\input{patterns/01_helloworld/main}
\input{patterns/015_prolog_epilogue/main}
\input{patterns/02_stack/main}
\input{patterns/03_printf/main}
\input{patterns/04_scanf/main}
\input{patterns/05_passing_arguments/main}
\input{patterns/06_return_results/main}
\input{patterns/061_pointers/main}
\input{patterns/065_GOTO/main}
\input{patterns/07_jcc/main}
\input{patterns/08_switch/main}
\input{patterns/09_loops/main}
\input{patterns/10_strings/main}
\input{patterns/11_arith_optimizations/main}
\input{patterns/12_FPU/main}
\input{patterns/13_arrays/main}
\input{patterns/14_bitfields/main}
\EN{\input{patterns/145_LCG/main_EN}}
\RU{\input{patterns/145_LCG/main_RU}}
\input{patterns/15_structs/main}
\input{patterns/17_unions/main}
\input{patterns/18_pointers_to_functions/main}
\input{patterns/185_64bit_in_32_env/main}

\EN{\input{patterns/19_SIMD/main_EN}}
\RU{\input{patterns/19_SIMD/main_RU}}
\DE{\input{patterns/19_SIMD/main_DE}}

\EN{\input{patterns/20_x64/main_EN}}
\RU{\input{patterns/20_x64/main_RU}}

\EN{\input{patterns/205_floating_SIMD/main_EN}}
\RU{\input{patterns/205_floating_SIMD/main_RU}}
\DE{\input{patterns/205_floating_SIMD/main_DE}}

\EN{\input{patterns/ARM/main_EN}}
\RU{\input{patterns/ARM/main_RU}}
\DE{\input{patterns/ARM/main_DE}}

\input{patterns/MIPS/main}

\ifdefined\SPANISH
\chapter{Patrones de código}
\fi % SPANISH

\ifdefined\GERMAN
\chapter{Code-Muster}
\fi % GERMAN

\ifdefined\ENGLISH
\chapter{Code Patterns}
\fi % ENGLISH

\ifdefined\ITALIAN
\chapter{Forme di codice}
\fi % ITALIAN

\ifdefined\RUSSIAN
\chapter{Образцы кода}
\fi % RUSSIAN

\ifdefined\BRAZILIAN
\chapter{Padrões de códigos}
\fi % BRAZILIAN

\ifdefined\THAI
\chapter{รูปแบบของโค้ด}
\fi % THAI

\ifdefined\FRENCH
\chapter{Modèle de code}
\fi % FRENCH

\ifdefined\POLISH
\chapter{\PLph{}}
\fi % POLISH

% sections
\EN{\input{patterns/patterns_opt_dbg_EN}}
\ES{\input{patterns/patterns_opt_dbg_ES}}
\ITA{\input{patterns/patterns_opt_dbg_ITA}}
\PTBR{\input{patterns/patterns_opt_dbg_PTBR}}
\RU{\input{patterns/patterns_opt_dbg_RU}}
\THA{\input{patterns/patterns_opt_dbg_THA}}
\DE{\input{patterns/patterns_opt_dbg_DE}}
\FR{\input{patterns/patterns_opt_dbg_FR}}
\PL{\input{patterns/patterns_opt_dbg_PL}}

\RU{\section{Некоторые базовые понятия}}
\EN{\section{Some basics}}
\DE{\section{Einige Grundlagen}}
\FR{\section{Quelques bases}}
\ES{\section{\ESph{}}}
\ITA{\section{Alcune basi teoriche}}
\PTBR{\section{\PTBRph{}}}
\THA{\section{\THAph{}}}
\PL{\section{\PLph{}}}

% sections:
\EN{\input{patterns/intro_CPU_ISA_EN}}
\ES{\input{patterns/intro_CPU_ISA_ES}}
\ITA{\input{patterns/intro_CPU_ISA_ITA}}
\PTBR{\input{patterns/intro_CPU_ISA_PTBR}}
\RU{\input{patterns/intro_CPU_ISA_RU}}
\DE{\input{patterns/intro_CPU_ISA_DE}}
\FR{\input{patterns/intro_CPU_ISA_FR}}
\PL{\input{patterns/intro_CPU_ISA_PL}}

\EN{\input{patterns/numeral_EN}}
\RU{\input{patterns/numeral_RU}}
\ITA{\input{patterns/numeral_ITA}}
\DE{\input{patterns/numeral_DE}}
\FR{\input{patterns/numeral_FR}}
\PL{\input{patterns/numeral_PL}}

% chapters
\input{patterns/00_empty/main}
\input{patterns/011_ret/main}
\input{patterns/01_helloworld/main}
\input{patterns/015_prolog_epilogue/main}
\input{patterns/02_stack/main}
\input{patterns/03_printf/main}
\input{patterns/04_scanf/main}
\input{patterns/05_passing_arguments/main}
\input{patterns/06_return_results/main}
\input{patterns/061_pointers/main}
\input{patterns/065_GOTO/main}
\input{patterns/07_jcc/main}
\input{patterns/08_switch/main}
\input{patterns/09_loops/main}
\input{patterns/10_strings/main}
\input{patterns/11_arith_optimizations/main}
\input{patterns/12_FPU/main}
\input{patterns/13_arrays/main}
\input{patterns/14_bitfields/main}
\EN{\input{patterns/145_LCG/main_EN}}
\RU{\input{patterns/145_LCG/main_RU}}
\input{patterns/15_structs/main}
\input{patterns/17_unions/main}
\input{patterns/18_pointers_to_functions/main}
\input{patterns/185_64bit_in_32_env/main}

\EN{\input{patterns/19_SIMD/main_EN}}
\RU{\input{patterns/19_SIMD/main_RU}}
\DE{\input{patterns/19_SIMD/main_DE}}

\EN{\input{patterns/20_x64/main_EN}}
\RU{\input{patterns/20_x64/main_RU}}

\EN{\input{patterns/205_floating_SIMD/main_EN}}
\RU{\input{patterns/205_floating_SIMD/main_RU}}
\DE{\input{patterns/205_floating_SIMD/main_DE}}

\EN{\input{patterns/ARM/main_EN}}
\RU{\input{patterns/ARM/main_RU}}
\DE{\input{patterns/ARM/main_DE}}

\input{patterns/MIPS/main}

\ifdefined\SPANISH
\chapter{Patrones de código}
\fi % SPANISH

\ifdefined\GERMAN
\chapter{Code-Muster}
\fi % GERMAN

\ifdefined\ENGLISH
\chapter{Code Patterns}
\fi % ENGLISH

\ifdefined\ITALIAN
\chapter{Forme di codice}
\fi % ITALIAN

\ifdefined\RUSSIAN
\chapter{Образцы кода}
\fi % RUSSIAN

\ifdefined\BRAZILIAN
\chapter{Padrões de códigos}
\fi % BRAZILIAN

\ifdefined\THAI
\chapter{รูปแบบของโค้ด}
\fi % THAI

\ifdefined\FRENCH
\chapter{Modèle de code}
\fi % FRENCH

\ifdefined\POLISH
\chapter{\PLph{}}
\fi % POLISH

% sections
\EN{\input{patterns/patterns_opt_dbg_EN}}
\ES{\input{patterns/patterns_opt_dbg_ES}}
\ITA{\input{patterns/patterns_opt_dbg_ITA}}
\PTBR{\input{patterns/patterns_opt_dbg_PTBR}}
\RU{\input{patterns/patterns_opt_dbg_RU}}
\THA{\input{patterns/patterns_opt_dbg_THA}}
\DE{\input{patterns/patterns_opt_dbg_DE}}
\FR{\input{patterns/patterns_opt_dbg_FR}}
\PL{\input{patterns/patterns_opt_dbg_PL}}

\RU{\section{Некоторые базовые понятия}}
\EN{\section{Some basics}}
\DE{\section{Einige Grundlagen}}
\FR{\section{Quelques bases}}
\ES{\section{\ESph{}}}
\ITA{\section{Alcune basi teoriche}}
\PTBR{\section{\PTBRph{}}}
\THA{\section{\THAph{}}}
\PL{\section{\PLph{}}}

% sections:
\EN{\input{patterns/intro_CPU_ISA_EN}}
\ES{\input{patterns/intro_CPU_ISA_ES}}
\ITA{\input{patterns/intro_CPU_ISA_ITA}}
\PTBR{\input{patterns/intro_CPU_ISA_PTBR}}
\RU{\input{patterns/intro_CPU_ISA_RU}}
\DE{\input{patterns/intro_CPU_ISA_DE}}
\FR{\input{patterns/intro_CPU_ISA_FR}}
\PL{\input{patterns/intro_CPU_ISA_PL}}

\EN{\input{patterns/numeral_EN}}
\RU{\input{patterns/numeral_RU}}
\ITA{\input{patterns/numeral_ITA}}
\DE{\input{patterns/numeral_DE}}
\FR{\input{patterns/numeral_FR}}
\PL{\input{patterns/numeral_PL}}

% chapters
\input{patterns/00_empty/main}
\input{patterns/011_ret/main}
\input{patterns/01_helloworld/main}
\input{patterns/015_prolog_epilogue/main}
\input{patterns/02_stack/main}
\input{patterns/03_printf/main}
\input{patterns/04_scanf/main}
\input{patterns/05_passing_arguments/main}
\input{patterns/06_return_results/main}
\input{patterns/061_pointers/main}
\input{patterns/065_GOTO/main}
\input{patterns/07_jcc/main}
\input{patterns/08_switch/main}
\input{patterns/09_loops/main}
\input{patterns/10_strings/main}
\input{patterns/11_arith_optimizations/main}
\input{patterns/12_FPU/main}
\input{patterns/13_arrays/main}
\input{patterns/14_bitfields/main}
\EN{\input{patterns/145_LCG/main_EN}}
\RU{\input{patterns/145_LCG/main_RU}}
\input{patterns/15_structs/main}
\input{patterns/17_unions/main}
\input{patterns/18_pointers_to_functions/main}
\input{patterns/185_64bit_in_32_env/main}

\EN{\input{patterns/19_SIMD/main_EN}}
\RU{\input{patterns/19_SIMD/main_RU}}
\DE{\input{patterns/19_SIMD/main_DE}}

\EN{\input{patterns/20_x64/main_EN}}
\RU{\input{patterns/20_x64/main_RU}}

\EN{\input{patterns/205_floating_SIMD/main_EN}}
\RU{\input{patterns/205_floating_SIMD/main_RU}}
\DE{\input{patterns/205_floating_SIMD/main_DE}}

\EN{\input{patterns/ARM/main_EN}}
\RU{\input{patterns/ARM/main_RU}}
\DE{\input{patterns/ARM/main_DE}}

\input{patterns/MIPS/main}

\ifdefined\SPANISH
\chapter{Patrones de código}
\fi % SPANISH

\ifdefined\GERMAN
\chapter{Code-Muster}
\fi % GERMAN

\ifdefined\ENGLISH
\chapter{Code Patterns}
\fi % ENGLISH

\ifdefined\ITALIAN
\chapter{Forme di codice}
\fi % ITALIAN

\ifdefined\RUSSIAN
\chapter{Образцы кода}
\fi % RUSSIAN

\ifdefined\BRAZILIAN
\chapter{Padrões de códigos}
\fi % BRAZILIAN

\ifdefined\THAI
\chapter{รูปแบบของโค้ด}
\fi % THAI

\ifdefined\FRENCH
\chapter{Modèle de code}
\fi % FRENCH

\ifdefined\POLISH
\chapter{\PLph{}}
\fi % POLISH

% sections
\EN{\input{patterns/patterns_opt_dbg_EN}}
\ES{\input{patterns/patterns_opt_dbg_ES}}
\ITA{\input{patterns/patterns_opt_dbg_ITA}}
\PTBR{\input{patterns/patterns_opt_dbg_PTBR}}
\RU{\input{patterns/patterns_opt_dbg_RU}}
\THA{\input{patterns/patterns_opt_dbg_THA}}
\DE{\input{patterns/patterns_opt_dbg_DE}}
\FR{\input{patterns/patterns_opt_dbg_FR}}
\PL{\input{patterns/patterns_opt_dbg_PL}}

\RU{\section{Некоторые базовые понятия}}
\EN{\section{Some basics}}
\DE{\section{Einige Grundlagen}}
\FR{\section{Quelques bases}}
\ES{\section{\ESph{}}}
\ITA{\section{Alcune basi teoriche}}
\PTBR{\section{\PTBRph{}}}
\THA{\section{\THAph{}}}
\PL{\section{\PLph{}}}

% sections:
\EN{\input{patterns/intro_CPU_ISA_EN}}
\ES{\input{patterns/intro_CPU_ISA_ES}}
\ITA{\input{patterns/intro_CPU_ISA_ITA}}
\PTBR{\input{patterns/intro_CPU_ISA_PTBR}}
\RU{\input{patterns/intro_CPU_ISA_RU}}
\DE{\input{patterns/intro_CPU_ISA_DE}}
\FR{\input{patterns/intro_CPU_ISA_FR}}
\PL{\input{patterns/intro_CPU_ISA_PL}}

\EN{\input{patterns/numeral_EN}}
\RU{\input{patterns/numeral_RU}}
\ITA{\input{patterns/numeral_ITA}}
\DE{\input{patterns/numeral_DE}}
\FR{\input{patterns/numeral_FR}}
\PL{\input{patterns/numeral_PL}}

% chapters
\input{patterns/00_empty/main}
\input{patterns/011_ret/main}
\input{patterns/01_helloworld/main}
\input{patterns/015_prolog_epilogue/main}
\input{patterns/02_stack/main}
\input{patterns/03_printf/main}
\input{patterns/04_scanf/main}
\input{patterns/05_passing_arguments/main}
\input{patterns/06_return_results/main}
\input{patterns/061_pointers/main}
\input{patterns/065_GOTO/main}
\input{patterns/07_jcc/main}
\input{patterns/08_switch/main}
\input{patterns/09_loops/main}
\input{patterns/10_strings/main}
\input{patterns/11_arith_optimizations/main}
\input{patterns/12_FPU/main}
\input{patterns/13_arrays/main}
\input{patterns/14_bitfields/main}
\EN{\input{patterns/145_LCG/main_EN}}
\RU{\input{patterns/145_LCG/main_RU}}
\input{patterns/15_structs/main}
\input{patterns/17_unions/main}
\input{patterns/18_pointers_to_functions/main}
\input{patterns/185_64bit_in_32_env/main}

\EN{\input{patterns/19_SIMD/main_EN}}
\RU{\input{patterns/19_SIMD/main_RU}}
\DE{\input{patterns/19_SIMD/main_DE}}

\EN{\input{patterns/20_x64/main_EN}}
\RU{\input{patterns/20_x64/main_RU}}

\EN{\input{patterns/205_floating_SIMD/main_EN}}
\RU{\input{patterns/205_floating_SIMD/main_RU}}
\DE{\input{patterns/205_floating_SIMD/main_DE}}

\EN{\input{patterns/ARM/main_EN}}
\RU{\input{patterns/ARM/main_RU}}
\DE{\input{patterns/ARM/main_DE}}

\input{patterns/MIPS/main}

\ifdefined\SPANISH
\chapter{Patrones de código}
\fi % SPANISH

\ifdefined\GERMAN
\chapter{Code-Muster}
\fi % GERMAN

\ifdefined\ENGLISH
\chapter{Code Patterns}
\fi % ENGLISH

\ifdefined\ITALIAN
\chapter{Forme di codice}
\fi % ITALIAN

\ifdefined\RUSSIAN
\chapter{Образцы кода}
\fi % RUSSIAN

\ifdefined\BRAZILIAN
\chapter{Padrões de códigos}
\fi % BRAZILIAN

\ifdefined\THAI
\chapter{รูปแบบของโค้ด}
\fi % THAI

\ifdefined\FRENCH
\chapter{Modèle de code}
\fi % FRENCH

\ifdefined\POLISH
\chapter{\PLph{}}
\fi % POLISH

% sections
\EN{\input{patterns/patterns_opt_dbg_EN}}
\ES{\input{patterns/patterns_opt_dbg_ES}}
\ITA{\input{patterns/patterns_opt_dbg_ITA}}
\PTBR{\input{patterns/patterns_opt_dbg_PTBR}}
\RU{\input{patterns/patterns_opt_dbg_RU}}
\THA{\input{patterns/patterns_opt_dbg_THA}}
\DE{\input{patterns/patterns_opt_dbg_DE}}
\FR{\input{patterns/patterns_opt_dbg_FR}}
\PL{\input{patterns/patterns_opt_dbg_PL}}

\RU{\section{Некоторые базовые понятия}}
\EN{\section{Some basics}}
\DE{\section{Einige Grundlagen}}
\FR{\section{Quelques bases}}
\ES{\section{\ESph{}}}
\ITA{\section{Alcune basi teoriche}}
\PTBR{\section{\PTBRph{}}}
\THA{\section{\THAph{}}}
\PL{\section{\PLph{}}}

% sections:
\EN{\input{patterns/intro_CPU_ISA_EN}}
\ES{\input{patterns/intro_CPU_ISA_ES}}
\ITA{\input{patterns/intro_CPU_ISA_ITA}}
\PTBR{\input{patterns/intro_CPU_ISA_PTBR}}
\RU{\input{patterns/intro_CPU_ISA_RU}}
\DE{\input{patterns/intro_CPU_ISA_DE}}
\FR{\input{patterns/intro_CPU_ISA_FR}}
\PL{\input{patterns/intro_CPU_ISA_PL}}

\EN{\input{patterns/numeral_EN}}
\RU{\input{patterns/numeral_RU}}
\ITA{\input{patterns/numeral_ITA}}
\DE{\input{patterns/numeral_DE}}
\FR{\input{patterns/numeral_FR}}
\PL{\input{patterns/numeral_PL}}

% chapters
\input{patterns/00_empty/main}
\input{patterns/011_ret/main}
\input{patterns/01_helloworld/main}
\input{patterns/015_prolog_epilogue/main}
\input{patterns/02_stack/main}
\input{patterns/03_printf/main}
\input{patterns/04_scanf/main}
\input{patterns/05_passing_arguments/main}
\input{patterns/06_return_results/main}
\input{patterns/061_pointers/main}
\input{patterns/065_GOTO/main}
\input{patterns/07_jcc/main}
\input{patterns/08_switch/main}
\input{patterns/09_loops/main}
\input{patterns/10_strings/main}
\input{patterns/11_arith_optimizations/main}
\input{patterns/12_FPU/main}
\input{patterns/13_arrays/main}
\input{patterns/14_bitfields/main}
\EN{\input{patterns/145_LCG/main_EN}}
\RU{\input{patterns/145_LCG/main_RU}}
\input{patterns/15_structs/main}
\input{patterns/17_unions/main}
\input{patterns/18_pointers_to_functions/main}
\input{patterns/185_64bit_in_32_env/main}

\EN{\input{patterns/19_SIMD/main_EN}}
\RU{\input{patterns/19_SIMD/main_RU}}
\DE{\input{patterns/19_SIMD/main_DE}}

\EN{\input{patterns/20_x64/main_EN}}
\RU{\input{patterns/20_x64/main_RU}}

\EN{\input{patterns/205_floating_SIMD/main_EN}}
\RU{\input{patterns/205_floating_SIMD/main_RU}}
\DE{\input{patterns/205_floating_SIMD/main_DE}}

\EN{\input{patterns/ARM/main_EN}}
\RU{\input{patterns/ARM/main_RU}}
\DE{\input{patterns/ARM/main_DE}}

\input{patterns/MIPS/main}

\ifdefined\SPANISH
\chapter{Patrones de código}
\fi % SPANISH

\ifdefined\GERMAN
\chapter{Code-Muster}
\fi % GERMAN

\ifdefined\ENGLISH
\chapter{Code Patterns}
\fi % ENGLISH

\ifdefined\ITALIAN
\chapter{Forme di codice}
\fi % ITALIAN

\ifdefined\RUSSIAN
\chapter{Образцы кода}
\fi % RUSSIAN

\ifdefined\BRAZILIAN
\chapter{Padrões de códigos}
\fi % BRAZILIAN

\ifdefined\THAI
\chapter{รูปแบบของโค้ด}
\fi % THAI

\ifdefined\FRENCH
\chapter{Modèle de code}
\fi % FRENCH

\ifdefined\POLISH
\chapter{\PLph{}}
\fi % POLISH

% sections
\EN{\input{patterns/patterns_opt_dbg_EN}}
\ES{\input{patterns/patterns_opt_dbg_ES}}
\ITA{\input{patterns/patterns_opt_dbg_ITA}}
\PTBR{\input{patterns/patterns_opt_dbg_PTBR}}
\RU{\input{patterns/patterns_opt_dbg_RU}}
\THA{\input{patterns/patterns_opt_dbg_THA}}
\DE{\input{patterns/patterns_opt_dbg_DE}}
\FR{\input{patterns/patterns_opt_dbg_FR}}
\PL{\input{patterns/patterns_opt_dbg_PL}}

\RU{\section{Некоторые базовые понятия}}
\EN{\section{Some basics}}
\DE{\section{Einige Grundlagen}}
\FR{\section{Quelques bases}}
\ES{\section{\ESph{}}}
\ITA{\section{Alcune basi teoriche}}
\PTBR{\section{\PTBRph{}}}
\THA{\section{\THAph{}}}
\PL{\section{\PLph{}}}

% sections:
\EN{\input{patterns/intro_CPU_ISA_EN}}
\ES{\input{patterns/intro_CPU_ISA_ES}}
\ITA{\input{patterns/intro_CPU_ISA_ITA}}
\PTBR{\input{patterns/intro_CPU_ISA_PTBR}}
\RU{\input{patterns/intro_CPU_ISA_RU}}
\DE{\input{patterns/intro_CPU_ISA_DE}}
\FR{\input{patterns/intro_CPU_ISA_FR}}
\PL{\input{patterns/intro_CPU_ISA_PL}}

\EN{\input{patterns/numeral_EN}}
\RU{\input{patterns/numeral_RU}}
\ITA{\input{patterns/numeral_ITA}}
\DE{\input{patterns/numeral_DE}}
\FR{\input{patterns/numeral_FR}}
\PL{\input{patterns/numeral_PL}}

% chapters
\input{patterns/00_empty/main}
\input{patterns/011_ret/main}
\input{patterns/01_helloworld/main}
\input{patterns/015_prolog_epilogue/main}
\input{patterns/02_stack/main}
\input{patterns/03_printf/main}
\input{patterns/04_scanf/main}
\input{patterns/05_passing_arguments/main}
\input{patterns/06_return_results/main}
\input{patterns/061_pointers/main}
\input{patterns/065_GOTO/main}
\input{patterns/07_jcc/main}
\input{patterns/08_switch/main}
\input{patterns/09_loops/main}
\input{patterns/10_strings/main}
\input{patterns/11_arith_optimizations/main}
\input{patterns/12_FPU/main}
\input{patterns/13_arrays/main}
\input{patterns/14_bitfields/main}
\EN{\input{patterns/145_LCG/main_EN}}
\RU{\input{patterns/145_LCG/main_RU}}
\input{patterns/15_structs/main}
\input{patterns/17_unions/main}
\input{patterns/18_pointers_to_functions/main}
\input{patterns/185_64bit_in_32_env/main}

\EN{\input{patterns/19_SIMD/main_EN}}
\RU{\input{patterns/19_SIMD/main_RU}}
\DE{\input{patterns/19_SIMD/main_DE}}

\EN{\input{patterns/20_x64/main_EN}}
\RU{\input{patterns/20_x64/main_RU}}

\EN{\input{patterns/205_floating_SIMD/main_EN}}
\RU{\input{patterns/205_floating_SIMD/main_RU}}
\DE{\input{patterns/205_floating_SIMD/main_DE}}

\EN{\input{patterns/ARM/main_EN}}
\RU{\input{patterns/ARM/main_RU}}
\DE{\input{patterns/ARM/main_DE}}

\input{patterns/MIPS/main}

\ifdefined\SPANISH
\chapter{Patrones de código}
\fi % SPANISH

\ifdefined\GERMAN
\chapter{Code-Muster}
\fi % GERMAN

\ifdefined\ENGLISH
\chapter{Code Patterns}
\fi % ENGLISH

\ifdefined\ITALIAN
\chapter{Forme di codice}
\fi % ITALIAN

\ifdefined\RUSSIAN
\chapter{Образцы кода}
\fi % RUSSIAN

\ifdefined\BRAZILIAN
\chapter{Padrões de códigos}
\fi % BRAZILIAN

\ifdefined\THAI
\chapter{รูปแบบของโค้ด}
\fi % THAI

\ifdefined\FRENCH
\chapter{Modèle de code}
\fi % FRENCH

\ifdefined\POLISH
\chapter{\PLph{}}
\fi % POLISH

% sections
\EN{\input{patterns/patterns_opt_dbg_EN}}
\ES{\input{patterns/patterns_opt_dbg_ES}}
\ITA{\input{patterns/patterns_opt_dbg_ITA}}
\PTBR{\input{patterns/patterns_opt_dbg_PTBR}}
\RU{\input{patterns/patterns_opt_dbg_RU}}
\THA{\input{patterns/patterns_opt_dbg_THA}}
\DE{\input{patterns/patterns_opt_dbg_DE}}
\FR{\input{patterns/patterns_opt_dbg_FR}}
\PL{\input{patterns/patterns_opt_dbg_PL}}

\RU{\section{Некоторые базовые понятия}}
\EN{\section{Some basics}}
\DE{\section{Einige Grundlagen}}
\FR{\section{Quelques bases}}
\ES{\section{\ESph{}}}
\ITA{\section{Alcune basi teoriche}}
\PTBR{\section{\PTBRph{}}}
\THA{\section{\THAph{}}}
\PL{\section{\PLph{}}}

% sections:
\EN{\input{patterns/intro_CPU_ISA_EN}}
\ES{\input{patterns/intro_CPU_ISA_ES}}
\ITA{\input{patterns/intro_CPU_ISA_ITA}}
\PTBR{\input{patterns/intro_CPU_ISA_PTBR}}
\RU{\input{patterns/intro_CPU_ISA_RU}}
\DE{\input{patterns/intro_CPU_ISA_DE}}
\FR{\input{patterns/intro_CPU_ISA_FR}}
\PL{\input{patterns/intro_CPU_ISA_PL}}

\EN{\input{patterns/numeral_EN}}
\RU{\input{patterns/numeral_RU}}
\ITA{\input{patterns/numeral_ITA}}
\DE{\input{patterns/numeral_DE}}
\FR{\input{patterns/numeral_FR}}
\PL{\input{patterns/numeral_PL}}

% chapters
\input{patterns/00_empty/main}
\input{patterns/011_ret/main}
\input{patterns/01_helloworld/main}
\input{patterns/015_prolog_epilogue/main}
\input{patterns/02_stack/main}
\input{patterns/03_printf/main}
\input{patterns/04_scanf/main}
\input{patterns/05_passing_arguments/main}
\input{patterns/06_return_results/main}
\input{patterns/061_pointers/main}
\input{patterns/065_GOTO/main}
\input{patterns/07_jcc/main}
\input{patterns/08_switch/main}
\input{patterns/09_loops/main}
\input{patterns/10_strings/main}
\input{patterns/11_arith_optimizations/main}
\input{patterns/12_FPU/main}
\input{patterns/13_arrays/main}
\input{patterns/14_bitfields/main}
\EN{\input{patterns/145_LCG/main_EN}}
\RU{\input{patterns/145_LCG/main_RU}}
\input{patterns/15_structs/main}
\input{patterns/17_unions/main}
\input{patterns/18_pointers_to_functions/main}
\input{patterns/185_64bit_in_32_env/main}

\EN{\input{patterns/19_SIMD/main_EN}}
\RU{\input{patterns/19_SIMD/main_RU}}
\DE{\input{patterns/19_SIMD/main_DE}}

\EN{\input{patterns/20_x64/main_EN}}
\RU{\input{patterns/20_x64/main_RU}}

\EN{\input{patterns/205_floating_SIMD/main_EN}}
\RU{\input{patterns/205_floating_SIMD/main_RU}}
\DE{\input{patterns/205_floating_SIMD/main_DE}}

\EN{\input{patterns/ARM/main_EN}}
\RU{\input{patterns/ARM/main_RU}}
\DE{\input{patterns/ARM/main_DE}}

\input{patterns/MIPS/main}

\ifdefined\SPANISH
\chapter{Patrones de código}
\fi % SPANISH

\ifdefined\GERMAN
\chapter{Code-Muster}
\fi % GERMAN

\ifdefined\ENGLISH
\chapter{Code Patterns}
\fi % ENGLISH

\ifdefined\ITALIAN
\chapter{Forme di codice}
\fi % ITALIAN

\ifdefined\RUSSIAN
\chapter{Образцы кода}
\fi % RUSSIAN

\ifdefined\BRAZILIAN
\chapter{Padrões de códigos}
\fi % BRAZILIAN

\ifdefined\THAI
\chapter{รูปแบบของโค้ด}
\fi % THAI

\ifdefined\FRENCH
\chapter{Modèle de code}
\fi % FRENCH

\ifdefined\POLISH
\chapter{\PLph{}}
\fi % POLISH

% sections
\EN{\input{patterns/patterns_opt_dbg_EN}}
\ES{\input{patterns/patterns_opt_dbg_ES}}
\ITA{\input{patterns/patterns_opt_dbg_ITA}}
\PTBR{\input{patterns/patterns_opt_dbg_PTBR}}
\RU{\input{patterns/patterns_opt_dbg_RU}}
\THA{\input{patterns/patterns_opt_dbg_THA}}
\DE{\input{patterns/patterns_opt_dbg_DE}}
\FR{\input{patterns/patterns_opt_dbg_FR}}
\PL{\input{patterns/patterns_opt_dbg_PL}}

\RU{\section{Некоторые базовые понятия}}
\EN{\section{Some basics}}
\DE{\section{Einige Grundlagen}}
\FR{\section{Quelques bases}}
\ES{\section{\ESph{}}}
\ITA{\section{Alcune basi teoriche}}
\PTBR{\section{\PTBRph{}}}
\THA{\section{\THAph{}}}
\PL{\section{\PLph{}}}

% sections:
\EN{\input{patterns/intro_CPU_ISA_EN}}
\ES{\input{patterns/intro_CPU_ISA_ES}}
\ITA{\input{patterns/intro_CPU_ISA_ITA}}
\PTBR{\input{patterns/intro_CPU_ISA_PTBR}}
\RU{\input{patterns/intro_CPU_ISA_RU}}
\DE{\input{patterns/intro_CPU_ISA_DE}}
\FR{\input{patterns/intro_CPU_ISA_FR}}
\PL{\input{patterns/intro_CPU_ISA_PL}}

\EN{\input{patterns/numeral_EN}}
\RU{\input{patterns/numeral_RU}}
\ITA{\input{patterns/numeral_ITA}}
\DE{\input{patterns/numeral_DE}}
\FR{\input{patterns/numeral_FR}}
\PL{\input{patterns/numeral_PL}}

% chapters
\input{patterns/00_empty/main}
\input{patterns/011_ret/main}
\input{patterns/01_helloworld/main}
\input{patterns/015_prolog_epilogue/main}
\input{patterns/02_stack/main}
\input{patterns/03_printf/main}
\input{patterns/04_scanf/main}
\input{patterns/05_passing_arguments/main}
\input{patterns/06_return_results/main}
\input{patterns/061_pointers/main}
\input{patterns/065_GOTO/main}
\input{patterns/07_jcc/main}
\input{patterns/08_switch/main}
\input{patterns/09_loops/main}
\input{patterns/10_strings/main}
\input{patterns/11_arith_optimizations/main}
\input{patterns/12_FPU/main}
\input{patterns/13_arrays/main}
\input{patterns/14_bitfields/main}
\EN{\input{patterns/145_LCG/main_EN}}
\RU{\input{patterns/145_LCG/main_RU}}
\input{patterns/15_structs/main}
\input{patterns/17_unions/main}
\input{patterns/18_pointers_to_functions/main}
\input{patterns/185_64bit_in_32_env/main}

\EN{\input{patterns/19_SIMD/main_EN}}
\RU{\input{patterns/19_SIMD/main_RU}}
\DE{\input{patterns/19_SIMD/main_DE}}

\EN{\input{patterns/20_x64/main_EN}}
\RU{\input{patterns/20_x64/main_RU}}

\EN{\input{patterns/205_floating_SIMD/main_EN}}
\RU{\input{patterns/205_floating_SIMD/main_RU}}
\DE{\input{patterns/205_floating_SIMD/main_DE}}

\EN{\input{patterns/ARM/main_EN}}
\RU{\input{patterns/ARM/main_RU}}
\DE{\input{patterns/ARM/main_DE}}

\input{patterns/MIPS/main}

\ifdefined\SPANISH
\chapter{Patrones de código}
\fi % SPANISH

\ifdefined\GERMAN
\chapter{Code-Muster}
\fi % GERMAN

\ifdefined\ENGLISH
\chapter{Code Patterns}
\fi % ENGLISH

\ifdefined\ITALIAN
\chapter{Forme di codice}
\fi % ITALIAN

\ifdefined\RUSSIAN
\chapter{Образцы кода}
\fi % RUSSIAN

\ifdefined\BRAZILIAN
\chapter{Padrões de códigos}
\fi % BRAZILIAN

\ifdefined\THAI
\chapter{รูปแบบของโค้ด}
\fi % THAI

\ifdefined\FRENCH
\chapter{Modèle de code}
\fi % FRENCH

\ifdefined\POLISH
\chapter{\PLph{}}
\fi % POLISH

% sections
\EN{\input{patterns/patterns_opt_dbg_EN}}
\ES{\input{patterns/patterns_opt_dbg_ES}}
\ITA{\input{patterns/patterns_opt_dbg_ITA}}
\PTBR{\input{patterns/patterns_opt_dbg_PTBR}}
\RU{\input{patterns/patterns_opt_dbg_RU}}
\THA{\input{patterns/patterns_opt_dbg_THA}}
\DE{\input{patterns/patterns_opt_dbg_DE}}
\FR{\input{patterns/patterns_opt_dbg_FR}}
\PL{\input{patterns/patterns_opt_dbg_PL}}

\RU{\section{Некоторые базовые понятия}}
\EN{\section{Some basics}}
\DE{\section{Einige Grundlagen}}
\FR{\section{Quelques bases}}
\ES{\section{\ESph{}}}
\ITA{\section{Alcune basi teoriche}}
\PTBR{\section{\PTBRph{}}}
\THA{\section{\THAph{}}}
\PL{\section{\PLph{}}}

% sections:
\EN{\input{patterns/intro_CPU_ISA_EN}}
\ES{\input{patterns/intro_CPU_ISA_ES}}
\ITA{\input{patterns/intro_CPU_ISA_ITA}}
\PTBR{\input{patterns/intro_CPU_ISA_PTBR}}
\RU{\input{patterns/intro_CPU_ISA_RU}}
\DE{\input{patterns/intro_CPU_ISA_DE}}
\FR{\input{patterns/intro_CPU_ISA_FR}}
\PL{\input{patterns/intro_CPU_ISA_PL}}

\EN{\input{patterns/numeral_EN}}
\RU{\input{patterns/numeral_RU}}
\ITA{\input{patterns/numeral_ITA}}
\DE{\input{patterns/numeral_DE}}
\FR{\input{patterns/numeral_FR}}
\PL{\input{patterns/numeral_PL}}

% chapters
\input{patterns/00_empty/main}
\input{patterns/011_ret/main}
\input{patterns/01_helloworld/main}
\input{patterns/015_prolog_epilogue/main}
\input{patterns/02_stack/main}
\input{patterns/03_printf/main}
\input{patterns/04_scanf/main}
\input{patterns/05_passing_arguments/main}
\input{patterns/06_return_results/main}
\input{patterns/061_pointers/main}
\input{patterns/065_GOTO/main}
\input{patterns/07_jcc/main}
\input{patterns/08_switch/main}
\input{patterns/09_loops/main}
\input{patterns/10_strings/main}
\input{patterns/11_arith_optimizations/main}
\input{patterns/12_FPU/main}
\input{patterns/13_arrays/main}
\input{patterns/14_bitfields/main}
\EN{\input{patterns/145_LCG/main_EN}}
\RU{\input{patterns/145_LCG/main_RU}}
\input{patterns/15_structs/main}
\input{patterns/17_unions/main}
\input{patterns/18_pointers_to_functions/main}
\input{patterns/185_64bit_in_32_env/main}

\EN{\input{patterns/19_SIMD/main_EN}}
\RU{\input{patterns/19_SIMD/main_RU}}
\DE{\input{patterns/19_SIMD/main_DE}}

\EN{\input{patterns/20_x64/main_EN}}
\RU{\input{patterns/20_x64/main_RU}}

\EN{\input{patterns/205_floating_SIMD/main_EN}}
\RU{\input{patterns/205_floating_SIMD/main_RU}}
\DE{\input{patterns/205_floating_SIMD/main_DE}}

\EN{\input{patterns/ARM/main_EN}}
\RU{\input{patterns/ARM/main_RU}}
\DE{\input{patterns/ARM/main_DE}}

\input{patterns/MIPS/main}

\EN{\input{patterns/12_FPU/main_EN}}
\RU{\input{patterns/12_FPU/main_RU}}
\DE{\input{patterns/12_FPU/main_DE}}
\FR{\input{patterns/12_FPU/main_FR}}


\ifdefined\SPANISH
\chapter{Patrones de código}
\fi % SPANISH

\ifdefined\GERMAN
\chapter{Code-Muster}
\fi % GERMAN

\ifdefined\ENGLISH
\chapter{Code Patterns}
\fi % ENGLISH

\ifdefined\ITALIAN
\chapter{Forme di codice}
\fi % ITALIAN

\ifdefined\RUSSIAN
\chapter{Образцы кода}
\fi % RUSSIAN

\ifdefined\BRAZILIAN
\chapter{Padrões de códigos}
\fi % BRAZILIAN

\ifdefined\THAI
\chapter{รูปแบบของโค้ด}
\fi % THAI

\ifdefined\FRENCH
\chapter{Modèle de code}
\fi % FRENCH

\ifdefined\POLISH
\chapter{\PLph{}}
\fi % POLISH

% sections
\EN{\input{patterns/patterns_opt_dbg_EN}}
\ES{\input{patterns/patterns_opt_dbg_ES}}
\ITA{\input{patterns/patterns_opt_dbg_ITA}}
\PTBR{\input{patterns/patterns_opt_dbg_PTBR}}
\RU{\input{patterns/patterns_opt_dbg_RU}}
\THA{\input{patterns/patterns_opt_dbg_THA}}
\DE{\input{patterns/patterns_opt_dbg_DE}}
\FR{\input{patterns/patterns_opt_dbg_FR}}
\PL{\input{patterns/patterns_opt_dbg_PL}}

\RU{\section{Некоторые базовые понятия}}
\EN{\section{Some basics}}
\DE{\section{Einige Grundlagen}}
\FR{\section{Quelques bases}}
\ES{\section{\ESph{}}}
\ITA{\section{Alcune basi teoriche}}
\PTBR{\section{\PTBRph{}}}
\THA{\section{\THAph{}}}
\PL{\section{\PLph{}}}

% sections:
\EN{\input{patterns/intro_CPU_ISA_EN}}
\ES{\input{patterns/intro_CPU_ISA_ES}}
\ITA{\input{patterns/intro_CPU_ISA_ITA}}
\PTBR{\input{patterns/intro_CPU_ISA_PTBR}}
\RU{\input{patterns/intro_CPU_ISA_RU}}
\DE{\input{patterns/intro_CPU_ISA_DE}}
\FR{\input{patterns/intro_CPU_ISA_FR}}
\PL{\input{patterns/intro_CPU_ISA_PL}}

\EN{\input{patterns/numeral_EN}}
\RU{\input{patterns/numeral_RU}}
\ITA{\input{patterns/numeral_ITA}}
\DE{\input{patterns/numeral_DE}}
\FR{\input{patterns/numeral_FR}}
\PL{\input{patterns/numeral_PL}}

% chapters
\input{patterns/00_empty/main}
\input{patterns/011_ret/main}
\input{patterns/01_helloworld/main}
\input{patterns/015_prolog_epilogue/main}
\input{patterns/02_stack/main}
\input{patterns/03_printf/main}
\input{patterns/04_scanf/main}
\input{patterns/05_passing_arguments/main}
\input{patterns/06_return_results/main}
\input{patterns/061_pointers/main}
\input{patterns/065_GOTO/main}
\input{patterns/07_jcc/main}
\input{patterns/08_switch/main}
\input{patterns/09_loops/main}
\input{patterns/10_strings/main}
\input{patterns/11_arith_optimizations/main}
\input{patterns/12_FPU/main}
\input{patterns/13_arrays/main}
\input{patterns/14_bitfields/main}
\EN{\input{patterns/145_LCG/main_EN}}
\RU{\input{patterns/145_LCG/main_RU}}
\input{patterns/15_structs/main}
\input{patterns/17_unions/main}
\input{patterns/18_pointers_to_functions/main}
\input{patterns/185_64bit_in_32_env/main}

\EN{\input{patterns/19_SIMD/main_EN}}
\RU{\input{patterns/19_SIMD/main_RU}}
\DE{\input{patterns/19_SIMD/main_DE}}

\EN{\input{patterns/20_x64/main_EN}}
\RU{\input{patterns/20_x64/main_RU}}

\EN{\input{patterns/205_floating_SIMD/main_EN}}
\RU{\input{patterns/205_floating_SIMD/main_RU}}
\DE{\input{patterns/205_floating_SIMD/main_DE}}

\EN{\input{patterns/ARM/main_EN}}
\RU{\input{patterns/ARM/main_RU}}
\DE{\input{patterns/ARM/main_DE}}

\input{patterns/MIPS/main}

\ifdefined\SPANISH
\chapter{Patrones de código}
\fi % SPANISH

\ifdefined\GERMAN
\chapter{Code-Muster}
\fi % GERMAN

\ifdefined\ENGLISH
\chapter{Code Patterns}
\fi % ENGLISH

\ifdefined\ITALIAN
\chapter{Forme di codice}
\fi % ITALIAN

\ifdefined\RUSSIAN
\chapter{Образцы кода}
\fi % RUSSIAN

\ifdefined\BRAZILIAN
\chapter{Padrões de códigos}
\fi % BRAZILIAN

\ifdefined\THAI
\chapter{รูปแบบของโค้ด}
\fi % THAI

\ifdefined\FRENCH
\chapter{Modèle de code}
\fi % FRENCH

\ifdefined\POLISH
\chapter{\PLph{}}
\fi % POLISH

% sections
\EN{\input{patterns/patterns_opt_dbg_EN}}
\ES{\input{patterns/patterns_opt_dbg_ES}}
\ITA{\input{patterns/patterns_opt_dbg_ITA}}
\PTBR{\input{patterns/patterns_opt_dbg_PTBR}}
\RU{\input{patterns/patterns_opt_dbg_RU}}
\THA{\input{patterns/patterns_opt_dbg_THA}}
\DE{\input{patterns/patterns_opt_dbg_DE}}
\FR{\input{patterns/patterns_opt_dbg_FR}}
\PL{\input{patterns/patterns_opt_dbg_PL}}

\RU{\section{Некоторые базовые понятия}}
\EN{\section{Some basics}}
\DE{\section{Einige Grundlagen}}
\FR{\section{Quelques bases}}
\ES{\section{\ESph{}}}
\ITA{\section{Alcune basi teoriche}}
\PTBR{\section{\PTBRph{}}}
\THA{\section{\THAph{}}}
\PL{\section{\PLph{}}}

% sections:
\EN{\input{patterns/intro_CPU_ISA_EN}}
\ES{\input{patterns/intro_CPU_ISA_ES}}
\ITA{\input{patterns/intro_CPU_ISA_ITA}}
\PTBR{\input{patterns/intro_CPU_ISA_PTBR}}
\RU{\input{patterns/intro_CPU_ISA_RU}}
\DE{\input{patterns/intro_CPU_ISA_DE}}
\FR{\input{patterns/intro_CPU_ISA_FR}}
\PL{\input{patterns/intro_CPU_ISA_PL}}

\EN{\input{patterns/numeral_EN}}
\RU{\input{patterns/numeral_RU}}
\ITA{\input{patterns/numeral_ITA}}
\DE{\input{patterns/numeral_DE}}
\FR{\input{patterns/numeral_FR}}
\PL{\input{patterns/numeral_PL}}

% chapters
\input{patterns/00_empty/main}
\input{patterns/011_ret/main}
\input{patterns/01_helloworld/main}
\input{patterns/015_prolog_epilogue/main}
\input{patterns/02_stack/main}
\input{patterns/03_printf/main}
\input{patterns/04_scanf/main}
\input{patterns/05_passing_arguments/main}
\input{patterns/06_return_results/main}
\input{patterns/061_pointers/main}
\input{patterns/065_GOTO/main}
\input{patterns/07_jcc/main}
\input{patterns/08_switch/main}
\input{patterns/09_loops/main}
\input{patterns/10_strings/main}
\input{patterns/11_arith_optimizations/main}
\input{patterns/12_FPU/main}
\input{patterns/13_arrays/main}
\input{patterns/14_bitfields/main}
\EN{\input{patterns/145_LCG/main_EN}}
\RU{\input{patterns/145_LCG/main_RU}}
\input{patterns/15_structs/main}
\input{patterns/17_unions/main}
\input{patterns/18_pointers_to_functions/main}
\input{patterns/185_64bit_in_32_env/main}

\EN{\input{patterns/19_SIMD/main_EN}}
\RU{\input{patterns/19_SIMD/main_RU}}
\DE{\input{patterns/19_SIMD/main_DE}}

\EN{\input{patterns/20_x64/main_EN}}
\RU{\input{patterns/20_x64/main_RU}}

\EN{\input{patterns/205_floating_SIMD/main_EN}}
\RU{\input{patterns/205_floating_SIMD/main_RU}}
\DE{\input{patterns/205_floating_SIMD/main_DE}}

\EN{\input{patterns/ARM/main_EN}}
\RU{\input{patterns/ARM/main_RU}}
\DE{\input{patterns/ARM/main_DE}}

\input{patterns/MIPS/main}

\EN{\section{Returning Values}
\label{ret_val_func}

Another simple function is the one that simply returns a constant value:

\lstinputlisting[caption=\EN{\CCpp Code},style=customc]{patterns/011_ret/1.c}

Let's compile it.

\subsection{x86}

Here's what both the GCC and MSVC compilers produce (with optimization) on the x86 platform:

\lstinputlisting[caption=\Optimizing GCC/MSVC (\assemblyOutput),style=customasmx86]{patterns/011_ret/1.s}

\myindex{x86!\Instructions!RET}
There are just two instructions: the first places the value 123 into the \EAX register,
which is used by convention for storing the return
value, and the second one is \RET, which returns execution to the \gls{caller}.

The caller will take the result from the \EAX register.

\subsection{ARM}

There are a few differences on the ARM platform:

\lstinputlisting[caption=\OptimizingKeilVI (\ARMMode) ASM Output,style=customasmARM]{patterns/011_ret/1_Keil_ARM_O3.s}

ARM uses the register \Reg{0} for returning the results of functions, so 123 is copied into \Reg{0}.

\myindex{ARM!\Instructions!MOV}
\myindex{x86!\Instructions!MOV}
It is worth noting that \MOV is a misleading name for the instruction in both the x86 and ARM \ac{ISA}s.

The data is not in fact \IT{moved}, but \IT{copied}.

\subsection{MIPS}

\label{MIPS_leaf_function_ex1}

The GCC assembly output below lists registers by number:

\lstinputlisting[caption=\Optimizing GCC 4.4.5 (\assemblyOutput),style=customasmMIPS]{patterns/011_ret/MIPS.s}

\dots while \IDA does it by their pseudo names:

\lstinputlisting[caption=\Optimizing GCC 4.4.5 (IDA),style=customasmMIPS]{patterns/011_ret/MIPS_IDA.lst}

The \$2 (or \$V0) register is used to store the function's return value.
\myindex{MIPS!\Pseudoinstructions!LI}
\INS{LI} stands for ``Load Immediate'' and is the MIPS equivalent to \MOV.

\myindex{MIPS!\Instructions!J}
The other instruction is the jump instruction (J or JR) which returns the execution flow to the \gls{caller}.

\myindex{MIPS!Branch delay slot}
You might be wondering why the positions of the load instruction (LI) and the jump instruction (J or JR) are swapped. This is due to a \ac{RISC} feature called ``branch delay slot''.

The reason this happens is a quirk in the architecture of some RISC \ac{ISA}s and isn't important for our
purposes---we must simply keep in mind that in MIPS, the instruction following a jump or branch instruction
is executed \IT{before} the jump/branch instruction itself.

As a consequence, branch instructions always swap places with the instruction executed immediately beforehand.


In practice, functions which merely return 1 (\IT{true}) or 0 (\IT{false}) are very frequent.

The smallest ever of the standard UNIX utilities, \IT{/bin/true} and \IT{/bin/false} return 0 and 1 respectively, as an exit code.
(Zero as an exit code usually means success, non-zero means error.)
}
\RU{\subsubsection{std::string}
\myindex{\Cpp!STL!std::string}
\label{std_string}

\myparagraph{Как устроена структура}

Многие строковые библиотеки \InSqBrackets{\CNotes 2.2} обеспечивают структуру содержащую ссылку 
на буфер собственно со строкой, переменная всегда содержащую длину строки 
(что очень удобно для массы функций \InSqBrackets{\CNotes 2.2.1}) и переменную содержащую текущий размер буфера.

Строка в буфере обыкновенно оканчивается нулем: это для того чтобы указатель на буфер можно было
передавать в функции требующие на вход обычную сишную \ac{ASCIIZ}-строку.

Стандарт \Cpp не описывает, как именно нужно реализовывать std::string,
но, как правило, они реализованы как описано выше, с небольшими дополнениями.

Строки в \Cpp это не класс (как, например, QString в Qt), а темплейт (basic\_string), 
это сделано для того чтобы поддерживать 
строки содержащие разного типа символы: как минимум \Tchar и \IT{wchar\_t}.

Так что, std::string это класс с базовым типом \Tchar.

А std::wstring это класс с базовым типом \IT{wchar\_t}.

\mysubparagraph{MSVC}

В реализации MSVC, вместо ссылки на буфер может содержаться сам буфер (если строка короче 16-и символов).

Это означает, что каждая короткая строка будет занимать в памяти по крайней мере $16 + 4 + 4 = 24$ 
байт для 32-битной среды либо $16 + 8 + 8 = 32$ 
байта в 64-битной, а если строка длиннее 16-и символов, то прибавьте еще длину самой строки.

\lstinputlisting[caption=пример для MSVC,style=customc]{\CURPATH/STL/string/MSVC_RU.cpp}

Собственно, из этого исходника почти всё ясно.

Несколько замечаний:

Если строка короче 16-и символов, 
то отдельный буфер для строки в \glslink{heap}{куче} выделяться не будет.

Это удобно потому что на практике, основная часть строк действительно короткие.
Вероятно, разработчики в Microsoft выбрали размер в 16 символов как разумный баланс.

Теперь очень важный момент в конце функции main(): мы не пользуемся методом c\_str(), тем не менее,
если это скомпилировать и запустить, то обе строки появятся в консоли!

Работает это вот почему.

В первом случае строка короче 16-и символов и в начале объекта std::string (его можно рассматривать
просто как структуру) расположен буфер с этой строкой.
\printf трактует указатель как указатель на массив символов оканчивающийся нулем и поэтому всё работает.

Вывод второй строки (длиннее 16-и символов) даже еще опаснее: это вообще типичная программистская ошибка 
(или опечатка), забыть дописать c\_str().
Это работает потому что в это время в начале структуры расположен указатель на буфер.
Это может надолго остаться незамеченным: до тех пока там не появится строка 
короче 16-и символов, тогда процесс упадет.

\mysubparagraph{GCC}

В реализации GCC в структуре есть еще одна переменная --- reference count.

Интересно, что указатель на экземпляр класса std::string в GCC указывает не на начало самой структуры, 
а на указатель на буфера.
В libstdc++-v3\textbackslash{}include\textbackslash{}bits\textbackslash{}basic\_string.h 
мы можем прочитать что это сделано для удобства отладки:

\begin{lstlisting}
   *  The reason you want _M_data pointing to the character %array and
   *  not the _Rep is so that the debugger can see the string
   *  contents. (Probably we should add a non-inline member to get
   *  the _Rep for the debugger to use, so users can check the actual
   *  string length.)
\end{lstlisting}

\href{http://go.yurichev.com/17085}{исходный код basic\_string.h}

В нашем примере мы учитываем это:

\lstinputlisting[caption=пример для GCC,style=customc]{\CURPATH/STL/string/GCC_RU.cpp}

Нужны еще небольшие хаки чтобы сымитировать типичную ошибку, которую мы уже видели выше, из-за
более ужесточенной проверки типов в GCC, тем не менее, printf() работает и здесь без c\_str().

\myparagraph{Чуть более сложный пример}

\lstinputlisting[style=customc]{\CURPATH/STL/string/3.cpp}

\lstinputlisting[caption=MSVC 2012,style=customasmx86]{\CURPATH/STL/string/3_MSVC_RU.asm}

Собственно, компилятор не конструирует строки статически: да в общем-то и как
это возможно, если буфер с ней нужно хранить в \glslink{heap}{куче}?

Вместо этого в сегменте данных хранятся обычные \ac{ASCIIZ}-строки, а позже, во время выполнения, 
при помощи метода \q{assign}, конструируются строки s1 и s2
.
При помощи \TT{operator+}, создается строка s3.

Обратите внимание на то что вызов метода c\_str() отсутствует,
потому что его код достаточно короткий и компилятор вставил его прямо здесь:
если строка короче 16-и байт, то в регистре EAX остается указатель на буфер,
а если длиннее, то из этого же места достается адрес на буфер расположенный в \glslink{heap}{куче}.

Далее следуют вызовы трех деструкторов, причем, они вызываются только если строка длиннее 16-и байт:
тогда нужно освободить буфера в \glslink{heap}{куче}.
В противном случае, так как все три объекта std::string хранятся в стеке,
они освобождаются автоматически после выхода из функции.

Следовательно, работа с короткими строками более быстрая из-за м\'{е}ньшего обращения к \glslink{heap}{куче}.

Код на GCC даже проще (из-за того, что в GCC, как мы уже видели, не реализована возможность хранить короткую
строку прямо в структуре):

% TODO1 comment each function meaning
\lstinputlisting[caption=GCC 4.8.1,style=customasmx86]{\CURPATH/STL/string/3_GCC_RU.s}

Можно заметить, что в деструкторы передается не указатель на объект,
а указатель на место за 12 байт (или 3 слова) перед ним, то есть, на настоящее начало структуры.

\myparagraph{std::string как глобальная переменная}
\label{sec:std_string_as_global_variable}

Опытные программисты на \Cpp знают, что глобальные переменные \ac{STL}-типов вполне можно объявлять.

Да, действительно:

\lstinputlisting[style=customc]{\CURPATH/STL/string/5.cpp}

Но как и где будет вызываться конструктор \TT{std::string}?

На самом деле, эта переменная будет инициализирована даже перед началом \main.

\lstinputlisting[caption=MSVC 2012: здесь конструируется глобальная переменная{,} а также регистрируется её деструктор,style=customasmx86]{\CURPATH/STL/string/5_MSVC_p2.asm}

\lstinputlisting[caption=MSVC 2012: здесь глобальная переменная используется в \main,style=customasmx86]{\CURPATH/STL/string/5_MSVC_p1.asm}

\lstinputlisting[caption=MSVC 2012: эта функция-деструктор вызывается перед выходом,style=customasmx86]{\CURPATH/STL/string/5_MSVC_p3.asm}

\myindex{\CStandardLibrary!atexit()}
В реальности, из \ac{CRT}, еще до вызова main(), вызывается специальная функция,
в которой перечислены все конструкторы подобных переменных.
Более того: при помощи atexit() регистрируется функция, которая будет вызвана в конце работы программы:
в этой функции компилятор собирает вызовы деструкторов всех подобных глобальных переменных.

GCC работает похожим образом:

\lstinputlisting[caption=GCC 4.8.1,style=customasmx86]{\CURPATH/STL/string/5_GCC.s}

Но он не выделяет отдельной функции в которой будут собраны деструкторы: 
каждый деструктор передается в atexit() по одному.

% TODO а если глобальная STL-переменная в другом модуле? надо проверить.

}
\ifdefined\SPANISH
\chapter{Patrones de código}
\fi % SPANISH

\ifdefined\GERMAN
\chapter{Code-Muster}
\fi % GERMAN

\ifdefined\ENGLISH
\chapter{Code Patterns}
\fi % ENGLISH

\ifdefined\ITALIAN
\chapter{Forme di codice}
\fi % ITALIAN

\ifdefined\RUSSIAN
\chapter{Образцы кода}
\fi % RUSSIAN

\ifdefined\BRAZILIAN
\chapter{Padrões de códigos}
\fi % BRAZILIAN

\ifdefined\THAI
\chapter{รูปแบบของโค้ด}
\fi % THAI

\ifdefined\FRENCH
\chapter{Modèle de code}
\fi % FRENCH

\ifdefined\POLISH
\chapter{\PLph{}}
\fi % POLISH

% sections
\EN{\input{patterns/patterns_opt_dbg_EN}}
\ES{\input{patterns/patterns_opt_dbg_ES}}
\ITA{\input{patterns/patterns_opt_dbg_ITA}}
\PTBR{\input{patterns/patterns_opt_dbg_PTBR}}
\RU{\input{patterns/patterns_opt_dbg_RU}}
\THA{\input{patterns/patterns_opt_dbg_THA}}
\DE{\input{patterns/patterns_opt_dbg_DE}}
\FR{\input{patterns/patterns_opt_dbg_FR}}
\PL{\input{patterns/patterns_opt_dbg_PL}}

\RU{\section{Некоторые базовые понятия}}
\EN{\section{Some basics}}
\DE{\section{Einige Grundlagen}}
\FR{\section{Quelques bases}}
\ES{\section{\ESph{}}}
\ITA{\section{Alcune basi teoriche}}
\PTBR{\section{\PTBRph{}}}
\THA{\section{\THAph{}}}
\PL{\section{\PLph{}}}

% sections:
\EN{\input{patterns/intro_CPU_ISA_EN}}
\ES{\input{patterns/intro_CPU_ISA_ES}}
\ITA{\input{patterns/intro_CPU_ISA_ITA}}
\PTBR{\input{patterns/intro_CPU_ISA_PTBR}}
\RU{\input{patterns/intro_CPU_ISA_RU}}
\DE{\input{patterns/intro_CPU_ISA_DE}}
\FR{\input{patterns/intro_CPU_ISA_FR}}
\PL{\input{patterns/intro_CPU_ISA_PL}}

\EN{\input{patterns/numeral_EN}}
\RU{\input{patterns/numeral_RU}}
\ITA{\input{patterns/numeral_ITA}}
\DE{\input{patterns/numeral_DE}}
\FR{\input{patterns/numeral_FR}}
\PL{\input{patterns/numeral_PL}}

% chapters
\input{patterns/00_empty/main}
\input{patterns/011_ret/main}
\input{patterns/01_helloworld/main}
\input{patterns/015_prolog_epilogue/main}
\input{patterns/02_stack/main}
\input{patterns/03_printf/main}
\input{patterns/04_scanf/main}
\input{patterns/05_passing_arguments/main}
\input{patterns/06_return_results/main}
\input{patterns/061_pointers/main}
\input{patterns/065_GOTO/main}
\input{patterns/07_jcc/main}
\input{patterns/08_switch/main}
\input{patterns/09_loops/main}
\input{patterns/10_strings/main}
\input{patterns/11_arith_optimizations/main}
\input{patterns/12_FPU/main}
\input{patterns/13_arrays/main}
\input{patterns/14_bitfields/main}
\EN{\input{patterns/145_LCG/main_EN}}
\RU{\input{patterns/145_LCG/main_RU}}
\input{patterns/15_structs/main}
\input{patterns/17_unions/main}
\input{patterns/18_pointers_to_functions/main}
\input{patterns/185_64bit_in_32_env/main}

\EN{\input{patterns/19_SIMD/main_EN}}
\RU{\input{patterns/19_SIMD/main_RU}}
\DE{\input{patterns/19_SIMD/main_DE}}

\EN{\input{patterns/20_x64/main_EN}}
\RU{\input{patterns/20_x64/main_RU}}

\EN{\input{patterns/205_floating_SIMD/main_EN}}
\RU{\input{patterns/205_floating_SIMD/main_RU}}
\DE{\input{patterns/205_floating_SIMD/main_DE}}

\EN{\input{patterns/ARM/main_EN}}
\RU{\input{patterns/ARM/main_RU}}
\DE{\input{patterns/ARM/main_DE}}

\input{patterns/MIPS/main}

\ifdefined\SPANISH
\chapter{Patrones de código}
\fi % SPANISH

\ifdefined\GERMAN
\chapter{Code-Muster}
\fi % GERMAN

\ifdefined\ENGLISH
\chapter{Code Patterns}
\fi % ENGLISH

\ifdefined\ITALIAN
\chapter{Forme di codice}
\fi % ITALIAN

\ifdefined\RUSSIAN
\chapter{Образцы кода}
\fi % RUSSIAN

\ifdefined\BRAZILIAN
\chapter{Padrões de códigos}
\fi % BRAZILIAN

\ifdefined\THAI
\chapter{รูปแบบของโค้ด}
\fi % THAI

\ifdefined\FRENCH
\chapter{Modèle de code}
\fi % FRENCH

\ifdefined\POLISH
\chapter{\PLph{}}
\fi % POLISH

% sections
\EN{\input{patterns/patterns_opt_dbg_EN}}
\ES{\input{patterns/patterns_opt_dbg_ES}}
\ITA{\input{patterns/patterns_opt_dbg_ITA}}
\PTBR{\input{patterns/patterns_opt_dbg_PTBR}}
\RU{\input{patterns/patterns_opt_dbg_RU}}
\THA{\input{patterns/patterns_opt_dbg_THA}}
\DE{\input{patterns/patterns_opt_dbg_DE}}
\FR{\input{patterns/patterns_opt_dbg_FR}}
\PL{\input{patterns/patterns_opt_dbg_PL}}

\RU{\section{Некоторые базовые понятия}}
\EN{\section{Some basics}}
\DE{\section{Einige Grundlagen}}
\FR{\section{Quelques bases}}
\ES{\section{\ESph{}}}
\ITA{\section{Alcune basi teoriche}}
\PTBR{\section{\PTBRph{}}}
\THA{\section{\THAph{}}}
\PL{\section{\PLph{}}}

% sections:
\EN{\input{patterns/intro_CPU_ISA_EN}}
\ES{\input{patterns/intro_CPU_ISA_ES}}
\ITA{\input{patterns/intro_CPU_ISA_ITA}}
\PTBR{\input{patterns/intro_CPU_ISA_PTBR}}
\RU{\input{patterns/intro_CPU_ISA_RU}}
\DE{\input{patterns/intro_CPU_ISA_DE}}
\FR{\input{patterns/intro_CPU_ISA_FR}}
\PL{\input{patterns/intro_CPU_ISA_PL}}

\EN{\input{patterns/numeral_EN}}
\RU{\input{patterns/numeral_RU}}
\ITA{\input{patterns/numeral_ITA}}
\DE{\input{patterns/numeral_DE}}
\FR{\input{patterns/numeral_FR}}
\PL{\input{patterns/numeral_PL}}

% chapters
\input{patterns/00_empty/main}
\input{patterns/011_ret/main}
\input{patterns/01_helloworld/main}
\input{patterns/015_prolog_epilogue/main}
\input{patterns/02_stack/main}
\input{patterns/03_printf/main}
\input{patterns/04_scanf/main}
\input{patterns/05_passing_arguments/main}
\input{patterns/06_return_results/main}
\input{patterns/061_pointers/main}
\input{patterns/065_GOTO/main}
\input{patterns/07_jcc/main}
\input{patterns/08_switch/main}
\input{patterns/09_loops/main}
\input{patterns/10_strings/main}
\input{patterns/11_arith_optimizations/main}
\input{patterns/12_FPU/main}
\input{patterns/13_arrays/main}
\input{patterns/14_bitfields/main}
\EN{\input{patterns/145_LCG/main_EN}}
\RU{\input{patterns/145_LCG/main_RU}}
\input{patterns/15_structs/main}
\input{patterns/17_unions/main}
\input{patterns/18_pointers_to_functions/main}
\input{patterns/185_64bit_in_32_env/main}

\EN{\input{patterns/19_SIMD/main_EN}}
\RU{\input{patterns/19_SIMD/main_RU}}
\DE{\input{patterns/19_SIMD/main_DE}}

\EN{\input{patterns/20_x64/main_EN}}
\RU{\input{patterns/20_x64/main_RU}}

\EN{\input{patterns/205_floating_SIMD/main_EN}}
\RU{\input{patterns/205_floating_SIMD/main_RU}}
\DE{\input{patterns/205_floating_SIMD/main_DE}}

\EN{\input{patterns/ARM/main_EN}}
\RU{\input{patterns/ARM/main_RU}}
\DE{\input{patterns/ARM/main_DE}}

\input{patterns/MIPS/main}

\ifdefined\SPANISH
\chapter{Patrones de código}
\fi % SPANISH

\ifdefined\GERMAN
\chapter{Code-Muster}
\fi % GERMAN

\ifdefined\ENGLISH
\chapter{Code Patterns}
\fi % ENGLISH

\ifdefined\ITALIAN
\chapter{Forme di codice}
\fi % ITALIAN

\ifdefined\RUSSIAN
\chapter{Образцы кода}
\fi % RUSSIAN

\ifdefined\BRAZILIAN
\chapter{Padrões de códigos}
\fi % BRAZILIAN

\ifdefined\THAI
\chapter{รูปแบบของโค้ด}
\fi % THAI

\ifdefined\FRENCH
\chapter{Modèle de code}
\fi % FRENCH

\ifdefined\POLISH
\chapter{\PLph{}}
\fi % POLISH

% sections
\EN{\input{patterns/patterns_opt_dbg_EN}}
\ES{\input{patterns/patterns_opt_dbg_ES}}
\ITA{\input{patterns/patterns_opt_dbg_ITA}}
\PTBR{\input{patterns/patterns_opt_dbg_PTBR}}
\RU{\input{patterns/patterns_opt_dbg_RU}}
\THA{\input{patterns/patterns_opt_dbg_THA}}
\DE{\input{patterns/patterns_opt_dbg_DE}}
\FR{\input{patterns/patterns_opt_dbg_FR}}
\PL{\input{patterns/patterns_opt_dbg_PL}}

\RU{\section{Некоторые базовые понятия}}
\EN{\section{Some basics}}
\DE{\section{Einige Grundlagen}}
\FR{\section{Quelques bases}}
\ES{\section{\ESph{}}}
\ITA{\section{Alcune basi teoriche}}
\PTBR{\section{\PTBRph{}}}
\THA{\section{\THAph{}}}
\PL{\section{\PLph{}}}

% sections:
\EN{\input{patterns/intro_CPU_ISA_EN}}
\ES{\input{patterns/intro_CPU_ISA_ES}}
\ITA{\input{patterns/intro_CPU_ISA_ITA}}
\PTBR{\input{patterns/intro_CPU_ISA_PTBR}}
\RU{\input{patterns/intro_CPU_ISA_RU}}
\DE{\input{patterns/intro_CPU_ISA_DE}}
\FR{\input{patterns/intro_CPU_ISA_FR}}
\PL{\input{patterns/intro_CPU_ISA_PL}}

\EN{\input{patterns/numeral_EN}}
\RU{\input{patterns/numeral_RU}}
\ITA{\input{patterns/numeral_ITA}}
\DE{\input{patterns/numeral_DE}}
\FR{\input{patterns/numeral_FR}}
\PL{\input{patterns/numeral_PL}}

% chapters
\input{patterns/00_empty/main}
\input{patterns/011_ret/main}
\input{patterns/01_helloworld/main}
\input{patterns/015_prolog_epilogue/main}
\input{patterns/02_stack/main}
\input{patterns/03_printf/main}
\input{patterns/04_scanf/main}
\input{patterns/05_passing_arguments/main}
\input{patterns/06_return_results/main}
\input{patterns/061_pointers/main}
\input{patterns/065_GOTO/main}
\input{patterns/07_jcc/main}
\input{patterns/08_switch/main}
\input{patterns/09_loops/main}
\input{patterns/10_strings/main}
\input{patterns/11_arith_optimizations/main}
\input{patterns/12_FPU/main}
\input{patterns/13_arrays/main}
\input{patterns/14_bitfields/main}
\EN{\input{patterns/145_LCG/main_EN}}
\RU{\input{patterns/145_LCG/main_RU}}
\input{patterns/15_structs/main}
\input{patterns/17_unions/main}
\input{patterns/18_pointers_to_functions/main}
\input{patterns/185_64bit_in_32_env/main}

\EN{\input{patterns/19_SIMD/main_EN}}
\RU{\input{patterns/19_SIMD/main_RU}}
\DE{\input{patterns/19_SIMD/main_DE}}

\EN{\input{patterns/20_x64/main_EN}}
\RU{\input{patterns/20_x64/main_RU}}

\EN{\input{patterns/205_floating_SIMD/main_EN}}
\RU{\input{patterns/205_floating_SIMD/main_RU}}
\DE{\input{patterns/205_floating_SIMD/main_DE}}

\EN{\input{patterns/ARM/main_EN}}
\RU{\input{patterns/ARM/main_RU}}
\DE{\input{patterns/ARM/main_DE}}

\input{patterns/MIPS/main}

\ifdefined\SPANISH
\chapter{Patrones de código}
\fi % SPANISH

\ifdefined\GERMAN
\chapter{Code-Muster}
\fi % GERMAN

\ifdefined\ENGLISH
\chapter{Code Patterns}
\fi % ENGLISH

\ifdefined\ITALIAN
\chapter{Forme di codice}
\fi % ITALIAN

\ifdefined\RUSSIAN
\chapter{Образцы кода}
\fi % RUSSIAN

\ifdefined\BRAZILIAN
\chapter{Padrões de códigos}
\fi % BRAZILIAN

\ifdefined\THAI
\chapter{รูปแบบของโค้ด}
\fi % THAI

\ifdefined\FRENCH
\chapter{Modèle de code}
\fi % FRENCH

\ifdefined\POLISH
\chapter{\PLph{}}
\fi % POLISH

% sections
\EN{\input{patterns/patterns_opt_dbg_EN}}
\ES{\input{patterns/patterns_opt_dbg_ES}}
\ITA{\input{patterns/patterns_opt_dbg_ITA}}
\PTBR{\input{patterns/patterns_opt_dbg_PTBR}}
\RU{\input{patterns/patterns_opt_dbg_RU}}
\THA{\input{patterns/patterns_opt_dbg_THA}}
\DE{\input{patterns/patterns_opt_dbg_DE}}
\FR{\input{patterns/patterns_opt_dbg_FR}}
\PL{\input{patterns/patterns_opt_dbg_PL}}

\RU{\section{Некоторые базовые понятия}}
\EN{\section{Some basics}}
\DE{\section{Einige Grundlagen}}
\FR{\section{Quelques bases}}
\ES{\section{\ESph{}}}
\ITA{\section{Alcune basi teoriche}}
\PTBR{\section{\PTBRph{}}}
\THA{\section{\THAph{}}}
\PL{\section{\PLph{}}}

% sections:
\EN{\input{patterns/intro_CPU_ISA_EN}}
\ES{\input{patterns/intro_CPU_ISA_ES}}
\ITA{\input{patterns/intro_CPU_ISA_ITA}}
\PTBR{\input{patterns/intro_CPU_ISA_PTBR}}
\RU{\input{patterns/intro_CPU_ISA_RU}}
\DE{\input{patterns/intro_CPU_ISA_DE}}
\FR{\input{patterns/intro_CPU_ISA_FR}}
\PL{\input{patterns/intro_CPU_ISA_PL}}

\EN{\input{patterns/numeral_EN}}
\RU{\input{patterns/numeral_RU}}
\ITA{\input{patterns/numeral_ITA}}
\DE{\input{patterns/numeral_DE}}
\FR{\input{patterns/numeral_FR}}
\PL{\input{patterns/numeral_PL}}

% chapters
\input{patterns/00_empty/main}
\input{patterns/011_ret/main}
\input{patterns/01_helloworld/main}
\input{patterns/015_prolog_epilogue/main}
\input{patterns/02_stack/main}
\input{patterns/03_printf/main}
\input{patterns/04_scanf/main}
\input{patterns/05_passing_arguments/main}
\input{patterns/06_return_results/main}
\input{patterns/061_pointers/main}
\input{patterns/065_GOTO/main}
\input{patterns/07_jcc/main}
\input{patterns/08_switch/main}
\input{patterns/09_loops/main}
\input{patterns/10_strings/main}
\input{patterns/11_arith_optimizations/main}
\input{patterns/12_FPU/main}
\input{patterns/13_arrays/main}
\input{patterns/14_bitfields/main}
\EN{\input{patterns/145_LCG/main_EN}}
\RU{\input{patterns/145_LCG/main_RU}}
\input{patterns/15_structs/main}
\input{patterns/17_unions/main}
\input{patterns/18_pointers_to_functions/main}
\input{patterns/185_64bit_in_32_env/main}

\EN{\input{patterns/19_SIMD/main_EN}}
\RU{\input{patterns/19_SIMD/main_RU}}
\DE{\input{patterns/19_SIMD/main_DE}}

\EN{\input{patterns/20_x64/main_EN}}
\RU{\input{patterns/20_x64/main_RU}}

\EN{\input{patterns/205_floating_SIMD/main_EN}}
\RU{\input{patterns/205_floating_SIMD/main_RU}}
\DE{\input{patterns/205_floating_SIMD/main_DE}}

\EN{\input{patterns/ARM/main_EN}}
\RU{\input{patterns/ARM/main_RU}}
\DE{\input{patterns/ARM/main_DE}}

\input{patterns/MIPS/main}


\EN{\section{Returning Values}
\label{ret_val_func}

Another simple function is the one that simply returns a constant value:

\lstinputlisting[caption=\EN{\CCpp Code},style=customc]{patterns/011_ret/1.c}

Let's compile it.

\subsection{x86}

Here's what both the GCC and MSVC compilers produce (with optimization) on the x86 platform:

\lstinputlisting[caption=\Optimizing GCC/MSVC (\assemblyOutput),style=customasmx86]{patterns/011_ret/1.s}

\myindex{x86!\Instructions!RET}
There are just two instructions: the first places the value 123 into the \EAX register,
which is used by convention for storing the return
value, and the second one is \RET, which returns execution to the \gls{caller}.

The caller will take the result from the \EAX register.

\subsection{ARM}

There are a few differences on the ARM platform:

\lstinputlisting[caption=\OptimizingKeilVI (\ARMMode) ASM Output,style=customasmARM]{patterns/011_ret/1_Keil_ARM_O3.s}

ARM uses the register \Reg{0} for returning the results of functions, so 123 is copied into \Reg{0}.

\myindex{ARM!\Instructions!MOV}
\myindex{x86!\Instructions!MOV}
It is worth noting that \MOV is a misleading name for the instruction in both the x86 and ARM \ac{ISA}s.

The data is not in fact \IT{moved}, but \IT{copied}.

\subsection{MIPS}

\label{MIPS_leaf_function_ex1}

The GCC assembly output below lists registers by number:

\lstinputlisting[caption=\Optimizing GCC 4.4.5 (\assemblyOutput),style=customasmMIPS]{patterns/011_ret/MIPS.s}

\dots while \IDA does it by their pseudo names:

\lstinputlisting[caption=\Optimizing GCC 4.4.5 (IDA),style=customasmMIPS]{patterns/011_ret/MIPS_IDA.lst}

The \$2 (or \$V0) register is used to store the function's return value.
\myindex{MIPS!\Pseudoinstructions!LI}
\INS{LI} stands for ``Load Immediate'' and is the MIPS equivalent to \MOV.

\myindex{MIPS!\Instructions!J}
The other instruction is the jump instruction (J or JR) which returns the execution flow to the \gls{caller}.

\myindex{MIPS!Branch delay slot}
You might be wondering why the positions of the load instruction (LI) and the jump instruction (J or JR) are swapped. This is due to a \ac{RISC} feature called ``branch delay slot''.

The reason this happens is a quirk in the architecture of some RISC \ac{ISA}s and isn't important for our
purposes---we must simply keep in mind that in MIPS, the instruction following a jump or branch instruction
is executed \IT{before} the jump/branch instruction itself.

As a consequence, branch instructions always swap places with the instruction executed immediately beforehand.


In practice, functions which merely return 1 (\IT{true}) or 0 (\IT{false}) are very frequent.

The smallest ever of the standard UNIX utilities, \IT{/bin/true} and \IT{/bin/false} return 0 and 1 respectively, as an exit code.
(Zero as an exit code usually means success, non-zero means error.)
}
\RU{\subsubsection{std::string}
\myindex{\Cpp!STL!std::string}
\label{std_string}

\myparagraph{Как устроена структура}

Многие строковые библиотеки \InSqBrackets{\CNotes 2.2} обеспечивают структуру содержащую ссылку 
на буфер собственно со строкой, переменная всегда содержащую длину строки 
(что очень удобно для массы функций \InSqBrackets{\CNotes 2.2.1}) и переменную содержащую текущий размер буфера.

Строка в буфере обыкновенно оканчивается нулем: это для того чтобы указатель на буфер можно было
передавать в функции требующие на вход обычную сишную \ac{ASCIIZ}-строку.

Стандарт \Cpp не описывает, как именно нужно реализовывать std::string,
но, как правило, они реализованы как описано выше, с небольшими дополнениями.

Строки в \Cpp это не класс (как, например, QString в Qt), а темплейт (basic\_string), 
это сделано для того чтобы поддерживать 
строки содержащие разного типа символы: как минимум \Tchar и \IT{wchar\_t}.

Так что, std::string это класс с базовым типом \Tchar.

А std::wstring это класс с базовым типом \IT{wchar\_t}.

\mysubparagraph{MSVC}

В реализации MSVC, вместо ссылки на буфер может содержаться сам буфер (если строка короче 16-и символов).

Это означает, что каждая короткая строка будет занимать в памяти по крайней мере $16 + 4 + 4 = 24$ 
байт для 32-битной среды либо $16 + 8 + 8 = 32$ 
байта в 64-битной, а если строка длиннее 16-и символов, то прибавьте еще длину самой строки.

\lstinputlisting[caption=пример для MSVC,style=customc]{\CURPATH/STL/string/MSVC_RU.cpp}

Собственно, из этого исходника почти всё ясно.

Несколько замечаний:

Если строка короче 16-и символов, 
то отдельный буфер для строки в \glslink{heap}{куче} выделяться не будет.

Это удобно потому что на практике, основная часть строк действительно короткие.
Вероятно, разработчики в Microsoft выбрали размер в 16 символов как разумный баланс.

Теперь очень важный момент в конце функции main(): мы не пользуемся методом c\_str(), тем не менее,
если это скомпилировать и запустить, то обе строки появятся в консоли!

Работает это вот почему.

В первом случае строка короче 16-и символов и в начале объекта std::string (его можно рассматривать
просто как структуру) расположен буфер с этой строкой.
\printf трактует указатель как указатель на массив символов оканчивающийся нулем и поэтому всё работает.

Вывод второй строки (длиннее 16-и символов) даже еще опаснее: это вообще типичная программистская ошибка 
(или опечатка), забыть дописать c\_str().
Это работает потому что в это время в начале структуры расположен указатель на буфер.
Это может надолго остаться незамеченным: до тех пока там не появится строка 
короче 16-и символов, тогда процесс упадет.

\mysubparagraph{GCC}

В реализации GCC в структуре есть еще одна переменная --- reference count.

Интересно, что указатель на экземпляр класса std::string в GCC указывает не на начало самой структуры, 
а на указатель на буфера.
В libstdc++-v3\textbackslash{}include\textbackslash{}bits\textbackslash{}basic\_string.h 
мы можем прочитать что это сделано для удобства отладки:

\begin{lstlisting}
   *  The reason you want _M_data pointing to the character %array and
   *  not the _Rep is so that the debugger can see the string
   *  contents. (Probably we should add a non-inline member to get
   *  the _Rep for the debugger to use, so users can check the actual
   *  string length.)
\end{lstlisting}

\href{http://go.yurichev.com/17085}{исходный код basic\_string.h}

В нашем примере мы учитываем это:

\lstinputlisting[caption=пример для GCC,style=customc]{\CURPATH/STL/string/GCC_RU.cpp}

Нужны еще небольшие хаки чтобы сымитировать типичную ошибку, которую мы уже видели выше, из-за
более ужесточенной проверки типов в GCC, тем не менее, printf() работает и здесь без c\_str().

\myparagraph{Чуть более сложный пример}

\lstinputlisting[style=customc]{\CURPATH/STL/string/3.cpp}

\lstinputlisting[caption=MSVC 2012,style=customasmx86]{\CURPATH/STL/string/3_MSVC_RU.asm}

Собственно, компилятор не конструирует строки статически: да в общем-то и как
это возможно, если буфер с ней нужно хранить в \glslink{heap}{куче}?

Вместо этого в сегменте данных хранятся обычные \ac{ASCIIZ}-строки, а позже, во время выполнения, 
при помощи метода \q{assign}, конструируются строки s1 и s2
.
При помощи \TT{operator+}, создается строка s3.

Обратите внимание на то что вызов метода c\_str() отсутствует,
потому что его код достаточно короткий и компилятор вставил его прямо здесь:
если строка короче 16-и байт, то в регистре EAX остается указатель на буфер,
а если длиннее, то из этого же места достается адрес на буфер расположенный в \glslink{heap}{куче}.

Далее следуют вызовы трех деструкторов, причем, они вызываются только если строка длиннее 16-и байт:
тогда нужно освободить буфера в \glslink{heap}{куче}.
В противном случае, так как все три объекта std::string хранятся в стеке,
они освобождаются автоматически после выхода из функции.

Следовательно, работа с короткими строками более быстрая из-за м\'{е}ньшего обращения к \glslink{heap}{куче}.

Код на GCC даже проще (из-за того, что в GCC, как мы уже видели, не реализована возможность хранить короткую
строку прямо в структуре):

% TODO1 comment each function meaning
\lstinputlisting[caption=GCC 4.8.1,style=customasmx86]{\CURPATH/STL/string/3_GCC_RU.s}

Можно заметить, что в деструкторы передается не указатель на объект,
а указатель на место за 12 байт (или 3 слова) перед ним, то есть, на настоящее начало структуры.

\myparagraph{std::string как глобальная переменная}
\label{sec:std_string_as_global_variable}

Опытные программисты на \Cpp знают, что глобальные переменные \ac{STL}-типов вполне можно объявлять.

Да, действительно:

\lstinputlisting[style=customc]{\CURPATH/STL/string/5.cpp}

Но как и где будет вызываться конструктор \TT{std::string}?

На самом деле, эта переменная будет инициализирована даже перед началом \main.

\lstinputlisting[caption=MSVC 2012: здесь конструируется глобальная переменная{,} а также регистрируется её деструктор,style=customasmx86]{\CURPATH/STL/string/5_MSVC_p2.asm}

\lstinputlisting[caption=MSVC 2012: здесь глобальная переменная используется в \main,style=customasmx86]{\CURPATH/STL/string/5_MSVC_p1.asm}

\lstinputlisting[caption=MSVC 2012: эта функция-деструктор вызывается перед выходом,style=customasmx86]{\CURPATH/STL/string/5_MSVC_p3.asm}

\myindex{\CStandardLibrary!atexit()}
В реальности, из \ac{CRT}, еще до вызова main(), вызывается специальная функция,
в которой перечислены все конструкторы подобных переменных.
Более того: при помощи atexit() регистрируется функция, которая будет вызвана в конце работы программы:
в этой функции компилятор собирает вызовы деструкторов всех подобных глобальных переменных.

GCC работает похожим образом:

\lstinputlisting[caption=GCC 4.8.1,style=customasmx86]{\CURPATH/STL/string/5_GCC.s}

Но он не выделяет отдельной функции в которой будут собраны деструкторы: 
каждый деструктор передается в atexit() по одному.

% TODO а если глобальная STL-переменная в другом модуле? надо проверить.

}
\DE{\subsection{Einfachste XOR-Verschlüsselung überhaupt}

Ich habe einmal eine Software gesehen, bei der alle Debugging-Ausgaben mit XOR mit dem Wert 3
verschlüsselt wurden. Mit anderen Worten, die beiden niedrigsten Bits aller Buchstaben wurden invertiert.

``Hello, world'' wurde zu ``Kfool/\#tlqog'':

\begin{lstlisting}
#!/usr/bin/python

msg="Hello, world!"

print "".join(map(lambda x: chr(ord(x)^3), msg))
\end{lstlisting}

Das ist eine ziemlich interessante Verschlüsselung (oder besser eine Verschleierung),
weil sie zwei wichtige Eigenschaften hat:
1) es ist eine einzige Funktion zum Verschlüsseln und entschlüsseln, sie muss nur wiederholt angewendet werden
2) die entstehenden Buchstaben befinden sich im druckbaren Bereich, also die ganze Zeichenkette kann ohne
Escape-Symbole im Code verwendet werden.

Die zweite Eigenschaft nutzt die Tatsache, dass alle druckbaren Zeichen in Reihen organisiert sind: 0x2x-0x7x,
und wenn die beiden niederwertigsten Bits invertiert werden, wird der Buchstabe um eine oder drei Stellen nach
links oder rechts \IT{verschoben}, aber niemals in eine andere Reihe:

\begin{figure}[H]
\centering
\includegraphics[width=0.7\textwidth]{ascii_clean.png}
\caption{7-Bit \ac{ASCII} Tabelle in Emacs}
\end{figure}

\dots mit dem Zeichen 0x7F als einziger Ausnahme.

Im Folgenden werden also beispielsweise die Zeichen A-Z \IT{verschlüsselt}:

\begin{lstlisting}
#!/usr/bin/python

msg="@ABCDEFGHIJKLMNO"

print "".join(map(lambda x: chr(ord(x)^3), msg))
\end{lstlisting}

Ergebnis:
% FIXME \verb  --  relevant comment for German?
\begin{lstlisting}
CBA@GFEDKJIHONML
\end{lstlisting}

Es sieht so aus als würden die Zeichen ``@'' und ``C'' sowie ``B'' und ``A'' vertauscht werden.

Hier ist noch ein interessantes Beispiel, in dem gezeigt wird, wie die Eigenschaften von XOR
ausgenutzt werden können: Exakt den gleichen Effekt, dass druckbare Zeichen auch druckbar bleiben,
kann man dadurch erzielen, dass irgendeine Kombination der niedrigsten vier Bits invertiert wird.
}

\EN{\section{Returning Values}
\label{ret_val_func}

Another simple function is the one that simply returns a constant value:

\lstinputlisting[caption=\EN{\CCpp Code},style=customc]{patterns/011_ret/1.c}

Let's compile it.

\subsection{x86}

Here's what both the GCC and MSVC compilers produce (with optimization) on the x86 platform:

\lstinputlisting[caption=\Optimizing GCC/MSVC (\assemblyOutput),style=customasmx86]{patterns/011_ret/1.s}

\myindex{x86!\Instructions!RET}
There are just two instructions: the first places the value 123 into the \EAX register,
which is used by convention for storing the return
value, and the second one is \RET, which returns execution to the \gls{caller}.

The caller will take the result from the \EAX register.

\subsection{ARM}

There are a few differences on the ARM platform:

\lstinputlisting[caption=\OptimizingKeilVI (\ARMMode) ASM Output,style=customasmARM]{patterns/011_ret/1_Keil_ARM_O3.s}

ARM uses the register \Reg{0} for returning the results of functions, so 123 is copied into \Reg{0}.

\myindex{ARM!\Instructions!MOV}
\myindex{x86!\Instructions!MOV}
It is worth noting that \MOV is a misleading name for the instruction in both the x86 and ARM \ac{ISA}s.

The data is not in fact \IT{moved}, but \IT{copied}.

\subsection{MIPS}

\label{MIPS_leaf_function_ex1}

The GCC assembly output below lists registers by number:

\lstinputlisting[caption=\Optimizing GCC 4.4.5 (\assemblyOutput),style=customasmMIPS]{patterns/011_ret/MIPS.s}

\dots while \IDA does it by their pseudo names:

\lstinputlisting[caption=\Optimizing GCC 4.4.5 (IDA),style=customasmMIPS]{patterns/011_ret/MIPS_IDA.lst}

The \$2 (or \$V0) register is used to store the function's return value.
\myindex{MIPS!\Pseudoinstructions!LI}
\INS{LI} stands for ``Load Immediate'' and is the MIPS equivalent to \MOV.

\myindex{MIPS!\Instructions!J}
The other instruction is the jump instruction (J or JR) which returns the execution flow to the \gls{caller}.

\myindex{MIPS!Branch delay slot}
You might be wondering why the positions of the load instruction (LI) and the jump instruction (J or JR) are swapped. This is due to a \ac{RISC} feature called ``branch delay slot''.

The reason this happens is a quirk in the architecture of some RISC \ac{ISA}s and isn't important for our
purposes---we must simply keep in mind that in MIPS, the instruction following a jump or branch instruction
is executed \IT{before} the jump/branch instruction itself.

As a consequence, branch instructions always swap places with the instruction executed immediately beforehand.


In practice, functions which merely return 1 (\IT{true}) or 0 (\IT{false}) are very frequent.

The smallest ever of the standard UNIX utilities, \IT{/bin/true} and \IT{/bin/false} return 0 and 1 respectively, as an exit code.
(Zero as an exit code usually means success, non-zero means error.)
}
\RU{\subsubsection{std::string}
\myindex{\Cpp!STL!std::string}
\label{std_string}

\myparagraph{Как устроена структура}

Многие строковые библиотеки \InSqBrackets{\CNotes 2.2} обеспечивают структуру содержащую ссылку 
на буфер собственно со строкой, переменная всегда содержащую длину строки 
(что очень удобно для массы функций \InSqBrackets{\CNotes 2.2.1}) и переменную содержащую текущий размер буфера.

Строка в буфере обыкновенно оканчивается нулем: это для того чтобы указатель на буфер можно было
передавать в функции требующие на вход обычную сишную \ac{ASCIIZ}-строку.

Стандарт \Cpp не описывает, как именно нужно реализовывать std::string,
но, как правило, они реализованы как описано выше, с небольшими дополнениями.

Строки в \Cpp это не класс (как, например, QString в Qt), а темплейт (basic\_string), 
это сделано для того чтобы поддерживать 
строки содержащие разного типа символы: как минимум \Tchar и \IT{wchar\_t}.

Так что, std::string это класс с базовым типом \Tchar.

А std::wstring это класс с базовым типом \IT{wchar\_t}.

\mysubparagraph{MSVC}

В реализации MSVC, вместо ссылки на буфер может содержаться сам буфер (если строка короче 16-и символов).

Это означает, что каждая короткая строка будет занимать в памяти по крайней мере $16 + 4 + 4 = 24$ 
байт для 32-битной среды либо $16 + 8 + 8 = 32$ 
байта в 64-битной, а если строка длиннее 16-и символов, то прибавьте еще длину самой строки.

\lstinputlisting[caption=пример для MSVC,style=customc]{\CURPATH/STL/string/MSVC_RU.cpp}

Собственно, из этого исходника почти всё ясно.

Несколько замечаний:

Если строка короче 16-и символов, 
то отдельный буфер для строки в \glslink{heap}{куче} выделяться не будет.

Это удобно потому что на практике, основная часть строк действительно короткие.
Вероятно, разработчики в Microsoft выбрали размер в 16 символов как разумный баланс.

Теперь очень важный момент в конце функции main(): мы не пользуемся методом c\_str(), тем не менее,
если это скомпилировать и запустить, то обе строки появятся в консоли!

Работает это вот почему.

В первом случае строка короче 16-и символов и в начале объекта std::string (его можно рассматривать
просто как структуру) расположен буфер с этой строкой.
\printf трактует указатель как указатель на массив символов оканчивающийся нулем и поэтому всё работает.

Вывод второй строки (длиннее 16-и символов) даже еще опаснее: это вообще типичная программистская ошибка 
(или опечатка), забыть дописать c\_str().
Это работает потому что в это время в начале структуры расположен указатель на буфер.
Это может надолго остаться незамеченным: до тех пока там не появится строка 
короче 16-и символов, тогда процесс упадет.

\mysubparagraph{GCC}

В реализации GCC в структуре есть еще одна переменная --- reference count.

Интересно, что указатель на экземпляр класса std::string в GCC указывает не на начало самой структуры, 
а на указатель на буфера.
В libstdc++-v3\textbackslash{}include\textbackslash{}bits\textbackslash{}basic\_string.h 
мы можем прочитать что это сделано для удобства отладки:

\begin{lstlisting}
   *  The reason you want _M_data pointing to the character %array and
   *  not the _Rep is so that the debugger can see the string
   *  contents. (Probably we should add a non-inline member to get
   *  the _Rep for the debugger to use, so users can check the actual
   *  string length.)
\end{lstlisting}

\href{http://go.yurichev.com/17085}{исходный код basic\_string.h}

В нашем примере мы учитываем это:

\lstinputlisting[caption=пример для GCC,style=customc]{\CURPATH/STL/string/GCC_RU.cpp}

Нужны еще небольшие хаки чтобы сымитировать типичную ошибку, которую мы уже видели выше, из-за
более ужесточенной проверки типов в GCC, тем не менее, printf() работает и здесь без c\_str().

\myparagraph{Чуть более сложный пример}

\lstinputlisting[style=customc]{\CURPATH/STL/string/3.cpp}

\lstinputlisting[caption=MSVC 2012,style=customasmx86]{\CURPATH/STL/string/3_MSVC_RU.asm}

Собственно, компилятор не конструирует строки статически: да в общем-то и как
это возможно, если буфер с ней нужно хранить в \glslink{heap}{куче}?

Вместо этого в сегменте данных хранятся обычные \ac{ASCIIZ}-строки, а позже, во время выполнения, 
при помощи метода \q{assign}, конструируются строки s1 и s2
.
При помощи \TT{operator+}, создается строка s3.

Обратите внимание на то что вызов метода c\_str() отсутствует,
потому что его код достаточно короткий и компилятор вставил его прямо здесь:
если строка короче 16-и байт, то в регистре EAX остается указатель на буфер,
а если длиннее, то из этого же места достается адрес на буфер расположенный в \glslink{heap}{куче}.

Далее следуют вызовы трех деструкторов, причем, они вызываются только если строка длиннее 16-и байт:
тогда нужно освободить буфера в \glslink{heap}{куче}.
В противном случае, так как все три объекта std::string хранятся в стеке,
они освобождаются автоматически после выхода из функции.

Следовательно, работа с короткими строками более быстрая из-за м\'{е}ньшего обращения к \glslink{heap}{куче}.

Код на GCC даже проще (из-за того, что в GCC, как мы уже видели, не реализована возможность хранить короткую
строку прямо в структуре):

% TODO1 comment each function meaning
\lstinputlisting[caption=GCC 4.8.1,style=customasmx86]{\CURPATH/STL/string/3_GCC_RU.s}

Можно заметить, что в деструкторы передается не указатель на объект,
а указатель на место за 12 байт (или 3 слова) перед ним, то есть, на настоящее начало структуры.

\myparagraph{std::string как глобальная переменная}
\label{sec:std_string_as_global_variable}

Опытные программисты на \Cpp знают, что глобальные переменные \ac{STL}-типов вполне можно объявлять.

Да, действительно:

\lstinputlisting[style=customc]{\CURPATH/STL/string/5.cpp}

Но как и где будет вызываться конструктор \TT{std::string}?

На самом деле, эта переменная будет инициализирована даже перед началом \main.

\lstinputlisting[caption=MSVC 2012: здесь конструируется глобальная переменная{,} а также регистрируется её деструктор,style=customasmx86]{\CURPATH/STL/string/5_MSVC_p2.asm}

\lstinputlisting[caption=MSVC 2012: здесь глобальная переменная используется в \main,style=customasmx86]{\CURPATH/STL/string/5_MSVC_p1.asm}

\lstinputlisting[caption=MSVC 2012: эта функция-деструктор вызывается перед выходом,style=customasmx86]{\CURPATH/STL/string/5_MSVC_p3.asm}

\myindex{\CStandardLibrary!atexit()}
В реальности, из \ac{CRT}, еще до вызова main(), вызывается специальная функция,
в которой перечислены все конструкторы подобных переменных.
Более того: при помощи atexit() регистрируется функция, которая будет вызвана в конце работы программы:
в этой функции компилятор собирает вызовы деструкторов всех подобных глобальных переменных.

GCC работает похожим образом:

\lstinputlisting[caption=GCC 4.8.1,style=customasmx86]{\CURPATH/STL/string/5_GCC.s}

Но он не выделяет отдельной функции в которой будут собраны деструкторы: 
каждый деструктор передается в atexit() по одному.

% TODO а если глобальная STL-переменная в другом модуле? надо проверить.

}

\EN{\section{Returning Values}
\label{ret_val_func}

Another simple function is the one that simply returns a constant value:

\lstinputlisting[caption=\EN{\CCpp Code},style=customc]{patterns/011_ret/1.c}

Let's compile it.

\subsection{x86}

Here's what both the GCC and MSVC compilers produce (with optimization) on the x86 platform:

\lstinputlisting[caption=\Optimizing GCC/MSVC (\assemblyOutput),style=customasmx86]{patterns/011_ret/1.s}

\myindex{x86!\Instructions!RET}
There are just two instructions: the first places the value 123 into the \EAX register,
which is used by convention for storing the return
value, and the second one is \RET, which returns execution to the \gls{caller}.

The caller will take the result from the \EAX register.

\subsection{ARM}

There are a few differences on the ARM platform:

\lstinputlisting[caption=\OptimizingKeilVI (\ARMMode) ASM Output,style=customasmARM]{patterns/011_ret/1_Keil_ARM_O3.s}

ARM uses the register \Reg{0} for returning the results of functions, so 123 is copied into \Reg{0}.

\myindex{ARM!\Instructions!MOV}
\myindex{x86!\Instructions!MOV}
It is worth noting that \MOV is a misleading name for the instruction in both the x86 and ARM \ac{ISA}s.

The data is not in fact \IT{moved}, but \IT{copied}.

\subsection{MIPS}

\label{MIPS_leaf_function_ex1}

The GCC assembly output below lists registers by number:

\lstinputlisting[caption=\Optimizing GCC 4.4.5 (\assemblyOutput),style=customasmMIPS]{patterns/011_ret/MIPS.s}

\dots while \IDA does it by their pseudo names:

\lstinputlisting[caption=\Optimizing GCC 4.4.5 (IDA),style=customasmMIPS]{patterns/011_ret/MIPS_IDA.lst}

The \$2 (or \$V0) register is used to store the function's return value.
\myindex{MIPS!\Pseudoinstructions!LI}
\INS{LI} stands for ``Load Immediate'' and is the MIPS equivalent to \MOV.

\myindex{MIPS!\Instructions!J}
The other instruction is the jump instruction (J or JR) which returns the execution flow to the \gls{caller}.

\myindex{MIPS!Branch delay slot}
You might be wondering why the positions of the load instruction (LI) and the jump instruction (J or JR) are swapped. This is due to a \ac{RISC} feature called ``branch delay slot''.

The reason this happens is a quirk in the architecture of some RISC \ac{ISA}s and isn't important for our
purposes---we must simply keep in mind that in MIPS, the instruction following a jump or branch instruction
is executed \IT{before} the jump/branch instruction itself.

As a consequence, branch instructions always swap places with the instruction executed immediately beforehand.


In practice, functions which merely return 1 (\IT{true}) or 0 (\IT{false}) are very frequent.

The smallest ever of the standard UNIX utilities, \IT{/bin/true} and \IT{/bin/false} return 0 and 1 respectively, as an exit code.
(Zero as an exit code usually means success, non-zero means error.)
}
\RU{\subsubsection{std::string}
\myindex{\Cpp!STL!std::string}
\label{std_string}

\myparagraph{Как устроена структура}

Многие строковые библиотеки \InSqBrackets{\CNotes 2.2} обеспечивают структуру содержащую ссылку 
на буфер собственно со строкой, переменная всегда содержащую длину строки 
(что очень удобно для массы функций \InSqBrackets{\CNotes 2.2.1}) и переменную содержащую текущий размер буфера.

Строка в буфере обыкновенно оканчивается нулем: это для того чтобы указатель на буфер можно было
передавать в функции требующие на вход обычную сишную \ac{ASCIIZ}-строку.

Стандарт \Cpp не описывает, как именно нужно реализовывать std::string,
но, как правило, они реализованы как описано выше, с небольшими дополнениями.

Строки в \Cpp это не класс (как, например, QString в Qt), а темплейт (basic\_string), 
это сделано для того чтобы поддерживать 
строки содержащие разного типа символы: как минимум \Tchar и \IT{wchar\_t}.

Так что, std::string это класс с базовым типом \Tchar.

А std::wstring это класс с базовым типом \IT{wchar\_t}.

\mysubparagraph{MSVC}

В реализации MSVC, вместо ссылки на буфер может содержаться сам буфер (если строка короче 16-и символов).

Это означает, что каждая короткая строка будет занимать в памяти по крайней мере $16 + 4 + 4 = 24$ 
байт для 32-битной среды либо $16 + 8 + 8 = 32$ 
байта в 64-битной, а если строка длиннее 16-и символов, то прибавьте еще длину самой строки.

\lstinputlisting[caption=пример для MSVC,style=customc]{\CURPATH/STL/string/MSVC_RU.cpp}

Собственно, из этого исходника почти всё ясно.

Несколько замечаний:

Если строка короче 16-и символов, 
то отдельный буфер для строки в \glslink{heap}{куче} выделяться не будет.

Это удобно потому что на практике, основная часть строк действительно короткие.
Вероятно, разработчики в Microsoft выбрали размер в 16 символов как разумный баланс.

Теперь очень важный момент в конце функции main(): мы не пользуемся методом c\_str(), тем не менее,
если это скомпилировать и запустить, то обе строки появятся в консоли!

Работает это вот почему.

В первом случае строка короче 16-и символов и в начале объекта std::string (его можно рассматривать
просто как структуру) расположен буфер с этой строкой.
\printf трактует указатель как указатель на массив символов оканчивающийся нулем и поэтому всё работает.

Вывод второй строки (длиннее 16-и символов) даже еще опаснее: это вообще типичная программистская ошибка 
(или опечатка), забыть дописать c\_str().
Это работает потому что в это время в начале структуры расположен указатель на буфер.
Это может надолго остаться незамеченным: до тех пока там не появится строка 
короче 16-и символов, тогда процесс упадет.

\mysubparagraph{GCC}

В реализации GCC в структуре есть еще одна переменная --- reference count.

Интересно, что указатель на экземпляр класса std::string в GCC указывает не на начало самой структуры, 
а на указатель на буфера.
В libstdc++-v3\textbackslash{}include\textbackslash{}bits\textbackslash{}basic\_string.h 
мы можем прочитать что это сделано для удобства отладки:

\begin{lstlisting}
   *  The reason you want _M_data pointing to the character %array and
   *  not the _Rep is so that the debugger can see the string
   *  contents. (Probably we should add a non-inline member to get
   *  the _Rep for the debugger to use, so users can check the actual
   *  string length.)
\end{lstlisting}

\href{http://go.yurichev.com/17085}{исходный код basic\_string.h}

В нашем примере мы учитываем это:

\lstinputlisting[caption=пример для GCC,style=customc]{\CURPATH/STL/string/GCC_RU.cpp}

Нужны еще небольшие хаки чтобы сымитировать типичную ошибку, которую мы уже видели выше, из-за
более ужесточенной проверки типов в GCC, тем не менее, printf() работает и здесь без c\_str().

\myparagraph{Чуть более сложный пример}

\lstinputlisting[style=customc]{\CURPATH/STL/string/3.cpp}

\lstinputlisting[caption=MSVC 2012,style=customasmx86]{\CURPATH/STL/string/3_MSVC_RU.asm}

Собственно, компилятор не конструирует строки статически: да в общем-то и как
это возможно, если буфер с ней нужно хранить в \glslink{heap}{куче}?

Вместо этого в сегменте данных хранятся обычные \ac{ASCIIZ}-строки, а позже, во время выполнения, 
при помощи метода \q{assign}, конструируются строки s1 и s2
.
При помощи \TT{operator+}, создается строка s3.

Обратите внимание на то что вызов метода c\_str() отсутствует,
потому что его код достаточно короткий и компилятор вставил его прямо здесь:
если строка короче 16-и байт, то в регистре EAX остается указатель на буфер,
а если длиннее, то из этого же места достается адрес на буфер расположенный в \glslink{heap}{куче}.

Далее следуют вызовы трех деструкторов, причем, они вызываются только если строка длиннее 16-и байт:
тогда нужно освободить буфера в \glslink{heap}{куче}.
В противном случае, так как все три объекта std::string хранятся в стеке,
они освобождаются автоматически после выхода из функции.

Следовательно, работа с короткими строками более быстрая из-за м\'{е}ньшего обращения к \glslink{heap}{куче}.

Код на GCC даже проще (из-за того, что в GCC, как мы уже видели, не реализована возможность хранить короткую
строку прямо в структуре):

% TODO1 comment each function meaning
\lstinputlisting[caption=GCC 4.8.1,style=customasmx86]{\CURPATH/STL/string/3_GCC_RU.s}

Можно заметить, что в деструкторы передается не указатель на объект,
а указатель на место за 12 байт (или 3 слова) перед ним, то есть, на настоящее начало структуры.

\myparagraph{std::string как глобальная переменная}
\label{sec:std_string_as_global_variable}

Опытные программисты на \Cpp знают, что глобальные переменные \ac{STL}-типов вполне можно объявлять.

Да, действительно:

\lstinputlisting[style=customc]{\CURPATH/STL/string/5.cpp}

Но как и где будет вызываться конструктор \TT{std::string}?

На самом деле, эта переменная будет инициализирована даже перед началом \main.

\lstinputlisting[caption=MSVC 2012: здесь конструируется глобальная переменная{,} а также регистрируется её деструктор,style=customasmx86]{\CURPATH/STL/string/5_MSVC_p2.asm}

\lstinputlisting[caption=MSVC 2012: здесь глобальная переменная используется в \main,style=customasmx86]{\CURPATH/STL/string/5_MSVC_p1.asm}

\lstinputlisting[caption=MSVC 2012: эта функция-деструктор вызывается перед выходом,style=customasmx86]{\CURPATH/STL/string/5_MSVC_p3.asm}

\myindex{\CStandardLibrary!atexit()}
В реальности, из \ac{CRT}, еще до вызова main(), вызывается специальная функция,
в которой перечислены все конструкторы подобных переменных.
Более того: при помощи atexit() регистрируется функция, которая будет вызвана в конце работы программы:
в этой функции компилятор собирает вызовы деструкторов всех подобных глобальных переменных.

GCC работает похожим образом:

\lstinputlisting[caption=GCC 4.8.1,style=customasmx86]{\CURPATH/STL/string/5_GCC.s}

Но он не выделяет отдельной функции в которой будут собраны деструкторы: 
каждый деструктор передается в atexit() по одному.

% TODO а если глобальная STL-переменная в другом модуле? надо проверить.

}
\DE{\subsection{Einfachste XOR-Verschlüsselung überhaupt}

Ich habe einmal eine Software gesehen, bei der alle Debugging-Ausgaben mit XOR mit dem Wert 3
verschlüsselt wurden. Mit anderen Worten, die beiden niedrigsten Bits aller Buchstaben wurden invertiert.

``Hello, world'' wurde zu ``Kfool/\#tlqog'':

\begin{lstlisting}
#!/usr/bin/python

msg="Hello, world!"

print "".join(map(lambda x: chr(ord(x)^3), msg))
\end{lstlisting}

Das ist eine ziemlich interessante Verschlüsselung (oder besser eine Verschleierung),
weil sie zwei wichtige Eigenschaften hat:
1) es ist eine einzige Funktion zum Verschlüsseln und entschlüsseln, sie muss nur wiederholt angewendet werden
2) die entstehenden Buchstaben befinden sich im druckbaren Bereich, also die ganze Zeichenkette kann ohne
Escape-Symbole im Code verwendet werden.

Die zweite Eigenschaft nutzt die Tatsache, dass alle druckbaren Zeichen in Reihen organisiert sind: 0x2x-0x7x,
und wenn die beiden niederwertigsten Bits invertiert werden, wird der Buchstabe um eine oder drei Stellen nach
links oder rechts \IT{verschoben}, aber niemals in eine andere Reihe:

\begin{figure}[H]
\centering
\includegraphics[width=0.7\textwidth]{ascii_clean.png}
\caption{7-Bit \ac{ASCII} Tabelle in Emacs}
\end{figure}

\dots mit dem Zeichen 0x7F als einziger Ausnahme.

Im Folgenden werden also beispielsweise die Zeichen A-Z \IT{verschlüsselt}:

\begin{lstlisting}
#!/usr/bin/python

msg="@ABCDEFGHIJKLMNO"

print "".join(map(lambda x: chr(ord(x)^3), msg))
\end{lstlisting}

Ergebnis:
% FIXME \verb  --  relevant comment for German?
\begin{lstlisting}
CBA@GFEDKJIHONML
\end{lstlisting}

Es sieht so aus als würden die Zeichen ``@'' und ``C'' sowie ``B'' und ``A'' vertauscht werden.

Hier ist noch ein interessantes Beispiel, in dem gezeigt wird, wie die Eigenschaften von XOR
ausgenutzt werden können: Exakt den gleichen Effekt, dass druckbare Zeichen auch druckbar bleiben,
kann man dadurch erzielen, dass irgendeine Kombination der niedrigsten vier Bits invertiert wird.
}

\EN{\section{Returning Values}
\label{ret_val_func}

Another simple function is the one that simply returns a constant value:

\lstinputlisting[caption=\EN{\CCpp Code},style=customc]{patterns/011_ret/1.c}

Let's compile it.

\subsection{x86}

Here's what both the GCC and MSVC compilers produce (with optimization) on the x86 platform:

\lstinputlisting[caption=\Optimizing GCC/MSVC (\assemblyOutput),style=customasmx86]{patterns/011_ret/1.s}

\myindex{x86!\Instructions!RET}
There are just two instructions: the first places the value 123 into the \EAX register,
which is used by convention for storing the return
value, and the second one is \RET, which returns execution to the \gls{caller}.

The caller will take the result from the \EAX register.

\subsection{ARM}

There are a few differences on the ARM platform:

\lstinputlisting[caption=\OptimizingKeilVI (\ARMMode) ASM Output,style=customasmARM]{patterns/011_ret/1_Keil_ARM_O3.s}

ARM uses the register \Reg{0} for returning the results of functions, so 123 is copied into \Reg{0}.

\myindex{ARM!\Instructions!MOV}
\myindex{x86!\Instructions!MOV}
It is worth noting that \MOV is a misleading name for the instruction in both the x86 and ARM \ac{ISA}s.

The data is not in fact \IT{moved}, but \IT{copied}.

\subsection{MIPS}

\label{MIPS_leaf_function_ex1}

The GCC assembly output below lists registers by number:

\lstinputlisting[caption=\Optimizing GCC 4.4.5 (\assemblyOutput),style=customasmMIPS]{patterns/011_ret/MIPS.s}

\dots while \IDA does it by their pseudo names:

\lstinputlisting[caption=\Optimizing GCC 4.4.5 (IDA),style=customasmMIPS]{patterns/011_ret/MIPS_IDA.lst}

The \$2 (or \$V0) register is used to store the function's return value.
\myindex{MIPS!\Pseudoinstructions!LI}
\INS{LI} stands for ``Load Immediate'' and is the MIPS equivalent to \MOV.

\myindex{MIPS!\Instructions!J}
The other instruction is the jump instruction (J or JR) which returns the execution flow to the \gls{caller}.

\myindex{MIPS!Branch delay slot}
You might be wondering why the positions of the load instruction (LI) and the jump instruction (J or JR) are swapped. This is due to a \ac{RISC} feature called ``branch delay slot''.

The reason this happens is a quirk in the architecture of some RISC \ac{ISA}s and isn't important for our
purposes---we must simply keep in mind that in MIPS, the instruction following a jump or branch instruction
is executed \IT{before} the jump/branch instruction itself.

As a consequence, branch instructions always swap places with the instruction executed immediately beforehand.


In practice, functions which merely return 1 (\IT{true}) or 0 (\IT{false}) are very frequent.

The smallest ever of the standard UNIX utilities, \IT{/bin/true} and \IT{/bin/false} return 0 and 1 respectively, as an exit code.
(Zero as an exit code usually means success, non-zero means error.)
}
\RU{\subsubsection{std::string}
\myindex{\Cpp!STL!std::string}
\label{std_string}

\myparagraph{Как устроена структура}

Многие строковые библиотеки \InSqBrackets{\CNotes 2.2} обеспечивают структуру содержащую ссылку 
на буфер собственно со строкой, переменная всегда содержащую длину строки 
(что очень удобно для массы функций \InSqBrackets{\CNotes 2.2.1}) и переменную содержащую текущий размер буфера.

Строка в буфере обыкновенно оканчивается нулем: это для того чтобы указатель на буфер можно было
передавать в функции требующие на вход обычную сишную \ac{ASCIIZ}-строку.

Стандарт \Cpp не описывает, как именно нужно реализовывать std::string,
но, как правило, они реализованы как описано выше, с небольшими дополнениями.

Строки в \Cpp это не класс (как, например, QString в Qt), а темплейт (basic\_string), 
это сделано для того чтобы поддерживать 
строки содержащие разного типа символы: как минимум \Tchar и \IT{wchar\_t}.

Так что, std::string это класс с базовым типом \Tchar.

А std::wstring это класс с базовым типом \IT{wchar\_t}.

\mysubparagraph{MSVC}

В реализации MSVC, вместо ссылки на буфер может содержаться сам буфер (если строка короче 16-и символов).

Это означает, что каждая короткая строка будет занимать в памяти по крайней мере $16 + 4 + 4 = 24$ 
байт для 32-битной среды либо $16 + 8 + 8 = 32$ 
байта в 64-битной, а если строка длиннее 16-и символов, то прибавьте еще длину самой строки.

\lstinputlisting[caption=пример для MSVC,style=customc]{\CURPATH/STL/string/MSVC_RU.cpp}

Собственно, из этого исходника почти всё ясно.

Несколько замечаний:

Если строка короче 16-и символов, 
то отдельный буфер для строки в \glslink{heap}{куче} выделяться не будет.

Это удобно потому что на практике, основная часть строк действительно короткие.
Вероятно, разработчики в Microsoft выбрали размер в 16 символов как разумный баланс.

Теперь очень важный момент в конце функции main(): мы не пользуемся методом c\_str(), тем не менее,
если это скомпилировать и запустить, то обе строки появятся в консоли!

Работает это вот почему.

В первом случае строка короче 16-и символов и в начале объекта std::string (его можно рассматривать
просто как структуру) расположен буфер с этой строкой.
\printf трактует указатель как указатель на массив символов оканчивающийся нулем и поэтому всё работает.

Вывод второй строки (длиннее 16-и символов) даже еще опаснее: это вообще типичная программистская ошибка 
(или опечатка), забыть дописать c\_str().
Это работает потому что в это время в начале структуры расположен указатель на буфер.
Это может надолго остаться незамеченным: до тех пока там не появится строка 
короче 16-и символов, тогда процесс упадет.

\mysubparagraph{GCC}

В реализации GCC в структуре есть еще одна переменная --- reference count.

Интересно, что указатель на экземпляр класса std::string в GCC указывает не на начало самой структуры, 
а на указатель на буфера.
В libstdc++-v3\textbackslash{}include\textbackslash{}bits\textbackslash{}basic\_string.h 
мы можем прочитать что это сделано для удобства отладки:

\begin{lstlisting}
   *  The reason you want _M_data pointing to the character %array and
   *  not the _Rep is so that the debugger can see the string
   *  contents. (Probably we should add a non-inline member to get
   *  the _Rep for the debugger to use, so users can check the actual
   *  string length.)
\end{lstlisting}

\href{http://go.yurichev.com/17085}{исходный код basic\_string.h}

В нашем примере мы учитываем это:

\lstinputlisting[caption=пример для GCC,style=customc]{\CURPATH/STL/string/GCC_RU.cpp}

Нужны еще небольшие хаки чтобы сымитировать типичную ошибку, которую мы уже видели выше, из-за
более ужесточенной проверки типов в GCC, тем не менее, printf() работает и здесь без c\_str().

\myparagraph{Чуть более сложный пример}

\lstinputlisting[style=customc]{\CURPATH/STL/string/3.cpp}

\lstinputlisting[caption=MSVC 2012,style=customasmx86]{\CURPATH/STL/string/3_MSVC_RU.asm}

Собственно, компилятор не конструирует строки статически: да в общем-то и как
это возможно, если буфер с ней нужно хранить в \glslink{heap}{куче}?

Вместо этого в сегменте данных хранятся обычные \ac{ASCIIZ}-строки, а позже, во время выполнения, 
при помощи метода \q{assign}, конструируются строки s1 и s2
.
При помощи \TT{operator+}, создается строка s3.

Обратите внимание на то что вызов метода c\_str() отсутствует,
потому что его код достаточно короткий и компилятор вставил его прямо здесь:
если строка короче 16-и байт, то в регистре EAX остается указатель на буфер,
а если длиннее, то из этого же места достается адрес на буфер расположенный в \glslink{heap}{куче}.

Далее следуют вызовы трех деструкторов, причем, они вызываются только если строка длиннее 16-и байт:
тогда нужно освободить буфера в \glslink{heap}{куче}.
В противном случае, так как все три объекта std::string хранятся в стеке,
они освобождаются автоматически после выхода из функции.

Следовательно, работа с короткими строками более быстрая из-за м\'{е}ньшего обращения к \glslink{heap}{куче}.

Код на GCC даже проще (из-за того, что в GCC, как мы уже видели, не реализована возможность хранить короткую
строку прямо в структуре):

% TODO1 comment each function meaning
\lstinputlisting[caption=GCC 4.8.1,style=customasmx86]{\CURPATH/STL/string/3_GCC_RU.s}

Можно заметить, что в деструкторы передается не указатель на объект,
а указатель на место за 12 байт (или 3 слова) перед ним, то есть, на настоящее начало структуры.

\myparagraph{std::string как глобальная переменная}
\label{sec:std_string_as_global_variable}

Опытные программисты на \Cpp знают, что глобальные переменные \ac{STL}-типов вполне можно объявлять.

Да, действительно:

\lstinputlisting[style=customc]{\CURPATH/STL/string/5.cpp}

Но как и где будет вызываться конструктор \TT{std::string}?

На самом деле, эта переменная будет инициализирована даже перед началом \main.

\lstinputlisting[caption=MSVC 2012: здесь конструируется глобальная переменная{,} а также регистрируется её деструктор,style=customasmx86]{\CURPATH/STL/string/5_MSVC_p2.asm}

\lstinputlisting[caption=MSVC 2012: здесь глобальная переменная используется в \main,style=customasmx86]{\CURPATH/STL/string/5_MSVC_p1.asm}

\lstinputlisting[caption=MSVC 2012: эта функция-деструктор вызывается перед выходом,style=customasmx86]{\CURPATH/STL/string/5_MSVC_p3.asm}

\myindex{\CStandardLibrary!atexit()}
В реальности, из \ac{CRT}, еще до вызова main(), вызывается специальная функция,
в которой перечислены все конструкторы подобных переменных.
Более того: при помощи atexit() регистрируется функция, которая будет вызвана в конце работы программы:
в этой функции компилятор собирает вызовы деструкторов всех подобных глобальных переменных.

GCC работает похожим образом:

\lstinputlisting[caption=GCC 4.8.1,style=customasmx86]{\CURPATH/STL/string/5_GCC.s}

Но он не выделяет отдельной функции в которой будут собраны деструкторы: 
каждый деструктор передается в atexit() по одному.

% TODO а если глобальная STL-переменная в другом модуле? надо проверить.

}
\DE{\subsection{Einfachste XOR-Verschlüsselung überhaupt}

Ich habe einmal eine Software gesehen, bei der alle Debugging-Ausgaben mit XOR mit dem Wert 3
verschlüsselt wurden. Mit anderen Worten, die beiden niedrigsten Bits aller Buchstaben wurden invertiert.

``Hello, world'' wurde zu ``Kfool/\#tlqog'':

\begin{lstlisting}
#!/usr/bin/python

msg="Hello, world!"

print "".join(map(lambda x: chr(ord(x)^3), msg))
\end{lstlisting}

Das ist eine ziemlich interessante Verschlüsselung (oder besser eine Verschleierung),
weil sie zwei wichtige Eigenschaften hat:
1) es ist eine einzige Funktion zum Verschlüsseln und entschlüsseln, sie muss nur wiederholt angewendet werden
2) die entstehenden Buchstaben befinden sich im druckbaren Bereich, also die ganze Zeichenkette kann ohne
Escape-Symbole im Code verwendet werden.

Die zweite Eigenschaft nutzt die Tatsache, dass alle druckbaren Zeichen in Reihen organisiert sind: 0x2x-0x7x,
und wenn die beiden niederwertigsten Bits invertiert werden, wird der Buchstabe um eine oder drei Stellen nach
links oder rechts \IT{verschoben}, aber niemals in eine andere Reihe:

\begin{figure}[H]
\centering
\includegraphics[width=0.7\textwidth]{ascii_clean.png}
\caption{7-Bit \ac{ASCII} Tabelle in Emacs}
\end{figure}

\dots mit dem Zeichen 0x7F als einziger Ausnahme.

Im Folgenden werden also beispielsweise die Zeichen A-Z \IT{verschlüsselt}:

\begin{lstlisting}
#!/usr/bin/python

msg="@ABCDEFGHIJKLMNO"

print "".join(map(lambda x: chr(ord(x)^3), msg))
\end{lstlisting}

Ergebnis:
% FIXME \verb  --  relevant comment for German?
\begin{lstlisting}
CBA@GFEDKJIHONML
\end{lstlisting}

Es sieht so aus als würden die Zeichen ``@'' und ``C'' sowie ``B'' und ``A'' vertauscht werden.

Hier ist noch ein interessantes Beispiel, in dem gezeigt wird, wie die Eigenschaften von XOR
ausgenutzt werden können: Exakt den gleichen Effekt, dass druckbare Zeichen auch druckbar bleiben,
kann man dadurch erzielen, dass irgendeine Kombination der niedrigsten vier Bits invertiert wird.
}

\ifdefined\SPANISH
\chapter{Patrones de código}
\fi % SPANISH

\ifdefined\GERMAN
\chapter{Code-Muster}
\fi % GERMAN

\ifdefined\ENGLISH
\chapter{Code Patterns}
\fi % ENGLISH

\ifdefined\ITALIAN
\chapter{Forme di codice}
\fi % ITALIAN

\ifdefined\RUSSIAN
\chapter{Образцы кода}
\fi % RUSSIAN

\ifdefined\BRAZILIAN
\chapter{Padrões de códigos}
\fi % BRAZILIAN

\ifdefined\THAI
\chapter{รูปแบบของโค้ด}
\fi % THAI

\ifdefined\FRENCH
\chapter{Modèle de code}
\fi % FRENCH

\ifdefined\POLISH
\chapter{\PLph{}}
\fi % POLISH

% sections
\EN{\input{patterns/patterns_opt_dbg_EN}}
\ES{\input{patterns/patterns_opt_dbg_ES}}
\ITA{\input{patterns/patterns_opt_dbg_ITA}}
\PTBR{\input{patterns/patterns_opt_dbg_PTBR}}
\RU{\input{patterns/patterns_opt_dbg_RU}}
\THA{\input{patterns/patterns_opt_dbg_THA}}
\DE{\input{patterns/patterns_opt_dbg_DE}}
\FR{\input{patterns/patterns_opt_dbg_FR}}
\PL{\input{patterns/patterns_opt_dbg_PL}}

\RU{\section{Некоторые базовые понятия}}
\EN{\section{Some basics}}
\DE{\section{Einige Grundlagen}}
\FR{\section{Quelques bases}}
\ES{\section{\ESph{}}}
\ITA{\section{Alcune basi teoriche}}
\PTBR{\section{\PTBRph{}}}
\THA{\section{\THAph{}}}
\PL{\section{\PLph{}}}

% sections:
\EN{\input{patterns/intro_CPU_ISA_EN}}
\ES{\input{patterns/intro_CPU_ISA_ES}}
\ITA{\input{patterns/intro_CPU_ISA_ITA}}
\PTBR{\input{patterns/intro_CPU_ISA_PTBR}}
\RU{\input{patterns/intro_CPU_ISA_RU}}
\DE{\input{patterns/intro_CPU_ISA_DE}}
\FR{\input{patterns/intro_CPU_ISA_FR}}
\PL{\input{patterns/intro_CPU_ISA_PL}}

\EN{\input{patterns/numeral_EN}}
\RU{\input{patterns/numeral_RU}}
\ITA{\input{patterns/numeral_ITA}}
\DE{\input{patterns/numeral_DE}}
\FR{\input{patterns/numeral_FR}}
\PL{\input{patterns/numeral_PL}}

% chapters
\input{patterns/00_empty/main}
\input{patterns/011_ret/main}
\input{patterns/01_helloworld/main}
\input{patterns/015_prolog_epilogue/main}
\input{patterns/02_stack/main}
\input{patterns/03_printf/main}
\input{patterns/04_scanf/main}
\input{patterns/05_passing_arguments/main}
\input{patterns/06_return_results/main}
\input{patterns/061_pointers/main}
\input{patterns/065_GOTO/main}
\input{patterns/07_jcc/main}
\input{patterns/08_switch/main}
\input{patterns/09_loops/main}
\input{patterns/10_strings/main}
\input{patterns/11_arith_optimizations/main}
\input{patterns/12_FPU/main}
\input{patterns/13_arrays/main}
\input{patterns/14_bitfields/main}
\EN{\input{patterns/145_LCG/main_EN}}
\RU{\input{patterns/145_LCG/main_RU}}
\input{patterns/15_structs/main}
\input{patterns/17_unions/main}
\input{patterns/18_pointers_to_functions/main}
\input{patterns/185_64bit_in_32_env/main}

\EN{\input{patterns/19_SIMD/main_EN}}
\RU{\input{patterns/19_SIMD/main_RU}}
\DE{\input{patterns/19_SIMD/main_DE}}

\EN{\input{patterns/20_x64/main_EN}}
\RU{\input{patterns/20_x64/main_RU}}

\EN{\input{patterns/205_floating_SIMD/main_EN}}
\RU{\input{patterns/205_floating_SIMD/main_RU}}
\DE{\input{patterns/205_floating_SIMD/main_DE}}

\EN{\input{patterns/ARM/main_EN}}
\RU{\input{patterns/ARM/main_RU}}
\DE{\input{patterns/ARM/main_DE}}

\input{patterns/MIPS/main}


\ifdefined\SPANISH
\chapter{Patrones de código}
\fi % SPANISH

\ifdefined\GERMAN
\chapter{Code-Muster}
\fi % GERMAN

\ifdefined\ENGLISH
\chapter{Code Patterns}
\fi % ENGLISH

\ifdefined\ITALIAN
\chapter{Forme di codice}
\fi % ITALIAN

\ifdefined\RUSSIAN
\chapter{Образцы кода}
\fi % RUSSIAN

\ifdefined\BRAZILIAN
\chapter{Padrões de códigos}
\fi % BRAZILIAN

\ifdefined\THAI
\chapter{รูปแบบของโค้ด}
\fi % THAI

\ifdefined\FRENCH
\chapter{Modèle de code}
\fi % FRENCH

\ifdefined\POLISH
\chapter{\PLph{}}
\fi % POLISH

% sections
\EN{\section{The method}

When the author of this book first started learning C and, later, \Cpp, he used to write small pieces of code, compile them,
and then look at the assembly language output. This made it very easy for him to understand what was going on in the code that he had written.
\footnote{In fact, he still does this when he can't understand what a particular bit of code does.}.
He did this so many times that the relationship between the \CCpp code and what the compiler produced was imprinted deeply in his mind.
It's now easy for him to imagine instantly a rough outline of a C code's appearance and function.
Perhaps this technique could be helpful for others.

%There are a lot of examples for both x86/x64 and ARM.
%Those who already familiar with one of architectures, may freely skim over pages.

By the way, there is a great website where you can do the same, with various compilers, instead of installing them on your box.
You can use it as well: \url{https://gcc.godbolt.org/}.

\section*{\Exercises}

When the author of this book studied assembly language, he also often compiled small C functions and then rewrote
them gradually to assembly, trying to make their code as short as possible.
This probably is not worth doing in real-world scenarios today,
because it's hard to compete with the latest compilers in terms of efficiency. It is, however, a very good way to gain a better understanding of assembly.
Feel free, therefore, to take any assembly code from this book and try to make it shorter.
However, don't forget to test what you have written.

% rewrote to show that debug\release and optimisations levels are orthogonal concepts.
\section*{Optimization levels and debug information}

Source code can be compiled by different compilers with various optimization levels.
A typical compiler has about three such levels, where level zero means that optimization is completely disabled.
Optimization can also be targeted towards code size or code speed.
A non-optimizing compiler is faster and produces more understandable (albeit verbose) code,
whereas an optimizing compiler is slower and tries to produce code that runs faster (but is not necessarily more compact).
In addition to optimization levels, a compiler can include some debug information in the resulting file,
producing code that is easy to debug.
One of the important features of the ´debug' code is that it might contain links
between each line of the source code and its respective machine code address.
Optimizing compilers, on the other hand, tend to produce output where entire lines of source code
can be optimized away and thus not even be present in the resulting machine code.
Reverse engineers can encounter either version, simply because some developers turn on the compiler's optimization flags and others do not.
Because of this, we'll try to work on examples of both debug and release versions of the code featured in this book, wherever possible.

Sometimes some pretty ancient compilers are used in this book, in order to get the shortest (or simplest) possible code snippet.
}
\ES{\input{patterns/patterns_opt_dbg_ES}}
\ITA{\input{patterns/patterns_opt_dbg_ITA}}
\PTBR{\input{patterns/patterns_opt_dbg_PTBR}}
\RU{\input{patterns/patterns_opt_dbg_RU}}
\THA{\input{patterns/patterns_opt_dbg_THA}}
\DE{\input{patterns/patterns_opt_dbg_DE}}
\FR{\input{patterns/patterns_opt_dbg_FR}}
\PL{\input{patterns/patterns_opt_dbg_PL}}

\RU{\section{Некоторые базовые понятия}}
\EN{\section{Some basics}}
\DE{\section{Einige Grundlagen}}
\FR{\section{Quelques bases}}
\ES{\section{\ESph{}}}
\ITA{\section{Alcune basi teoriche}}
\PTBR{\section{\PTBRph{}}}
\THA{\section{\THAph{}}}
\PL{\section{\PLph{}}}

% sections:
\EN{\input{patterns/intro_CPU_ISA_EN}}
\ES{\input{patterns/intro_CPU_ISA_ES}}
\ITA{\input{patterns/intro_CPU_ISA_ITA}}
\PTBR{\input{patterns/intro_CPU_ISA_PTBR}}
\RU{\input{patterns/intro_CPU_ISA_RU}}
\DE{\input{patterns/intro_CPU_ISA_DE}}
\FR{\input{patterns/intro_CPU_ISA_FR}}
\PL{\input{patterns/intro_CPU_ISA_PL}}

\EN{\subsection{Numeral Systems}

Humans have become accustomed to a decimal numeral system, probably because almost everyone has 10 fingers.
Nevertheless, the number \q{10} has no significant meaning in science and mathematics.
The natural numeral system in digital electronics is binary: 0 is for an absence of current in the wire, and 1 for presence.
10 in binary is 2 in decimal, 100 in binary is 4 in decimal, and so on.

% This sentence is a bit unweildy - maybe try 'Our ten-digit system would be described as having a radix...' - Renaissance
If the numeral system has 10 digits, it has a \IT{radix} (or \IT{base}) of 10.
The binary numeral system has a \IT{radix} of 2.

Important things to recall:

1) A \IT{number} is a number, while a \IT{digit} is a term from writing systems, and is usually one character

% The original is 'number' is not changed; I think the intent is value, and changed it - Renaissance
2) The value of a number does not change when converted to another radix; only the writing notation for that value has changed (and therefore the way of representing it in \ac{RAM}).

\subsection{Converting From One Radix To Another}

Positional notation is used almost every numerical system. This means that a digit has weight relative to where it is placed inside of the larger number.
If 2 is placed at the rightmost place, it's 2, but if it's placed one digit before rightmost, it's 20.

What does $1234$ stand for?

$10^3 \cdot 1 + 10^2 \cdot 2 + 10^1 \cdot 3 + 1 \cdot 4 = 1234$ or
$1000 \cdot 1 + 100 \cdot 2 + 10 \cdot 3 + 4 = 1234$

It's the same story for binary numbers, but the base is 2 instead of 10.
What does 0b101011 stand for?

$2^5 \cdot 1 + 2^4 \cdot 0 + 2^3 \cdot 1 + 2^2 \cdot 0 + 2^1 \cdot 1 + 2^0 \cdot 1 = 43$ or
$32 \cdot 1 + 16 \cdot 0 + 8 \cdot 1 + 4 \cdot 0 + 2 \cdot 1 + 1 = 43$

There is such a thing as non-positional notation, such as the Roman numeral system.
\footnote{About numeric system evolution, see \InSqBrackets{\TAOCPvolII{}, 195--213.}}.
% Maybe add a sentence to fill in that X is always 10, and is therefore non-positional, even though putting an I before subtracts and after adds, and is in that sense positional
Perhaps, humankind switched to positional notation because it's easier to do basic operations (addition, multiplication, etc.) on paper by hand.

Binary numbers can be added, subtracted and so on in the very same as taught in schools, but only 2 digits are available.

Binary numbers are bulky when represented in source code and dumps, so that is where the hexadecimal numeral system can be useful.
A hexadecimal radix uses the digits 0..9, and also 6 Latin characters: A..F.
Each hexadecimal digit takes 4 bits or 4 binary digits, so it's very easy to convert from binary number to hexadecimal and back, even manually, in one's mind.

\begin{center}
\begin{longtable}{ | l | l | l | }
\hline
\HeaderColor hexadecimal & \HeaderColor binary & \HeaderColor decimal \\
\hline
0	&0000	&0 \\
1	&0001	&1 \\
2	&0010	&2 \\
3	&0011	&3 \\
4	&0100	&4 \\
5	&0101	&5 \\
6	&0110	&6 \\
7	&0111	&7 \\
8	&1000	&8 \\
9	&1001	&9 \\
A	&1010	&10 \\
B	&1011	&11 \\
C	&1100	&12 \\
D	&1101	&13 \\
E	&1110	&14 \\
F	&1111	&15 \\
\hline
\end{longtable}
\end{center}

How can one tell which radix is being used in a specific instance?

Decimal numbers are usually written as is, i.e., 1234. Some assemblers allow an identifier on decimal radix numbers, in which the number would be written with a "d" suffix: 1234d.

Binary numbers are sometimes prepended with the "0b" prefix: 0b100110111 (\ac{GCC} has a non-standard language extension for this\footnote{\url{https://gcc.gnu.org/onlinedocs/gcc/Binary-constants.html}}).
There is also another way: using a "b" suffix, for example: 100110111b.
This book tries to use the "0b" prefix consistently throughout the book for binary numbers.

Hexadecimal numbers are prepended with "0x" prefix in \CCpp and other \ac{PL}s: 0x1234ABCD.
Alternatively, they are given a "h" suffix: 1234ABCDh. This is common way of representing them in assemblers and debuggers.
In this convention, if the number is started with a Latin (A..F) digit, a 0 is added at the beginning: 0ABCDEFh.
There was also convention that was popular in 8-bit home computers era, using \$ prefix, like \$ABCD.
The book will try to stick to "0x" prefix throughout the book for hexadecimal numbers.

Should one learn to convert numbers mentally? A table of 1-digit hexadecimal numbers can easily be memorized.
As for larger numbers, it's probably not worth tormenting yourself.

Perhaps the most visible hexadecimal numbers are in \ac{URL}s.
This is the way that non-Latin characters are encoded.
For example:
\url{https://en.wiktionary.org/wiki/na\%C3\%AFvet\%C3\%A9} is the \ac{URL} of Wiktionary article about \q{naïveté} word.

\subsubsection{Octal Radix}

Another numeral system heavily used in the past of computer programming is octal. In octal there are 8 digits (0..7), and each is mapped to 3 bits, so it's easy to convert numbers back and forth.
It has been superseded by the hexadecimal system almost everywhere, but, surprisingly, there is a *NIX utility, used often by many people, which takes octal numbers as argument: \TT{chmod}.

\myindex{UNIX!chmod}
As many *NIX users know, \TT{chmod} argument can be a number of 3 digits. The first digit represents the rights of the owner of the file (read, write and/or execute), the second is the rights for the group to which the file belongs, and the third is for everyone else.
Each digit that \TT{chmod} takes can be represented in binary form:

\begin{center}
\begin{longtable}{ | l | l | l | }
\hline
\HeaderColor decimal & \HeaderColor binary & \HeaderColor meaning \\
\hline
7	&111	&\textbf{rwx} \\
6	&110	&\textbf{rw-} \\
5	&101	&\textbf{r-x} \\
4	&100	&\textbf{r-{}-} \\
3	&011	&\textbf{-wx} \\
2	&010	&\textbf{-w-} \\
1	&001	&\textbf{-{}-x} \\
0	&000	&\textbf{-{}-{}-} \\
\hline
\end{longtable}
\end{center}

So each bit is mapped to a flag: read/write/execute.

The importance of \TT{chmod} here is that the whole number in argument can be represented as octal number.
Let's take, for example, 644.
When you run \TT{chmod 644 file}, you set read/write permissions for owner, read permissions for group and again, read permissions for everyone else.
If we convert the octal number 644 to binary, it would be \TT{110100100}, or, in groups of 3 bits, \TT{110 100 100}.

Now we see that each triplet describe permissions for owner/group/others: first is \TT{rw-}, second is \TT{r--} and third is \TT{r--}.

The octal numeral system was also popular on old computers like PDP-8, because word there could be 12, 24 or 36 bits, and these numbers are all divisible by 3, so the octal system was natural in that environment.
Nowadays, all popular computers employ word/address sizes of 16, 32 or 64 bits, and these numbers are all divisible by 4, so the hexadecimal system is more natural there.

The octal numeral system is supported by all standard \CCpp compilers.
This is a source of confusion sometimes, because octal numbers are encoded with a zero prepended, for example, 0377 is 255.
Sometimes, you might make a typo and write "09" instead of 9, and the compiler would report an error.
GCC might report something like this:\\
\TT{error: invalid digit "9" in octal constant}.

Also, the octal system is somewhat popular in Java. When the IDA shows Java strings with non-printable characters,
they are encoded in the octal system instead of hexadecimal.
\myindex{JAD}
The JAD Java decompiler behaves the same way.

\subsubsection{Divisibility}

When you see a decimal number like 120, you can quickly deduce that it's divisible by 10, because the last digit is zero.
In the same way, 123400 is divisible by 100, because the two last digits are zeros.

Likewise, the hexadecimal number 0x1230 is divisible by 0x10 (or 16), 0x123000 is divisible by 0x1000 (or 4096), etc.

The binary number 0b1000101000 is divisible by 0b1000 (8), etc.

This property can often be used to quickly realize if the size of some block in memory is padded to some boundary.
For example, sections in \ac{PE} files are almost always started at addresses ending with 3 hexadecimal zeros: 0x41000, 0x10001000, etc.
The reason behind this is the fact that almost all \ac{PE} sections are padded to a boundary of 0x1000 (4096) bytes.

\subsubsection{Multi-Precision Arithmetic and Radix}

\index{RSA}
Multi-precision arithmetic can use huge numbers, and each one may be stored in several bytes.
For example, RSA keys, both public and private, span up to 4096 bits, and maybe even more.

% I'm not sure how to change this, but the normal format for quoting would be just to mention the author or book, and footnote to the full reference
In \InSqBrackets{\TAOCPvolII, 265} we find the following idea: when you store a multi-precision number in several bytes,
the whole number can be represented as having a radix of $2^8=256$, and each digit goes to the corresponding byte.
Likewise, if you store a multi-precision number in several 32-bit integer values, each digit goes to each 32-bit slot,
and you may think about this number as stored in radix of $2^{32}$.

\subsubsection{How to Pronounce Non-Decimal Numbers}

Numbers in a non-decimal base are usually pronounced by digit by digit: ``one-zero-zero-one-one-...''.
Words like ``ten'' and ``thousand'' are usually not pronounced, to prevent confusion with the decimal base system.

\subsubsection{Floating point numbers}

To distinguish floating point numbers from integers, they are usually written with ``.0'' at the end,
like $0.0$, $123.0$, etc.
}
\RU{\subsection{Представление чисел}

Люди привыкли к десятичной системе счисления вероятно потому что почти у каждого есть по 10 пальцев.
Тем не менее, число 10 не имеет особого значения в науке и математике.
Двоичная система естествена для цифровой электроники: 0 означает отсутствие тока в проводе и 1 --- его присутствие.
10 в двоичной системе это 2 в десятичной; 100 в двоичной это 4 в десятичной, итд.

Если в системе счисления есть 10 цифр, её \IT{основание} или \IT{radix} это 10.
Двоичная система имеет \IT{основание} 2.

Важные вещи, которые полезно вспомнить:
1) \IT{число} это число, в то время как \IT{цифра} это термин из системы письменности, и это обычно один символ;
2) само число не меняется, когда конвертируется из одного основания в другое: меняется способ его записи (или представления
в памяти).

Как сконвертировать число из одного основания в другое?

Позиционная нотация используется почти везде, это означает, что всякая цифра имеет свой вес, в зависимости от её расположения
внутри числа.
Если 2 расположена в самом последнем месте справа, это 2.
Если она расположена в месте перед последним, это 20.

Что означает $1234$?

$10^3 \cdot 1 + 10^2 \cdot 2 + 10^1 \cdot 3 + 1 \cdot 4$ = 1234 или
$1000 \cdot 1 + 100 \cdot 2 + 10 \cdot 3 + 4 = 1234$

Та же история и для двоичных чисел, только основание там 2 вместо 10.
Что означает 0b101011?

$2^5 \cdot 1 + 2^4 \cdot 0 + 2^3 \cdot 1 + 2^2 \cdot 0 + 2^1 \cdot 1 + 2^0 \cdot 1 = 43$ или
$32 \cdot 1 + 16 \cdot 0 + 8 \cdot 1 + 4 \cdot 0 + 2 \cdot 1 + 1 = 43$

Позиционную нотацию можно противопоставить непозиционной нотации, такой как римская система записи чисел
\footnote{Об эволюции способов записи чисел, см.также: \InSqBrackets{\TAOCPvolII{}, 195--213.}}.
Вероятно, человечество перешло на позиционную нотацию, потому что так проще работать с числами (сложение, умножение, итд)
на бумаге, в ручную.

Действительно, двоичные числа можно складывать, вычитать, итд, точно также, как этому обычно обучают в школах,
только доступны лишь 2 цифры.

Двоичные числа громоздки, когда их используют в исходных кодах и дампах, так что в этих случаях применяется шестнадцатеричная
система.
Используются цифры 0..9 и еще 6 латинских букв: A..F.
Каждая шестнадцатеричная цифра занимает 4 бита или 4 двоичных цифры, так что конвертировать из двоичной системы в
шестнадцатеричную и назад, можно легко вручную, или даже в уме.

\begin{center}
\begin{longtable}{ | l | l | l | }
\hline
\HeaderColor шестнадцатеричная & \HeaderColor двоичная & \HeaderColor десятичная \\
\hline
0	&0000	&0 \\
1	&0001	&1 \\
2	&0010	&2 \\
3	&0011	&3 \\
4	&0100	&4 \\
5	&0101	&5 \\
6	&0110	&6 \\
7	&0111	&7 \\
8	&1000	&8 \\
9	&1001	&9 \\
A	&1010	&10 \\
B	&1011	&11 \\
C	&1100	&12 \\
D	&1101	&13 \\
E	&1110	&14 \\
F	&1111	&15 \\
\hline
\end{longtable}
\end{center}

Как понять, какое основание используется в конкретном месте?

Десятичные числа обычно записываются как есть, т.е., 1234. Но некоторые ассемблеры позволяют подчеркивать
этот факт для ясности, и это число может быть дополнено суффиксом "d": 1234d.

К двоичным числам иногда спереди добавляют префикс "0b": 0b100110111
(В \ac{GCC} для этого есть нестандартное расширение языка
\footnote{\url{https://gcc.gnu.org/onlinedocs/gcc/Binary-constants.html}}).
Есть также еще один способ: суффикс "b", например: 100110111b.
В этой книге я буду пытаться придерживаться префикса "0b" для двоичных чисел.

Шестнадцатеричные числа имеют префикс "0x" в \CCpp и некоторых других \ac{PL}: 0x1234ABCD.
Либо они имеют суффикс "h": 1234ABCDh --- обычно так они представляются в ассемблерах и отладчиках.
Если число начинается с цифры A..F, перед ним добавляется 0: 0ABCDEFh.
Во времена 8-битных домашних компьютеров, был также способ записи чисел используя префикс \$, например, \$ABCD.
В книге я попытаюсь придерживаться префикса "0x" для шестнадцатеричных чисел.

Нужно ли учиться конвертировать числа в уме? Таблицу шестнадцатеричных чисел из одной цифры легко запомнить.
А запоминать б\'{о}льшие числа, наверное, не стоит.

Наверное, чаще всего шестнадцатеричные числа можно увидеть в \ac{URL}-ах.
Так кодируются буквы не из числа латинских.
Например:
\url{https://en.wiktionary.org/wiki/na\%C3\%AFvet\%C3\%A9} это \ac{URL} страницы в Wiktionary о слове \q{naïveté}.

\subsubsection{Восьмеричная система}

Еще одна система, которая в прошлом много использовалась в программировании это восьмеричная: есть 8 цифр (0..7) и каждая
описывает 3 бита, так что легко конвертировать числа туда и назад.
Она почти везде была заменена шестнадцатеричной, но удивительно, в *NIX имеется утилита использующаяся многими людьми,
которая принимает на вход восьмеричное число: \TT{chmod}.

\myindex{UNIX!chmod}
Как знают многие пользователи *NIX, аргумент \TT{chmod} это число из трех цифр. Первая цифра это права владельца файла,
вторая это права группы (которой файл принадлежит), третья для всех остальных.
И каждая цифра может быть представлена в двоичном виде:

\begin{center}
\begin{longtable}{ | l | l | l | }
\hline
\HeaderColor десятичная & \HeaderColor двоичная & \HeaderColor значение \\
\hline
7	&111	&\textbf{rwx} \\
6	&110	&\textbf{rw-} \\
5	&101	&\textbf{r-x} \\
4	&100	&\textbf{r-{}-} \\
3	&011	&\textbf{-wx} \\
2	&010	&\textbf{-w-} \\
1	&001	&\textbf{-{}-x} \\
0	&000	&\textbf{-{}-{}-} \\
\hline
\end{longtable}
\end{center}

Так что каждый бит привязан к флагу: read/write/execute (чтение/запись/исполнение).

И вот почему я вспомнил здесь о \TT{chmod}, это потому что всё число может быть представлено как число в восьмеричной системе.
Для примера возьмем 644.
Когда вы запускаете \TT{chmod 644 file}, вы выставляете права read/write для владельца, права read для группы, и снова,
read для всех остальных.
Сконвертируем число 644 из восьмеричной системы в двоичную, это будет \TT{110100100}, или (в группах по 3 бита) \TT{110 100 100}.

Теперь мы видим, что каждая тройка описывает права для владельца/группы/остальных:
первая это \TT{rw-}, вторая это \TT{r--} и третья это \TT{r--}.

Восьмеричная система была также популярная на старых компьютерах вроде PDP-8, потому что слово там могло содержать 12, 24 или
36 бит, и эти числа делятся на 3, так что выбор восьмеричной системы в той среде был логичен.
Сейчас, все популярные компьютеры имеют размер слова/адреса 16, 32 или 64 бита, и эти числа делятся на 4,
так что шестнадцатеричная система здесь удобнее.

Восьмеричная система поддерживается всеми стандартными компиляторами \CCpp{}.
Это иногда источник недоумения, потому что восьмеричные числа кодируются с нулем вперед, например, 0377 это 255.
И иногда, вы можете сделать опечатку, и написать "09" вместо 9, и компилятор выдаст ошибку.
GCC может выдать что-то вроде:\\
\TT{error: invalid digit "9" in octal constant}.

Также, восьмеричная система популярна в Java: когда IDA показывает строку с непечатаемыми символами,
они кодируются в восьмеричной системе вместо шестнадцатеричной.
\myindex{JAD}
Точно также себя ведет декомпилятор с Java JAD.

\subsubsection{Делимость}

Когда вы видите десятичное число вроде 120, вы можете быстро понять что оно делится на 10, потому что последняя цифра это 0.
Точно также, 123400 делится на 100, потому что две последних цифры это нули.

Точно также, шестнадцатеричное число 0x1230 делится на 0x10 (или 16), 0x123000 делится на 0x1000 (или 4096), итд.

Двоичное число 0b1000101000 делится на 0b1000 (8), итд.

Это свойство можно часто использовать, чтобы быстро понять,
что длина какого-либо блока в памяти выровнена по некоторой границе.
Например, секции в \ac{PE}-файлах почти всегда начинаются с адресов заканчивающихся 3 шестнадцатеричными нулями:
0x41000, 0x10001000, итд.
Причина в том, что почти все секции в \ac{PE} выровнены по границе 0x1000 (4096) байт.

\subsubsection{Арифметика произвольной точности и основание}

\index{RSA}
Арифметика произвольной точности (multi-precision arithmetic) может использовать огромные числа,
которые могут храниться в нескольких байтах.
Например, ключи RSA, и открытые и закрытые, могут занимать до 4096 бит и даже больше.

В \InSqBrackets{\TAOCPvolII, 265} можно найти такую идею: когда вы сохраняете число произвольной точности в нескольких байтах,
всё число может быть представлено как имеющую систему счисления по основанию $2^8=256$, и каждая цифра находится
в соответствующем байте.
Точно также, если вы сохраняете число произвольной точности в нескольких 32-битных целочисленных значениях,
каждая цифра отправляется в каждый 32-битный слот, и вы можете считать что это число записано в системе с основанием $2^{32}$.

\subsubsection{Произношение}

Числа в недесятичных системах счислениях обычно произносятся по одной цифре: ``один-ноль-ноль-один-один-...''.
Слова вроде ``десять'', ``тысяча'', итд, обычно не произносятся, потому что тогда можно спутать с десятичной системой.

\subsubsection{Числа с плавающей запятой}

Чтобы отличать числа с плавающей запятой от целочисленных, часто, в конце добавляют ``.0'',
например $0.0$, $123.0$, итд.

}
\ITA{\input{patterns/numeral_ITA}}
\DE{\input{patterns/numeral_DE}}
\FR{\input{patterns/numeral_FR}}
\PL{\input{patterns/numeral_PL}}

% chapters
\ifdefined\SPANISH
\chapter{Patrones de código}
\fi % SPANISH

\ifdefined\GERMAN
\chapter{Code-Muster}
\fi % GERMAN

\ifdefined\ENGLISH
\chapter{Code Patterns}
\fi % ENGLISH

\ifdefined\ITALIAN
\chapter{Forme di codice}
\fi % ITALIAN

\ifdefined\RUSSIAN
\chapter{Образцы кода}
\fi % RUSSIAN

\ifdefined\BRAZILIAN
\chapter{Padrões de códigos}
\fi % BRAZILIAN

\ifdefined\THAI
\chapter{รูปแบบของโค้ด}
\fi % THAI

\ifdefined\FRENCH
\chapter{Modèle de code}
\fi % FRENCH

\ifdefined\POLISH
\chapter{\PLph{}}
\fi % POLISH

% sections
\EN{\input{patterns/patterns_opt_dbg_EN}}
\ES{\input{patterns/patterns_opt_dbg_ES}}
\ITA{\input{patterns/patterns_opt_dbg_ITA}}
\PTBR{\input{patterns/patterns_opt_dbg_PTBR}}
\RU{\input{patterns/patterns_opt_dbg_RU}}
\THA{\input{patterns/patterns_opt_dbg_THA}}
\DE{\input{patterns/patterns_opt_dbg_DE}}
\FR{\input{patterns/patterns_opt_dbg_FR}}
\PL{\input{patterns/patterns_opt_dbg_PL}}

\RU{\section{Некоторые базовые понятия}}
\EN{\section{Some basics}}
\DE{\section{Einige Grundlagen}}
\FR{\section{Quelques bases}}
\ES{\section{\ESph{}}}
\ITA{\section{Alcune basi teoriche}}
\PTBR{\section{\PTBRph{}}}
\THA{\section{\THAph{}}}
\PL{\section{\PLph{}}}

% sections:
\EN{\input{patterns/intro_CPU_ISA_EN}}
\ES{\input{patterns/intro_CPU_ISA_ES}}
\ITA{\input{patterns/intro_CPU_ISA_ITA}}
\PTBR{\input{patterns/intro_CPU_ISA_PTBR}}
\RU{\input{patterns/intro_CPU_ISA_RU}}
\DE{\input{patterns/intro_CPU_ISA_DE}}
\FR{\input{patterns/intro_CPU_ISA_FR}}
\PL{\input{patterns/intro_CPU_ISA_PL}}

\EN{\input{patterns/numeral_EN}}
\RU{\input{patterns/numeral_RU}}
\ITA{\input{patterns/numeral_ITA}}
\DE{\input{patterns/numeral_DE}}
\FR{\input{patterns/numeral_FR}}
\PL{\input{patterns/numeral_PL}}

% chapters
\input{patterns/00_empty/main}
\input{patterns/011_ret/main}
\input{patterns/01_helloworld/main}
\input{patterns/015_prolog_epilogue/main}
\input{patterns/02_stack/main}
\input{patterns/03_printf/main}
\input{patterns/04_scanf/main}
\input{patterns/05_passing_arguments/main}
\input{patterns/06_return_results/main}
\input{patterns/061_pointers/main}
\input{patterns/065_GOTO/main}
\input{patterns/07_jcc/main}
\input{patterns/08_switch/main}
\input{patterns/09_loops/main}
\input{patterns/10_strings/main}
\input{patterns/11_arith_optimizations/main}
\input{patterns/12_FPU/main}
\input{patterns/13_arrays/main}
\input{patterns/14_bitfields/main}
\EN{\input{patterns/145_LCG/main_EN}}
\RU{\input{patterns/145_LCG/main_RU}}
\input{patterns/15_structs/main}
\input{patterns/17_unions/main}
\input{patterns/18_pointers_to_functions/main}
\input{patterns/185_64bit_in_32_env/main}

\EN{\input{patterns/19_SIMD/main_EN}}
\RU{\input{patterns/19_SIMD/main_RU}}
\DE{\input{patterns/19_SIMD/main_DE}}

\EN{\input{patterns/20_x64/main_EN}}
\RU{\input{patterns/20_x64/main_RU}}

\EN{\input{patterns/205_floating_SIMD/main_EN}}
\RU{\input{patterns/205_floating_SIMD/main_RU}}
\DE{\input{patterns/205_floating_SIMD/main_DE}}

\EN{\input{patterns/ARM/main_EN}}
\RU{\input{patterns/ARM/main_RU}}
\DE{\input{patterns/ARM/main_DE}}

\input{patterns/MIPS/main}

\ifdefined\SPANISH
\chapter{Patrones de código}
\fi % SPANISH

\ifdefined\GERMAN
\chapter{Code-Muster}
\fi % GERMAN

\ifdefined\ENGLISH
\chapter{Code Patterns}
\fi % ENGLISH

\ifdefined\ITALIAN
\chapter{Forme di codice}
\fi % ITALIAN

\ifdefined\RUSSIAN
\chapter{Образцы кода}
\fi % RUSSIAN

\ifdefined\BRAZILIAN
\chapter{Padrões de códigos}
\fi % BRAZILIAN

\ifdefined\THAI
\chapter{รูปแบบของโค้ด}
\fi % THAI

\ifdefined\FRENCH
\chapter{Modèle de code}
\fi % FRENCH

\ifdefined\POLISH
\chapter{\PLph{}}
\fi % POLISH

% sections
\EN{\input{patterns/patterns_opt_dbg_EN}}
\ES{\input{patterns/patterns_opt_dbg_ES}}
\ITA{\input{patterns/patterns_opt_dbg_ITA}}
\PTBR{\input{patterns/patterns_opt_dbg_PTBR}}
\RU{\input{patterns/patterns_opt_dbg_RU}}
\THA{\input{patterns/patterns_opt_dbg_THA}}
\DE{\input{patterns/patterns_opt_dbg_DE}}
\FR{\input{patterns/patterns_opt_dbg_FR}}
\PL{\input{patterns/patterns_opt_dbg_PL}}

\RU{\section{Некоторые базовые понятия}}
\EN{\section{Some basics}}
\DE{\section{Einige Grundlagen}}
\FR{\section{Quelques bases}}
\ES{\section{\ESph{}}}
\ITA{\section{Alcune basi teoriche}}
\PTBR{\section{\PTBRph{}}}
\THA{\section{\THAph{}}}
\PL{\section{\PLph{}}}

% sections:
\EN{\input{patterns/intro_CPU_ISA_EN}}
\ES{\input{patterns/intro_CPU_ISA_ES}}
\ITA{\input{patterns/intro_CPU_ISA_ITA}}
\PTBR{\input{patterns/intro_CPU_ISA_PTBR}}
\RU{\input{patterns/intro_CPU_ISA_RU}}
\DE{\input{patterns/intro_CPU_ISA_DE}}
\FR{\input{patterns/intro_CPU_ISA_FR}}
\PL{\input{patterns/intro_CPU_ISA_PL}}

\EN{\input{patterns/numeral_EN}}
\RU{\input{patterns/numeral_RU}}
\ITA{\input{patterns/numeral_ITA}}
\DE{\input{patterns/numeral_DE}}
\FR{\input{patterns/numeral_FR}}
\PL{\input{patterns/numeral_PL}}

% chapters
\input{patterns/00_empty/main}
\input{patterns/011_ret/main}
\input{patterns/01_helloworld/main}
\input{patterns/015_prolog_epilogue/main}
\input{patterns/02_stack/main}
\input{patterns/03_printf/main}
\input{patterns/04_scanf/main}
\input{patterns/05_passing_arguments/main}
\input{patterns/06_return_results/main}
\input{patterns/061_pointers/main}
\input{patterns/065_GOTO/main}
\input{patterns/07_jcc/main}
\input{patterns/08_switch/main}
\input{patterns/09_loops/main}
\input{patterns/10_strings/main}
\input{patterns/11_arith_optimizations/main}
\input{patterns/12_FPU/main}
\input{patterns/13_arrays/main}
\input{patterns/14_bitfields/main}
\EN{\input{patterns/145_LCG/main_EN}}
\RU{\input{patterns/145_LCG/main_RU}}
\input{patterns/15_structs/main}
\input{patterns/17_unions/main}
\input{patterns/18_pointers_to_functions/main}
\input{patterns/185_64bit_in_32_env/main}

\EN{\input{patterns/19_SIMD/main_EN}}
\RU{\input{patterns/19_SIMD/main_RU}}
\DE{\input{patterns/19_SIMD/main_DE}}

\EN{\input{patterns/20_x64/main_EN}}
\RU{\input{patterns/20_x64/main_RU}}

\EN{\input{patterns/205_floating_SIMD/main_EN}}
\RU{\input{patterns/205_floating_SIMD/main_RU}}
\DE{\input{patterns/205_floating_SIMD/main_DE}}

\EN{\input{patterns/ARM/main_EN}}
\RU{\input{patterns/ARM/main_RU}}
\DE{\input{patterns/ARM/main_DE}}

\input{patterns/MIPS/main}

\ifdefined\SPANISH
\chapter{Patrones de código}
\fi % SPANISH

\ifdefined\GERMAN
\chapter{Code-Muster}
\fi % GERMAN

\ifdefined\ENGLISH
\chapter{Code Patterns}
\fi % ENGLISH

\ifdefined\ITALIAN
\chapter{Forme di codice}
\fi % ITALIAN

\ifdefined\RUSSIAN
\chapter{Образцы кода}
\fi % RUSSIAN

\ifdefined\BRAZILIAN
\chapter{Padrões de códigos}
\fi % BRAZILIAN

\ifdefined\THAI
\chapter{รูปแบบของโค้ด}
\fi % THAI

\ifdefined\FRENCH
\chapter{Modèle de code}
\fi % FRENCH

\ifdefined\POLISH
\chapter{\PLph{}}
\fi % POLISH

% sections
\EN{\input{patterns/patterns_opt_dbg_EN}}
\ES{\input{patterns/patterns_opt_dbg_ES}}
\ITA{\input{patterns/patterns_opt_dbg_ITA}}
\PTBR{\input{patterns/patterns_opt_dbg_PTBR}}
\RU{\input{patterns/patterns_opt_dbg_RU}}
\THA{\input{patterns/patterns_opt_dbg_THA}}
\DE{\input{patterns/patterns_opt_dbg_DE}}
\FR{\input{patterns/patterns_opt_dbg_FR}}
\PL{\input{patterns/patterns_opt_dbg_PL}}

\RU{\section{Некоторые базовые понятия}}
\EN{\section{Some basics}}
\DE{\section{Einige Grundlagen}}
\FR{\section{Quelques bases}}
\ES{\section{\ESph{}}}
\ITA{\section{Alcune basi teoriche}}
\PTBR{\section{\PTBRph{}}}
\THA{\section{\THAph{}}}
\PL{\section{\PLph{}}}

% sections:
\EN{\input{patterns/intro_CPU_ISA_EN}}
\ES{\input{patterns/intro_CPU_ISA_ES}}
\ITA{\input{patterns/intro_CPU_ISA_ITA}}
\PTBR{\input{patterns/intro_CPU_ISA_PTBR}}
\RU{\input{patterns/intro_CPU_ISA_RU}}
\DE{\input{patterns/intro_CPU_ISA_DE}}
\FR{\input{patterns/intro_CPU_ISA_FR}}
\PL{\input{patterns/intro_CPU_ISA_PL}}

\EN{\input{patterns/numeral_EN}}
\RU{\input{patterns/numeral_RU}}
\ITA{\input{patterns/numeral_ITA}}
\DE{\input{patterns/numeral_DE}}
\FR{\input{patterns/numeral_FR}}
\PL{\input{patterns/numeral_PL}}

% chapters
\input{patterns/00_empty/main}
\input{patterns/011_ret/main}
\input{patterns/01_helloworld/main}
\input{patterns/015_prolog_epilogue/main}
\input{patterns/02_stack/main}
\input{patterns/03_printf/main}
\input{patterns/04_scanf/main}
\input{patterns/05_passing_arguments/main}
\input{patterns/06_return_results/main}
\input{patterns/061_pointers/main}
\input{patterns/065_GOTO/main}
\input{patterns/07_jcc/main}
\input{patterns/08_switch/main}
\input{patterns/09_loops/main}
\input{patterns/10_strings/main}
\input{patterns/11_arith_optimizations/main}
\input{patterns/12_FPU/main}
\input{patterns/13_arrays/main}
\input{patterns/14_bitfields/main}
\EN{\input{patterns/145_LCG/main_EN}}
\RU{\input{patterns/145_LCG/main_RU}}
\input{patterns/15_structs/main}
\input{patterns/17_unions/main}
\input{patterns/18_pointers_to_functions/main}
\input{patterns/185_64bit_in_32_env/main}

\EN{\input{patterns/19_SIMD/main_EN}}
\RU{\input{patterns/19_SIMD/main_RU}}
\DE{\input{patterns/19_SIMD/main_DE}}

\EN{\input{patterns/20_x64/main_EN}}
\RU{\input{patterns/20_x64/main_RU}}

\EN{\input{patterns/205_floating_SIMD/main_EN}}
\RU{\input{patterns/205_floating_SIMD/main_RU}}
\DE{\input{patterns/205_floating_SIMD/main_DE}}

\EN{\input{patterns/ARM/main_EN}}
\RU{\input{patterns/ARM/main_RU}}
\DE{\input{patterns/ARM/main_DE}}

\input{patterns/MIPS/main}

\ifdefined\SPANISH
\chapter{Patrones de código}
\fi % SPANISH

\ifdefined\GERMAN
\chapter{Code-Muster}
\fi % GERMAN

\ifdefined\ENGLISH
\chapter{Code Patterns}
\fi % ENGLISH

\ifdefined\ITALIAN
\chapter{Forme di codice}
\fi % ITALIAN

\ifdefined\RUSSIAN
\chapter{Образцы кода}
\fi % RUSSIAN

\ifdefined\BRAZILIAN
\chapter{Padrões de códigos}
\fi % BRAZILIAN

\ifdefined\THAI
\chapter{รูปแบบของโค้ด}
\fi % THAI

\ifdefined\FRENCH
\chapter{Modèle de code}
\fi % FRENCH

\ifdefined\POLISH
\chapter{\PLph{}}
\fi % POLISH

% sections
\EN{\input{patterns/patterns_opt_dbg_EN}}
\ES{\input{patterns/patterns_opt_dbg_ES}}
\ITA{\input{patterns/patterns_opt_dbg_ITA}}
\PTBR{\input{patterns/patterns_opt_dbg_PTBR}}
\RU{\input{patterns/patterns_opt_dbg_RU}}
\THA{\input{patterns/patterns_opt_dbg_THA}}
\DE{\input{patterns/patterns_opt_dbg_DE}}
\FR{\input{patterns/patterns_opt_dbg_FR}}
\PL{\input{patterns/patterns_opt_dbg_PL}}

\RU{\section{Некоторые базовые понятия}}
\EN{\section{Some basics}}
\DE{\section{Einige Grundlagen}}
\FR{\section{Quelques bases}}
\ES{\section{\ESph{}}}
\ITA{\section{Alcune basi teoriche}}
\PTBR{\section{\PTBRph{}}}
\THA{\section{\THAph{}}}
\PL{\section{\PLph{}}}

% sections:
\EN{\input{patterns/intro_CPU_ISA_EN}}
\ES{\input{patterns/intro_CPU_ISA_ES}}
\ITA{\input{patterns/intro_CPU_ISA_ITA}}
\PTBR{\input{patterns/intro_CPU_ISA_PTBR}}
\RU{\input{patterns/intro_CPU_ISA_RU}}
\DE{\input{patterns/intro_CPU_ISA_DE}}
\FR{\input{patterns/intro_CPU_ISA_FR}}
\PL{\input{patterns/intro_CPU_ISA_PL}}

\EN{\input{patterns/numeral_EN}}
\RU{\input{patterns/numeral_RU}}
\ITA{\input{patterns/numeral_ITA}}
\DE{\input{patterns/numeral_DE}}
\FR{\input{patterns/numeral_FR}}
\PL{\input{patterns/numeral_PL}}

% chapters
\input{patterns/00_empty/main}
\input{patterns/011_ret/main}
\input{patterns/01_helloworld/main}
\input{patterns/015_prolog_epilogue/main}
\input{patterns/02_stack/main}
\input{patterns/03_printf/main}
\input{patterns/04_scanf/main}
\input{patterns/05_passing_arguments/main}
\input{patterns/06_return_results/main}
\input{patterns/061_pointers/main}
\input{patterns/065_GOTO/main}
\input{patterns/07_jcc/main}
\input{patterns/08_switch/main}
\input{patterns/09_loops/main}
\input{patterns/10_strings/main}
\input{patterns/11_arith_optimizations/main}
\input{patterns/12_FPU/main}
\input{patterns/13_arrays/main}
\input{patterns/14_bitfields/main}
\EN{\input{patterns/145_LCG/main_EN}}
\RU{\input{patterns/145_LCG/main_RU}}
\input{patterns/15_structs/main}
\input{patterns/17_unions/main}
\input{patterns/18_pointers_to_functions/main}
\input{patterns/185_64bit_in_32_env/main}

\EN{\input{patterns/19_SIMD/main_EN}}
\RU{\input{patterns/19_SIMD/main_RU}}
\DE{\input{patterns/19_SIMD/main_DE}}

\EN{\input{patterns/20_x64/main_EN}}
\RU{\input{patterns/20_x64/main_RU}}

\EN{\input{patterns/205_floating_SIMD/main_EN}}
\RU{\input{patterns/205_floating_SIMD/main_RU}}
\DE{\input{patterns/205_floating_SIMD/main_DE}}

\EN{\input{patterns/ARM/main_EN}}
\RU{\input{patterns/ARM/main_RU}}
\DE{\input{patterns/ARM/main_DE}}

\input{patterns/MIPS/main}

\ifdefined\SPANISH
\chapter{Patrones de código}
\fi % SPANISH

\ifdefined\GERMAN
\chapter{Code-Muster}
\fi % GERMAN

\ifdefined\ENGLISH
\chapter{Code Patterns}
\fi % ENGLISH

\ifdefined\ITALIAN
\chapter{Forme di codice}
\fi % ITALIAN

\ifdefined\RUSSIAN
\chapter{Образцы кода}
\fi % RUSSIAN

\ifdefined\BRAZILIAN
\chapter{Padrões de códigos}
\fi % BRAZILIAN

\ifdefined\THAI
\chapter{รูปแบบของโค้ด}
\fi % THAI

\ifdefined\FRENCH
\chapter{Modèle de code}
\fi % FRENCH

\ifdefined\POLISH
\chapter{\PLph{}}
\fi % POLISH

% sections
\EN{\input{patterns/patterns_opt_dbg_EN}}
\ES{\input{patterns/patterns_opt_dbg_ES}}
\ITA{\input{patterns/patterns_opt_dbg_ITA}}
\PTBR{\input{patterns/patterns_opt_dbg_PTBR}}
\RU{\input{patterns/patterns_opt_dbg_RU}}
\THA{\input{patterns/patterns_opt_dbg_THA}}
\DE{\input{patterns/patterns_opt_dbg_DE}}
\FR{\input{patterns/patterns_opt_dbg_FR}}
\PL{\input{patterns/patterns_opt_dbg_PL}}

\RU{\section{Некоторые базовые понятия}}
\EN{\section{Some basics}}
\DE{\section{Einige Grundlagen}}
\FR{\section{Quelques bases}}
\ES{\section{\ESph{}}}
\ITA{\section{Alcune basi teoriche}}
\PTBR{\section{\PTBRph{}}}
\THA{\section{\THAph{}}}
\PL{\section{\PLph{}}}

% sections:
\EN{\input{patterns/intro_CPU_ISA_EN}}
\ES{\input{patterns/intro_CPU_ISA_ES}}
\ITA{\input{patterns/intro_CPU_ISA_ITA}}
\PTBR{\input{patterns/intro_CPU_ISA_PTBR}}
\RU{\input{patterns/intro_CPU_ISA_RU}}
\DE{\input{patterns/intro_CPU_ISA_DE}}
\FR{\input{patterns/intro_CPU_ISA_FR}}
\PL{\input{patterns/intro_CPU_ISA_PL}}

\EN{\input{patterns/numeral_EN}}
\RU{\input{patterns/numeral_RU}}
\ITA{\input{patterns/numeral_ITA}}
\DE{\input{patterns/numeral_DE}}
\FR{\input{patterns/numeral_FR}}
\PL{\input{patterns/numeral_PL}}

% chapters
\input{patterns/00_empty/main}
\input{patterns/011_ret/main}
\input{patterns/01_helloworld/main}
\input{patterns/015_prolog_epilogue/main}
\input{patterns/02_stack/main}
\input{patterns/03_printf/main}
\input{patterns/04_scanf/main}
\input{patterns/05_passing_arguments/main}
\input{patterns/06_return_results/main}
\input{patterns/061_pointers/main}
\input{patterns/065_GOTO/main}
\input{patterns/07_jcc/main}
\input{patterns/08_switch/main}
\input{patterns/09_loops/main}
\input{patterns/10_strings/main}
\input{patterns/11_arith_optimizations/main}
\input{patterns/12_FPU/main}
\input{patterns/13_arrays/main}
\input{patterns/14_bitfields/main}
\EN{\input{patterns/145_LCG/main_EN}}
\RU{\input{patterns/145_LCG/main_RU}}
\input{patterns/15_structs/main}
\input{patterns/17_unions/main}
\input{patterns/18_pointers_to_functions/main}
\input{patterns/185_64bit_in_32_env/main}

\EN{\input{patterns/19_SIMD/main_EN}}
\RU{\input{patterns/19_SIMD/main_RU}}
\DE{\input{patterns/19_SIMD/main_DE}}

\EN{\input{patterns/20_x64/main_EN}}
\RU{\input{patterns/20_x64/main_RU}}

\EN{\input{patterns/205_floating_SIMD/main_EN}}
\RU{\input{patterns/205_floating_SIMD/main_RU}}
\DE{\input{patterns/205_floating_SIMD/main_DE}}

\EN{\input{patterns/ARM/main_EN}}
\RU{\input{patterns/ARM/main_RU}}
\DE{\input{patterns/ARM/main_DE}}

\input{patterns/MIPS/main}

\ifdefined\SPANISH
\chapter{Patrones de código}
\fi % SPANISH

\ifdefined\GERMAN
\chapter{Code-Muster}
\fi % GERMAN

\ifdefined\ENGLISH
\chapter{Code Patterns}
\fi % ENGLISH

\ifdefined\ITALIAN
\chapter{Forme di codice}
\fi % ITALIAN

\ifdefined\RUSSIAN
\chapter{Образцы кода}
\fi % RUSSIAN

\ifdefined\BRAZILIAN
\chapter{Padrões de códigos}
\fi % BRAZILIAN

\ifdefined\THAI
\chapter{รูปแบบของโค้ด}
\fi % THAI

\ifdefined\FRENCH
\chapter{Modèle de code}
\fi % FRENCH

\ifdefined\POLISH
\chapter{\PLph{}}
\fi % POLISH

% sections
\EN{\input{patterns/patterns_opt_dbg_EN}}
\ES{\input{patterns/patterns_opt_dbg_ES}}
\ITA{\input{patterns/patterns_opt_dbg_ITA}}
\PTBR{\input{patterns/patterns_opt_dbg_PTBR}}
\RU{\input{patterns/patterns_opt_dbg_RU}}
\THA{\input{patterns/patterns_opt_dbg_THA}}
\DE{\input{patterns/patterns_opt_dbg_DE}}
\FR{\input{patterns/patterns_opt_dbg_FR}}
\PL{\input{patterns/patterns_opt_dbg_PL}}

\RU{\section{Некоторые базовые понятия}}
\EN{\section{Some basics}}
\DE{\section{Einige Grundlagen}}
\FR{\section{Quelques bases}}
\ES{\section{\ESph{}}}
\ITA{\section{Alcune basi teoriche}}
\PTBR{\section{\PTBRph{}}}
\THA{\section{\THAph{}}}
\PL{\section{\PLph{}}}

% sections:
\EN{\input{patterns/intro_CPU_ISA_EN}}
\ES{\input{patterns/intro_CPU_ISA_ES}}
\ITA{\input{patterns/intro_CPU_ISA_ITA}}
\PTBR{\input{patterns/intro_CPU_ISA_PTBR}}
\RU{\input{patterns/intro_CPU_ISA_RU}}
\DE{\input{patterns/intro_CPU_ISA_DE}}
\FR{\input{patterns/intro_CPU_ISA_FR}}
\PL{\input{patterns/intro_CPU_ISA_PL}}

\EN{\input{patterns/numeral_EN}}
\RU{\input{patterns/numeral_RU}}
\ITA{\input{patterns/numeral_ITA}}
\DE{\input{patterns/numeral_DE}}
\FR{\input{patterns/numeral_FR}}
\PL{\input{patterns/numeral_PL}}

% chapters
\input{patterns/00_empty/main}
\input{patterns/011_ret/main}
\input{patterns/01_helloworld/main}
\input{patterns/015_prolog_epilogue/main}
\input{patterns/02_stack/main}
\input{patterns/03_printf/main}
\input{patterns/04_scanf/main}
\input{patterns/05_passing_arguments/main}
\input{patterns/06_return_results/main}
\input{patterns/061_pointers/main}
\input{patterns/065_GOTO/main}
\input{patterns/07_jcc/main}
\input{patterns/08_switch/main}
\input{patterns/09_loops/main}
\input{patterns/10_strings/main}
\input{patterns/11_arith_optimizations/main}
\input{patterns/12_FPU/main}
\input{patterns/13_arrays/main}
\input{patterns/14_bitfields/main}
\EN{\input{patterns/145_LCG/main_EN}}
\RU{\input{patterns/145_LCG/main_RU}}
\input{patterns/15_structs/main}
\input{patterns/17_unions/main}
\input{patterns/18_pointers_to_functions/main}
\input{patterns/185_64bit_in_32_env/main}

\EN{\input{patterns/19_SIMD/main_EN}}
\RU{\input{patterns/19_SIMD/main_RU}}
\DE{\input{patterns/19_SIMD/main_DE}}

\EN{\input{patterns/20_x64/main_EN}}
\RU{\input{patterns/20_x64/main_RU}}

\EN{\input{patterns/205_floating_SIMD/main_EN}}
\RU{\input{patterns/205_floating_SIMD/main_RU}}
\DE{\input{patterns/205_floating_SIMD/main_DE}}

\EN{\input{patterns/ARM/main_EN}}
\RU{\input{patterns/ARM/main_RU}}
\DE{\input{patterns/ARM/main_DE}}

\input{patterns/MIPS/main}

\ifdefined\SPANISH
\chapter{Patrones de código}
\fi % SPANISH

\ifdefined\GERMAN
\chapter{Code-Muster}
\fi % GERMAN

\ifdefined\ENGLISH
\chapter{Code Patterns}
\fi % ENGLISH

\ifdefined\ITALIAN
\chapter{Forme di codice}
\fi % ITALIAN

\ifdefined\RUSSIAN
\chapter{Образцы кода}
\fi % RUSSIAN

\ifdefined\BRAZILIAN
\chapter{Padrões de códigos}
\fi % BRAZILIAN

\ifdefined\THAI
\chapter{รูปแบบของโค้ด}
\fi % THAI

\ifdefined\FRENCH
\chapter{Modèle de code}
\fi % FRENCH

\ifdefined\POLISH
\chapter{\PLph{}}
\fi % POLISH

% sections
\EN{\input{patterns/patterns_opt_dbg_EN}}
\ES{\input{patterns/patterns_opt_dbg_ES}}
\ITA{\input{patterns/patterns_opt_dbg_ITA}}
\PTBR{\input{patterns/patterns_opt_dbg_PTBR}}
\RU{\input{patterns/patterns_opt_dbg_RU}}
\THA{\input{patterns/patterns_opt_dbg_THA}}
\DE{\input{patterns/patterns_opt_dbg_DE}}
\FR{\input{patterns/patterns_opt_dbg_FR}}
\PL{\input{patterns/patterns_opt_dbg_PL}}

\RU{\section{Некоторые базовые понятия}}
\EN{\section{Some basics}}
\DE{\section{Einige Grundlagen}}
\FR{\section{Quelques bases}}
\ES{\section{\ESph{}}}
\ITA{\section{Alcune basi teoriche}}
\PTBR{\section{\PTBRph{}}}
\THA{\section{\THAph{}}}
\PL{\section{\PLph{}}}

% sections:
\EN{\input{patterns/intro_CPU_ISA_EN}}
\ES{\input{patterns/intro_CPU_ISA_ES}}
\ITA{\input{patterns/intro_CPU_ISA_ITA}}
\PTBR{\input{patterns/intro_CPU_ISA_PTBR}}
\RU{\input{patterns/intro_CPU_ISA_RU}}
\DE{\input{patterns/intro_CPU_ISA_DE}}
\FR{\input{patterns/intro_CPU_ISA_FR}}
\PL{\input{patterns/intro_CPU_ISA_PL}}

\EN{\input{patterns/numeral_EN}}
\RU{\input{patterns/numeral_RU}}
\ITA{\input{patterns/numeral_ITA}}
\DE{\input{patterns/numeral_DE}}
\FR{\input{patterns/numeral_FR}}
\PL{\input{patterns/numeral_PL}}

% chapters
\input{patterns/00_empty/main}
\input{patterns/011_ret/main}
\input{patterns/01_helloworld/main}
\input{patterns/015_prolog_epilogue/main}
\input{patterns/02_stack/main}
\input{patterns/03_printf/main}
\input{patterns/04_scanf/main}
\input{patterns/05_passing_arguments/main}
\input{patterns/06_return_results/main}
\input{patterns/061_pointers/main}
\input{patterns/065_GOTO/main}
\input{patterns/07_jcc/main}
\input{patterns/08_switch/main}
\input{patterns/09_loops/main}
\input{patterns/10_strings/main}
\input{patterns/11_arith_optimizations/main}
\input{patterns/12_FPU/main}
\input{patterns/13_arrays/main}
\input{patterns/14_bitfields/main}
\EN{\input{patterns/145_LCG/main_EN}}
\RU{\input{patterns/145_LCG/main_RU}}
\input{patterns/15_structs/main}
\input{patterns/17_unions/main}
\input{patterns/18_pointers_to_functions/main}
\input{patterns/185_64bit_in_32_env/main}

\EN{\input{patterns/19_SIMD/main_EN}}
\RU{\input{patterns/19_SIMD/main_RU}}
\DE{\input{patterns/19_SIMD/main_DE}}

\EN{\input{patterns/20_x64/main_EN}}
\RU{\input{patterns/20_x64/main_RU}}

\EN{\input{patterns/205_floating_SIMD/main_EN}}
\RU{\input{patterns/205_floating_SIMD/main_RU}}
\DE{\input{patterns/205_floating_SIMD/main_DE}}

\EN{\input{patterns/ARM/main_EN}}
\RU{\input{patterns/ARM/main_RU}}
\DE{\input{patterns/ARM/main_DE}}

\input{patterns/MIPS/main}

\ifdefined\SPANISH
\chapter{Patrones de código}
\fi % SPANISH

\ifdefined\GERMAN
\chapter{Code-Muster}
\fi % GERMAN

\ifdefined\ENGLISH
\chapter{Code Patterns}
\fi % ENGLISH

\ifdefined\ITALIAN
\chapter{Forme di codice}
\fi % ITALIAN

\ifdefined\RUSSIAN
\chapter{Образцы кода}
\fi % RUSSIAN

\ifdefined\BRAZILIAN
\chapter{Padrões de códigos}
\fi % BRAZILIAN

\ifdefined\THAI
\chapter{รูปแบบของโค้ด}
\fi % THAI

\ifdefined\FRENCH
\chapter{Modèle de code}
\fi % FRENCH

\ifdefined\POLISH
\chapter{\PLph{}}
\fi % POLISH

% sections
\EN{\input{patterns/patterns_opt_dbg_EN}}
\ES{\input{patterns/patterns_opt_dbg_ES}}
\ITA{\input{patterns/patterns_opt_dbg_ITA}}
\PTBR{\input{patterns/patterns_opt_dbg_PTBR}}
\RU{\input{patterns/patterns_opt_dbg_RU}}
\THA{\input{patterns/patterns_opt_dbg_THA}}
\DE{\input{patterns/patterns_opt_dbg_DE}}
\FR{\input{patterns/patterns_opt_dbg_FR}}
\PL{\input{patterns/patterns_opt_dbg_PL}}

\RU{\section{Некоторые базовые понятия}}
\EN{\section{Some basics}}
\DE{\section{Einige Grundlagen}}
\FR{\section{Quelques bases}}
\ES{\section{\ESph{}}}
\ITA{\section{Alcune basi teoriche}}
\PTBR{\section{\PTBRph{}}}
\THA{\section{\THAph{}}}
\PL{\section{\PLph{}}}

% sections:
\EN{\input{patterns/intro_CPU_ISA_EN}}
\ES{\input{patterns/intro_CPU_ISA_ES}}
\ITA{\input{patterns/intro_CPU_ISA_ITA}}
\PTBR{\input{patterns/intro_CPU_ISA_PTBR}}
\RU{\input{patterns/intro_CPU_ISA_RU}}
\DE{\input{patterns/intro_CPU_ISA_DE}}
\FR{\input{patterns/intro_CPU_ISA_FR}}
\PL{\input{patterns/intro_CPU_ISA_PL}}

\EN{\input{patterns/numeral_EN}}
\RU{\input{patterns/numeral_RU}}
\ITA{\input{patterns/numeral_ITA}}
\DE{\input{patterns/numeral_DE}}
\FR{\input{patterns/numeral_FR}}
\PL{\input{patterns/numeral_PL}}

% chapters
\input{patterns/00_empty/main}
\input{patterns/011_ret/main}
\input{patterns/01_helloworld/main}
\input{patterns/015_prolog_epilogue/main}
\input{patterns/02_stack/main}
\input{patterns/03_printf/main}
\input{patterns/04_scanf/main}
\input{patterns/05_passing_arguments/main}
\input{patterns/06_return_results/main}
\input{patterns/061_pointers/main}
\input{patterns/065_GOTO/main}
\input{patterns/07_jcc/main}
\input{patterns/08_switch/main}
\input{patterns/09_loops/main}
\input{patterns/10_strings/main}
\input{patterns/11_arith_optimizations/main}
\input{patterns/12_FPU/main}
\input{patterns/13_arrays/main}
\input{patterns/14_bitfields/main}
\EN{\input{patterns/145_LCG/main_EN}}
\RU{\input{patterns/145_LCG/main_RU}}
\input{patterns/15_structs/main}
\input{patterns/17_unions/main}
\input{patterns/18_pointers_to_functions/main}
\input{patterns/185_64bit_in_32_env/main}

\EN{\input{patterns/19_SIMD/main_EN}}
\RU{\input{patterns/19_SIMD/main_RU}}
\DE{\input{patterns/19_SIMD/main_DE}}

\EN{\input{patterns/20_x64/main_EN}}
\RU{\input{patterns/20_x64/main_RU}}

\EN{\input{patterns/205_floating_SIMD/main_EN}}
\RU{\input{patterns/205_floating_SIMD/main_RU}}
\DE{\input{patterns/205_floating_SIMD/main_DE}}

\EN{\input{patterns/ARM/main_EN}}
\RU{\input{patterns/ARM/main_RU}}
\DE{\input{patterns/ARM/main_DE}}

\input{patterns/MIPS/main}

\ifdefined\SPANISH
\chapter{Patrones de código}
\fi % SPANISH

\ifdefined\GERMAN
\chapter{Code-Muster}
\fi % GERMAN

\ifdefined\ENGLISH
\chapter{Code Patterns}
\fi % ENGLISH

\ifdefined\ITALIAN
\chapter{Forme di codice}
\fi % ITALIAN

\ifdefined\RUSSIAN
\chapter{Образцы кода}
\fi % RUSSIAN

\ifdefined\BRAZILIAN
\chapter{Padrões de códigos}
\fi % BRAZILIAN

\ifdefined\THAI
\chapter{รูปแบบของโค้ด}
\fi % THAI

\ifdefined\FRENCH
\chapter{Modèle de code}
\fi % FRENCH

\ifdefined\POLISH
\chapter{\PLph{}}
\fi % POLISH

% sections
\EN{\input{patterns/patterns_opt_dbg_EN}}
\ES{\input{patterns/patterns_opt_dbg_ES}}
\ITA{\input{patterns/patterns_opt_dbg_ITA}}
\PTBR{\input{patterns/patterns_opt_dbg_PTBR}}
\RU{\input{patterns/patterns_opt_dbg_RU}}
\THA{\input{patterns/patterns_opt_dbg_THA}}
\DE{\input{patterns/patterns_opt_dbg_DE}}
\FR{\input{patterns/patterns_opt_dbg_FR}}
\PL{\input{patterns/patterns_opt_dbg_PL}}

\RU{\section{Некоторые базовые понятия}}
\EN{\section{Some basics}}
\DE{\section{Einige Grundlagen}}
\FR{\section{Quelques bases}}
\ES{\section{\ESph{}}}
\ITA{\section{Alcune basi teoriche}}
\PTBR{\section{\PTBRph{}}}
\THA{\section{\THAph{}}}
\PL{\section{\PLph{}}}

% sections:
\EN{\input{patterns/intro_CPU_ISA_EN}}
\ES{\input{patterns/intro_CPU_ISA_ES}}
\ITA{\input{patterns/intro_CPU_ISA_ITA}}
\PTBR{\input{patterns/intro_CPU_ISA_PTBR}}
\RU{\input{patterns/intro_CPU_ISA_RU}}
\DE{\input{patterns/intro_CPU_ISA_DE}}
\FR{\input{patterns/intro_CPU_ISA_FR}}
\PL{\input{patterns/intro_CPU_ISA_PL}}

\EN{\input{patterns/numeral_EN}}
\RU{\input{patterns/numeral_RU}}
\ITA{\input{patterns/numeral_ITA}}
\DE{\input{patterns/numeral_DE}}
\FR{\input{patterns/numeral_FR}}
\PL{\input{patterns/numeral_PL}}

% chapters
\input{patterns/00_empty/main}
\input{patterns/011_ret/main}
\input{patterns/01_helloworld/main}
\input{patterns/015_prolog_epilogue/main}
\input{patterns/02_stack/main}
\input{patterns/03_printf/main}
\input{patterns/04_scanf/main}
\input{patterns/05_passing_arguments/main}
\input{patterns/06_return_results/main}
\input{patterns/061_pointers/main}
\input{patterns/065_GOTO/main}
\input{patterns/07_jcc/main}
\input{patterns/08_switch/main}
\input{patterns/09_loops/main}
\input{patterns/10_strings/main}
\input{patterns/11_arith_optimizations/main}
\input{patterns/12_FPU/main}
\input{patterns/13_arrays/main}
\input{patterns/14_bitfields/main}
\EN{\input{patterns/145_LCG/main_EN}}
\RU{\input{patterns/145_LCG/main_RU}}
\input{patterns/15_structs/main}
\input{patterns/17_unions/main}
\input{patterns/18_pointers_to_functions/main}
\input{patterns/185_64bit_in_32_env/main}

\EN{\input{patterns/19_SIMD/main_EN}}
\RU{\input{patterns/19_SIMD/main_RU}}
\DE{\input{patterns/19_SIMD/main_DE}}

\EN{\input{patterns/20_x64/main_EN}}
\RU{\input{patterns/20_x64/main_RU}}

\EN{\input{patterns/205_floating_SIMD/main_EN}}
\RU{\input{patterns/205_floating_SIMD/main_RU}}
\DE{\input{patterns/205_floating_SIMD/main_DE}}

\EN{\input{patterns/ARM/main_EN}}
\RU{\input{patterns/ARM/main_RU}}
\DE{\input{patterns/ARM/main_DE}}

\input{patterns/MIPS/main}

\ifdefined\SPANISH
\chapter{Patrones de código}
\fi % SPANISH

\ifdefined\GERMAN
\chapter{Code-Muster}
\fi % GERMAN

\ifdefined\ENGLISH
\chapter{Code Patterns}
\fi % ENGLISH

\ifdefined\ITALIAN
\chapter{Forme di codice}
\fi % ITALIAN

\ifdefined\RUSSIAN
\chapter{Образцы кода}
\fi % RUSSIAN

\ifdefined\BRAZILIAN
\chapter{Padrões de códigos}
\fi % BRAZILIAN

\ifdefined\THAI
\chapter{รูปแบบของโค้ด}
\fi % THAI

\ifdefined\FRENCH
\chapter{Modèle de code}
\fi % FRENCH

\ifdefined\POLISH
\chapter{\PLph{}}
\fi % POLISH

% sections
\EN{\input{patterns/patterns_opt_dbg_EN}}
\ES{\input{patterns/patterns_opt_dbg_ES}}
\ITA{\input{patterns/patterns_opt_dbg_ITA}}
\PTBR{\input{patterns/patterns_opt_dbg_PTBR}}
\RU{\input{patterns/patterns_opt_dbg_RU}}
\THA{\input{patterns/patterns_opt_dbg_THA}}
\DE{\input{patterns/patterns_opt_dbg_DE}}
\FR{\input{patterns/patterns_opt_dbg_FR}}
\PL{\input{patterns/patterns_opt_dbg_PL}}

\RU{\section{Некоторые базовые понятия}}
\EN{\section{Some basics}}
\DE{\section{Einige Grundlagen}}
\FR{\section{Quelques bases}}
\ES{\section{\ESph{}}}
\ITA{\section{Alcune basi teoriche}}
\PTBR{\section{\PTBRph{}}}
\THA{\section{\THAph{}}}
\PL{\section{\PLph{}}}

% sections:
\EN{\input{patterns/intro_CPU_ISA_EN}}
\ES{\input{patterns/intro_CPU_ISA_ES}}
\ITA{\input{patterns/intro_CPU_ISA_ITA}}
\PTBR{\input{patterns/intro_CPU_ISA_PTBR}}
\RU{\input{patterns/intro_CPU_ISA_RU}}
\DE{\input{patterns/intro_CPU_ISA_DE}}
\FR{\input{patterns/intro_CPU_ISA_FR}}
\PL{\input{patterns/intro_CPU_ISA_PL}}

\EN{\input{patterns/numeral_EN}}
\RU{\input{patterns/numeral_RU}}
\ITA{\input{patterns/numeral_ITA}}
\DE{\input{patterns/numeral_DE}}
\FR{\input{patterns/numeral_FR}}
\PL{\input{patterns/numeral_PL}}

% chapters
\input{patterns/00_empty/main}
\input{patterns/011_ret/main}
\input{patterns/01_helloworld/main}
\input{patterns/015_prolog_epilogue/main}
\input{patterns/02_stack/main}
\input{patterns/03_printf/main}
\input{patterns/04_scanf/main}
\input{patterns/05_passing_arguments/main}
\input{patterns/06_return_results/main}
\input{patterns/061_pointers/main}
\input{patterns/065_GOTO/main}
\input{patterns/07_jcc/main}
\input{patterns/08_switch/main}
\input{patterns/09_loops/main}
\input{patterns/10_strings/main}
\input{patterns/11_arith_optimizations/main}
\input{patterns/12_FPU/main}
\input{patterns/13_arrays/main}
\input{patterns/14_bitfields/main}
\EN{\input{patterns/145_LCG/main_EN}}
\RU{\input{patterns/145_LCG/main_RU}}
\input{patterns/15_structs/main}
\input{patterns/17_unions/main}
\input{patterns/18_pointers_to_functions/main}
\input{patterns/185_64bit_in_32_env/main}

\EN{\input{patterns/19_SIMD/main_EN}}
\RU{\input{patterns/19_SIMD/main_RU}}
\DE{\input{patterns/19_SIMD/main_DE}}

\EN{\input{patterns/20_x64/main_EN}}
\RU{\input{patterns/20_x64/main_RU}}

\EN{\input{patterns/205_floating_SIMD/main_EN}}
\RU{\input{patterns/205_floating_SIMD/main_RU}}
\DE{\input{patterns/205_floating_SIMD/main_DE}}

\EN{\input{patterns/ARM/main_EN}}
\RU{\input{patterns/ARM/main_RU}}
\DE{\input{patterns/ARM/main_DE}}

\input{patterns/MIPS/main}

\ifdefined\SPANISH
\chapter{Patrones de código}
\fi % SPANISH

\ifdefined\GERMAN
\chapter{Code-Muster}
\fi % GERMAN

\ifdefined\ENGLISH
\chapter{Code Patterns}
\fi % ENGLISH

\ifdefined\ITALIAN
\chapter{Forme di codice}
\fi % ITALIAN

\ifdefined\RUSSIAN
\chapter{Образцы кода}
\fi % RUSSIAN

\ifdefined\BRAZILIAN
\chapter{Padrões de códigos}
\fi % BRAZILIAN

\ifdefined\THAI
\chapter{รูปแบบของโค้ด}
\fi % THAI

\ifdefined\FRENCH
\chapter{Modèle de code}
\fi % FRENCH

\ifdefined\POLISH
\chapter{\PLph{}}
\fi % POLISH

% sections
\EN{\input{patterns/patterns_opt_dbg_EN}}
\ES{\input{patterns/patterns_opt_dbg_ES}}
\ITA{\input{patterns/patterns_opt_dbg_ITA}}
\PTBR{\input{patterns/patterns_opt_dbg_PTBR}}
\RU{\input{patterns/patterns_opt_dbg_RU}}
\THA{\input{patterns/patterns_opt_dbg_THA}}
\DE{\input{patterns/patterns_opt_dbg_DE}}
\FR{\input{patterns/patterns_opt_dbg_FR}}
\PL{\input{patterns/patterns_opt_dbg_PL}}

\RU{\section{Некоторые базовые понятия}}
\EN{\section{Some basics}}
\DE{\section{Einige Grundlagen}}
\FR{\section{Quelques bases}}
\ES{\section{\ESph{}}}
\ITA{\section{Alcune basi teoriche}}
\PTBR{\section{\PTBRph{}}}
\THA{\section{\THAph{}}}
\PL{\section{\PLph{}}}

% sections:
\EN{\input{patterns/intro_CPU_ISA_EN}}
\ES{\input{patterns/intro_CPU_ISA_ES}}
\ITA{\input{patterns/intro_CPU_ISA_ITA}}
\PTBR{\input{patterns/intro_CPU_ISA_PTBR}}
\RU{\input{patterns/intro_CPU_ISA_RU}}
\DE{\input{patterns/intro_CPU_ISA_DE}}
\FR{\input{patterns/intro_CPU_ISA_FR}}
\PL{\input{patterns/intro_CPU_ISA_PL}}

\EN{\input{patterns/numeral_EN}}
\RU{\input{patterns/numeral_RU}}
\ITA{\input{patterns/numeral_ITA}}
\DE{\input{patterns/numeral_DE}}
\FR{\input{patterns/numeral_FR}}
\PL{\input{patterns/numeral_PL}}

% chapters
\input{patterns/00_empty/main}
\input{patterns/011_ret/main}
\input{patterns/01_helloworld/main}
\input{patterns/015_prolog_epilogue/main}
\input{patterns/02_stack/main}
\input{patterns/03_printf/main}
\input{patterns/04_scanf/main}
\input{patterns/05_passing_arguments/main}
\input{patterns/06_return_results/main}
\input{patterns/061_pointers/main}
\input{patterns/065_GOTO/main}
\input{patterns/07_jcc/main}
\input{patterns/08_switch/main}
\input{patterns/09_loops/main}
\input{patterns/10_strings/main}
\input{patterns/11_arith_optimizations/main}
\input{patterns/12_FPU/main}
\input{patterns/13_arrays/main}
\input{patterns/14_bitfields/main}
\EN{\input{patterns/145_LCG/main_EN}}
\RU{\input{patterns/145_LCG/main_RU}}
\input{patterns/15_structs/main}
\input{patterns/17_unions/main}
\input{patterns/18_pointers_to_functions/main}
\input{patterns/185_64bit_in_32_env/main}

\EN{\input{patterns/19_SIMD/main_EN}}
\RU{\input{patterns/19_SIMD/main_RU}}
\DE{\input{patterns/19_SIMD/main_DE}}

\EN{\input{patterns/20_x64/main_EN}}
\RU{\input{patterns/20_x64/main_RU}}

\EN{\input{patterns/205_floating_SIMD/main_EN}}
\RU{\input{patterns/205_floating_SIMD/main_RU}}
\DE{\input{patterns/205_floating_SIMD/main_DE}}

\EN{\input{patterns/ARM/main_EN}}
\RU{\input{patterns/ARM/main_RU}}
\DE{\input{patterns/ARM/main_DE}}

\input{patterns/MIPS/main}

\ifdefined\SPANISH
\chapter{Patrones de código}
\fi % SPANISH

\ifdefined\GERMAN
\chapter{Code-Muster}
\fi % GERMAN

\ifdefined\ENGLISH
\chapter{Code Patterns}
\fi % ENGLISH

\ifdefined\ITALIAN
\chapter{Forme di codice}
\fi % ITALIAN

\ifdefined\RUSSIAN
\chapter{Образцы кода}
\fi % RUSSIAN

\ifdefined\BRAZILIAN
\chapter{Padrões de códigos}
\fi % BRAZILIAN

\ifdefined\THAI
\chapter{รูปแบบของโค้ด}
\fi % THAI

\ifdefined\FRENCH
\chapter{Modèle de code}
\fi % FRENCH

\ifdefined\POLISH
\chapter{\PLph{}}
\fi % POLISH

% sections
\EN{\input{patterns/patterns_opt_dbg_EN}}
\ES{\input{patterns/patterns_opt_dbg_ES}}
\ITA{\input{patterns/patterns_opt_dbg_ITA}}
\PTBR{\input{patterns/patterns_opt_dbg_PTBR}}
\RU{\input{patterns/patterns_opt_dbg_RU}}
\THA{\input{patterns/patterns_opt_dbg_THA}}
\DE{\input{patterns/patterns_opt_dbg_DE}}
\FR{\input{patterns/patterns_opt_dbg_FR}}
\PL{\input{patterns/patterns_opt_dbg_PL}}

\RU{\section{Некоторые базовые понятия}}
\EN{\section{Some basics}}
\DE{\section{Einige Grundlagen}}
\FR{\section{Quelques bases}}
\ES{\section{\ESph{}}}
\ITA{\section{Alcune basi teoriche}}
\PTBR{\section{\PTBRph{}}}
\THA{\section{\THAph{}}}
\PL{\section{\PLph{}}}

% sections:
\EN{\input{patterns/intro_CPU_ISA_EN}}
\ES{\input{patterns/intro_CPU_ISA_ES}}
\ITA{\input{patterns/intro_CPU_ISA_ITA}}
\PTBR{\input{patterns/intro_CPU_ISA_PTBR}}
\RU{\input{patterns/intro_CPU_ISA_RU}}
\DE{\input{patterns/intro_CPU_ISA_DE}}
\FR{\input{patterns/intro_CPU_ISA_FR}}
\PL{\input{patterns/intro_CPU_ISA_PL}}

\EN{\input{patterns/numeral_EN}}
\RU{\input{patterns/numeral_RU}}
\ITA{\input{patterns/numeral_ITA}}
\DE{\input{patterns/numeral_DE}}
\FR{\input{patterns/numeral_FR}}
\PL{\input{patterns/numeral_PL}}

% chapters
\input{patterns/00_empty/main}
\input{patterns/011_ret/main}
\input{patterns/01_helloworld/main}
\input{patterns/015_prolog_epilogue/main}
\input{patterns/02_stack/main}
\input{patterns/03_printf/main}
\input{patterns/04_scanf/main}
\input{patterns/05_passing_arguments/main}
\input{patterns/06_return_results/main}
\input{patterns/061_pointers/main}
\input{patterns/065_GOTO/main}
\input{patterns/07_jcc/main}
\input{patterns/08_switch/main}
\input{patterns/09_loops/main}
\input{patterns/10_strings/main}
\input{patterns/11_arith_optimizations/main}
\input{patterns/12_FPU/main}
\input{patterns/13_arrays/main}
\input{patterns/14_bitfields/main}
\EN{\input{patterns/145_LCG/main_EN}}
\RU{\input{patterns/145_LCG/main_RU}}
\input{patterns/15_structs/main}
\input{patterns/17_unions/main}
\input{patterns/18_pointers_to_functions/main}
\input{patterns/185_64bit_in_32_env/main}

\EN{\input{patterns/19_SIMD/main_EN}}
\RU{\input{patterns/19_SIMD/main_RU}}
\DE{\input{patterns/19_SIMD/main_DE}}

\EN{\input{patterns/20_x64/main_EN}}
\RU{\input{patterns/20_x64/main_RU}}

\EN{\input{patterns/205_floating_SIMD/main_EN}}
\RU{\input{patterns/205_floating_SIMD/main_RU}}
\DE{\input{patterns/205_floating_SIMD/main_DE}}

\EN{\input{patterns/ARM/main_EN}}
\RU{\input{patterns/ARM/main_RU}}
\DE{\input{patterns/ARM/main_DE}}

\input{patterns/MIPS/main}

\ifdefined\SPANISH
\chapter{Patrones de código}
\fi % SPANISH

\ifdefined\GERMAN
\chapter{Code-Muster}
\fi % GERMAN

\ifdefined\ENGLISH
\chapter{Code Patterns}
\fi % ENGLISH

\ifdefined\ITALIAN
\chapter{Forme di codice}
\fi % ITALIAN

\ifdefined\RUSSIAN
\chapter{Образцы кода}
\fi % RUSSIAN

\ifdefined\BRAZILIAN
\chapter{Padrões de códigos}
\fi % BRAZILIAN

\ifdefined\THAI
\chapter{รูปแบบของโค้ด}
\fi % THAI

\ifdefined\FRENCH
\chapter{Modèle de code}
\fi % FRENCH

\ifdefined\POLISH
\chapter{\PLph{}}
\fi % POLISH

% sections
\EN{\input{patterns/patterns_opt_dbg_EN}}
\ES{\input{patterns/patterns_opt_dbg_ES}}
\ITA{\input{patterns/patterns_opt_dbg_ITA}}
\PTBR{\input{patterns/patterns_opt_dbg_PTBR}}
\RU{\input{patterns/patterns_opt_dbg_RU}}
\THA{\input{patterns/patterns_opt_dbg_THA}}
\DE{\input{patterns/patterns_opt_dbg_DE}}
\FR{\input{patterns/patterns_opt_dbg_FR}}
\PL{\input{patterns/patterns_opt_dbg_PL}}

\RU{\section{Некоторые базовые понятия}}
\EN{\section{Some basics}}
\DE{\section{Einige Grundlagen}}
\FR{\section{Quelques bases}}
\ES{\section{\ESph{}}}
\ITA{\section{Alcune basi teoriche}}
\PTBR{\section{\PTBRph{}}}
\THA{\section{\THAph{}}}
\PL{\section{\PLph{}}}

% sections:
\EN{\input{patterns/intro_CPU_ISA_EN}}
\ES{\input{patterns/intro_CPU_ISA_ES}}
\ITA{\input{patterns/intro_CPU_ISA_ITA}}
\PTBR{\input{patterns/intro_CPU_ISA_PTBR}}
\RU{\input{patterns/intro_CPU_ISA_RU}}
\DE{\input{patterns/intro_CPU_ISA_DE}}
\FR{\input{patterns/intro_CPU_ISA_FR}}
\PL{\input{patterns/intro_CPU_ISA_PL}}

\EN{\input{patterns/numeral_EN}}
\RU{\input{patterns/numeral_RU}}
\ITA{\input{patterns/numeral_ITA}}
\DE{\input{patterns/numeral_DE}}
\FR{\input{patterns/numeral_FR}}
\PL{\input{patterns/numeral_PL}}

% chapters
\input{patterns/00_empty/main}
\input{patterns/011_ret/main}
\input{patterns/01_helloworld/main}
\input{patterns/015_prolog_epilogue/main}
\input{patterns/02_stack/main}
\input{patterns/03_printf/main}
\input{patterns/04_scanf/main}
\input{patterns/05_passing_arguments/main}
\input{patterns/06_return_results/main}
\input{patterns/061_pointers/main}
\input{patterns/065_GOTO/main}
\input{patterns/07_jcc/main}
\input{patterns/08_switch/main}
\input{patterns/09_loops/main}
\input{patterns/10_strings/main}
\input{patterns/11_arith_optimizations/main}
\input{patterns/12_FPU/main}
\input{patterns/13_arrays/main}
\input{patterns/14_bitfields/main}
\EN{\input{patterns/145_LCG/main_EN}}
\RU{\input{patterns/145_LCG/main_RU}}
\input{patterns/15_structs/main}
\input{patterns/17_unions/main}
\input{patterns/18_pointers_to_functions/main}
\input{patterns/185_64bit_in_32_env/main}

\EN{\input{patterns/19_SIMD/main_EN}}
\RU{\input{patterns/19_SIMD/main_RU}}
\DE{\input{patterns/19_SIMD/main_DE}}

\EN{\input{patterns/20_x64/main_EN}}
\RU{\input{patterns/20_x64/main_RU}}

\EN{\input{patterns/205_floating_SIMD/main_EN}}
\RU{\input{patterns/205_floating_SIMD/main_RU}}
\DE{\input{patterns/205_floating_SIMD/main_DE}}

\EN{\input{patterns/ARM/main_EN}}
\RU{\input{patterns/ARM/main_RU}}
\DE{\input{patterns/ARM/main_DE}}

\input{patterns/MIPS/main}

\ifdefined\SPANISH
\chapter{Patrones de código}
\fi % SPANISH

\ifdefined\GERMAN
\chapter{Code-Muster}
\fi % GERMAN

\ifdefined\ENGLISH
\chapter{Code Patterns}
\fi % ENGLISH

\ifdefined\ITALIAN
\chapter{Forme di codice}
\fi % ITALIAN

\ifdefined\RUSSIAN
\chapter{Образцы кода}
\fi % RUSSIAN

\ifdefined\BRAZILIAN
\chapter{Padrões de códigos}
\fi % BRAZILIAN

\ifdefined\THAI
\chapter{รูปแบบของโค้ด}
\fi % THAI

\ifdefined\FRENCH
\chapter{Modèle de code}
\fi % FRENCH

\ifdefined\POLISH
\chapter{\PLph{}}
\fi % POLISH

% sections
\EN{\input{patterns/patterns_opt_dbg_EN}}
\ES{\input{patterns/patterns_opt_dbg_ES}}
\ITA{\input{patterns/patterns_opt_dbg_ITA}}
\PTBR{\input{patterns/patterns_opt_dbg_PTBR}}
\RU{\input{patterns/patterns_opt_dbg_RU}}
\THA{\input{patterns/patterns_opt_dbg_THA}}
\DE{\input{patterns/patterns_opt_dbg_DE}}
\FR{\input{patterns/patterns_opt_dbg_FR}}
\PL{\input{patterns/patterns_opt_dbg_PL}}

\RU{\section{Некоторые базовые понятия}}
\EN{\section{Some basics}}
\DE{\section{Einige Grundlagen}}
\FR{\section{Quelques bases}}
\ES{\section{\ESph{}}}
\ITA{\section{Alcune basi teoriche}}
\PTBR{\section{\PTBRph{}}}
\THA{\section{\THAph{}}}
\PL{\section{\PLph{}}}

% sections:
\EN{\input{patterns/intro_CPU_ISA_EN}}
\ES{\input{patterns/intro_CPU_ISA_ES}}
\ITA{\input{patterns/intro_CPU_ISA_ITA}}
\PTBR{\input{patterns/intro_CPU_ISA_PTBR}}
\RU{\input{patterns/intro_CPU_ISA_RU}}
\DE{\input{patterns/intro_CPU_ISA_DE}}
\FR{\input{patterns/intro_CPU_ISA_FR}}
\PL{\input{patterns/intro_CPU_ISA_PL}}

\EN{\input{patterns/numeral_EN}}
\RU{\input{patterns/numeral_RU}}
\ITA{\input{patterns/numeral_ITA}}
\DE{\input{patterns/numeral_DE}}
\FR{\input{patterns/numeral_FR}}
\PL{\input{patterns/numeral_PL}}

% chapters
\input{patterns/00_empty/main}
\input{patterns/011_ret/main}
\input{patterns/01_helloworld/main}
\input{patterns/015_prolog_epilogue/main}
\input{patterns/02_stack/main}
\input{patterns/03_printf/main}
\input{patterns/04_scanf/main}
\input{patterns/05_passing_arguments/main}
\input{patterns/06_return_results/main}
\input{patterns/061_pointers/main}
\input{patterns/065_GOTO/main}
\input{patterns/07_jcc/main}
\input{patterns/08_switch/main}
\input{patterns/09_loops/main}
\input{patterns/10_strings/main}
\input{patterns/11_arith_optimizations/main}
\input{patterns/12_FPU/main}
\input{patterns/13_arrays/main}
\input{patterns/14_bitfields/main}
\EN{\input{patterns/145_LCG/main_EN}}
\RU{\input{patterns/145_LCG/main_RU}}
\input{patterns/15_structs/main}
\input{patterns/17_unions/main}
\input{patterns/18_pointers_to_functions/main}
\input{patterns/185_64bit_in_32_env/main}

\EN{\input{patterns/19_SIMD/main_EN}}
\RU{\input{patterns/19_SIMD/main_RU}}
\DE{\input{patterns/19_SIMD/main_DE}}

\EN{\input{patterns/20_x64/main_EN}}
\RU{\input{patterns/20_x64/main_RU}}

\EN{\input{patterns/205_floating_SIMD/main_EN}}
\RU{\input{patterns/205_floating_SIMD/main_RU}}
\DE{\input{patterns/205_floating_SIMD/main_DE}}

\EN{\input{patterns/ARM/main_EN}}
\RU{\input{patterns/ARM/main_RU}}
\DE{\input{patterns/ARM/main_DE}}

\input{patterns/MIPS/main}

\ifdefined\SPANISH
\chapter{Patrones de código}
\fi % SPANISH

\ifdefined\GERMAN
\chapter{Code-Muster}
\fi % GERMAN

\ifdefined\ENGLISH
\chapter{Code Patterns}
\fi % ENGLISH

\ifdefined\ITALIAN
\chapter{Forme di codice}
\fi % ITALIAN

\ifdefined\RUSSIAN
\chapter{Образцы кода}
\fi % RUSSIAN

\ifdefined\BRAZILIAN
\chapter{Padrões de códigos}
\fi % BRAZILIAN

\ifdefined\THAI
\chapter{รูปแบบของโค้ด}
\fi % THAI

\ifdefined\FRENCH
\chapter{Modèle de code}
\fi % FRENCH

\ifdefined\POLISH
\chapter{\PLph{}}
\fi % POLISH

% sections
\EN{\input{patterns/patterns_opt_dbg_EN}}
\ES{\input{patterns/patterns_opt_dbg_ES}}
\ITA{\input{patterns/patterns_opt_dbg_ITA}}
\PTBR{\input{patterns/patterns_opt_dbg_PTBR}}
\RU{\input{patterns/patterns_opt_dbg_RU}}
\THA{\input{patterns/patterns_opt_dbg_THA}}
\DE{\input{patterns/patterns_opt_dbg_DE}}
\FR{\input{patterns/patterns_opt_dbg_FR}}
\PL{\input{patterns/patterns_opt_dbg_PL}}

\RU{\section{Некоторые базовые понятия}}
\EN{\section{Some basics}}
\DE{\section{Einige Grundlagen}}
\FR{\section{Quelques bases}}
\ES{\section{\ESph{}}}
\ITA{\section{Alcune basi teoriche}}
\PTBR{\section{\PTBRph{}}}
\THA{\section{\THAph{}}}
\PL{\section{\PLph{}}}

% sections:
\EN{\input{patterns/intro_CPU_ISA_EN}}
\ES{\input{patterns/intro_CPU_ISA_ES}}
\ITA{\input{patterns/intro_CPU_ISA_ITA}}
\PTBR{\input{patterns/intro_CPU_ISA_PTBR}}
\RU{\input{patterns/intro_CPU_ISA_RU}}
\DE{\input{patterns/intro_CPU_ISA_DE}}
\FR{\input{patterns/intro_CPU_ISA_FR}}
\PL{\input{patterns/intro_CPU_ISA_PL}}

\EN{\input{patterns/numeral_EN}}
\RU{\input{patterns/numeral_RU}}
\ITA{\input{patterns/numeral_ITA}}
\DE{\input{patterns/numeral_DE}}
\FR{\input{patterns/numeral_FR}}
\PL{\input{patterns/numeral_PL}}

% chapters
\input{patterns/00_empty/main}
\input{patterns/011_ret/main}
\input{patterns/01_helloworld/main}
\input{patterns/015_prolog_epilogue/main}
\input{patterns/02_stack/main}
\input{patterns/03_printf/main}
\input{patterns/04_scanf/main}
\input{patterns/05_passing_arguments/main}
\input{patterns/06_return_results/main}
\input{patterns/061_pointers/main}
\input{patterns/065_GOTO/main}
\input{patterns/07_jcc/main}
\input{patterns/08_switch/main}
\input{patterns/09_loops/main}
\input{patterns/10_strings/main}
\input{patterns/11_arith_optimizations/main}
\input{patterns/12_FPU/main}
\input{patterns/13_arrays/main}
\input{patterns/14_bitfields/main}
\EN{\input{patterns/145_LCG/main_EN}}
\RU{\input{patterns/145_LCG/main_RU}}
\input{patterns/15_structs/main}
\input{patterns/17_unions/main}
\input{patterns/18_pointers_to_functions/main}
\input{patterns/185_64bit_in_32_env/main}

\EN{\input{patterns/19_SIMD/main_EN}}
\RU{\input{patterns/19_SIMD/main_RU}}
\DE{\input{patterns/19_SIMD/main_DE}}

\EN{\input{patterns/20_x64/main_EN}}
\RU{\input{patterns/20_x64/main_RU}}

\EN{\input{patterns/205_floating_SIMD/main_EN}}
\RU{\input{patterns/205_floating_SIMD/main_RU}}
\DE{\input{patterns/205_floating_SIMD/main_DE}}

\EN{\input{patterns/ARM/main_EN}}
\RU{\input{patterns/ARM/main_RU}}
\DE{\input{patterns/ARM/main_DE}}

\input{patterns/MIPS/main}

\ifdefined\SPANISH
\chapter{Patrones de código}
\fi % SPANISH

\ifdefined\GERMAN
\chapter{Code-Muster}
\fi % GERMAN

\ifdefined\ENGLISH
\chapter{Code Patterns}
\fi % ENGLISH

\ifdefined\ITALIAN
\chapter{Forme di codice}
\fi % ITALIAN

\ifdefined\RUSSIAN
\chapter{Образцы кода}
\fi % RUSSIAN

\ifdefined\BRAZILIAN
\chapter{Padrões de códigos}
\fi % BRAZILIAN

\ifdefined\THAI
\chapter{รูปแบบของโค้ด}
\fi % THAI

\ifdefined\FRENCH
\chapter{Modèle de code}
\fi % FRENCH

\ifdefined\POLISH
\chapter{\PLph{}}
\fi % POLISH

% sections
\EN{\input{patterns/patterns_opt_dbg_EN}}
\ES{\input{patterns/patterns_opt_dbg_ES}}
\ITA{\input{patterns/patterns_opt_dbg_ITA}}
\PTBR{\input{patterns/patterns_opt_dbg_PTBR}}
\RU{\input{patterns/patterns_opt_dbg_RU}}
\THA{\input{patterns/patterns_opt_dbg_THA}}
\DE{\input{patterns/patterns_opt_dbg_DE}}
\FR{\input{patterns/patterns_opt_dbg_FR}}
\PL{\input{patterns/patterns_opt_dbg_PL}}

\RU{\section{Некоторые базовые понятия}}
\EN{\section{Some basics}}
\DE{\section{Einige Grundlagen}}
\FR{\section{Quelques bases}}
\ES{\section{\ESph{}}}
\ITA{\section{Alcune basi teoriche}}
\PTBR{\section{\PTBRph{}}}
\THA{\section{\THAph{}}}
\PL{\section{\PLph{}}}

% sections:
\EN{\input{patterns/intro_CPU_ISA_EN}}
\ES{\input{patterns/intro_CPU_ISA_ES}}
\ITA{\input{patterns/intro_CPU_ISA_ITA}}
\PTBR{\input{patterns/intro_CPU_ISA_PTBR}}
\RU{\input{patterns/intro_CPU_ISA_RU}}
\DE{\input{patterns/intro_CPU_ISA_DE}}
\FR{\input{patterns/intro_CPU_ISA_FR}}
\PL{\input{patterns/intro_CPU_ISA_PL}}

\EN{\input{patterns/numeral_EN}}
\RU{\input{patterns/numeral_RU}}
\ITA{\input{patterns/numeral_ITA}}
\DE{\input{patterns/numeral_DE}}
\FR{\input{patterns/numeral_FR}}
\PL{\input{patterns/numeral_PL}}

% chapters
\input{patterns/00_empty/main}
\input{patterns/011_ret/main}
\input{patterns/01_helloworld/main}
\input{patterns/015_prolog_epilogue/main}
\input{patterns/02_stack/main}
\input{patterns/03_printf/main}
\input{patterns/04_scanf/main}
\input{patterns/05_passing_arguments/main}
\input{patterns/06_return_results/main}
\input{patterns/061_pointers/main}
\input{patterns/065_GOTO/main}
\input{patterns/07_jcc/main}
\input{patterns/08_switch/main}
\input{patterns/09_loops/main}
\input{patterns/10_strings/main}
\input{patterns/11_arith_optimizations/main}
\input{patterns/12_FPU/main}
\input{patterns/13_arrays/main}
\input{patterns/14_bitfields/main}
\EN{\input{patterns/145_LCG/main_EN}}
\RU{\input{patterns/145_LCG/main_RU}}
\input{patterns/15_structs/main}
\input{patterns/17_unions/main}
\input{patterns/18_pointers_to_functions/main}
\input{patterns/185_64bit_in_32_env/main}

\EN{\input{patterns/19_SIMD/main_EN}}
\RU{\input{patterns/19_SIMD/main_RU}}
\DE{\input{patterns/19_SIMD/main_DE}}

\EN{\input{patterns/20_x64/main_EN}}
\RU{\input{patterns/20_x64/main_RU}}

\EN{\input{patterns/205_floating_SIMD/main_EN}}
\RU{\input{patterns/205_floating_SIMD/main_RU}}
\DE{\input{patterns/205_floating_SIMD/main_DE}}

\EN{\input{patterns/ARM/main_EN}}
\RU{\input{patterns/ARM/main_RU}}
\DE{\input{patterns/ARM/main_DE}}

\input{patterns/MIPS/main}

\EN{\input{patterns/12_FPU/main_EN}}
\RU{\input{patterns/12_FPU/main_RU}}
\DE{\input{patterns/12_FPU/main_DE}}
\FR{\input{patterns/12_FPU/main_FR}}


\ifdefined\SPANISH
\chapter{Patrones de código}
\fi % SPANISH

\ifdefined\GERMAN
\chapter{Code-Muster}
\fi % GERMAN

\ifdefined\ENGLISH
\chapter{Code Patterns}
\fi % ENGLISH

\ifdefined\ITALIAN
\chapter{Forme di codice}
\fi % ITALIAN

\ifdefined\RUSSIAN
\chapter{Образцы кода}
\fi % RUSSIAN

\ifdefined\BRAZILIAN
\chapter{Padrões de códigos}
\fi % BRAZILIAN

\ifdefined\THAI
\chapter{รูปแบบของโค้ด}
\fi % THAI

\ifdefined\FRENCH
\chapter{Modèle de code}
\fi % FRENCH

\ifdefined\POLISH
\chapter{\PLph{}}
\fi % POLISH

% sections
\EN{\input{patterns/patterns_opt_dbg_EN}}
\ES{\input{patterns/patterns_opt_dbg_ES}}
\ITA{\input{patterns/patterns_opt_dbg_ITA}}
\PTBR{\input{patterns/patterns_opt_dbg_PTBR}}
\RU{\input{patterns/patterns_opt_dbg_RU}}
\THA{\input{patterns/patterns_opt_dbg_THA}}
\DE{\input{patterns/patterns_opt_dbg_DE}}
\FR{\input{patterns/patterns_opt_dbg_FR}}
\PL{\input{patterns/patterns_opt_dbg_PL}}

\RU{\section{Некоторые базовые понятия}}
\EN{\section{Some basics}}
\DE{\section{Einige Grundlagen}}
\FR{\section{Quelques bases}}
\ES{\section{\ESph{}}}
\ITA{\section{Alcune basi teoriche}}
\PTBR{\section{\PTBRph{}}}
\THA{\section{\THAph{}}}
\PL{\section{\PLph{}}}

% sections:
\EN{\input{patterns/intro_CPU_ISA_EN}}
\ES{\input{patterns/intro_CPU_ISA_ES}}
\ITA{\input{patterns/intro_CPU_ISA_ITA}}
\PTBR{\input{patterns/intro_CPU_ISA_PTBR}}
\RU{\input{patterns/intro_CPU_ISA_RU}}
\DE{\input{patterns/intro_CPU_ISA_DE}}
\FR{\input{patterns/intro_CPU_ISA_FR}}
\PL{\input{patterns/intro_CPU_ISA_PL}}

\EN{\input{patterns/numeral_EN}}
\RU{\input{patterns/numeral_RU}}
\ITA{\input{patterns/numeral_ITA}}
\DE{\input{patterns/numeral_DE}}
\FR{\input{patterns/numeral_FR}}
\PL{\input{patterns/numeral_PL}}

% chapters
\input{patterns/00_empty/main}
\input{patterns/011_ret/main}
\input{patterns/01_helloworld/main}
\input{patterns/015_prolog_epilogue/main}
\input{patterns/02_stack/main}
\input{patterns/03_printf/main}
\input{patterns/04_scanf/main}
\input{patterns/05_passing_arguments/main}
\input{patterns/06_return_results/main}
\input{patterns/061_pointers/main}
\input{patterns/065_GOTO/main}
\input{patterns/07_jcc/main}
\input{patterns/08_switch/main}
\input{patterns/09_loops/main}
\input{patterns/10_strings/main}
\input{patterns/11_arith_optimizations/main}
\input{patterns/12_FPU/main}
\input{patterns/13_arrays/main}
\input{patterns/14_bitfields/main}
\EN{\input{patterns/145_LCG/main_EN}}
\RU{\input{patterns/145_LCG/main_RU}}
\input{patterns/15_structs/main}
\input{patterns/17_unions/main}
\input{patterns/18_pointers_to_functions/main}
\input{patterns/185_64bit_in_32_env/main}

\EN{\input{patterns/19_SIMD/main_EN}}
\RU{\input{patterns/19_SIMD/main_RU}}
\DE{\input{patterns/19_SIMD/main_DE}}

\EN{\input{patterns/20_x64/main_EN}}
\RU{\input{patterns/20_x64/main_RU}}

\EN{\input{patterns/205_floating_SIMD/main_EN}}
\RU{\input{patterns/205_floating_SIMD/main_RU}}
\DE{\input{patterns/205_floating_SIMD/main_DE}}

\EN{\input{patterns/ARM/main_EN}}
\RU{\input{patterns/ARM/main_RU}}
\DE{\input{patterns/ARM/main_DE}}

\input{patterns/MIPS/main}

\ifdefined\SPANISH
\chapter{Patrones de código}
\fi % SPANISH

\ifdefined\GERMAN
\chapter{Code-Muster}
\fi % GERMAN

\ifdefined\ENGLISH
\chapter{Code Patterns}
\fi % ENGLISH

\ifdefined\ITALIAN
\chapter{Forme di codice}
\fi % ITALIAN

\ifdefined\RUSSIAN
\chapter{Образцы кода}
\fi % RUSSIAN

\ifdefined\BRAZILIAN
\chapter{Padrões de códigos}
\fi % BRAZILIAN

\ifdefined\THAI
\chapter{รูปแบบของโค้ด}
\fi % THAI

\ifdefined\FRENCH
\chapter{Modèle de code}
\fi % FRENCH

\ifdefined\POLISH
\chapter{\PLph{}}
\fi % POLISH

% sections
\EN{\input{patterns/patterns_opt_dbg_EN}}
\ES{\input{patterns/patterns_opt_dbg_ES}}
\ITA{\input{patterns/patterns_opt_dbg_ITA}}
\PTBR{\input{patterns/patterns_opt_dbg_PTBR}}
\RU{\input{patterns/patterns_opt_dbg_RU}}
\THA{\input{patterns/patterns_opt_dbg_THA}}
\DE{\input{patterns/patterns_opt_dbg_DE}}
\FR{\input{patterns/patterns_opt_dbg_FR}}
\PL{\input{patterns/patterns_opt_dbg_PL}}

\RU{\section{Некоторые базовые понятия}}
\EN{\section{Some basics}}
\DE{\section{Einige Grundlagen}}
\FR{\section{Quelques bases}}
\ES{\section{\ESph{}}}
\ITA{\section{Alcune basi teoriche}}
\PTBR{\section{\PTBRph{}}}
\THA{\section{\THAph{}}}
\PL{\section{\PLph{}}}

% sections:
\EN{\input{patterns/intro_CPU_ISA_EN}}
\ES{\input{patterns/intro_CPU_ISA_ES}}
\ITA{\input{patterns/intro_CPU_ISA_ITA}}
\PTBR{\input{patterns/intro_CPU_ISA_PTBR}}
\RU{\input{patterns/intro_CPU_ISA_RU}}
\DE{\input{patterns/intro_CPU_ISA_DE}}
\FR{\input{patterns/intro_CPU_ISA_FR}}
\PL{\input{patterns/intro_CPU_ISA_PL}}

\EN{\input{patterns/numeral_EN}}
\RU{\input{patterns/numeral_RU}}
\ITA{\input{patterns/numeral_ITA}}
\DE{\input{patterns/numeral_DE}}
\FR{\input{patterns/numeral_FR}}
\PL{\input{patterns/numeral_PL}}

% chapters
\input{patterns/00_empty/main}
\input{patterns/011_ret/main}
\input{patterns/01_helloworld/main}
\input{patterns/015_prolog_epilogue/main}
\input{patterns/02_stack/main}
\input{patterns/03_printf/main}
\input{patterns/04_scanf/main}
\input{patterns/05_passing_arguments/main}
\input{patterns/06_return_results/main}
\input{patterns/061_pointers/main}
\input{patterns/065_GOTO/main}
\input{patterns/07_jcc/main}
\input{patterns/08_switch/main}
\input{patterns/09_loops/main}
\input{patterns/10_strings/main}
\input{patterns/11_arith_optimizations/main}
\input{patterns/12_FPU/main}
\input{patterns/13_arrays/main}
\input{patterns/14_bitfields/main}
\EN{\input{patterns/145_LCG/main_EN}}
\RU{\input{patterns/145_LCG/main_RU}}
\input{patterns/15_structs/main}
\input{patterns/17_unions/main}
\input{patterns/18_pointers_to_functions/main}
\input{patterns/185_64bit_in_32_env/main}

\EN{\input{patterns/19_SIMD/main_EN}}
\RU{\input{patterns/19_SIMD/main_RU}}
\DE{\input{patterns/19_SIMD/main_DE}}

\EN{\input{patterns/20_x64/main_EN}}
\RU{\input{patterns/20_x64/main_RU}}

\EN{\input{patterns/205_floating_SIMD/main_EN}}
\RU{\input{patterns/205_floating_SIMD/main_RU}}
\DE{\input{patterns/205_floating_SIMD/main_DE}}

\EN{\input{patterns/ARM/main_EN}}
\RU{\input{patterns/ARM/main_RU}}
\DE{\input{patterns/ARM/main_DE}}

\input{patterns/MIPS/main}

\EN{\section{Returning Values}
\label{ret_val_func}

Another simple function is the one that simply returns a constant value:

\lstinputlisting[caption=\EN{\CCpp Code},style=customc]{patterns/011_ret/1.c}

Let's compile it.

\subsection{x86}

Here's what both the GCC and MSVC compilers produce (with optimization) on the x86 platform:

\lstinputlisting[caption=\Optimizing GCC/MSVC (\assemblyOutput),style=customasmx86]{patterns/011_ret/1.s}

\myindex{x86!\Instructions!RET}
There are just two instructions: the first places the value 123 into the \EAX register,
which is used by convention for storing the return
value, and the second one is \RET, which returns execution to the \gls{caller}.

The caller will take the result from the \EAX register.

\subsection{ARM}

There are a few differences on the ARM platform:

\lstinputlisting[caption=\OptimizingKeilVI (\ARMMode) ASM Output,style=customasmARM]{patterns/011_ret/1_Keil_ARM_O3.s}

ARM uses the register \Reg{0} for returning the results of functions, so 123 is copied into \Reg{0}.

\myindex{ARM!\Instructions!MOV}
\myindex{x86!\Instructions!MOV}
It is worth noting that \MOV is a misleading name for the instruction in both the x86 and ARM \ac{ISA}s.

The data is not in fact \IT{moved}, but \IT{copied}.

\subsection{MIPS}

\label{MIPS_leaf_function_ex1}

The GCC assembly output below lists registers by number:

\lstinputlisting[caption=\Optimizing GCC 4.4.5 (\assemblyOutput),style=customasmMIPS]{patterns/011_ret/MIPS.s}

\dots while \IDA does it by their pseudo names:

\lstinputlisting[caption=\Optimizing GCC 4.4.5 (IDA),style=customasmMIPS]{patterns/011_ret/MIPS_IDA.lst}

The \$2 (or \$V0) register is used to store the function's return value.
\myindex{MIPS!\Pseudoinstructions!LI}
\INS{LI} stands for ``Load Immediate'' and is the MIPS equivalent to \MOV.

\myindex{MIPS!\Instructions!J}
The other instruction is the jump instruction (J or JR) which returns the execution flow to the \gls{caller}.

\myindex{MIPS!Branch delay slot}
You might be wondering why the positions of the load instruction (LI) and the jump instruction (J or JR) are swapped. This is due to a \ac{RISC} feature called ``branch delay slot''.

The reason this happens is a quirk in the architecture of some RISC \ac{ISA}s and isn't important for our
purposes---we must simply keep in mind that in MIPS, the instruction following a jump or branch instruction
is executed \IT{before} the jump/branch instruction itself.

As a consequence, branch instructions always swap places with the instruction executed immediately beforehand.


In practice, functions which merely return 1 (\IT{true}) or 0 (\IT{false}) are very frequent.

The smallest ever of the standard UNIX utilities, \IT{/bin/true} and \IT{/bin/false} return 0 and 1 respectively, as an exit code.
(Zero as an exit code usually means success, non-zero means error.)
}
\RU{\subsubsection{std::string}
\myindex{\Cpp!STL!std::string}
\label{std_string}

\myparagraph{Как устроена структура}

Многие строковые библиотеки \InSqBrackets{\CNotes 2.2} обеспечивают структуру содержащую ссылку 
на буфер собственно со строкой, переменная всегда содержащую длину строки 
(что очень удобно для массы функций \InSqBrackets{\CNotes 2.2.1}) и переменную содержащую текущий размер буфера.

Строка в буфере обыкновенно оканчивается нулем: это для того чтобы указатель на буфер можно было
передавать в функции требующие на вход обычную сишную \ac{ASCIIZ}-строку.

Стандарт \Cpp не описывает, как именно нужно реализовывать std::string,
но, как правило, они реализованы как описано выше, с небольшими дополнениями.

Строки в \Cpp это не класс (как, например, QString в Qt), а темплейт (basic\_string), 
это сделано для того чтобы поддерживать 
строки содержащие разного типа символы: как минимум \Tchar и \IT{wchar\_t}.

Так что, std::string это класс с базовым типом \Tchar.

А std::wstring это класс с базовым типом \IT{wchar\_t}.

\mysubparagraph{MSVC}

В реализации MSVC, вместо ссылки на буфер может содержаться сам буфер (если строка короче 16-и символов).

Это означает, что каждая короткая строка будет занимать в памяти по крайней мере $16 + 4 + 4 = 24$ 
байт для 32-битной среды либо $16 + 8 + 8 = 32$ 
байта в 64-битной, а если строка длиннее 16-и символов, то прибавьте еще длину самой строки.

\lstinputlisting[caption=пример для MSVC,style=customc]{\CURPATH/STL/string/MSVC_RU.cpp}

Собственно, из этого исходника почти всё ясно.

Несколько замечаний:

Если строка короче 16-и символов, 
то отдельный буфер для строки в \glslink{heap}{куче} выделяться не будет.

Это удобно потому что на практике, основная часть строк действительно короткие.
Вероятно, разработчики в Microsoft выбрали размер в 16 символов как разумный баланс.

Теперь очень важный момент в конце функции main(): мы не пользуемся методом c\_str(), тем не менее,
если это скомпилировать и запустить, то обе строки появятся в консоли!

Работает это вот почему.

В первом случае строка короче 16-и символов и в начале объекта std::string (его можно рассматривать
просто как структуру) расположен буфер с этой строкой.
\printf трактует указатель как указатель на массив символов оканчивающийся нулем и поэтому всё работает.

Вывод второй строки (длиннее 16-и символов) даже еще опаснее: это вообще типичная программистская ошибка 
(или опечатка), забыть дописать c\_str().
Это работает потому что в это время в начале структуры расположен указатель на буфер.
Это может надолго остаться незамеченным: до тех пока там не появится строка 
короче 16-и символов, тогда процесс упадет.

\mysubparagraph{GCC}

В реализации GCC в структуре есть еще одна переменная --- reference count.

Интересно, что указатель на экземпляр класса std::string в GCC указывает не на начало самой структуры, 
а на указатель на буфера.
В libstdc++-v3\textbackslash{}include\textbackslash{}bits\textbackslash{}basic\_string.h 
мы можем прочитать что это сделано для удобства отладки:

\begin{lstlisting}
   *  The reason you want _M_data pointing to the character %array and
   *  not the _Rep is so that the debugger can see the string
   *  contents. (Probably we should add a non-inline member to get
   *  the _Rep for the debugger to use, so users can check the actual
   *  string length.)
\end{lstlisting}

\href{http://go.yurichev.com/17085}{исходный код basic\_string.h}

В нашем примере мы учитываем это:

\lstinputlisting[caption=пример для GCC,style=customc]{\CURPATH/STL/string/GCC_RU.cpp}

Нужны еще небольшие хаки чтобы сымитировать типичную ошибку, которую мы уже видели выше, из-за
более ужесточенной проверки типов в GCC, тем не менее, printf() работает и здесь без c\_str().

\myparagraph{Чуть более сложный пример}

\lstinputlisting[style=customc]{\CURPATH/STL/string/3.cpp}

\lstinputlisting[caption=MSVC 2012,style=customasmx86]{\CURPATH/STL/string/3_MSVC_RU.asm}

Собственно, компилятор не конструирует строки статически: да в общем-то и как
это возможно, если буфер с ней нужно хранить в \glslink{heap}{куче}?

Вместо этого в сегменте данных хранятся обычные \ac{ASCIIZ}-строки, а позже, во время выполнения, 
при помощи метода \q{assign}, конструируются строки s1 и s2
.
При помощи \TT{operator+}, создается строка s3.

Обратите внимание на то что вызов метода c\_str() отсутствует,
потому что его код достаточно короткий и компилятор вставил его прямо здесь:
если строка короче 16-и байт, то в регистре EAX остается указатель на буфер,
а если длиннее, то из этого же места достается адрес на буфер расположенный в \glslink{heap}{куче}.

Далее следуют вызовы трех деструкторов, причем, они вызываются только если строка длиннее 16-и байт:
тогда нужно освободить буфера в \glslink{heap}{куче}.
В противном случае, так как все три объекта std::string хранятся в стеке,
они освобождаются автоматически после выхода из функции.

Следовательно, работа с короткими строками более быстрая из-за м\'{е}ньшего обращения к \glslink{heap}{куче}.

Код на GCC даже проще (из-за того, что в GCC, как мы уже видели, не реализована возможность хранить короткую
строку прямо в структуре):

% TODO1 comment each function meaning
\lstinputlisting[caption=GCC 4.8.1,style=customasmx86]{\CURPATH/STL/string/3_GCC_RU.s}

Можно заметить, что в деструкторы передается не указатель на объект,
а указатель на место за 12 байт (или 3 слова) перед ним, то есть, на настоящее начало структуры.

\myparagraph{std::string как глобальная переменная}
\label{sec:std_string_as_global_variable}

Опытные программисты на \Cpp знают, что глобальные переменные \ac{STL}-типов вполне можно объявлять.

Да, действительно:

\lstinputlisting[style=customc]{\CURPATH/STL/string/5.cpp}

Но как и где будет вызываться конструктор \TT{std::string}?

На самом деле, эта переменная будет инициализирована даже перед началом \main.

\lstinputlisting[caption=MSVC 2012: здесь конструируется глобальная переменная{,} а также регистрируется её деструктор,style=customasmx86]{\CURPATH/STL/string/5_MSVC_p2.asm}

\lstinputlisting[caption=MSVC 2012: здесь глобальная переменная используется в \main,style=customasmx86]{\CURPATH/STL/string/5_MSVC_p1.asm}

\lstinputlisting[caption=MSVC 2012: эта функция-деструктор вызывается перед выходом,style=customasmx86]{\CURPATH/STL/string/5_MSVC_p3.asm}

\myindex{\CStandardLibrary!atexit()}
В реальности, из \ac{CRT}, еще до вызова main(), вызывается специальная функция,
в которой перечислены все конструкторы подобных переменных.
Более того: при помощи atexit() регистрируется функция, которая будет вызвана в конце работы программы:
в этой функции компилятор собирает вызовы деструкторов всех подобных глобальных переменных.

GCC работает похожим образом:

\lstinputlisting[caption=GCC 4.8.1,style=customasmx86]{\CURPATH/STL/string/5_GCC.s}

Но он не выделяет отдельной функции в которой будут собраны деструкторы: 
каждый деструктор передается в atexit() по одному.

% TODO а если глобальная STL-переменная в другом модуле? надо проверить.

}
\ifdefined\SPANISH
\chapter{Patrones de código}
\fi % SPANISH

\ifdefined\GERMAN
\chapter{Code-Muster}
\fi % GERMAN

\ifdefined\ENGLISH
\chapter{Code Patterns}
\fi % ENGLISH

\ifdefined\ITALIAN
\chapter{Forme di codice}
\fi % ITALIAN

\ifdefined\RUSSIAN
\chapter{Образцы кода}
\fi % RUSSIAN

\ifdefined\BRAZILIAN
\chapter{Padrões de códigos}
\fi % BRAZILIAN

\ifdefined\THAI
\chapter{รูปแบบของโค้ด}
\fi % THAI

\ifdefined\FRENCH
\chapter{Modèle de code}
\fi % FRENCH

\ifdefined\POLISH
\chapter{\PLph{}}
\fi % POLISH

% sections
\EN{\input{patterns/patterns_opt_dbg_EN}}
\ES{\input{patterns/patterns_opt_dbg_ES}}
\ITA{\input{patterns/patterns_opt_dbg_ITA}}
\PTBR{\input{patterns/patterns_opt_dbg_PTBR}}
\RU{\input{patterns/patterns_opt_dbg_RU}}
\THA{\input{patterns/patterns_opt_dbg_THA}}
\DE{\input{patterns/patterns_opt_dbg_DE}}
\FR{\input{patterns/patterns_opt_dbg_FR}}
\PL{\input{patterns/patterns_opt_dbg_PL}}

\RU{\section{Некоторые базовые понятия}}
\EN{\section{Some basics}}
\DE{\section{Einige Grundlagen}}
\FR{\section{Quelques bases}}
\ES{\section{\ESph{}}}
\ITA{\section{Alcune basi teoriche}}
\PTBR{\section{\PTBRph{}}}
\THA{\section{\THAph{}}}
\PL{\section{\PLph{}}}

% sections:
\EN{\input{patterns/intro_CPU_ISA_EN}}
\ES{\input{patterns/intro_CPU_ISA_ES}}
\ITA{\input{patterns/intro_CPU_ISA_ITA}}
\PTBR{\input{patterns/intro_CPU_ISA_PTBR}}
\RU{\input{patterns/intro_CPU_ISA_RU}}
\DE{\input{patterns/intro_CPU_ISA_DE}}
\FR{\input{patterns/intro_CPU_ISA_FR}}
\PL{\input{patterns/intro_CPU_ISA_PL}}

\EN{\input{patterns/numeral_EN}}
\RU{\input{patterns/numeral_RU}}
\ITA{\input{patterns/numeral_ITA}}
\DE{\input{patterns/numeral_DE}}
\FR{\input{patterns/numeral_FR}}
\PL{\input{patterns/numeral_PL}}

% chapters
\input{patterns/00_empty/main}
\input{patterns/011_ret/main}
\input{patterns/01_helloworld/main}
\input{patterns/015_prolog_epilogue/main}
\input{patterns/02_stack/main}
\input{patterns/03_printf/main}
\input{patterns/04_scanf/main}
\input{patterns/05_passing_arguments/main}
\input{patterns/06_return_results/main}
\input{patterns/061_pointers/main}
\input{patterns/065_GOTO/main}
\input{patterns/07_jcc/main}
\input{patterns/08_switch/main}
\input{patterns/09_loops/main}
\input{patterns/10_strings/main}
\input{patterns/11_arith_optimizations/main}
\input{patterns/12_FPU/main}
\input{patterns/13_arrays/main}
\input{patterns/14_bitfields/main}
\EN{\input{patterns/145_LCG/main_EN}}
\RU{\input{patterns/145_LCG/main_RU}}
\input{patterns/15_structs/main}
\input{patterns/17_unions/main}
\input{patterns/18_pointers_to_functions/main}
\input{patterns/185_64bit_in_32_env/main}

\EN{\input{patterns/19_SIMD/main_EN}}
\RU{\input{patterns/19_SIMD/main_RU}}
\DE{\input{patterns/19_SIMD/main_DE}}

\EN{\input{patterns/20_x64/main_EN}}
\RU{\input{patterns/20_x64/main_RU}}

\EN{\input{patterns/205_floating_SIMD/main_EN}}
\RU{\input{patterns/205_floating_SIMD/main_RU}}
\DE{\input{patterns/205_floating_SIMD/main_DE}}

\EN{\input{patterns/ARM/main_EN}}
\RU{\input{patterns/ARM/main_RU}}
\DE{\input{patterns/ARM/main_DE}}

\input{patterns/MIPS/main}

\ifdefined\SPANISH
\chapter{Patrones de código}
\fi % SPANISH

\ifdefined\GERMAN
\chapter{Code-Muster}
\fi % GERMAN

\ifdefined\ENGLISH
\chapter{Code Patterns}
\fi % ENGLISH

\ifdefined\ITALIAN
\chapter{Forme di codice}
\fi % ITALIAN

\ifdefined\RUSSIAN
\chapter{Образцы кода}
\fi % RUSSIAN

\ifdefined\BRAZILIAN
\chapter{Padrões de códigos}
\fi % BRAZILIAN

\ifdefined\THAI
\chapter{รูปแบบของโค้ด}
\fi % THAI

\ifdefined\FRENCH
\chapter{Modèle de code}
\fi % FRENCH

\ifdefined\POLISH
\chapter{\PLph{}}
\fi % POLISH

% sections
\EN{\input{patterns/patterns_opt_dbg_EN}}
\ES{\input{patterns/patterns_opt_dbg_ES}}
\ITA{\input{patterns/patterns_opt_dbg_ITA}}
\PTBR{\input{patterns/patterns_opt_dbg_PTBR}}
\RU{\input{patterns/patterns_opt_dbg_RU}}
\THA{\input{patterns/patterns_opt_dbg_THA}}
\DE{\input{patterns/patterns_opt_dbg_DE}}
\FR{\input{patterns/patterns_opt_dbg_FR}}
\PL{\input{patterns/patterns_opt_dbg_PL}}

\RU{\section{Некоторые базовые понятия}}
\EN{\section{Some basics}}
\DE{\section{Einige Grundlagen}}
\FR{\section{Quelques bases}}
\ES{\section{\ESph{}}}
\ITA{\section{Alcune basi teoriche}}
\PTBR{\section{\PTBRph{}}}
\THA{\section{\THAph{}}}
\PL{\section{\PLph{}}}

% sections:
\EN{\input{patterns/intro_CPU_ISA_EN}}
\ES{\input{patterns/intro_CPU_ISA_ES}}
\ITA{\input{patterns/intro_CPU_ISA_ITA}}
\PTBR{\input{patterns/intro_CPU_ISA_PTBR}}
\RU{\input{patterns/intro_CPU_ISA_RU}}
\DE{\input{patterns/intro_CPU_ISA_DE}}
\FR{\input{patterns/intro_CPU_ISA_FR}}
\PL{\input{patterns/intro_CPU_ISA_PL}}

\EN{\input{patterns/numeral_EN}}
\RU{\input{patterns/numeral_RU}}
\ITA{\input{patterns/numeral_ITA}}
\DE{\input{patterns/numeral_DE}}
\FR{\input{patterns/numeral_FR}}
\PL{\input{patterns/numeral_PL}}

% chapters
\input{patterns/00_empty/main}
\input{patterns/011_ret/main}
\input{patterns/01_helloworld/main}
\input{patterns/015_prolog_epilogue/main}
\input{patterns/02_stack/main}
\input{patterns/03_printf/main}
\input{patterns/04_scanf/main}
\input{patterns/05_passing_arguments/main}
\input{patterns/06_return_results/main}
\input{patterns/061_pointers/main}
\input{patterns/065_GOTO/main}
\input{patterns/07_jcc/main}
\input{patterns/08_switch/main}
\input{patterns/09_loops/main}
\input{patterns/10_strings/main}
\input{patterns/11_arith_optimizations/main}
\input{patterns/12_FPU/main}
\input{patterns/13_arrays/main}
\input{patterns/14_bitfields/main}
\EN{\input{patterns/145_LCG/main_EN}}
\RU{\input{patterns/145_LCG/main_RU}}
\input{patterns/15_structs/main}
\input{patterns/17_unions/main}
\input{patterns/18_pointers_to_functions/main}
\input{patterns/185_64bit_in_32_env/main}

\EN{\input{patterns/19_SIMD/main_EN}}
\RU{\input{patterns/19_SIMD/main_RU}}
\DE{\input{patterns/19_SIMD/main_DE}}

\EN{\input{patterns/20_x64/main_EN}}
\RU{\input{patterns/20_x64/main_RU}}

\EN{\input{patterns/205_floating_SIMD/main_EN}}
\RU{\input{patterns/205_floating_SIMD/main_RU}}
\DE{\input{patterns/205_floating_SIMD/main_DE}}

\EN{\input{patterns/ARM/main_EN}}
\RU{\input{patterns/ARM/main_RU}}
\DE{\input{patterns/ARM/main_DE}}

\input{patterns/MIPS/main}

\ifdefined\SPANISH
\chapter{Patrones de código}
\fi % SPANISH

\ifdefined\GERMAN
\chapter{Code-Muster}
\fi % GERMAN

\ifdefined\ENGLISH
\chapter{Code Patterns}
\fi % ENGLISH

\ifdefined\ITALIAN
\chapter{Forme di codice}
\fi % ITALIAN

\ifdefined\RUSSIAN
\chapter{Образцы кода}
\fi % RUSSIAN

\ifdefined\BRAZILIAN
\chapter{Padrões de códigos}
\fi % BRAZILIAN

\ifdefined\THAI
\chapter{รูปแบบของโค้ด}
\fi % THAI

\ifdefined\FRENCH
\chapter{Modèle de code}
\fi % FRENCH

\ifdefined\POLISH
\chapter{\PLph{}}
\fi % POLISH

% sections
\EN{\input{patterns/patterns_opt_dbg_EN}}
\ES{\input{patterns/patterns_opt_dbg_ES}}
\ITA{\input{patterns/patterns_opt_dbg_ITA}}
\PTBR{\input{patterns/patterns_opt_dbg_PTBR}}
\RU{\input{patterns/patterns_opt_dbg_RU}}
\THA{\input{patterns/patterns_opt_dbg_THA}}
\DE{\input{patterns/patterns_opt_dbg_DE}}
\FR{\input{patterns/patterns_opt_dbg_FR}}
\PL{\input{patterns/patterns_opt_dbg_PL}}

\RU{\section{Некоторые базовые понятия}}
\EN{\section{Some basics}}
\DE{\section{Einige Grundlagen}}
\FR{\section{Quelques bases}}
\ES{\section{\ESph{}}}
\ITA{\section{Alcune basi teoriche}}
\PTBR{\section{\PTBRph{}}}
\THA{\section{\THAph{}}}
\PL{\section{\PLph{}}}

% sections:
\EN{\input{patterns/intro_CPU_ISA_EN}}
\ES{\input{patterns/intro_CPU_ISA_ES}}
\ITA{\input{patterns/intro_CPU_ISA_ITA}}
\PTBR{\input{patterns/intro_CPU_ISA_PTBR}}
\RU{\input{patterns/intro_CPU_ISA_RU}}
\DE{\input{patterns/intro_CPU_ISA_DE}}
\FR{\input{patterns/intro_CPU_ISA_FR}}
\PL{\input{patterns/intro_CPU_ISA_PL}}

\EN{\input{patterns/numeral_EN}}
\RU{\input{patterns/numeral_RU}}
\ITA{\input{patterns/numeral_ITA}}
\DE{\input{patterns/numeral_DE}}
\FR{\input{patterns/numeral_FR}}
\PL{\input{patterns/numeral_PL}}

% chapters
\input{patterns/00_empty/main}
\input{patterns/011_ret/main}
\input{patterns/01_helloworld/main}
\input{patterns/015_prolog_epilogue/main}
\input{patterns/02_stack/main}
\input{patterns/03_printf/main}
\input{patterns/04_scanf/main}
\input{patterns/05_passing_arguments/main}
\input{patterns/06_return_results/main}
\input{patterns/061_pointers/main}
\input{patterns/065_GOTO/main}
\input{patterns/07_jcc/main}
\input{patterns/08_switch/main}
\input{patterns/09_loops/main}
\input{patterns/10_strings/main}
\input{patterns/11_arith_optimizations/main}
\input{patterns/12_FPU/main}
\input{patterns/13_arrays/main}
\input{patterns/14_bitfields/main}
\EN{\input{patterns/145_LCG/main_EN}}
\RU{\input{patterns/145_LCG/main_RU}}
\input{patterns/15_structs/main}
\input{patterns/17_unions/main}
\input{patterns/18_pointers_to_functions/main}
\input{patterns/185_64bit_in_32_env/main}

\EN{\input{patterns/19_SIMD/main_EN}}
\RU{\input{patterns/19_SIMD/main_RU}}
\DE{\input{patterns/19_SIMD/main_DE}}

\EN{\input{patterns/20_x64/main_EN}}
\RU{\input{patterns/20_x64/main_RU}}

\EN{\input{patterns/205_floating_SIMD/main_EN}}
\RU{\input{patterns/205_floating_SIMD/main_RU}}
\DE{\input{patterns/205_floating_SIMD/main_DE}}

\EN{\input{patterns/ARM/main_EN}}
\RU{\input{patterns/ARM/main_RU}}
\DE{\input{patterns/ARM/main_DE}}

\input{patterns/MIPS/main}

\ifdefined\SPANISH
\chapter{Patrones de código}
\fi % SPANISH

\ifdefined\GERMAN
\chapter{Code-Muster}
\fi % GERMAN

\ifdefined\ENGLISH
\chapter{Code Patterns}
\fi % ENGLISH

\ifdefined\ITALIAN
\chapter{Forme di codice}
\fi % ITALIAN

\ifdefined\RUSSIAN
\chapter{Образцы кода}
\fi % RUSSIAN

\ifdefined\BRAZILIAN
\chapter{Padrões de códigos}
\fi % BRAZILIAN

\ifdefined\THAI
\chapter{รูปแบบของโค้ด}
\fi % THAI

\ifdefined\FRENCH
\chapter{Modèle de code}
\fi % FRENCH

\ifdefined\POLISH
\chapter{\PLph{}}
\fi % POLISH

% sections
\EN{\input{patterns/patterns_opt_dbg_EN}}
\ES{\input{patterns/patterns_opt_dbg_ES}}
\ITA{\input{patterns/patterns_opt_dbg_ITA}}
\PTBR{\input{patterns/patterns_opt_dbg_PTBR}}
\RU{\input{patterns/patterns_opt_dbg_RU}}
\THA{\input{patterns/patterns_opt_dbg_THA}}
\DE{\input{patterns/patterns_opt_dbg_DE}}
\FR{\input{patterns/patterns_opt_dbg_FR}}
\PL{\input{patterns/patterns_opt_dbg_PL}}

\RU{\section{Некоторые базовые понятия}}
\EN{\section{Some basics}}
\DE{\section{Einige Grundlagen}}
\FR{\section{Quelques bases}}
\ES{\section{\ESph{}}}
\ITA{\section{Alcune basi teoriche}}
\PTBR{\section{\PTBRph{}}}
\THA{\section{\THAph{}}}
\PL{\section{\PLph{}}}

% sections:
\EN{\input{patterns/intro_CPU_ISA_EN}}
\ES{\input{patterns/intro_CPU_ISA_ES}}
\ITA{\input{patterns/intro_CPU_ISA_ITA}}
\PTBR{\input{patterns/intro_CPU_ISA_PTBR}}
\RU{\input{patterns/intro_CPU_ISA_RU}}
\DE{\input{patterns/intro_CPU_ISA_DE}}
\FR{\input{patterns/intro_CPU_ISA_FR}}
\PL{\input{patterns/intro_CPU_ISA_PL}}

\EN{\input{patterns/numeral_EN}}
\RU{\input{patterns/numeral_RU}}
\ITA{\input{patterns/numeral_ITA}}
\DE{\input{patterns/numeral_DE}}
\FR{\input{patterns/numeral_FR}}
\PL{\input{patterns/numeral_PL}}

% chapters
\input{patterns/00_empty/main}
\input{patterns/011_ret/main}
\input{patterns/01_helloworld/main}
\input{patterns/015_prolog_epilogue/main}
\input{patterns/02_stack/main}
\input{patterns/03_printf/main}
\input{patterns/04_scanf/main}
\input{patterns/05_passing_arguments/main}
\input{patterns/06_return_results/main}
\input{patterns/061_pointers/main}
\input{patterns/065_GOTO/main}
\input{patterns/07_jcc/main}
\input{patterns/08_switch/main}
\input{patterns/09_loops/main}
\input{patterns/10_strings/main}
\input{patterns/11_arith_optimizations/main}
\input{patterns/12_FPU/main}
\input{patterns/13_arrays/main}
\input{patterns/14_bitfields/main}
\EN{\input{patterns/145_LCG/main_EN}}
\RU{\input{patterns/145_LCG/main_RU}}
\input{patterns/15_structs/main}
\input{patterns/17_unions/main}
\input{patterns/18_pointers_to_functions/main}
\input{patterns/185_64bit_in_32_env/main}

\EN{\input{patterns/19_SIMD/main_EN}}
\RU{\input{patterns/19_SIMD/main_RU}}
\DE{\input{patterns/19_SIMD/main_DE}}

\EN{\input{patterns/20_x64/main_EN}}
\RU{\input{patterns/20_x64/main_RU}}

\EN{\input{patterns/205_floating_SIMD/main_EN}}
\RU{\input{patterns/205_floating_SIMD/main_RU}}
\DE{\input{patterns/205_floating_SIMD/main_DE}}

\EN{\input{patterns/ARM/main_EN}}
\RU{\input{patterns/ARM/main_RU}}
\DE{\input{patterns/ARM/main_DE}}

\input{patterns/MIPS/main}


\EN{\section{Returning Values}
\label{ret_val_func}

Another simple function is the one that simply returns a constant value:

\lstinputlisting[caption=\EN{\CCpp Code},style=customc]{patterns/011_ret/1.c}

Let's compile it.

\subsection{x86}

Here's what both the GCC and MSVC compilers produce (with optimization) on the x86 platform:

\lstinputlisting[caption=\Optimizing GCC/MSVC (\assemblyOutput),style=customasmx86]{patterns/011_ret/1.s}

\myindex{x86!\Instructions!RET}
There are just two instructions: the first places the value 123 into the \EAX register,
which is used by convention for storing the return
value, and the second one is \RET, which returns execution to the \gls{caller}.

The caller will take the result from the \EAX register.

\subsection{ARM}

There are a few differences on the ARM platform:

\lstinputlisting[caption=\OptimizingKeilVI (\ARMMode) ASM Output,style=customasmARM]{patterns/011_ret/1_Keil_ARM_O3.s}

ARM uses the register \Reg{0} for returning the results of functions, so 123 is copied into \Reg{0}.

\myindex{ARM!\Instructions!MOV}
\myindex{x86!\Instructions!MOV}
It is worth noting that \MOV is a misleading name for the instruction in both the x86 and ARM \ac{ISA}s.

The data is not in fact \IT{moved}, but \IT{copied}.

\subsection{MIPS}

\label{MIPS_leaf_function_ex1}

The GCC assembly output below lists registers by number:

\lstinputlisting[caption=\Optimizing GCC 4.4.5 (\assemblyOutput),style=customasmMIPS]{patterns/011_ret/MIPS.s}

\dots while \IDA does it by their pseudo names:

\lstinputlisting[caption=\Optimizing GCC 4.4.5 (IDA),style=customasmMIPS]{patterns/011_ret/MIPS_IDA.lst}

The \$2 (or \$V0) register is used to store the function's return value.
\myindex{MIPS!\Pseudoinstructions!LI}
\INS{LI} stands for ``Load Immediate'' and is the MIPS equivalent to \MOV.

\myindex{MIPS!\Instructions!J}
The other instruction is the jump instruction (J or JR) which returns the execution flow to the \gls{caller}.

\myindex{MIPS!Branch delay slot}
You might be wondering why the positions of the load instruction (LI) and the jump instruction (J or JR) are swapped. This is due to a \ac{RISC} feature called ``branch delay slot''.

The reason this happens is a quirk in the architecture of some RISC \ac{ISA}s and isn't important for our
purposes---we must simply keep in mind that in MIPS, the instruction following a jump or branch instruction
is executed \IT{before} the jump/branch instruction itself.

As a consequence, branch instructions always swap places with the instruction executed immediately beforehand.


In practice, functions which merely return 1 (\IT{true}) or 0 (\IT{false}) are very frequent.

The smallest ever of the standard UNIX utilities, \IT{/bin/true} and \IT{/bin/false} return 0 and 1 respectively, as an exit code.
(Zero as an exit code usually means success, non-zero means error.)
}
\RU{\subsubsection{std::string}
\myindex{\Cpp!STL!std::string}
\label{std_string}

\myparagraph{Как устроена структура}

Многие строковые библиотеки \InSqBrackets{\CNotes 2.2} обеспечивают структуру содержащую ссылку 
на буфер собственно со строкой, переменная всегда содержащую длину строки 
(что очень удобно для массы функций \InSqBrackets{\CNotes 2.2.1}) и переменную содержащую текущий размер буфера.

Строка в буфере обыкновенно оканчивается нулем: это для того чтобы указатель на буфер можно было
передавать в функции требующие на вход обычную сишную \ac{ASCIIZ}-строку.

Стандарт \Cpp не описывает, как именно нужно реализовывать std::string,
но, как правило, они реализованы как описано выше, с небольшими дополнениями.

Строки в \Cpp это не класс (как, например, QString в Qt), а темплейт (basic\_string), 
это сделано для того чтобы поддерживать 
строки содержащие разного типа символы: как минимум \Tchar и \IT{wchar\_t}.

Так что, std::string это класс с базовым типом \Tchar.

А std::wstring это класс с базовым типом \IT{wchar\_t}.

\mysubparagraph{MSVC}

В реализации MSVC, вместо ссылки на буфер может содержаться сам буфер (если строка короче 16-и символов).

Это означает, что каждая короткая строка будет занимать в памяти по крайней мере $16 + 4 + 4 = 24$ 
байт для 32-битной среды либо $16 + 8 + 8 = 32$ 
байта в 64-битной, а если строка длиннее 16-и символов, то прибавьте еще длину самой строки.

\lstinputlisting[caption=пример для MSVC,style=customc]{\CURPATH/STL/string/MSVC_RU.cpp}

Собственно, из этого исходника почти всё ясно.

Несколько замечаний:

Если строка короче 16-и символов, 
то отдельный буфер для строки в \glslink{heap}{куче} выделяться не будет.

Это удобно потому что на практике, основная часть строк действительно короткие.
Вероятно, разработчики в Microsoft выбрали размер в 16 символов как разумный баланс.

Теперь очень важный момент в конце функции main(): мы не пользуемся методом c\_str(), тем не менее,
если это скомпилировать и запустить, то обе строки появятся в консоли!

Работает это вот почему.

В первом случае строка короче 16-и символов и в начале объекта std::string (его можно рассматривать
просто как структуру) расположен буфер с этой строкой.
\printf трактует указатель как указатель на массив символов оканчивающийся нулем и поэтому всё работает.

Вывод второй строки (длиннее 16-и символов) даже еще опаснее: это вообще типичная программистская ошибка 
(или опечатка), забыть дописать c\_str().
Это работает потому что в это время в начале структуры расположен указатель на буфер.
Это может надолго остаться незамеченным: до тех пока там не появится строка 
короче 16-и символов, тогда процесс упадет.

\mysubparagraph{GCC}

В реализации GCC в структуре есть еще одна переменная --- reference count.

Интересно, что указатель на экземпляр класса std::string в GCC указывает не на начало самой структуры, 
а на указатель на буфера.
В libstdc++-v3\textbackslash{}include\textbackslash{}bits\textbackslash{}basic\_string.h 
мы можем прочитать что это сделано для удобства отладки:

\begin{lstlisting}
   *  The reason you want _M_data pointing to the character %array and
   *  not the _Rep is so that the debugger can see the string
   *  contents. (Probably we should add a non-inline member to get
   *  the _Rep for the debugger to use, so users can check the actual
   *  string length.)
\end{lstlisting}

\href{http://go.yurichev.com/17085}{исходный код basic\_string.h}

В нашем примере мы учитываем это:

\lstinputlisting[caption=пример для GCC,style=customc]{\CURPATH/STL/string/GCC_RU.cpp}

Нужны еще небольшие хаки чтобы сымитировать типичную ошибку, которую мы уже видели выше, из-за
более ужесточенной проверки типов в GCC, тем не менее, printf() работает и здесь без c\_str().

\myparagraph{Чуть более сложный пример}

\lstinputlisting[style=customc]{\CURPATH/STL/string/3.cpp}

\lstinputlisting[caption=MSVC 2012,style=customasmx86]{\CURPATH/STL/string/3_MSVC_RU.asm}

Собственно, компилятор не конструирует строки статически: да в общем-то и как
это возможно, если буфер с ней нужно хранить в \glslink{heap}{куче}?

Вместо этого в сегменте данных хранятся обычные \ac{ASCIIZ}-строки, а позже, во время выполнения, 
при помощи метода \q{assign}, конструируются строки s1 и s2
.
При помощи \TT{operator+}, создается строка s3.

Обратите внимание на то что вызов метода c\_str() отсутствует,
потому что его код достаточно короткий и компилятор вставил его прямо здесь:
если строка короче 16-и байт, то в регистре EAX остается указатель на буфер,
а если длиннее, то из этого же места достается адрес на буфер расположенный в \glslink{heap}{куче}.

Далее следуют вызовы трех деструкторов, причем, они вызываются только если строка длиннее 16-и байт:
тогда нужно освободить буфера в \glslink{heap}{куче}.
В противном случае, так как все три объекта std::string хранятся в стеке,
они освобождаются автоматически после выхода из функции.

Следовательно, работа с короткими строками более быстрая из-за м\'{е}ньшего обращения к \glslink{heap}{куче}.

Код на GCC даже проще (из-за того, что в GCC, как мы уже видели, не реализована возможность хранить короткую
строку прямо в структуре):

% TODO1 comment each function meaning
\lstinputlisting[caption=GCC 4.8.1,style=customasmx86]{\CURPATH/STL/string/3_GCC_RU.s}

Можно заметить, что в деструкторы передается не указатель на объект,
а указатель на место за 12 байт (или 3 слова) перед ним, то есть, на настоящее начало структуры.

\myparagraph{std::string как глобальная переменная}
\label{sec:std_string_as_global_variable}

Опытные программисты на \Cpp знают, что глобальные переменные \ac{STL}-типов вполне можно объявлять.

Да, действительно:

\lstinputlisting[style=customc]{\CURPATH/STL/string/5.cpp}

Но как и где будет вызываться конструктор \TT{std::string}?

На самом деле, эта переменная будет инициализирована даже перед началом \main.

\lstinputlisting[caption=MSVC 2012: здесь конструируется глобальная переменная{,} а также регистрируется её деструктор,style=customasmx86]{\CURPATH/STL/string/5_MSVC_p2.asm}

\lstinputlisting[caption=MSVC 2012: здесь глобальная переменная используется в \main,style=customasmx86]{\CURPATH/STL/string/5_MSVC_p1.asm}

\lstinputlisting[caption=MSVC 2012: эта функция-деструктор вызывается перед выходом,style=customasmx86]{\CURPATH/STL/string/5_MSVC_p3.asm}

\myindex{\CStandardLibrary!atexit()}
В реальности, из \ac{CRT}, еще до вызова main(), вызывается специальная функция,
в которой перечислены все конструкторы подобных переменных.
Более того: при помощи atexit() регистрируется функция, которая будет вызвана в конце работы программы:
в этой функции компилятор собирает вызовы деструкторов всех подобных глобальных переменных.

GCC работает похожим образом:

\lstinputlisting[caption=GCC 4.8.1,style=customasmx86]{\CURPATH/STL/string/5_GCC.s}

Но он не выделяет отдельной функции в которой будут собраны деструкторы: 
каждый деструктор передается в atexit() по одному.

% TODO а если глобальная STL-переменная в другом модуле? надо проверить.

}
\DE{\subsection{Einfachste XOR-Verschlüsselung überhaupt}

Ich habe einmal eine Software gesehen, bei der alle Debugging-Ausgaben mit XOR mit dem Wert 3
verschlüsselt wurden. Mit anderen Worten, die beiden niedrigsten Bits aller Buchstaben wurden invertiert.

``Hello, world'' wurde zu ``Kfool/\#tlqog'':

\begin{lstlisting}
#!/usr/bin/python

msg="Hello, world!"

print "".join(map(lambda x: chr(ord(x)^3), msg))
\end{lstlisting}

Das ist eine ziemlich interessante Verschlüsselung (oder besser eine Verschleierung),
weil sie zwei wichtige Eigenschaften hat:
1) es ist eine einzige Funktion zum Verschlüsseln und entschlüsseln, sie muss nur wiederholt angewendet werden
2) die entstehenden Buchstaben befinden sich im druckbaren Bereich, also die ganze Zeichenkette kann ohne
Escape-Symbole im Code verwendet werden.

Die zweite Eigenschaft nutzt die Tatsache, dass alle druckbaren Zeichen in Reihen organisiert sind: 0x2x-0x7x,
und wenn die beiden niederwertigsten Bits invertiert werden, wird der Buchstabe um eine oder drei Stellen nach
links oder rechts \IT{verschoben}, aber niemals in eine andere Reihe:

\begin{figure}[H]
\centering
\includegraphics[width=0.7\textwidth]{ascii_clean.png}
\caption{7-Bit \ac{ASCII} Tabelle in Emacs}
\end{figure}

\dots mit dem Zeichen 0x7F als einziger Ausnahme.

Im Folgenden werden also beispielsweise die Zeichen A-Z \IT{verschlüsselt}:

\begin{lstlisting}
#!/usr/bin/python

msg="@ABCDEFGHIJKLMNO"

print "".join(map(lambda x: chr(ord(x)^3), msg))
\end{lstlisting}

Ergebnis:
% FIXME \verb  --  relevant comment for German?
\begin{lstlisting}
CBA@GFEDKJIHONML
\end{lstlisting}

Es sieht so aus als würden die Zeichen ``@'' und ``C'' sowie ``B'' und ``A'' vertauscht werden.

Hier ist noch ein interessantes Beispiel, in dem gezeigt wird, wie die Eigenschaften von XOR
ausgenutzt werden können: Exakt den gleichen Effekt, dass druckbare Zeichen auch druckbar bleiben,
kann man dadurch erzielen, dass irgendeine Kombination der niedrigsten vier Bits invertiert wird.
}

\EN{\section{Returning Values}
\label{ret_val_func}

Another simple function is the one that simply returns a constant value:

\lstinputlisting[caption=\EN{\CCpp Code},style=customc]{patterns/011_ret/1.c}

Let's compile it.

\subsection{x86}

Here's what both the GCC and MSVC compilers produce (with optimization) on the x86 platform:

\lstinputlisting[caption=\Optimizing GCC/MSVC (\assemblyOutput),style=customasmx86]{patterns/011_ret/1.s}

\myindex{x86!\Instructions!RET}
There are just two instructions: the first places the value 123 into the \EAX register,
which is used by convention for storing the return
value, and the second one is \RET, which returns execution to the \gls{caller}.

The caller will take the result from the \EAX register.

\subsection{ARM}

There are a few differences on the ARM platform:

\lstinputlisting[caption=\OptimizingKeilVI (\ARMMode) ASM Output,style=customasmARM]{patterns/011_ret/1_Keil_ARM_O3.s}

ARM uses the register \Reg{0} for returning the results of functions, so 123 is copied into \Reg{0}.

\myindex{ARM!\Instructions!MOV}
\myindex{x86!\Instructions!MOV}
It is worth noting that \MOV is a misleading name for the instruction in both the x86 and ARM \ac{ISA}s.

The data is not in fact \IT{moved}, but \IT{copied}.

\subsection{MIPS}

\label{MIPS_leaf_function_ex1}

The GCC assembly output below lists registers by number:

\lstinputlisting[caption=\Optimizing GCC 4.4.5 (\assemblyOutput),style=customasmMIPS]{patterns/011_ret/MIPS.s}

\dots while \IDA does it by their pseudo names:

\lstinputlisting[caption=\Optimizing GCC 4.4.5 (IDA),style=customasmMIPS]{patterns/011_ret/MIPS_IDA.lst}

The \$2 (or \$V0) register is used to store the function's return value.
\myindex{MIPS!\Pseudoinstructions!LI}
\INS{LI} stands for ``Load Immediate'' and is the MIPS equivalent to \MOV.

\myindex{MIPS!\Instructions!J}
The other instruction is the jump instruction (J or JR) which returns the execution flow to the \gls{caller}.

\myindex{MIPS!Branch delay slot}
You might be wondering why the positions of the load instruction (LI) and the jump instruction (J or JR) are swapped. This is due to a \ac{RISC} feature called ``branch delay slot''.

The reason this happens is a quirk in the architecture of some RISC \ac{ISA}s and isn't important for our
purposes---we must simply keep in mind that in MIPS, the instruction following a jump or branch instruction
is executed \IT{before} the jump/branch instruction itself.

As a consequence, branch instructions always swap places with the instruction executed immediately beforehand.


In practice, functions which merely return 1 (\IT{true}) or 0 (\IT{false}) are very frequent.

The smallest ever of the standard UNIX utilities, \IT{/bin/true} and \IT{/bin/false} return 0 and 1 respectively, as an exit code.
(Zero as an exit code usually means success, non-zero means error.)
}
\RU{\subsubsection{std::string}
\myindex{\Cpp!STL!std::string}
\label{std_string}

\myparagraph{Как устроена структура}

Многие строковые библиотеки \InSqBrackets{\CNotes 2.2} обеспечивают структуру содержащую ссылку 
на буфер собственно со строкой, переменная всегда содержащую длину строки 
(что очень удобно для массы функций \InSqBrackets{\CNotes 2.2.1}) и переменную содержащую текущий размер буфера.

Строка в буфере обыкновенно оканчивается нулем: это для того чтобы указатель на буфер можно было
передавать в функции требующие на вход обычную сишную \ac{ASCIIZ}-строку.

Стандарт \Cpp не описывает, как именно нужно реализовывать std::string,
но, как правило, они реализованы как описано выше, с небольшими дополнениями.

Строки в \Cpp это не класс (как, например, QString в Qt), а темплейт (basic\_string), 
это сделано для того чтобы поддерживать 
строки содержащие разного типа символы: как минимум \Tchar и \IT{wchar\_t}.

Так что, std::string это класс с базовым типом \Tchar.

А std::wstring это класс с базовым типом \IT{wchar\_t}.

\mysubparagraph{MSVC}

В реализации MSVC, вместо ссылки на буфер может содержаться сам буфер (если строка короче 16-и символов).

Это означает, что каждая короткая строка будет занимать в памяти по крайней мере $16 + 4 + 4 = 24$ 
байт для 32-битной среды либо $16 + 8 + 8 = 32$ 
байта в 64-битной, а если строка длиннее 16-и символов, то прибавьте еще длину самой строки.

\lstinputlisting[caption=пример для MSVC,style=customc]{\CURPATH/STL/string/MSVC_RU.cpp}

Собственно, из этого исходника почти всё ясно.

Несколько замечаний:

Если строка короче 16-и символов, 
то отдельный буфер для строки в \glslink{heap}{куче} выделяться не будет.

Это удобно потому что на практике, основная часть строк действительно короткие.
Вероятно, разработчики в Microsoft выбрали размер в 16 символов как разумный баланс.

Теперь очень важный момент в конце функции main(): мы не пользуемся методом c\_str(), тем не менее,
если это скомпилировать и запустить, то обе строки появятся в консоли!

Работает это вот почему.

В первом случае строка короче 16-и символов и в начале объекта std::string (его можно рассматривать
просто как структуру) расположен буфер с этой строкой.
\printf трактует указатель как указатель на массив символов оканчивающийся нулем и поэтому всё работает.

Вывод второй строки (длиннее 16-и символов) даже еще опаснее: это вообще типичная программистская ошибка 
(или опечатка), забыть дописать c\_str().
Это работает потому что в это время в начале структуры расположен указатель на буфер.
Это может надолго остаться незамеченным: до тех пока там не появится строка 
короче 16-и символов, тогда процесс упадет.

\mysubparagraph{GCC}

В реализации GCC в структуре есть еще одна переменная --- reference count.

Интересно, что указатель на экземпляр класса std::string в GCC указывает не на начало самой структуры, 
а на указатель на буфера.
В libstdc++-v3\textbackslash{}include\textbackslash{}bits\textbackslash{}basic\_string.h 
мы можем прочитать что это сделано для удобства отладки:

\begin{lstlisting}
   *  The reason you want _M_data pointing to the character %array and
   *  not the _Rep is so that the debugger can see the string
   *  contents. (Probably we should add a non-inline member to get
   *  the _Rep for the debugger to use, so users can check the actual
   *  string length.)
\end{lstlisting}

\href{http://go.yurichev.com/17085}{исходный код basic\_string.h}

В нашем примере мы учитываем это:

\lstinputlisting[caption=пример для GCC,style=customc]{\CURPATH/STL/string/GCC_RU.cpp}

Нужны еще небольшие хаки чтобы сымитировать типичную ошибку, которую мы уже видели выше, из-за
более ужесточенной проверки типов в GCC, тем не менее, printf() работает и здесь без c\_str().

\myparagraph{Чуть более сложный пример}

\lstinputlisting[style=customc]{\CURPATH/STL/string/3.cpp}

\lstinputlisting[caption=MSVC 2012,style=customasmx86]{\CURPATH/STL/string/3_MSVC_RU.asm}

Собственно, компилятор не конструирует строки статически: да в общем-то и как
это возможно, если буфер с ней нужно хранить в \glslink{heap}{куче}?

Вместо этого в сегменте данных хранятся обычные \ac{ASCIIZ}-строки, а позже, во время выполнения, 
при помощи метода \q{assign}, конструируются строки s1 и s2
.
При помощи \TT{operator+}, создается строка s3.

Обратите внимание на то что вызов метода c\_str() отсутствует,
потому что его код достаточно короткий и компилятор вставил его прямо здесь:
если строка короче 16-и байт, то в регистре EAX остается указатель на буфер,
а если длиннее, то из этого же места достается адрес на буфер расположенный в \glslink{heap}{куче}.

Далее следуют вызовы трех деструкторов, причем, они вызываются только если строка длиннее 16-и байт:
тогда нужно освободить буфера в \glslink{heap}{куче}.
В противном случае, так как все три объекта std::string хранятся в стеке,
они освобождаются автоматически после выхода из функции.

Следовательно, работа с короткими строками более быстрая из-за м\'{е}ньшего обращения к \glslink{heap}{куче}.

Код на GCC даже проще (из-за того, что в GCC, как мы уже видели, не реализована возможность хранить короткую
строку прямо в структуре):

% TODO1 comment each function meaning
\lstinputlisting[caption=GCC 4.8.1,style=customasmx86]{\CURPATH/STL/string/3_GCC_RU.s}

Можно заметить, что в деструкторы передается не указатель на объект,
а указатель на место за 12 байт (или 3 слова) перед ним, то есть, на настоящее начало структуры.

\myparagraph{std::string как глобальная переменная}
\label{sec:std_string_as_global_variable}

Опытные программисты на \Cpp знают, что глобальные переменные \ac{STL}-типов вполне можно объявлять.

Да, действительно:

\lstinputlisting[style=customc]{\CURPATH/STL/string/5.cpp}

Но как и где будет вызываться конструктор \TT{std::string}?

На самом деле, эта переменная будет инициализирована даже перед началом \main.

\lstinputlisting[caption=MSVC 2012: здесь конструируется глобальная переменная{,} а также регистрируется её деструктор,style=customasmx86]{\CURPATH/STL/string/5_MSVC_p2.asm}

\lstinputlisting[caption=MSVC 2012: здесь глобальная переменная используется в \main,style=customasmx86]{\CURPATH/STL/string/5_MSVC_p1.asm}

\lstinputlisting[caption=MSVC 2012: эта функция-деструктор вызывается перед выходом,style=customasmx86]{\CURPATH/STL/string/5_MSVC_p3.asm}

\myindex{\CStandardLibrary!atexit()}
В реальности, из \ac{CRT}, еще до вызова main(), вызывается специальная функция,
в которой перечислены все конструкторы подобных переменных.
Более того: при помощи atexit() регистрируется функция, которая будет вызвана в конце работы программы:
в этой функции компилятор собирает вызовы деструкторов всех подобных глобальных переменных.

GCC работает похожим образом:

\lstinputlisting[caption=GCC 4.8.1,style=customasmx86]{\CURPATH/STL/string/5_GCC.s}

Но он не выделяет отдельной функции в которой будут собраны деструкторы: 
каждый деструктор передается в atexit() по одному.

% TODO а если глобальная STL-переменная в другом модуле? надо проверить.

}

\EN{\section{Returning Values}
\label{ret_val_func}

Another simple function is the one that simply returns a constant value:

\lstinputlisting[caption=\EN{\CCpp Code},style=customc]{patterns/011_ret/1.c}

Let's compile it.

\subsection{x86}

Here's what both the GCC and MSVC compilers produce (with optimization) on the x86 platform:

\lstinputlisting[caption=\Optimizing GCC/MSVC (\assemblyOutput),style=customasmx86]{patterns/011_ret/1.s}

\myindex{x86!\Instructions!RET}
There are just two instructions: the first places the value 123 into the \EAX register,
which is used by convention for storing the return
value, and the second one is \RET, which returns execution to the \gls{caller}.

The caller will take the result from the \EAX register.

\subsection{ARM}

There are a few differences on the ARM platform:

\lstinputlisting[caption=\OptimizingKeilVI (\ARMMode) ASM Output,style=customasmARM]{patterns/011_ret/1_Keil_ARM_O3.s}

ARM uses the register \Reg{0} for returning the results of functions, so 123 is copied into \Reg{0}.

\myindex{ARM!\Instructions!MOV}
\myindex{x86!\Instructions!MOV}
It is worth noting that \MOV is a misleading name for the instruction in both the x86 and ARM \ac{ISA}s.

The data is not in fact \IT{moved}, but \IT{copied}.

\subsection{MIPS}

\label{MIPS_leaf_function_ex1}

The GCC assembly output below lists registers by number:

\lstinputlisting[caption=\Optimizing GCC 4.4.5 (\assemblyOutput),style=customasmMIPS]{patterns/011_ret/MIPS.s}

\dots while \IDA does it by their pseudo names:

\lstinputlisting[caption=\Optimizing GCC 4.4.5 (IDA),style=customasmMIPS]{patterns/011_ret/MIPS_IDA.lst}

The \$2 (or \$V0) register is used to store the function's return value.
\myindex{MIPS!\Pseudoinstructions!LI}
\INS{LI} stands for ``Load Immediate'' and is the MIPS equivalent to \MOV.

\myindex{MIPS!\Instructions!J}
The other instruction is the jump instruction (J or JR) which returns the execution flow to the \gls{caller}.

\myindex{MIPS!Branch delay slot}
You might be wondering why the positions of the load instruction (LI) and the jump instruction (J or JR) are swapped. This is due to a \ac{RISC} feature called ``branch delay slot''.

The reason this happens is a quirk in the architecture of some RISC \ac{ISA}s and isn't important for our
purposes---we must simply keep in mind that in MIPS, the instruction following a jump or branch instruction
is executed \IT{before} the jump/branch instruction itself.

As a consequence, branch instructions always swap places with the instruction executed immediately beforehand.


In practice, functions which merely return 1 (\IT{true}) or 0 (\IT{false}) are very frequent.

The smallest ever of the standard UNIX utilities, \IT{/bin/true} and \IT{/bin/false} return 0 and 1 respectively, as an exit code.
(Zero as an exit code usually means success, non-zero means error.)
}
\RU{\subsubsection{std::string}
\myindex{\Cpp!STL!std::string}
\label{std_string}

\myparagraph{Как устроена структура}

Многие строковые библиотеки \InSqBrackets{\CNotes 2.2} обеспечивают структуру содержащую ссылку 
на буфер собственно со строкой, переменная всегда содержащую длину строки 
(что очень удобно для массы функций \InSqBrackets{\CNotes 2.2.1}) и переменную содержащую текущий размер буфера.

Строка в буфере обыкновенно оканчивается нулем: это для того чтобы указатель на буфер можно было
передавать в функции требующие на вход обычную сишную \ac{ASCIIZ}-строку.

Стандарт \Cpp не описывает, как именно нужно реализовывать std::string,
но, как правило, они реализованы как описано выше, с небольшими дополнениями.

Строки в \Cpp это не класс (как, например, QString в Qt), а темплейт (basic\_string), 
это сделано для того чтобы поддерживать 
строки содержащие разного типа символы: как минимум \Tchar и \IT{wchar\_t}.

Так что, std::string это класс с базовым типом \Tchar.

А std::wstring это класс с базовым типом \IT{wchar\_t}.

\mysubparagraph{MSVC}

В реализации MSVC, вместо ссылки на буфер может содержаться сам буфер (если строка короче 16-и символов).

Это означает, что каждая короткая строка будет занимать в памяти по крайней мере $16 + 4 + 4 = 24$ 
байт для 32-битной среды либо $16 + 8 + 8 = 32$ 
байта в 64-битной, а если строка длиннее 16-и символов, то прибавьте еще длину самой строки.

\lstinputlisting[caption=пример для MSVC,style=customc]{\CURPATH/STL/string/MSVC_RU.cpp}

Собственно, из этого исходника почти всё ясно.

Несколько замечаний:

Если строка короче 16-и символов, 
то отдельный буфер для строки в \glslink{heap}{куче} выделяться не будет.

Это удобно потому что на практике, основная часть строк действительно короткие.
Вероятно, разработчики в Microsoft выбрали размер в 16 символов как разумный баланс.

Теперь очень важный момент в конце функции main(): мы не пользуемся методом c\_str(), тем не менее,
если это скомпилировать и запустить, то обе строки появятся в консоли!

Работает это вот почему.

В первом случае строка короче 16-и символов и в начале объекта std::string (его можно рассматривать
просто как структуру) расположен буфер с этой строкой.
\printf трактует указатель как указатель на массив символов оканчивающийся нулем и поэтому всё работает.

Вывод второй строки (длиннее 16-и символов) даже еще опаснее: это вообще типичная программистская ошибка 
(или опечатка), забыть дописать c\_str().
Это работает потому что в это время в начале структуры расположен указатель на буфер.
Это может надолго остаться незамеченным: до тех пока там не появится строка 
короче 16-и символов, тогда процесс упадет.

\mysubparagraph{GCC}

В реализации GCC в структуре есть еще одна переменная --- reference count.

Интересно, что указатель на экземпляр класса std::string в GCC указывает не на начало самой структуры, 
а на указатель на буфера.
В libstdc++-v3\textbackslash{}include\textbackslash{}bits\textbackslash{}basic\_string.h 
мы можем прочитать что это сделано для удобства отладки:

\begin{lstlisting}
   *  The reason you want _M_data pointing to the character %array and
   *  not the _Rep is so that the debugger can see the string
   *  contents. (Probably we should add a non-inline member to get
   *  the _Rep for the debugger to use, so users can check the actual
   *  string length.)
\end{lstlisting}

\href{http://go.yurichev.com/17085}{исходный код basic\_string.h}

В нашем примере мы учитываем это:

\lstinputlisting[caption=пример для GCC,style=customc]{\CURPATH/STL/string/GCC_RU.cpp}

Нужны еще небольшие хаки чтобы сымитировать типичную ошибку, которую мы уже видели выше, из-за
более ужесточенной проверки типов в GCC, тем не менее, printf() работает и здесь без c\_str().

\myparagraph{Чуть более сложный пример}

\lstinputlisting[style=customc]{\CURPATH/STL/string/3.cpp}

\lstinputlisting[caption=MSVC 2012,style=customasmx86]{\CURPATH/STL/string/3_MSVC_RU.asm}

Собственно, компилятор не конструирует строки статически: да в общем-то и как
это возможно, если буфер с ней нужно хранить в \glslink{heap}{куче}?

Вместо этого в сегменте данных хранятся обычные \ac{ASCIIZ}-строки, а позже, во время выполнения, 
при помощи метода \q{assign}, конструируются строки s1 и s2
.
При помощи \TT{operator+}, создается строка s3.

Обратите внимание на то что вызов метода c\_str() отсутствует,
потому что его код достаточно короткий и компилятор вставил его прямо здесь:
если строка короче 16-и байт, то в регистре EAX остается указатель на буфер,
а если длиннее, то из этого же места достается адрес на буфер расположенный в \glslink{heap}{куче}.

Далее следуют вызовы трех деструкторов, причем, они вызываются только если строка длиннее 16-и байт:
тогда нужно освободить буфера в \glslink{heap}{куче}.
В противном случае, так как все три объекта std::string хранятся в стеке,
они освобождаются автоматически после выхода из функции.

Следовательно, работа с короткими строками более быстрая из-за м\'{е}ньшего обращения к \glslink{heap}{куче}.

Код на GCC даже проще (из-за того, что в GCC, как мы уже видели, не реализована возможность хранить короткую
строку прямо в структуре):

% TODO1 comment each function meaning
\lstinputlisting[caption=GCC 4.8.1,style=customasmx86]{\CURPATH/STL/string/3_GCC_RU.s}

Можно заметить, что в деструкторы передается не указатель на объект,
а указатель на место за 12 байт (или 3 слова) перед ним, то есть, на настоящее начало структуры.

\myparagraph{std::string как глобальная переменная}
\label{sec:std_string_as_global_variable}

Опытные программисты на \Cpp знают, что глобальные переменные \ac{STL}-типов вполне можно объявлять.

Да, действительно:

\lstinputlisting[style=customc]{\CURPATH/STL/string/5.cpp}

Но как и где будет вызываться конструктор \TT{std::string}?

На самом деле, эта переменная будет инициализирована даже перед началом \main.

\lstinputlisting[caption=MSVC 2012: здесь конструируется глобальная переменная{,} а также регистрируется её деструктор,style=customasmx86]{\CURPATH/STL/string/5_MSVC_p2.asm}

\lstinputlisting[caption=MSVC 2012: здесь глобальная переменная используется в \main,style=customasmx86]{\CURPATH/STL/string/5_MSVC_p1.asm}

\lstinputlisting[caption=MSVC 2012: эта функция-деструктор вызывается перед выходом,style=customasmx86]{\CURPATH/STL/string/5_MSVC_p3.asm}

\myindex{\CStandardLibrary!atexit()}
В реальности, из \ac{CRT}, еще до вызова main(), вызывается специальная функция,
в которой перечислены все конструкторы подобных переменных.
Более того: при помощи atexit() регистрируется функция, которая будет вызвана в конце работы программы:
в этой функции компилятор собирает вызовы деструкторов всех подобных глобальных переменных.

GCC работает похожим образом:

\lstinputlisting[caption=GCC 4.8.1,style=customasmx86]{\CURPATH/STL/string/5_GCC.s}

Но он не выделяет отдельной функции в которой будут собраны деструкторы: 
каждый деструктор передается в atexit() по одному.

% TODO а если глобальная STL-переменная в другом модуле? надо проверить.

}
\DE{\subsection{Einfachste XOR-Verschlüsselung überhaupt}

Ich habe einmal eine Software gesehen, bei der alle Debugging-Ausgaben mit XOR mit dem Wert 3
verschlüsselt wurden. Mit anderen Worten, die beiden niedrigsten Bits aller Buchstaben wurden invertiert.

``Hello, world'' wurde zu ``Kfool/\#tlqog'':

\begin{lstlisting}
#!/usr/bin/python

msg="Hello, world!"

print "".join(map(lambda x: chr(ord(x)^3), msg))
\end{lstlisting}

Das ist eine ziemlich interessante Verschlüsselung (oder besser eine Verschleierung),
weil sie zwei wichtige Eigenschaften hat:
1) es ist eine einzige Funktion zum Verschlüsseln und entschlüsseln, sie muss nur wiederholt angewendet werden
2) die entstehenden Buchstaben befinden sich im druckbaren Bereich, also die ganze Zeichenkette kann ohne
Escape-Symbole im Code verwendet werden.

Die zweite Eigenschaft nutzt die Tatsache, dass alle druckbaren Zeichen in Reihen organisiert sind: 0x2x-0x7x,
und wenn die beiden niederwertigsten Bits invertiert werden, wird der Buchstabe um eine oder drei Stellen nach
links oder rechts \IT{verschoben}, aber niemals in eine andere Reihe:

\begin{figure}[H]
\centering
\includegraphics[width=0.7\textwidth]{ascii_clean.png}
\caption{7-Bit \ac{ASCII} Tabelle in Emacs}
\end{figure}

\dots mit dem Zeichen 0x7F als einziger Ausnahme.

Im Folgenden werden also beispielsweise die Zeichen A-Z \IT{verschlüsselt}:

\begin{lstlisting}
#!/usr/bin/python

msg="@ABCDEFGHIJKLMNO"

print "".join(map(lambda x: chr(ord(x)^3), msg))
\end{lstlisting}

Ergebnis:
% FIXME \verb  --  relevant comment for German?
\begin{lstlisting}
CBA@GFEDKJIHONML
\end{lstlisting}

Es sieht so aus als würden die Zeichen ``@'' und ``C'' sowie ``B'' und ``A'' vertauscht werden.

Hier ist noch ein interessantes Beispiel, in dem gezeigt wird, wie die Eigenschaften von XOR
ausgenutzt werden können: Exakt den gleichen Effekt, dass druckbare Zeichen auch druckbar bleiben,
kann man dadurch erzielen, dass irgendeine Kombination der niedrigsten vier Bits invertiert wird.
}

\EN{\section{Returning Values}
\label{ret_val_func}

Another simple function is the one that simply returns a constant value:

\lstinputlisting[caption=\EN{\CCpp Code},style=customc]{patterns/011_ret/1.c}

Let's compile it.

\subsection{x86}

Here's what both the GCC and MSVC compilers produce (with optimization) on the x86 platform:

\lstinputlisting[caption=\Optimizing GCC/MSVC (\assemblyOutput),style=customasmx86]{patterns/011_ret/1.s}

\myindex{x86!\Instructions!RET}
There are just two instructions: the first places the value 123 into the \EAX register,
which is used by convention for storing the return
value, and the second one is \RET, which returns execution to the \gls{caller}.

The caller will take the result from the \EAX register.

\subsection{ARM}

There are a few differences on the ARM platform:

\lstinputlisting[caption=\OptimizingKeilVI (\ARMMode) ASM Output,style=customasmARM]{patterns/011_ret/1_Keil_ARM_O3.s}

ARM uses the register \Reg{0} for returning the results of functions, so 123 is copied into \Reg{0}.

\myindex{ARM!\Instructions!MOV}
\myindex{x86!\Instructions!MOV}
It is worth noting that \MOV is a misleading name for the instruction in both the x86 and ARM \ac{ISA}s.

The data is not in fact \IT{moved}, but \IT{copied}.

\subsection{MIPS}

\label{MIPS_leaf_function_ex1}

The GCC assembly output below lists registers by number:

\lstinputlisting[caption=\Optimizing GCC 4.4.5 (\assemblyOutput),style=customasmMIPS]{patterns/011_ret/MIPS.s}

\dots while \IDA does it by their pseudo names:

\lstinputlisting[caption=\Optimizing GCC 4.4.5 (IDA),style=customasmMIPS]{patterns/011_ret/MIPS_IDA.lst}

The \$2 (or \$V0) register is used to store the function's return value.
\myindex{MIPS!\Pseudoinstructions!LI}
\INS{LI} stands for ``Load Immediate'' and is the MIPS equivalent to \MOV.

\myindex{MIPS!\Instructions!J}
The other instruction is the jump instruction (J or JR) which returns the execution flow to the \gls{caller}.

\myindex{MIPS!Branch delay slot}
You might be wondering why the positions of the load instruction (LI) and the jump instruction (J or JR) are swapped. This is due to a \ac{RISC} feature called ``branch delay slot''.

The reason this happens is a quirk in the architecture of some RISC \ac{ISA}s and isn't important for our
purposes---we must simply keep in mind that in MIPS, the instruction following a jump or branch instruction
is executed \IT{before} the jump/branch instruction itself.

As a consequence, branch instructions always swap places with the instruction executed immediately beforehand.


In practice, functions which merely return 1 (\IT{true}) or 0 (\IT{false}) are very frequent.

The smallest ever of the standard UNIX utilities, \IT{/bin/true} and \IT{/bin/false} return 0 and 1 respectively, as an exit code.
(Zero as an exit code usually means success, non-zero means error.)
}
\RU{\subsubsection{std::string}
\myindex{\Cpp!STL!std::string}
\label{std_string}

\myparagraph{Как устроена структура}

Многие строковые библиотеки \InSqBrackets{\CNotes 2.2} обеспечивают структуру содержащую ссылку 
на буфер собственно со строкой, переменная всегда содержащую длину строки 
(что очень удобно для массы функций \InSqBrackets{\CNotes 2.2.1}) и переменную содержащую текущий размер буфера.

Строка в буфере обыкновенно оканчивается нулем: это для того чтобы указатель на буфер можно было
передавать в функции требующие на вход обычную сишную \ac{ASCIIZ}-строку.

Стандарт \Cpp не описывает, как именно нужно реализовывать std::string,
но, как правило, они реализованы как описано выше, с небольшими дополнениями.

Строки в \Cpp это не класс (как, например, QString в Qt), а темплейт (basic\_string), 
это сделано для того чтобы поддерживать 
строки содержащие разного типа символы: как минимум \Tchar и \IT{wchar\_t}.

Так что, std::string это класс с базовым типом \Tchar.

А std::wstring это класс с базовым типом \IT{wchar\_t}.

\mysubparagraph{MSVC}

В реализации MSVC, вместо ссылки на буфер может содержаться сам буфер (если строка короче 16-и символов).

Это означает, что каждая короткая строка будет занимать в памяти по крайней мере $16 + 4 + 4 = 24$ 
байт для 32-битной среды либо $16 + 8 + 8 = 32$ 
байта в 64-битной, а если строка длиннее 16-и символов, то прибавьте еще длину самой строки.

\lstinputlisting[caption=пример для MSVC,style=customc]{\CURPATH/STL/string/MSVC_RU.cpp}

Собственно, из этого исходника почти всё ясно.

Несколько замечаний:

Если строка короче 16-и символов, 
то отдельный буфер для строки в \glslink{heap}{куче} выделяться не будет.

Это удобно потому что на практике, основная часть строк действительно короткие.
Вероятно, разработчики в Microsoft выбрали размер в 16 символов как разумный баланс.

Теперь очень важный момент в конце функции main(): мы не пользуемся методом c\_str(), тем не менее,
если это скомпилировать и запустить, то обе строки появятся в консоли!

Работает это вот почему.

В первом случае строка короче 16-и символов и в начале объекта std::string (его можно рассматривать
просто как структуру) расположен буфер с этой строкой.
\printf трактует указатель как указатель на массив символов оканчивающийся нулем и поэтому всё работает.

Вывод второй строки (длиннее 16-и символов) даже еще опаснее: это вообще типичная программистская ошибка 
(или опечатка), забыть дописать c\_str().
Это работает потому что в это время в начале структуры расположен указатель на буфер.
Это может надолго остаться незамеченным: до тех пока там не появится строка 
короче 16-и символов, тогда процесс упадет.

\mysubparagraph{GCC}

В реализации GCC в структуре есть еще одна переменная --- reference count.

Интересно, что указатель на экземпляр класса std::string в GCC указывает не на начало самой структуры, 
а на указатель на буфера.
В libstdc++-v3\textbackslash{}include\textbackslash{}bits\textbackslash{}basic\_string.h 
мы можем прочитать что это сделано для удобства отладки:

\begin{lstlisting}
   *  The reason you want _M_data pointing to the character %array and
   *  not the _Rep is so that the debugger can see the string
   *  contents. (Probably we should add a non-inline member to get
   *  the _Rep for the debugger to use, so users can check the actual
   *  string length.)
\end{lstlisting}

\href{http://go.yurichev.com/17085}{исходный код basic\_string.h}

В нашем примере мы учитываем это:

\lstinputlisting[caption=пример для GCC,style=customc]{\CURPATH/STL/string/GCC_RU.cpp}

Нужны еще небольшие хаки чтобы сымитировать типичную ошибку, которую мы уже видели выше, из-за
более ужесточенной проверки типов в GCC, тем не менее, printf() работает и здесь без c\_str().

\myparagraph{Чуть более сложный пример}

\lstinputlisting[style=customc]{\CURPATH/STL/string/3.cpp}

\lstinputlisting[caption=MSVC 2012,style=customasmx86]{\CURPATH/STL/string/3_MSVC_RU.asm}

Собственно, компилятор не конструирует строки статически: да в общем-то и как
это возможно, если буфер с ней нужно хранить в \glslink{heap}{куче}?

Вместо этого в сегменте данных хранятся обычные \ac{ASCIIZ}-строки, а позже, во время выполнения, 
при помощи метода \q{assign}, конструируются строки s1 и s2
.
При помощи \TT{operator+}, создается строка s3.

Обратите внимание на то что вызов метода c\_str() отсутствует,
потому что его код достаточно короткий и компилятор вставил его прямо здесь:
если строка короче 16-и байт, то в регистре EAX остается указатель на буфер,
а если длиннее, то из этого же места достается адрес на буфер расположенный в \glslink{heap}{куче}.

Далее следуют вызовы трех деструкторов, причем, они вызываются только если строка длиннее 16-и байт:
тогда нужно освободить буфера в \glslink{heap}{куче}.
В противном случае, так как все три объекта std::string хранятся в стеке,
они освобождаются автоматически после выхода из функции.

Следовательно, работа с короткими строками более быстрая из-за м\'{е}ньшего обращения к \glslink{heap}{куче}.

Код на GCC даже проще (из-за того, что в GCC, как мы уже видели, не реализована возможность хранить короткую
строку прямо в структуре):

% TODO1 comment each function meaning
\lstinputlisting[caption=GCC 4.8.1,style=customasmx86]{\CURPATH/STL/string/3_GCC_RU.s}

Можно заметить, что в деструкторы передается не указатель на объект,
а указатель на место за 12 байт (или 3 слова) перед ним, то есть, на настоящее начало структуры.

\myparagraph{std::string как глобальная переменная}
\label{sec:std_string_as_global_variable}

Опытные программисты на \Cpp знают, что глобальные переменные \ac{STL}-типов вполне можно объявлять.

Да, действительно:

\lstinputlisting[style=customc]{\CURPATH/STL/string/5.cpp}

Но как и где будет вызываться конструктор \TT{std::string}?

На самом деле, эта переменная будет инициализирована даже перед началом \main.

\lstinputlisting[caption=MSVC 2012: здесь конструируется глобальная переменная{,} а также регистрируется её деструктор,style=customasmx86]{\CURPATH/STL/string/5_MSVC_p2.asm}

\lstinputlisting[caption=MSVC 2012: здесь глобальная переменная используется в \main,style=customasmx86]{\CURPATH/STL/string/5_MSVC_p1.asm}

\lstinputlisting[caption=MSVC 2012: эта функция-деструктор вызывается перед выходом,style=customasmx86]{\CURPATH/STL/string/5_MSVC_p3.asm}

\myindex{\CStandardLibrary!atexit()}
В реальности, из \ac{CRT}, еще до вызова main(), вызывается специальная функция,
в которой перечислены все конструкторы подобных переменных.
Более того: при помощи atexit() регистрируется функция, которая будет вызвана в конце работы программы:
в этой функции компилятор собирает вызовы деструкторов всех подобных глобальных переменных.

GCC работает похожим образом:

\lstinputlisting[caption=GCC 4.8.1,style=customasmx86]{\CURPATH/STL/string/5_GCC.s}

Но он не выделяет отдельной функции в которой будут собраны деструкторы: 
каждый деструктор передается в atexit() по одному.

% TODO а если глобальная STL-переменная в другом модуле? надо проверить.

}
\DE{\subsection{Einfachste XOR-Verschlüsselung überhaupt}

Ich habe einmal eine Software gesehen, bei der alle Debugging-Ausgaben mit XOR mit dem Wert 3
verschlüsselt wurden. Mit anderen Worten, die beiden niedrigsten Bits aller Buchstaben wurden invertiert.

``Hello, world'' wurde zu ``Kfool/\#tlqog'':

\begin{lstlisting}
#!/usr/bin/python

msg="Hello, world!"

print "".join(map(lambda x: chr(ord(x)^3), msg))
\end{lstlisting}

Das ist eine ziemlich interessante Verschlüsselung (oder besser eine Verschleierung),
weil sie zwei wichtige Eigenschaften hat:
1) es ist eine einzige Funktion zum Verschlüsseln und entschlüsseln, sie muss nur wiederholt angewendet werden
2) die entstehenden Buchstaben befinden sich im druckbaren Bereich, also die ganze Zeichenkette kann ohne
Escape-Symbole im Code verwendet werden.

Die zweite Eigenschaft nutzt die Tatsache, dass alle druckbaren Zeichen in Reihen organisiert sind: 0x2x-0x7x,
und wenn die beiden niederwertigsten Bits invertiert werden, wird der Buchstabe um eine oder drei Stellen nach
links oder rechts \IT{verschoben}, aber niemals in eine andere Reihe:

\begin{figure}[H]
\centering
\includegraphics[width=0.7\textwidth]{ascii_clean.png}
\caption{7-Bit \ac{ASCII} Tabelle in Emacs}
\end{figure}

\dots mit dem Zeichen 0x7F als einziger Ausnahme.

Im Folgenden werden also beispielsweise die Zeichen A-Z \IT{verschlüsselt}:

\begin{lstlisting}
#!/usr/bin/python

msg="@ABCDEFGHIJKLMNO"

print "".join(map(lambda x: chr(ord(x)^3), msg))
\end{lstlisting}

Ergebnis:
% FIXME \verb  --  relevant comment for German?
\begin{lstlisting}
CBA@GFEDKJIHONML
\end{lstlisting}

Es sieht so aus als würden die Zeichen ``@'' und ``C'' sowie ``B'' und ``A'' vertauscht werden.

Hier ist noch ein interessantes Beispiel, in dem gezeigt wird, wie die Eigenschaften von XOR
ausgenutzt werden können: Exakt den gleichen Effekt, dass druckbare Zeichen auch druckbar bleiben,
kann man dadurch erzielen, dass irgendeine Kombination der niedrigsten vier Bits invertiert wird.
}

\ifdefined\SPANISH
\chapter{Patrones de código}
\fi % SPANISH

\ifdefined\GERMAN
\chapter{Code-Muster}
\fi % GERMAN

\ifdefined\ENGLISH
\chapter{Code Patterns}
\fi % ENGLISH

\ifdefined\ITALIAN
\chapter{Forme di codice}
\fi % ITALIAN

\ifdefined\RUSSIAN
\chapter{Образцы кода}
\fi % RUSSIAN

\ifdefined\BRAZILIAN
\chapter{Padrões de códigos}
\fi % BRAZILIAN

\ifdefined\THAI
\chapter{รูปแบบของโค้ด}
\fi % THAI

\ifdefined\FRENCH
\chapter{Modèle de code}
\fi % FRENCH

\ifdefined\POLISH
\chapter{\PLph{}}
\fi % POLISH

% sections
\EN{\input{patterns/patterns_opt_dbg_EN}}
\ES{\input{patterns/patterns_opt_dbg_ES}}
\ITA{\input{patterns/patterns_opt_dbg_ITA}}
\PTBR{\input{patterns/patterns_opt_dbg_PTBR}}
\RU{\input{patterns/patterns_opt_dbg_RU}}
\THA{\input{patterns/patterns_opt_dbg_THA}}
\DE{\input{patterns/patterns_opt_dbg_DE}}
\FR{\input{patterns/patterns_opt_dbg_FR}}
\PL{\input{patterns/patterns_opt_dbg_PL}}

\RU{\section{Некоторые базовые понятия}}
\EN{\section{Some basics}}
\DE{\section{Einige Grundlagen}}
\FR{\section{Quelques bases}}
\ES{\section{\ESph{}}}
\ITA{\section{Alcune basi teoriche}}
\PTBR{\section{\PTBRph{}}}
\THA{\section{\THAph{}}}
\PL{\section{\PLph{}}}

% sections:
\EN{\input{patterns/intro_CPU_ISA_EN}}
\ES{\input{patterns/intro_CPU_ISA_ES}}
\ITA{\input{patterns/intro_CPU_ISA_ITA}}
\PTBR{\input{patterns/intro_CPU_ISA_PTBR}}
\RU{\input{patterns/intro_CPU_ISA_RU}}
\DE{\input{patterns/intro_CPU_ISA_DE}}
\FR{\input{patterns/intro_CPU_ISA_FR}}
\PL{\input{patterns/intro_CPU_ISA_PL}}

\EN{\input{patterns/numeral_EN}}
\RU{\input{patterns/numeral_RU}}
\ITA{\input{patterns/numeral_ITA}}
\DE{\input{patterns/numeral_DE}}
\FR{\input{patterns/numeral_FR}}
\PL{\input{patterns/numeral_PL}}

% chapters
\input{patterns/00_empty/main}
\input{patterns/011_ret/main}
\input{patterns/01_helloworld/main}
\input{patterns/015_prolog_epilogue/main}
\input{patterns/02_stack/main}
\input{patterns/03_printf/main}
\input{patterns/04_scanf/main}
\input{patterns/05_passing_arguments/main}
\input{patterns/06_return_results/main}
\input{patterns/061_pointers/main}
\input{patterns/065_GOTO/main}
\input{patterns/07_jcc/main}
\input{patterns/08_switch/main}
\input{patterns/09_loops/main}
\input{patterns/10_strings/main}
\input{patterns/11_arith_optimizations/main}
\input{patterns/12_FPU/main}
\input{patterns/13_arrays/main}
\input{patterns/14_bitfields/main}
\EN{\input{patterns/145_LCG/main_EN}}
\RU{\input{patterns/145_LCG/main_RU}}
\input{patterns/15_structs/main}
\input{patterns/17_unions/main}
\input{patterns/18_pointers_to_functions/main}
\input{patterns/185_64bit_in_32_env/main}

\EN{\input{patterns/19_SIMD/main_EN}}
\RU{\input{patterns/19_SIMD/main_RU}}
\DE{\input{patterns/19_SIMD/main_DE}}

\EN{\input{patterns/20_x64/main_EN}}
\RU{\input{patterns/20_x64/main_RU}}

\EN{\input{patterns/205_floating_SIMD/main_EN}}
\RU{\input{patterns/205_floating_SIMD/main_RU}}
\DE{\input{patterns/205_floating_SIMD/main_DE}}

\EN{\input{patterns/ARM/main_EN}}
\RU{\input{patterns/ARM/main_RU}}
\DE{\input{patterns/ARM/main_DE}}

\input{patterns/MIPS/main}



\EN{\section{Returning Values}
\label{ret_val_func}

Another simple function is the one that simply returns a constant value:

\lstinputlisting[caption=\EN{\CCpp Code},style=customc]{patterns/011_ret/1.c}

Let's compile it.

\subsection{x86}

Here's what both the GCC and MSVC compilers produce (with optimization) on the x86 platform:

\lstinputlisting[caption=\Optimizing GCC/MSVC (\assemblyOutput),style=customasmx86]{patterns/011_ret/1.s}

\myindex{x86!\Instructions!RET}
There are just two instructions: the first places the value 123 into the \EAX register,
which is used by convention for storing the return
value, and the second one is \RET, which returns execution to the \gls{caller}.

The caller will take the result from the \EAX register.

\subsection{ARM}

There are a few differences on the ARM platform:

\lstinputlisting[caption=\OptimizingKeilVI (\ARMMode) ASM Output,style=customasmARM]{patterns/011_ret/1_Keil_ARM_O3.s}

ARM uses the register \Reg{0} for returning the results of functions, so 123 is copied into \Reg{0}.

\myindex{ARM!\Instructions!MOV}
\myindex{x86!\Instructions!MOV}
It is worth noting that \MOV is a misleading name for the instruction in both the x86 and ARM \ac{ISA}s.

The data is not in fact \IT{moved}, but \IT{copied}.

\subsection{MIPS}

\label{MIPS_leaf_function_ex1}

The GCC assembly output below lists registers by number:

\lstinputlisting[caption=\Optimizing GCC 4.4.5 (\assemblyOutput),style=customasmMIPS]{patterns/011_ret/MIPS.s}

\dots while \IDA does it by their pseudo names:

\lstinputlisting[caption=\Optimizing GCC 4.4.5 (IDA),style=customasmMIPS]{patterns/011_ret/MIPS_IDA.lst}

The \$2 (or \$V0) register is used to store the function's return value.
\myindex{MIPS!\Pseudoinstructions!LI}
\INS{LI} stands for ``Load Immediate'' and is the MIPS equivalent to \MOV.

\myindex{MIPS!\Instructions!J}
The other instruction is the jump instruction (J or JR) which returns the execution flow to the \gls{caller}.

\myindex{MIPS!Branch delay slot}
You might be wondering why the positions of the load instruction (LI) and the jump instruction (J or JR) are swapped. This is due to a \ac{RISC} feature called ``branch delay slot''.

The reason this happens is a quirk in the architecture of some RISC \ac{ISA}s and isn't important for our
purposes---we must simply keep in mind that in MIPS, the instruction following a jump or branch instruction
is executed \IT{before} the jump/branch instruction itself.

As a consequence, branch instructions always swap places with the instruction executed immediately beforehand.


In practice, functions which merely return 1 (\IT{true}) or 0 (\IT{false}) are very frequent.

The smallest ever of the standard UNIX utilities, \IT{/bin/true} and \IT{/bin/false} return 0 and 1 respectively, as an exit code.
(Zero as an exit code usually means success, non-zero means error.)
}
\RU{\subsubsection{std::string}
\myindex{\Cpp!STL!std::string}
\label{std_string}

\myparagraph{Как устроена структура}

Многие строковые библиотеки \InSqBrackets{\CNotes 2.2} обеспечивают структуру содержащую ссылку 
на буфер собственно со строкой, переменная всегда содержащую длину строки 
(что очень удобно для массы функций \InSqBrackets{\CNotes 2.2.1}) и переменную содержащую текущий размер буфера.

Строка в буфере обыкновенно оканчивается нулем: это для того чтобы указатель на буфер можно было
передавать в функции требующие на вход обычную сишную \ac{ASCIIZ}-строку.

Стандарт \Cpp не описывает, как именно нужно реализовывать std::string,
но, как правило, они реализованы как описано выше, с небольшими дополнениями.

Строки в \Cpp это не класс (как, например, QString в Qt), а темплейт (basic\_string), 
это сделано для того чтобы поддерживать 
строки содержащие разного типа символы: как минимум \Tchar и \IT{wchar\_t}.

Так что, std::string это класс с базовым типом \Tchar.

А std::wstring это класс с базовым типом \IT{wchar\_t}.

\mysubparagraph{MSVC}

В реализации MSVC, вместо ссылки на буфер может содержаться сам буфер (если строка короче 16-и символов).

Это означает, что каждая короткая строка будет занимать в памяти по крайней мере $16 + 4 + 4 = 24$ 
байт для 32-битной среды либо $16 + 8 + 8 = 32$ 
байта в 64-битной, а если строка длиннее 16-и символов, то прибавьте еще длину самой строки.

\lstinputlisting[caption=пример для MSVC,style=customc]{\CURPATH/STL/string/MSVC_RU.cpp}

Собственно, из этого исходника почти всё ясно.

Несколько замечаний:

Если строка короче 16-и символов, 
то отдельный буфер для строки в \glslink{heap}{куче} выделяться не будет.

Это удобно потому что на практике, основная часть строк действительно короткие.
Вероятно, разработчики в Microsoft выбрали размер в 16 символов как разумный баланс.

Теперь очень важный момент в конце функции main(): мы не пользуемся методом c\_str(), тем не менее,
если это скомпилировать и запустить, то обе строки появятся в консоли!

Работает это вот почему.

В первом случае строка короче 16-и символов и в начале объекта std::string (его можно рассматривать
просто как структуру) расположен буфер с этой строкой.
\printf трактует указатель как указатель на массив символов оканчивающийся нулем и поэтому всё работает.

Вывод второй строки (длиннее 16-и символов) даже еще опаснее: это вообще типичная программистская ошибка 
(или опечатка), забыть дописать c\_str().
Это работает потому что в это время в начале структуры расположен указатель на буфер.
Это может надолго остаться незамеченным: до тех пока там не появится строка 
короче 16-и символов, тогда процесс упадет.

\mysubparagraph{GCC}

В реализации GCC в структуре есть еще одна переменная --- reference count.

Интересно, что указатель на экземпляр класса std::string в GCC указывает не на начало самой структуры, 
а на указатель на буфера.
В libstdc++-v3\textbackslash{}include\textbackslash{}bits\textbackslash{}basic\_string.h 
мы можем прочитать что это сделано для удобства отладки:

\begin{lstlisting}
   *  The reason you want _M_data pointing to the character %array and
   *  not the _Rep is so that the debugger can see the string
   *  contents. (Probably we should add a non-inline member to get
   *  the _Rep for the debugger to use, so users can check the actual
   *  string length.)
\end{lstlisting}

\href{http://go.yurichev.com/17085}{исходный код basic\_string.h}

В нашем примере мы учитываем это:

\lstinputlisting[caption=пример для GCC,style=customc]{\CURPATH/STL/string/GCC_RU.cpp}

Нужны еще небольшие хаки чтобы сымитировать типичную ошибку, которую мы уже видели выше, из-за
более ужесточенной проверки типов в GCC, тем не менее, printf() работает и здесь без c\_str().

\myparagraph{Чуть более сложный пример}

\lstinputlisting[style=customc]{\CURPATH/STL/string/3.cpp}

\lstinputlisting[caption=MSVC 2012,style=customasmx86]{\CURPATH/STL/string/3_MSVC_RU.asm}

Собственно, компилятор не конструирует строки статически: да в общем-то и как
это возможно, если буфер с ней нужно хранить в \glslink{heap}{куче}?

Вместо этого в сегменте данных хранятся обычные \ac{ASCIIZ}-строки, а позже, во время выполнения, 
при помощи метода \q{assign}, конструируются строки s1 и s2
.
При помощи \TT{operator+}, создается строка s3.

Обратите внимание на то что вызов метода c\_str() отсутствует,
потому что его код достаточно короткий и компилятор вставил его прямо здесь:
если строка короче 16-и байт, то в регистре EAX остается указатель на буфер,
а если длиннее, то из этого же места достается адрес на буфер расположенный в \glslink{heap}{куче}.

Далее следуют вызовы трех деструкторов, причем, они вызываются только если строка длиннее 16-и байт:
тогда нужно освободить буфера в \glslink{heap}{куче}.
В противном случае, так как все три объекта std::string хранятся в стеке,
они освобождаются автоматически после выхода из функции.

Следовательно, работа с короткими строками более быстрая из-за м\'{е}ньшего обращения к \glslink{heap}{куче}.

Код на GCC даже проще (из-за того, что в GCC, как мы уже видели, не реализована возможность хранить короткую
строку прямо в структуре):

% TODO1 comment each function meaning
\lstinputlisting[caption=GCC 4.8.1,style=customasmx86]{\CURPATH/STL/string/3_GCC_RU.s}

Можно заметить, что в деструкторы передается не указатель на объект,
а указатель на место за 12 байт (или 3 слова) перед ним, то есть, на настоящее начало структуры.

\myparagraph{std::string как глобальная переменная}
\label{sec:std_string_as_global_variable}

Опытные программисты на \Cpp знают, что глобальные переменные \ac{STL}-типов вполне можно объявлять.

Да, действительно:

\lstinputlisting[style=customc]{\CURPATH/STL/string/5.cpp}

Но как и где будет вызываться конструктор \TT{std::string}?

На самом деле, эта переменная будет инициализирована даже перед началом \main.

\lstinputlisting[caption=MSVC 2012: здесь конструируется глобальная переменная{,} а также регистрируется её деструктор,style=customasmx86]{\CURPATH/STL/string/5_MSVC_p2.asm}

\lstinputlisting[caption=MSVC 2012: здесь глобальная переменная используется в \main,style=customasmx86]{\CURPATH/STL/string/5_MSVC_p1.asm}

\lstinputlisting[caption=MSVC 2012: эта функция-деструктор вызывается перед выходом,style=customasmx86]{\CURPATH/STL/string/5_MSVC_p3.asm}

\myindex{\CStandardLibrary!atexit()}
В реальности, из \ac{CRT}, еще до вызова main(), вызывается специальная функция,
в которой перечислены все конструкторы подобных переменных.
Более того: при помощи atexit() регистрируется функция, которая будет вызвана в конце работы программы:
в этой функции компилятор собирает вызовы деструкторов всех подобных глобальных переменных.

GCC работает похожим образом:

\lstinputlisting[caption=GCC 4.8.1,style=customasmx86]{\CURPATH/STL/string/5_GCC.s}

Но он не выделяет отдельной функции в которой будут собраны деструкторы: 
каждый деструктор передается в atexit() по одному.

% TODO а если глобальная STL-переменная в другом модуле? надо проверить.

}
\DE{\subsection{Einfachste XOR-Verschlüsselung überhaupt}

Ich habe einmal eine Software gesehen, bei der alle Debugging-Ausgaben mit XOR mit dem Wert 3
verschlüsselt wurden. Mit anderen Worten, die beiden niedrigsten Bits aller Buchstaben wurden invertiert.

``Hello, world'' wurde zu ``Kfool/\#tlqog'':

\begin{lstlisting}
#!/usr/bin/python

msg="Hello, world!"

print "".join(map(lambda x: chr(ord(x)^3), msg))
\end{lstlisting}

Das ist eine ziemlich interessante Verschlüsselung (oder besser eine Verschleierung),
weil sie zwei wichtige Eigenschaften hat:
1) es ist eine einzige Funktion zum Verschlüsseln und entschlüsseln, sie muss nur wiederholt angewendet werden
2) die entstehenden Buchstaben befinden sich im druckbaren Bereich, also die ganze Zeichenkette kann ohne
Escape-Symbole im Code verwendet werden.

Die zweite Eigenschaft nutzt die Tatsache, dass alle druckbaren Zeichen in Reihen organisiert sind: 0x2x-0x7x,
und wenn die beiden niederwertigsten Bits invertiert werden, wird der Buchstabe um eine oder drei Stellen nach
links oder rechts \IT{verschoben}, aber niemals in eine andere Reihe:

\begin{figure}[H]
\centering
\includegraphics[width=0.7\textwidth]{ascii_clean.png}
\caption{7-Bit \ac{ASCII} Tabelle in Emacs}
\end{figure}

\dots mit dem Zeichen 0x7F als einziger Ausnahme.

Im Folgenden werden also beispielsweise die Zeichen A-Z \IT{verschlüsselt}:

\begin{lstlisting}
#!/usr/bin/python

msg="@ABCDEFGHIJKLMNO"

print "".join(map(lambda x: chr(ord(x)^3), msg))
\end{lstlisting}

Ergebnis:
% FIXME \verb  --  relevant comment for German?
\begin{lstlisting}
CBA@GFEDKJIHONML
\end{lstlisting}

Es sieht so aus als würden die Zeichen ``@'' und ``C'' sowie ``B'' und ``A'' vertauscht werden.

Hier ist noch ein interessantes Beispiel, in dem gezeigt wird, wie die Eigenschaften von XOR
ausgenutzt werden können: Exakt den gleichen Effekt, dass druckbare Zeichen auch druckbar bleiben,
kann man dadurch erzielen, dass irgendeine Kombination der niedrigsten vier Bits invertiert wird.
}

\EN{\section{Returning Values}
\label{ret_val_func}

Another simple function is the one that simply returns a constant value:

\lstinputlisting[caption=\EN{\CCpp Code},style=customc]{patterns/011_ret/1.c}

Let's compile it.

\subsection{x86}

Here's what both the GCC and MSVC compilers produce (with optimization) on the x86 platform:

\lstinputlisting[caption=\Optimizing GCC/MSVC (\assemblyOutput),style=customasmx86]{patterns/011_ret/1.s}

\myindex{x86!\Instructions!RET}
There are just two instructions: the first places the value 123 into the \EAX register,
which is used by convention for storing the return
value, and the second one is \RET, which returns execution to the \gls{caller}.

The caller will take the result from the \EAX register.

\subsection{ARM}

There are a few differences on the ARM platform:

\lstinputlisting[caption=\OptimizingKeilVI (\ARMMode) ASM Output,style=customasmARM]{patterns/011_ret/1_Keil_ARM_O3.s}

ARM uses the register \Reg{0} for returning the results of functions, so 123 is copied into \Reg{0}.

\myindex{ARM!\Instructions!MOV}
\myindex{x86!\Instructions!MOV}
It is worth noting that \MOV is a misleading name for the instruction in both the x86 and ARM \ac{ISA}s.

The data is not in fact \IT{moved}, but \IT{copied}.

\subsection{MIPS}

\label{MIPS_leaf_function_ex1}

The GCC assembly output below lists registers by number:

\lstinputlisting[caption=\Optimizing GCC 4.4.5 (\assemblyOutput),style=customasmMIPS]{patterns/011_ret/MIPS.s}

\dots while \IDA does it by their pseudo names:

\lstinputlisting[caption=\Optimizing GCC 4.4.5 (IDA),style=customasmMIPS]{patterns/011_ret/MIPS_IDA.lst}

The \$2 (or \$V0) register is used to store the function's return value.
\myindex{MIPS!\Pseudoinstructions!LI}
\INS{LI} stands for ``Load Immediate'' and is the MIPS equivalent to \MOV.

\myindex{MIPS!\Instructions!J}
The other instruction is the jump instruction (J or JR) which returns the execution flow to the \gls{caller}.

\myindex{MIPS!Branch delay slot}
You might be wondering why the positions of the load instruction (LI) and the jump instruction (J or JR) are swapped. This is due to a \ac{RISC} feature called ``branch delay slot''.

The reason this happens is a quirk in the architecture of some RISC \ac{ISA}s and isn't important for our
purposes---we must simply keep in mind that in MIPS, the instruction following a jump or branch instruction
is executed \IT{before} the jump/branch instruction itself.

As a consequence, branch instructions always swap places with the instruction executed immediately beforehand.


In practice, functions which merely return 1 (\IT{true}) or 0 (\IT{false}) are very frequent.

The smallest ever of the standard UNIX utilities, \IT{/bin/true} and \IT{/bin/false} return 0 and 1 respectively, as an exit code.
(Zero as an exit code usually means success, non-zero means error.)
}
\RU{\subsubsection{std::string}
\myindex{\Cpp!STL!std::string}
\label{std_string}

\myparagraph{Как устроена структура}

Многие строковые библиотеки \InSqBrackets{\CNotes 2.2} обеспечивают структуру содержащую ссылку 
на буфер собственно со строкой, переменная всегда содержащую длину строки 
(что очень удобно для массы функций \InSqBrackets{\CNotes 2.2.1}) и переменную содержащую текущий размер буфера.

Строка в буфере обыкновенно оканчивается нулем: это для того чтобы указатель на буфер можно было
передавать в функции требующие на вход обычную сишную \ac{ASCIIZ}-строку.

Стандарт \Cpp не описывает, как именно нужно реализовывать std::string,
но, как правило, они реализованы как описано выше, с небольшими дополнениями.

Строки в \Cpp это не класс (как, например, QString в Qt), а темплейт (basic\_string), 
это сделано для того чтобы поддерживать 
строки содержащие разного типа символы: как минимум \Tchar и \IT{wchar\_t}.

Так что, std::string это класс с базовым типом \Tchar.

А std::wstring это класс с базовым типом \IT{wchar\_t}.

\mysubparagraph{MSVC}

В реализации MSVC, вместо ссылки на буфер может содержаться сам буфер (если строка короче 16-и символов).

Это означает, что каждая короткая строка будет занимать в памяти по крайней мере $16 + 4 + 4 = 24$ 
байт для 32-битной среды либо $16 + 8 + 8 = 32$ 
байта в 64-битной, а если строка длиннее 16-и символов, то прибавьте еще длину самой строки.

\lstinputlisting[caption=пример для MSVC,style=customc]{\CURPATH/STL/string/MSVC_RU.cpp}

Собственно, из этого исходника почти всё ясно.

Несколько замечаний:

Если строка короче 16-и символов, 
то отдельный буфер для строки в \glslink{heap}{куче} выделяться не будет.

Это удобно потому что на практике, основная часть строк действительно короткие.
Вероятно, разработчики в Microsoft выбрали размер в 16 символов как разумный баланс.

Теперь очень важный момент в конце функции main(): мы не пользуемся методом c\_str(), тем не менее,
если это скомпилировать и запустить, то обе строки появятся в консоли!

Работает это вот почему.

В первом случае строка короче 16-и символов и в начале объекта std::string (его можно рассматривать
просто как структуру) расположен буфер с этой строкой.
\printf трактует указатель как указатель на массив символов оканчивающийся нулем и поэтому всё работает.

Вывод второй строки (длиннее 16-и символов) даже еще опаснее: это вообще типичная программистская ошибка 
(или опечатка), забыть дописать c\_str().
Это работает потому что в это время в начале структуры расположен указатель на буфер.
Это может надолго остаться незамеченным: до тех пока там не появится строка 
короче 16-и символов, тогда процесс упадет.

\mysubparagraph{GCC}

В реализации GCC в структуре есть еще одна переменная --- reference count.

Интересно, что указатель на экземпляр класса std::string в GCC указывает не на начало самой структуры, 
а на указатель на буфера.
В libstdc++-v3\textbackslash{}include\textbackslash{}bits\textbackslash{}basic\_string.h 
мы можем прочитать что это сделано для удобства отладки:

\begin{lstlisting}
   *  The reason you want _M_data pointing to the character %array and
   *  not the _Rep is so that the debugger can see the string
   *  contents. (Probably we should add a non-inline member to get
   *  the _Rep for the debugger to use, so users can check the actual
   *  string length.)
\end{lstlisting}

\href{http://go.yurichev.com/17085}{исходный код basic\_string.h}

В нашем примере мы учитываем это:

\lstinputlisting[caption=пример для GCC,style=customc]{\CURPATH/STL/string/GCC_RU.cpp}

Нужны еще небольшие хаки чтобы сымитировать типичную ошибку, которую мы уже видели выше, из-за
более ужесточенной проверки типов в GCC, тем не менее, printf() работает и здесь без c\_str().

\myparagraph{Чуть более сложный пример}

\lstinputlisting[style=customc]{\CURPATH/STL/string/3.cpp}

\lstinputlisting[caption=MSVC 2012,style=customasmx86]{\CURPATH/STL/string/3_MSVC_RU.asm}

Собственно, компилятор не конструирует строки статически: да в общем-то и как
это возможно, если буфер с ней нужно хранить в \glslink{heap}{куче}?

Вместо этого в сегменте данных хранятся обычные \ac{ASCIIZ}-строки, а позже, во время выполнения, 
при помощи метода \q{assign}, конструируются строки s1 и s2
.
При помощи \TT{operator+}, создается строка s3.

Обратите внимание на то что вызов метода c\_str() отсутствует,
потому что его код достаточно короткий и компилятор вставил его прямо здесь:
если строка короче 16-и байт, то в регистре EAX остается указатель на буфер,
а если длиннее, то из этого же места достается адрес на буфер расположенный в \glslink{heap}{куче}.

Далее следуют вызовы трех деструкторов, причем, они вызываются только если строка длиннее 16-и байт:
тогда нужно освободить буфера в \glslink{heap}{куче}.
В противном случае, так как все три объекта std::string хранятся в стеке,
они освобождаются автоматически после выхода из функции.

Следовательно, работа с короткими строками более быстрая из-за м\'{е}ньшего обращения к \glslink{heap}{куче}.

Код на GCC даже проще (из-за того, что в GCC, как мы уже видели, не реализована возможность хранить короткую
строку прямо в структуре):

% TODO1 comment each function meaning
\lstinputlisting[caption=GCC 4.8.1,style=customasmx86]{\CURPATH/STL/string/3_GCC_RU.s}

Можно заметить, что в деструкторы передается не указатель на объект,
а указатель на место за 12 байт (или 3 слова) перед ним, то есть, на настоящее начало структуры.

\myparagraph{std::string как глобальная переменная}
\label{sec:std_string_as_global_variable}

Опытные программисты на \Cpp знают, что глобальные переменные \ac{STL}-типов вполне можно объявлять.

Да, действительно:

\lstinputlisting[style=customc]{\CURPATH/STL/string/5.cpp}

Но как и где будет вызываться конструктор \TT{std::string}?

На самом деле, эта переменная будет инициализирована даже перед началом \main.

\lstinputlisting[caption=MSVC 2012: здесь конструируется глобальная переменная{,} а также регистрируется её деструктор,style=customasmx86]{\CURPATH/STL/string/5_MSVC_p2.asm}

\lstinputlisting[caption=MSVC 2012: здесь глобальная переменная используется в \main,style=customasmx86]{\CURPATH/STL/string/5_MSVC_p1.asm}

\lstinputlisting[caption=MSVC 2012: эта функция-деструктор вызывается перед выходом,style=customasmx86]{\CURPATH/STL/string/5_MSVC_p3.asm}

\myindex{\CStandardLibrary!atexit()}
В реальности, из \ac{CRT}, еще до вызова main(), вызывается специальная функция,
в которой перечислены все конструкторы подобных переменных.
Более того: при помощи atexit() регистрируется функция, которая будет вызвана в конце работы программы:
в этой функции компилятор собирает вызовы деструкторов всех подобных глобальных переменных.

GCC работает похожим образом:

\lstinputlisting[caption=GCC 4.8.1,style=customasmx86]{\CURPATH/STL/string/5_GCC.s}

Но он не выделяет отдельной функции в которой будут собраны деструкторы: 
каждый деструктор передается в atexit() по одному.

% TODO а если глобальная STL-переменная в другом модуле? надо проверить.

}

\EN{\section{Returning Values}
\label{ret_val_func}

Another simple function is the one that simply returns a constant value:

\lstinputlisting[caption=\EN{\CCpp Code},style=customc]{patterns/011_ret/1.c}

Let's compile it.

\subsection{x86}

Here's what both the GCC and MSVC compilers produce (with optimization) on the x86 platform:

\lstinputlisting[caption=\Optimizing GCC/MSVC (\assemblyOutput),style=customasmx86]{patterns/011_ret/1.s}

\myindex{x86!\Instructions!RET}
There are just two instructions: the first places the value 123 into the \EAX register,
which is used by convention for storing the return
value, and the second one is \RET, which returns execution to the \gls{caller}.

The caller will take the result from the \EAX register.

\subsection{ARM}

There are a few differences on the ARM platform:

\lstinputlisting[caption=\OptimizingKeilVI (\ARMMode) ASM Output,style=customasmARM]{patterns/011_ret/1_Keil_ARM_O3.s}

ARM uses the register \Reg{0} for returning the results of functions, so 123 is copied into \Reg{0}.

\myindex{ARM!\Instructions!MOV}
\myindex{x86!\Instructions!MOV}
It is worth noting that \MOV is a misleading name for the instruction in both the x86 and ARM \ac{ISA}s.

The data is not in fact \IT{moved}, but \IT{copied}.

\subsection{MIPS}

\label{MIPS_leaf_function_ex1}

The GCC assembly output below lists registers by number:

\lstinputlisting[caption=\Optimizing GCC 4.4.5 (\assemblyOutput),style=customasmMIPS]{patterns/011_ret/MIPS.s}

\dots while \IDA does it by their pseudo names:

\lstinputlisting[caption=\Optimizing GCC 4.4.5 (IDA),style=customasmMIPS]{patterns/011_ret/MIPS_IDA.lst}

The \$2 (or \$V0) register is used to store the function's return value.
\myindex{MIPS!\Pseudoinstructions!LI}
\INS{LI} stands for ``Load Immediate'' and is the MIPS equivalent to \MOV.

\myindex{MIPS!\Instructions!J}
The other instruction is the jump instruction (J or JR) which returns the execution flow to the \gls{caller}.

\myindex{MIPS!Branch delay slot}
You might be wondering why the positions of the load instruction (LI) and the jump instruction (J or JR) are swapped. This is due to a \ac{RISC} feature called ``branch delay slot''.

The reason this happens is a quirk in the architecture of some RISC \ac{ISA}s and isn't important for our
purposes---we must simply keep in mind that in MIPS, the instruction following a jump or branch instruction
is executed \IT{before} the jump/branch instruction itself.

As a consequence, branch instructions always swap places with the instruction executed immediately beforehand.


In practice, functions which merely return 1 (\IT{true}) or 0 (\IT{false}) are very frequent.

The smallest ever of the standard UNIX utilities, \IT{/bin/true} and \IT{/bin/false} return 0 and 1 respectively, as an exit code.
(Zero as an exit code usually means success, non-zero means error.)
}
\RU{\subsubsection{std::string}
\myindex{\Cpp!STL!std::string}
\label{std_string}

\myparagraph{Как устроена структура}

Многие строковые библиотеки \InSqBrackets{\CNotes 2.2} обеспечивают структуру содержащую ссылку 
на буфер собственно со строкой, переменная всегда содержащую длину строки 
(что очень удобно для массы функций \InSqBrackets{\CNotes 2.2.1}) и переменную содержащую текущий размер буфера.

Строка в буфере обыкновенно оканчивается нулем: это для того чтобы указатель на буфер можно было
передавать в функции требующие на вход обычную сишную \ac{ASCIIZ}-строку.

Стандарт \Cpp не описывает, как именно нужно реализовывать std::string,
но, как правило, они реализованы как описано выше, с небольшими дополнениями.

Строки в \Cpp это не класс (как, например, QString в Qt), а темплейт (basic\_string), 
это сделано для того чтобы поддерживать 
строки содержащие разного типа символы: как минимум \Tchar и \IT{wchar\_t}.

Так что, std::string это класс с базовым типом \Tchar.

А std::wstring это класс с базовым типом \IT{wchar\_t}.

\mysubparagraph{MSVC}

В реализации MSVC, вместо ссылки на буфер может содержаться сам буфер (если строка короче 16-и символов).

Это означает, что каждая короткая строка будет занимать в памяти по крайней мере $16 + 4 + 4 = 24$ 
байт для 32-битной среды либо $16 + 8 + 8 = 32$ 
байта в 64-битной, а если строка длиннее 16-и символов, то прибавьте еще длину самой строки.

\lstinputlisting[caption=пример для MSVC,style=customc]{\CURPATH/STL/string/MSVC_RU.cpp}

Собственно, из этого исходника почти всё ясно.

Несколько замечаний:

Если строка короче 16-и символов, 
то отдельный буфер для строки в \glslink{heap}{куче} выделяться не будет.

Это удобно потому что на практике, основная часть строк действительно короткие.
Вероятно, разработчики в Microsoft выбрали размер в 16 символов как разумный баланс.

Теперь очень важный момент в конце функции main(): мы не пользуемся методом c\_str(), тем не менее,
если это скомпилировать и запустить, то обе строки появятся в консоли!

Работает это вот почему.

В первом случае строка короче 16-и символов и в начале объекта std::string (его можно рассматривать
просто как структуру) расположен буфер с этой строкой.
\printf трактует указатель как указатель на массив символов оканчивающийся нулем и поэтому всё работает.

Вывод второй строки (длиннее 16-и символов) даже еще опаснее: это вообще типичная программистская ошибка 
(или опечатка), забыть дописать c\_str().
Это работает потому что в это время в начале структуры расположен указатель на буфер.
Это может надолго остаться незамеченным: до тех пока там не появится строка 
короче 16-и символов, тогда процесс упадет.

\mysubparagraph{GCC}

В реализации GCC в структуре есть еще одна переменная --- reference count.

Интересно, что указатель на экземпляр класса std::string в GCC указывает не на начало самой структуры, 
а на указатель на буфера.
В libstdc++-v3\textbackslash{}include\textbackslash{}bits\textbackslash{}basic\_string.h 
мы можем прочитать что это сделано для удобства отладки:

\begin{lstlisting}
   *  The reason you want _M_data pointing to the character %array and
   *  not the _Rep is so that the debugger can see the string
   *  contents. (Probably we should add a non-inline member to get
   *  the _Rep for the debugger to use, so users can check the actual
   *  string length.)
\end{lstlisting}

\href{http://go.yurichev.com/17085}{исходный код basic\_string.h}

В нашем примере мы учитываем это:

\lstinputlisting[caption=пример для GCC,style=customc]{\CURPATH/STL/string/GCC_RU.cpp}

Нужны еще небольшие хаки чтобы сымитировать типичную ошибку, которую мы уже видели выше, из-за
более ужесточенной проверки типов в GCC, тем не менее, printf() работает и здесь без c\_str().

\myparagraph{Чуть более сложный пример}

\lstinputlisting[style=customc]{\CURPATH/STL/string/3.cpp}

\lstinputlisting[caption=MSVC 2012,style=customasmx86]{\CURPATH/STL/string/3_MSVC_RU.asm}

Собственно, компилятор не конструирует строки статически: да в общем-то и как
это возможно, если буфер с ней нужно хранить в \glslink{heap}{куче}?

Вместо этого в сегменте данных хранятся обычные \ac{ASCIIZ}-строки, а позже, во время выполнения, 
при помощи метода \q{assign}, конструируются строки s1 и s2
.
При помощи \TT{operator+}, создается строка s3.

Обратите внимание на то что вызов метода c\_str() отсутствует,
потому что его код достаточно короткий и компилятор вставил его прямо здесь:
если строка короче 16-и байт, то в регистре EAX остается указатель на буфер,
а если длиннее, то из этого же места достается адрес на буфер расположенный в \glslink{heap}{куче}.

Далее следуют вызовы трех деструкторов, причем, они вызываются только если строка длиннее 16-и байт:
тогда нужно освободить буфера в \glslink{heap}{куче}.
В противном случае, так как все три объекта std::string хранятся в стеке,
они освобождаются автоматически после выхода из функции.

Следовательно, работа с короткими строками более быстрая из-за м\'{е}ньшего обращения к \glslink{heap}{куче}.

Код на GCC даже проще (из-за того, что в GCC, как мы уже видели, не реализована возможность хранить короткую
строку прямо в структуре):

% TODO1 comment each function meaning
\lstinputlisting[caption=GCC 4.8.1,style=customasmx86]{\CURPATH/STL/string/3_GCC_RU.s}

Можно заметить, что в деструкторы передается не указатель на объект,
а указатель на место за 12 байт (или 3 слова) перед ним, то есть, на настоящее начало структуры.

\myparagraph{std::string как глобальная переменная}
\label{sec:std_string_as_global_variable}

Опытные программисты на \Cpp знают, что глобальные переменные \ac{STL}-типов вполне можно объявлять.

Да, действительно:

\lstinputlisting[style=customc]{\CURPATH/STL/string/5.cpp}

Но как и где будет вызываться конструктор \TT{std::string}?

На самом деле, эта переменная будет инициализирована даже перед началом \main.

\lstinputlisting[caption=MSVC 2012: здесь конструируется глобальная переменная{,} а также регистрируется её деструктор,style=customasmx86]{\CURPATH/STL/string/5_MSVC_p2.asm}

\lstinputlisting[caption=MSVC 2012: здесь глобальная переменная используется в \main,style=customasmx86]{\CURPATH/STL/string/5_MSVC_p1.asm}

\lstinputlisting[caption=MSVC 2012: эта функция-деструктор вызывается перед выходом,style=customasmx86]{\CURPATH/STL/string/5_MSVC_p3.asm}

\myindex{\CStandardLibrary!atexit()}
В реальности, из \ac{CRT}, еще до вызова main(), вызывается специальная функция,
в которой перечислены все конструкторы подобных переменных.
Более того: при помощи atexit() регистрируется функция, которая будет вызвана в конце работы программы:
в этой функции компилятор собирает вызовы деструкторов всех подобных глобальных переменных.

GCC работает похожим образом:

\lstinputlisting[caption=GCC 4.8.1,style=customasmx86]{\CURPATH/STL/string/5_GCC.s}

Но он не выделяет отдельной функции в которой будут собраны деструкторы: 
каждый деструктор передается в atexit() по одному.

% TODO а если глобальная STL-переменная в другом модуле? надо проверить.

}
\DE{\subsection{Einfachste XOR-Verschlüsselung überhaupt}

Ich habe einmal eine Software gesehen, bei der alle Debugging-Ausgaben mit XOR mit dem Wert 3
verschlüsselt wurden. Mit anderen Worten, die beiden niedrigsten Bits aller Buchstaben wurden invertiert.

``Hello, world'' wurde zu ``Kfool/\#tlqog'':

\begin{lstlisting}
#!/usr/bin/python

msg="Hello, world!"

print "".join(map(lambda x: chr(ord(x)^3), msg))
\end{lstlisting}

Das ist eine ziemlich interessante Verschlüsselung (oder besser eine Verschleierung),
weil sie zwei wichtige Eigenschaften hat:
1) es ist eine einzige Funktion zum Verschlüsseln und entschlüsseln, sie muss nur wiederholt angewendet werden
2) die entstehenden Buchstaben befinden sich im druckbaren Bereich, also die ganze Zeichenkette kann ohne
Escape-Symbole im Code verwendet werden.

Die zweite Eigenschaft nutzt die Tatsache, dass alle druckbaren Zeichen in Reihen organisiert sind: 0x2x-0x7x,
und wenn die beiden niederwertigsten Bits invertiert werden, wird der Buchstabe um eine oder drei Stellen nach
links oder rechts \IT{verschoben}, aber niemals in eine andere Reihe:

\begin{figure}[H]
\centering
\includegraphics[width=0.7\textwidth]{ascii_clean.png}
\caption{7-Bit \ac{ASCII} Tabelle in Emacs}
\end{figure}

\dots mit dem Zeichen 0x7F als einziger Ausnahme.

Im Folgenden werden also beispielsweise die Zeichen A-Z \IT{verschlüsselt}:

\begin{lstlisting}
#!/usr/bin/python

msg="@ABCDEFGHIJKLMNO"

print "".join(map(lambda x: chr(ord(x)^3), msg))
\end{lstlisting}

Ergebnis:
% FIXME \verb  --  relevant comment for German?
\begin{lstlisting}
CBA@GFEDKJIHONML
\end{lstlisting}

Es sieht so aus als würden die Zeichen ``@'' und ``C'' sowie ``B'' und ``A'' vertauscht werden.

Hier ist noch ein interessantes Beispiel, in dem gezeigt wird, wie die Eigenschaften von XOR
ausgenutzt werden können: Exakt den gleichen Effekt, dass druckbare Zeichen auch druckbar bleiben,
kann man dadurch erzielen, dass irgendeine Kombination der niedrigsten vier Bits invertiert wird.
}

\EN{\section{Returning Values}
\label{ret_val_func}

Another simple function is the one that simply returns a constant value:

\lstinputlisting[caption=\EN{\CCpp Code},style=customc]{patterns/011_ret/1.c}

Let's compile it.

\subsection{x86}

Here's what both the GCC and MSVC compilers produce (with optimization) on the x86 platform:

\lstinputlisting[caption=\Optimizing GCC/MSVC (\assemblyOutput),style=customasmx86]{patterns/011_ret/1.s}

\myindex{x86!\Instructions!RET}
There are just two instructions: the first places the value 123 into the \EAX register,
which is used by convention for storing the return
value, and the second one is \RET, which returns execution to the \gls{caller}.

The caller will take the result from the \EAX register.

\subsection{ARM}

There are a few differences on the ARM platform:

\lstinputlisting[caption=\OptimizingKeilVI (\ARMMode) ASM Output,style=customasmARM]{patterns/011_ret/1_Keil_ARM_O3.s}

ARM uses the register \Reg{0} for returning the results of functions, so 123 is copied into \Reg{0}.

\myindex{ARM!\Instructions!MOV}
\myindex{x86!\Instructions!MOV}
It is worth noting that \MOV is a misleading name for the instruction in both the x86 and ARM \ac{ISA}s.

The data is not in fact \IT{moved}, but \IT{copied}.

\subsection{MIPS}

\label{MIPS_leaf_function_ex1}

The GCC assembly output below lists registers by number:

\lstinputlisting[caption=\Optimizing GCC 4.4.5 (\assemblyOutput),style=customasmMIPS]{patterns/011_ret/MIPS.s}

\dots while \IDA does it by their pseudo names:

\lstinputlisting[caption=\Optimizing GCC 4.4.5 (IDA),style=customasmMIPS]{patterns/011_ret/MIPS_IDA.lst}

The \$2 (or \$V0) register is used to store the function's return value.
\myindex{MIPS!\Pseudoinstructions!LI}
\INS{LI} stands for ``Load Immediate'' and is the MIPS equivalent to \MOV.

\myindex{MIPS!\Instructions!J}
The other instruction is the jump instruction (J or JR) which returns the execution flow to the \gls{caller}.

\myindex{MIPS!Branch delay slot}
You might be wondering why the positions of the load instruction (LI) and the jump instruction (J or JR) are swapped. This is due to a \ac{RISC} feature called ``branch delay slot''.

The reason this happens is a quirk in the architecture of some RISC \ac{ISA}s and isn't important for our
purposes---we must simply keep in mind that in MIPS, the instruction following a jump or branch instruction
is executed \IT{before} the jump/branch instruction itself.

As a consequence, branch instructions always swap places with the instruction executed immediately beforehand.


In practice, functions which merely return 1 (\IT{true}) or 0 (\IT{false}) are very frequent.

The smallest ever of the standard UNIX utilities, \IT{/bin/true} and \IT{/bin/false} return 0 and 1 respectively, as an exit code.
(Zero as an exit code usually means success, non-zero means error.)
}
\RU{\subsubsection{std::string}
\myindex{\Cpp!STL!std::string}
\label{std_string}

\myparagraph{Как устроена структура}

Многие строковые библиотеки \InSqBrackets{\CNotes 2.2} обеспечивают структуру содержащую ссылку 
на буфер собственно со строкой, переменная всегда содержащую длину строки 
(что очень удобно для массы функций \InSqBrackets{\CNotes 2.2.1}) и переменную содержащую текущий размер буфера.

Строка в буфере обыкновенно оканчивается нулем: это для того чтобы указатель на буфер можно было
передавать в функции требующие на вход обычную сишную \ac{ASCIIZ}-строку.

Стандарт \Cpp не описывает, как именно нужно реализовывать std::string,
но, как правило, они реализованы как описано выше, с небольшими дополнениями.

Строки в \Cpp это не класс (как, например, QString в Qt), а темплейт (basic\_string), 
это сделано для того чтобы поддерживать 
строки содержащие разного типа символы: как минимум \Tchar и \IT{wchar\_t}.

Так что, std::string это класс с базовым типом \Tchar.

А std::wstring это класс с базовым типом \IT{wchar\_t}.

\mysubparagraph{MSVC}

В реализации MSVC, вместо ссылки на буфер может содержаться сам буфер (если строка короче 16-и символов).

Это означает, что каждая короткая строка будет занимать в памяти по крайней мере $16 + 4 + 4 = 24$ 
байт для 32-битной среды либо $16 + 8 + 8 = 32$ 
байта в 64-битной, а если строка длиннее 16-и символов, то прибавьте еще длину самой строки.

\lstinputlisting[caption=пример для MSVC,style=customc]{\CURPATH/STL/string/MSVC_RU.cpp}

Собственно, из этого исходника почти всё ясно.

Несколько замечаний:

Если строка короче 16-и символов, 
то отдельный буфер для строки в \glslink{heap}{куче} выделяться не будет.

Это удобно потому что на практике, основная часть строк действительно короткие.
Вероятно, разработчики в Microsoft выбрали размер в 16 символов как разумный баланс.

Теперь очень важный момент в конце функции main(): мы не пользуемся методом c\_str(), тем не менее,
если это скомпилировать и запустить, то обе строки появятся в консоли!

Работает это вот почему.

В первом случае строка короче 16-и символов и в начале объекта std::string (его можно рассматривать
просто как структуру) расположен буфер с этой строкой.
\printf трактует указатель как указатель на массив символов оканчивающийся нулем и поэтому всё работает.

Вывод второй строки (длиннее 16-и символов) даже еще опаснее: это вообще типичная программистская ошибка 
(или опечатка), забыть дописать c\_str().
Это работает потому что в это время в начале структуры расположен указатель на буфер.
Это может надолго остаться незамеченным: до тех пока там не появится строка 
короче 16-и символов, тогда процесс упадет.

\mysubparagraph{GCC}

В реализации GCC в структуре есть еще одна переменная --- reference count.

Интересно, что указатель на экземпляр класса std::string в GCC указывает не на начало самой структуры, 
а на указатель на буфера.
В libstdc++-v3\textbackslash{}include\textbackslash{}bits\textbackslash{}basic\_string.h 
мы можем прочитать что это сделано для удобства отладки:

\begin{lstlisting}
   *  The reason you want _M_data pointing to the character %array and
   *  not the _Rep is so that the debugger can see the string
   *  contents. (Probably we should add a non-inline member to get
   *  the _Rep for the debugger to use, so users can check the actual
   *  string length.)
\end{lstlisting}

\href{http://go.yurichev.com/17085}{исходный код basic\_string.h}

В нашем примере мы учитываем это:

\lstinputlisting[caption=пример для GCC,style=customc]{\CURPATH/STL/string/GCC_RU.cpp}

Нужны еще небольшие хаки чтобы сымитировать типичную ошибку, которую мы уже видели выше, из-за
более ужесточенной проверки типов в GCC, тем не менее, printf() работает и здесь без c\_str().

\myparagraph{Чуть более сложный пример}

\lstinputlisting[style=customc]{\CURPATH/STL/string/3.cpp}

\lstinputlisting[caption=MSVC 2012,style=customasmx86]{\CURPATH/STL/string/3_MSVC_RU.asm}

Собственно, компилятор не конструирует строки статически: да в общем-то и как
это возможно, если буфер с ней нужно хранить в \glslink{heap}{куче}?

Вместо этого в сегменте данных хранятся обычные \ac{ASCIIZ}-строки, а позже, во время выполнения, 
при помощи метода \q{assign}, конструируются строки s1 и s2
.
При помощи \TT{operator+}, создается строка s3.

Обратите внимание на то что вызов метода c\_str() отсутствует,
потому что его код достаточно короткий и компилятор вставил его прямо здесь:
если строка короче 16-и байт, то в регистре EAX остается указатель на буфер,
а если длиннее, то из этого же места достается адрес на буфер расположенный в \glslink{heap}{куче}.

Далее следуют вызовы трех деструкторов, причем, они вызываются только если строка длиннее 16-и байт:
тогда нужно освободить буфера в \glslink{heap}{куче}.
В противном случае, так как все три объекта std::string хранятся в стеке,
они освобождаются автоматически после выхода из функции.

Следовательно, работа с короткими строками более быстрая из-за м\'{е}ньшего обращения к \glslink{heap}{куче}.

Код на GCC даже проще (из-за того, что в GCC, как мы уже видели, не реализована возможность хранить короткую
строку прямо в структуре):

% TODO1 comment each function meaning
\lstinputlisting[caption=GCC 4.8.1,style=customasmx86]{\CURPATH/STL/string/3_GCC_RU.s}

Можно заметить, что в деструкторы передается не указатель на объект,
а указатель на место за 12 байт (или 3 слова) перед ним, то есть, на настоящее начало структуры.

\myparagraph{std::string как глобальная переменная}
\label{sec:std_string_as_global_variable}

Опытные программисты на \Cpp знают, что глобальные переменные \ac{STL}-типов вполне можно объявлять.

Да, действительно:

\lstinputlisting[style=customc]{\CURPATH/STL/string/5.cpp}

Но как и где будет вызываться конструктор \TT{std::string}?

На самом деле, эта переменная будет инициализирована даже перед началом \main.

\lstinputlisting[caption=MSVC 2012: здесь конструируется глобальная переменная{,} а также регистрируется её деструктор,style=customasmx86]{\CURPATH/STL/string/5_MSVC_p2.asm}

\lstinputlisting[caption=MSVC 2012: здесь глобальная переменная используется в \main,style=customasmx86]{\CURPATH/STL/string/5_MSVC_p1.asm}

\lstinputlisting[caption=MSVC 2012: эта функция-деструктор вызывается перед выходом,style=customasmx86]{\CURPATH/STL/string/5_MSVC_p3.asm}

\myindex{\CStandardLibrary!atexit()}
В реальности, из \ac{CRT}, еще до вызова main(), вызывается специальная функция,
в которой перечислены все конструкторы подобных переменных.
Более того: при помощи atexit() регистрируется функция, которая будет вызвана в конце работы программы:
в этой функции компилятор собирает вызовы деструкторов всех подобных глобальных переменных.

GCC работает похожим образом:

\lstinputlisting[caption=GCC 4.8.1,style=customasmx86]{\CURPATH/STL/string/5_GCC.s}

Но он не выделяет отдельной функции в которой будут собраны деструкторы: 
каждый деструктор передается в atexit() по одному.

% TODO а если глобальная STL-переменная в другом модуле? надо проверить.

}
\DE{\subsection{Einfachste XOR-Verschlüsselung überhaupt}

Ich habe einmal eine Software gesehen, bei der alle Debugging-Ausgaben mit XOR mit dem Wert 3
verschlüsselt wurden. Mit anderen Worten, die beiden niedrigsten Bits aller Buchstaben wurden invertiert.

``Hello, world'' wurde zu ``Kfool/\#tlqog'':

\begin{lstlisting}
#!/usr/bin/python

msg="Hello, world!"

print "".join(map(lambda x: chr(ord(x)^3), msg))
\end{lstlisting}

Das ist eine ziemlich interessante Verschlüsselung (oder besser eine Verschleierung),
weil sie zwei wichtige Eigenschaften hat:
1) es ist eine einzige Funktion zum Verschlüsseln und entschlüsseln, sie muss nur wiederholt angewendet werden
2) die entstehenden Buchstaben befinden sich im druckbaren Bereich, also die ganze Zeichenkette kann ohne
Escape-Symbole im Code verwendet werden.

Die zweite Eigenschaft nutzt die Tatsache, dass alle druckbaren Zeichen in Reihen organisiert sind: 0x2x-0x7x,
und wenn die beiden niederwertigsten Bits invertiert werden, wird der Buchstabe um eine oder drei Stellen nach
links oder rechts \IT{verschoben}, aber niemals in eine andere Reihe:

\begin{figure}[H]
\centering
\includegraphics[width=0.7\textwidth]{ascii_clean.png}
\caption{7-Bit \ac{ASCII} Tabelle in Emacs}
\end{figure}

\dots mit dem Zeichen 0x7F als einziger Ausnahme.

Im Folgenden werden also beispielsweise die Zeichen A-Z \IT{verschlüsselt}:

\begin{lstlisting}
#!/usr/bin/python

msg="@ABCDEFGHIJKLMNO"

print "".join(map(lambda x: chr(ord(x)^3), msg))
\end{lstlisting}

Ergebnis:
% FIXME \verb  --  relevant comment for German?
\begin{lstlisting}
CBA@GFEDKJIHONML
\end{lstlisting}

Es sieht so aus als würden die Zeichen ``@'' und ``C'' sowie ``B'' und ``A'' vertauscht werden.

Hier ist noch ein interessantes Beispiel, in dem gezeigt wird, wie die Eigenschaften von XOR
ausgenutzt werden können: Exakt den gleichen Effekt, dass druckbare Zeichen auch druckbar bleiben,
kann man dadurch erzielen, dass irgendeine Kombination der niedrigsten vier Bits invertiert wird.
}

\ifdefined\SPANISH
\chapter{Patrones de código}
\fi % SPANISH

\ifdefined\GERMAN
\chapter{Code-Muster}
\fi % GERMAN

\ifdefined\ENGLISH
\chapter{Code Patterns}
\fi % ENGLISH

\ifdefined\ITALIAN
\chapter{Forme di codice}
\fi % ITALIAN

\ifdefined\RUSSIAN
\chapter{Образцы кода}
\fi % RUSSIAN

\ifdefined\BRAZILIAN
\chapter{Padrões de códigos}
\fi % BRAZILIAN

\ifdefined\THAI
\chapter{รูปแบบของโค้ด}
\fi % THAI

\ifdefined\FRENCH
\chapter{Modèle de code}
\fi % FRENCH

\ifdefined\POLISH
\chapter{\PLph{}}
\fi % POLISH

% sections
\EN{\section{The method}

When the author of this book first started learning C and, later, \Cpp, he used to write small pieces of code, compile them,
and then look at the assembly language output. This made it very easy for him to understand what was going on in the code that he had written.
\footnote{In fact, he still does this when he can't understand what a particular bit of code does.}.
He did this so many times that the relationship between the \CCpp code and what the compiler produced was imprinted deeply in his mind.
It's now easy for him to imagine instantly a rough outline of a C code's appearance and function.
Perhaps this technique could be helpful for others.

%There are a lot of examples for both x86/x64 and ARM.
%Those who already familiar with one of architectures, may freely skim over pages.

By the way, there is a great website where you can do the same, with various compilers, instead of installing them on your box.
You can use it as well: \url{https://gcc.godbolt.org/}.

\section*{\Exercises}

When the author of this book studied assembly language, he also often compiled small C functions and then rewrote
them gradually to assembly, trying to make their code as short as possible.
This probably is not worth doing in real-world scenarios today,
because it's hard to compete with the latest compilers in terms of efficiency. It is, however, a very good way to gain a better understanding of assembly.
Feel free, therefore, to take any assembly code from this book and try to make it shorter.
However, don't forget to test what you have written.

% rewrote to show that debug\release and optimisations levels are orthogonal concepts.
\section*{Optimization levels and debug information}

Source code can be compiled by different compilers with various optimization levels.
A typical compiler has about three such levels, where level zero means that optimization is completely disabled.
Optimization can also be targeted towards code size or code speed.
A non-optimizing compiler is faster and produces more understandable (albeit verbose) code,
whereas an optimizing compiler is slower and tries to produce code that runs faster (but is not necessarily more compact).
In addition to optimization levels, a compiler can include some debug information in the resulting file,
producing code that is easy to debug.
One of the important features of the ´debug' code is that it might contain links
between each line of the source code and its respective machine code address.
Optimizing compilers, on the other hand, tend to produce output where entire lines of source code
can be optimized away and thus not even be present in the resulting machine code.
Reverse engineers can encounter either version, simply because some developers turn on the compiler's optimization flags and others do not.
Because of this, we'll try to work on examples of both debug and release versions of the code featured in this book, wherever possible.

Sometimes some pretty ancient compilers are used in this book, in order to get the shortest (or simplest) possible code snippet.
}
\ES{\input{patterns/patterns_opt_dbg_ES}}
\ITA{\input{patterns/patterns_opt_dbg_ITA}}
\PTBR{\input{patterns/patterns_opt_dbg_PTBR}}
\RU{\input{patterns/patterns_opt_dbg_RU}}
\THA{\input{patterns/patterns_opt_dbg_THA}}
\DE{\input{patterns/patterns_opt_dbg_DE}}
\FR{\input{patterns/patterns_opt_dbg_FR}}
\PL{\input{patterns/patterns_opt_dbg_PL}}

\RU{\section{Некоторые базовые понятия}}
\EN{\section{Some basics}}
\DE{\section{Einige Grundlagen}}
\FR{\section{Quelques bases}}
\ES{\section{\ESph{}}}
\ITA{\section{Alcune basi teoriche}}
\PTBR{\section{\PTBRph{}}}
\THA{\section{\THAph{}}}
\PL{\section{\PLph{}}}

% sections:
\EN{\input{patterns/intro_CPU_ISA_EN}}
\ES{\input{patterns/intro_CPU_ISA_ES}}
\ITA{\input{patterns/intro_CPU_ISA_ITA}}
\PTBR{\input{patterns/intro_CPU_ISA_PTBR}}
\RU{\input{patterns/intro_CPU_ISA_RU}}
\DE{\input{patterns/intro_CPU_ISA_DE}}
\FR{\input{patterns/intro_CPU_ISA_FR}}
\PL{\input{patterns/intro_CPU_ISA_PL}}

\EN{\subsection{Numeral Systems}

Humans have become accustomed to a decimal numeral system, probably because almost everyone has 10 fingers.
Nevertheless, the number \q{10} has no significant meaning in science and mathematics.
The natural numeral system in digital electronics is binary: 0 is for an absence of current in the wire, and 1 for presence.
10 in binary is 2 in decimal, 100 in binary is 4 in decimal, and so on.

% This sentence is a bit unweildy - maybe try 'Our ten-digit system would be described as having a radix...' - Renaissance
If the numeral system has 10 digits, it has a \IT{radix} (or \IT{base}) of 10.
The binary numeral system has a \IT{radix} of 2.

Important things to recall:

1) A \IT{number} is a number, while a \IT{digit} is a term from writing systems, and is usually one character

% The original is 'number' is not changed; I think the intent is value, and changed it - Renaissance
2) The value of a number does not change when converted to another radix; only the writing notation for that value has changed (and therefore the way of representing it in \ac{RAM}).

\subsection{Converting From One Radix To Another}

Positional notation is used almost every numerical system. This means that a digit has weight relative to where it is placed inside of the larger number.
If 2 is placed at the rightmost place, it's 2, but if it's placed one digit before rightmost, it's 20.

What does $1234$ stand for?

$10^3 \cdot 1 + 10^2 \cdot 2 + 10^1 \cdot 3 + 1 \cdot 4 = 1234$ or
$1000 \cdot 1 + 100 \cdot 2 + 10 \cdot 3 + 4 = 1234$

It's the same story for binary numbers, but the base is 2 instead of 10.
What does 0b101011 stand for?

$2^5 \cdot 1 + 2^4 \cdot 0 + 2^3 \cdot 1 + 2^2 \cdot 0 + 2^1 \cdot 1 + 2^0 \cdot 1 = 43$ or
$32 \cdot 1 + 16 \cdot 0 + 8 \cdot 1 + 4 \cdot 0 + 2 \cdot 1 + 1 = 43$

There is such a thing as non-positional notation, such as the Roman numeral system.
\footnote{About numeric system evolution, see \InSqBrackets{\TAOCPvolII{}, 195--213.}}.
% Maybe add a sentence to fill in that X is always 10, and is therefore non-positional, even though putting an I before subtracts and after adds, and is in that sense positional
Perhaps, humankind switched to positional notation because it's easier to do basic operations (addition, multiplication, etc.) on paper by hand.

Binary numbers can be added, subtracted and so on in the very same as taught in schools, but only 2 digits are available.

Binary numbers are bulky when represented in source code and dumps, so that is where the hexadecimal numeral system can be useful.
A hexadecimal radix uses the digits 0..9, and also 6 Latin characters: A..F.
Each hexadecimal digit takes 4 bits or 4 binary digits, so it's very easy to convert from binary number to hexadecimal and back, even manually, in one's mind.

\begin{center}
\begin{longtable}{ | l | l | l | }
\hline
\HeaderColor hexadecimal & \HeaderColor binary & \HeaderColor decimal \\
\hline
0	&0000	&0 \\
1	&0001	&1 \\
2	&0010	&2 \\
3	&0011	&3 \\
4	&0100	&4 \\
5	&0101	&5 \\
6	&0110	&6 \\
7	&0111	&7 \\
8	&1000	&8 \\
9	&1001	&9 \\
A	&1010	&10 \\
B	&1011	&11 \\
C	&1100	&12 \\
D	&1101	&13 \\
E	&1110	&14 \\
F	&1111	&15 \\
\hline
\end{longtable}
\end{center}

How can one tell which radix is being used in a specific instance?

Decimal numbers are usually written as is, i.e., 1234. Some assemblers allow an identifier on decimal radix numbers, in which the number would be written with a "d" suffix: 1234d.

Binary numbers are sometimes prepended with the "0b" prefix: 0b100110111 (\ac{GCC} has a non-standard language extension for this\footnote{\url{https://gcc.gnu.org/onlinedocs/gcc/Binary-constants.html}}).
There is also another way: using a "b" suffix, for example: 100110111b.
This book tries to use the "0b" prefix consistently throughout the book for binary numbers.

Hexadecimal numbers are prepended with "0x" prefix in \CCpp and other \ac{PL}s: 0x1234ABCD.
Alternatively, they are given a "h" suffix: 1234ABCDh. This is common way of representing them in assemblers and debuggers.
In this convention, if the number is started with a Latin (A..F) digit, a 0 is added at the beginning: 0ABCDEFh.
There was also convention that was popular in 8-bit home computers era, using \$ prefix, like \$ABCD.
The book will try to stick to "0x" prefix throughout the book for hexadecimal numbers.

Should one learn to convert numbers mentally? A table of 1-digit hexadecimal numbers can easily be memorized.
As for larger numbers, it's probably not worth tormenting yourself.

Perhaps the most visible hexadecimal numbers are in \ac{URL}s.
This is the way that non-Latin characters are encoded.
For example:
\url{https://en.wiktionary.org/wiki/na\%C3\%AFvet\%C3\%A9} is the \ac{URL} of Wiktionary article about \q{naïveté} word.

\subsubsection{Octal Radix}

Another numeral system heavily used in the past of computer programming is octal. In octal there are 8 digits (0..7), and each is mapped to 3 bits, so it's easy to convert numbers back and forth.
It has been superseded by the hexadecimal system almost everywhere, but, surprisingly, there is a *NIX utility, used often by many people, which takes octal numbers as argument: \TT{chmod}.

\myindex{UNIX!chmod}
As many *NIX users know, \TT{chmod} argument can be a number of 3 digits. The first digit represents the rights of the owner of the file (read, write and/or execute), the second is the rights for the group to which the file belongs, and the third is for everyone else.
Each digit that \TT{chmod} takes can be represented in binary form:

\begin{center}
\begin{longtable}{ | l | l | l | }
\hline
\HeaderColor decimal & \HeaderColor binary & \HeaderColor meaning \\
\hline
7	&111	&\textbf{rwx} \\
6	&110	&\textbf{rw-} \\
5	&101	&\textbf{r-x} \\
4	&100	&\textbf{r-{}-} \\
3	&011	&\textbf{-wx} \\
2	&010	&\textbf{-w-} \\
1	&001	&\textbf{-{}-x} \\
0	&000	&\textbf{-{}-{}-} \\
\hline
\end{longtable}
\end{center}

So each bit is mapped to a flag: read/write/execute.

The importance of \TT{chmod} here is that the whole number in argument can be represented as octal number.
Let's take, for example, 644.
When you run \TT{chmod 644 file}, you set read/write permissions for owner, read permissions for group and again, read permissions for everyone else.
If we convert the octal number 644 to binary, it would be \TT{110100100}, or, in groups of 3 bits, \TT{110 100 100}.

Now we see that each triplet describe permissions for owner/group/others: first is \TT{rw-}, second is \TT{r--} and third is \TT{r--}.

The octal numeral system was also popular on old computers like PDP-8, because word there could be 12, 24 or 36 bits, and these numbers are all divisible by 3, so the octal system was natural in that environment.
Nowadays, all popular computers employ word/address sizes of 16, 32 or 64 bits, and these numbers are all divisible by 4, so the hexadecimal system is more natural there.

The octal numeral system is supported by all standard \CCpp compilers.
This is a source of confusion sometimes, because octal numbers are encoded with a zero prepended, for example, 0377 is 255.
Sometimes, you might make a typo and write "09" instead of 9, and the compiler would report an error.
GCC might report something like this:\\
\TT{error: invalid digit "9" in octal constant}.

Also, the octal system is somewhat popular in Java. When the IDA shows Java strings with non-printable characters,
they are encoded in the octal system instead of hexadecimal.
\myindex{JAD}
The JAD Java decompiler behaves the same way.

\subsubsection{Divisibility}

When you see a decimal number like 120, you can quickly deduce that it's divisible by 10, because the last digit is zero.
In the same way, 123400 is divisible by 100, because the two last digits are zeros.

Likewise, the hexadecimal number 0x1230 is divisible by 0x10 (or 16), 0x123000 is divisible by 0x1000 (or 4096), etc.

The binary number 0b1000101000 is divisible by 0b1000 (8), etc.

This property can often be used to quickly realize if the size of some block in memory is padded to some boundary.
For example, sections in \ac{PE} files are almost always started at addresses ending with 3 hexadecimal zeros: 0x41000, 0x10001000, etc.
The reason behind this is the fact that almost all \ac{PE} sections are padded to a boundary of 0x1000 (4096) bytes.

\subsubsection{Multi-Precision Arithmetic and Radix}

\index{RSA}
Multi-precision arithmetic can use huge numbers, and each one may be stored in several bytes.
For example, RSA keys, both public and private, span up to 4096 bits, and maybe even more.

% I'm not sure how to change this, but the normal format for quoting would be just to mention the author or book, and footnote to the full reference
In \InSqBrackets{\TAOCPvolII, 265} we find the following idea: when you store a multi-precision number in several bytes,
the whole number can be represented as having a radix of $2^8=256$, and each digit goes to the corresponding byte.
Likewise, if you store a multi-precision number in several 32-bit integer values, each digit goes to each 32-bit slot,
and you may think about this number as stored in radix of $2^{32}$.

\subsubsection{How to Pronounce Non-Decimal Numbers}

Numbers in a non-decimal base are usually pronounced by digit by digit: ``one-zero-zero-one-one-...''.
Words like ``ten'' and ``thousand'' are usually not pronounced, to prevent confusion with the decimal base system.

\subsubsection{Floating point numbers}

To distinguish floating point numbers from integers, they are usually written with ``.0'' at the end,
like $0.0$, $123.0$, etc.
}
\RU{\subsection{Представление чисел}

Люди привыкли к десятичной системе счисления вероятно потому что почти у каждого есть по 10 пальцев.
Тем не менее, число 10 не имеет особого значения в науке и математике.
Двоичная система естествена для цифровой электроники: 0 означает отсутствие тока в проводе и 1 --- его присутствие.
10 в двоичной системе это 2 в десятичной; 100 в двоичной это 4 в десятичной, итд.

Если в системе счисления есть 10 цифр, её \IT{основание} или \IT{radix} это 10.
Двоичная система имеет \IT{основание} 2.

Важные вещи, которые полезно вспомнить:
1) \IT{число} это число, в то время как \IT{цифра} это термин из системы письменности, и это обычно один символ;
2) само число не меняется, когда конвертируется из одного основания в другое: меняется способ его записи (или представления
в памяти).

Как сконвертировать число из одного основания в другое?

Позиционная нотация используется почти везде, это означает, что всякая цифра имеет свой вес, в зависимости от её расположения
внутри числа.
Если 2 расположена в самом последнем месте справа, это 2.
Если она расположена в месте перед последним, это 20.

Что означает $1234$?

$10^3 \cdot 1 + 10^2 \cdot 2 + 10^1 \cdot 3 + 1 \cdot 4$ = 1234 или
$1000 \cdot 1 + 100 \cdot 2 + 10 \cdot 3 + 4 = 1234$

Та же история и для двоичных чисел, только основание там 2 вместо 10.
Что означает 0b101011?

$2^5 \cdot 1 + 2^4 \cdot 0 + 2^3 \cdot 1 + 2^2 \cdot 0 + 2^1 \cdot 1 + 2^0 \cdot 1 = 43$ или
$32 \cdot 1 + 16 \cdot 0 + 8 \cdot 1 + 4 \cdot 0 + 2 \cdot 1 + 1 = 43$

Позиционную нотацию можно противопоставить непозиционной нотации, такой как римская система записи чисел
\footnote{Об эволюции способов записи чисел, см.также: \InSqBrackets{\TAOCPvolII{}, 195--213.}}.
Вероятно, человечество перешло на позиционную нотацию, потому что так проще работать с числами (сложение, умножение, итд)
на бумаге, в ручную.

Действительно, двоичные числа можно складывать, вычитать, итд, точно также, как этому обычно обучают в школах,
только доступны лишь 2 цифры.

Двоичные числа громоздки, когда их используют в исходных кодах и дампах, так что в этих случаях применяется шестнадцатеричная
система.
Используются цифры 0..9 и еще 6 латинских букв: A..F.
Каждая шестнадцатеричная цифра занимает 4 бита или 4 двоичных цифры, так что конвертировать из двоичной системы в
шестнадцатеричную и назад, можно легко вручную, или даже в уме.

\begin{center}
\begin{longtable}{ | l | l | l | }
\hline
\HeaderColor шестнадцатеричная & \HeaderColor двоичная & \HeaderColor десятичная \\
\hline
0	&0000	&0 \\
1	&0001	&1 \\
2	&0010	&2 \\
3	&0011	&3 \\
4	&0100	&4 \\
5	&0101	&5 \\
6	&0110	&6 \\
7	&0111	&7 \\
8	&1000	&8 \\
9	&1001	&9 \\
A	&1010	&10 \\
B	&1011	&11 \\
C	&1100	&12 \\
D	&1101	&13 \\
E	&1110	&14 \\
F	&1111	&15 \\
\hline
\end{longtable}
\end{center}

Как понять, какое основание используется в конкретном месте?

Десятичные числа обычно записываются как есть, т.е., 1234. Но некоторые ассемблеры позволяют подчеркивать
этот факт для ясности, и это число может быть дополнено суффиксом "d": 1234d.

К двоичным числам иногда спереди добавляют префикс "0b": 0b100110111
(В \ac{GCC} для этого есть нестандартное расширение языка
\footnote{\url{https://gcc.gnu.org/onlinedocs/gcc/Binary-constants.html}}).
Есть также еще один способ: суффикс "b", например: 100110111b.
В этой книге я буду пытаться придерживаться префикса "0b" для двоичных чисел.

Шестнадцатеричные числа имеют префикс "0x" в \CCpp и некоторых других \ac{PL}: 0x1234ABCD.
Либо они имеют суффикс "h": 1234ABCDh --- обычно так они представляются в ассемблерах и отладчиках.
Если число начинается с цифры A..F, перед ним добавляется 0: 0ABCDEFh.
Во времена 8-битных домашних компьютеров, был также способ записи чисел используя префикс \$, например, \$ABCD.
В книге я попытаюсь придерживаться префикса "0x" для шестнадцатеричных чисел.

Нужно ли учиться конвертировать числа в уме? Таблицу шестнадцатеричных чисел из одной цифры легко запомнить.
А запоминать б\'{о}льшие числа, наверное, не стоит.

Наверное, чаще всего шестнадцатеричные числа можно увидеть в \ac{URL}-ах.
Так кодируются буквы не из числа латинских.
Например:
\url{https://en.wiktionary.org/wiki/na\%C3\%AFvet\%C3\%A9} это \ac{URL} страницы в Wiktionary о слове \q{naïveté}.

\subsubsection{Восьмеричная система}

Еще одна система, которая в прошлом много использовалась в программировании это восьмеричная: есть 8 цифр (0..7) и каждая
описывает 3 бита, так что легко конвертировать числа туда и назад.
Она почти везде была заменена шестнадцатеричной, но удивительно, в *NIX имеется утилита использующаяся многими людьми,
которая принимает на вход восьмеричное число: \TT{chmod}.

\myindex{UNIX!chmod}
Как знают многие пользователи *NIX, аргумент \TT{chmod} это число из трех цифр. Первая цифра это права владельца файла,
вторая это права группы (которой файл принадлежит), третья для всех остальных.
И каждая цифра может быть представлена в двоичном виде:

\begin{center}
\begin{longtable}{ | l | l | l | }
\hline
\HeaderColor десятичная & \HeaderColor двоичная & \HeaderColor значение \\
\hline
7	&111	&\textbf{rwx} \\
6	&110	&\textbf{rw-} \\
5	&101	&\textbf{r-x} \\
4	&100	&\textbf{r-{}-} \\
3	&011	&\textbf{-wx} \\
2	&010	&\textbf{-w-} \\
1	&001	&\textbf{-{}-x} \\
0	&000	&\textbf{-{}-{}-} \\
\hline
\end{longtable}
\end{center}

Так что каждый бит привязан к флагу: read/write/execute (чтение/запись/исполнение).

И вот почему я вспомнил здесь о \TT{chmod}, это потому что всё число может быть представлено как число в восьмеричной системе.
Для примера возьмем 644.
Когда вы запускаете \TT{chmod 644 file}, вы выставляете права read/write для владельца, права read для группы, и снова,
read для всех остальных.
Сконвертируем число 644 из восьмеричной системы в двоичную, это будет \TT{110100100}, или (в группах по 3 бита) \TT{110 100 100}.

Теперь мы видим, что каждая тройка описывает права для владельца/группы/остальных:
первая это \TT{rw-}, вторая это \TT{r--} и третья это \TT{r--}.

Восьмеричная система была также популярная на старых компьютерах вроде PDP-8, потому что слово там могло содержать 12, 24 или
36 бит, и эти числа делятся на 3, так что выбор восьмеричной системы в той среде был логичен.
Сейчас, все популярные компьютеры имеют размер слова/адреса 16, 32 или 64 бита, и эти числа делятся на 4,
так что шестнадцатеричная система здесь удобнее.

Восьмеричная система поддерживается всеми стандартными компиляторами \CCpp{}.
Это иногда источник недоумения, потому что восьмеричные числа кодируются с нулем вперед, например, 0377 это 255.
И иногда, вы можете сделать опечатку, и написать "09" вместо 9, и компилятор выдаст ошибку.
GCC может выдать что-то вроде:\\
\TT{error: invalid digit "9" in octal constant}.

Также, восьмеричная система популярна в Java: когда IDA показывает строку с непечатаемыми символами,
они кодируются в восьмеричной системе вместо шестнадцатеричной.
\myindex{JAD}
Точно также себя ведет декомпилятор с Java JAD.

\subsubsection{Делимость}

Когда вы видите десятичное число вроде 120, вы можете быстро понять что оно делится на 10, потому что последняя цифра это 0.
Точно также, 123400 делится на 100, потому что две последних цифры это нули.

Точно также, шестнадцатеричное число 0x1230 делится на 0x10 (или 16), 0x123000 делится на 0x1000 (или 4096), итд.

Двоичное число 0b1000101000 делится на 0b1000 (8), итд.

Это свойство можно часто использовать, чтобы быстро понять,
что длина какого-либо блока в памяти выровнена по некоторой границе.
Например, секции в \ac{PE}-файлах почти всегда начинаются с адресов заканчивающихся 3 шестнадцатеричными нулями:
0x41000, 0x10001000, итд.
Причина в том, что почти все секции в \ac{PE} выровнены по границе 0x1000 (4096) байт.

\subsubsection{Арифметика произвольной точности и основание}

\index{RSA}
Арифметика произвольной точности (multi-precision arithmetic) может использовать огромные числа,
которые могут храниться в нескольких байтах.
Например, ключи RSA, и открытые и закрытые, могут занимать до 4096 бит и даже больше.

В \InSqBrackets{\TAOCPvolII, 265} можно найти такую идею: когда вы сохраняете число произвольной точности в нескольких байтах,
всё число может быть представлено как имеющую систему счисления по основанию $2^8=256$, и каждая цифра находится
в соответствующем байте.
Точно также, если вы сохраняете число произвольной точности в нескольких 32-битных целочисленных значениях,
каждая цифра отправляется в каждый 32-битный слот, и вы можете считать что это число записано в системе с основанием $2^{32}$.

\subsubsection{Произношение}

Числа в недесятичных системах счислениях обычно произносятся по одной цифре: ``один-ноль-ноль-один-один-...''.
Слова вроде ``десять'', ``тысяча'', итд, обычно не произносятся, потому что тогда можно спутать с десятичной системой.

\subsubsection{Числа с плавающей запятой}

Чтобы отличать числа с плавающей запятой от целочисленных, часто, в конце добавляют ``.0'',
например $0.0$, $123.0$, итд.

}
\ITA{\input{patterns/numeral_ITA}}
\DE{\input{patterns/numeral_DE}}
\FR{\input{patterns/numeral_FR}}
\PL{\input{patterns/numeral_PL}}

% chapters
\ifdefined\SPANISH
\chapter{Patrones de código}
\fi % SPANISH

\ifdefined\GERMAN
\chapter{Code-Muster}
\fi % GERMAN

\ifdefined\ENGLISH
\chapter{Code Patterns}
\fi % ENGLISH

\ifdefined\ITALIAN
\chapter{Forme di codice}
\fi % ITALIAN

\ifdefined\RUSSIAN
\chapter{Образцы кода}
\fi % RUSSIAN

\ifdefined\BRAZILIAN
\chapter{Padrões de códigos}
\fi % BRAZILIAN

\ifdefined\THAI
\chapter{รูปแบบของโค้ด}
\fi % THAI

\ifdefined\FRENCH
\chapter{Modèle de code}
\fi % FRENCH

\ifdefined\POLISH
\chapter{\PLph{}}
\fi % POLISH

% sections
\EN{\input{patterns/patterns_opt_dbg_EN}}
\ES{\input{patterns/patterns_opt_dbg_ES}}
\ITA{\input{patterns/patterns_opt_dbg_ITA}}
\PTBR{\input{patterns/patterns_opt_dbg_PTBR}}
\RU{\input{patterns/patterns_opt_dbg_RU}}
\THA{\input{patterns/patterns_opt_dbg_THA}}
\DE{\input{patterns/patterns_opt_dbg_DE}}
\FR{\input{patterns/patterns_opt_dbg_FR}}
\PL{\input{patterns/patterns_opt_dbg_PL}}

\RU{\section{Некоторые базовые понятия}}
\EN{\section{Some basics}}
\DE{\section{Einige Grundlagen}}
\FR{\section{Quelques bases}}
\ES{\section{\ESph{}}}
\ITA{\section{Alcune basi teoriche}}
\PTBR{\section{\PTBRph{}}}
\THA{\section{\THAph{}}}
\PL{\section{\PLph{}}}

% sections:
\EN{\input{patterns/intro_CPU_ISA_EN}}
\ES{\input{patterns/intro_CPU_ISA_ES}}
\ITA{\input{patterns/intro_CPU_ISA_ITA}}
\PTBR{\input{patterns/intro_CPU_ISA_PTBR}}
\RU{\input{patterns/intro_CPU_ISA_RU}}
\DE{\input{patterns/intro_CPU_ISA_DE}}
\FR{\input{patterns/intro_CPU_ISA_FR}}
\PL{\input{patterns/intro_CPU_ISA_PL}}

\EN{\input{patterns/numeral_EN}}
\RU{\input{patterns/numeral_RU}}
\ITA{\input{patterns/numeral_ITA}}
\DE{\input{patterns/numeral_DE}}
\FR{\input{patterns/numeral_FR}}
\PL{\input{patterns/numeral_PL}}

% chapters
\input{patterns/00_empty/main}
\input{patterns/011_ret/main}
\input{patterns/01_helloworld/main}
\input{patterns/015_prolog_epilogue/main}
\input{patterns/02_stack/main}
\input{patterns/03_printf/main}
\input{patterns/04_scanf/main}
\input{patterns/05_passing_arguments/main}
\input{patterns/06_return_results/main}
\input{patterns/061_pointers/main}
\input{patterns/065_GOTO/main}
\input{patterns/07_jcc/main}
\input{patterns/08_switch/main}
\input{patterns/09_loops/main}
\input{patterns/10_strings/main}
\input{patterns/11_arith_optimizations/main}
\input{patterns/12_FPU/main}
\input{patterns/13_arrays/main}
\input{patterns/14_bitfields/main}
\EN{\input{patterns/145_LCG/main_EN}}
\RU{\input{patterns/145_LCG/main_RU}}
\input{patterns/15_structs/main}
\input{patterns/17_unions/main}
\input{patterns/18_pointers_to_functions/main}
\input{patterns/185_64bit_in_32_env/main}

\EN{\input{patterns/19_SIMD/main_EN}}
\RU{\input{patterns/19_SIMD/main_RU}}
\DE{\input{patterns/19_SIMD/main_DE}}

\EN{\input{patterns/20_x64/main_EN}}
\RU{\input{patterns/20_x64/main_RU}}

\EN{\input{patterns/205_floating_SIMD/main_EN}}
\RU{\input{patterns/205_floating_SIMD/main_RU}}
\DE{\input{patterns/205_floating_SIMD/main_DE}}

\EN{\input{patterns/ARM/main_EN}}
\RU{\input{patterns/ARM/main_RU}}
\DE{\input{patterns/ARM/main_DE}}

\input{patterns/MIPS/main}

\ifdefined\SPANISH
\chapter{Patrones de código}
\fi % SPANISH

\ifdefined\GERMAN
\chapter{Code-Muster}
\fi % GERMAN

\ifdefined\ENGLISH
\chapter{Code Patterns}
\fi % ENGLISH

\ifdefined\ITALIAN
\chapter{Forme di codice}
\fi % ITALIAN

\ifdefined\RUSSIAN
\chapter{Образцы кода}
\fi % RUSSIAN

\ifdefined\BRAZILIAN
\chapter{Padrões de códigos}
\fi % BRAZILIAN

\ifdefined\THAI
\chapter{รูปแบบของโค้ด}
\fi % THAI

\ifdefined\FRENCH
\chapter{Modèle de code}
\fi % FRENCH

\ifdefined\POLISH
\chapter{\PLph{}}
\fi % POLISH

% sections
\EN{\input{patterns/patterns_opt_dbg_EN}}
\ES{\input{patterns/patterns_opt_dbg_ES}}
\ITA{\input{patterns/patterns_opt_dbg_ITA}}
\PTBR{\input{patterns/patterns_opt_dbg_PTBR}}
\RU{\input{patterns/patterns_opt_dbg_RU}}
\THA{\input{patterns/patterns_opt_dbg_THA}}
\DE{\input{patterns/patterns_opt_dbg_DE}}
\FR{\input{patterns/patterns_opt_dbg_FR}}
\PL{\input{patterns/patterns_opt_dbg_PL}}

\RU{\section{Некоторые базовые понятия}}
\EN{\section{Some basics}}
\DE{\section{Einige Grundlagen}}
\FR{\section{Quelques bases}}
\ES{\section{\ESph{}}}
\ITA{\section{Alcune basi teoriche}}
\PTBR{\section{\PTBRph{}}}
\THA{\section{\THAph{}}}
\PL{\section{\PLph{}}}

% sections:
\EN{\input{patterns/intro_CPU_ISA_EN}}
\ES{\input{patterns/intro_CPU_ISA_ES}}
\ITA{\input{patterns/intro_CPU_ISA_ITA}}
\PTBR{\input{patterns/intro_CPU_ISA_PTBR}}
\RU{\input{patterns/intro_CPU_ISA_RU}}
\DE{\input{patterns/intro_CPU_ISA_DE}}
\FR{\input{patterns/intro_CPU_ISA_FR}}
\PL{\input{patterns/intro_CPU_ISA_PL}}

\EN{\input{patterns/numeral_EN}}
\RU{\input{patterns/numeral_RU}}
\ITA{\input{patterns/numeral_ITA}}
\DE{\input{patterns/numeral_DE}}
\FR{\input{patterns/numeral_FR}}
\PL{\input{patterns/numeral_PL}}

% chapters
\input{patterns/00_empty/main}
\input{patterns/011_ret/main}
\input{patterns/01_helloworld/main}
\input{patterns/015_prolog_epilogue/main}
\input{patterns/02_stack/main}
\input{patterns/03_printf/main}
\input{patterns/04_scanf/main}
\input{patterns/05_passing_arguments/main}
\input{patterns/06_return_results/main}
\input{patterns/061_pointers/main}
\input{patterns/065_GOTO/main}
\input{patterns/07_jcc/main}
\input{patterns/08_switch/main}
\input{patterns/09_loops/main}
\input{patterns/10_strings/main}
\input{patterns/11_arith_optimizations/main}
\input{patterns/12_FPU/main}
\input{patterns/13_arrays/main}
\input{patterns/14_bitfields/main}
\EN{\input{patterns/145_LCG/main_EN}}
\RU{\input{patterns/145_LCG/main_RU}}
\input{patterns/15_structs/main}
\input{patterns/17_unions/main}
\input{patterns/18_pointers_to_functions/main}
\input{patterns/185_64bit_in_32_env/main}

\EN{\input{patterns/19_SIMD/main_EN}}
\RU{\input{patterns/19_SIMD/main_RU}}
\DE{\input{patterns/19_SIMD/main_DE}}

\EN{\input{patterns/20_x64/main_EN}}
\RU{\input{patterns/20_x64/main_RU}}

\EN{\input{patterns/205_floating_SIMD/main_EN}}
\RU{\input{patterns/205_floating_SIMD/main_RU}}
\DE{\input{patterns/205_floating_SIMD/main_DE}}

\EN{\input{patterns/ARM/main_EN}}
\RU{\input{patterns/ARM/main_RU}}
\DE{\input{patterns/ARM/main_DE}}

\input{patterns/MIPS/main}

\ifdefined\SPANISH
\chapter{Patrones de código}
\fi % SPANISH

\ifdefined\GERMAN
\chapter{Code-Muster}
\fi % GERMAN

\ifdefined\ENGLISH
\chapter{Code Patterns}
\fi % ENGLISH

\ifdefined\ITALIAN
\chapter{Forme di codice}
\fi % ITALIAN

\ifdefined\RUSSIAN
\chapter{Образцы кода}
\fi % RUSSIAN

\ifdefined\BRAZILIAN
\chapter{Padrões de códigos}
\fi % BRAZILIAN

\ifdefined\THAI
\chapter{รูปแบบของโค้ด}
\fi % THAI

\ifdefined\FRENCH
\chapter{Modèle de code}
\fi % FRENCH

\ifdefined\POLISH
\chapter{\PLph{}}
\fi % POLISH

% sections
\EN{\input{patterns/patterns_opt_dbg_EN}}
\ES{\input{patterns/patterns_opt_dbg_ES}}
\ITA{\input{patterns/patterns_opt_dbg_ITA}}
\PTBR{\input{patterns/patterns_opt_dbg_PTBR}}
\RU{\input{patterns/patterns_opt_dbg_RU}}
\THA{\input{patterns/patterns_opt_dbg_THA}}
\DE{\input{patterns/patterns_opt_dbg_DE}}
\FR{\input{patterns/patterns_opt_dbg_FR}}
\PL{\input{patterns/patterns_opt_dbg_PL}}

\RU{\section{Некоторые базовые понятия}}
\EN{\section{Some basics}}
\DE{\section{Einige Grundlagen}}
\FR{\section{Quelques bases}}
\ES{\section{\ESph{}}}
\ITA{\section{Alcune basi teoriche}}
\PTBR{\section{\PTBRph{}}}
\THA{\section{\THAph{}}}
\PL{\section{\PLph{}}}

% sections:
\EN{\input{patterns/intro_CPU_ISA_EN}}
\ES{\input{patterns/intro_CPU_ISA_ES}}
\ITA{\input{patterns/intro_CPU_ISA_ITA}}
\PTBR{\input{patterns/intro_CPU_ISA_PTBR}}
\RU{\input{patterns/intro_CPU_ISA_RU}}
\DE{\input{patterns/intro_CPU_ISA_DE}}
\FR{\input{patterns/intro_CPU_ISA_FR}}
\PL{\input{patterns/intro_CPU_ISA_PL}}

\EN{\input{patterns/numeral_EN}}
\RU{\input{patterns/numeral_RU}}
\ITA{\input{patterns/numeral_ITA}}
\DE{\input{patterns/numeral_DE}}
\FR{\input{patterns/numeral_FR}}
\PL{\input{patterns/numeral_PL}}

% chapters
\input{patterns/00_empty/main}
\input{patterns/011_ret/main}
\input{patterns/01_helloworld/main}
\input{patterns/015_prolog_epilogue/main}
\input{patterns/02_stack/main}
\input{patterns/03_printf/main}
\input{patterns/04_scanf/main}
\input{patterns/05_passing_arguments/main}
\input{patterns/06_return_results/main}
\input{patterns/061_pointers/main}
\input{patterns/065_GOTO/main}
\input{patterns/07_jcc/main}
\input{patterns/08_switch/main}
\input{patterns/09_loops/main}
\input{patterns/10_strings/main}
\input{patterns/11_arith_optimizations/main}
\input{patterns/12_FPU/main}
\input{patterns/13_arrays/main}
\input{patterns/14_bitfields/main}
\EN{\input{patterns/145_LCG/main_EN}}
\RU{\input{patterns/145_LCG/main_RU}}
\input{patterns/15_structs/main}
\input{patterns/17_unions/main}
\input{patterns/18_pointers_to_functions/main}
\input{patterns/185_64bit_in_32_env/main}

\EN{\input{patterns/19_SIMD/main_EN}}
\RU{\input{patterns/19_SIMD/main_RU}}
\DE{\input{patterns/19_SIMD/main_DE}}

\EN{\input{patterns/20_x64/main_EN}}
\RU{\input{patterns/20_x64/main_RU}}

\EN{\input{patterns/205_floating_SIMD/main_EN}}
\RU{\input{patterns/205_floating_SIMD/main_RU}}
\DE{\input{patterns/205_floating_SIMD/main_DE}}

\EN{\input{patterns/ARM/main_EN}}
\RU{\input{patterns/ARM/main_RU}}
\DE{\input{patterns/ARM/main_DE}}

\input{patterns/MIPS/main}

\ifdefined\SPANISH
\chapter{Patrones de código}
\fi % SPANISH

\ifdefined\GERMAN
\chapter{Code-Muster}
\fi % GERMAN

\ifdefined\ENGLISH
\chapter{Code Patterns}
\fi % ENGLISH

\ifdefined\ITALIAN
\chapter{Forme di codice}
\fi % ITALIAN

\ifdefined\RUSSIAN
\chapter{Образцы кода}
\fi % RUSSIAN

\ifdefined\BRAZILIAN
\chapter{Padrões de códigos}
\fi % BRAZILIAN

\ifdefined\THAI
\chapter{รูปแบบของโค้ด}
\fi % THAI

\ifdefined\FRENCH
\chapter{Modèle de code}
\fi % FRENCH

\ifdefined\POLISH
\chapter{\PLph{}}
\fi % POLISH

% sections
\EN{\input{patterns/patterns_opt_dbg_EN}}
\ES{\input{patterns/patterns_opt_dbg_ES}}
\ITA{\input{patterns/patterns_opt_dbg_ITA}}
\PTBR{\input{patterns/patterns_opt_dbg_PTBR}}
\RU{\input{patterns/patterns_opt_dbg_RU}}
\THA{\input{patterns/patterns_opt_dbg_THA}}
\DE{\input{patterns/patterns_opt_dbg_DE}}
\FR{\input{patterns/patterns_opt_dbg_FR}}
\PL{\input{patterns/patterns_opt_dbg_PL}}

\RU{\section{Некоторые базовые понятия}}
\EN{\section{Some basics}}
\DE{\section{Einige Grundlagen}}
\FR{\section{Quelques bases}}
\ES{\section{\ESph{}}}
\ITA{\section{Alcune basi teoriche}}
\PTBR{\section{\PTBRph{}}}
\THA{\section{\THAph{}}}
\PL{\section{\PLph{}}}

% sections:
\EN{\input{patterns/intro_CPU_ISA_EN}}
\ES{\input{patterns/intro_CPU_ISA_ES}}
\ITA{\input{patterns/intro_CPU_ISA_ITA}}
\PTBR{\input{patterns/intro_CPU_ISA_PTBR}}
\RU{\input{patterns/intro_CPU_ISA_RU}}
\DE{\input{patterns/intro_CPU_ISA_DE}}
\FR{\input{patterns/intro_CPU_ISA_FR}}
\PL{\input{patterns/intro_CPU_ISA_PL}}

\EN{\input{patterns/numeral_EN}}
\RU{\input{patterns/numeral_RU}}
\ITA{\input{patterns/numeral_ITA}}
\DE{\input{patterns/numeral_DE}}
\FR{\input{patterns/numeral_FR}}
\PL{\input{patterns/numeral_PL}}

% chapters
\input{patterns/00_empty/main}
\input{patterns/011_ret/main}
\input{patterns/01_helloworld/main}
\input{patterns/015_prolog_epilogue/main}
\input{patterns/02_stack/main}
\input{patterns/03_printf/main}
\input{patterns/04_scanf/main}
\input{patterns/05_passing_arguments/main}
\input{patterns/06_return_results/main}
\input{patterns/061_pointers/main}
\input{patterns/065_GOTO/main}
\input{patterns/07_jcc/main}
\input{patterns/08_switch/main}
\input{patterns/09_loops/main}
\input{patterns/10_strings/main}
\input{patterns/11_arith_optimizations/main}
\input{patterns/12_FPU/main}
\input{patterns/13_arrays/main}
\input{patterns/14_bitfields/main}
\EN{\input{patterns/145_LCG/main_EN}}
\RU{\input{patterns/145_LCG/main_RU}}
\input{patterns/15_structs/main}
\input{patterns/17_unions/main}
\input{patterns/18_pointers_to_functions/main}
\input{patterns/185_64bit_in_32_env/main}

\EN{\input{patterns/19_SIMD/main_EN}}
\RU{\input{patterns/19_SIMD/main_RU}}
\DE{\input{patterns/19_SIMD/main_DE}}

\EN{\input{patterns/20_x64/main_EN}}
\RU{\input{patterns/20_x64/main_RU}}

\EN{\input{patterns/205_floating_SIMD/main_EN}}
\RU{\input{patterns/205_floating_SIMD/main_RU}}
\DE{\input{patterns/205_floating_SIMD/main_DE}}

\EN{\input{patterns/ARM/main_EN}}
\RU{\input{patterns/ARM/main_RU}}
\DE{\input{patterns/ARM/main_DE}}

\input{patterns/MIPS/main}

\ifdefined\SPANISH
\chapter{Patrones de código}
\fi % SPANISH

\ifdefined\GERMAN
\chapter{Code-Muster}
\fi % GERMAN

\ifdefined\ENGLISH
\chapter{Code Patterns}
\fi % ENGLISH

\ifdefined\ITALIAN
\chapter{Forme di codice}
\fi % ITALIAN

\ifdefined\RUSSIAN
\chapter{Образцы кода}
\fi % RUSSIAN

\ifdefined\BRAZILIAN
\chapter{Padrões de códigos}
\fi % BRAZILIAN

\ifdefined\THAI
\chapter{รูปแบบของโค้ด}
\fi % THAI

\ifdefined\FRENCH
\chapter{Modèle de code}
\fi % FRENCH

\ifdefined\POLISH
\chapter{\PLph{}}
\fi % POLISH

% sections
\EN{\input{patterns/patterns_opt_dbg_EN}}
\ES{\input{patterns/patterns_opt_dbg_ES}}
\ITA{\input{patterns/patterns_opt_dbg_ITA}}
\PTBR{\input{patterns/patterns_opt_dbg_PTBR}}
\RU{\input{patterns/patterns_opt_dbg_RU}}
\THA{\input{patterns/patterns_opt_dbg_THA}}
\DE{\input{patterns/patterns_opt_dbg_DE}}
\FR{\input{patterns/patterns_opt_dbg_FR}}
\PL{\input{patterns/patterns_opt_dbg_PL}}

\RU{\section{Некоторые базовые понятия}}
\EN{\section{Some basics}}
\DE{\section{Einige Grundlagen}}
\FR{\section{Quelques bases}}
\ES{\section{\ESph{}}}
\ITA{\section{Alcune basi teoriche}}
\PTBR{\section{\PTBRph{}}}
\THA{\section{\THAph{}}}
\PL{\section{\PLph{}}}

% sections:
\EN{\input{patterns/intro_CPU_ISA_EN}}
\ES{\input{patterns/intro_CPU_ISA_ES}}
\ITA{\input{patterns/intro_CPU_ISA_ITA}}
\PTBR{\input{patterns/intro_CPU_ISA_PTBR}}
\RU{\input{patterns/intro_CPU_ISA_RU}}
\DE{\input{patterns/intro_CPU_ISA_DE}}
\FR{\input{patterns/intro_CPU_ISA_FR}}
\PL{\input{patterns/intro_CPU_ISA_PL}}

\EN{\input{patterns/numeral_EN}}
\RU{\input{patterns/numeral_RU}}
\ITA{\input{patterns/numeral_ITA}}
\DE{\input{patterns/numeral_DE}}
\FR{\input{patterns/numeral_FR}}
\PL{\input{patterns/numeral_PL}}

% chapters
\input{patterns/00_empty/main}
\input{patterns/011_ret/main}
\input{patterns/01_helloworld/main}
\input{patterns/015_prolog_epilogue/main}
\input{patterns/02_stack/main}
\input{patterns/03_printf/main}
\input{patterns/04_scanf/main}
\input{patterns/05_passing_arguments/main}
\input{patterns/06_return_results/main}
\input{patterns/061_pointers/main}
\input{patterns/065_GOTO/main}
\input{patterns/07_jcc/main}
\input{patterns/08_switch/main}
\input{patterns/09_loops/main}
\input{patterns/10_strings/main}
\input{patterns/11_arith_optimizations/main}
\input{patterns/12_FPU/main}
\input{patterns/13_arrays/main}
\input{patterns/14_bitfields/main}
\EN{\input{patterns/145_LCG/main_EN}}
\RU{\input{patterns/145_LCG/main_RU}}
\input{patterns/15_structs/main}
\input{patterns/17_unions/main}
\input{patterns/18_pointers_to_functions/main}
\input{patterns/185_64bit_in_32_env/main}

\EN{\input{patterns/19_SIMD/main_EN}}
\RU{\input{patterns/19_SIMD/main_RU}}
\DE{\input{patterns/19_SIMD/main_DE}}

\EN{\input{patterns/20_x64/main_EN}}
\RU{\input{patterns/20_x64/main_RU}}

\EN{\input{patterns/205_floating_SIMD/main_EN}}
\RU{\input{patterns/205_floating_SIMD/main_RU}}
\DE{\input{patterns/205_floating_SIMD/main_DE}}

\EN{\input{patterns/ARM/main_EN}}
\RU{\input{patterns/ARM/main_RU}}
\DE{\input{patterns/ARM/main_DE}}

\input{patterns/MIPS/main}

\ifdefined\SPANISH
\chapter{Patrones de código}
\fi % SPANISH

\ifdefined\GERMAN
\chapter{Code-Muster}
\fi % GERMAN

\ifdefined\ENGLISH
\chapter{Code Patterns}
\fi % ENGLISH

\ifdefined\ITALIAN
\chapter{Forme di codice}
\fi % ITALIAN

\ifdefined\RUSSIAN
\chapter{Образцы кода}
\fi % RUSSIAN

\ifdefined\BRAZILIAN
\chapter{Padrões de códigos}
\fi % BRAZILIAN

\ifdefined\THAI
\chapter{รูปแบบของโค้ด}
\fi % THAI

\ifdefined\FRENCH
\chapter{Modèle de code}
\fi % FRENCH

\ifdefined\POLISH
\chapter{\PLph{}}
\fi % POLISH

% sections
\EN{\input{patterns/patterns_opt_dbg_EN}}
\ES{\input{patterns/patterns_opt_dbg_ES}}
\ITA{\input{patterns/patterns_opt_dbg_ITA}}
\PTBR{\input{patterns/patterns_opt_dbg_PTBR}}
\RU{\input{patterns/patterns_opt_dbg_RU}}
\THA{\input{patterns/patterns_opt_dbg_THA}}
\DE{\input{patterns/patterns_opt_dbg_DE}}
\FR{\input{patterns/patterns_opt_dbg_FR}}
\PL{\input{patterns/patterns_opt_dbg_PL}}

\RU{\section{Некоторые базовые понятия}}
\EN{\section{Some basics}}
\DE{\section{Einige Grundlagen}}
\FR{\section{Quelques bases}}
\ES{\section{\ESph{}}}
\ITA{\section{Alcune basi teoriche}}
\PTBR{\section{\PTBRph{}}}
\THA{\section{\THAph{}}}
\PL{\section{\PLph{}}}

% sections:
\EN{\input{patterns/intro_CPU_ISA_EN}}
\ES{\input{patterns/intro_CPU_ISA_ES}}
\ITA{\input{patterns/intro_CPU_ISA_ITA}}
\PTBR{\input{patterns/intro_CPU_ISA_PTBR}}
\RU{\input{patterns/intro_CPU_ISA_RU}}
\DE{\input{patterns/intro_CPU_ISA_DE}}
\FR{\input{patterns/intro_CPU_ISA_FR}}
\PL{\input{patterns/intro_CPU_ISA_PL}}

\EN{\input{patterns/numeral_EN}}
\RU{\input{patterns/numeral_RU}}
\ITA{\input{patterns/numeral_ITA}}
\DE{\input{patterns/numeral_DE}}
\FR{\input{patterns/numeral_FR}}
\PL{\input{patterns/numeral_PL}}

% chapters
\input{patterns/00_empty/main}
\input{patterns/011_ret/main}
\input{patterns/01_helloworld/main}
\input{patterns/015_prolog_epilogue/main}
\input{patterns/02_stack/main}
\input{patterns/03_printf/main}
\input{patterns/04_scanf/main}
\input{patterns/05_passing_arguments/main}
\input{patterns/06_return_results/main}
\input{patterns/061_pointers/main}
\input{patterns/065_GOTO/main}
\input{patterns/07_jcc/main}
\input{patterns/08_switch/main}
\input{patterns/09_loops/main}
\input{patterns/10_strings/main}
\input{patterns/11_arith_optimizations/main}
\input{patterns/12_FPU/main}
\input{patterns/13_arrays/main}
\input{patterns/14_bitfields/main}
\EN{\input{patterns/145_LCG/main_EN}}
\RU{\input{patterns/145_LCG/main_RU}}
\input{patterns/15_structs/main}
\input{patterns/17_unions/main}
\input{patterns/18_pointers_to_functions/main}
\input{patterns/185_64bit_in_32_env/main}

\EN{\input{patterns/19_SIMD/main_EN}}
\RU{\input{patterns/19_SIMD/main_RU}}
\DE{\input{patterns/19_SIMD/main_DE}}

\EN{\input{patterns/20_x64/main_EN}}
\RU{\input{patterns/20_x64/main_RU}}

\EN{\input{patterns/205_floating_SIMD/main_EN}}
\RU{\input{patterns/205_floating_SIMD/main_RU}}
\DE{\input{patterns/205_floating_SIMD/main_DE}}

\EN{\input{patterns/ARM/main_EN}}
\RU{\input{patterns/ARM/main_RU}}
\DE{\input{patterns/ARM/main_DE}}

\input{patterns/MIPS/main}

\ifdefined\SPANISH
\chapter{Patrones de código}
\fi % SPANISH

\ifdefined\GERMAN
\chapter{Code-Muster}
\fi % GERMAN

\ifdefined\ENGLISH
\chapter{Code Patterns}
\fi % ENGLISH

\ifdefined\ITALIAN
\chapter{Forme di codice}
\fi % ITALIAN

\ifdefined\RUSSIAN
\chapter{Образцы кода}
\fi % RUSSIAN

\ifdefined\BRAZILIAN
\chapter{Padrões de códigos}
\fi % BRAZILIAN

\ifdefined\THAI
\chapter{รูปแบบของโค้ด}
\fi % THAI

\ifdefined\FRENCH
\chapter{Modèle de code}
\fi % FRENCH

\ifdefined\POLISH
\chapter{\PLph{}}
\fi % POLISH

% sections
\EN{\input{patterns/patterns_opt_dbg_EN}}
\ES{\input{patterns/patterns_opt_dbg_ES}}
\ITA{\input{patterns/patterns_opt_dbg_ITA}}
\PTBR{\input{patterns/patterns_opt_dbg_PTBR}}
\RU{\input{patterns/patterns_opt_dbg_RU}}
\THA{\input{patterns/patterns_opt_dbg_THA}}
\DE{\input{patterns/patterns_opt_dbg_DE}}
\FR{\input{patterns/patterns_opt_dbg_FR}}
\PL{\input{patterns/patterns_opt_dbg_PL}}

\RU{\section{Некоторые базовые понятия}}
\EN{\section{Some basics}}
\DE{\section{Einige Grundlagen}}
\FR{\section{Quelques bases}}
\ES{\section{\ESph{}}}
\ITA{\section{Alcune basi teoriche}}
\PTBR{\section{\PTBRph{}}}
\THA{\section{\THAph{}}}
\PL{\section{\PLph{}}}

% sections:
\EN{\input{patterns/intro_CPU_ISA_EN}}
\ES{\input{patterns/intro_CPU_ISA_ES}}
\ITA{\input{patterns/intro_CPU_ISA_ITA}}
\PTBR{\input{patterns/intro_CPU_ISA_PTBR}}
\RU{\input{patterns/intro_CPU_ISA_RU}}
\DE{\input{patterns/intro_CPU_ISA_DE}}
\FR{\input{patterns/intro_CPU_ISA_FR}}
\PL{\input{patterns/intro_CPU_ISA_PL}}

\EN{\input{patterns/numeral_EN}}
\RU{\input{patterns/numeral_RU}}
\ITA{\input{patterns/numeral_ITA}}
\DE{\input{patterns/numeral_DE}}
\FR{\input{patterns/numeral_FR}}
\PL{\input{patterns/numeral_PL}}

% chapters
\input{patterns/00_empty/main}
\input{patterns/011_ret/main}
\input{patterns/01_helloworld/main}
\input{patterns/015_prolog_epilogue/main}
\input{patterns/02_stack/main}
\input{patterns/03_printf/main}
\input{patterns/04_scanf/main}
\input{patterns/05_passing_arguments/main}
\input{patterns/06_return_results/main}
\input{patterns/061_pointers/main}
\input{patterns/065_GOTO/main}
\input{patterns/07_jcc/main}
\input{patterns/08_switch/main}
\input{patterns/09_loops/main}
\input{patterns/10_strings/main}
\input{patterns/11_arith_optimizations/main}
\input{patterns/12_FPU/main}
\input{patterns/13_arrays/main}
\input{patterns/14_bitfields/main}
\EN{\input{patterns/145_LCG/main_EN}}
\RU{\input{patterns/145_LCG/main_RU}}
\input{patterns/15_structs/main}
\input{patterns/17_unions/main}
\input{patterns/18_pointers_to_functions/main}
\input{patterns/185_64bit_in_32_env/main}

\EN{\input{patterns/19_SIMD/main_EN}}
\RU{\input{patterns/19_SIMD/main_RU}}
\DE{\input{patterns/19_SIMD/main_DE}}

\EN{\input{patterns/20_x64/main_EN}}
\RU{\input{patterns/20_x64/main_RU}}

\EN{\input{patterns/205_floating_SIMD/main_EN}}
\RU{\input{patterns/205_floating_SIMD/main_RU}}
\DE{\input{patterns/205_floating_SIMD/main_DE}}

\EN{\input{patterns/ARM/main_EN}}
\RU{\input{patterns/ARM/main_RU}}
\DE{\input{patterns/ARM/main_DE}}

\input{patterns/MIPS/main}

\ifdefined\SPANISH
\chapter{Patrones de código}
\fi % SPANISH

\ifdefined\GERMAN
\chapter{Code-Muster}
\fi % GERMAN

\ifdefined\ENGLISH
\chapter{Code Patterns}
\fi % ENGLISH

\ifdefined\ITALIAN
\chapter{Forme di codice}
\fi % ITALIAN

\ifdefined\RUSSIAN
\chapter{Образцы кода}
\fi % RUSSIAN

\ifdefined\BRAZILIAN
\chapter{Padrões de códigos}
\fi % BRAZILIAN

\ifdefined\THAI
\chapter{รูปแบบของโค้ด}
\fi % THAI

\ifdefined\FRENCH
\chapter{Modèle de code}
\fi % FRENCH

\ifdefined\POLISH
\chapter{\PLph{}}
\fi % POLISH

% sections
\EN{\input{patterns/patterns_opt_dbg_EN}}
\ES{\input{patterns/patterns_opt_dbg_ES}}
\ITA{\input{patterns/patterns_opt_dbg_ITA}}
\PTBR{\input{patterns/patterns_opt_dbg_PTBR}}
\RU{\input{patterns/patterns_opt_dbg_RU}}
\THA{\input{patterns/patterns_opt_dbg_THA}}
\DE{\input{patterns/patterns_opt_dbg_DE}}
\FR{\input{patterns/patterns_opt_dbg_FR}}
\PL{\input{patterns/patterns_opt_dbg_PL}}

\RU{\section{Некоторые базовые понятия}}
\EN{\section{Some basics}}
\DE{\section{Einige Grundlagen}}
\FR{\section{Quelques bases}}
\ES{\section{\ESph{}}}
\ITA{\section{Alcune basi teoriche}}
\PTBR{\section{\PTBRph{}}}
\THA{\section{\THAph{}}}
\PL{\section{\PLph{}}}

% sections:
\EN{\input{patterns/intro_CPU_ISA_EN}}
\ES{\input{patterns/intro_CPU_ISA_ES}}
\ITA{\input{patterns/intro_CPU_ISA_ITA}}
\PTBR{\input{patterns/intro_CPU_ISA_PTBR}}
\RU{\input{patterns/intro_CPU_ISA_RU}}
\DE{\input{patterns/intro_CPU_ISA_DE}}
\FR{\input{patterns/intro_CPU_ISA_FR}}
\PL{\input{patterns/intro_CPU_ISA_PL}}

\EN{\input{patterns/numeral_EN}}
\RU{\input{patterns/numeral_RU}}
\ITA{\input{patterns/numeral_ITA}}
\DE{\input{patterns/numeral_DE}}
\FR{\input{patterns/numeral_FR}}
\PL{\input{patterns/numeral_PL}}

% chapters
\input{patterns/00_empty/main}
\input{patterns/011_ret/main}
\input{patterns/01_helloworld/main}
\input{patterns/015_prolog_epilogue/main}
\input{patterns/02_stack/main}
\input{patterns/03_printf/main}
\input{patterns/04_scanf/main}
\input{patterns/05_passing_arguments/main}
\input{patterns/06_return_results/main}
\input{patterns/061_pointers/main}
\input{patterns/065_GOTO/main}
\input{patterns/07_jcc/main}
\input{patterns/08_switch/main}
\input{patterns/09_loops/main}
\input{patterns/10_strings/main}
\input{patterns/11_arith_optimizations/main}
\input{patterns/12_FPU/main}
\input{patterns/13_arrays/main}
\input{patterns/14_bitfields/main}
\EN{\input{patterns/145_LCG/main_EN}}
\RU{\input{patterns/145_LCG/main_RU}}
\input{patterns/15_structs/main}
\input{patterns/17_unions/main}
\input{patterns/18_pointers_to_functions/main}
\input{patterns/185_64bit_in_32_env/main}

\EN{\input{patterns/19_SIMD/main_EN}}
\RU{\input{patterns/19_SIMD/main_RU}}
\DE{\input{patterns/19_SIMD/main_DE}}

\EN{\input{patterns/20_x64/main_EN}}
\RU{\input{patterns/20_x64/main_RU}}

\EN{\input{patterns/205_floating_SIMD/main_EN}}
\RU{\input{patterns/205_floating_SIMD/main_RU}}
\DE{\input{patterns/205_floating_SIMD/main_DE}}

\EN{\input{patterns/ARM/main_EN}}
\RU{\input{patterns/ARM/main_RU}}
\DE{\input{patterns/ARM/main_DE}}

\input{patterns/MIPS/main}

\ifdefined\SPANISH
\chapter{Patrones de código}
\fi % SPANISH

\ifdefined\GERMAN
\chapter{Code-Muster}
\fi % GERMAN

\ifdefined\ENGLISH
\chapter{Code Patterns}
\fi % ENGLISH

\ifdefined\ITALIAN
\chapter{Forme di codice}
\fi % ITALIAN

\ifdefined\RUSSIAN
\chapter{Образцы кода}
\fi % RUSSIAN

\ifdefined\BRAZILIAN
\chapter{Padrões de códigos}
\fi % BRAZILIAN

\ifdefined\THAI
\chapter{รูปแบบของโค้ด}
\fi % THAI

\ifdefined\FRENCH
\chapter{Modèle de code}
\fi % FRENCH

\ifdefined\POLISH
\chapter{\PLph{}}
\fi % POLISH

% sections
\EN{\input{patterns/patterns_opt_dbg_EN}}
\ES{\input{patterns/patterns_opt_dbg_ES}}
\ITA{\input{patterns/patterns_opt_dbg_ITA}}
\PTBR{\input{patterns/patterns_opt_dbg_PTBR}}
\RU{\input{patterns/patterns_opt_dbg_RU}}
\THA{\input{patterns/patterns_opt_dbg_THA}}
\DE{\input{patterns/patterns_opt_dbg_DE}}
\FR{\input{patterns/patterns_opt_dbg_FR}}
\PL{\input{patterns/patterns_opt_dbg_PL}}

\RU{\section{Некоторые базовые понятия}}
\EN{\section{Some basics}}
\DE{\section{Einige Grundlagen}}
\FR{\section{Quelques bases}}
\ES{\section{\ESph{}}}
\ITA{\section{Alcune basi teoriche}}
\PTBR{\section{\PTBRph{}}}
\THA{\section{\THAph{}}}
\PL{\section{\PLph{}}}

% sections:
\EN{\input{patterns/intro_CPU_ISA_EN}}
\ES{\input{patterns/intro_CPU_ISA_ES}}
\ITA{\input{patterns/intro_CPU_ISA_ITA}}
\PTBR{\input{patterns/intro_CPU_ISA_PTBR}}
\RU{\input{patterns/intro_CPU_ISA_RU}}
\DE{\input{patterns/intro_CPU_ISA_DE}}
\FR{\input{patterns/intro_CPU_ISA_FR}}
\PL{\input{patterns/intro_CPU_ISA_PL}}

\EN{\input{patterns/numeral_EN}}
\RU{\input{patterns/numeral_RU}}
\ITA{\input{patterns/numeral_ITA}}
\DE{\input{patterns/numeral_DE}}
\FR{\input{patterns/numeral_FR}}
\PL{\input{patterns/numeral_PL}}

% chapters
\input{patterns/00_empty/main}
\input{patterns/011_ret/main}
\input{patterns/01_helloworld/main}
\input{patterns/015_prolog_epilogue/main}
\input{patterns/02_stack/main}
\input{patterns/03_printf/main}
\input{patterns/04_scanf/main}
\input{patterns/05_passing_arguments/main}
\input{patterns/06_return_results/main}
\input{patterns/061_pointers/main}
\input{patterns/065_GOTO/main}
\input{patterns/07_jcc/main}
\input{patterns/08_switch/main}
\input{patterns/09_loops/main}
\input{patterns/10_strings/main}
\input{patterns/11_arith_optimizations/main}
\input{patterns/12_FPU/main}
\input{patterns/13_arrays/main}
\input{patterns/14_bitfields/main}
\EN{\input{patterns/145_LCG/main_EN}}
\RU{\input{patterns/145_LCG/main_RU}}
\input{patterns/15_structs/main}
\input{patterns/17_unions/main}
\input{patterns/18_pointers_to_functions/main}
\input{patterns/185_64bit_in_32_env/main}

\EN{\input{patterns/19_SIMD/main_EN}}
\RU{\input{patterns/19_SIMD/main_RU}}
\DE{\input{patterns/19_SIMD/main_DE}}

\EN{\input{patterns/20_x64/main_EN}}
\RU{\input{patterns/20_x64/main_RU}}

\EN{\input{patterns/205_floating_SIMD/main_EN}}
\RU{\input{patterns/205_floating_SIMD/main_RU}}
\DE{\input{patterns/205_floating_SIMD/main_DE}}

\EN{\input{patterns/ARM/main_EN}}
\RU{\input{patterns/ARM/main_RU}}
\DE{\input{patterns/ARM/main_DE}}

\input{patterns/MIPS/main}

\ifdefined\SPANISH
\chapter{Patrones de código}
\fi % SPANISH

\ifdefined\GERMAN
\chapter{Code-Muster}
\fi % GERMAN

\ifdefined\ENGLISH
\chapter{Code Patterns}
\fi % ENGLISH

\ifdefined\ITALIAN
\chapter{Forme di codice}
\fi % ITALIAN

\ifdefined\RUSSIAN
\chapter{Образцы кода}
\fi % RUSSIAN

\ifdefined\BRAZILIAN
\chapter{Padrões de códigos}
\fi % BRAZILIAN

\ifdefined\THAI
\chapter{รูปแบบของโค้ด}
\fi % THAI

\ifdefined\FRENCH
\chapter{Modèle de code}
\fi % FRENCH

\ifdefined\POLISH
\chapter{\PLph{}}
\fi % POLISH

% sections
\EN{\input{patterns/patterns_opt_dbg_EN}}
\ES{\input{patterns/patterns_opt_dbg_ES}}
\ITA{\input{patterns/patterns_opt_dbg_ITA}}
\PTBR{\input{patterns/patterns_opt_dbg_PTBR}}
\RU{\input{patterns/patterns_opt_dbg_RU}}
\THA{\input{patterns/patterns_opt_dbg_THA}}
\DE{\input{patterns/patterns_opt_dbg_DE}}
\FR{\input{patterns/patterns_opt_dbg_FR}}
\PL{\input{patterns/patterns_opt_dbg_PL}}

\RU{\section{Некоторые базовые понятия}}
\EN{\section{Some basics}}
\DE{\section{Einige Grundlagen}}
\FR{\section{Quelques bases}}
\ES{\section{\ESph{}}}
\ITA{\section{Alcune basi teoriche}}
\PTBR{\section{\PTBRph{}}}
\THA{\section{\THAph{}}}
\PL{\section{\PLph{}}}

% sections:
\EN{\input{patterns/intro_CPU_ISA_EN}}
\ES{\input{patterns/intro_CPU_ISA_ES}}
\ITA{\input{patterns/intro_CPU_ISA_ITA}}
\PTBR{\input{patterns/intro_CPU_ISA_PTBR}}
\RU{\input{patterns/intro_CPU_ISA_RU}}
\DE{\input{patterns/intro_CPU_ISA_DE}}
\FR{\input{patterns/intro_CPU_ISA_FR}}
\PL{\input{patterns/intro_CPU_ISA_PL}}

\EN{\input{patterns/numeral_EN}}
\RU{\input{patterns/numeral_RU}}
\ITA{\input{patterns/numeral_ITA}}
\DE{\input{patterns/numeral_DE}}
\FR{\input{patterns/numeral_FR}}
\PL{\input{patterns/numeral_PL}}

% chapters
\input{patterns/00_empty/main}
\input{patterns/011_ret/main}
\input{patterns/01_helloworld/main}
\input{patterns/015_prolog_epilogue/main}
\input{patterns/02_stack/main}
\input{patterns/03_printf/main}
\input{patterns/04_scanf/main}
\input{patterns/05_passing_arguments/main}
\input{patterns/06_return_results/main}
\input{patterns/061_pointers/main}
\input{patterns/065_GOTO/main}
\input{patterns/07_jcc/main}
\input{patterns/08_switch/main}
\input{patterns/09_loops/main}
\input{patterns/10_strings/main}
\input{patterns/11_arith_optimizations/main}
\input{patterns/12_FPU/main}
\input{patterns/13_arrays/main}
\input{patterns/14_bitfields/main}
\EN{\input{patterns/145_LCG/main_EN}}
\RU{\input{patterns/145_LCG/main_RU}}
\input{patterns/15_structs/main}
\input{patterns/17_unions/main}
\input{patterns/18_pointers_to_functions/main}
\input{patterns/185_64bit_in_32_env/main}

\EN{\input{patterns/19_SIMD/main_EN}}
\RU{\input{patterns/19_SIMD/main_RU}}
\DE{\input{patterns/19_SIMD/main_DE}}

\EN{\input{patterns/20_x64/main_EN}}
\RU{\input{patterns/20_x64/main_RU}}

\EN{\input{patterns/205_floating_SIMD/main_EN}}
\RU{\input{patterns/205_floating_SIMD/main_RU}}
\DE{\input{patterns/205_floating_SIMD/main_DE}}

\EN{\input{patterns/ARM/main_EN}}
\RU{\input{patterns/ARM/main_RU}}
\DE{\input{patterns/ARM/main_DE}}

\input{patterns/MIPS/main}

\ifdefined\SPANISH
\chapter{Patrones de código}
\fi % SPANISH

\ifdefined\GERMAN
\chapter{Code-Muster}
\fi % GERMAN

\ifdefined\ENGLISH
\chapter{Code Patterns}
\fi % ENGLISH

\ifdefined\ITALIAN
\chapter{Forme di codice}
\fi % ITALIAN

\ifdefined\RUSSIAN
\chapter{Образцы кода}
\fi % RUSSIAN

\ifdefined\BRAZILIAN
\chapter{Padrões de códigos}
\fi % BRAZILIAN

\ifdefined\THAI
\chapter{รูปแบบของโค้ด}
\fi % THAI

\ifdefined\FRENCH
\chapter{Modèle de code}
\fi % FRENCH

\ifdefined\POLISH
\chapter{\PLph{}}
\fi % POLISH

% sections
\EN{\input{patterns/patterns_opt_dbg_EN}}
\ES{\input{patterns/patterns_opt_dbg_ES}}
\ITA{\input{patterns/patterns_opt_dbg_ITA}}
\PTBR{\input{patterns/patterns_opt_dbg_PTBR}}
\RU{\input{patterns/patterns_opt_dbg_RU}}
\THA{\input{patterns/patterns_opt_dbg_THA}}
\DE{\input{patterns/patterns_opt_dbg_DE}}
\FR{\input{patterns/patterns_opt_dbg_FR}}
\PL{\input{patterns/patterns_opt_dbg_PL}}

\RU{\section{Некоторые базовые понятия}}
\EN{\section{Some basics}}
\DE{\section{Einige Grundlagen}}
\FR{\section{Quelques bases}}
\ES{\section{\ESph{}}}
\ITA{\section{Alcune basi teoriche}}
\PTBR{\section{\PTBRph{}}}
\THA{\section{\THAph{}}}
\PL{\section{\PLph{}}}

% sections:
\EN{\input{patterns/intro_CPU_ISA_EN}}
\ES{\input{patterns/intro_CPU_ISA_ES}}
\ITA{\input{patterns/intro_CPU_ISA_ITA}}
\PTBR{\input{patterns/intro_CPU_ISA_PTBR}}
\RU{\input{patterns/intro_CPU_ISA_RU}}
\DE{\input{patterns/intro_CPU_ISA_DE}}
\FR{\input{patterns/intro_CPU_ISA_FR}}
\PL{\input{patterns/intro_CPU_ISA_PL}}

\EN{\input{patterns/numeral_EN}}
\RU{\input{patterns/numeral_RU}}
\ITA{\input{patterns/numeral_ITA}}
\DE{\input{patterns/numeral_DE}}
\FR{\input{patterns/numeral_FR}}
\PL{\input{patterns/numeral_PL}}

% chapters
\input{patterns/00_empty/main}
\input{patterns/011_ret/main}
\input{patterns/01_helloworld/main}
\input{patterns/015_prolog_epilogue/main}
\input{patterns/02_stack/main}
\input{patterns/03_printf/main}
\input{patterns/04_scanf/main}
\input{patterns/05_passing_arguments/main}
\input{patterns/06_return_results/main}
\input{patterns/061_pointers/main}
\input{patterns/065_GOTO/main}
\input{patterns/07_jcc/main}
\input{patterns/08_switch/main}
\input{patterns/09_loops/main}
\input{patterns/10_strings/main}
\input{patterns/11_arith_optimizations/main}
\input{patterns/12_FPU/main}
\input{patterns/13_arrays/main}
\input{patterns/14_bitfields/main}
\EN{\input{patterns/145_LCG/main_EN}}
\RU{\input{patterns/145_LCG/main_RU}}
\input{patterns/15_structs/main}
\input{patterns/17_unions/main}
\input{patterns/18_pointers_to_functions/main}
\input{patterns/185_64bit_in_32_env/main}

\EN{\input{patterns/19_SIMD/main_EN}}
\RU{\input{patterns/19_SIMD/main_RU}}
\DE{\input{patterns/19_SIMD/main_DE}}

\EN{\input{patterns/20_x64/main_EN}}
\RU{\input{patterns/20_x64/main_RU}}

\EN{\input{patterns/205_floating_SIMD/main_EN}}
\RU{\input{patterns/205_floating_SIMD/main_RU}}
\DE{\input{patterns/205_floating_SIMD/main_DE}}

\EN{\input{patterns/ARM/main_EN}}
\RU{\input{patterns/ARM/main_RU}}
\DE{\input{patterns/ARM/main_DE}}

\input{patterns/MIPS/main}

\ifdefined\SPANISH
\chapter{Patrones de código}
\fi % SPANISH

\ifdefined\GERMAN
\chapter{Code-Muster}
\fi % GERMAN

\ifdefined\ENGLISH
\chapter{Code Patterns}
\fi % ENGLISH

\ifdefined\ITALIAN
\chapter{Forme di codice}
\fi % ITALIAN

\ifdefined\RUSSIAN
\chapter{Образцы кода}
\fi % RUSSIAN

\ifdefined\BRAZILIAN
\chapter{Padrões de códigos}
\fi % BRAZILIAN

\ifdefined\THAI
\chapter{รูปแบบของโค้ด}
\fi % THAI

\ifdefined\FRENCH
\chapter{Modèle de code}
\fi % FRENCH

\ifdefined\POLISH
\chapter{\PLph{}}
\fi % POLISH

% sections
\EN{\input{patterns/patterns_opt_dbg_EN}}
\ES{\input{patterns/patterns_opt_dbg_ES}}
\ITA{\input{patterns/patterns_opt_dbg_ITA}}
\PTBR{\input{patterns/patterns_opt_dbg_PTBR}}
\RU{\input{patterns/patterns_opt_dbg_RU}}
\THA{\input{patterns/patterns_opt_dbg_THA}}
\DE{\input{patterns/patterns_opt_dbg_DE}}
\FR{\input{patterns/patterns_opt_dbg_FR}}
\PL{\input{patterns/patterns_opt_dbg_PL}}

\RU{\section{Некоторые базовые понятия}}
\EN{\section{Some basics}}
\DE{\section{Einige Grundlagen}}
\FR{\section{Quelques bases}}
\ES{\section{\ESph{}}}
\ITA{\section{Alcune basi teoriche}}
\PTBR{\section{\PTBRph{}}}
\THA{\section{\THAph{}}}
\PL{\section{\PLph{}}}

% sections:
\EN{\input{patterns/intro_CPU_ISA_EN}}
\ES{\input{patterns/intro_CPU_ISA_ES}}
\ITA{\input{patterns/intro_CPU_ISA_ITA}}
\PTBR{\input{patterns/intro_CPU_ISA_PTBR}}
\RU{\input{patterns/intro_CPU_ISA_RU}}
\DE{\input{patterns/intro_CPU_ISA_DE}}
\FR{\input{patterns/intro_CPU_ISA_FR}}
\PL{\input{patterns/intro_CPU_ISA_PL}}

\EN{\input{patterns/numeral_EN}}
\RU{\input{patterns/numeral_RU}}
\ITA{\input{patterns/numeral_ITA}}
\DE{\input{patterns/numeral_DE}}
\FR{\input{patterns/numeral_FR}}
\PL{\input{patterns/numeral_PL}}

% chapters
\input{patterns/00_empty/main}
\input{patterns/011_ret/main}
\input{patterns/01_helloworld/main}
\input{patterns/015_prolog_epilogue/main}
\input{patterns/02_stack/main}
\input{patterns/03_printf/main}
\input{patterns/04_scanf/main}
\input{patterns/05_passing_arguments/main}
\input{patterns/06_return_results/main}
\input{patterns/061_pointers/main}
\input{patterns/065_GOTO/main}
\input{patterns/07_jcc/main}
\input{patterns/08_switch/main}
\input{patterns/09_loops/main}
\input{patterns/10_strings/main}
\input{patterns/11_arith_optimizations/main}
\input{patterns/12_FPU/main}
\input{patterns/13_arrays/main}
\input{patterns/14_bitfields/main}
\EN{\input{patterns/145_LCG/main_EN}}
\RU{\input{patterns/145_LCG/main_RU}}
\input{patterns/15_structs/main}
\input{patterns/17_unions/main}
\input{patterns/18_pointers_to_functions/main}
\input{patterns/185_64bit_in_32_env/main}

\EN{\input{patterns/19_SIMD/main_EN}}
\RU{\input{patterns/19_SIMD/main_RU}}
\DE{\input{patterns/19_SIMD/main_DE}}

\EN{\input{patterns/20_x64/main_EN}}
\RU{\input{patterns/20_x64/main_RU}}

\EN{\input{patterns/205_floating_SIMD/main_EN}}
\RU{\input{patterns/205_floating_SIMD/main_RU}}
\DE{\input{patterns/205_floating_SIMD/main_DE}}

\EN{\input{patterns/ARM/main_EN}}
\RU{\input{patterns/ARM/main_RU}}
\DE{\input{patterns/ARM/main_DE}}

\input{patterns/MIPS/main}

\ifdefined\SPANISH
\chapter{Patrones de código}
\fi % SPANISH

\ifdefined\GERMAN
\chapter{Code-Muster}
\fi % GERMAN

\ifdefined\ENGLISH
\chapter{Code Patterns}
\fi % ENGLISH

\ifdefined\ITALIAN
\chapter{Forme di codice}
\fi % ITALIAN

\ifdefined\RUSSIAN
\chapter{Образцы кода}
\fi % RUSSIAN

\ifdefined\BRAZILIAN
\chapter{Padrões de códigos}
\fi % BRAZILIAN

\ifdefined\THAI
\chapter{รูปแบบของโค้ด}
\fi % THAI

\ifdefined\FRENCH
\chapter{Modèle de code}
\fi % FRENCH

\ifdefined\POLISH
\chapter{\PLph{}}
\fi % POLISH

% sections
\EN{\input{patterns/patterns_opt_dbg_EN}}
\ES{\input{patterns/patterns_opt_dbg_ES}}
\ITA{\input{patterns/patterns_opt_dbg_ITA}}
\PTBR{\input{patterns/patterns_opt_dbg_PTBR}}
\RU{\input{patterns/patterns_opt_dbg_RU}}
\THA{\input{patterns/patterns_opt_dbg_THA}}
\DE{\input{patterns/patterns_opt_dbg_DE}}
\FR{\input{patterns/patterns_opt_dbg_FR}}
\PL{\input{patterns/patterns_opt_dbg_PL}}

\RU{\section{Некоторые базовые понятия}}
\EN{\section{Some basics}}
\DE{\section{Einige Grundlagen}}
\FR{\section{Quelques bases}}
\ES{\section{\ESph{}}}
\ITA{\section{Alcune basi teoriche}}
\PTBR{\section{\PTBRph{}}}
\THA{\section{\THAph{}}}
\PL{\section{\PLph{}}}

% sections:
\EN{\input{patterns/intro_CPU_ISA_EN}}
\ES{\input{patterns/intro_CPU_ISA_ES}}
\ITA{\input{patterns/intro_CPU_ISA_ITA}}
\PTBR{\input{patterns/intro_CPU_ISA_PTBR}}
\RU{\input{patterns/intro_CPU_ISA_RU}}
\DE{\input{patterns/intro_CPU_ISA_DE}}
\FR{\input{patterns/intro_CPU_ISA_FR}}
\PL{\input{patterns/intro_CPU_ISA_PL}}

\EN{\input{patterns/numeral_EN}}
\RU{\input{patterns/numeral_RU}}
\ITA{\input{patterns/numeral_ITA}}
\DE{\input{patterns/numeral_DE}}
\FR{\input{patterns/numeral_FR}}
\PL{\input{patterns/numeral_PL}}

% chapters
\input{patterns/00_empty/main}
\input{patterns/011_ret/main}
\input{patterns/01_helloworld/main}
\input{patterns/015_prolog_epilogue/main}
\input{patterns/02_stack/main}
\input{patterns/03_printf/main}
\input{patterns/04_scanf/main}
\input{patterns/05_passing_arguments/main}
\input{patterns/06_return_results/main}
\input{patterns/061_pointers/main}
\input{patterns/065_GOTO/main}
\input{patterns/07_jcc/main}
\input{patterns/08_switch/main}
\input{patterns/09_loops/main}
\input{patterns/10_strings/main}
\input{patterns/11_arith_optimizations/main}
\input{patterns/12_FPU/main}
\input{patterns/13_arrays/main}
\input{patterns/14_bitfields/main}
\EN{\input{patterns/145_LCG/main_EN}}
\RU{\input{patterns/145_LCG/main_RU}}
\input{patterns/15_structs/main}
\input{patterns/17_unions/main}
\input{patterns/18_pointers_to_functions/main}
\input{patterns/185_64bit_in_32_env/main}

\EN{\input{patterns/19_SIMD/main_EN}}
\RU{\input{patterns/19_SIMD/main_RU}}
\DE{\input{patterns/19_SIMD/main_DE}}

\EN{\input{patterns/20_x64/main_EN}}
\RU{\input{patterns/20_x64/main_RU}}

\EN{\input{patterns/205_floating_SIMD/main_EN}}
\RU{\input{patterns/205_floating_SIMD/main_RU}}
\DE{\input{patterns/205_floating_SIMD/main_DE}}

\EN{\input{patterns/ARM/main_EN}}
\RU{\input{patterns/ARM/main_RU}}
\DE{\input{patterns/ARM/main_DE}}

\input{patterns/MIPS/main}

\ifdefined\SPANISH
\chapter{Patrones de código}
\fi % SPANISH

\ifdefined\GERMAN
\chapter{Code-Muster}
\fi % GERMAN

\ifdefined\ENGLISH
\chapter{Code Patterns}
\fi % ENGLISH

\ifdefined\ITALIAN
\chapter{Forme di codice}
\fi % ITALIAN

\ifdefined\RUSSIAN
\chapter{Образцы кода}
\fi % RUSSIAN

\ifdefined\BRAZILIAN
\chapter{Padrões de códigos}
\fi % BRAZILIAN

\ifdefined\THAI
\chapter{รูปแบบของโค้ด}
\fi % THAI

\ifdefined\FRENCH
\chapter{Modèle de code}
\fi % FRENCH

\ifdefined\POLISH
\chapter{\PLph{}}
\fi % POLISH

% sections
\EN{\input{patterns/patterns_opt_dbg_EN}}
\ES{\input{patterns/patterns_opt_dbg_ES}}
\ITA{\input{patterns/patterns_opt_dbg_ITA}}
\PTBR{\input{patterns/patterns_opt_dbg_PTBR}}
\RU{\input{patterns/patterns_opt_dbg_RU}}
\THA{\input{patterns/patterns_opt_dbg_THA}}
\DE{\input{patterns/patterns_opt_dbg_DE}}
\FR{\input{patterns/patterns_opt_dbg_FR}}
\PL{\input{patterns/patterns_opt_dbg_PL}}

\RU{\section{Некоторые базовые понятия}}
\EN{\section{Some basics}}
\DE{\section{Einige Grundlagen}}
\FR{\section{Quelques bases}}
\ES{\section{\ESph{}}}
\ITA{\section{Alcune basi teoriche}}
\PTBR{\section{\PTBRph{}}}
\THA{\section{\THAph{}}}
\PL{\section{\PLph{}}}

% sections:
\EN{\input{patterns/intro_CPU_ISA_EN}}
\ES{\input{patterns/intro_CPU_ISA_ES}}
\ITA{\input{patterns/intro_CPU_ISA_ITA}}
\PTBR{\input{patterns/intro_CPU_ISA_PTBR}}
\RU{\input{patterns/intro_CPU_ISA_RU}}
\DE{\input{patterns/intro_CPU_ISA_DE}}
\FR{\input{patterns/intro_CPU_ISA_FR}}
\PL{\input{patterns/intro_CPU_ISA_PL}}

\EN{\input{patterns/numeral_EN}}
\RU{\input{patterns/numeral_RU}}
\ITA{\input{patterns/numeral_ITA}}
\DE{\input{patterns/numeral_DE}}
\FR{\input{patterns/numeral_FR}}
\PL{\input{patterns/numeral_PL}}

% chapters
\input{patterns/00_empty/main}
\input{patterns/011_ret/main}
\input{patterns/01_helloworld/main}
\input{patterns/015_prolog_epilogue/main}
\input{patterns/02_stack/main}
\input{patterns/03_printf/main}
\input{patterns/04_scanf/main}
\input{patterns/05_passing_arguments/main}
\input{patterns/06_return_results/main}
\input{patterns/061_pointers/main}
\input{patterns/065_GOTO/main}
\input{patterns/07_jcc/main}
\input{patterns/08_switch/main}
\input{patterns/09_loops/main}
\input{patterns/10_strings/main}
\input{patterns/11_arith_optimizations/main}
\input{patterns/12_FPU/main}
\input{patterns/13_arrays/main}
\input{patterns/14_bitfields/main}
\EN{\input{patterns/145_LCG/main_EN}}
\RU{\input{patterns/145_LCG/main_RU}}
\input{patterns/15_structs/main}
\input{patterns/17_unions/main}
\input{patterns/18_pointers_to_functions/main}
\input{patterns/185_64bit_in_32_env/main}

\EN{\input{patterns/19_SIMD/main_EN}}
\RU{\input{patterns/19_SIMD/main_RU}}
\DE{\input{patterns/19_SIMD/main_DE}}

\EN{\input{patterns/20_x64/main_EN}}
\RU{\input{patterns/20_x64/main_RU}}

\EN{\input{patterns/205_floating_SIMD/main_EN}}
\RU{\input{patterns/205_floating_SIMD/main_RU}}
\DE{\input{patterns/205_floating_SIMD/main_DE}}

\EN{\input{patterns/ARM/main_EN}}
\RU{\input{patterns/ARM/main_RU}}
\DE{\input{patterns/ARM/main_DE}}

\input{patterns/MIPS/main}

\ifdefined\SPANISH
\chapter{Patrones de código}
\fi % SPANISH

\ifdefined\GERMAN
\chapter{Code-Muster}
\fi % GERMAN

\ifdefined\ENGLISH
\chapter{Code Patterns}
\fi % ENGLISH

\ifdefined\ITALIAN
\chapter{Forme di codice}
\fi % ITALIAN

\ifdefined\RUSSIAN
\chapter{Образцы кода}
\fi % RUSSIAN

\ifdefined\BRAZILIAN
\chapter{Padrões de códigos}
\fi % BRAZILIAN

\ifdefined\THAI
\chapter{รูปแบบของโค้ด}
\fi % THAI

\ifdefined\FRENCH
\chapter{Modèle de code}
\fi % FRENCH

\ifdefined\POLISH
\chapter{\PLph{}}
\fi % POLISH

% sections
\EN{\input{patterns/patterns_opt_dbg_EN}}
\ES{\input{patterns/patterns_opt_dbg_ES}}
\ITA{\input{patterns/patterns_opt_dbg_ITA}}
\PTBR{\input{patterns/patterns_opt_dbg_PTBR}}
\RU{\input{patterns/patterns_opt_dbg_RU}}
\THA{\input{patterns/patterns_opt_dbg_THA}}
\DE{\input{patterns/patterns_opt_dbg_DE}}
\FR{\input{patterns/patterns_opt_dbg_FR}}
\PL{\input{patterns/patterns_opt_dbg_PL}}

\RU{\section{Некоторые базовые понятия}}
\EN{\section{Some basics}}
\DE{\section{Einige Grundlagen}}
\FR{\section{Quelques bases}}
\ES{\section{\ESph{}}}
\ITA{\section{Alcune basi teoriche}}
\PTBR{\section{\PTBRph{}}}
\THA{\section{\THAph{}}}
\PL{\section{\PLph{}}}

% sections:
\EN{\input{patterns/intro_CPU_ISA_EN}}
\ES{\input{patterns/intro_CPU_ISA_ES}}
\ITA{\input{patterns/intro_CPU_ISA_ITA}}
\PTBR{\input{patterns/intro_CPU_ISA_PTBR}}
\RU{\input{patterns/intro_CPU_ISA_RU}}
\DE{\input{patterns/intro_CPU_ISA_DE}}
\FR{\input{patterns/intro_CPU_ISA_FR}}
\PL{\input{patterns/intro_CPU_ISA_PL}}

\EN{\input{patterns/numeral_EN}}
\RU{\input{patterns/numeral_RU}}
\ITA{\input{patterns/numeral_ITA}}
\DE{\input{patterns/numeral_DE}}
\FR{\input{patterns/numeral_FR}}
\PL{\input{patterns/numeral_PL}}

% chapters
\input{patterns/00_empty/main}
\input{patterns/011_ret/main}
\input{patterns/01_helloworld/main}
\input{patterns/015_prolog_epilogue/main}
\input{patterns/02_stack/main}
\input{patterns/03_printf/main}
\input{patterns/04_scanf/main}
\input{patterns/05_passing_arguments/main}
\input{patterns/06_return_results/main}
\input{patterns/061_pointers/main}
\input{patterns/065_GOTO/main}
\input{patterns/07_jcc/main}
\input{patterns/08_switch/main}
\input{patterns/09_loops/main}
\input{patterns/10_strings/main}
\input{patterns/11_arith_optimizations/main}
\input{patterns/12_FPU/main}
\input{patterns/13_arrays/main}
\input{patterns/14_bitfields/main}
\EN{\input{patterns/145_LCG/main_EN}}
\RU{\input{patterns/145_LCG/main_RU}}
\input{patterns/15_structs/main}
\input{patterns/17_unions/main}
\input{patterns/18_pointers_to_functions/main}
\input{patterns/185_64bit_in_32_env/main}

\EN{\input{patterns/19_SIMD/main_EN}}
\RU{\input{patterns/19_SIMD/main_RU}}
\DE{\input{patterns/19_SIMD/main_DE}}

\EN{\input{patterns/20_x64/main_EN}}
\RU{\input{patterns/20_x64/main_RU}}

\EN{\input{patterns/205_floating_SIMD/main_EN}}
\RU{\input{patterns/205_floating_SIMD/main_RU}}
\DE{\input{patterns/205_floating_SIMD/main_DE}}

\EN{\input{patterns/ARM/main_EN}}
\RU{\input{patterns/ARM/main_RU}}
\DE{\input{patterns/ARM/main_DE}}

\input{patterns/MIPS/main}

\ifdefined\SPANISH
\chapter{Patrones de código}
\fi % SPANISH

\ifdefined\GERMAN
\chapter{Code-Muster}
\fi % GERMAN

\ifdefined\ENGLISH
\chapter{Code Patterns}
\fi % ENGLISH

\ifdefined\ITALIAN
\chapter{Forme di codice}
\fi % ITALIAN

\ifdefined\RUSSIAN
\chapter{Образцы кода}
\fi % RUSSIAN

\ifdefined\BRAZILIAN
\chapter{Padrões de códigos}
\fi % BRAZILIAN

\ifdefined\THAI
\chapter{รูปแบบของโค้ด}
\fi % THAI

\ifdefined\FRENCH
\chapter{Modèle de code}
\fi % FRENCH

\ifdefined\POLISH
\chapter{\PLph{}}
\fi % POLISH

% sections
\EN{\input{patterns/patterns_opt_dbg_EN}}
\ES{\input{patterns/patterns_opt_dbg_ES}}
\ITA{\input{patterns/patterns_opt_dbg_ITA}}
\PTBR{\input{patterns/patterns_opt_dbg_PTBR}}
\RU{\input{patterns/patterns_opt_dbg_RU}}
\THA{\input{patterns/patterns_opt_dbg_THA}}
\DE{\input{patterns/patterns_opt_dbg_DE}}
\FR{\input{patterns/patterns_opt_dbg_FR}}
\PL{\input{patterns/patterns_opt_dbg_PL}}

\RU{\section{Некоторые базовые понятия}}
\EN{\section{Some basics}}
\DE{\section{Einige Grundlagen}}
\FR{\section{Quelques bases}}
\ES{\section{\ESph{}}}
\ITA{\section{Alcune basi teoriche}}
\PTBR{\section{\PTBRph{}}}
\THA{\section{\THAph{}}}
\PL{\section{\PLph{}}}

% sections:
\EN{\input{patterns/intro_CPU_ISA_EN}}
\ES{\input{patterns/intro_CPU_ISA_ES}}
\ITA{\input{patterns/intro_CPU_ISA_ITA}}
\PTBR{\input{patterns/intro_CPU_ISA_PTBR}}
\RU{\input{patterns/intro_CPU_ISA_RU}}
\DE{\input{patterns/intro_CPU_ISA_DE}}
\FR{\input{patterns/intro_CPU_ISA_FR}}
\PL{\input{patterns/intro_CPU_ISA_PL}}

\EN{\input{patterns/numeral_EN}}
\RU{\input{patterns/numeral_RU}}
\ITA{\input{patterns/numeral_ITA}}
\DE{\input{patterns/numeral_DE}}
\FR{\input{patterns/numeral_FR}}
\PL{\input{patterns/numeral_PL}}

% chapters
\input{patterns/00_empty/main}
\input{patterns/011_ret/main}
\input{patterns/01_helloworld/main}
\input{patterns/015_prolog_epilogue/main}
\input{patterns/02_stack/main}
\input{patterns/03_printf/main}
\input{patterns/04_scanf/main}
\input{patterns/05_passing_arguments/main}
\input{patterns/06_return_results/main}
\input{patterns/061_pointers/main}
\input{patterns/065_GOTO/main}
\input{patterns/07_jcc/main}
\input{patterns/08_switch/main}
\input{patterns/09_loops/main}
\input{patterns/10_strings/main}
\input{patterns/11_arith_optimizations/main}
\input{patterns/12_FPU/main}
\input{patterns/13_arrays/main}
\input{patterns/14_bitfields/main}
\EN{\input{patterns/145_LCG/main_EN}}
\RU{\input{patterns/145_LCG/main_RU}}
\input{patterns/15_structs/main}
\input{patterns/17_unions/main}
\input{patterns/18_pointers_to_functions/main}
\input{patterns/185_64bit_in_32_env/main}

\EN{\input{patterns/19_SIMD/main_EN}}
\RU{\input{patterns/19_SIMD/main_RU}}
\DE{\input{patterns/19_SIMD/main_DE}}

\EN{\input{patterns/20_x64/main_EN}}
\RU{\input{patterns/20_x64/main_RU}}

\EN{\input{patterns/205_floating_SIMD/main_EN}}
\RU{\input{patterns/205_floating_SIMD/main_RU}}
\DE{\input{patterns/205_floating_SIMD/main_DE}}

\EN{\input{patterns/ARM/main_EN}}
\RU{\input{patterns/ARM/main_RU}}
\DE{\input{patterns/ARM/main_DE}}

\input{patterns/MIPS/main}

\EN{\input{patterns/12_FPU/main_EN}}
\RU{\input{patterns/12_FPU/main_RU}}
\DE{\input{patterns/12_FPU/main_DE}}
\FR{\input{patterns/12_FPU/main_FR}}


\ifdefined\SPANISH
\chapter{Patrones de código}
\fi % SPANISH

\ifdefined\GERMAN
\chapter{Code-Muster}
\fi % GERMAN

\ifdefined\ENGLISH
\chapter{Code Patterns}
\fi % ENGLISH

\ifdefined\ITALIAN
\chapter{Forme di codice}
\fi % ITALIAN

\ifdefined\RUSSIAN
\chapter{Образцы кода}
\fi % RUSSIAN

\ifdefined\BRAZILIAN
\chapter{Padrões de códigos}
\fi % BRAZILIAN

\ifdefined\THAI
\chapter{รูปแบบของโค้ด}
\fi % THAI

\ifdefined\FRENCH
\chapter{Modèle de code}
\fi % FRENCH

\ifdefined\POLISH
\chapter{\PLph{}}
\fi % POLISH

% sections
\EN{\input{patterns/patterns_opt_dbg_EN}}
\ES{\input{patterns/patterns_opt_dbg_ES}}
\ITA{\input{patterns/patterns_opt_dbg_ITA}}
\PTBR{\input{patterns/patterns_opt_dbg_PTBR}}
\RU{\input{patterns/patterns_opt_dbg_RU}}
\THA{\input{patterns/patterns_opt_dbg_THA}}
\DE{\input{patterns/patterns_opt_dbg_DE}}
\FR{\input{patterns/patterns_opt_dbg_FR}}
\PL{\input{patterns/patterns_opt_dbg_PL}}

\RU{\section{Некоторые базовые понятия}}
\EN{\section{Some basics}}
\DE{\section{Einige Grundlagen}}
\FR{\section{Quelques bases}}
\ES{\section{\ESph{}}}
\ITA{\section{Alcune basi teoriche}}
\PTBR{\section{\PTBRph{}}}
\THA{\section{\THAph{}}}
\PL{\section{\PLph{}}}

% sections:
\EN{\input{patterns/intro_CPU_ISA_EN}}
\ES{\input{patterns/intro_CPU_ISA_ES}}
\ITA{\input{patterns/intro_CPU_ISA_ITA}}
\PTBR{\input{patterns/intro_CPU_ISA_PTBR}}
\RU{\input{patterns/intro_CPU_ISA_RU}}
\DE{\input{patterns/intro_CPU_ISA_DE}}
\FR{\input{patterns/intro_CPU_ISA_FR}}
\PL{\input{patterns/intro_CPU_ISA_PL}}

\EN{\input{patterns/numeral_EN}}
\RU{\input{patterns/numeral_RU}}
\ITA{\input{patterns/numeral_ITA}}
\DE{\input{patterns/numeral_DE}}
\FR{\input{patterns/numeral_FR}}
\PL{\input{patterns/numeral_PL}}

% chapters
\input{patterns/00_empty/main}
\input{patterns/011_ret/main}
\input{patterns/01_helloworld/main}
\input{patterns/015_prolog_epilogue/main}
\input{patterns/02_stack/main}
\input{patterns/03_printf/main}
\input{patterns/04_scanf/main}
\input{patterns/05_passing_arguments/main}
\input{patterns/06_return_results/main}
\input{patterns/061_pointers/main}
\input{patterns/065_GOTO/main}
\input{patterns/07_jcc/main}
\input{patterns/08_switch/main}
\input{patterns/09_loops/main}
\input{patterns/10_strings/main}
\input{patterns/11_arith_optimizations/main}
\input{patterns/12_FPU/main}
\input{patterns/13_arrays/main}
\input{patterns/14_bitfields/main}
\EN{\input{patterns/145_LCG/main_EN}}
\RU{\input{patterns/145_LCG/main_RU}}
\input{patterns/15_structs/main}
\input{patterns/17_unions/main}
\input{patterns/18_pointers_to_functions/main}
\input{patterns/185_64bit_in_32_env/main}

\EN{\input{patterns/19_SIMD/main_EN}}
\RU{\input{patterns/19_SIMD/main_RU}}
\DE{\input{patterns/19_SIMD/main_DE}}

\EN{\input{patterns/20_x64/main_EN}}
\RU{\input{patterns/20_x64/main_RU}}

\EN{\input{patterns/205_floating_SIMD/main_EN}}
\RU{\input{patterns/205_floating_SIMD/main_RU}}
\DE{\input{patterns/205_floating_SIMD/main_DE}}

\EN{\input{patterns/ARM/main_EN}}
\RU{\input{patterns/ARM/main_RU}}
\DE{\input{patterns/ARM/main_DE}}

\input{patterns/MIPS/main}

\ifdefined\SPANISH
\chapter{Patrones de código}
\fi % SPANISH

\ifdefined\GERMAN
\chapter{Code-Muster}
\fi % GERMAN

\ifdefined\ENGLISH
\chapter{Code Patterns}
\fi % ENGLISH

\ifdefined\ITALIAN
\chapter{Forme di codice}
\fi % ITALIAN

\ifdefined\RUSSIAN
\chapter{Образцы кода}
\fi % RUSSIAN

\ifdefined\BRAZILIAN
\chapter{Padrões de códigos}
\fi % BRAZILIAN

\ifdefined\THAI
\chapter{รูปแบบของโค้ด}
\fi % THAI

\ifdefined\FRENCH
\chapter{Modèle de code}
\fi % FRENCH

\ifdefined\POLISH
\chapter{\PLph{}}
\fi % POLISH

% sections
\EN{\input{patterns/patterns_opt_dbg_EN}}
\ES{\input{patterns/patterns_opt_dbg_ES}}
\ITA{\input{patterns/patterns_opt_dbg_ITA}}
\PTBR{\input{patterns/patterns_opt_dbg_PTBR}}
\RU{\input{patterns/patterns_opt_dbg_RU}}
\THA{\input{patterns/patterns_opt_dbg_THA}}
\DE{\input{patterns/patterns_opt_dbg_DE}}
\FR{\input{patterns/patterns_opt_dbg_FR}}
\PL{\input{patterns/patterns_opt_dbg_PL}}

\RU{\section{Некоторые базовые понятия}}
\EN{\section{Some basics}}
\DE{\section{Einige Grundlagen}}
\FR{\section{Quelques bases}}
\ES{\section{\ESph{}}}
\ITA{\section{Alcune basi teoriche}}
\PTBR{\section{\PTBRph{}}}
\THA{\section{\THAph{}}}
\PL{\section{\PLph{}}}

% sections:
\EN{\input{patterns/intro_CPU_ISA_EN}}
\ES{\input{patterns/intro_CPU_ISA_ES}}
\ITA{\input{patterns/intro_CPU_ISA_ITA}}
\PTBR{\input{patterns/intro_CPU_ISA_PTBR}}
\RU{\input{patterns/intro_CPU_ISA_RU}}
\DE{\input{patterns/intro_CPU_ISA_DE}}
\FR{\input{patterns/intro_CPU_ISA_FR}}
\PL{\input{patterns/intro_CPU_ISA_PL}}

\EN{\input{patterns/numeral_EN}}
\RU{\input{patterns/numeral_RU}}
\ITA{\input{patterns/numeral_ITA}}
\DE{\input{patterns/numeral_DE}}
\FR{\input{patterns/numeral_FR}}
\PL{\input{patterns/numeral_PL}}

% chapters
\input{patterns/00_empty/main}
\input{patterns/011_ret/main}
\input{patterns/01_helloworld/main}
\input{patterns/015_prolog_epilogue/main}
\input{patterns/02_stack/main}
\input{patterns/03_printf/main}
\input{patterns/04_scanf/main}
\input{patterns/05_passing_arguments/main}
\input{patterns/06_return_results/main}
\input{patterns/061_pointers/main}
\input{patterns/065_GOTO/main}
\input{patterns/07_jcc/main}
\input{patterns/08_switch/main}
\input{patterns/09_loops/main}
\input{patterns/10_strings/main}
\input{patterns/11_arith_optimizations/main}
\input{patterns/12_FPU/main}
\input{patterns/13_arrays/main}
\input{patterns/14_bitfields/main}
\EN{\input{patterns/145_LCG/main_EN}}
\RU{\input{patterns/145_LCG/main_RU}}
\input{patterns/15_structs/main}
\input{patterns/17_unions/main}
\input{patterns/18_pointers_to_functions/main}
\input{patterns/185_64bit_in_32_env/main}

\EN{\input{patterns/19_SIMD/main_EN}}
\RU{\input{patterns/19_SIMD/main_RU}}
\DE{\input{patterns/19_SIMD/main_DE}}

\EN{\input{patterns/20_x64/main_EN}}
\RU{\input{patterns/20_x64/main_RU}}

\EN{\input{patterns/205_floating_SIMD/main_EN}}
\RU{\input{patterns/205_floating_SIMD/main_RU}}
\DE{\input{patterns/205_floating_SIMD/main_DE}}

\EN{\input{patterns/ARM/main_EN}}
\RU{\input{patterns/ARM/main_RU}}
\DE{\input{patterns/ARM/main_DE}}

\input{patterns/MIPS/main}

\EN{\section{Returning Values}
\label{ret_val_func}

Another simple function is the one that simply returns a constant value:

\lstinputlisting[caption=\EN{\CCpp Code},style=customc]{patterns/011_ret/1.c}

Let's compile it.

\subsection{x86}

Here's what both the GCC and MSVC compilers produce (with optimization) on the x86 platform:

\lstinputlisting[caption=\Optimizing GCC/MSVC (\assemblyOutput),style=customasmx86]{patterns/011_ret/1.s}

\myindex{x86!\Instructions!RET}
There are just two instructions: the first places the value 123 into the \EAX register,
which is used by convention for storing the return
value, and the second one is \RET, which returns execution to the \gls{caller}.

The caller will take the result from the \EAX register.

\subsection{ARM}

There are a few differences on the ARM platform:

\lstinputlisting[caption=\OptimizingKeilVI (\ARMMode) ASM Output,style=customasmARM]{patterns/011_ret/1_Keil_ARM_O3.s}

ARM uses the register \Reg{0} for returning the results of functions, so 123 is copied into \Reg{0}.

\myindex{ARM!\Instructions!MOV}
\myindex{x86!\Instructions!MOV}
It is worth noting that \MOV is a misleading name for the instruction in both the x86 and ARM \ac{ISA}s.

The data is not in fact \IT{moved}, but \IT{copied}.

\subsection{MIPS}

\label{MIPS_leaf_function_ex1}

The GCC assembly output below lists registers by number:

\lstinputlisting[caption=\Optimizing GCC 4.4.5 (\assemblyOutput),style=customasmMIPS]{patterns/011_ret/MIPS.s}

\dots while \IDA does it by their pseudo names:

\lstinputlisting[caption=\Optimizing GCC 4.4.5 (IDA),style=customasmMIPS]{patterns/011_ret/MIPS_IDA.lst}

The \$2 (or \$V0) register is used to store the function's return value.
\myindex{MIPS!\Pseudoinstructions!LI}
\INS{LI} stands for ``Load Immediate'' and is the MIPS equivalent to \MOV.

\myindex{MIPS!\Instructions!J}
The other instruction is the jump instruction (J or JR) which returns the execution flow to the \gls{caller}.

\myindex{MIPS!Branch delay slot}
You might be wondering why the positions of the load instruction (LI) and the jump instruction (J or JR) are swapped. This is due to a \ac{RISC} feature called ``branch delay slot''.

The reason this happens is a quirk in the architecture of some RISC \ac{ISA}s and isn't important for our
purposes---we must simply keep in mind that in MIPS, the instruction following a jump or branch instruction
is executed \IT{before} the jump/branch instruction itself.

As a consequence, branch instructions always swap places with the instruction executed immediately beforehand.


In practice, functions which merely return 1 (\IT{true}) or 0 (\IT{false}) are very frequent.

The smallest ever of the standard UNIX utilities, \IT{/bin/true} and \IT{/bin/false} return 0 and 1 respectively, as an exit code.
(Zero as an exit code usually means success, non-zero means error.)
}
\RU{\subsubsection{std::string}
\myindex{\Cpp!STL!std::string}
\label{std_string}

\myparagraph{Как устроена структура}

Многие строковые библиотеки \InSqBrackets{\CNotes 2.2} обеспечивают структуру содержащую ссылку 
на буфер собственно со строкой, переменная всегда содержащую длину строки 
(что очень удобно для массы функций \InSqBrackets{\CNotes 2.2.1}) и переменную содержащую текущий размер буфера.

Строка в буфере обыкновенно оканчивается нулем: это для того чтобы указатель на буфер можно было
передавать в функции требующие на вход обычную сишную \ac{ASCIIZ}-строку.

Стандарт \Cpp не описывает, как именно нужно реализовывать std::string,
но, как правило, они реализованы как описано выше, с небольшими дополнениями.

Строки в \Cpp это не класс (как, например, QString в Qt), а темплейт (basic\_string), 
это сделано для того чтобы поддерживать 
строки содержащие разного типа символы: как минимум \Tchar и \IT{wchar\_t}.

Так что, std::string это класс с базовым типом \Tchar.

А std::wstring это класс с базовым типом \IT{wchar\_t}.

\mysubparagraph{MSVC}

В реализации MSVC, вместо ссылки на буфер может содержаться сам буфер (если строка короче 16-и символов).

Это означает, что каждая короткая строка будет занимать в памяти по крайней мере $16 + 4 + 4 = 24$ 
байт для 32-битной среды либо $16 + 8 + 8 = 32$ 
байта в 64-битной, а если строка длиннее 16-и символов, то прибавьте еще длину самой строки.

\lstinputlisting[caption=пример для MSVC,style=customc]{\CURPATH/STL/string/MSVC_RU.cpp}

Собственно, из этого исходника почти всё ясно.

Несколько замечаний:

Если строка короче 16-и символов, 
то отдельный буфер для строки в \glslink{heap}{куче} выделяться не будет.

Это удобно потому что на практике, основная часть строк действительно короткие.
Вероятно, разработчики в Microsoft выбрали размер в 16 символов как разумный баланс.

Теперь очень важный момент в конце функции main(): мы не пользуемся методом c\_str(), тем не менее,
если это скомпилировать и запустить, то обе строки появятся в консоли!

Работает это вот почему.

В первом случае строка короче 16-и символов и в начале объекта std::string (его можно рассматривать
просто как структуру) расположен буфер с этой строкой.
\printf трактует указатель как указатель на массив символов оканчивающийся нулем и поэтому всё работает.

Вывод второй строки (длиннее 16-и символов) даже еще опаснее: это вообще типичная программистская ошибка 
(или опечатка), забыть дописать c\_str().
Это работает потому что в это время в начале структуры расположен указатель на буфер.
Это может надолго остаться незамеченным: до тех пока там не появится строка 
короче 16-и символов, тогда процесс упадет.

\mysubparagraph{GCC}

В реализации GCC в структуре есть еще одна переменная --- reference count.

Интересно, что указатель на экземпляр класса std::string в GCC указывает не на начало самой структуры, 
а на указатель на буфера.
В libstdc++-v3\textbackslash{}include\textbackslash{}bits\textbackslash{}basic\_string.h 
мы можем прочитать что это сделано для удобства отладки:

\begin{lstlisting}
   *  The reason you want _M_data pointing to the character %array and
   *  not the _Rep is so that the debugger can see the string
   *  contents. (Probably we should add a non-inline member to get
   *  the _Rep for the debugger to use, so users can check the actual
   *  string length.)
\end{lstlisting}

\href{http://go.yurichev.com/17085}{исходный код basic\_string.h}

В нашем примере мы учитываем это:

\lstinputlisting[caption=пример для GCC,style=customc]{\CURPATH/STL/string/GCC_RU.cpp}

Нужны еще небольшие хаки чтобы сымитировать типичную ошибку, которую мы уже видели выше, из-за
более ужесточенной проверки типов в GCC, тем не менее, printf() работает и здесь без c\_str().

\myparagraph{Чуть более сложный пример}

\lstinputlisting[style=customc]{\CURPATH/STL/string/3.cpp}

\lstinputlisting[caption=MSVC 2012,style=customasmx86]{\CURPATH/STL/string/3_MSVC_RU.asm}

Собственно, компилятор не конструирует строки статически: да в общем-то и как
это возможно, если буфер с ней нужно хранить в \glslink{heap}{куче}?

Вместо этого в сегменте данных хранятся обычные \ac{ASCIIZ}-строки, а позже, во время выполнения, 
при помощи метода \q{assign}, конструируются строки s1 и s2
.
При помощи \TT{operator+}, создается строка s3.

Обратите внимание на то что вызов метода c\_str() отсутствует,
потому что его код достаточно короткий и компилятор вставил его прямо здесь:
если строка короче 16-и байт, то в регистре EAX остается указатель на буфер,
а если длиннее, то из этого же места достается адрес на буфер расположенный в \glslink{heap}{куче}.

Далее следуют вызовы трех деструкторов, причем, они вызываются только если строка длиннее 16-и байт:
тогда нужно освободить буфера в \glslink{heap}{куче}.
В противном случае, так как все три объекта std::string хранятся в стеке,
они освобождаются автоматически после выхода из функции.

Следовательно, работа с короткими строками более быстрая из-за м\'{е}ньшего обращения к \glslink{heap}{куче}.

Код на GCC даже проще (из-за того, что в GCC, как мы уже видели, не реализована возможность хранить короткую
строку прямо в структуре):

% TODO1 comment each function meaning
\lstinputlisting[caption=GCC 4.8.1,style=customasmx86]{\CURPATH/STL/string/3_GCC_RU.s}

Можно заметить, что в деструкторы передается не указатель на объект,
а указатель на место за 12 байт (или 3 слова) перед ним, то есть, на настоящее начало структуры.

\myparagraph{std::string как глобальная переменная}
\label{sec:std_string_as_global_variable}

Опытные программисты на \Cpp знают, что глобальные переменные \ac{STL}-типов вполне можно объявлять.

Да, действительно:

\lstinputlisting[style=customc]{\CURPATH/STL/string/5.cpp}

Но как и где будет вызываться конструктор \TT{std::string}?

На самом деле, эта переменная будет инициализирована даже перед началом \main.

\lstinputlisting[caption=MSVC 2012: здесь конструируется глобальная переменная{,} а также регистрируется её деструктор,style=customasmx86]{\CURPATH/STL/string/5_MSVC_p2.asm}

\lstinputlisting[caption=MSVC 2012: здесь глобальная переменная используется в \main,style=customasmx86]{\CURPATH/STL/string/5_MSVC_p1.asm}

\lstinputlisting[caption=MSVC 2012: эта функция-деструктор вызывается перед выходом,style=customasmx86]{\CURPATH/STL/string/5_MSVC_p3.asm}

\myindex{\CStandardLibrary!atexit()}
В реальности, из \ac{CRT}, еще до вызова main(), вызывается специальная функция,
в которой перечислены все конструкторы подобных переменных.
Более того: при помощи atexit() регистрируется функция, которая будет вызвана в конце работы программы:
в этой функции компилятор собирает вызовы деструкторов всех подобных глобальных переменных.

GCC работает похожим образом:

\lstinputlisting[caption=GCC 4.8.1,style=customasmx86]{\CURPATH/STL/string/5_GCC.s}

Но он не выделяет отдельной функции в которой будут собраны деструкторы: 
каждый деструктор передается в atexit() по одному.

% TODO а если глобальная STL-переменная в другом модуле? надо проверить.

}
\ifdefined\SPANISH
\chapter{Patrones de código}
\fi % SPANISH

\ifdefined\GERMAN
\chapter{Code-Muster}
\fi % GERMAN

\ifdefined\ENGLISH
\chapter{Code Patterns}
\fi % ENGLISH

\ifdefined\ITALIAN
\chapter{Forme di codice}
\fi % ITALIAN

\ifdefined\RUSSIAN
\chapter{Образцы кода}
\fi % RUSSIAN

\ifdefined\BRAZILIAN
\chapter{Padrões de códigos}
\fi % BRAZILIAN

\ifdefined\THAI
\chapter{รูปแบบของโค้ด}
\fi % THAI

\ifdefined\FRENCH
\chapter{Modèle de code}
\fi % FRENCH

\ifdefined\POLISH
\chapter{\PLph{}}
\fi % POLISH

% sections
\EN{\input{patterns/patterns_opt_dbg_EN}}
\ES{\input{patterns/patterns_opt_dbg_ES}}
\ITA{\input{patterns/patterns_opt_dbg_ITA}}
\PTBR{\input{patterns/patterns_opt_dbg_PTBR}}
\RU{\input{patterns/patterns_opt_dbg_RU}}
\THA{\input{patterns/patterns_opt_dbg_THA}}
\DE{\input{patterns/patterns_opt_dbg_DE}}
\FR{\input{patterns/patterns_opt_dbg_FR}}
\PL{\input{patterns/patterns_opt_dbg_PL}}

\RU{\section{Некоторые базовые понятия}}
\EN{\section{Some basics}}
\DE{\section{Einige Grundlagen}}
\FR{\section{Quelques bases}}
\ES{\section{\ESph{}}}
\ITA{\section{Alcune basi teoriche}}
\PTBR{\section{\PTBRph{}}}
\THA{\section{\THAph{}}}
\PL{\section{\PLph{}}}

% sections:
\EN{\input{patterns/intro_CPU_ISA_EN}}
\ES{\input{patterns/intro_CPU_ISA_ES}}
\ITA{\input{patterns/intro_CPU_ISA_ITA}}
\PTBR{\input{patterns/intro_CPU_ISA_PTBR}}
\RU{\input{patterns/intro_CPU_ISA_RU}}
\DE{\input{patterns/intro_CPU_ISA_DE}}
\FR{\input{patterns/intro_CPU_ISA_FR}}
\PL{\input{patterns/intro_CPU_ISA_PL}}

\EN{\input{patterns/numeral_EN}}
\RU{\input{patterns/numeral_RU}}
\ITA{\input{patterns/numeral_ITA}}
\DE{\input{patterns/numeral_DE}}
\FR{\input{patterns/numeral_FR}}
\PL{\input{patterns/numeral_PL}}

% chapters
\input{patterns/00_empty/main}
\input{patterns/011_ret/main}
\input{patterns/01_helloworld/main}
\input{patterns/015_prolog_epilogue/main}
\input{patterns/02_stack/main}
\input{patterns/03_printf/main}
\input{patterns/04_scanf/main}
\input{patterns/05_passing_arguments/main}
\input{patterns/06_return_results/main}
\input{patterns/061_pointers/main}
\input{patterns/065_GOTO/main}
\input{patterns/07_jcc/main}
\input{patterns/08_switch/main}
\input{patterns/09_loops/main}
\input{patterns/10_strings/main}
\input{patterns/11_arith_optimizations/main}
\input{patterns/12_FPU/main}
\input{patterns/13_arrays/main}
\input{patterns/14_bitfields/main}
\EN{\input{patterns/145_LCG/main_EN}}
\RU{\input{patterns/145_LCG/main_RU}}
\input{patterns/15_structs/main}
\input{patterns/17_unions/main}
\input{patterns/18_pointers_to_functions/main}
\input{patterns/185_64bit_in_32_env/main}

\EN{\input{patterns/19_SIMD/main_EN}}
\RU{\input{patterns/19_SIMD/main_RU}}
\DE{\input{patterns/19_SIMD/main_DE}}

\EN{\input{patterns/20_x64/main_EN}}
\RU{\input{patterns/20_x64/main_RU}}

\EN{\input{patterns/205_floating_SIMD/main_EN}}
\RU{\input{patterns/205_floating_SIMD/main_RU}}
\DE{\input{patterns/205_floating_SIMD/main_DE}}

\EN{\input{patterns/ARM/main_EN}}
\RU{\input{patterns/ARM/main_RU}}
\DE{\input{patterns/ARM/main_DE}}

\input{patterns/MIPS/main}

\ifdefined\SPANISH
\chapter{Patrones de código}
\fi % SPANISH

\ifdefined\GERMAN
\chapter{Code-Muster}
\fi % GERMAN

\ifdefined\ENGLISH
\chapter{Code Patterns}
\fi % ENGLISH

\ifdefined\ITALIAN
\chapter{Forme di codice}
\fi % ITALIAN

\ifdefined\RUSSIAN
\chapter{Образцы кода}
\fi % RUSSIAN

\ifdefined\BRAZILIAN
\chapter{Padrões de códigos}
\fi % BRAZILIAN

\ifdefined\THAI
\chapter{รูปแบบของโค้ด}
\fi % THAI

\ifdefined\FRENCH
\chapter{Modèle de code}
\fi % FRENCH

\ifdefined\POLISH
\chapter{\PLph{}}
\fi % POLISH

% sections
\EN{\input{patterns/patterns_opt_dbg_EN}}
\ES{\input{patterns/patterns_opt_dbg_ES}}
\ITA{\input{patterns/patterns_opt_dbg_ITA}}
\PTBR{\input{patterns/patterns_opt_dbg_PTBR}}
\RU{\input{patterns/patterns_opt_dbg_RU}}
\THA{\input{patterns/patterns_opt_dbg_THA}}
\DE{\input{patterns/patterns_opt_dbg_DE}}
\FR{\input{patterns/patterns_opt_dbg_FR}}
\PL{\input{patterns/patterns_opt_dbg_PL}}

\RU{\section{Некоторые базовые понятия}}
\EN{\section{Some basics}}
\DE{\section{Einige Grundlagen}}
\FR{\section{Quelques bases}}
\ES{\section{\ESph{}}}
\ITA{\section{Alcune basi teoriche}}
\PTBR{\section{\PTBRph{}}}
\THA{\section{\THAph{}}}
\PL{\section{\PLph{}}}

% sections:
\EN{\input{patterns/intro_CPU_ISA_EN}}
\ES{\input{patterns/intro_CPU_ISA_ES}}
\ITA{\input{patterns/intro_CPU_ISA_ITA}}
\PTBR{\input{patterns/intro_CPU_ISA_PTBR}}
\RU{\input{patterns/intro_CPU_ISA_RU}}
\DE{\input{patterns/intro_CPU_ISA_DE}}
\FR{\input{patterns/intro_CPU_ISA_FR}}
\PL{\input{patterns/intro_CPU_ISA_PL}}

\EN{\input{patterns/numeral_EN}}
\RU{\input{patterns/numeral_RU}}
\ITA{\input{patterns/numeral_ITA}}
\DE{\input{patterns/numeral_DE}}
\FR{\input{patterns/numeral_FR}}
\PL{\input{patterns/numeral_PL}}

% chapters
\input{patterns/00_empty/main}
\input{patterns/011_ret/main}
\input{patterns/01_helloworld/main}
\input{patterns/015_prolog_epilogue/main}
\input{patterns/02_stack/main}
\input{patterns/03_printf/main}
\input{patterns/04_scanf/main}
\input{patterns/05_passing_arguments/main}
\input{patterns/06_return_results/main}
\input{patterns/061_pointers/main}
\input{patterns/065_GOTO/main}
\input{patterns/07_jcc/main}
\input{patterns/08_switch/main}
\input{patterns/09_loops/main}
\input{patterns/10_strings/main}
\input{patterns/11_arith_optimizations/main}
\input{patterns/12_FPU/main}
\input{patterns/13_arrays/main}
\input{patterns/14_bitfields/main}
\EN{\input{patterns/145_LCG/main_EN}}
\RU{\input{patterns/145_LCG/main_RU}}
\input{patterns/15_structs/main}
\input{patterns/17_unions/main}
\input{patterns/18_pointers_to_functions/main}
\input{patterns/185_64bit_in_32_env/main}

\EN{\input{patterns/19_SIMD/main_EN}}
\RU{\input{patterns/19_SIMD/main_RU}}
\DE{\input{patterns/19_SIMD/main_DE}}

\EN{\input{patterns/20_x64/main_EN}}
\RU{\input{patterns/20_x64/main_RU}}

\EN{\input{patterns/205_floating_SIMD/main_EN}}
\RU{\input{patterns/205_floating_SIMD/main_RU}}
\DE{\input{patterns/205_floating_SIMD/main_DE}}

\EN{\input{patterns/ARM/main_EN}}
\RU{\input{patterns/ARM/main_RU}}
\DE{\input{patterns/ARM/main_DE}}

\input{patterns/MIPS/main}

\ifdefined\SPANISH
\chapter{Patrones de código}
\fi % SPANISH

\ifdefined\GERMAN
\chapter{Code-Muster}
\fi % GERMAN

\ifdefined\ENGLISH
\chapter{Code Patterns}
\fi % ENGLISH

\ifdefined\ITALIAN
\chapter{Forme di codice}
\fi % ITALIAN

\ifdefined\RUSSIAN
\chapter{Образцы кода}
\fi % RUSSIAN

\ifdefined\BRAZILIAN
\chapter{Padrões de códigos}
\fi % BRAZILIAN

\ifdefined\THAI
\chapter{รูปแบบของโค้ด}
\fi % THAI

\ifdefined\FRENCH
\chapter{Modèle de code}
\fi % FRENCH

\ifdefined\POLISH
\chapter{\PLph{}}
\fi % POLISH

% sections
\EN{\input{patterns/patterns_opt_dbg_EN}}
\ES{\input{patterns/patterns_opt_dbg_ES}}
\ITA{\input{patterns/patterns_opt_dbg_ITA}}
\PTBR{\input{patterns/patterns_opt_dbg_PTBR}}
\RU{\input{patterns/patterns_opt_dbg_RU}}
\THA{\input{patterns/patterns_opt_dbg_THA}}
\DE{\input{patterns/patterns_opt_dbg_DE}}
\FR{\input{patterns/patterns_opt_dbg_FR}}
\PL{\input{patterns/patterns_opt_dbg_PL}}

\RU{\section{Некоторые базовые понятия}}
\EN{\section{Some basics}}
\DE{\section{Einige Grundlagen}}
\FR{\section{Quelques bases}}
\ES{\section{\ESph{}}}
\ITA{\section{Alcune basi teoriche}}
\PTBR{\section{\PTBRph{}}}
\THA{\section{\THAph{}}}
\PL{\section{\PLph{}}}

% sections:
\EN{\input{patterns/intro_CPU_ISA_EN}}
\ES{\input{patterns/intro_CPU_ISA_ES}}
\ITA{\input{patterns/intro_CPU_ISA_ITA}}
\PTBR{\input{patterns/intro_CPU_ISA_PTBR}}
\RU{\input{patterns/intro_CPU_ISA_RU}}
\DE{\input{patterns/intro_CPU_ISA_DE}}
\FR{\input{patterns/intro_CPU_ISA_FR}}
\PL{\input{patterns/intro_CPU_ISA_PL}}

\EN{\input{patterns/numeral_EN}}
\RU{\input{patterns/numeral_RU}}
\ITA{\input{patterns/numeral_ITA}}
\DE{\input{patterns/numeral_DE}}
\FR{\input{patterns/numeral_FR}}
\PL{\input{patterns/numeral_PL}}

% chapters
\input{patterns/00_empty/main}
\input{patterns/011_ret/main}
\input{patterns/01_helloworld/main}
\input{patterns/015_prolog_epilogue/main}
\input{patterns/02_stack/main}
\input{patterns/03_printf/main}
\input{patterns/04_scanf/main}
\input{patterns/05_passing_arguments/main}
\input{patterns/06_return_results/main}
\input{patterns/061_pointers/main}
\input{patterns/065_GOTO/main}
\input{patterns/07_jcc/main}
\input{patterns/08_switch/main}
\input{patterns/09_loops/main}
\input{patterns/10_strings/main}
\input{patterns/11_arith_optimizations/main}
\input{patterns/12_FPU/main}
\input{patterns/13_arrays/main}
\input{patterns/14_bitfields/main}
\EN{\input{patterns/145_LCG/main_EN}}
\RU{\input{patterns/145_LCG/main_RU}}
\input{patterns/15_structs/main}
\input{patterns/17_unions/main}
\input{patterns/18_pointers_to_functions/main}
\input{patterns/185_64bit_in_32_env/main}

\EN{\input{patterns/19_SIMD/main_EN}}
\RU{\input{patterns/19_SIMD/main_RU}}
\DE{\input{patterns/19_SIMD/main_DE}}

\EN{\input{patterns/20_x64/main_EN}}
\RU{\input{patterns/20_x64/main_RU}}

\EN{\input{patterns/205_floating_SIMD/main_EN}}
\RU{\input{patterns/205_floating_SIMD/main_RU}}
\DE{\input{patterns/205_floating_SIMD/main_DE}}

\EN{\input{patterns/ARM/main_EN}}
\RU{\input{patterns/ARM/main_RU}}
\DE{\input{patterns/ARM/main_DE}}

\input{patterns/MIPS/main}

\ifdefined\SPANISH
\chapter{Patrones de código}
\fi % SPANISH

\ifdefined\GERMAN
\chapter{Code-Muster}
\fi % GERMAN

\ifdefined\ENGLISH
\chapter{Code Patterns}
\fi % ENGLISH

\ifdefined\ITALIAN
\chapter{Forme di codice}
\fi % ITALIAN

\ifdefined\RUSSIAN
\chapter{Образцы кода}
\fi % RUSSIAN

\ifdefined\BRAZILIAN
\chapter{Padrões de códigos}
\fi % BRAZILIAN

\ifdefined\THAI
\chapter{รูปแบบของโค้ด}
\fi % THAI

\ifdefined\FRENCH
\chapter{Modèle de code}
\fi % FRENCH

\ifdefined\POLISH
\chapter{\PLph{}}
\fi % POLISH

% sections
\EN{\input{patterns/patterns_opt_dbg_EN}}
\ES{\input{patterns/patterns_opt_dbg_ES}}
\ITA{\input{patterns/patterns_opt_dbg_ITA}}
\PTBR{\input{patterns/patterns_opt_dbg_PTBR}}
\RU{\input{patterns/patterns_opt_dbg_RU}}
\THA{\input{patterns/patterns_opt_dbg_THA}}
\DE{\input{patterns/patterns_opt_dbg_DE}}
\FR{\input{patterns/patterns_opt_dbg_FR}}
\PL{\input{patterns/patterns_opt_dbg_PL}}

\RU{\section{Некоторые базовые понятия}}
\EN{\section{Some basics}}
\DE{\section{Einige Grundlagen}}
\FR{\section{Quelques bases}}
\ES{\section{\ESph{}}}
\ITA{\section{Alcune basi teoriche}}
\PTBR{\section{\PTBRph{}}}
\THA{\section{\THAph{}}}
\PL{\section{\PLph{}}}

% sections:
\EN{\input{patterns/intro_CPU_ISA_EN}}
\ES{\input{patterns/intro_CPU_ISA_ES}}
\ITA{\input{patterns/intro_CPU_ISA_ITA}}
\PTBR{\input{patterns/intro_CPU_ISA_PTBR}}
\RU{\input{patterns/intro_CPU_ISA_RU}}
\DE{\input{patterns/intro_CPU_ISA_DE}}
\FR{\input{patterns/intro_CPU_ISA_FR}}
\PL{\input{patterns/intro_CPU_ISA_PL}}

\EN{\input{patterns/numeral_EN}}
\RU{\input{patterns/numeral_RU}}
\ITA{\input{patterns/numeral_ITA}}
\DE{\input{patterns/numeral_DE}}
\FR{\input{patterns/numeral_FR}}
\PL{\input{patterns/numeral_PL}}

% chapters
\input{patterns/00_empty/main}
\input{patterns/011_ret/main}
\input{patterns/01_helloworld/main}
\input{patterns/015_prolog_epilogue/main}
\input{patterns/02_stack/main}
\input{patterns/03_printf/main}
\input{patterns/04_scanf/main}
\input{patterns/05_passing_arguments/main}
\input{patterns/06_return_results/main}
\input{patterns/061_pointers/main}
\input{patterns/065_GOTO/main}
\input{patterns/07_jcc/main}
\input{patterns/08_switch/main}
\input{patterns/09_loops/main}
\input{patterns/10_strings/main}
\input{patterns/11_arith_optimizations/main}
\input{patterns/12_FPU/main}
\input{patterns/13_arrays/main}
\input{patterns/14_bitfields/main}
\EN{\input{patterns/145_LCG/main_EN}}
\RU{\input{patterns/145_LCG/main_RU}}
\input{patterns/15_structs/main}
\input{patterns/17_unions/main}
\input{patterns/18_pointers_to_functions/main}
\input{patterns/185_64bit_in_32_env/main}

\EN{\input{patterns/19_SIMD/main_EN}}
\RU{\input{patterns/19_SIMD/main_RU}}
\DE{\input{patterns/19_SIMD/main_DE}}

\EN{\input{patterns/20_x64/main_EN}}
\RU{\input{patterns/20_x64/main_RU}}

\EN{\input{patterns/205_floating_SIMD/main_EN}}
\RU{\input{patterns/205_floating_SIMD/main_RU}}
\DE{\input{patterns/205_floating_SIMD/main_DE}}

\EN{\input{patterns/ARM/main_EN}}
\RU{\input{patterns/ARM/main_RU}}
\DE{\input{patterns/ARM/main_DE}}

\input{patterns/MIPS/main}


\EN{\section{Returning Values}
\label{ret_val_func}

Another simple function is the one that simply returns a constant value:

\lstinputlisting[caption=\EN{\CCpp Code},style=customc]{patterns/011_ret/1.c}

Let's compile it.

\subsection{x86}

Here's what both the GCC and MSVC compilers produce (with optimization) on the x86 platform:

\lstinputlisting[caption=\Optimizing GCC/MSVC (\assemblyOutput),style=customasmx86]{patterns/011_ret/1.s}

\myindex{x86!\Instructions!RET}
There are just two instructions: the first places the value 123 into the \EAX register,
which is used by convention for storing the return
value, and the second one is \RET, which returns execution to the \gls{caller}.

The caller will take the result from the \EAX register.

\subsection{ARM}

There are a few differences on the ARM platform:

\lstinputlisting[caption=\OptimizingKeilVI (\ARMMode) ASM Output,style=customasmARM]{patterns/011_ret/1_Keil_ARM_O3.s}

ARM uses the register \Reg{0} for returning the results of functions, so 123 is copied into \Reg{0}.

\myindex{ARM!\Instructions!MOV}
\myindex{x86!\Instructions!MOV}
It is worth noting that \MOV is a misleading name for the instruction in both the x86 and ARM \ac{ISA}s.

The data is not in fact \IT{moved}, but \IT{copied}.

\subsection{MIPS}

\label{MIPS_leaf_function_ex1}

The GCC assembly output below lists registers by number:

\lstinputlisting[caption=\Optimizing GCC 4.4.5 (\assemblyOutput),style=customasmMIPS]{patterns/011_ret/MIPS.s}

\dots while \IDA does it by their pseudo names:

\lstinputlisting[caption=\Optimizing GCC 4.4.5 (IDA),style=customasmMIPS]{patterns/011_ret/MIPS_IDA.lst}

The \$2 (or \$V0) register is used to store the function's return value.
\myindex{MIPS!\Pseudoinstructions!LI}
\INS{LI} stands for ``Load Immediate'' and is the MIPS equivalent to \MOV.

\myindex{MIPS!\Instructions!J}
The other instruction is the jump instruction (J or JR) which returns the execution flow to the \gls{caller}.

\myindex{MIPS!Branch delay slot}
You might be wondering why the positions of the load instruction (LI) and the jump instruction (J or JR) are swapped. This is due to a \ac{RISC} feature called ``branch delay slot''.

The reason this happens is a quirk in the architecture of some RISC \ac{ISA}s and isn't important for our
purposes---we must simply keep in mind that in MIPS, the instruction following a jump or branch instruction
is executed \IT{before} the jump/branch instruction itself.

As a consequence, branch instructions always swap places with the instruction executed immediately beforehand.


In practice, functions which merely return 1 (\IT{true}) or 0 (\IT{false}) are very frequent.

The smallest ever of the standard UNIX utilities, \IT{/bin/true} and \IT{/bin/false} return 0 and 1 respectively, as an exit code.
(Zero as an exit code usually means success, non-zero means error.)
}
\RU{\subsubsection{std::string}
\myindex{\Cpp!STL!std::string}
\label{std_string}

\myparagraph{Как устроена структура}

Многие строковые библиотеки \InSqBrackets{\CNotes 2.2} обеспечивают структуру содержащую ссылку 
на буфер собственно со строкой, переменная всегда содержащую длину строки 
(что очень удобно для массы функций \InSqBrackets{\CNotes 2.2.1}) и переменную содержащую текущий размер буфера.

Строка в буфере обыкновенно оканчивается нулем: это для того чтобы указатель на буфер можно было
передавать в функции требующие на вход обычную сишную \ac{ASCIIZ}-строку.

Стандарт \Cpp не описывает, как именно нужно реализовывать std::string,
но, как правило, они реализованы как описано выше, с небольшими дополнениями.

Строки в \Cpp это не класс (как, например, QString в Qt), а темплейт (basic\_string), 
это сделано для того чтобы поддерживать 
строки содержащие разного типа символы: как минимум \Tchar и \IT{wchar\_t}.

Так что, std::string это класс с базовым типом \Tchar.

А std::wstring это класс с базовым типом \IT{wchar\_t}.

\mysubparagraph{MSVC}

В реализации MSVC, вместо ссылки на буфер может содержаться сам буфер (если строка короче 16-и символов).

Это означает, что каждая короткая строка будет занимать в памяти по крайней мере $16 + 4 + 4 = 24$ 
байт для 32-битной среды либо $16 + 8 + 8 = 32$ 
байта в 64-битной, а если строка длиннее 16-и символов, то прибавьте еще длину самой строки.

\lstinputlisting[caption=пример для MSVC,style=customc]{\CURPATH/STL/string/MSVC_RU.cpp}

Собственно, из этого исходника почти всё ясно.

Несколько замечаний:

Если строка короче 16-и символов, 
то отдельный буфер для строки в \glslink{heap}{куче} выделяться не будет.

Это удобно потому что на практике, основная часть строк действительно короткие.
Вероятно, разработчики в Microsoft выбрали размер в 16 символов как разумный баланс.

Теперь очень важный момент в конце функции main(): мы не пользуемся методом c\_str(), тем не менее,
если это скомпилировать и запустить, то обе строки появятся в консоли!

Работает это вот почему.

В первом случае строка короче 16-и символов и в начале объекта std::string (его можно рассматривать
просто как структуру) расположен буфер с этой строкой.
\printf трактует указатель как указатель на массив символов оканчивающийся нулем и поэтому всё работает.

Вывод второй строки (длиннее 16-и символов) даже еще опаснее: это вообще типичная программистская ошибка 
(или опечатка), забыть дописать c\_str().
Это работает потому что в это время в начале структуры расположен указатель на буфер.
Это может надолго остаться незамеченным: до тех пока там не появится строка 
короче 16-и символов, тогда процесс упадет.

\mysubparagraph{GCC}

В реализации GCC в структуре есть еще одна переменная --- reference count.

Интересно, что указатель на экземпляр класса std::string в GCC указывает не на начало самой структуры, 
а на указатель на буфера.
В libstdc++-v3\textbackslash{}include\textbackslash{}bits\textbackslash{}basic\_string.h 
мы можем прочитать что это сделано для удобства отладки:

\begin{lstlisting}
   *  The reason you want _M_data pointing to the character %array and
   *  not the _Rep is so that the debugger can see the string
   *  contents. (Probably we should add a non-inline member to get
   *  the _Rep for the debugger to use, so users can check the actual
   *  string length.)
\end{lstlisting}

\href{http://go.yurichev.com/17085}{исходный код basic\_string.h}

В нашем примере мы учитываем это:

\lstinputlisting[caption=пример для GCC,style=customc]{\CURPATH/STL/string/GCC_RU.cpp}

Нужны еще небольшие хаки чтобы сымитировать типичную ошибку, которую мы уже видели выше, из-за
более ужесточенной проверки типов в GCC, тем не менее, printf() работает и здесь без c\_str().

\myparagraph{Чуть более сложный пример}

\lstinputlisting[style=customc]{\CURPATH/STL/string/3.cpp}

\lstinputlisting[caption=MSVC 2012,style=customasmx86]{\CURPATH/STL/string/3_MSVC_RU.asm}

Собственно, компилятор не конструирует строки статически: да в общем-то и как
это возможно, если буфер с ней нужно хранить в \glslink{heap}{куче}?

Вместо этого в сегменте данных хранятся обычные \ac{ASCIIZ}-строки, а позже, во время выполнения, 
при помощи метода \q{assign}, конструируются строки s1 и s2
.
При помощи \TT{operator+}, создается строка s3.

Обратите внимание на то что вызов метода c\_str() отсутствует,
потому что его код достаточно короткий и компилятор вставил его прямо здесь:
если строка короче 16-и байт, то в регистре EAX остается указатель на буфер,
а если длиннее, то из этого же места достается адрес на буфер расположенный в \glslink{heap}{куче}.

Далее следуют вызовы трех деструкторов, причем, они вызываются только если строка длиннее 16-и байт:
тогда нужно освободить буфера в \glslink{heap}{куче}.
В противном случае, так как все три объекта std::string хранятся в стеке,
они освобождаются автоматически после выхода из функции.

Следовательно, работа с короткими строками более быстрая из-за м\'{е}ньшего обращения к \glslink{heap}{куче}.

Код на GCC даже проще (из-за того, что в GCC, как мы уже видели, не реализована возможность хранить короткую
строку прямо в структуре):

% TODO1 comment each function meaning
\lstinputlisting[caption=GCC 4.8.1,style=customasmx86]{\CURPATH/STL/string/3_GCC_RU.s}

Можно заметить, что в деструкторы передается не указатель на объект,
а указатель на место за 12 байт (или 3 слова) перед ним, то есть, на настоящее начало структуры.

\myparagraph{std::string как глобальная переменная}
\label{sec:std_string_as_global_variable}

Опытные программисты на \Cpp знают, что глобальные переменные \ac{STL}-типов вполне можно объявлять.

Да, действительно:

\lstinputlisting[style=customc]{\CURPATH/STL/string/5.cpp}

Но как и где будет вызываться конструктор \TT{std::string}?

На самом деле, эта переменная будет инициализирована даже перед началом \main.

\lstinputlisting[caption=MSVC 2012: здесь конструируется глобальная переменная{,} а также регистрируется её деструктор,style=customasmx86]{\CURPATH/STL/string/5_MSVC_p2.asm}

\lstinputlisting[caption=MSVC 2012: здесь глобальная переменная используется в \main,style=customasmx86]{\CURPATH/STL/string/5_MSVC_p1.asm}

\lstinputlisting[caption=MSVC 2012: эта функция-деструктор вызывается перед выходом,style=customasmx86]{\CURPATH/STL/string/5_MSVC_p3.asm}

\myindex{\CStandardLibrary!atexit()}
В реальности, из \ac{CRT}, еще до вызова main(), вызывается специальная функция,
в которой перечислены все конструкторы подобных переменных.
Более того: при помощи atexit() регистрируется функция, которая будет вызвана в конце работы программы:
в этой функции компилятор собирает вызовы деструкторов всех подобных глобальных переменных.

GCC работает похожим образом:

\lstinputlisting[caption=GCC 4.8.1,style=customasmx86]{\CURPATH/STL/string/5_GCC.s}

Но он не выделяет отдельной функции в которой будут собраны деструкторы: 
каждый деструктор передается в atexit() по одному.

% TODO а если глобальная STL-переменная в другом модуле? надо проверить.

}
\DE{\subsection{Einfachste XOR-Verschlüsselung überhaupt}

Ich habe einmal eine Software gesehen, bei der alle Debugging-Ausgaben mit XOR mit dem Wert 3
verschlüsselt wurden. Mit anderen Worten, die beiden niedrigsten Bits aller Buchstaben wurden invertiert.

``Hello, world'' wurde zu ``Kfool/\#tlqog'':

\begin{lstlisting}
#!/usr/bin/python

msg="Hello, world!"

print "".join(map(lambda x: chr(ord(x)^3), msg))
\end{lstlisting}

Das ist eine ziemlich interessante Verschlüsselung (oder besser eine Verschleierung),
weil sie zwei wichtige Eigenschaften hat:
1) es ist eine einzige Funktion zum Verschlüsseln und entschlüsseln, sie muss nur wiederholt angewendet werden
2) die entstehenden Buchstaben befinden sich im druckbaren Bereich, also die ganze Zeichenkette kann ohne
Escape-Symbole im Code verwendet werden.

Die zweite Eigenschaft nutzt die Tatsache, dass alle druckbaren Zeichen in Reihen organisiert sind: 0x2x-0x7x,
und wenn die beiden niederwertigsten Bits invertiert werden, wird der Buchstabe um eine oder drei Stellen nach
links oder rechts \IT{verschoben}, aber niemals in eine andere Reihe:

\begin{figure}[H]
\centering
\includegraphics[width=0.7\textwidth]{ascii_clean.png}
\caption{7-Bit \ac{ASCII} Tabelle in Emacs}
\end{figure}

\dots mit dem Zeichen 0x7F als einziger Ausnahme.

Im Folgenden werden also beispielsweise die Zeichen A-Z \IT{verschlüsselt}:

\begin{lstlisting}
#!/usr/bin/python

msg="@ABCDEFGHIJKLMNO"

print "".join(map(lambda x: chr(ord(x)^3), msg))
\end{lstlisting}

Ergebnis:
% FIXME \verb  --  relevant comment for German?
\begin{lstlisting}
CBA@GFEDKJIHONML
\end{lstlisting}

Es sieht so aus als würden die Zeichen ``@'' und ``C'' sowie ``B'' und ``A'' vertauscht werden.

Hier ist noch ein interessantes Beispiel, in dem gezeigt wird, wie die Eigenschaften von XOR
ausgenutzt werden können: Exakt den gleichen Effekt, dass druckbare Zeichen auch druckbar bleiben,
kann man dadurch erzielen, dass irgendeine Kombination der niedrigsten vier Bits invertiert wird.
}

\EN{\section{Returning Values}
\label{ret_val_func}

Another simple function is the one that simply returns a constant value:

\lstinputlisting[caption=\EN{\CCpp Code},style=customc]{patterns/011_ret/1.c}

Let's compile it.

\subsection{x86}

Here's what both the GCC and MSVC compilers produce (with optimization) on the x86 platform:

\lstinputlisting[caption=\Optimizing GCC/MSVC (\assemblyOutput),style=customasmx86]{patterns/011_ret/1.s}

\myindex{x86!\Instructions!RET}
There are just two instructions: the first places the value 123 into the \EAX register,
which is used by convention for storing the return
value, and the second one is \RET, which returns execution to the \gls{caller}.

The caller will take the result from the \EAX register.

\subsection{ARM}

There are a few differences on the ARM platform:

\lstinputlisting[caption=\OptimizingKeilVI (\ARMMode) ASM Output,style=customasmARM]{patterns/011_ret/1_Keil_ARM_O3.s}

ARM uses the register \Reg{0} for returning the results of functions, so 123 is copied into \Reg{0}.

\myindex{ARM!\Instructions!MOV}
\myindex{x86!\Instructions!MOV}
It is worth noting that \MOV is a misleading name for the instruction in both the x86 and ARM \ac{ISA}s.

The data is not in fact \IT{moved}, but \IT{copied}.

\subsection{MIPS}

\label{MIPS_leaf_function_ex1}

The GCC assembly output below lists registers by number:

\lstinputlisting[caption=\Optimizing GCC 4.4.5 (\assemblyOutput),style=customasmMIPS]{patterns/011_ret/MIPS.s}

\dots while \IDA does it by their pseudo names:

\lstinputlisting[caption=\Optimizing GCC 4.4.5 (IDA),style=customasmMIPS]{patterns/011_ret/MIPS_IDA.lst}

The \$2 (or \$V0) register is used to store the function's return value.
\myindex{MIPS!\Pseudoinstructions!LI}
\INS{LI} stands for ``Load Immediate'' and is the MIPS equivalent to \MOV.

\myindex{MIPS!\Instructions!J}
The other instruction is the jump instruction (J or JR) which returns the execution flow to the \gls{caller}.

\myindex{MIPS!Branch delay slot}
You might be wondering why the positions of the load instruction (LI) and the jump instruction (J or JR) are swapped. This is due to a \ac{RISC} feature called ``branch delay slot''.

The reason this happens is a quirk in the architecture of some RISC \ac{ISA}s and isn't important for our
purposes---we must simply keep in mind that in MIPS, the instruction following a jump or branch instruction
is executed \IT{before} the jump/branch instruction itself.

As a consequence, branch instructions always swap places with the instruction executed immediately beforehand.


In practice, functions which merely return 1 (\IT{true}) or 0 (\IT{false}) are very frequent.

The smallest ever of the standard UNIX utilities, \IT{/bin/true} and \IT{/bin/false} return 0 and 1 respectively, as an exit code.
(Zero as an exit code usually means success, non-zero means error.)
}
\RU{\subsubsection{std::string}
\myindex{\Cpp!STL!std::string}
\label{std_string}

\myparagraph{Как устроена структура}

Многие строковые библиотеки \InSqBrackets{\CNotes 2.2} обеспечивают структуру содержащую ссылку 
на буфер собственно со строкой, переменная всегда содержащую длину строки 
(что очень удобно для массы функций \InSqBrackets{\CNotes 2.2.1}) и переменную содержащую текущий размер буфера.

Строка в буфере обыкновенно оканчивается нулем: это для того чтобы указатель на буфер можно было
передавать в функции требующие на вход обычную сишную \ac{ASCIIZ}-строку.

Стандарт \Cpp не описывает, как именно нужно реализовывать std::string,
но, как правило, они реализованы как описано выше, с небольшими дополнениями.

Строки в \Cpp это не класс (как, например, QString в Qt), а темплейт (basic\_string), 
это сделано для того чтобы поддерживать 
строки содержащие разного типа символы: как минимум \Tchar и \IT{wchar\_t}.

Так что, std::string это класс с базовым типом \Tchar.

А std::wstring это класс с базовым типом \IT{wchar\_t}.

\mysubparagraph{MSVC}

В реализации MSVC, вместо ссылки на буфер может содержаться сам буфер (если строка короче 16-и символов).

Это означает, что каждая короткая строка будет занимать в памяти по крайней мере $16 + 4 + 4 = 24$ 
байт для 32-битной среды либо $16 + 8 + 8 = 32$ 
байта в 64-битной, а если строка длиннее 16-и символов, то прибавьте еще длину самой строки.

\lstinputlisting[caption=пример для MSVC,style=customc]{\CURPATH/STL/string/MSVC_RU.cpp}

Собственно, из этого исходника почти всё ясно.

Несколько замечаний:

Если строка короче 16-и символов, 
то отдельный буфер для строки в \glslink{heap}{куче} выделяться не будет.

Это удобно потому что на практике, основная часть строк действительно короткие.
Вероятно, разработчики в Microsoft выбрали размер в 16 символов как разумный баланс.

Теперь очень важный момент в конце функции main(): мы не пользуемся методом c\_str(), тем не менее,
если это скомпилировать и запустить, то обе строки появятся в консоли!

Работает это вот почему.

В первом случае строка короче 16-и символов и в начале объекта std::string (его можно рассматривать
просто как структуру) расположен буфер с этой строкой.
\printf трактует указатель как указатель на массив символов оканчивающийся нулем и поэтому всё работает.

Вывод второй строки (длиннее 16-и символов) даже еще опаснее: это вообще типичная программистская ошибка 
(или опечатка), забыть дописать c\_str().
Это работает потому что в это время в начале структуры расположен указатель на буфер.
Это может надолго остаться незамеченным: до тех пока там не появится строка 
короче 16-и символов, тогда процесс упадет.

\mysubparagraph{GCC}

В реализации GCC в структуре есть еще одна переменная --- reference count.

Интересно, что указатель на экземпляр класса std::string в GCC указывает не на начало самой структуры, 
а на указатель на буфера.
В libstdc++-v3\textbackslash{}include\textbackslash{}bits\textbackslash{}basic\_string.h 
мы можем прочитать что это сделано для удобства отладки:

\begin{lstlisting}
   *  The reason you want _M_data pointing to the character %array and
   *  not the _Rep is so that the debugger can see the string
   *  contents. (Probably we should add a non-inline member to get
   *  the _Rep for the debugger to use, so users can check the actual
   *  string length.)
\end{lstlisting}

\href{http://go.yurichev.com/17085}{исходный код basic\_string.h}

В нашем примере мы учитываем это:

\lstinputlisting[caption=пример для GCC,style=customc]{\CURPATH/STL/string/GCC_RU.cpp}

Нужны еще небольшие хаки чтобы сымитировать типичную ошибку, которую мы уже видели выше, из-за
более ужесточенной проверки типов в GCC, тем не менее, printf() работает и здесь без c\_str().

\myparagraph{Чуть более сложный пример}

\lstinputlisting[style=customc]{\CURPATH/STL/string/3.cpp}

\lstinputlisting[caption=MSVC 2012,style=customasmx86]{\CURPATH/STL/string/3_MSVC_RU.asm}

Собственно, компилятор не конструирует строки статически: да в общем-то и как
это возможно, если буфер с ней нужно хранить в \glslink{heap}{куче}?

Вместо этого в сегменте данных хранятся обычные \ac{ASCIIZ}-строки, а позже, во время выполнения, 
при помощи метода \q{assign}, конструируются строки s1 и s2
.
При помощи \TT{operator+}, создается строка s3.

Обратите внимание на то что вызов метода c\_str() отсутствует,
потому что его код достаточно короткий и компилятор вставил его прямо здесь:
если строка короче 16-и байт, то в регистре EAX остается указатель на буфер,
а если длиннее, то из этого же места достается адрес на буфер расположенный в \glslink{heap}{куче}.

Далее следуют вызовы трех деструкторов, причем, они вызываются только если строка длиннее 16-и байт:
тогда нужно освободить буфера в \glslink{heap}{куче}.
В противном случае, так как все три объекта std::string хранятся в стеке,
они освобождаются автоматически после выхода из функции.

Следовательно, работа с короткими строками более быстрая из-за м\'{е}ньшего обращения к \glslink{heap}{куче}.

Код на GCC даже проще (из-за того, что в GCC, как мы уже видели, не реализована возможность хранить короткую
строку прямо в структуре):

% TODO1 comment each function meaning
\lstinputlisting[caption=GCC 4.8.1,style=customasmx86]{\CURPATH/STL/string/3_GCC_RU.s}

Можно заметить, что в деструкторы передается не указатель на объект,
а указатель на место за 12 байт (или 3 слова) перед ним, то есть, на настоящее начало структуры.

\myparagraph{std::string как глобальная переменная}
\label{sec:std_string_as_global_variable}

Опытные программисты на \Cpp знают, что глобальные переменные \ac{STL}-типов вполне можно объявлять.

Да, действительно:

\lstinputlisting[style=customc]{\CURPATH/STL/string/5.cpp}

Но как и где будет вызываться конструктор \TT{std::string}?

На самом деле, эта переменная будет инициализирована даже перед началом \main.

\lstinputlisting[caption=MSVC 2012: здесь конструируется глобальная переменная{,} а также регистрируется её деструктор,style=customasmx86]{\CURPATH/STL/string/5_MSVC_p2.asm}

\lstinputlisting[caption=MSVC 2012: здесь глобальная переменная используется в \main,style=customasmx86]{\CURPATH/STL/string/5_MSVC_p1.asm}

\lstinputlisting[caption=MSVC 2012: эта функция-деструктор вызывается перед выходом,style=customasmx86]{\CURPATH/STL/string/5_MSVC_p3.asm}

\myindex{\CStandardLibrary!atexit()}
В реальности, из \ac{CRT}, еще до вызова main(), вызывается специальная функция,
в которой перечислены все конструкторы подобных переменных.
Более того: при помощи atexit() регистрируется функция, которая будет вызвана в конце работы программы:
в этой функции компилятор собирает вызовы деструкторов всех подобных глобальных переменных.

GCC работает похожим образом:

\lstinputlisting[caption=GCC 4.8.1,style=customasmx86]{\CURPATH/STL/string/5_GCC.s}

Но он не выделяет отдельной функции в которой будут собраны деструкторы: 
каждый деструктор передается в atexit() по одному.

% TODO а если глобальная STL-переменная в другом модуле? надо проверить.

}

\EN{\section{Returning Values}
\label{ret_val_func}

Another simple function is the one that simply returns a constant value:

\lstinputlisting[caption=\EN{\CCpp Code},style=customc]{patterns/011_ret/1.c}

Let's compile it.

\subsection{x86}

Here's what both the GCC and MSVC compilers produce (with optimization) on the x86 platform:

\lstinputlisting[caption=\Optimizing GCC/MSVC (\assemblyOutput),style=customasmx86]{patterns/011_ret/1.s}

\myindex{x86!\Instructions!RET}
There are just two instructions: the first places the value 123 into the \EAX register,
which is used by convention for storing the return
value, and the second one is \RET, which returns execution to the \gls{caller}.

The caller will take the result from the \EAX register.

\subsection{ARM}

There are a few differences on the ARM platform:

\lstinputlisting[caption=\OptimizingKeilVI (\ARMMode) ASM Output,style=customasmARM]{patterns/011_ret/1_Keil_ARM_O3.s}

ARM uses the register \Reg{0} for returning the results of functions, so 123 is copied into \Reg{0}.

\myindex{ARM!\Instructions!MOV}
\myindex{x86!\Instructions!MOV}
It is worth noting that \MOV is a misleading name for the instruction in both the x86 and ARM \ac{ISA}s.

The data is not in fact \IT{moved}, but \IT{copied}.

\subsection{MIPS}

\label{MIPS_leaf_function_ex1}

The GCC assembly output below lists registers by number:

\lstinputlisting[caption=\Optimizing GCC 4.4.5 (\assemblyOutput),style=customasmMIPS]{patterns/011_ret/MIPS.s}

\dots while \IDA does it by their pseudo names:

\lstinputlisting[caption=\Optimizing GCC 4.4.5 (IDA),style=customasmMIPS]{patterns/011_ret/MIPS_IDA.lst}

The \$2 (or \$V0) register is used to store the function's return value.
\myindex{MIPS!\Pseudoinstructions!LI}
\INS{LI} stands for ``Load Immediate'' and is the MIPS equivalent to \MOV.

\myindex{MIPS!\Instructions!J}
The other instruction is the jump instruction (J or JR) which returns the execution flow to the \gls{caller}.

\myindex{MIPS!Branch delay slot}
You might be wondering why the positions of the load instruction (LI) and the jump instruction (J or JR) are swapped. This is due to a \ac{RISC} feature called ``branch delay slot''.

The reason this happens is a quirk in the architecture of some RISC \ac{ISA}s and isn't important for our
purposes---we must simply keep in mind that in MIPS, the instruction following a jump or branch instruction
is executed \IT{before} the jump/branch instruction itself.

As a consequence, branch instructions always swap places with the instruction executed immediately beforehand.


In practice, functions which merely return 1 (\IT{true}) or 0 (\IT{false}) are very frequent.

The smallest ever of the standard UNIX utilities, \IT{/bin/true} and \IT{/bin/false} return 0 and 1 respectively, as an exit code.
(Zero as an exit code usually means success, non-zero means error.)
}
\RU{\subsubsection{std::string}
\myindex{\Cpp!STL!std::string}
\label{std_string}

\myparagraph{Как устроена структура}

Многие строковые библиотеки \InSqBrackets{\CNotes 2.2} обеспечивают структуру содержащую ссылку 
на буфер собственно со строкой, переменная всегда содержащую длину строки 
(что очень удобно для массы функций \InSqBrackets{\CNotes 2.2.1}) и переменную содержащую текущий размер буфера.

Строка в буфере обыкновенно оканчивается нулем: это для того чтобы указатель на буфер можно было
передавать в функции требующие на вход обычную сишную \ac{ASCIIZ}-строку.

Стандарт \Cpp не описывает, как именно нужно реализовывать std::string,
но, как правило, они реализованы как описано выше, с небольшими дополнениями.

Строки в \Cpp это не класс (как, например, QString в Qt), а темплейт (basic\_string), 
это сделано для того чтобы поддерживать 
строки содержащие разного типа символы: как минимум \Tchar и \IT{wchar\_t}.

Так что, std::string это класс с базовым типом \Tchar.

А std::wstring это класс с базовым типом \IT{wchar\_t}.

\mysubparagraph{MSVC}

В реализации MSVC, вместо ссылки на буфер может содержаться сам буфер (если строка короче 16-и символов).

Это означает, что каждая короткая строка будет занимать в памяти по крайней мере $16 + 4 + 4 = 24$ 
байт для 32-битной среды либо $16 + 8 + 8 = 32$ 
байта в 64-битной, а если строка длиннее 16-и символов, то прибавьте еще длину самой строки.

\lstinputlisting[caption=пример для MSVC,style=customc]{\CURPATH/STL/string/MSVC_RU.cpp}

Собственно, из этого исходника почти всё ясно.

Несколько замечаний:

Если строка короче 16-и символов, 
то отдельный буфер для строки в \glslink{heap}{куче} выделяться не будет.

Это удобно потому что на практике, основная часть строк действительно короткие.
Вероятно, разработчики в Microsoft выбрали размер в 16 символов как разумный баланс.

Теперь очень важный момент в конце функции main(): мы не пользуемся методом c\_str(), тем не менее,
если это скомпилировать и запустить, то обе строки появятся в консоли!

Работает это вот почему.

В первом случае строка короче 16-и символов и в начале объекта std::string (его можно рассматривать
просто как структуру) расположен буфер с этой строкой.
\printf трактует указатель как указатель на массив символов оканчивающийся нулем и поэтому всё работает.

Вывод второй строки (длиннее 16-и символов) даже еще опаснее: это вообще типичная программистская ошибка 
(или опечатка), забыть дописать c\_str().
Это работает потому что в это время в начале структуры расположен указатель на буфер.
Это может надолго остаться незамеченным: до тех пока там не появится строка 
короче 16-и символов, тогда процесс упадет.

\mysubparagraph{GCC}

В реализации GCC в структуре есть еще одна переменная --- reference count.

Интересно, что указатель на экземпляр класса std::string в GCC указывает не на начало самой структуры, 
а на указатель на буфера.
В libstdc++-v3\textbackslash{}include\textbackslash{}bits\textbackslash{}basic\_string.h 
мы можем прочитать что это сделано для удобства отладки:

\begin{lstlisting}
   *  The reason you want _M_data pointing to the character %array and
   *  not the _Rep is so that the debugger can see the string
   *  contents. (Probably we should add a non-inline member to get
   *  the _Rep for the debugger to use, so users can check the actual
   *  string length.)
\end{lstlisting}

\href{http://go.yurichev.com/17085}{исходный код basic\_string.h}

В нашем примере мы учитываем это:

\lstinputlisting[caption=пример для GCC,style=customc]{\CURPATH/STL/string/GCC_RU.cpp}

Нужны еще небольшие хаки чтобы сымитировать типичную ошибку, которую мы уже видели выше, из-за
более ужесточенной проверки типов в GCC, тем не менее, printf() работает и здесь без c\_str().

\myparagraph{Чуть более сложный пример}

\lstinputlisting[style=customc]{\CURPATH/STL/string/3.cpp}

\lstinputlisting[caption=MSVC 2012,style=customasmx86]{\CURPATH/STL/string/3_MSVC_RU.asm}

Собственно, компилятор не конструирует строки статически: да в общем-то и как
это возможно, если буфер с ней нужно хранить в \glslink{heap}{куче}?

Вместо этого в сегменте данных хранятся обычные \ac{ASCIIZ}-строки, а позже, во время выполнения, 
при помощи метода \q{assign}, конструируются строки s1 и s2
.
При помощи \TT{operator+}, создается строка s3.

Обратите внимание на то что вызов метода c\_str() отсутствует,
потому что его код достаточно короткий и компилятор вставил его прямо здесь:
если строка короче 16-и байт, то в регистре EAX остается указатель на буфер,
а если длиннее, то из этого же места достается адрес на буфер расположенный в \glslink{heap}{куче}.

Далее следуют вызовы трех деструкторов, причем, они вызываются только если строка длиннее 16-и байт:
тогда нужно освободить буфера в \glslink{heap}{куче}.
В противном случае, так как все три объекта std::string хранятся в стеке,
они освобождаются автоматически после выхода из функции.

Следовательно, работа с короткими строками более быстрая из-за м\'{е}ньшего обращения к \glslink{heap}{куче}.

Код на GCC даже проще (из-за того, что в GCC, как мы уже видели, не реализована возможность хранить короткую
строку прямо в структуре):

% TODO1 comment each function meaning
\lstinputlisting[caption=GCC 4.8.1,style=customasmx86]{\CURPATH/STL/string/3_GCC_RU.s}

Можно заметить, что в деструкторы передается не указатель на объект,
а указатель на место за 12 байт (или 3 слова) перед ним, то есть, на настоящее начало структуры.

\myparagraph{std::string как глобальная переменная}
\label{sec:std_string_as_global_variable}

Опытные программисты на \Cpp знают, что глобальные переменные \ac{STL}-типов вполне можно объявлять.

Да, действительно:

\lstinputlisting[style=customc]{\CURPATH/STL/string/5.cpp}

Но как и где будет вызываться конструктор \TT{std::string}?

На самом деле, эта переменная будет инициализирована даже перед началом \main.

\lstinputlisting[caption=MSVC 2012: здесь конструируется глобальная переменная{,} а также регистрируется её деструктор,style=customasmx86]{\CURPATH/STL/string/5_MSVC_p2.asm}

\lstinputlisting[caption=MSVC 2012: здесь глобальная переменная используется в \main,style=customasmx86]{\CURPATH/STL/string/5_MSVC_p1.asm}

\lstinputlisting[caption=MSVC 2012: эта функция-деструктор вызывается перед выходом,style=customasmx86]{\CURPATH/STL/string/5_MSVC_p3.asm}

\myindex{\CStandardLibrary!atexit()}
В реальности, из \ac{CRT}, еще до вызова main(), вызывается специальная функция,
в которой перечислены все конструкторы подобных переменных.
Более того: при помощи atexit() регистрируется функция, которая будет вызвана в конце работы программы:
в этой функции компилятор собирает вызовы деструкторов всех подобных глобальных переменных.

GCC работает похожим образом:

\lstinputlisting[caption=GCC 4.8.1,style=customasmx86]{\CURPATH/STL/string/5_GCC.s}

Но он не выделяет отдельной функции в которой будут собраны деструкторы: 
каждый деструктор передается в atexit() по одному.

% TODO а если глобальная STL-переменная в другом модуле? надо проверить.

}
\DE{\subsection{Einfachste XOR-Verschlüsselung überhaupt}

Ich habe einmal eine Software gesehen, bei der alle Debugging-Ausgaben mit XOR mit dem Wert 3
verschlüsselt wurden. Mit anderen Worten, die beiden niedrigsten Bits aller Buchstaben wurden invertiert.

``Hello, world'' wurde zu ``Kfool/\#tlqog'':

\begin{lstlisting}
#!/usr/bin/python

msg="Hello, world!"

print "".join(map(lambda x: chr(ord(x)^3), msg))
\end{lstlisting}

Das ist eine ziemlich interessante Verschlüsselung (oder besser eine Verschleierung),
weil sie zwei wichtige Eigenschaften hat:
1) es ist eine einzige Funktion zum Verschlüsseln und entschlüsseln, sie muss nur wiederholt angewendet werden
2) die entstehenden Buchstaben befinden sich im druckbaren Bereich, also die ganze Zeichenkette kann ohne
Escape-Symbole im Code verwendet werden.

Die zweite Eigenschaft nutzt die Tatsache, dass alle druckbaren Zeichen in Reihen organisiert sind: 0x2x-0x7x,
und wenn die beiden niederwertigsten Bits invertiert werden, wird der Buchstabe um eine oder drei Stellen nach
links oder rechts \IT{verschoben}, aber niemals in eine andere Reihe:

\begin{figure}[H]
\centering
\includegraphics[width=0.7\textwidth]{ascii_clean.png}
\caption{7-Bit \ac{ASCII} Tabelle in Emacs}
\end{figure}

\dots mit dem Zeichen 0x7F als einziger Ausnahme.

Im Folgenden werden also beispielsweise die Zeichen A-Z \IT{verschlüsselt}:

\begin{lstlisting}
#!/usr/bin/python

msg="@ABCDEFGHIJKLMNO"

print "".join(map(lambda x: chr(ord(x)^3), msg))
\end{lstlisting}

Ergebnis:
% FIXME \verb  --  relevant comment for German?
\begin{lstlisting}
CBA@GFEDKJIHONML
\end{lstlisting}

Es sieht so aus als würden die Zeichen ``@'' und ``C'' sowie ``B'' und ``A'' vertauscht werden.

Hier ist noch ein interessantes Beispiel, in dem gezeigt wird, wie die Eigenschaften von XOR
ausgenutzt werden können: Exakt den gleichen Effekt, dass druckbare Zeichen auch druckbar bleiben,
kann man dadurch erzielen, dass irgendeine Kombination der niedrigsten vier Bits invertiert wird.
}

\EN{\section{Returning Values}
\label{ret_val_func}

Another simple function is the one that simply returns a constant value:

\lstinputlisting[caption=\EN{\CCpp Code},style=customc]{patterns/011_ret/1.c}

Let's compile it.

\subsection{x86}

Here's what both the GCC and MSVC compilers produce (with optimization) on the x86 platform:

\lstinputlisting[caption=\Optimizing GCC/MSVC (\assemblyOutput),style=customasmx86]{patterns/011_ret/1.s}

\myindex{x86!\Instructions!RET}
There are just two instructions: the first places the value 123 into the \EAX register,
which is used by convention for storing the return
value, and the second one is \RET, which returns execution to the \gls{caller}.

The caller will take the result from the \EAX register.

\subsection{ARM}

There are a few differences on the ARM platform:

\lstinputlisting[caption=\OptimizingKeilVI (\ARMMode) ASM Output,style=customasmARM]{patterns/011_ret/1_Keil_ARM_O3.s}

ARM uses the register \Reg{0} for returning the results of functions, so 123 is copied into \Reg{0}.

\myindex{ARM!\Instructions!MOV}
\myindex{x86!\Instructions!MOV}
It is worth noting that \MOV is a misleading name for the instruction in both the x86 and ARM \ac{ISA}s.

The data is not in fact \IT{moved}, but \IT{copied}.

\subsection{MIPS}

\label{MIPS_leaf_function_ex1}

The GCC assembly output below lists registers by number:

\lstinputlisting[caption=\Optimizing GCC 4.4.5 (\assemblyOutput),style=customasmMIPS]{patterns/011_ret/MIPS.s}

\dots while \IDA does it by their pseudo names:

\lstinputlisting[caption=\Optimizing GCC 4.4.5 (IDA),style=customasmMIPS]{patterns/011_ret/MIPS_IDA.lst}

The \$2 (or \$V0) register is used to store the function's return value.
\myindex{MIPS!\Pseudoinstructions!LI}
\INS{LI} stands for ``Load Immediate'' and is the MIPS equivalent to \MOV.

\myindex{MIPS!\Instructions!J}
The other instruction is the jump instruction (J or JR) which returns the execution flow to the \gls{caller}.

\myindex{MIPS!Branch delay slot}
You might be wondering why the positions of the load instruction (LI) and the jump instruction (J or JR) are swapped. This is due to a \ac{RISC} feature called ``branch delay slot''.

The reason this happens is a quirk in the architecture of some RISC \ac{ISA}s and isn't important for our
purposes---we must simply keep in mind that in MIPS, the instruction following a jump or branch instruction
is executed \IT{before} the jump/branch instruction itself.

As a consequence, branch instructions always swap places with the instruction executed immediately beforehand.


In practice, functions which merely return 1 (\IT{true}) or 0 (\IT{false}) are very frequent.

The smallest ever of the standard UNIX utilities, \IT{/bin/true} and \IT{/bin/false} return 0 and 1 respectively, as an exit code.
(Zero as an exit code usually means success, non-zero means error.)
}
\RU{\subsubsection{std::string}
\myindex{\Cpp!STL!std::string}
\label{std_string}

\myparagraph{Как устроена структура}

Многие строковые библиотеки \InSqBrackets{\CNotes 2.2} обеспечивают структуру содержащую ссылку 
на буфер собственно со строкой, переменная всегда содержащую длину строки 
(что очень удобно для массы функций \InSqBrackets{\CNotes 2.2.1}) и переменную содержащую текущий размер буфера.

Строка в буфере обыкновенно оканчивается нулем: это для того чтобы указатель на буфер можно было
передавать в функции требующие на вход обычную сишную \ac{ASCIIZ}-строку.

Стандарт \Cpp не описывает, как именно нужно реализовывать std::string,
но, как правило, они реализованы как описано выше, с небольшими дополнениями.

Строки в \Cpp это не класс (как, например, QString в Qt), а темплейт (basic\_string), 
это сделано для того чтобы поддерживать 
строки содержащие разного типа символы: как минимум \Tchar и \IT{wchar\_t}.

Так что, std::string это класс с базовым типом \Tchar.

А std::wstring это класс с базовым типом \IT{wchar\_t}.

\mysubparagraph{MSVC}

В реализации MSVC, вместо ссылки на буфер может содержаться сам буфер (если строка короче 16-и символов).

Это означает, что каждая короткая строка будет занимать в памяти по крайней мере $16 + 4 + 4 = 24$ 
байт для 32-битной среды либо $16 + 8 + 8 = 32$ 
байта в 64-битной, а если строка длиннее 16-и символов, то прибавьте еще длину самой строки.

\lstinputlisting[caption=пример для MSVC,style=customc]{\CURPATH/STL/string/MSVC_RU.cpp}

Собственно, из этого исходника почти всё ясно.

Несколько замечаний:

Если строка короче 16-и символов, 
то отдельный буфер для строки в \glslink{heap}{куче} выделяться не будет.

Это удобно потому что на практике, основная часть строк действительно короткие.
Вероятно, разработчики в Microsoft выбрали размер в 16 символов как разумный баланс.

Теперь очень важный момент в конце функции main(): мы не пользуемся методом c\_str(), тем не менее,
если это скомпилировать и запустить, то обе строки появятся в консоли!

Работает это вот почему.

В первом случае строка короче 16-и символов и в начале объекта std::string (его можно рассматривать
просто как структуру) расположен буфер с этой строкой.
\printf трактует указатель как указатель на массив символов оканчивающийся нулем и поэтому всё работает.

Вывод второй строки (длиннее 16-и символов) даже еще опаснее: это вообще типичная программистская ошибка 
(или опечатка), забыть дописать c\_str().
Это работает потому что в это время в начале структуры расположен указатель на буфер.
Это может надолго остаться незамеченным: до тех пока там не появится строка 
короче 16-и символов, тогда процесс упадет.

\mysubparagraph{GCC}

В реализации GCC в структуре есть еще одна переменная --- reference count.

Интересно, что указатель на экземпляр класса std::string в GCC указывает не на начало самой структуры, 
а на указатель на буфера.
В libstdc++-v3\textbackslash{}include\textbackslash{}bits\textbackslash{}basic\_string.h 
мы можем прочитать что это сделано для удобства отладки:

\begin{lstlisting}
   *  The reason you want _M_data pointing to the character %array and
   *  not the _Rep is so that the debugger can see the string
   *  contents. (Probably we should add a non-inline member to get
   *  the _Rep for the debugger to use, so users can check the actual
   *  string length.)
\end{lstlisting}

\href{http://go.yurichev.com/17085}{исходный код basic\_string.h}

В нашем примере мы учитываем это:

\lstinputlisting[caption=пример для GCC,style=customc]{\CURPATH/STL/string/GCC_RU.cpp}

Нужны еще небольшие хаки чтобы сымитировать типичную ошибку, которую мы уже видели выше, из-за
более ужесточенной проверки типов в GCC, тем не менее, printf() работает и здесь без c\_str().

\myparagraph{Чуть более сложный пример}

\lstinputlisting[style=customc]{\CURPATH/STL/string/3.cpp}

\lstinputlisting[caption=MSVC 2012,style=customasmx86]{\CURPATH/STL/string/3_MSVC_RU.asm}

Собственно, компилятор не конструирует строки статически: да в общем-то и как
это возможно, если буфер с ней нужно хранить в \glslink{heap}{куче}?

Вместо этого в сегменте данных хранятся обычные \ac{ASCIIZ}-строки, а позже, во время выполнения, 
при помощи метода \q{assign}, конструируются строки s1 и s2
.
При помощи \TT{operator+}, создается строка s3.

Обратите внимание на то что вызов метода c\_str() отсутствует,
потому что его код достаточно короткий и компилятор вставил его прямо здесь:
если строка короче 16-и байт, то в регистре EAX остается указатель на буфер,
а если длиннее, то из этого же места достается адрес на буфер расположенный в \glslink{heap}{куче}.

Далее следуют вызовы трех деструкторов, причем, они вызываются только если строка длиннее 16-и байт:
тогда нужно освободить буфера в \glslink{heap}{куче}.
В противном случае, так как все три объекта std::string хранятся в стеке,
они освобождаются автоматически после выхода из функции.

Следовательно, работа с короткими строками более быстрая из-за м\'{е}ньшего обращения к \glslink{heap}{куче}.

Код на GCC даже проще (из-за того, что в GCC, как мы уже видели, не реализована возможность хранить короткую
строку прямо в структуре):

% TODO1 comment each function meaning
\lstinputlisting[caption=GCC 4.8.1,style=customasmx86]{\CURPATH/STL/string/3_GCC_RU.s}

Можно заметить, что в деструкторы передается не указатель на объект,
а указатель на место за 12 байт (или 3 слова) перед ним, то есть, на настоящее начало структуры.

\myparagraph{std::string как глобальная переменная}
\label{sec:std_string_as_global_variable}

Опытные программисты на \Cpp знают, что глобальные переменные \ac{STL}-типов вполне можно объявлять.

Да, действительно:

\lstinputlisting[style=customc]{\CURPATH/STL/string/5.cpp}

Но как и где будет вызываться конструктор \TT{std::string}?

На самом деле, эта переменная будет инициализирована даже перед началом \main.

\lstinputlisting[caption=MSVC 2012: здесь конструируется глобальная переменная{,} а также регистрируется её деструктор,style=customasmx86]{\CURPATH/STL/string/5_MSVC_p2.asm}

\lstinputlisting[caption=MSVC 2012: здесь глобальная переменная используется в \main,style=customasmx86]{\CURPATH/STL/string/5_MSVC_p1.asm}

\lstinputlisting[caption=MSVC 2012: эта функция-деструктор вызывается перед выходом,style=customasmx86]{\CURPATH/STL/string/5_MSVC_p3.asm}

\myindex{\CStandardLibrary!atexit()}
В реальности, из \ac{CRT}, еще до вызова main(), вызывается специальная функция,
в которой перечислены все конструкторы подобных переменных.
Более того: при помощи atexit() регистрируется функция, которая будет вызвана в конце работы программы:
в этой функции компилятор собирает вызовы деструкторов всех подобных глобальных переменных.

GCC работает похожим образом:

\lstinputlisting[caption=GCC 4.8.1,style=customasmx86]{\CURPATH/STL/string/5_GCC.s}

Но он не выделяет отдельной функции в которой будут собраны деструкторы: 
каждый деструктор передается в atexit() по одному.

% TODO а если глобальная STL-переменная в другом модуле? надо проверить.

}
\DE{\subsection{Einfachste XOR-Verschlüsselung überhaupt}

Ich habe einmal eine Software gesehen, bei der alle Debugging-Ausgaben mit XOR mit dem Wert 3
verschlüsselt wurden. Mit anderen Worten, die beiden niedrigsten Bits aller Buchstaben wurden invertiert.

``Hello, world'' wurde zu ``Kfool/\#tlqog'':

\begin{lstlisting}
#!/usr/bin/python

msg="Hello, world!"

print "".join(map(lambda x: chr(ord(x)^3), msg))
\end{lstlisting}

Das ist eine ziemlich interessante Verschlüsselung (oder besser eine Verschleierung),
weil sie zwei wichtige Eigenschaften hat:
1) es ist eine einzige Funktion zum Verschlüsseln und entschlüsseln, sie muss nur wiederholt angewendet werden
2) die entstehenden Buchstaben befinden sich im druckbaren Bereich, also die ganze Zeichenkette kann ohne
Escape-Symbole im Code verwendet werden.

Die zweite Eigenschaft nutzt die Tatsache, dass alle druckbaren Zeichen in Reihen organisiert sind: 0x2x-0x7x,
und wenn die beiden niederwertigsten Bits invertiert werden, wird der Buchstabe um eine oder drei Stellen nach
links oder rechts \IT{verschoben}, aber niemals in eine andere Reihe:

\begin{figure}[H]
\centering
\includegraphics[width=0.7\textwidth]{ascii_clean.png}
\caption{7-Bit \ac{ASCII} Tabelle in Emacs}
\end{figure}

\dots mit dem Zeichen 0x7F als einziger Ausnahme.

Im Folgenden werden also beispielsweise die Zeichen A-Z \IT{verschlüsselt}:

\begin{lstlisting}
#!/usr/bin/python

msg="@ABCDEFGHIJKLMNO"

print "".join(map(lambda x: chr(ord(x)^3), msg))
\end{lstlisting}

Ergebnis:
% FIXME \verb  --  relevant comment for German?
\begin{lstlisting}
CBA@GFEDKJIHONML
\end{lstlisting}

Es sieht so aus als würden die Zeichen ``@'' und ``C'' sowie ``B'' und ``A'' vertauscht werden.

Hier ist noch ein interessantes Beispiel, in dem gezeigt wird, wie die Eigenschaften von XOR
ausgenutzt werden können: Exakt den gleichen Effekt, dass druckbare Zeichen auch druckbar bleiben,
kann man dadurch erzielen, dass irgendeine Kombination der niedrigsten vier Bits invertiert wird.
}

\ifdefined\SPANISH
\chapter{Patrones de código}
\fi % SPANISH

\ifdefined\GERMAN
\chapter{Code-Muster}
\fi % GERMAN

\ifdefined\ENGLISH
\chapter{Code Patterns}
\fi % ENGLISH

\ifdefined\ITALIAN
\chapter{Forme di codice}
\fi % ITALIAN

\ifdefined\RUSSIAN
\chapter{Образцы кода}
\fi % RUSSIAN

\ifdefined\BRAZILIAN
\chapter{Padrões de códigos}
\fi % BRAZILIAN

\ifdefined\THAI
\chapter{รูปแบบของโค้ด}
\fi % THAI

\ifdefined\FRENCH
\chapter{Modèle de code}
\fi % FRENCH

\ifdefined\POLISH
\chapter{\PLph{}}
\fi % POLISH

% sections
\EN{\input{patterns/patterns_opt_dbg_EN}}
\ES{\input{patterns/patterns_opt_dbg_ES}}
\ITA{\input{patterns/patterns_opt_dbg_ITA}}
\PTBR{\input{patterns/patterns_opt_dbg_PTBR}}
\RU{\input{patterns/patterns_opt_dbg_RU}}
\THA{\input{patterns/patterns_opt_dbg_THA}}
\DE{\input{patterns/patterns_opt_dbg_DE}}
\FR{\input{patterns/patterns_opt_dbg_FR}}
\PL{\input{patterns/patterns_opt_dbg_PL}}

\RU{\section{Некоторые базовые понятия}}
\EN{\section{Some basics}}
\DE{\section{Einige Grundlagen}}
\FR{\section{Quelques bases}}
\ES{\section{\ESph{}}}
\ITA{\section{Alcune basi teoriche}}
\PTBR{\section{\PTBRph{}}}
\THA{\section{\THAph{}}}
\PL{\section{\PLph{}}}

% sections:
\EN{\input{patterns/intro_CPU_ISA_EN}}
\ES{\input{patterns/intro_CPU_ISA_ES}}
\ITA{\input{patterns/intro_CPU_ISA_ITA}}
\PTBR{\input{patterns/intro_CPU_ISA_PTBR}}
\RU{\input{patterns/intro_CPU_ISA_RU}}
\DE{\input{patterns/intro_CPU_ISA_DE}}
\FR{\input{patterns/intro_CPU_ISA_FR}}
\PL{\input{patterns/intro_CPU_ISA_PL}}

\EN{\input{patterns/numeral_EN}}
\RU{\input{patterns/numeral_RU}}
\ITA{\input{patterns/numeral_ITA}}
\DE{\input{patterns/numeral_DE}}
\FR{\input{patterns/numeral_FR}}
\PL{\input{patterns/numeral_PL}}

% chapters
\input{patterns/00_empty/main}
\input{patterns/011_ret/main}
\input{patterns/01_helloworld/main}
\input{patterns/015_prolog_epilogue/main}
\input{patterns/02_stack/main}
\input{patterns/03_printf/main}
\input{patterns/04_scanf/main}
\input{patterns/05_passing_arguments/main}
\input{patterns/06_return_results/main}
\input{patterns/061_pointers/main}
\input{patterns/065_GOTO/main}
\input{patterns/07_jcc/main}
\input{patterns/08_switch/main}
\input{patterns/09_loops/main}
\input{patterns/10_strings/main}
\input{patterns/11_arith_optimizations/main}
\input{patterns/12_FPU/main}
\input{patterns/13_arrays/main}
\input{patterns/14_bitfields/main}
\EN{\input{patterns/145_LCG/main_EN}}
\RU{\input{patterns/145_LCG/main_RU}}
\input{patterns/15_structs/main}
\input{patterns/17_unions/main}
\input{patterns/18_pointers_to_functions/main}
\input{patterns/185_64bit_in_32_env/main}

\EN{\input{patterns/19_SIMD/main_EN}}
\RU{\input{patterns/19_SIMD/main_RU}}
\DE{\input{patterns/19_SIMD/main_DE}}

\EN{\input{patterns/20_x64/main_EN}}
\RU{\input{patterns/20_x64/main_RU}}

\EN{\input{patterns/205_floating_SIMD/main_EN}}
\RU{\input{patterns/205_floating_SIMD/main_RU}}
\DE{\input{patterns/205_floating_SIMD/main_DE}}

\EN{\input{patterns/ARM/main_EN}}
\RU{\input{patterns/ARM/main_RU}}
\DE{\input{patterns/ARM/main_DE}}

\input{patterns/MIPS/main}




\subsection{Some constants}

It's easy to find representations of some constants in Wikipedia for IEEE 754 encoded numbers.
It's interesting to know that 0.0 in IEEE 754 is represented as 32 zero bits (for single precision) or 64 zero bits
(for double).
So in order to set a floating point variable to 0.0 in register or memory, one can use \MOV or \TT{XOR reg, reg} instruction.
\myindex{\CStandardLibrary!memset()}
This is suitable for structures where many variables present of various data types.
With usual memset() function one can set all integer variables to 0, all boolean variables to \IT{false}, all pointers
to NULL, and all floating point variables (of any precision) to 0.0.

\subsection{Copying}

One may think inertially that \INS{FLD}/\INS{FST} instructions must be used to load and store (and hence, copy) IEEE 754 values.
Nevertheless, same can be achieved easier by usual \INS{MOV} instruction, which, of course, copies values bitwisely.

\subsection{Stack, calculators and reverse Polish notation}

\myindex{Reverse Polish notation}

Now we understand why some old calculators use reverse Polish notation
\footnote{\href{http://go.yurichev.com/17354}{wikipedia.org/wiki/Reverse\_Polish\_notation}}.

For example, for addition of 12 and 34 one has to enter 12, then 34, then press \q{plus} sign.

It's because old calculators were just stack machine implementations, and this was much simpler
than to handle complex parenthesized expressions.

\subsection{80 bits?}

\myindex{Punched card}
Internal numbers representation in FPU --- 80-bit.
Strange number, because the number not in $2^n$ form.
There is a hypothesis that this is probably due to historical reasons---the standard IBM puched card can encode 12 rows of 80 bits.
$80\cdot 25$ text mode resolution was also popular in past.
If you know better, please a drop email to the author: \EMAIL{}.

\subsection{x64}

On how floating point numbers are processed in x86-64, read more here: \myref{floating_SIMD}.

% sections
\input{patterns/12_FPU/exercises}
