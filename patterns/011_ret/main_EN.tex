\section{Returning Values}
\label{ret_val_func}

Another simple function is the one that simply returns a constant value:

\lstinputlisting[caption=\EN{\CCpp Code},style=customc]{patterns/011_ret/1.c}

Let's compile it.

\subsection{x86}

Here's what both the GCC and MSVC compilers produce (with optimization) on the x86 platform:

\lstinputlisting[caption=\Optimizing GCC/MSVC (\assemblyOutput),style=customasmx86]{patterns/011_ret/1.s}

\myindex{x86!\Instructions!RET}
There are just two instructions: the first places the value 123 into the \EAX register,
which is used by convention for storing the return
value, and the second one is \RET, which returns execution to the \gls{caller}.

The caller will take the result from the \EAX register.

\subsection{ARM}

There are a few differences on the ARM platform:

\lstinputlisting[caption=\OptimizingKeilVI (\ARMMode) ASM Output,style=customasmARM]{patterns/011_ret/1_Keil_ARM_O3.s}

ARM uses the register \Reg{0} for returning the results of functions, so 123 is copied into \Reg{0}.

\myindex{ARM!\Instructions!MOV}
\myindex{x86!\Instructions!MOV}
It is worth noting that \MOV is a misleading name for the instruction in both the x86 and ARM \ac{ISA}s.

The data is not in fact \IT{moved}, but \IT{copied}.

\subsection{MIPS}

\label{MIPS_leaf_function_ex1}

The GCC assembly output below lists registers by number:

\lstinputlisting[caption=\Optimizing GCC 4.4.5 (\assemblyOutput),style=customasmMIPS]{patterns/011_ret/MIPS.s}

\dots while \IDA does it by their pseudo names:

\lstinputlisting[caption=\Optimizing GCC 4.4.5 (IDA),style=customasmMIPS]{patterns/011_ret/MIPS_IDA.lst}

The \$2 (or \$V0) register is used to store the function's return value.
\myindex{MIPS!\Pseudoinstructions!LI}
\INS{LI} stands for ``Load Immediate'' and is the MIPS equivalent to \MOV.

\myindex{MIPS!\Instructions!J}
The other instruction is the jump instruction (J or JR) which returns the execution flow to the \gls{caller}.

\myindex{MIPS!Branch delay slot}
You might be wondering why the positions of the load instruction (LI) and the jump instruction (J or JR) are swapped. This is due to a \ac{RISC} feature called ``branch delay slot''.

The reason this happens is a quirk in the architecture of some RISC \ac{ISA}s and isn't important for our
purposes---we must simply keep in mind that in MIPS, the instruction following a jump or branch instruction
is executed \IT{before} the jump/branch instruction itself.

As a consequence, branch instructions always swap places with the instruction executed immediately beforehand.


In practice, functions which merely return 1 (\IT{true}) or 0 (\IT{false}) are very frequent.

The smallest ever of the standard UNIX utilities, \IT{/bin/true} and \IT{/bin/false} return 0 and 1 respectively, as an exit code.
(Zero as an exit code usually means success, non-zero means error.)
