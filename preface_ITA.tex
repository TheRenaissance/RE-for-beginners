% TODO to be synced with EN version
\section*{Prefazione}

Ci sono diversi significati per il termine \q{\gls{reverse engineering}}:
1) Il reverse engineering del software: fare ricerca su programmi compilati;
2) La scansione delle strutture 3D e le successive manipolazioni digitali necessarie per duplicarle;
3) Ricreare struttura \ac{DBMS}.
Questo libro riguarda il primo significato.

\subsection*{Esercizi e compiti}

\dots 
spostati al seguente sito: \url{http://challenges.re}.

\subsection*{A proposito dell'autore}
\begin{tabularx}{\textwidth}{ l X }

\raisebox{-\totalheight}{
\includegraphics[scale=0.60]{Dennis_Yurichev.jpg}
}

&
Dennis Yurichev è un esperto reverse engineer.
Può essere contattato al seguente indirizzo e-mail: \textbf{\EMAIL{}}.

% FIXME: no link. \tablefootnote doesn't work
\end{tabularx}

% subsections:
\input{praise}
\ifdefined\RUSSIAN
\newcommand{\PeopleMistakesInaccuraciesRusEng}{Станислав \q{Beaver} Бобрицкий, Александр Лысенко, Александр \q{Solar Designer} Песляк, Федерико Рамондино, Марк Уилсон, Ксения Галинская, Разихова Мейрамгуль Кайратовна, Анатолий Прокофьев}
\else
\newcommand{\PeopleMistakesInaccuraciesRusEng}{Stanislav \q{Beaver} Bobrytskyy, Alexander Lysenko, Alexander \q{Solar Designer} Peslyak, Federico Ramondino, Mark Wilson, Xenia Galinskaya, Razikhova Meiramgul Kayratovna, Anatoly Prokofiev}
\fi

\newcommand{\PeopleMistakesInaccuracies}{\PeopleMistakesInaccuraciesRusEng{}, Shell Rocket, Zhu Ruijin, Changmin Heo, Vitor Vidal, Stijn Crevits, Jean-Gregoire Foulon\footnote{\url{https://github.com/pixjuan}}, Ben L., Etienne Khan, Norbert Szetei\footnote{\url{https://github.com/73696e65}}, Marc Remy, Michael Hansen, Derk Barten, The Renaissance\footnote{\url{https://github.com/TheRenaissance}}.}

\newcommand{\PeopleItalianTranslators}{Federico Ramondino\footnote{\url{https://github.com/pinkrab}},
Paolo Stivanin\footnote{\url{https://github.com/paolostivanin}}, twyK}

\newcommand{\PeopleFrenchTranslators}{Florent Besnard\footnote{\url{https://github.com/besnardf}}, Marc Remy\footnote{\url{https://github.com/mremy}}, Baudouin Landais, Téo Dacquet\footnote{\url{https://github.com/T30rix}}}

\newcommand{\PeopleGermanTranslators}{Dennis Siekmeier\footnote{\url{https://github.com/DSiekmeier}},
Julius Angres\footnote{\url{https://github.com/JAngres}}, Dirk Loser\footnote{\url{https://github.com/PolymathMonkey}}, Clemens Tamme}

\newcommand{\PeopleSpanishTranslators}{Diego Boy, Luis Alberto Espinosa Calvo, Fernando Guida, Diogo Mussi}

\newcommand{\PeoplePTBRTranslators}{Thales Stevan de A. Gois, Diogo Mussi}

\newcommand{\PeoplePolishTranslators}{Kateryna Rozanova, Aleksander Mistewicz}

\newcommand{\FNGithubContributors}{\footnote{\url{https://github.com/dennis714/RE-for-beginners/graphs/contributors}}}

\EN{% Suggested name change: Acknowledgements:
%\subsection*{Acknowledgements}
\subsection*{Thanks}

For patiently answering all my questions: \HERMIT, Slava \q{Avid} Kazakov, SkullC0DEr.

For sending me notes about mistakes and inaccuracies: \PeopleMistakesInaccuracies{}

For helping me in other ways:
Andrew Zubinski,
Arnaud Patard (rtp on \#debian-arm IRC),
noshadow on \#gcc IRC,
Aliaksandr Autayeu,
Mohsen Mostafa Jokar.

For translating the book into Simplified Chinese:
Antiy Labs (\href{http://antiy.cn}{antiy.cn}), Archer.

For translating the book into Korean: Byungho Min.

For translating the book into Dutch: Cedric Sambre (AKA Midas).

For translating the book into Spanish: \PeopleSpanishTranslators{}.

For translating the book into Portuguese: \PeoplePTBRTranslators{}.

For translating the book into Italian: \PeopleItalianTranslators{}.

For translating the book into French: \PeopleFrenchTranslators{}.

For translating the book into German: \PeopleGermanTranslators{}.

For translating the book into Polish: \PeoplePolishTranslators{}.

For proofreading:
Alexander \q{Lstar} Chernenkiy,
Vladimir Botov,
Andrei Brazhuk,
Mark ``Logxen'' Cooper, Yuan Jochen Kang, Mal Malakov, Lewis Porter, Jarle Thorsen, Hong Xie.

Vasil Kolev\footnote{\url{https://vasil.ludost.net/}} did a great amount of work in proofreading and correcting many mistakes.

For illustrations and cover art: Andy Nechaevsky.

Thanks also to all the folks on github.com who have contributed notes and corrections\FNGithubContributors{}.

Many \LaTeX\ packages were used: I would like to thank the authors as well.

%This is where I think the subsections about the Korean, etc. versions should go - Renaissance:
%\subsection*{About the Korean translation}

%In January 2015, the Acorn publishing company (\href{http://www.acornpub.co.kr}{www.acornpub.co.kr}) in South Korea did a huge amount of work in translating and publishing
%this book (as it was in August 2014) into Korean.

%It's available now at \href{http://go.yurichev.com/17343}{their website}.

%\iffalse
%\begin{figure}[H]
%\centering
%\includegraphics[scale=0.3]{acorn_cover.jpg}
%\end{figure}
%\fi

%The translator is Byungho Min (\href{http://go.yurichev.com/17344}{twitter/tais9}).
%The cover art was done by the artistic Andy Nechaevsky, a friend of the author:
%\href{http://go.yurichev.com/17023}{facebook/andydinka}.
%Acorn also holds the copyright to the Korean translation.

%So, if you want to have a \IT{real} book on your shelf in Korean and
%want to support this work, it is now available for purchase.

%\subsection*{About the Persian/Farsi translation}

%In 2016 the book was translated by Mohsen Mostafa Jokar (who is also known to Iranian community for his translation of Radare manual\footnote{\url{http://rada.re/get/radare2book-persian.pdf}}).
%It is available on the publisher’s website\footnote{\url{http://goo.gl/2Tzx0H}} (Pendare Pars).

%Here is a link to a 40-page excerpt: \url{https://beginners.re/farsi.pdf}.

%National Library of Iran registration information: \url{http://opac.nlai.ir/opac-prod/bibliographic/4473995}.

%\subsection*{About the Chinese translation}

%In April 2017, translation to Chinese was completed by Chinese PTPress. They are also the Chinese translation copyright holders.

% The Chinese version is available for order here: \url{http://www.epubit.com.cn/book/details/4174}. A partial review and history behind the translation can be found here: \url{http://www.cptoday.cn/news/detail/3155}.

%% If you're using first-person (I, me) in the acknowledgments, you should change the following back to first-person, too - Ren.
%The principal translator is Archer, to whom the author owes very much. He was extremely meticulous (in a good sense) and reported most of the known mistakes and bugs, which is very important in literature such as this book.
%The author would recommend his services to any other author!

%The guys from \href{http://www.antiy.net/}{Antiy Labs} has also helped with translation. \href{http://www.epubit.com.cn/book/onlinechapter/51413}{Here is preface} written by them.


\subsubsection*{Donors}

Those who supported me during the time when I wrote significant part of the book:

2 * Oleg Vygovsky (50+100 UAH), 
Daniel Bilar ($\$$50), 
James Truscott ($\$$4.5),
Luis Rocha ($\$$63), 
Joris van de Vis ($\$$127), 
Richard S Shultz ($\$$20), 
Jang Minchang ($\$$20), 
Shade Atlas (5 AUD), 
Yao Xiao ($\$$10),
Pawel Szczur (40 CHF), 
Justin Simms ($\$$20), 
Shawn the R0ck ($\$$27), 
Ki Chan Ahn ($\$$50), 
Triop AB (100 SEK), 
Ange Albertini (\euro{}10+50),
Sergey Lukianov (300 RUR), 
Ludvig Gislason (200 SEK), 
Gérard Labadie (\euro{}40), 
Sergey Volchkov (10 AUD),
Vankayala Vigneswararao ($\$$50),
Philippe Teuwen ($\$$4),
Martin Haeberli ($\$$10),
Victor Cazacov (\euro{}5),
Tobias Sturzenegger (10 CHF),
Sonny Thai ($\$$15),
Bayna AlZaabi ($\$$75),
Redfive B.V. (\euro{}25),
Joona Oskari Heikkilä (\euro{}5),
Marshall Bishop ($\$$50),
Nicolas Werner (\euro{}12),
Jeremy Brown ($\$$100),
Alexandre Borges ($\$$25),
Vladimir Dikovski (\euro{}50),
Jiarui Hong (100.00 SEK),
Jim Di (500 RUR),
Tan Vincent ($\$$30),
Sri Harsha Kandrakota (10 AUD),
Pillay Harish (10 SGD),
Timur Valiev (230 RUR),
Carlos Garcia Prado (\euro{}10),
Salikov Alexander (500 RUR),
Oliver Whitehouse (30 GBP),
Katy Moe ($\$$14),
Maxim Dyakonov ($\$$3),
Sebastian Aguilera (\euro{}20),
Hans-Martin Münch (\euro{}15),
Jarle Thorsen (100 NOK),
Vitaly Osipov ($\$$100),
Yuri Romanov (1000 RUR),
Aliaksandr Autayeu (\euro{}10),
Tudor Azoitei ($\$$40),
Z0vsky (\euro{}10),
Yu Dai ($\$$10),
Anonymous ($\$$15),
Vladislav Chelnokov (($\$$25),
Nenad Noveljic (($\$$50).



Thanks a lot to every donor!
}
\ES{\input{thanks_ES}}
\NL{\input{thanks_NL}}
\RU{\input{thanks_RU}}
\ITA{\input{thanks_ITA}}
\FR{\input{thanks_FR}}
\DE{\input{thanks_DE}}
%\CN{\input{thanks_CN}}

\input{FAQ_ITA}

\subsection*{A proposito della traduzione Coreana}

A Gennaio 2015, la compagnia editrice Acorn (\href{http://www.acornpub.co.kr}{www.acornpub.co.kr}) sita in Sud Corea ha svolto un enorme lavoro nel tradurre e pubblicare il mio libro in Coreano.

Il libro è quindi disponibili tramite il \href{http://go.yurichev.com/17343}{loro sito}.

\iffalse
\begin{figure}[H]
\centering
\includegraphics[scale=0.3]{acorn_cover.jpg}
\end{figure}
\fi

Il traduttore è Byungho Min (\href{http://go.yurichev.com/17344}{twitter/tais9}).
La copertina è stata fatta dall'artista (e mio amico) Andy Nechaevsky:
\href{http://go.yurichev.com/17023}{facebook/andydinka}.
Loro detengono inoltre il copyright della versione Coreana.

Se vuoi avere una versione \IT{reale} (in Coreano) del mio libro nella tua libreria e 
vuoi inoltre supportare il mio lavoro, allora puoi acquistare la suddetta copia direttamente da loro.

%\subsection*{About the Persian/Farsi translation}
%TBT

