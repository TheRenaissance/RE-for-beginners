\documentclass[a4paper,oneside]{book}

% http://www.tex.ac.uk/FAQ-noroom.html
\usepackage{etex}

\usepackage[table,usenames,dvipsnames]{xcolor}

\usepackage{fontspec}
% fonts
%\setmonofont{DroidSansMono}
%\setmainfont[Ligatures=TeX]{PT Sans}
%\setmainfont{DroidSans}
\setmainfont{DejaVu Sans}
\setmonofont{DejaVu Sans Mono}
\usepackage{polyglossia}
\defaultfontfeatures{Scale=MatchLowercase} % ensure all fonts have the same 1ex
\usepackage{ucharclasses}
\usepackage{csquotes}

\ifdefined\ENGLISH
\setmainlanguage{english}
\setotherlanguage{russian}
\fi

\ifdefined\RUSSIAN
\setmainlanguage{russian}
%\newfontfamily\cyrillicfont{LiberationSans}
%\newfontfamily\cyrillicfonttt{LiberationMono}
%\newfontfamily\cyrillicfontsf{lmsans10-regular.otf}
\setotherlanguage{english}
\fi

\ifdefined\GERMAN
%\wlog{main GERMAN defined OK}
\setmainlanguage{german}
\setotherlanguage{english}
\fi

\ifdefined\SPANISH
\setmainlanguage{spanish}
\setotherlanguage{english}
\fi

\ifdefined\ITALIAN
\setmainlanguage{italian}
\setotherlanguage{english}
\fi

\ifdefined\BRAZILIAN
\setmainlanguage{portuges}
\setotherlanguage{english}
\fi

\ifdefined\POLISH
\setmainlanguage{polish}
\setotherlanguage{english}
\fi

\ifdefined\DUTCH
\setmainlanguage{dutch}
\setotherlanguage{english}
\fi

\ifdefined\THAI
\setmainlanguage{thai}
%\usepackage[thai]{babel}
%\usepackage{fonts-tlwg}
\setmainfont[Script=Thai]{TH SarabunPSK}
\newfontfamily{\thaifont}[Script=Thai]{TH SarabunPSK}
\let\thaifonttt\ttfamily
\setotherlanguage{english}
\fi

\ifdefined\FRENCH
\setmainlanguage{french}
\setotherlanguage{english}
\fi

\usepackage{microtype}
\usepackage{fancyhdr}
\usepackage{listings}
\usepackage{ulem}
\usepackage{url}
\usepackage{graphicx}
\usepackage{makeidx}
\usepackage[cm]{fullpage}
%\usepackage{color}
\usepackage{fancyvrb}
\usepackage{xspace}
\usepackage{tabularx}
\usepackage{framed}
\usepackage{parskip}
\usepackage{epigraph}
\usepackage{ccicons}
\usepackage[nottoc]{tocbibind}
\usepackage{longtable}
\usepackage[footnote,printonlyused,withpage]{acronym}
\usepackage[]{bookmark,hyperref} % must be last
\usepackage[official]{eurosym}
\usepackage[usestackEOL]{stackengine}

% ************** myref
% http://tex.stackexchange.com/questions/228286/how-to-mix-ref-and-pageref#228292
\ifdefined\RUSSIAN
\newcommand{\myref}[1]{%
  \ref{#1}
  (стр.~\pageref{#1})%
  }
% FIXME: I wasn't able to force varioref to output russian text...
\else
\usepackage{varioref}
\newcommand{\myref}[1]{\vref{#1}}
\fi
% ************** myref

\usepackage{glossaries}
\usepackage{tikz}
%\usepackage{fixltx2e}
\usepackage{bytefield}

\usepackage{amsmath}
\usepackage{MnSymbol}
\undef\mathdollar

\usepackage{float}

\usepackage{shorttoc}
\usetikzlibrary{calc,positioning,chains,arrows}
\usepackage[margin=0.5in,headheight=15.5pt]{geometry}

\newcommand{\footnoteref}[1]{\textsuperscript{\ref{#1}}}

%\definecolor{lstbgcolor}{rgb}{0.94,0.94,0.94}

% I don't know why this voodoo works, but without all-caps, it can't find LIGHT-GRAY color. WTF?
% see also: https://tex.stackexchange.com/questions/64298/error-with-xcolor-package
\definecolor{light-gray}{gray}{0.87}
\definecolor{LIGHT-GRAY}{gray}{0.87}
\definecolor{RED}{rgb}{1,0,0}
\makeindex

\include{macros}
\include{glossary}

\makeglossaries

\hypersetup{
    colorlinks=true,
    allcolors=blue,
    pdfauthor={\AUTHOR},
    pdftitle={\TITLE}
    }

%\ifdefined\RUSSIAN
\newcommand{\LstStyle}{\ttfamily\small}
%\else
%\newcommand{\LstStyle}{\ttfamily}
%\fi

% inspired by http://prismjs.com/
\definecolor{digits}{RGB}{0,0,0}
\definecolor{bg}{RGB}{255,252,250}
\definecolor{col1}{RGB}{154,20,150}
\definecolor{col2}{RGB}{112,128,144}
\definecolor{col3}{RGB}{10,120,180}
\definecolor{col4}{RGB}{106,164,108}


\lstset{
    %backgroundcolor=\color{lstbgcolor},
    %backgroundcolor=\color{light-gray},
    backgroundcolor=\color{bg},
    basicstyle=\LstStyle,
    breaklines=true,
    %prebreak=\raisebox{0ex}[0ex][0ex]{->},
    %postbreak=\raisebox{0ex}[0ex][0ex]{->},
    prebreak=\raisebox{0ex}[0ex][0ex]{\ensuremath{\rhookswarrow}},
    postbreak=\raisebox{0ex}[0ex][0ex]{\ensuremath{\rcurvearrowse\space}},
    frame=single,
    columns=fullflexible,keepspaces,
    escapeinside=§§,
    inputencoding=utf8
}

% I'm giving up with syntax highlighting so far.
% one problem is: make hexadecimal numbers and keywords distinct
% anyone can try to solve it
% see also: http://tex.stackexchange.com/questions/347784/package-listing-colorizing-assembly-listings

\lstdefinestyle{customc}{
  %language=C,
  %showstringspaces=false,
  %backgroundcolor=\color{bg},
  %keywordstyle=\color{col3},
  %commentstyle=\color{col2},
  %identifierstyle=\color{col4},
  %stringstyle=\color{col1}
}

\lstdefinestyle{custommath}{
  %language=Mathematica,
  %showstringspaces=false,
  %backgroundcolor=\color{bg},
  %keywordstyle=\color{col3},
  %commentstyle=\color{col2},
  %identifierstyle=\color{col4},
  %stringstyle=\color{col1}
}

\lstdefinestyle{custompy}{
  %language=Python,
  %showstringspaces=false,
  %backgroundcolor=\color{bg},
  %keywordstyle=\color{col3},
  %commentstyle=\color{col2},
  %identifierstyle=\color{col4},
  %stringstyle=\color{col1}
}

\lstdefinestyle{customjava}{
  %language=Java,
  %showstringspaces=false,
  %backgroundcolor=\color{bg},
  %keywordstyle=\color{col3},
  %commentstyle=\color{col2},
  %identifierstyle=\color{col4},
  %stringstyle=\color{col1}
}

\lstdefinestyle{customasmx86}{
  %morekeywords={push,mov,sub,call,and,leave,ret,lea,and,retn,xor,add,db,%
%	  eax,ebx,ecx,edx,esi,edi,esp,ebp,eip%
%	  rax,rbx,rcx,rdx,rsi,rdi,rbp,rsp,rip},%
  %alsoletter=.,alsodigit=?,%
  %alsodigit=abcdefhABCDEFhH,%
  %sensitive=f,%
  %morestring=[b]",%
  %morestring=[b]',%
  %morecomment=[l];%
  %showstringspaces=false,
  %basicstyle=\color{red},
  %backgroundcolor=\color{bg},
  %keywordstyle=\color{red},
  %commentstyle=\color{green},
  %identifierstyle=\color{black},
  %identifierstyle=\color{blue},
  %stringstyle=\color{yellow},
}

\lstdefinestyle{customasmARM}{
  %morekeywords={stmfd,mov,adr,bl,ldmfd,push,movs,pop,b,stp,add,adrp,ldp,ret,str,sub,movt,stmfa,movw,mov.w,movt.w,add.w,stmia.w,str.w,blx,%
%		  sp,pc,r0,r1,r2,r3,r4,r5,r6,r7,r8,r9,r10,r11,r12,lr,%
%		  x0,x1,x2,x3,x4,x5,x6,x7,x8,x9,x10,%
%		  w0,w1,w2,w3,w4,w5,w6,w7,w8,w9,w10},
  %alsoletter=.,alsodigit=?,%
  %sensitive=f,%
  %morestring=[b]",%
  %morestring=[b]',%
  %morecomment=[l];%
  %showstringspaces=false,
  %backgroundcolor=\color{bg},
  %keywordstyle=\color{col3},
  %commentstyle=\color{col2},
  %identifierstyle=\color{black},
  %identifierstyle=\color{col4},
  %stringstyle=\color{col1},
}

\lstdefinestyle{customasmMIPS}{
  %morekeywords={lui,addiu,sw,lw,li,jalr,move,j,la,jr,or,%
  %		\$0,\$1, \$2, \$3, \$4, \$5, \$6, \$7, \$8, \$9, \$10, \$11, \$12, \$13, \$14, \$15, \$16, \$17, \$18, \$19,%
  %		\$20, \$21, \$22, \$23, \$24, \$25, \$26, \$27, \$28, \$29, \$30,\$31,\$sp,\$gp,\$fp,%
%		\%lo,\%hi,
%		},
  %alsoletter=.,alsodigit=?,%
  %sensitive=f,%
  %morestring=[b]",%
  %morestring=[b]',%
  %morecomment=[l];%
  %showstringspaces=false,
  %backgroundcolor=\color{bg},
  %keywordstyle=\color{col3},
  %commentstyle=\color{col2},
  %identifierstyle=\color{black},
  %stringstyle=\color{col1},
}

\lstdefinestyle{customasmPPC}{
  %showstringspaces=false,
}

\ifdefined\RUSSIAN
\renewcommand\lstlistingname{Листинг}
\renewcommand\lstlistlistingname{Листинг}
\fi

\DeclareMathSizes{12}{30}{16}{12}%

% see also:
% http://tex.stackexchange.com/questions/129225/how-can-i-get-get-makeindex-to-ignore-capital-letters
% http://tex.stackexchange.com/questions/18336/correct-sorting-of-index-entries-containing-macros
\def\myindex#1{\expandafter\index\expandafter{#1}}

\begin{document}

% fancyhdr
\pagestyle{fancy}
\setlength{\headheight}{13pt}
\fancyhead[R]{} % suppress chapter name

\VerbatimFootnotes

\frontmatter

% I'm suggesting putting \include{praise.tex} here - Renaissance

% Suggested taking out 1st_page, putting the information in different places -Ren
\RU{\vspace*{\fill}

\huge Пожалуйста, жертвуйте
\normalsize

\bigskip
\bigskip
\bigskip

\dots чтобы я мог продолжать работать над этой книгой и другими статьями: \\
\url{https://yurichev.com/donate.html}.

\bigskip
\bigskip
\bigskip

\huge Внимание: соц.опрос
\normalsize

\bigskip
\bigskip
\bigskip

У меня есть идея заменить все примеры на OllyDbg в этой книге на примеры с использованием другого отладчика.
Я не имею ничего против OllyDbg, но он GUI-шный и использует маленькие шрифты, и скриншоты не очень подходят для книги.
Может я бы использовал GDB, radare или WinDbg.
Или может быть, какой-нибудь другой консольный отладчик?

Что вы об этом думаете?
Должен ли я оставить примеры с OllyDbg, или примеры с radare будут ОК?

E-Mail: \GTT{\EMAIL}.

\vspace*{\fill}
\vfill
}
\EN{\vspace*{\fill}

% See comments at the beginning of preface_EN.tex -
% I'd make this into a new file - reverse_engineering_services.tex,
% and put it at the end - Renaissance
\huge Reverse Engineering Services
\normalsize

\bigskip
\bigskip
\bigskip


I have tried many jobs in my life, but, surprisingly (even to myself),
the job I'm the most proud of is rewriting large piece(s) of compiled code back to C/C++.
This is an extremely boring and slow process. I once spent more than a year on rewriting 100KB DLL to pure C,
and it was like a full-time job.

And expensive.

% I think I got the intent right (original: By the way, a lot of tricks has been added to this book during this work.)
A lot of tricks have been added to this book as a result of this work.

I assume such a service could be interesting to those who inherited some compiled code with no source code.
You are welcome to contact me at:

E-Mail: \GTT{\EMAIL}.

\bigskip
\bigskip
\bigskip

% I'd turn this into a new section, and keep it towards the beginning, probably
% right after the Acknowledgements - Ren.
\huge Please Donate
\normalsize

\bigskip
\bigskip
\bigskip

% added detail - Renaissance
\dots to this project so I can continue to work on the book and other articles: \\
\url{https://yurichev.com/donate.html}.

%% Suggested:
% It's with help of readers like you that I created this work (and still am creating
% it). Anyone who donates will get added to the list in the Acknowledgments (if they
% like).
\bigskip
\bigskip
\bigskip

% I'd make this into it's own thing, and put it before the Appendix
\huge Attention: Opinion Poll
\normalsize

\bigskip
\bigskip
\bigskip

I have an idea to replace all the OllyDbg examples in the book with examples using some other debugger.
I have nothing against OllyDbg, but it has a GUI and uses small fonts, and the screenshots are somewhat unsuitable for the book.
Maybe I could use GDB, rada.re, WinDbg, or maybe some other console debugger?

What do you think about it?
Should I leave OllyDbg examples, or would GDB examples would be OK?

E-Mail: \GTT{\EMAIL}.

\vspace*{\fill}
\vfill
}
%\DE{\include{1st_page_DE}}\RU{\vspace*{\fill}

\huge Пожалуйста, жертвуйте
\normalsize

\bigskip
\bigskip
\bigskip

\dots чтобы я мог продолжать работать над этой книгой и другими статьями: \\
\url{https://yurichev.com/donate.html}.

\bigskip
\bigskip
\bigskip

\huge Внимание: соц.опрос
\normalsize

\bigskip
\bigskip
\bigskip

У меня есть идея заменить все примеры на OllyDbg в этой книге на примеры с использованием другого отладчика.
Я не имею ничего против OllyDbg, но он GUI-шный и использует маленькие шрифты, и скриншоты не очень подходят для книги.
Может я бы использовал GDB, radare или WinDbg.
Или может быть, какой-нибудь другой консольный отладчик?

Что вы об этом думаете?
Должен ли я оставить примеры с OllyDbg, или примеры с radare будут ОК?

E-Mail: \GTT{\EMAIL}.

\vspace*{\fill}
\vfill
}\CN{\include{1st_page_CN}}


\include{page_after_cover}
\include{call_for_translators}

\shorttoc{%
    \RU{Краткое оглавление}%
    \EN{Abridged contents}%
    \ES{Contenidos abreviados}%
    \PTBRph{}%
    \DE{Inhaltsverzeichnis (gekürzt)}%
    \PLph{}%
    \ITAph{}%
    \THAph{}\NLph{}%
    \FR{Contenus abrégés}
}{0}

\tableofcontents
\cleardoublepage

% \include{page_after_cover} - as per suggestion above - Ren.

\cleardoublepage
\include{preface}

\mainmatter

\include{parts}
% I'm not sure how to exclude reverse_engineering_services, but I think it's \exclude{reverse_engineering_services} - Ren

%Should there be a \backmatter here?
% If the opinion poll is its own file (see in preface_EN):
% \huge Attention: Opinion Poll
\normalsize

\bigskip
\bigskip
\bigskip

I have an idea to replace all the OllyDbg examples in the book with examples using some other debugger.
I have nothing against OllyDbg, but it has a GUI and uses small fonts, and the screenshots are somewhat unsuitable for the book.
Maybe I could use GDB, rada.re, WinDbg, or maybe some other console debugger?

What do you think about it?
Should I leave OllyDbg examples, or would GDB examples would be OK?

E-Mail: \GTT{\EMAIL}.

\vspace*{\fill}
\vfill

\EN{\include{appendix/appendix}}\RU{\include{appendix/appendix}}
% TODO split
\part*{%
	\RU{Список принятых сокращений}%
	\EN{Acronyms used}%
	\NL{Gebruikte afkortingen}%
	\ES{Acr\'onimos utilizados}%
	\PTBRph{}%
	\DE{Verwendete Abkürzungen}%
	\PLph{}%
	\ITAph{}%
	\FR{Acronymes utilisés}
}
\addcontentsline{toc}{part}{%
	\RU{Список принятых сокращений}%
	\EN{Acronyms used}%
	\ES{Acr\'onimos utilizados}%
	\NL{Gebruikte afkortingen}%
	\PTBRph{}%
	\DE{Verwendete Abkürzungen}%
	\PLph{}%
	\ITAph{}%
	\FR{Acronymes utilisés}
}
\begin{acronym}
\RU{
	\acro{OS}[ОС]{Операционная Система}
	\acro{FAQ}[ЧаВО]{Часто задаваемые вопросы}
	\acro{OOP}[ООП]{Объектно-Ориентированное Программирование}
	\acro{PL}[ЯП]{Язык Программирования}
	\acro{PRNG}[ГПСЧ]{Генератор псевдослучайных чисел}
	\acro{ROM}[ПЗУ]{Постоянное запоминающее устройство}
	\acro{ALU}[АЛУ]{Арифметико-логическое устройство}
	\acro{PID}{ID программы/процесса}
	\acro{LF}{Line feed (подача строки) (10 или '\textbackslash{}n' в \CCpp)}
	\acro{CR}{Carriage return (возврат каретки) (13 или '\textbackslash{}r' в \CCpp)}
	\acro{LIFO}{Last In First Out (последним вошел, первым вышел)}
	\acro{MSB}{Most significant bit (самый старший бит)} % NOT BYTE!
	\acro{LSB}{Least significant bit (самый младший бит)} % NOT BYTE!
	\acro{NSA}[АНБ]{Агентство национальной безопасности}
	\acro{CFB}{Режим обратной связи по шифротексту (Cipher Feedback)}
	\acro{CSPRNG}{Криптографически стойкий генератор псевдослучайных чисел (cryptographically secure pseudorandom number generator)}
	\acro{SICP}{Структура и интерпретация компьютерных программ (Structure and Interpretation of Computer Programs)}
}%
\EN{
	\acro{OS}{Operating System}
	\acro{FAQ}{Frequently Asked Questions}
	\acro{OOP}{Object-Oriented Programming}
	\acro{PL}{Programming Language}
	\acro{PRNG}{Pseudorandom Number Generator}
	\acro{ROM}{Read-Only Memory}
	\acro{ALU}{Arithmetic Logic Unit}
	\acro{PID}{Program/process ID}
	\acro{LF}{Line Feed (10 or '\textbackslash{}n' in \CCpp)}
	\acro{CR}{Carriage Return (13 or '\textbackslash{}r' in \CCpp)}
	\acro{LIFO}{Last In First Out}
	\acro{MSB}{Most Significant Bit} % NOT BYTE!
	\acro{LSB}{Least Significant Bit} % NOT BYTE!
	\acro{NSA}{National Security Agency}
	\acro{CFB}{Cipher Feedback}
	\acro{CSPRNG}{Cryptographically Secure Pseudorandom Number Generator}
	\acro{SICP}{Structure and Interpretation of Computer Programs}
	%\acro{ABI}{Application Binary Interface}
}%
\ES{
	\acro{OS}[SO]{Sistema Operativo}
	\acro{FAQ}{Preguntas Frecuentes}
	\acro{OOP}[POO]{Programaci\'on Orientada a Objetos}
	\acro{PL}[LP]{Lenguaje de Programaci\'on}
	\acro{PRNG}[GPAN]{Generador Pseudo-Aleatorio de N\'umeros}
	\acro{ROM}{Memoria de Solo Lectura}
	\acro{ALU}{Unidad Aritm\'etica L\'ogica}
	\acro{NSA}{\ESph{}}
}%
\PTBR{
	\acro{OS}{\PTBRph{}}
	\acro{FAQ}{\PTBRph{}}
	\acro{OOP}{\PTBRph{}}
	\acro{PL}{\PTBRph{}}
	\acro{PRNG}{\PTBRph{}}
	\acro{ROM}{\PTBRph{}}
	\acro{ALU}{\PTBRph{}}
	\acro{NSA}{\PTBRph{}}
}%
\PL{
	\acro{OS}{\PLph{}}
	\acro{FAQ}{\PLph{}}
	\acro{OOP}{\PLph{}}
	\acro{PL}{\PLph{}}
	\acro{PRNG}{\PLph{}}
	\acro{ROM}{\PLph{}}
	\acro{ALU}{\PLph{}}
}%
\DE{
	\acro{OS}[BS]{Betriebssystem}
	\acro{FAQ}{Häufig gestellte Fragen}
	\acro{OOP}{Objektorientierte Programmierung}
	\acro{PL}[PS]{Programmiersprache}
	\acro{PRNG}{Pseudozufallszahlen-Generator}
	\acro{ROM}{\DEph{}}
	\acro{ALU}[ALE]{Arithmetisch-logische Einheit}
	\acro{NSA}{\DEph{}}
}%
\ITA{
	\acro{OS}{\ITAph{}}
	\acro{FAQ}{\ITAph{}}
	\acro{OOP}{\ITAph{}}
	\acro{PL}{\ITAph{}}
	\acro{PRNG}{\ITAph{}}
	\acro{ROM}{\ITAph{}}
	\acro{ALU}{\ITAph{}}
}%
\THA{
	\acro{OS}{\THAph{}}
	\acro{FAQ}{\THAph{}}
	\acro{OOP}{\THAph{}}
	\acro{PL}{\THAph{}}
	\acro{PRNG}{\THAph{}}
	\acro{ROM}{\THAph{}}
	\acro{ALU}{\THAph{}}
}%
\NL{
	\acro{OS}{\NL{Operating System}}
	\acro{FAQ}{\NL{Veelvoorkomende vragen}}
	\acro{OOP}{\NL{Object-Oriented Programmeren}}
	\acro{PL}[PT]{\NL{Programmeertaal}}
	\acro{PRNG}{\NL{Pseudorandom number generator}}
	\acro{ROM}{\NL{Read-only memory}}
	\acro{ALU}{\NL{Arithmetic logic unit}}
}%
\FR{
	\acro{OS}[OS]{Système d'exploitation (Operating System)}
	\acro{FAQ}{Foire Aux Questions}
	\acro{OOP}[POO]{Programmation orientée objet}
	\acro{PL}[LP]{Language de programmation}
	\acro{PRNG}{Nombre généré pseudo-aléatoirement}
	\acro{ROM}{Mémoire morte}
	\acro{ALU}[UAL]{Unité arithmétique et logique}
	\acro{PID}{ID d'un programme/processus}
	\acro{LF}{Line feed (10 ou '\textbackslash{}n' en \CCpp)}
	\acro{CR}{Carriage return (13 ou '\textbackslash{}r' en \CCpp)}
	\acro{LIFO}{Dernier entré, premier sorti}
	\acro{MSB}{Bit le plus significatif} % NOT BYTE!
	\acro{LSB}{Bit le moins significatif} % NOT BYTE!
	\acro{NSA}{\FRph{}}
}%
\acro{RA}{\ReturnAddress}
\acro{PE}{Portable Executable}
\acro{SP}{\gls{stack pointer}. SP/ESP/RSP \InENRU x86/x64. SP \InENRU ARM.}
\acro{DLL}{Dynamic-Link Library}
\acro{PC}{Program Counter. IP/EIP/RIP \InENRU x86/64. PC \InENRU ARM.}
\acro{LR}{Link Register}
\acro{IDA}{
	\RU{Интерактивный дизассемблер и отладчик, разработан \href{https://hex-rays.com/}{Hex-Rays}}%
	\EN{Interactive Disassembler and Debugger developed by \href{https://hex-rays.com/}{Hex-Rays}}%
	\ES{Desensamblador Interactivo y depurador desarrollado por \href{https://hex-rays.com/}{Hex-Rays}}%
	\NL{Interactive Disassembler en debugger ontwikkeld door \href{https://hex-rays.com}{Hex-Rays}}
	\PTBRph{}%
	\PLph{}%
	\DE{Interaktiver Disassembler und Debugger entwickelt von \href{https://hex-rays.com/}{Hex-Rays}}%
	\ITAph{}%
	\THAph{}%
	\FR{Désassembleur interactif et débuggueur développé par \href{https://hex-rays.com/}{Hex-Rays}}%
}
\acro{IAT}{Import Address Table}
\acro{INT}{Import Name Table}
\acro{RVA}{Relative Virtual Address}
\acro{VA}{Virtual Address}
\acro{OEP}{Original Entry Point}
\acro{MSVC}{Microsoft Visual C++}
\acro{MSVS}{Microsoft Visual Studio}
\acro{ASLR}{Address Space Layout Randomization}
\acro{MFC}{Microsoft Foundation Classes}
\acro{TLS}{Thread Local Storage}
\acro{AKA}{
        \EN{Also Known As}%
	\FR{Aussi connu sous le nom de}%
	\RU{ - (Также известный как)}%
	\ES{ - (Tambi\'en Conocido Como)}%
	\NL{ - (Ook gekend als)}%
	\PTBRph{}%
	\PLph{}%
	\DEph{}%
	\ITAph{}%
	\THAph{}%
}
\acro{CRT}{C Runtime library}
\acro{CPU}{Central Processing Unit}
\acro{GPU}{Graphics Processing Unit}
\acro{FPU}{Floating-Point Unit}
\acro{CISC}{Complex Instruction Set Computing}
\acro{RISC}{Reduced Instruction Set Computing}
\acro{GUI}{Graphical User Interface}
\acro{RTTI}{Run-Time Type Information}
\acro{BSS}{Block Started by Symbol}
\acro{SIMD}{Single Instruction, Multiple Data}
\acro{BSOD}{Blue Screen of Death}
\acro{DBMS}{Database Management Systems}
\acro{ISA}{Instruction Set Architecture\RU{ (Архитектура набора команд)}}
\acro{CGI}{Common Gateway Interface}
\acro{HPC}{High-Performance Computing}
\acro{SOC}{System on Chip}
\acro{SEH}{Structured Exception Handling}
\acro{ELF}{\RU{Формат исполняемых файлов, использующийся в Linux и некоторых других *NIX}
\EN{Executable File format widely used in *NIX systems including Linux}\ESph{}\PTBRph{}\PLph{}\ITAph{}\DEph{}\NLph{}
\FR{Format de fichier exécutable couramment utilisé sur les systèmes *NIX, Linux inclus}}
\acro{TIB}{Thread Information Block}
\acro{TEA}{Tiny Encryption Algorithm}
\acro{PIC}{Position Independent Code}
\acro{NAN}{Not a Number}
\acro{NOP}{No Operation}
\acro{BEQ}{(PowerPC, ARM) Branch if Equal}
\acro{BNE}{(PowerPC, ARM) Branch if Not Equal}
\acro{BLR}{(PowerPC) Branch to Link Register}
\acro{XOR}{eXclusive OR\RU{ (исключающее \q{ИЛИ})}\FR{ (OU exclusif)}}
\acro{MCU}{Microcontroller Unit}
\acro{RAM}{Random-Access Memory}
\acro{GCC}{GNU Compiler Collection}
\acro{EGA}{Enhanced Graphics Adapter}
\acro{VGA}{Video Graphics Array}
\acro{API}{Application Programming Interface}
\acro{ASCII}{American Standard Code for Information Interchange}
\acro{ASCIIZ}{ASCII Zero (\RU{ASCII-строка заканчивающаяся нулем}\EN{null-terminated ASCII string}
\FR{chaîne ASCII terminée par un octet nul (à zéro)}\ESph{}\PTBRph{}\PLph{}\ITAph{}\DEph{}\NLph{})}
\acro{IA64}{Intel Architecture 64 (Itanium): \myref{itanium}}
\acro{EPIC}{Explicitly Parallel Instruction Computing}
\acro{OOE}{Out-of-Order Execution}
\acro{MSDN}{Microsoft Developer Network}
\acro{STL}{(\Cpp) Standard Template Library: \myref{sec:STL}}
\acro{PODT}{(\Cpp) Plain Old Data Type}
\acro{HDD}{Hard Disk Drive}
\acro{VM}{Virtual Memory\RU{ (виртуальная память)}\FR{ (mémoire virtuelle)}}
\acro{WRK}{Windows Research Kernel}
\acro{GPR}{General Purpose Registers\RU{ (регистры общего пользования)}}
\acro{SSDT}{System Service Dispatch Table}
\acro{RE}{Reverse Engineering}
\acro{RAID}{Redundant Array of Independent Disks}
\acro{SSE}{Streaming SIMD Extensions}
\acro{BCD}{Binary-Coded Decimal}
\acro{BOM}{Byte Order Mark}
\acro{GDB}{GNU Debugger}
\acro{FP}{Frame Pointer}
\acro{MBR}{Master Boot Record}
\acro{JPE}{Jump Parity Even (\RU{инструкция x86}\EN{x86 instruction}\FR{instruction x86})}
\acro{CIDR}{Classless Inter-Domain Routing}
\acro{STMFD}{Store Multiple Full Descending (\RU{инструкция ARM}\EN{ARM instruction}\FR{instruction ARM})}
\acro{LDMFD}{Load Multiple Full Descending (\RU{инструкция ARM}\EN{ARM instruction}\FR{instruction ARM})}
\acro{STMED}{Store Multiple Empty Descending (\RU{инструкция ARM}\EN{ARM instruction}\FR{instruction ARM})}
\acro{LDMED}{Load Multiple Empty Descending (\RU{инструкция ARM}\EN{ARM instruction}\FR{instruction ARM})}
\acro{STMFA}{Store Multiple Full Ascending (\RU{инструкция ARM}\EN{ARM instruction}\FR{instruction ARM})}
\acro{LDMFA}{Load Multiple Full Ascending (\RU{инструкция ARM}\EN{ARM instruction}\FR{instruction ARM})}
\acro{STMEA}{Store Multiple Empty Ascending (\RU{инструкция ARM}\EN{ARM instruction}\FR{instruction ARM})}
\acro{LDMEA}{Load Multiple Empty Ascending (\RU{инструкция ARM}\EN{ARM instruction}\FR{instruction ARM})}
\acro{APSR}{(ARM) Application Program Status Register}
\acro{FPSCR}{(ARM) Floating-Point Status and Control Register}
\acro{RFC}{Request for Comments}
\acro{TOS}{Top of Stack\RU{ (вершина стека)}}
\acro{LVA}{(Java) Local Variable Array\RU{ (массив локальных переменных)}}
\acro{JVM}{Java Virtual Machine}
\acro{JIT}{Just-In-Time compilation}
\acro{CDFS}{Compact Disc File System}
\acro{CD}{Compact Disc}
\acro{ADC}{Analog-to-Digital Converter}
\acro{EOF}{End of File\RU{ (конец файла)}\FR{ (fin de fichier)}}
\acro{TBT}{To be Translated. The presence of this acronym in this place means that the English version has some new/modified content which is to be translated and placed right here.}
\acro{DIY}{Do It Yourself}
\acro{MMU}{Memory Management Unit}
\acro{DES}{Data Encryption Standard}
\acro{MIME}{Multipurpose Internet Mail Extensions}
\acro{DBI}{Dynamic Binary Instrumentation}
\acro{XML}{Extensible Markup Language}
\acro{JSON}{JavaScript Object Notation}
\acro{URL}{Uniform Resource Locator}
\acro{ISP}{Internet Service Provider}
\acro{IV}{Initialization Vector}
\end{acronym}


\bookmarksetup{startatroot}

\clearpage
\phantomsection
\addcontentsline{toc}{chapter}{%
    \RU{Глоссарий}%
    \EN{Glossary}%
    \ES{Glosario}%
    \PTBRph{}%
    \DE{Glossar}%
    \PLph{}%
    \ITAph{}%
    \THAph{}\NLph{}%
    \FR{Glossaire}
}
\printglossaries

\clearpage
\phantomsection
\printindex

%I think % Suggested taking from 1st_page and putting after index - Renaissance

%\\part\huge{Reverse Engineering Services}
%\addcontentsline{toc}{part}{Reverse Engineering Services}
\normalsize

\bigskip
\bigskip
\bigskip


I have tried many jobs in my life, but, surprisingly (even to myself),
the job I'm the most proud of is rewriting large piece(s) of compiled code back to C/C++.
This is an extremely boring and slow process. I once spent more than a year on rewriting 100KB DLL to pure C,
and it was like a full-time job.

And expensive.

% I think I got the intent right (original: By the way, a lot of tricks has been added to this book during this work.)
A lot of tricks have been added to this book as a result of this work.

I assume such a service could be interesting to those who inherited some compiled code with no source code.
You are welcome to contact me at:

E-Mail: \GTT{\EMAIL}.
 would put in the file here - Ren

\end{document}
