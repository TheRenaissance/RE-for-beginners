% TODO png blur? too wide listings
\section{Information entropy}
\label{entropy}
\myindex{Entropy}

For the sake of simplification, I would say, information entropy is a measure,
how tightly some piece of data can be compressed.
For example, it is usually not possible to compress already compressed archive file, so it has high entropy.
On the other hand, 1MiB of zero bytes can be compressed to a tiny output file.
Indeed, in plain English language, one million of zeros can be described just as
``resulting file is one million zero bytes''.
Compressed files are usually a list of instructions to decompressor, like this:
``put 1000 zeros, then 0x23 byte, then 0x45 byte, then put a block of size 10 bytes which we've seen 500 bytes back, etc.''

Texts written in natural languages are also can be compressed tightly, 
because natural languages has a lot of redundancy
(otherwise, a tiny typo will always lead to misunderstanding, 
like any toggled bit in compressed archive make decompression nearly impossible), 
some words are used very often, etc.
In everyday speech, it's possible to drop up to half of words and it still be recognizable.

Code for CPUs is also can be compressed, because some \ac{ISA} instructions are used much more often than others.
\myindex{x86!\Instructions!MOV}
\myindex{x86!\Instructions!PUSH}
\myindex{x86!\Instructions!CALL}
In x86, most used instructions are \INS{MOV}/\INS{PUSH}/\INS{CALL} (\myref{correctly_disasmed_code}).

Data compressors and ciphers tend to produce very high entropy results.
Good \ac{PRNG} also produce data which cannot be compressed 
(it is possible to measure their quality by this sign).

So, in other words, entropy is a measure which can help to probe contents of unknown data block.

\input{ff/entropy/math_EN}

\subsection{Conclusion}

Information entropy can be used as a quick-n-dirty method for inspecting unknown binary files.
In particular, it is a very quick way to find compressed/encrypted pieces of data.
Someone say it's possible to find \ac{RSA} (and other asymmetric cryptographic algorithms) public/private keys 
in executable code (keys has high entropy as well), but I didn't try this myself.

\subsection{Tools}

Handy Linux \IT{ent} utility to measure entropy of a file\footnote{\url{http://www.fourmilab.ch/random/}}.

There is a great online entropy visualizer made by Aldo Cortesi, 
which I tried to mimic using Mathematica: \url{http://binvis.io}.
His articles about entropy visualization are worth reading:
\url{http://corte.si/posts/visualisation/entropy/index.html},
\url{http://corte.si/posts/visualisation/malware/index.html},
\url{http://corte.si/posts/visualisation/binvis/index.html}.

\myindex{radare2}
radare2 framework has \IT{\#entropy} command for this.

A tool for IDA: IDAtropy\footnote{\url{https://github.com/danigargu/IDAtropy}}.

\subsection{A word about primitive encryption like XORing}

It's interesting that simple XOR encryption doesn't affect entropy of data.
I've shown this in \IT{Norton Guide} example in the book (\myref{norton_guide}).

Generalizing: encryption by substitution cipher also doesn't affect entropy of data
(and XOR can be viewed as substitution cipher).
The reason of that is because entropy calculation algorithm view data on byte-level.
On the other hand, the data encrypted by 2 or 4-byte XOR pattern will result in another level of entropy.

Nevertheless, low entropy is usually a good sign of weak amateur cryptography
(which is also used in license keys/files, etc.).

\subsection{More about entropy of executable code}

It is quickly noticeable that probably a biggest source of high-entropy in executable code
are relative offsets encoded in opcodes.
For example, these two consequent instructions will have different relative offsets in their opcodes, 
while they are in fact pointing to the same function:

\begin{lstlisting}[style=customasmx86]
function proc
...
function endp

...

CALL function
...
CALL function
\end{lstlisting}

Ideal executable code compressor would encode information like this:
\IT{there is a CALL to a ``function'' at address X and the same CALL at address Y} without necessity to encode
address of the \IT{function} twice.

\myindex{UPX}
To deal with this, executable compressors are sometimes able to reduce entropy here.
One example is UPX: \url{http://sourceforge.net/p/upx/code/ci/default/tree/doc/filter.txt}.

\subsection{\ac{PRNG}}

\myindex{GnuPG}
When I run GnuPG to generate new private (secret) key, it asking for some entropy \dots

\begin{lstlisting}
We need to generate a lot of random bytes. It is a good idea to perform
some other action (type on the keyboard, move the mouse, utilize the
disks) during the prime generation; this gives the random number
generator a better chance to gain enough entropy.

Not enough random bytes available.  Please do some other work to give
the OS a chance to collect more entropy! (Need 169 more bytes)
\end{lstlisting}

This means that good a \ac{PRNG} produces long high-entropy results,
and this is what the secret asymmetrical cryptographical key needs.
But \ac{CPRNG} is tricky (because computer is highly deterministic device itself),
so the GnuPG asking for some additional randomness from the user.

\subsection{More examples}

Here is a case where I try to calculate entropy of some blocks with unknown contents: \myref{encrypted_DB1}.

\input{ff/entropy/files_EN}

\subsection{Making lower level of entropy}

The author of these lines once saw a software which stored each byte of encrypted data in 3 bytes:
each has {\Large $\approx \frac{byte}{3}$} value, so reconstructing encrypted byte back involving summing up 3 consecutive bytes.
Looks absurdly.

But some people say this was done in order to conceal the very fact
the data has something encrypted inside: measuring entropy of such block will show much lower level of it.

