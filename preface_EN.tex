% Okay, I'd like to suggest completely restructuring the beginning:
% Most books I see put the praise right after the cover, even before the info
% you have in the page_after_cover.tex, which should be right before the
% table of contents.

% I'm debating whether the call_for_translators should be before or after the
% table of contents, but I think it should be pretty visible and at the beginning.
% That way, potential contributors already have the idea in their head when they start.

% I would recommend taking the details about yourself from this files, adding some
% more details about yourself, and making an 'About the Author' (if you want it to be more formal)
% or an 'About Myself' (more informal). (Many place the 'About the Author'
% after the whole book, but especially on technical works, many put them before.
% Wiley (who did a bunch of Schneier's books) puts them at the beginning, for instance. Especially in pdf
% format, I would recommmend the beginning.)

% Continuing like this, I would put the
% thanks.tex right after that (consider calling it Acknowledgments, by the way),
% but I've also got some suggestions for that. I would make a subsection there
% called 'About the Editions of Reverse Engineering for Beginners' which would
% bring in the subsections from this file there about the Chinese, Korean and
% Farsi.(I'll also point out that it is acceptable and common
% to use first person ('I', 'me') in acknoledgments, which gives the thanks a
% more personal feel.)

% I would then put a page break, and put the FAQ.tex with some additions I'll
% suggest there. This will more or less take out the need for any of the text in
% this file, though you might want to still use it as the backbone of the intro

% I would also recommend putting the tools and communities in the beginning,
% not the end. Both are already useful at the beginning of the reader's journey.
% Truth is, the afterword should probably be rolled into the call for translators
% or made more unique, too.

% 1st_page.tex has three parts: one where you offer your services for interperting
% source code, a call for donations, and the opinion poll about the debuggers.
% I think that it should probably be split into three parts.
% The Reverse Engineering Services should be moved to the end, as sort of a last page,
% probably even after the index, though make sure it's in the Table of Contents then,
% and with a more identifiable name, like Reverse Engineering Services.
% I think the opinion poll should also be in the end, though maybe before the
% appendix. Beginners are unlikely to have an opinion, I think, while
% those who have finished are more than entitled to an opinion.
% The donations, on the other hand, should stay towards the beginning.

% For this file, the above would mean cutting out everything unbtil where the
% thanks and FAQ sections are put together - everything else will be covered
% somewhere else (I've put it in comments in each file already)
\section*{Preface}

There are several popular meanings of the term \q{\gls{reverse engineering}}:

1) The reverse engineering of software; researching compiled programs

2) The scanning of 3D structures and the subsequent digital manipulation required in order to duplicate them

3) Recreating \ac{DBMS} structure

This book is about the first meaning.

\subsection*{Prerequisites}

Basic knowledge of the C \ac{PL}.
Recommended reading: \myref{CCppBooks}.

\subsection*{Exercises and tasks}

\dots
can be found at: \url{http://challenges.re}.

\subsection*{About the author}
\begin{tabularx}{\textwidth}{ l X }

\raisebox{-\totalheight}{
\includegraphics[scale=0.60]{Dennis_Yurichev.jpg}
}

&
Dennis Yurichev is an experienced reverse engineer and programmer.
He can be contacted by email: \textbf{\EMAIL{}} or Skype: \textbf{dennis.yurichev}.

% FIXME: no link. \tablefootnote doesn't work
\end{tabularx}

% subsections:
\input{praise}
\ifdefined\RUSSIAN
\newcommand{\PeopleMistakesInaccuraciesRusEng}{Станислав \q{Beaver} Бобрицкий, Александр Лысенко, Александр \q{Solar Designer} Песляк, Федерико Рамондино, Марк Уилсон, Ксения Галинская, Разихова Мейрамгуль Кайратовна, Анатолий Прокофьев}
\else
\newcommand{\PeopleMistakesInaccuraciesRusEng}{Stanislav \q{Beaver} Bobrytskyy, Alexander Lysenko, Alexander \q{Solar Designer} Peslyak, Federico Ramondino, Mark Wilson, Xenia Galinskaya, Razikhova Meiramgul Kayratovna, Anatoly Prokofiev}
\fi

\newcommand{\PeopleMistakesInaccuracies}{\PeopleMistakesInaccuraciesRusEng{}, Shell Rocket, Zhu Ruijin, Changmin Heo, Vitor Vidal, Stijn Crevits, Jean-Gregoire Foulon\footnote{\url{https://github.com/pixjuan}}, Ben L., Etienne Khan, Norbert Szetei\footnote{\url{https://github.com/73696e65}}, Marc Remy, Michael Hansen, Derk Barten, The Renaissance\footnote{\url{https://github.com/TheRenaissance}}.}

\newcommand{\PeopleItalianTranslators}{Federico Ramondino\footnote{\url{https://github.com/pinkrab}},
Paolo Stivanin\footnote{\url{https://github.com/paolostivanin}}, twyK}

\newcommand{\PeopleFrenchTranslators}{Florent Besnard\footnote{\url{https://github.com/besnardf}}, Marc Remy\footnote{\url{https://github.com/mremy}}, Baudouin Landais, Téo Dacquet\footnote{\url{https://github.com/T30rix}}}

\newcommand{\PeopleGermanTranslators}{Dennis Siekmeier\footnote{\url{https://github.com/DSiekmeier}},
Julius Angres\footnote{\url{https://github.com/JAngres}}, Dirk Loser\footnote{\url{https://github.com/PolymathMonkey}}, Clemens Tamme}

\newcommand{\PeopleSpanishTranslators}{Diego Boy, Luis Alberto Espinosa Calvo, Fernando Guida, Diogo Mussi}

\newcommand{\PeoplePTBRTranslators}{Thales Stevan de A. Gois, Diogo Mussi}

\newcommand{\PeoplePolishTranslators}{Kateryna Rozanova, Aleksander Mistewicz}

\newcommand{\FNGithubContributors}{\footnote{\url{https://github.com/dennis714/RE-for-beginners/graphs/contributors}}}

\EN{% Suggested name change: Acknowledgements:
%\subsection*{Acknowledgements}
\subsection*{Thanks}

For patiently answering all my questions: \HERMIT, Slava \q{Avid} Kazakov, SkullC0DEr.

For sending me notes about mistakes and inaccuracies: \PeopleMistakesInaccuracies{}

For helping me in other ways:
Andrew Zubinski,
Arnaud Patard (rtp on \#debian-arm IRC),
noshadow on \#gcc IRC,
Aliaksandr Autayeu,
Mohsen Mostafa Jokar.

For translating the book into Simplified Chinese:
Antiy Labs (\href{http://antiy.cn}{antiy.cn}), Archer.

For translating the book into Korean: Byungho Min.

For translating the book into Dutch: Cedric Sambre (AKA Midas).

For translating the book into Spanish: \PeopleSpanishTranslators{}.

For translating the book into Portuguese: \PeoplePTBRTranslators{}.

For translating the book into Italian: \PeopleItalianTranslators{}.

For translating the book into French: \PeopleFrenchTranslators{}.

For translating the book into German: \PeopleGermanTranslators{}.

For translating the book into Polish: \PeoplePolishTranslators{}.

For proofreading:
Alexander \q{Lstar} Chernenkiy,
Vladimir Botov,
Andrei Brazhuk,
Mark ``Logxen'' Cooper, Yuan Jochen Kang, Mal Malakov, Lewis Porter, Jarle Thorsen, Hong Xie.

Vasil Kolev\footnote{\url{https://vasil.ludost.net/}} did a great amount of work in proofreading and correcting many mistakes.

For illustrations and cover art: Andy Nechaevsky.

Thanks also to all the folks on github.com who have contributed notes and corrections\FNGithubContributors{}.

Many \LaTeX\ packages were used: I would like to thank the authors as well.

%This is where I think the subsections about the Korean, etc. versions should go - Renaissance:
%\subsection*{About the Korean translation}

%In January 2015, the Acorn publishing company (\href{http://www.acornpub.co.kr}{www.acornpub.co.kr}) in South Korea did a huge amount of work in translating and publishing
%this book (as it was in August 2014) into Korean.

%It's available now at \href{http://go.yurichev.com/17343}{their website}.

%\iffalse
%\begin{figure}[H]
%\centering
%\includegraphics[scale=0.3]{acorn_cover.jpg}
%\end{figure}
%\fi

%The translator is Byungho Min (\href{http://go.yurichev.com/17344}{twitter/tais9}).
%The cover art was done by the artistic Andy Nechaevsky, a friend of the author:
%\href{http://go.yurichev.com/17023}{facebook/andydinka}.
%Acorn also holds the copyright to the Korean translation.

%So, if you want to have a \IT{real} book on your shelf in Korean and
%want to support this work, it is now available for purchase.

%\subsection*{About the Persian/Farsi translation}

%In 2016 the book was translated by Mohsen Mostafa Jokar (who is also known to Iranian community for his translation of Radare manual\footnote{\url{http://rada.re/get/radare2book-persian.pdf}}).
%It is available on the publisher’s website\footnote{\url{http://goo.gl/2Tzx0H}} (Pendare Pars).

%Here is a link to a 40-page excerpt: \url{https://beginners.re/farsi.pdf}.

%National Library of Iran registration information: \url{http://opac.nlai.ir/opac-prod/bibliographic/4473995}.

%\subsection*{About the Chinese translation}

%In April 2017, translation to Chinese was completed by Chinese PTPress. They are also the Chinese translation copyright holders.

% The Chinese version is available for order here: \url{http://www.epubit.com.cn/book/details/4174}. A partial review and history behind the translation can be found here: \url{http://www.cptoday.cn/news/detail/3155}.

%% If you're using first-person (I, me) in the acknowledgments, you should change the following back to first-person, too - Ren.
%The principal translator is Archer, to whom the author owes very much. He was extremely meticulous (in a good sense) and reported most of the known mistakes and bugs, which is very important in literature such as this book.
%The author would recommend his services to any other author!

%The guys from \href{http://www.antiy.net/}{Antiy Labs} has also helped with translation. \href{http://www.epubit.com.cn/book/onlinechapter/51413}{Here is preface} written by them.


\subsubsection*{Donors}

Those who supported me during the time when I wrote significant part of the book:

2 * Oleg Vygovsky (50+100 UAH), 
Daniel Bilar ($\$$50), 
James Truscott ($\$$4.5),
Luis Rocha ($\$$63), 
Joris van de Vis ($\$$127), 
Richard S Shultz ($\$$20), 
Jang Minchang ($\$$20), 
Shade Atlas (5 AUD), 
Yao Xiao ($\$$10),
Pawel Szczur (40 CHF), 
Justin Simms ($\$$20), 
Shawn the R0ck ($\$$27), 
Ki Chan Ahn ($\$$50), 
Triop AB (100 SEK), 
Ange Albertini (\euro{}10+50),
Sergey Lukianov (300 RUR), 
Ludvig Gislason (200 SEK), 
Gérard Labadie (\euro{}40), 
Sergey Volchkov (10 AUD),
Vankayala Vigneswararao ($\$$50),
Philippe Teuwen ($\$$4),
Martin Haeberli ($\$$10),
Victor Cazacov (\euro{}5),
Tobias Sturzenegger (10 CHF),
Sonny Thai ($\$$15),
Bayna AlZaabi ($\$$75),
Redfive B.V. (\euro{}25),
Joona Oskari Heikkilä (\euro{}5),
Marshall Bishop ($\$$50),
Nicolas Werner (\euro{}12),
Jeremy Brown ($\$$100),
Alexandre Borges ($\$$25),
Vladimir Dikovski (\euro{}50),
Jiarui Hong (100.00 SEK),
Jim Di (500 RUR),
Tan Vincent ($\$$30),
Sri Harsha Kandrakota (10 AUD),
Pillay Harish (10 SGD),
Timur Valiev (230 RUR),
Carlos Garcia Prado (\euro{}10),
Salikov Alexander (500 RUR),
Oliver Whitehouse (30 GBP),
Katy Moe ($\$$14),
Maxim Dyakonov ($\$$3),
Sebastian Aguilera (\euro{}20),
Hans-Martin Münch (\euro{}15),
Jarle Thorsen (100 NOK),
Vitaly Osipov ($\$$100),
Yuri Romanov (1000 RUR),
Aliaksandr Autayeu (\euro{}10),
Tudor Azoitei ($\$$40),
Z0vsky (\euro{}10),
Yu Dai ($\$$10),
Anonymous ($\$$15),
Vladislav Chelnokov (($\$$25),
Nenad Noveljic (($\$$50).



Thanks a lot to every donor!
}
\ES{\input{thanks_ES}}
\NL{\input{thanks_NL}}
\RU{\input{thanks_RU}}
\ITA{\input{thanks_ITA}}
\FR{\input{thanks_FR}}
\DE{\input{thanks_DE}}
%\CN{\input{thanks_CN}}

\subsection*{mini-FAQ}
% I'm putting in commented-out additions as per what I wrote in preface.tex - Renaissance

%\par Q: What is \q{\gls{reverse engineering}?
%\par A:There are several popular meanings of the term \q{\gls{reverse engineering}}:

%1) The reverse engineering of software; researching compiled programs

%2) The scanning of 3D structures and the subsequent digital manipulation required in order to duplicate them

%3) Recreating \ac{DBMS} structure

%This book is about the first meaning.

\par Q: What are the prerequisites for reading this book?
\par A: A basic understanding of C/C++ is desirable.

%\par Q: Which topics will you be covering?
%\par A: x86/x64, ARM/ARM64, MIPS and Java/JVM are all covered in depth.

\par Q: Should I really learn x86/x64/ARM and MIPS at once? Isn't it too much?
\par A: Starters can read about just x86/x64, while skipping or skimming the ARM and MIPS parts.

\par Q: Can I buy a Russian or English hard copy/paper book?
\par A: Unfortunately, no. No publisher got interested in publishing a Russian or English version so far.
Meanwhile, you can ask your favorite copy shop to print and bind it.

\par Q: Is there an epub or mobi version?
\par A: No. The book is highly dependent on TeX/LaTeX-specific hacks, so converting to HTML (epub/mobi are a set of HTMLs)
would not be easy.

\par Q: Why should one learn assembly language these days?
\par A: Unless you are an \ac{OS} developer, you probably don't need to code in assembly\textemdash{}the latest compilers (2010s) are much better at performing optimizations than humans \footnote{A very good text on this topic: \InSqBrackets{\AgnerFog}}.

Also, the latest \ac{CPU}s are very complex devices, and assembly knowledge doesn't really help towards understand their internals.

That being said, there are at least two areas where a good understanding of assembly can be helpful:
First and foremost, for security/malware research. It is also a good way to gain a better understanding of your compiled code while debugging.
This book is therefore intended for those who want to understand assembly language rather
than to code in it, which is why there are many examples of compiler output contained within.

\par Q: I clicked on a hyperlink inside a PDF-document, how do I go back?
\par A: In Adobe Acrobat Reader click Alt+LeftArrow. In Evince click ``<'' button.

\par Q: May I print this book / use it for teaching?
\par A: Of course! That's why the book is licensed under the Creative Commons license (CC BY-SA 4.0).

\par Q: Why is this book free? You've done great job. This is suspicious, as with many other free things.
\par A: In my own experience, authors of technical literature write mostly for self-advertisement purposes. It's not possible to make any decent money from such work.

\par Q: How does one get a job in reverse engineering?
\par A: There are hiring threads that appear from time to time on reddit, devoted to RE\FNURLREDDIT{}
(\RedditHiringThread{}).
Try looking there.

A somewhat related hiring thread can be found in the \q{netsec} subreddit: \NetsecHiringThread{}.

\myindex{LISP}
\par Q: How can I learn programming in general?
\par A: Mastering both C and LISP languages makes programmer's life much, much easier.
I would recommend solving exercises from [\KRBook] and \ac{SICP}.

\par Q: I have a question...
\par A: Send it to me by email (\EMAIL).

%% Based on afterword.tex:
%\par Q: I have a suggestion/correction.
%\par A: Don't hesitate to send me suggestions and corrections, even petty corrections. (Do you see how my English is?)
%Do not worry to bother me while writing me about any petty mistakes you found, even if you are not very confident.
%I'm writing for beginners, after all, so beginners' opinions and comments are crucial for my job.

%One note, though: I'm working on the book a lot, so page and listing numbers change rapidly. Please do not refer to these
%in emails. Take a screenshot of the page, and underline the error in a graphics editor.
%If you're familiar with GitHub and \Latex\, you can fix the error in the source code:

%\href{http://go.yurichev.com/17089}{GitHub}.


% Here's the structure based off of the suggestions above:
%% As per suggestion at the beginning of preface_EN.tex - Renaissance

\subsection*{About the author}
\begin{tabularx}{\textwidth}{ l X }

\raisebox{-\totalheight}{
\includegraphics[scale=0.60]{Dennis_Yurichev.jpg}
}
% In my humble opinion, you should fill this in a bit, but that's big words coming
% from an anonymous newbie contributor
&
Dennis Yurichev is an experienced reverse engineer and programmer.
He can be contacted by email: \textbf{\EMAIL{}} or Skype: \textbf{dennis.yurichev}.

\huge Please Donate
\normalsize

\bigskip
\bigskip
\bigskip

% added detail - Renaissance
\dots to this project so I can continue to work on the book and other articles: \\
\url{https://yurichev.com/donate.html}.

%% Suggested:
% It's with help of readers like you that I created this work (and still am creating
% it). Anyone who donates will get added to the list in the Acknowledgments (if they
% like).

%\input{acknowledgments}
%\subsection*{mini-FAQ}
% I'm putting in commented-out additions as per what I wrote in preface.tex - Renaissance

%\par Q: What is \q{\gls{reverse engineering}?
%\par A:There are several popular meanings of the term \q{\gls{reverse engineering}}:

%1) The reverse engineering of software; researching compiled programs

%2) The scanning of 3D structures and the subsequent digital manipulation required in order to duplicate them

%3) Recreating \ac{DBMS} structure

%This book is about the first meaning.

\par Q: What are the prerequisites for reading this book?
\par A: A basic understanding of C/C++ is desirable.

%\par Q: Which topics will you be covering?
%\par A: x86/x64, ARM/ARM64, MIPS and Java/JVM are all covered in depth.

\par Q: Should I really learn x86/x64/ARM and MIPS at once? Isn't it too much?
\par A: Starters can read about just x86/x64, while skipping or skimming the ARM and MIPS parts.

\par Q: Can I buy a Russian or English hard copy/paper book?
\par A: Unfortunately, no. No publisher got interested in publishing a Russian or English version so far.
Meanwhile, you can ask your favorite copy shop to print and bind it.

\par Q: Is there an epub or mobi version?
\par A: No. The book is highly dependent on TeX/LaTeX-specific hacks, so converting to HTML (epub/mobi are a set of HTMLs)
would not be easy.

\par Q: Why should one learn assembly language these days?
\par A: Unless you are an \ac{OS} developer, you probably don't need to code in assembly\textemdash{}the latest compilers (2010s) are much better at performing optimizations than humans \footnote{A very good text on this topic: \InSqBrackets{\AgnerFog}}.

Also, the latest \ac{CPU}s are very complex devices, and assembly knowledge doesn't really help towards understand their internals.

That being said, there are at least two areas where a good understanding of assembly can be helpful:
First and foremost, for security/malware research. It is also a good way to gain a better understanding of your compiled code while debugging.
This book is therefore intended for those who want to understand assembly language rather
than to code in it, which is why there are many examples of compiler output contained within.

\par Q: I clicked on a hyperlink inside a PDF-document, how do I go back?
\par A: In Adobe Acrobat Reader click Alt+LeftArrow. In Evince click ``<'' button.

\par Q: May I print this book / use it for teaching?
\par A: Of course! That's why the book is licensed under the Creative Commons license (CC BY-SA 4.0).

\par Q: Why is this book free? You've done great job. This is suspicious, as with many other free things.
\par A: In my own experience, authors of technical literature write mostly for self-advertisement purposes. It's not possible to make any decent money from such work.

\par Q: How does one get a job in reverse engineering?
\par A: There are hiring threads that appear from time to time on reddit, devoted to RE\FNURLREDDIT{}
(\RedditHiringThread{}).
Try looking there.

A somewhat related hiring thread can be found in the \q{netsec} subreddit: \NetsecHiringThread{}.

\myindex{LISP}
\par Q: How can I learn programming in general?
\par A: Mastering both C and LISP languages makes programmer's life much, much easier.
I would recommend solving exercises from [\KRBook] and \ac{SICP}.

\par Q: I have a question...
\par A: Send it to me by email (\EMAIL).

%% Based on afterword.tex:
%\par Q: I have a suggestion/correction.
%\par A: Don't hesitate to send me suggestions and corrections, even petty corrections. (Do you see how my English is?)
%Do not worry to bother me while writing me about any petty mistakes you found, even if you are not very confident.
%I'm writing for beginners, after all, so beginners' opinions and comments are crucial for my job.

%One note, though: I'm working on the book a lot, so page and listing numbers change rapidly. Please do not refer to these
%in emails. Take a screenshot of the page, and underline the error in a graphics editor.
%If you're familiar with GitHub and \Latex\, you can fix the error in the source code:

%\href{http://go.yurichev.com/17089}{GitHub}.



% According to above suggestions, the following should be in thanks.tex
\subsection*{About the Korean translation}

In January 2015, the Acorn publishing company (\href{http://www.acornpub.co.kr}{www.acornpub.co.kr}) in South Korea did a huge amount of work in translating and publishing
this book (as it was in August 2014) into Korean.

It's available now at \href{http://go.yurichev.com/17343}{their website}.

\iffalse
\begin{figure}[H]
\centering
\includegraphics[scale=0.3]{acorn_cover.jpg}
\end{figure}
\fi

The translator is Byungho Min (\href{http://go.yurichev.com/17344}{twitter/tais9}).
The cover art was done by the artistic Andy Nechaevsky, a friend of the author:
\href{http://go.yurichev.com/17023}{facebook/andydinka}.
Acorn also holds the copyright to the Korean translation.

So, if you want to have a \IT{real} book on your shelf in Korean and
want to support this work, it is now available for purchase.

\subsection*{About the Persian/Farsi translation}

In 2016 the book was translated by Mohsen Mostafa Jokar (who is also known to Iranian community for his translation of Radare manual\footnote{\url{http://rada.re/get/radare2book-persian.pdf}}).
It is available on the publisher’s website\footnote{\url{http://goo.gl/2Tzx0H}} (Pendare Pars).

Here is a link to a 40-page excerpt: \url{https://beginners.re/farsi.pdf}.

National Library of Iran registration information: \url{http://opac.nlai.ir/opac-prod/bibliographic/4473995}.

\subsection*{About the Chinese translation}

In April 2017, translation to Chinese was completed by Chinese PTPress. They are also the Chinese translation copyright holders.

 The Chinese version is available for order here: \url{http://www.epubit.com.cn/book/details/4174}. A partial review and history behind the translation can be found here: \url{http://www.cptoday.cn/news/detail/3155}.

The principal translator is Archer, to whom the author owes very much. He was extremely meticulous (in a good sense) and reported most of the known mistakes and bugs, which is very important in literature such as this book.
The author would recommend his services to any other author!

The guys from \href{http://www.antiy.net/}{Antiy Labs} has also helped with translation. \href{http://www.epubit.com.cn/book/onlinechapter/51413}{Here is preface} written by them.
