% Okay, I'd like to suggest completely restructuring the beginning:
% Most books I see put the praise right after the cover, even before the info
% you have in the page_after_cover.tex, which should be right before the
% table of contents.

% I'm debating whether the call_for_translators should be before or after the
% table of contents, but I think it should be pretty visible and at the beginning.
% That way, potential contributors already have the idea in their head when they start.

% I would recommend taking the details about yourself from this files, adding some
% more details about yourself, and making an 'About the Author' (if you want it to be more formal)
% or an 'About Myself' (more informal). (Many place the 'About the Author'
% after the whole book, but especially on technical works, many put them before.
% Wiley (who did a bunch of Schneier's books) puts them at the beginning, for instance. Especially in pdf
% format, I would recommmend the beginning.)

% Continuing like this, I would put the
% thanks.tex right after that (consider calling it Acknowledgments, by the way),
% but I've also got some suggestions for that. I would make a subsection there
% called 'About the Editions of Reverse Engineering for Beginners' which would
% bring in the subsections from this file there about the Chinese, Korean and
% Farsi.(I'll also point out that it is acceptable and common
% to use first person ('I', 'me') in acknoledgments, which gives the thanks a
% more personal feel.)

% I would then put a page break, and put the FAQ.tex with some additions I'll
% suggest there. This will more or less take out the need for any of the text in
% this file, though you might want to still use it as the backbone of the intro

% I would also recommend putting the tools and communities in the beginning,
% not the end. Both are already useful at the beginning of the reader's journey.
% Truth is, the afterword should probably be rolled into the call for translators
% or made more unique, too.

% 1st_page.tex has three parts: one where you offer your services for interperting
% source code, a call for donations, and the opinion poll about the debuggers.
% I think that it should probably be split into three parts.
% The Reverse Engineering Services should be moved to the end, as sort of a last page,
% probably even after the index, though make sure it's in the Table of Contents then,
% and with a more identifiable name, like Reverse Engineering Services.
% I think the opinion poll should also be in the end, though maybe before the
% appendix. Beginners are unlikely to have an opinion, I think, while
% those who have finished are more than entitled to an opinion.
% The donations, on the other hand, should stay towards the beginning.

% For this file, the above would mean cutting out everything unbtil where the
% thanks and FAQ sections are put together - everything else will be covered
% somewhere else (I've put it in comments in each file already)
\section*{Preface}

There are several popular meanings of the term \q{\gls{reverse engineering}}:

1) The reverse engineering of software; researching compiled programs

2) The scanning of 3D structures and the subsequent digital manipulation required in order to duplicate them

3) Recreating \ac{DBMS} structure

This book is about the first meaning.

\subsection*{Prerequisites}

Basic knowledge of the C \ac{PL}.
Recommended reading: \myref{CCppBooks}.

\subsection*{Exercises and tasks}

\dots
can be found at: \url{http://challenges.re}.

\subsection*{About the author}
\begin{tabularx}{\textwidth}{ l X }

\raisebox{-\totalheight}{
\includegraphics[scale=0.60]{Dennis_Yurichev.jpg}
}

&
Dennis Yurichev is an experienced reverse engineer and programmer.
He can be contacted by email: \textbf{\EMAIL{}} or Skype: \textbf{dennis.yurichev}.

% FIXME: no link. \tablefootnote doesn't work
\end{tabularx}

% subsections:
\input{praise}
\input{thanks}
\input{FAQ_EN}

% Here's the structure based off of the suggestions above:
%% As per suggestion at the beginning of preface_EN.tex - Renaissance

\subsection*{About the author}
\begin{tabularx}{\textwidth}{ l X }

\raisebox{-\totalheight}{
\includegraphics[scale=0.60]{Dennis_Yurichev.jpg}
}
% In my humble opinion, you should fill this in a bit, but that's big words coming
% from an anonymous newbie contributor
&
Dennis Yurichev is an experienced reverse engineer and programmer.
He can be contacted by email: \textbf{\EMAIL{}} or Skype: \textbf{dennis.yurichev}.

\huge Please Donate
\normalsize

\bigskip
\bigskip
\bigskip

% added detail - Renaissance
\dots to this project so I can continue to work on the book and other articles: \\
\url{https://yurichev.com/donate.html}.

%% Suggested:
% It's with help of readers like you that I created this work (and still am creating
% it). Anyone who donates will get added to the list in the Acknowledgments (if they
% like).

%\input{acknowledgments}
%\input{FAQ_EN}


% According to above suggestions, the following should be in thanks.tex
\subsection*{About the Korean translation}

In January 2015, the Acorn publishing company (\href{http://www.acornpub.co.kr}{www.acornpub.co.kr}) in South Korea did a huge amount of work in translating and publishing
this book (as it was in August 2014) into Korean.

It's available now at \href{http://go.yurichev.com/17343}{their website}.

\iffalse
\begin{figure}[H]
\centering
\includegraphics[scale=0.3]{acorn_cover.jpg}
\end{figure}
\fi

The translator is Byungho Min (\href{http://go.yurichev.com/17344}{twitter/tais9}).
The cover art was done by the artistic Andy Nechaevsky, a friend of the author:
\href{http://go.yurichev.com/17023}{facebook/andydinka}.
Acorn also holds the copyright to the Korean translation.

So, if you want to have a \IT{real} book on your shelf in Korean and
want to support this work, it is now available for purchase.

\subsection*{About the Persian/Farsi translation}

In 2016 the book was translated by Mohsen Mostafa Jokar (who is also known to Iranian community for his translation of Radare manual\footnote{\url{http://rada.re/get/radare2book-persian.pdf}}).
It is available on the publisher’s website\footnote{\url{http://goo.gl/2Tzx0H}} (Pendare Pars).

Here is a link to a 40-page excerpt: \url{https://beginners.re/farsi.pdf}.

National Library of Iran registration information: \url{http://opac.nlai.ir/opac-prod/bibliographic/4473995}.

\subsection*{About the Chinese translation}

In April 2017, translation to Chinese was completed by Chinese PTPress. They are also the Chinese translation copyright holders.

 The Chinese version is available for order here: \url{http://www.epubit.com.cn/book/details/4174}. A partial review and history behind the translation can be found here: \url{http://www.cptoday.cn/news/detail/3155}.

The principal translator is Archer, to whom the author owes very much. He was extremely meticulous (in a good sense) and reported most of the known mistakes and bugs, which is very important in literature such as this book.
The author would recommend his services to any other author!

The guys from \href{http://www.antiy.net/}{Antiy Labs} has also helped with translation. \href{http://www.epubit.com.cn/book/onlinechapter/51413}{Here is preface} written by them.
